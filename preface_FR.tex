% TODO to be synced with EN version
\section*{Préface}

Il existe plusieurs définitions pour l'expression \q{ingénierie inverse ou rétro-ingénierie \gls{reverse engineering}} :

1) L'ingénierie inverse de logiciels : examiner des programmes compilés;

2) Le balayage des structures en 3D et la manipulation numérique nécessaire afin de les reproduire;

3) Recréer une structure de base de données.

Ce livre concerne la première définition.

\subsection*{Prérequis}

Connaissance basique du C \ac{PL}.
Il est recommandé de lire: \myref{CCppBooks}.

\subsection*{Exercices et tâches}

\dots 
ont été déplacés sur un site différent : \url{http://challenges.re}.

\subsection*{A propos de l'auteur}
\begin{tabularx}{\textwidth}{ l X }

\raisebox{-\totalheight}{
\includegraphics[scale=0.60]{Dennis_Yurichev.jpg}
}

&
Dennis Yurichev est un ingénieur expérimenté en rétro-ingénierie et un programmeur.
Il peut être contacté par email : \textbf{\EMAIL{}}.

% FIXME: no link. \tablefootnote doesn't work
\end{tabularx}

% subsections:
\subsection*{%
	\RU{Отзывы о книге}%
	\EN{Praise for}%
	\ES{Elogios para}%
	\PTBRph{}%
	\DEph{}\PLph{}%
	\ITAph{}
	\IT{\TITLE}%
}

\begin{itemize}
% expanded URLs to make it more robust for printouts. In electronic editions people will click anyway, so tracking will keep working
\item \q{It's very well done .. and for free .. amazing.}\footnote{\href{http://go.yurichev.com/17095}{twitter.com/daniel\_bilar/status/436578617221742593}} Daniel Bilar, Siege Technologies, LLC.

\item \q{... excellent and free}\footnote{\href{http://go.yurichev.com/17096}{twitter.com/petefinnigan/status/400551705797869568}} Pete Finnigan,%
	\RU{гуру по безопасности}%
	\ES{gur\'u de seguridad en}%
	\PTBRph{}%
	\DEph{}\PLph{}%
	\ITAph{}
\oracle
	\EN{security guru}.

\item \q{... book is interesting, great job!} Michael Sikorski,
	\RU{автор книги}%
	\EN{author of}%
	\ES{autor de}%
	\PTBRph{}%
	\DEph{}\PLph{}%
	\ITAph{}
\IT{Practical Malware Analysis: The Hands-On Guide to Dissecting Malicious Software}.

\item \q{... my compliments for the very nice tutorial!} Herbert Bos,
	\RU{профессор университета}%
	\EN{full professor at the}%
	\ES{catedr\'atico de tiempo completo en la}%
	\PTBRph{}%
	\DEph{}\PLph{}%
	\ITAph{}
Vrije Universiteit Amsterdam,
	\RU{соавтор}%
	\EN{co-author of}%
	\ES{coautor de}%
	\PTBRph{}%
	\DEph{}\PLph{}%
	\ITAph{}
\IT{Modern Operating Systems (4th Edition)}.

\item \q{... It is amazing and unbelievable.} Luis Rocha, CISSP / ISSAP, Technical Manager, Network \& Information Security at Verizon Business.

\item \q{Thanks for the great work and your book.} Joris van de Vis,
	\RU{специалист по}%
	\ES{especialista en}%
	\PTBRph{}%
	\DEph{}\PLph{}%
	\ITAph{}
SAP Netweaver \& Security
	\EN{specialist}.

\item \q{... reasonable intro to some of the techniques.}\footnote{\href{http://go.yurichev.com/17099}{reddit}} Mike Stay,
	\RU{преподаватель в}%
	\EN{teacher at the}%
	\ES{profesor en el}%
	\PTBRph{}%
	\DEph{}\PLph{}%
	\ITAph{}
Federal Law Enforcement Training Center, Georgia, US.

\item \q{I love this book! I have several students reading it at the moment, plan to use it in graduate course.}\footnote{\href{http://go.yurichev.com/17097}{twitter.com/sergeybratus/status/505590326560833536}}
	\RU{Сергей Братусь}%
	\EN{Sergey Bratus}%
	\ES{Sergey Bratus}%
	\PTBRph{}%
	\DEph{}\PLph{}%
	\ITAph{},
Research Assistant Professor
	\RU{в отделе Computer Science в}%
	\EN{at the Computer Science Department at}%
	\ES{en el Departamento de Ciencias de la Computaci\'on en}%
	\PTBRph{}%
	\DEph{}\PLph{}%
	\ITAph{}
Dartmouth College

\item \q{Dennis @Yurichev has published an impressive (and free!) book on reverse engineering}\footnote{\href{http://go.yurichev.com/17098}{twitter.com/TanelPoder/status/524668104065159169}} Tanel Poder,
	\RU{эксперт по настройке производительности Oracle RDBMS}%
	\EN{Oracle RDBMS performance tuning expert}%
	\ES{experto en afinaci\'on de rendimiento de Oracle RDBMS}%
	\PTBRph{}%
	\DEph{}\PLph{}
	\ITAph{}.

\item \q{This book is some kind of Wikipedia to beginners...} Archer, Chinese Translator, IT Security Researcher.

\RU{\item \q{Прочел Вашу книгу~--- отличная работа, рекомендую на своих курсах студентам
в качестве учебного пособия}. Николай Ильин, преподаватель в ФТИ НТУУ \q{КПИ} и DefCon-UA}
\end{itemize}

\ifdefined\RUSSIAN
\newcommand{\PeopleMistakesInaccuracies}{Станислав \q{Beaver} Бобрицкий, Александр Лысенко, Shell Rocket, Zhu Ruijin, Changmin Heo, Александр \q{Solar Designer} Песляк, Vitor Vidal, Марк Уилсон.}
\else
\newcommand{\PeopleMistakesInaccuracies}{Stanislav \q{Beaver} Bobrytskyy, Alexander Lysenko, Shell Rocket, Zhu Ruijin, Changmin Heo, Alexander \q{Solar Designer} Peslyak, Vitor Vidal, Mark Wilson.}
\fi

\EN{\subsection*{Thanks}

For patiently answering all my questions: \HERMIT, Slava \q{Avid} Kazakov.

For sending me notes about mistakes and inaccuracies: \PeopleMistakesInaccuracies{}.

For helping me in other ways:
Andrew Zubinski,
Arnaud Patard (rtp on \#debian-arm IRC),
noshadow on \#gcc IRC,
Aliaksandr Autayeu,
Mohsen Mostafa Jokar.

For translating the book into Simplified Chinese:
Antiy Labs (\href{http://antiy.cn}{antiy.cn}), Archer.

For translating the book into Korean: Byungho Min.

For translating the book into Dutch: Cedric Sambre (AKA Midas).

For translating the book into Spanish: \PeopleSpanishTranslators{}.

For translating the book into Portuguese: Thales Stevan de A. Gois.

For translating the book into Italian: \PeopleItalianTranslators{}.

For translating the book into French: \PeopleFrenchTranslators{}.

For translating the book into German: \PeopleGermanTranslators{}.

For proofreading:
Alexander \q{Lstar} Chernenkiy,
Vladimir Botov,
Andrei Brazhuk,
Mark ``Logxen'' Cooper, Yuan Jochen Kang, Mal Malakov, Lewis Porter, Jarle Thorsen, Hong Xie.

Vasil Kolev\footnote{\url{https://vasil.ludost.net/}} did a great amount of work in proofreading and correcting many mistakes.

For illustrations and cover art: Andy Nechaevsky.

Thanks also to all the folks on github.com who have contributed notes and corrections\FNGithubContributors{}.

Many \LaTeX\ packages were used: I would like to thank the authors as well.

\subsubsection*{Donors}

Those who supported me during the time when I wrote significant part of the book:

\subsubsection*{\RU{Жертвователи}\EN{Donors}}

10 * \RU{аноним}\EN{anonymous}, 2 * \RU{Олег Выговский}\EN{Oleg Vygovsky}, Daniel Bilar, James Truscott,
Luis Rocha, Joris van de Vis, Richard S Shultz, Jang Minchang, Shade Atlas, Yao Xiao,
Pawel Szczur, Justin Simms, Shawn the R0ck, Ki Chan Ahn, Triop AB, Ange Albertini,
\RU{Сергей Лукьянов}\EN{Sergey Lukianov}, Ludvig Gislason, Gérard Labadie, Sergey Volchkov.


Thanks a lot to every donor!
}
\ES{\subsection*{Agradecimientos}

Por contestar pacientemente a todas mis preguntas: \HERMIT, Slava \q{Avid} Kazakov.

Por enviarme notas acerca de errores e inexactitudes: \PeopleMistakesInaccuracies{}.

Por ayudarme de otras formas:
Andrew Zubinski,
Arnaud Patard (rtp en \#debian-arm IRC),
noshadow en \#gcc IRC,
Aliaksandr Autayeu,
Mohsen Mostafa Jokar.

Por traducir el libro a Chino Simplificado:
Antiy Labs (\href{http://antiy.cn}{antiy.cn}), Archer.

Por traducir el libro a Coreano: Byungho Min.

\ESph{}: Cedric Sambre (AKA Midas).

\ESph{}: \PeopleSpanishTranslators{}.

\ESph{}: Thales Stevan de A. Gois.

\ESph{}: \PeopleItalianTranslators{}.

\ESph{}: \PeopleFrenchTranslators{}.

\DEph{}: \PeopleGermanTranslators{}.

\ES{Por correcci\'on de pruebas}%
Alexander \q{Lstar} Chernenkiy,
Vladimir Botov,
Andrei Brazhuk,
Mark ``Logxen'' Cooper, Yuan Jochen Kang, Mal Malakov, Lewis Porter, Jarle Thorsen, Hong Xie.

Vasil Kolev\footnote{\url{https://vasil.ludost.net/}} realiz\'o una gran cantidad de trabajo en correcci\'on de pruebas y correcci\'on de muchos errores.

Por las ilustraciones y el arte de la portada: Andy Nechaevsky.

Gracias a toda la gente en github.com que ha contribuido con notas y correcciones\FNGithubContributors{}.

Muchos paquetes de \LaTeX\ fueron utiliados: quiero agradecer tambi\'en a sus autores.

\subsubsection*{Donadores}

Aquellos que me apoyaron durante el tiempo que escrib\'i una parte significativa del libro:

\subsubsection*{\RU{Жертвователи}\EN{Donors}}

10 * \RU{аноним}\EN{anonymous}, 2 * \RU{Олег Выговский}\EN{Oleg Vygovsky}, Daniel Bilar, James Truscott,
Luis Rocha, Joris van de Vis, Richard S Shultz, Jang Minchang, Shade Atlas, Yao Xiao,
Pawel Szczur, Justin Simms, Shawn the R0ck, Ki Chan Ahn, Triop AB, Ange Albertini,
\RU{Сергей Лукьянов}\EN{Sergey Lukianov}, Ludvig Gislason, Gérard Labadie, Sergey Volchkov.


!`Gracias a cada donante!

}
\NL{\subsection*{Dankwoord}

Voor al mijn vragen geduldig te beantwoorden: \HERMIT, Slava \q{Avid} Kazakov.

Om me nota\'s over fouten en onnauwkeurigheden toe te sturen: \PeopleMistakesInaccuracies{}.

Om me te helpen op andere manieren:
Andrew Zubinski,
Arnaud Patard (rtp op \#debian-arm IRC),
noshadow op \#gcc IRC,
Aliaksandr Autayeu, Mohsen Mostafa Jokar.

Om het boek te vertalen naar het Vereenvoudigd Chinees:
Antiy Labs (\href{http://antiy.cn}{antiy.cn}), Archer.

Om dit boek te vertalen in het Koreaans: Byungho Min.

\NLph{}: Cedric Sambre (AKA Midas).

\NLph{}: \PeopleSpanishTranslators{}.

\NLph{}: Thales Stevan de A. Gois.

\NLph{}: \PeopleItalianTranslators{}.

\NLph{}: \PeopleFrenchTranslators{}.

\NLph{}: \PeopleGermanTranslators{}.

Voor proofreading:
Alexander \q{Lstar} Chernenkiy,
Vladimir Botov,
Andrei Brazhuk,
Mark ``Logxen'' Cooper, Yuan Jochen Kang, Mal Malakov, Lewis Porter, Jarle Thorsen, Hong Xie.

Vasil Kolev\footnote{\url{https://vasil.ludost.net/}}, voor het vele werk in proofreading en het verbeteren van vele fouten.

Voor de illustraties en cover art: Andy Nechaevsky.

Dank aan al de mensen op github.com die hebben nota\'s en correcties hebben bijgedragen\FNGithubContributors{}.

Veel \LaTeX\ packages zijn gebruikt. Ik zou de auteurs hiervan ook graag bedanken.

\subsubsection*{Donaties}

Zij die me gesteund hebben tijdens het schrijven van een groot deel van dit boek:

\subsubsection*{\RU{Жертвователи}\EN{Donors}}

10 * \RU{аноним}\EN{anonymous}, 2 * \RU{Олег Выговский}\EN{Oleg Vygovsky}, Daniel Bilar, James Truscott,
Luis Rocha, Joris van de Vis, Richard S Shultz, Jang Minchang, Shade Atlas, Yao Xiao,
Pawel Szczur, Justin Simms, Shawn the R0ck, Ki Chan Ahn, Triop AB, Ange Albertini,
\RU{Сергей Лукьянов}\EN{Sergey Lukianov}, Ludvig Gislason, Gérard Labadie, Sergey Volchkov.


Veel dank aan elke donor!
}
\RU{\subsection*{Благодарности}

Тем, кто много помогал мне отвечая на массу вопросов: \HERMIT, Слава \q{Avid} Казаков.

Тем, кто присылал замечания об ошибках и неточностях: \PeopleMistakesInaccuracies{}.

Просто помогали разными способами:
Андрей Зубинский,
Arnaud Patard (rtp на \#debian-arm IRC),
noshadow на \#gcc IRC,
Александр Автаев,
Mohsen Mostafa Jokar.

Переводчикам на китайский язык:
Antiy Labs (\href{http://antiy.cn}{antiy.cn}), Archer.

Переводчику на корейский язык: Byungho Min.

Переводчику на голландский язык: Cedric Sambre (AKA Midas).

Переводчикам на испанский язык: \PeopleSpanishTranslators{}.

Переводчикам на португальский язык: Thales Stevan de A. Gois.

Переводчикам на итальянский язык: \PeopleItalianTranslators{}.

Переводчикам на французский язык: \PeopleFrenchTranslators{}.

Переводчикам на немецкий язык: \PeopleGermanTranslators{}.

Корректорам:
Александр \q{Lstar} Черненький,
Владимир Ботов,
Андрей Бражук,
Марк ``Logxen'' Купер, Yuan Jochen Kang, Mal Malakov, Lewis Porter, Jarle Thorsen, Hong Xie.

Васил Колев\footnote{\url{https://vasil.ludost.net/}} сделал очень много исправлений и указал на многие ошибки.

За иллюстрации и обложку: Андрей Нечаевский.

И ещё всем тем на github.com кто присылал замечания и исправления\FNGithubContributors{}.

Было использовано множество пакетов \LaTeX. Их авторов я также хотел бы поблагодарить.

\subsubsection*{Жертвователи}

Те, кто поддерживал меня во время написании этой книги:

\subsubsection*{\RU{Жертвователи}\EN{Donors}}

10 * \RU{аноним}\EN{anonymous}, 2 * \RU{Олег Выговский}\EN{Oleg Vygovsky}, Daniel Bilar, James Truscott,
Luis Rocha, Joris van de Vis, Richard S Shultz, Jang Minchang, Shade Atlas, Yao Xiao,
Pawel Szczur, Justin Simms, Shawn the R0ck, Ki Chan Ahn, Triop AB, Ange Albertini,
\RU{Сергей Лукьянов}\EN{Sergey Lukianov}, Ludvig Gislason, Gérard Labadie, Sergey Volchkov.


Огромное спасибо каждому!

}


\subsection*{mini-FAQ}

\par Q: Quels sont les prérequis nécessaires avant de lire ce livre ?
\par A: Une compréhension de base du C/C++ serait l'idéal.

\par Q: Puis-je acheter une version papier du livre en Russe / Anglais ?
\par A: Malheureusement non, aucune maison d'édition n'a été intéressée pour publier une version en russe ou en anglais du livre jusqu'à présent.
Cependant, vous pouvez demander à votre imprimerie préférée de l'imprimer et de le relier.

\par Q: Y a-il une version ePub/Mobi ?
\par A: Non, mais il existe une version PDF en format A5 pour les lecteurs ebooks.
Le livre dépend majoritairement de TeX/LaTeX, il n'est donc pas évident de le convertir en version ePub/Mobi.

\par Q: Pourquoi devrait-on apprendre l'assembleur de nos jours ?
\par A: A moins d'être un développeur d'\ac{OS}, vous n'aurez probablement pas besoin d'écrire en assembleur\textemdash{}les derniers compilateurs (ceux de notre décennie) sont meilleurs que les êtres humains en terme d'optimisation. \footnote{Un très bon article à ce sujet : \InSqBrackets{\AgnerFog}}.

De plus, les derniers \ac{CPU}s sont des appareils complexes et la connaissance de l'assembleur n'aide pas vraiment à comprendre leurs mécanismes internes.

Cela dit, il existe au moins deux domaines dans lesquels une bonne connaissance de l'assembleur peut être utile : 
Tout d'abord, pour de la recherche en sécurité ou sur des malwares. C'est également un bon moyen de comprendre un code compilé lorsqu'on le debug.
Ce livre est donc destiné à ceux qui veulent comprendre l'assembleur plutôt que d'écrire en assembleur, ce qui explique pourquoi il y a de nombreux exemples de résultats issus de compilateurs dans ce livre. 

\par Q: J'ai cliqué sur un lien dans le document PDF, comment puis-je retourner en arrière ?
\par A: Dans Adobe Acrobat Reader, appuyez sur Alt + Flèche gauche. Dans Evince, appuyez sur le bouton ``<''.

\par Q: Puis-je imprimer ce livre / l'utiliser pour de l'enseignement ?
\par A: Bien sûr ! C'est la raison pour laquelle le livre est sous licence Creative Commons (CC BY-SA 4.0).

\par Q: Pourquoi ce livre est-il gratuit ? Vous avez fait du bon boulot. C'est suspect, comme nombre de choses gratuites.
\par A: D'après ma propre expérience, les auteurs d'ouvrages techniques font cela pour l'auto-publicité. Il n'est pas possible de se faire beaucoup d'argent d'une telle manière.

\par Q: Comment trouver du travail dans le domaine de la rétro-ingénierie ?
\par A: Il existe des topics d'embauche qui apparaissent de temps en temps sur Reddit, dédiés à la rétro-ingénierie (cf. reverse engineering ou RE)\FNURLREDDIT{}
(\href{http://go.yurichev.com/17333}{2013 Q3}, 
\href{http://go.yurichev.com/17334}{2014}).
Jetez un oeil ici.

Un topic d'embauche quelque peu lié peut être trouvé dans le subreddit \q{netsec} :
\href{http://go.yurichev.com/17335}{2014 Q2}.

\par Q: J'ai une question...
\par A: Envoyez la moi par email (\EMAIL).



\subsection*{A propos de la traduction en Coréen}

En Janvier 2015, la maison d'édition Acorn (\href{http://www.acornpub.co.kr}{www.acornpub.co.kr}) en Corée du Sud a réalisé un énorme travail en traduisant et en publiant mon livre (dans son état en Août 2014) en Coréen.

Il est désormais disponible sur \href{http://go.yurichev.com/17343}{leur site web}.

\iffalse
\begin{figure}[H]
\centering
\includegraphics[scale=0.3]{acorn_cover.jpg}
\end{figure}
\fi

Le traducteur est Byungho Min (\href{http://go.yurichev.com/17344}{twitter/tais9}).
L'illustration de couverture a été réalisée l'artiste, Andy Nechaevsky, un ami de l'auteur:
\href{http://go.yurichev.com/17023}{facebook/andydinka}.
Ils détiennent également les droits d'auteurs sur la traduction coréenne.

Donc si vous souhaitez avoir un livre \IT{réel} en coréen sur votre étagère et que vous souhaitez soutenir ce travail, il est désormais disponible à l'achat.

\subsection*{Á propos de la traduction en Farsi/Perse}

En 2016, ce livre a été traduit par Mohsen Mostafa Jokar (qui est aussi connu dans
la communauté iranienne pour sa traduction du manuel de Radare\footnote{\url{http://rada.re/get/radare2book-persian.pdf}}).
Il est disponible sur le site web de l'éditeur\footnote{\url{http://goo.gl/2Tzx0H}}
(Pendare Pars).

Extrait de 40 pages: \url{https://beginners.re/farsi.pdf}.

Enregistrement du livre à la Bibliothèque Nationale d'Iran: \url{http://opac.nlai.ir/opac-prod/bibliographic/4473995}.

\subsection*{Á propos de la traduction en Chinois}

En avril 2017, la traduction en Chinois a été terminée par Chinese PTPress. Ils sont
également les détenteurs des droits de la traduction en Chinois.

La version chinoise est disponible à l'achat ici: \url{http://www.epubit.com.cn/book/details/4174}.
Une revue partielle et l'historique de la traduction peut être trouvé ici: \url{http://www.cptoday.cn/news/detail/3155}.


Le traducteur principal est Archer, à qui je dois beaucoup. Il a été très méticuleux
(dans le bon sens du terme) et a signalé la plupart des erreurs et bugs connus, ce
qui est très important dans le genre de littérature de ce livre.
Je recommanderais ses services à tout autre auteur!

Les gens de \href{http://www.antiy.net/}{Antiy Labs} ont aussi aidé pour la traduction.
\href{http://www.epubit.com.cn/book/onlinechapter/51413}{Voici la préface} écrite par eux.
