\part{\RU{Что стоит почитать}\EN{Books/blogs worth reading}}
% TODO названия книг а не ссылки!
\chapter{\RU{Книги}\EN{Books}}

\section{Windows}

\cite{Russinovich}.

\section{\CCpp}

\cite{CPP11}.

\section{x86 / x86-64}

\cite{Intel}, \cite{AMD}

\section{ARM}

\RU{Документация от ARM}\EN{ARM manuals}: \url{http://go.yurichev.com/17024}

\chapter{\RU{Блоги}\EN{Blogs}}

\section{Windows}

\begin{itemize}
\item
\href{http://go.yurichev.com/17025}{Microsoft: Raymond Chen}
\item
\href{http://go.yurichev.com/17026}{nynaeve.net}
\end{itemize}

\chapter{\RU{Прочее}\EN{Other}}

\RU{Имеются два отличных субреддита на reddit.com посвященных \ac{RE}}
\EN{There are two excellent \ac{RE}-related subreddits on reddit.com}: 
\href{http://go.yurichev.com/17027}{ReverseEngineering} \AndENRU 
\href{http://go.yurichev.com/17028}{REMath}
(\RU{для тем посвященных пересечению \ac{RE} и математики}
\EN{for the topics on the intersection of \ac{RE} and mathematics}).

\RU{Имеется также часть сайта}\EN{There are also \ac{RE} part of} Stack Exchange 
\RU{посвященная \ac{RE}}\EN{website}:\\
\href{http://go.yurichev.com/17029}{reverseengineering.stackexchange.com}.

\RU{На IRC есть канал}\EN{On IRC there are} \#\#re \RU{на}\EN{channel on} FreeNode\footnote{\href{http://go.yurichev.com/17030}{freenode.net}}.
