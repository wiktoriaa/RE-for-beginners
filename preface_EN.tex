\section*{Preface}

There are several popular meanings of the term \q{\gls{reverse engineering}}:

1) The reverse engineering of software; researching compiled programs

2) The scanning of 3D structures and the subsequent digital manipulation required in order to duplicate them

3) Recreating \ac{DBMS} structure

This book is about the first meaning.

\subsection*{Topics discussed in-depth}

x86/x64, ARM/ARM64, MIPS, Java/JVM

\subsection*{Topics touched upon}

\oracle (\myref{oracle}),
Itanium (\myref{itanium}),
copy-protection dongles (\myref{dongles}),
LD\_PRELOAD (\myref{ld_preload}),
stack overflow,
\ac{ELF},
win32 PE file format (\myref{win32_pe}),
x86-64 (\myref{x86-64}),
critical sections (\myref{critical_sections}),
syscalls (\myref{syscalls}),
\ac{TLS},
position-independent code (\ac{PIC}) (\myref{sec:PIC}),
profile-guided optimization (\myref{PGO}),
C++ STL (\myref{cpp_STL}),
OpenMP (\myref{openmp}),
SEH (\myref{sec:SEH}).

\subsection*{Prerequisites}

Basic C \ac{PL} knowledge.
Recommended reading: \myref{CCppBooks}.

\subsection*{Exercises and tasks}

\dots
can be found at: \url{http://challenges.re}.

\subsection*{About the author}
\begin{tabularx}{\textwidth}{ l X }

\raisebox{-\totalheight}{
\includegraphics[scale=0.60]{Dennis_Yurichev.jpg}
}

&
Dennis Yurichev is an experienced reverse engineer and programmer.
He can be contacted by email: \textbf{\EMAIL{}} or Skype: \textbf{dennis.yurichev}.

% FIXME: no link. \tablefootnote doesn't work
\end{tabularx}

% subsections:
\subsection*{%
	\RU{Отзывы о книге}%
	\EN{Praise for}%
	\ES{Elogios para}%
	\PTBRph{}%
	\DEph{}\PLph{}%
	\ITAph{}
	\IT{\TITLE}%
}

\begin{itemize}
% expanded URLs to make it more robust for printouts. In electronic editions people will click anyway, so tracking will keep working
\item \q{It's very well done .. and for free .. amazing.}\footnote{\href{http://go.yurichev.com/17095}{twitter.com/daniel\_bilar/status/436578617221742593}} Daniel Bilar, Siege Technologies, LLC.

\item \q{... excellent and free}\footnote{\href{http://go.yurichev.com/17096}{twitter.com/petefinnigan/status/400551705797869568}} Pete Finnigan,%
	\RU{гуру по безопасности}%
	\ES{gur\'u de seguridad en}%
	\PTBRph{}%
	\DEph{}\PLph{}%
	\ITAph{}
\oracle
	\EN{security guru}.

\item \q{... book is interesting, great job!} Michael Sikorski,
	\RU{автор книги}%
	\EN{author of}%
	\ES{autor de}%
	\PTBRph{}%
	\DEph{}\PLph{}%
	\ITAph{}
\IT{Practical Malware Analysis: The Hands-On Guide to Dissecting Malicious Software}.

\item \q{... my compliments for the very nice tutorial!} Herbert Bos,
	\RU{профессор университета}%
	\EN{full professor at the}%
	\ES{catedr\'atico de tiempo completo en la}%
	\PTBRph{}%
	\DEph{}\PLph{}%
	\ITAph{}
Vrije Universiteit Amsterdam,
	\RU{соавтор}%
	\EN{co-author of}%
	\ES{coautor de}%
	\PTBRph{}%
	\DEph{}\PLph{}%
	\ITAph{}
\IT{Modern Operating Systems (4th Edition)}.

\item \q{... It is amazing and unbelievable.} Luis Rocha, CISSP / ISSAP, Technical Manager, Network \& Information Security at Verizon Business.

\item \q{Thanks for the great work and your book.} Joris van de Vis,
	\RU{специалист по}%
	\ES{especialista en}%
	\PTBRph{}%
	\DEph{}\PLph{}%
	\ITAph{}
SAP Netweaver \& Security
	\EN{specialist}.

\item \q{... reasonable intro to some of the techniques.}\footnote{\href{http://go.yurichev.com/17099}{reddit}} Mike Stay,
	\RU{преподаватель в}%
	\EN{teacher at the}%
	\ES{profesor en el}%
	\PTBRph{}%
	\DEph{}\PLph{}%
	\ITAph{}
Federal Law Enforcement Training Center, Georgia, US.

\item \q{I love this book! I have several students reading it at the moment, plan to use it in graduate course.}\footnote{\href{http://go.yurichev.com/17097}{twitter.com/sergeybratus/status/505590326560833536}}
	\RU{Сергей Братусь}%
	\EN{Sergey Bratus}%
	\ES{Sergey Bratus}%
	\PTBRph{}%
	\DEph{}\PLph{}%
	\ITAph{},
Research Assistant Professor
	\RU{в отделе Computer Science в}%
	\EN{at the Computer Science Department at}%
	\ES{en el Departamento de Ciencias de la Computaci\'on en}%
	\PTBRph{}%
	\DEph{}\PLph{}%
	\ITAph{}
Dartmouth College

\item \q{Dennis @Yurichev has published an impressive (and free!) book on reverse engineering}\footnote{\href{http://go.yurichev.com/17098}{twitter.com/TanelPoder/status/524668104065159169}} Tanel Poder,
	\RU{эксперт по настройке производительности Oracle RDBMS}%
	\EN{Oracle RDBMS performance tuning expert}%
	\ES{experto en afinaci\'on de rendimiento de Oracle RDBMS}%
	\PTBRph{}%
	\DEph{}\PLph{}
	\ITAph{}.

\item \q{This book is some kind of Wikipedia to beginners...} Archer, Chinese Translator, IT Security Researcher.

\RU{\item \q{Прочел Вашу книгу~--- отличная работа, рекомендую на своих курсах студентам
в качестве учебного пособия}. Николай Ильин, преподаватель в ФТИ НТУУ \q{КПИ} и DefCon-UA}
\end{itemize}

\ifdefined\RUSSIAN
\newcommand{\PeopleMistakesInaccuracies}{Станислав \q{Beaver} Бобрицкий, Александр Лысенко, Shell Rocket, Zhu Ruijin, Changmin Heo, Александр \q{Solar Designer} Песляк, Vitor Vidal, Марк Уилсон.}
\else
\newcommand{\PeopleMistakesInaccuracies}{Stanislav \q{Beaver} Bobrytskyy, Alexander Lysenko, Shell Rocket, Zhu Ruijin, Changmin Heo, Alexander \q{Solar Designer} Peslyak, Vitor Vidal, Mark Wilson.}
\fi

\EN{\subsection*{Thanks}

For patiently answering all my questions: \HERMIT, Slava \q{Avid} Kazakov.

For sending me notes about mistakes and inaccuracies: \PeopleMistakesInaccuracies{}.

For helping me in other ways:
Andrew Zubinski,
Arnaud Patard (rtp on \#debian-arm IRC),
noshadow on \#gcc IRC,
Aliaksandr Autayeu,
Mohsen Mostafa Jokar.

For translating the book into Simplified Chinese:
Antiy Labs (\href{http://antiy.cn}{antiy.cn}), Archer.

For translating the book into Korean: Byungho Min.

For translating the book into Dutch: Cedric Sambre (AKA Midas).

For translating the book into Spanish: \PeopleSpanishTranslators{}.

For translating the book into Portuguese: Thales Stevan de A. Gois.

For translating the book into Italian: \PeopleItalianTranslators{}.

For translating the book into French: \PeopleFrenchTranslators{}.

For translating the book into German: \PeopleGermanTranslators{}.

For proofreading:
Alexander \q{Lstar} Chernenkiy,
Vladimir Botov,
Andrei Brazhuk,
Mark ``Logxen'' Cooper, Yuan Jochen Kang, Mal Malakov, Lewis Porter, Jarle Thorsen, Hong Xie.

Vasil Kolev\footnote{\url{https://vasil.ludost.net/}} did a great amount of work in proofreading and correcting many mistakes.

For illustrations and cover art: Andy Nechaevsky.

Thanks also to all the folks on github.com who have contributed notes and corrections\FNGithubContributors{}.

Many \LaTeX\ packages were used: I would like to thank the authors as well.

\subsubsection*{Donors}

Those who supported me during the time when I wrote significant part of the book:

\subsubsection*{\RU{Жертвователи}\EN{Donors}}

10 * \RU{аноним}\EN{anonymous}, 2 * \RU{Олег Выговский}\EN{Oleg Vygovsky}, Daniel Bilar, James Truscott,
Luis Rocha, Joris van de Vis, Richard S Shultz, Jang Minchang, Shade Atlas, Yao Xiao,
Pawel Szczur, Justin Simms, Shawn the R0ck, Ki Chan Ahn, Triop AB, Ange Albertini,
\RU{Сергей Лукьянов}\EN{Sergey Lukianov}, Ludvig Gislason, Gérard Labadie, Sergey Volchkov.


Thanks a lot to every donor!
}
\ES{\subsection*{Agradecimientos}

Por contestar pacientemente a todas mis preguntas: \HERMIT, Slava \q{Avid} Kazakov.

Por enviarme notas acerca de errores e inexactitudes: \PeopleMistakesInaccuracies{}.

Por ayudarme de otras formas:
Andrew Zubinski,
Arnaud Patard (rtp en \#debian-arm IRC),
noshadow en \#gcc IRC,
Aliaksandr Autayeu,
Mohsen Mostafa Jokar.

Por traducir el libro a Chino Simplificado:
Antiy Labs (\href{http://antiy.cn}{antiy.cn}), Archer.

Por traducir el libro a Coreano: Byungho Min.

\ESph{}: Cedric Sambre (AKA Midas).

\ESph{}: \PeopleSpanishTranslators{}.

\ESph{}: Thales Stevan de A. Gois.

\ESph{}: \PeopleItalianTranslators{}.

\ESph{}: \PeopleFrenchTranslators{}.

\DEph{}: \PeopleGermanTranslators{}.

\ES{Por correcci\'on de pruebas}%
Alexander \q{Lstar} Chernenkiy,
Vladimir Botov,
Andrei Brazhuk,
Mark ``Logxen'' Cooper, Yuan Jochen Kang, Mal Malakov, Lewis Porter, Jarle Thorsen, Hong Xie.

Vasil Kolev\footnote{\url{https://vasil.ludost.net/}} realiz\'o una gran cantidad de trabajo en correcci\'on de pruebas y correcci\'on de muchos errores.

Por las ilustraciones y el arte de la portada: Andy Nechaevsky.

Gracias a toda la gente en github.com que ha contribuido con notas y correcciones\FNGithubContributors{}.

Muchos paquetes de \LaTeX\ fueron utiliados: quiero agradecer tambi\'en a sus autores.

\subsubsection*{Donadores}

Aquellos que me apoyaron durante el tiempo que escrib\'i una parte significativa del libro:

\subsubsection*{\RU{Жертвователи}\EN{Donors}}

10 * \RU{аноним}\EN{anonymous}, 2 * \RU{Олег Выговский}\EN{Oleg Vygovsky}, Daniel Bilar, James Truscott,
Luis Rocha, Joris van de Vis, Richard S Shultz, Jang Minchang, Shade Atlas, Yao Xiao,
Pawel Szczur, Justin Simms, Shawn the R0ck, Ki Chan Ahn, Triop AB, Ange Albertini,
\RU{Сергей Лукьянов}\EN{Sergey Lukianov}, Ludvig Gislason, Gérard Labadie, Sergey Volchkov.


!`Gracias a cada donante!

}
\NL{\subsection*{Dankwoord}

Voor al mijn vragen geduldig te beantwoorden: \HERMIT, Slava \q{Avid} Kazakov.

Om me nota\'s over fouten en onnauwkeurigheden toe te sturen: \PeopleMistakesInaccuracies{}.

Om me te helpen op andere manieren:
Andrew Zubinski,
Arnaud Patard (rtp op \#debian-arm IRC),
noshadow op \#gcc IRC,
Aliaksandr Autayeu, Mohsen Mostafa Jokar.

Om het boek te vertalen naar het Vereenvoudigd Chinees:
Antiy Labs (\href{http://antiy.cn}{antiy.cn}), Archer.

Om dit boek te vertalen in het Koreaans: Byungho Min.

\NLph{}: Cedric Sambre (AKA Midas).

\NLph{}: \PeopleSpanishTranslators{}.

\NLph{}: Thales Stevan de A. Gois.

\NLph{}: \PeopleItalianTranslators{}.

\NLph{}: \PeopleFrenchTranslators{}.

\NLph{}: \PeopleGermanTranslators{}.

Voor proofreading:
Alexander \q{Lstar} Chernenkiy,
Vladimir Botov,
Andrei Brazhuk,
Mark ``Logxen'' Cooper, Yuan Jochen Kang, Mal Malakov, Lewis Porter, Jarle Thorsen, Hong Xie.

Vasil Kolev\footnote{\url{https://vasil.ludost.net/}}, voor het vele werk in proofreading en het verbeteren van vele fouten.

Voor de illustraties en cover art: Andy Nechaevsky.

Dank aan al de mensen op github.com die hebben nota\'s en correcties hebben bijgedragen\FNGithubContributors{}.

Veel \LaTeX\ packages zijn gebruikt. Ik zou de auteurs hiervan ook graag bedanken.

\subsubsection*{Donaties}

Zij die me gesteund hebben tijdens het schrijven van een groot deel van dit boek:

\subsubsection*{\RU{Жертвователи}\EN{Donors}}

10 * \RU{аноним}\EN{anonymous}, 2 * \RU{Олег Выговский}\EN{Oleg Vygovsky}, Daniel Bilar, James Truscott,
Luis Rocha, Joris van de Vis, Richard S Shultz, Jang Minchang, Shade Atlas, Yao Xiao,
Pawel Szczur, Justin Simms, Shawn the R0ck, Ki Chan Ahn, Triop AB, Ange Albertini,
\RU{Сергей Лукьянов}\EN{Sergey Lukianov}, Ludvig Gislason, Gérard Labadie, Sergey Volchkov.


Veel dank aan elke donor!
}
\RU{\subsection*{Благодарности}

Тем, кто много помогал мне отвечая на массу вопросов: \HERMIT, Слава \q{Avid} Казаков.

Тем, кто присылал замечания об ошибках и неточностях: \PeopleMistakesInaccuracies{}.

Просто помогали разными способами:
Андрей Зубинский,
Arnaud Patard (rtp на \#debian-arm IRC),
noshadow на \#gcc IRC,
Александр Автаев,
Mohsen Mostafa Jokar.

Переводчикам на китайский язык:
Antiy Labs (\href{http://antiy.cn}{antiy.cn}), Archer.

Переводчику на корейский язык: Byungho Min.

Переводчику на голландский язык: Cedric Sambre (AKA Midas).

Переводчикам на испанский язык: \PeopleSpanishTranslators{}.

Переводчикам на португальский язык: Thales Stevan de A. Gois.

Переводчикам на итальянский язык: \PeopleItalianTranslators{}.

Переводчикам на французский язык: \PeopleFrenchTranslators{}.

Переводчикам на немецкий язык: \PeopleGermanTranslators{}.

Корректорам:
Александр \q{Lstar} Черненький,
Владимир Ботов,
Андрей Бражук,
Марк ``Logxen'' Купер, Yuan Jochen Kang, Mal Malakov, Lewis Porter, Jarle Thorsen, Hong Xie.

Васил Колев\footnote{\url{https://vasil.ludost.net/}} сделал очень много исправлений и указал на многие ошибки.

За иллюстрации и обложку: Андрей Нечаевский.

И ещё всем тем на github.com кто присылал замечания и исправления\FNGithubContributors{}.

Было использовано множество пакетов \LaTeX. Их авторов я также хотел бы поблагодарить.

\subsubsection*{Жертвователи}

Те, кто поддерживал меня во время написании этой книги:

\subsubsection*{\RU{Жертвователи}\EN{Donors}}

10 * \RU{аноним}\EN{anonymous}, 2 * \RU{Олег Выговский}\EN{Oleg Vygovsky}, Daniel Bilar, James Truscott,
Luis Rocha, Joris van de Vis, Richard S Shultz, Jang Minchang, Shade Atlas, Yao Xiao,
Pawel Szczur, Justin Simms, Shawn the R0ck, Ki Chan Ahn, Triop AB, Ange Albertini,
\RU{Сергей Лукьянов}\EN{Sergey Lukianov}, Ludvig Gislason, Gérard Labadie, Sergey Volchkov.


Огромное спасибо каждому!

}


\subsection*{mini-FAQ}

\par Q: What are prerequisites for reading this book?
\par A: Basic understanding of C/C++ is desirable.

\par Q: Can I buy Russian/English hardcopy/paper book?
\par A: Unfortunately no, no publisher got interested in publishing Russian or English version so far.
Meanwhile, you can ask your favorite copy shop to print/bind it.

\par Q: Is there epub/mobi version?
\par A: The book is highly dependent on TeX/LaTeX-specific hacks, so converting to HTML (epub/mobi is a set of HTMLs)
will not be easy.

\par Q: Why should one learn assembly language these days?
\par A: Unless you are an \ac{OS} developer, you probably don't need to code in assembly\textemdash{}latest compilers (2010s) are much better at performing optimizations than humans \footnote{A very good text about this topic: \InSqBrackets{\AgnerFog}}.

Also, latest \ac{CPU}s are very complex devices and assembly knowledge doesn't really help one to understand their internals.

That being said, there are at least two areas where a good understanding of assembly can be helpful: 
First and foremost, security/malware research. It is also a good way to gain a better understanding of your compiled code whilst debugging.
This book is therefore intended for those who want to understand assembly language rather 
than to code in it, which is why there are many examples of compiler output contained within.

\par Q: I clicked on a hyperlink inside a PDF-document, how do I go back?
\par A: In Adobe Acrobat Reader click Alt+LeftArrow. In Evince click ``<'' button.

\par Q: May I print this book / use it for teaching?
\par A: Of course! That's why the book is licensed under the Creative Commons license (CC BY-SA 4.0).

\par Q: Why is this book free? You've done great job. This is suspicious, as many other free things.
\par A: In my own experience, authors of technical literature do this mostly for self-advertisement purposes. It's not possible to get any decent money from such work.

\par Q: How does one get a job in reverse engineering?
\par A: There are hiring threads that appear from time to time on reddit, devoted to RE\FNURLREDDIT{}
(\RedditHiringThread{}).
Try looking there.

A somewhat related hiring thread can be found in the \q{netsec} subreddit: \NetsecHiringThread{}.

\par Q: I have a question...
\par A: Send it to me by email (\EMAIL).



\subsection*{About the Korean translation}

In January 2015, the Acorn publishing company (\href{http://www.acornpub.co.kr}{www.acornpub.co.kr}) in South Korea did a huge amount of work in translating and publishing
this book (as it was in August 2014) into Korean.

It's available now at \href{http://go.yurichev.com/17343}{their website}.

\iffalse
\begin{figure}[H]
\centering
\includegraphics[scale=0.3]{acorn_cover.jpg}
\end{figure}
\fi

The translator is Byungho Min (\href{http://go.yurichev.com/17344}{twitter/tais9}).
The cover art was done by the artistic Andy Nechaevsky, a friend of the author's:
\href{http://go.yurichev.com/17023}{facebook/andydinka}.
Acorn also holds the copyright to the Korean translation.

So, if you want to have a \IT{real} book on your shelf in Korean and
want to support this work, it is now available for purchase.

\subsection*{About the Persian/Farsi translation}

In 2016 the book was translated by Mohsen Mostafa Jokar (who is also known to Iranian community for his translation of Radare manual\footnote{\url{http://rada.re/get/radare2book-persian.pdf}}).
It is available on the publisher’s website\footnote{\url{http://goo.gl/2Tzx0H}} (Pendare Pars).

Here is a link to a 40-page excerpt: \url{https://beginners.re/farsi.pdf}.

National Library of Iran registration information: \url{http://opac.nlai.ir/opac-prod/bibliographic/4473995}.

\subsection*{About the Chinese translation}

In April 2017, translation to Chinese was completed by Chinese PTPress. They are also the Chinese translation copyright holders.

 The Chinese version is available for order here: \url{http://www.epubit.com.cn/book/details/4174}. A partial review and history behind the translation can be found here: \url{http://www.cptoday.cn/news/detail/3155}.

The principal translator is Archer, to whom the author owes very much. He was extremely meticulous (in a good sense) and reported most of the known mistakes and bugs, which is very important in literature such as this book.
The author would recommend his services to any other author!

The guys from \href{http://www.antiy.net/}{Antiy Labs} has also helped with translation. \href{http://www.epubit.com.cn/book/onlinechapter/51413}{Here is preface} written by them.
