% TODO translation
% TODO proof-reading
\section{Bitfields}

All bit-wise operations works just like in any other instruction set:

\begin{lstlisting}
	public static int set (int a, int b) 
	{
		return a | 1<<b;
	}

	public static int clear (int a, int b) 
	{
		return a & (~(1<<b));
	}
\end{lstlisting}

\begin{lstlisting}
  public static int set(int, int);
    flags: ACC_PUBLIC, ACC_STATIC
    Code:
      stack=3, locals=2, args_size=2
         0: iload_0       
         1: iconst_1      
         2: iload_1       
         3: ishl          
         4: ior           
         5: ireturn       

  public static int clear(int, int);
    flags: ACC_PUBLIC, ACC_STATIC
    Code:
      stack=3, locals=2, args_size=2
         0: iload_0       
         1: iconst_1      
         2: iload_1       
         3: ishl          
         4: iconst_m1     
         5: ixor          
         6: iand          
         7: ireturn       
\end{lstlisting}

\TT{iconst\_m1} loads -1 into stack, this is the same as 0xFFFFFFFF number.
XORing with 0xFFFFFFFF is the same effect as inverting all bits.

I also extended all types to 64-bit long:

\begin{lstlisting}
	public static long lset (long a, int b) 
	{
		return a | 1<<b;
	}

	public static long lclear (long a, int b) 
	{
		return a & (~(1<<b));
	}
\end{lstlisting}

\begin{lstlisting}
  public static long lset(long, int);
    flags: ACC_PUBLIC, ACC_STATIC
    Code:
      stack=4, locals=3, args_size=2
         0: lload_0       
         1: iconst_1      
         2: iload_2       
         3: ishl          
         4: i2l           
         5: lor           
         6: lreturn       

  public static long lclear(long, int);
    flags: ACC_PUBLIC, ACC_STATIC
    Code:
      stack=4, locals=3, args_size=2
         0: lload_0       
         1: iconst_1      
         2: iload_2       
         3: ishl          
         4: iconst_m1     
         5: ixor          
         6: i2l           
         7: land          
         8: lreturn       
\end{lstlisting}

The code is the same, but instructions with \IT{l} prefix is used, which operates on 64-bit values.
Also, second functions argument still has \IT{int} type, and when a 32-bit value in it 
needs to be promoted to 64-bit value i2l instruction is used, which essentially extend \IT{integer}
value to \IT{long} one.
