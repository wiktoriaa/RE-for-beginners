% TODO proof-reading
\subsubsection{Двухмерные массивы}


Двухмерные массивы в Java это просто одномерные массивы \IT{reference}-в на другие одномерные 
массивы.

Создадим двухмерный массив:

\begin{lstlisting}[style=customjava]
	public static void main(String[] args)
	{
		int[][] a = new int[5][10];
		a[1][2]=3;
	}
\end{lstlisting}

\begin{lstlisting}
  public static void main(java.lang.String[]);
    flags: ACC_PUBLIC, ACC_STATIC
    Code:
      stack=3, locals=2, args_size=1
         0: iconst_5      
         1: bipush        10
         3: multianewarray #2,  2      // class "[[I"
         7: astore_1      
         8: aload_1       
         9: iconst_1      
        10: aaload        
        11: iconst_2      
        12: iconst_3      
        13: iastore       
        14: return        
\end{lstlisting}


Он создается при помощи инструкции \TT{multianewarray}: тип объекта и размерность передаются
в операндах.

Размер массива (10*5) остается в стеке (используя инструкции \TT{iconst\_5} и \TT{bipush}).


\IT{Reference} на строку \#1 загружается по смещению 10 (\TT{iconst\_1} и \TT{aaload}).

Выборка столбца происходит используя инструкцию \TT{iconst\_2} по смещению 11.

Значение для записи устанавливается по смещению 12.

\TT{iastore} на 13 записывает элемент массива.

Как его прочитать?

\begin{lstlisting}[style=customjava]
	public static int get12 (int[][] in)
	{
		return in[1][2];
	}
\end{lstlisting}

\begin{lstlisting}
  public static int get12(int[][]);
    flags: ACC_PUBLIC, ACC_STATIC
    Code:
      stack=2, locals=1, args_size=1
         0: aload_0       
         1: iconst_1      
         2: aaload        
         3: iconst_2      
         4: iaload        
         5: ireturn       
\end{lstlisting}


\IT{Reference} на строку массива загружается по смещению 2, 
столбец устанавливается по смещению 3, \TT{iaload} загружает элемент массива.
