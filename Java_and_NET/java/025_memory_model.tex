% TODO translation
% TODO proof-reading
\section{\ac{JVM} memory model}

x86 and other low-level environments uses stack for arguments passing and 
as local variables storage.
\ac{JVM} is slightly different.

It has: 

\begin{itemize}
\item Local variable array (\ac{LVA}).
It is used for incoming function arguments and local variables.
Instructions like \TT{iload\_0} loads values from it.
\TT{istore} stores values to it.
First, function arguments are came: starting at 0 or at 1 
(if zeroth argument is occupied by \IT{this} pointer).
Then local variables are allocated.

Each slot has size of 32-bit. Hence, values of \IT{long} and \IT{double} data types 
occupy two slots.

\item Operand stack (or just ``stack''). 
It's used for computations and passing arguments to caller functions.
Unlike low-level environments like x86, it's not possible to access the stack without using
instructions which explicitely pushes or pops values to/from it.

\item Heap. It is used as storage for objects and arrays.
\end{itemize}

These 3 areas are isolated from each other.
