%FIXME: requires PTBR and ES revision (dbmussi)
% for index
\newcommand{\GrepUsage}{\RU{Использование grep}\EN{grep usage}\PTBR{Uso do grep}\ES{Uso de grep}}
\newcommand{\SyntacticSugar}{\RU{Синтаксический сахар}\EN{Syntactic Sugar}\PTBR{Açúcar sintático}\ES{Azúcar sintáctica}}
\newcommand{\CompilerAnomaly}{\RU{Аномалии компиляторов}\EN{Compiler's anomalies}\PTBR{Anomalias do compilador}\ES{Anomalías del compilador}}
\newcommand{\CLanguageElements}{\RU{Элементы языка Си}\EN{C language elements}\PTBR{Elementos da linguagem C}\ES{Elementos del lenguaje C}}
\newcommand{\CStandardLibrary}{\RU{Стандартная библиотека Си}\EN{C standard library}\PTBR{Biblioteca padrão C}\ES{Librería stándar C}}
\newcommand{\Instructions}{\RU{Инструкции}\EN{Instructions}\PTBR{Instruções}\ES{Instrucciones}}
\newcommand{\Pseudoinstructions}{\RU{Псевдоинструкции}\EN{Pseudoinstructions}\PTBR{Pseudo-instruções}\ES{Pseudo-instrucciones}}
\newcommand{\Prefixes}{\RU{Префиксы}\EN{Prefixes}\PTBR{Prefixos}\ES{Prefijos}}

\newcommand{\Flags}{\RU{Флаги}\EN{Flags}\PTBR{Flags}\ES{Flags}}
\newcommand{\Registers}{\RU{Регистры}\EN{Registers}\PTBR{Registradores}\ES{Registros}}
\newcommand{\registers}{\RU{регистры}\EN{registers}\PTBR{registradores}\ES{registros}}
\newcommand{\Stack}{\RU{Стек}\EN{Stack}\PTBR{Pilha}\ES{Pila}}
\newcommand{\Recursion}{\RU{Рекурсия}\EN{Recursion}\PTBR{Recursividade}\ES{Recursión}}
\newcommand{\RAM}{\RU{ОЗУ}\EN{RAM}\PTBR{RAM}\ES{RAM}}
\newcommand{\ROM}{\RU{ПЗУ}\EN{ROM}\PTBR{ROM}\ES{ROM}}
\newcommand{\Pointers}{\RU{Указатели}\EN{Pointers}\PTBR{Ponteiros}\ES{Apuntadores}}
\newcommand{\BufferOverflow}{\RU{Переполнение буфера}\EN{Buffer Overflow}\PTBR{Buffer Overflow}\ES{Buffer Overflow}}
\newcommand{\Conclusion}{\RU{Вывод}\EN{Conclusion}\PTBR{Conclusão}\ES{Conclusión}}

\newcommand{\Exercise}{\RU{Упражнение}\EN{Exercise}\PTBR{Exercício}\ES{Ejercicio}\xspace}
\newcommand{\Exercises}{\RU{Упражнения}\EN{Exercises}\PTBR{Exercícios}\ES{Ejercicios}\xspace}
\newcommand{\Arrays}{\RU{Массивы}\EN{Arrays}\PTBR{Matriz}\ES{Matriz}}
\newcommand{\Cpp}{\RU{Си++}\EN{C++}\PTBR{C++}\ES{C++}\xspace}
\newcommand{\CCpp}{\RU{Си/Си++}\EN{C/C++}\PTBR{C/C++}\ES{C/C++}\xspace}
\newcommand{\NonOptimizing}{\RU{Неоптимизирующий}\EN{Non-optimizing}\PTBR{Sem otimização}\ES{Sin optimización}\xspace}
\newcommand{\Optimizing}{\RU{Оптимизирующий}\EN{Optimizing}\PTBR{Com otimização}\ES{Con optimización}\xspace}
\newcommand{\NonOptimizingKeilVI}{\NonOptimizing Keil 6/2013\xspace}
\newcommand{\OptimizingKeilVI}{\Optimizing Keil 6/2013\xspace}
\newcommand{\NonOptimizingXcodeIV}{\NonOptimizing Xcode 4.6.3 (LLVM)\xspace}
\newcommand{\OptimizingXcodeIV}{\Optimizing Xcode 4.6.3 (LLVM)\xspace}
\newcommand{\ARMMode}{\RU{Режим ARM}\EN{ARM mode}\PTBR{Modo ARM}\ES{Modo ARM}\xspace}
\newcommand{\ThumbMode}{\RU{Режим Thumb}\EN{Thumb mode}\PTBR{Modo Thumb}\ES{Modo Thumb}\xspace}
\newcommand{\ThumbTwoMode}{\RU{Режим Thumb-2}\EN{Thumb-2 mode}\PTBR{Modo Thumb-2}\ES{Modo Thumb-2}\xspace}
\newcommand{\AndENRU}{\RU{и}\EN{and}\PTBR{e}\ES{y}\xspace}
\newcommand{\OrENRU}{\RU{или}\EN{or}\PTBR{ou}\ES{o}\xspace}
\newcommand{\InENRU}{\RU{в}\EN{in}\PTBR{em}\ES{en}\xspace}
\newcommand{\ForENRU}{\RU{для}\EN{for}\PTBR{para}\ES{para}\xspace}
\newcommand{\LineENRU}{\RU{строка}\EN{line}\PTBR{linha}\ES{línea}\xspace}

\newcommand{\DataProcessingInstructionsFootNote}{
	\RU{Эти инструкции также называются}
	\EN{These instructions are also called}
	\PTBR{Estas intruções também são chamadas}
	\ES{Estas instrucciones también se llaman} \q{data processing instructions}
}

% for .bib files
\newcommand{\AlsoAvailableAs}{\RU{Также доступно здесь:}\EN{Also available as}\PTBR{Também disponível como}\ES{También disponible como}\xspace}

% section names
\newcommand{\ShiftsSectionName}{\RU{Сдвиги}\EN{Shifts}\PTBR{Shifts}\ES{Shifts}}
\newcommand{\SignedNumbersSectionName}{\RU{Представление знака в числах}\EN{Signed number representations}}
\newcommand{\HelloWorldSectionName}{Hello, world!}
\newcommand{\SwitchCaseDefaultSectionName}{switch()/case/default}
\newcommand{\PrintfSeveralArgumentsSectionName}{\printf \RU{с несколькими аргументами}\EN{with several arguments}\PTBR{com vários argumentos}\ES{con varios argumentos}}
\newcommand{\BitfieldsChapter}{\RU{Работа с отдельными битами}\EN{Manipulating specific bit(s)}\PTBR{Manipulando bit(s) específicos}\ES{Manipulando bit(s) específicas}}
\newcommand{\ArithOptimizations}{
	\RU{Замена одних арифметических инструкций на другие}
	\EN{Replacing arithmetic instructions to other ones}
	\PTBR{Substituição de instruções aritiméticas por outras}
	\ES{Substituición de instrucciones aritméticas por otros}
	}
\newcommand{\FPUChapterName}{\RU{Работа с FPU}\EN{Floating-point unit}\PTBR{Unidade de Ponto flutuante}\ES{Unidad de Punto flotante}}
\newcommand{\SimpleStringsProcessings}{\RU{Простая работа с Си-строками}\EN{Simple C-strings processing}\PTBR{Processamento de strings C simples}\ES{Procesamiento de strings C simples}}
\newcommand{\DivisionByNineSectionName}{\RU{Деление на 9}\EN{Division by 9}\PTBR{Divisão por 9}\ES{División por 9}}
\newcommand{\Answer}{\RU{Ответ}\EN{Answer}\PTBR{Responda}\ES{Responda}}
\newcommand{\WhatThisCodeDoes}{\RU{Что делает этот код}\EN{What does this code do}\PTBR{O que este código faz}\ES{Lo que hace el código}?}
\newcommand{\WorkingWithFloatAsWithStructSubSubSectionName}{
\RU{Работа с типом float как со структурой}\EN{Working with the float type as with a structure}\PTBR{Trabalhando com o tipo float como uma estrutura}\ES{Trabajando con el tipo float como una estructura}}

\newcommand{\MinesweeperWinXPExampleChapterName}{\RU{Сапёр}\EN{Minesweeper}\PTBR{Campo minado}\ES{Buscaminas} (Windows XP)}

\newcommand{\StructurePackingSectionName}{\RU{Упаковка полей в структуре}\EN{Fields packing in structure}\PTBR{Organização de campos na estrutura}\ES{Organización de campos en la estructura}}
\newcommand{\StructuresChapterName}{\RU{Структуры}\EN{Structures}\PTBR{Estruturas}\ES{Estructuras}}
\newcommand{\PICcode}{\RU{адресно-независимый код}\EN{position-independent code}\PTBR{código independente de posição}\ES{código independiente de la posición}}
\newcommand{\CapitalPICcode}{\RU{Адресно-независимый код}\EN{Position-independent code}\PTBR{Código independente de posição}\ES{Código independiente de lá posición}}
\newcommand{\Loops}{\RU{Циклы}\EN{Loops}\PTBR{Laços}\ES{Lazos}}

% C
\newcommand{\PostIncrement}{\RU{Пост-инкремент}\EN{Post-increment}\PTBR{Pós-incremento}\ES{Post-incremento}}
\newcommand{\PostDecrement}{\RU{Пост-декремент}\EN{Post-decrement}\PTBR{Pós-decremento}\ES{Post-decremento}}
\newcommand{\PreIncrement}{\RU{Пре-инкремент}\EN{Pre-increment}\PTBR{Pré-incremento}\ES{Pre-incremento}}
\newcommand{\PreDecrement}{\RU{Пре-декремент}\EN{Pre-decrement}\PTBR{Pré-decremento}\ES{Pre-decremento}}

% MIPS
\newcommand{\GlobalPointer}{\RU{Глобальный указатель}\EN{Global Pointer}\PTBR{Ponteiro Global}\ES{Puntero Global}}

% other
\newcommand{\garbage}{\RU{мусор}\EN{garbage}\PTBR{Lixo}\ES{Basura}}
\newcommand{\IntelSyntax}{\RU{Синтаксис Intel}\EN{Intel syntax}\PTBR{Sintaxe Intel}\ES{Sintaxis Intel}}
\newcommand{\ATTSyntax}{\RU{Синтаксис AT\&T}\EN{AT\&T syntax}\PTBR{Sintaxe AT\&T}\ES{Sintaxis AT\&T}}
\newcommand{\randomNoise}{\RU{случайный шум}\EN{random noise}\PTBR{Ruído aleatório}\ES{Ruido aleatorio}}
\newcommand{\Example}{\RU{Пример}\EN{Example}\PTBR{Exemplo}\ES{Ejemplo}}
\newcommand{\argument}{\RU{аргумент}\EN{argument}\PTBR{argumento}\ES{argumento}}
\newcommand{\MarkedInIDAAs}{\RU{маркируется в \IDA как}\EN{marked in \IDA as}\PTBR{Marcado no \IDA como}\ES{Marcado en \IDA como}}
\newcommand{\HERMIT}{\RU{Андрей}\EN{Andrey}\PTBR{Andrey}\ES{Andrey} \q{herm1t} \RU{Баранович}\EN{Baranovich}\PTBR{Baranovich}\ES{Baranovich}}
\newcommand{\stepover}{\RU{сделать шаг, не входя в функцию}\EN{step over}\PTBR{passar por cima}\ES{pasar por encima}}
\newcommand{\ShortHotKeyCheatsheet}{\RU{Краткий справочник горячих клавиш}\EN{Hot-keys cheatsheet}\PTBR{Cheatsheet de teclas de atalho}\ES{Cheatsheet de teclas de acceso rápido}}

\newcommand{\assemblyOutput}{\RU{вывод на ассемблере}\EN{assembly output}\PTBR{saída do assembly}\ES{salida de assembly}}

