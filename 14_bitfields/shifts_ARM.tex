\subsubsection{ARM + \OptimizingXcode + \ARMMode}

\begin{lstlisting}[caption=\OptimizingXcode + \ARMMode]
                MOV             R1, R0
                MOV             R0, #0
                MOV             R2, #1
                MOV             R3, R0
loc_2E54
                TST             R1, R2,LSL R3 ; set flags according to R1 & (R2<<R3)
                ADD             R3, R3, #1    ; R3++
                ADDNE           R0, R0, #1    ; if ZF flag is cleared by TST, R0++
                CMP             R3, #32
                BNE             loc_2E54
                BX              LR
\end{lstlisting}

\TT{TST} это то же что и \TEST в x86.

Как я уже указывал~\ref{shifts_in_ARM_mode}, в режиме ARM нет отдельной инструкции для сдвигов, 
но модификаторами 
LSL (\IT{Logical Shift Left}), 
LSR (\IT{Logical Shift Right}), 
ASR (\IT{Arithmetic Shift Right}), 
ROR (\IT{Rotate Right}) и 
RRX (\IT{Rotate Right with Extend}) можно дополнять некоторые инструкции, такие как \MOV, \TT{TST},
\CMP, \ADD, \SUB, \TT{RSB}\footnote{Эти инструкции также называются ``data processing Instructions''}.

Таким образом, инструкция \TT{``TST R1, R2,LSL R3''} здесь работает как $R1 \land (R2 \ll R3)$.

\subsubsection{ARM + \OptimizingXcode + \ThumbTwoMode}

Почти такое же, только там применяется пара инструкций \TT{LSL.W}/\TT{TST} вместо одной \TT{TST},
ведь в режиме thumb нельзя добавлять указывать модификатор \TT{LSL} прямо в \TT{TST}.

\begin{lstlisting}
                MOV             R1, R0
                MOVS            R0, #0
                MOV.W           R9, #1
                MOVS            R3, #0
loc_2F7A
                LSL.W           R2, R9, R3
                TST             R2, R1
                ADD.W           R3, R3, #1
                IT NE
                ADDNE           R0, #1
                CMP             R3, #32
                BNE             loc_2F7A
                BX              LR
\end{lstlisting}

