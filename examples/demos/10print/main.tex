\section{10 PRINT CHR\$(205.5+RND(1)); : GOTO 10}

\RU{Все примеры здесь для .COM-файлов под MS-DOS}\EN{All examples here are MS-DOS .COM files}.
%FIXME -> about .COM files

\index{MS-DOS}
\RU{В}\EN{In} \cite{10PRINT} \RU{можно прочитать об одном из простейших генераторов случайных лабиринтов}
\EN{we can read about one of the most simplest possible random maze generators}.
\RU{Он просто бесконечно и случайно печатает символ слэша или обратный слэша, выдавая в итоге что-то вроде}
\EN{It just prints slash or backslash character randomly and endlessly, resulting something like}:

\begin{figure}[H]
\centering
\includegraphics[scale=\NormalScale]{examples/demos/10print/10print.png}
\end{figure}

\RU{Здесь несколько известных реализаций для 16-битного x86}
\EN{There are some known implementations for 16-bit x86}.

\subsection{\RU{Версия 42-х байт от Trixter}\EN{Trixter's 42 byte version}}

\newcommand{\FNURLTRIXTER}{\footnote{\url{http://go.yurichev.com/17305}}}

\RU{Листинг взят с его сайта}\EN{The listing taken from his website}\FNURLTRIXTER, \RU{но комментарии --- мои}
\EN{but comments are mine}.

\lstinputlisting{examples/demos/10print/10print_42.lst.\LANG}

\index{Intel!8253}
\RU{Псевдослучайное число на самом деле это время, прошедшее со старта системы, получаемое из чипа таймера 8253, 
это значение
увеличивается на единицу 18.2 раза в секунду}\EN{Pseudo-random value here is in fact the time 
passed from the system boot, taken from 8253 time chip, the value increases by one 18.2 times per second}.

\RU{Записывая ноль в порт}\EN{By writing zero to port} \TT{43h}, 
\RU{мы имеем ввиду что команда это}\EN{we mean the command is} "\RU{выбрать счетчик}\EN{select counter} 0", 
"counter latch", 
"\RU{двоичный счетчик}\EN{binary counter}" (\RU{а не значение \ac{BCD}}\EN{not \ac{BCD} value}).

\index{x86!\Instructions!POPF}
\RU{Прерывания снова разрешаются при помощи инструкции}\EN{Interrupts enabled back with} \TT{POPF}\RU{, которая
также возвращает флаг \TT{IF}}\EN{ instruction, which restores \TT{IF} flag as well}.

\index{x86!\Instructions!IN}
\RU{Инструкцию \TT{IN} нельзя использовать с другими регистрами кроме}\EN{It is not possible to 
use \TT{IN} instruction with other registers instead of} \TT{AL}, \RU{поэтому здесь перетасовка}
\EN{hence that shuffling}.

\subsection{\RU{Моя попытка укоротить версию Trixter: 27 байт}
\EN{My attempt to reduce Trixter's version: 27 bytes}}

\RU{Мы можем сказать, что мы используем таймер не для того чтобы получить точное время, но псевдослучайное число,
так что мы можем не тратить время (и код) на запрещение прерываний}\EN{We can say that since we use timer not 
to get precise time value, but pseudo-random one, so we may not
spent time (and code) to disable interrupts}.
\RU{Еще можно сказать, что так как мы берем бит из младшей 8-битной части, то мы можем считывать только её}
\EN{Another thing we might say that we need only bit from a low 8-bit part, so let's read only it}.

\RU{Я немного укоротил код и вышло 27 байт}\EN{I reduced the code slightly and I've got 27 bytes}:

\lstinputlisting{examples/demos/10print/10print_27.lst.\LANG}

\subsection{\RU{Использование случайного мусора в памяти как источника случайных чисел}
\EN{Take a random memory garbage as a source of randomness}}

\RU{Так как это}\EN{Since it is} MS-DOS, \RU{защиты памяти здесь нет вовсе, так что мы можем читать с какого
угодно адреса}\EN{there are no memory protection at all, we can read from whatever address}.
\index{x86!\Instructions!LODSB}
\RU{И даже более того: простая инструкция}\EN{Even more than that: simple} \TT{LODSB} 
\RU{будет читать байт по адресу}\EN{instruction will read byte from} \TT{DS:SI}\RU{, но это не проблема
если правильные значения не установлены в регистры, пусть она читает 1) случайные байты; 2) из случайного
места в памяти!}\EN{ address, but it's not a problem
if register values are not setted up, let it read 1) random bytes; 2) from random memory place!}

\RU{Так что на странице Trixter-а}\EN{So it is suggested in Trixter webpage}\FNURLTRIXTER 
\RU{можно найти предложение использовать}\EN{to use} \TT{LODSB} \RU{без всякой инициализации}\EN{without any setup}.

\index{x86!\Instructions!SCASB}
\RU{Есть также предложение использовать инструкцию}\EN{It is also suggested that} \TT{SCASB} 
\RU{вместо, потому что она выставляет флаги в соответствии с прочитанным значением}
\EN{instruction can be used instead, because it sets flag according to the byte it read}.

\RU{Еще одна идея насчет минимизации кода --- это использовать прерывание DOS}
\EN{Another idea to minimize code is to use} \TT{INT 29h} \RU{которое просто печатает символ на экране
из регистра}\EN{DOS syscall, which just prints character stored in} \TT{AL}\EN{ register}.

\RU{Это то что сделали}\EN{That is what} Peter Ferrie \AndENRU \HERMIT{}\EN{ did} (11 \AndENRU 
10 \RU{байт}\EN{bytes})
\footnote{\url{http://go.yurichev.com/17087}}:

\lstinputlisting[caption=\HERMIT: 11 \RU{байт}\EN{bytes}]{examples/demos/10print/herm1t_11.lst.\LANG}

\index{x86!\Instructions!SCASB}
\TT{SCASB} \RU{также использует значение в регистре}\EN{also use value in} \TT{AL}\RU{, она вычитает значение
случайного байта в памяти из}
\EN{ register, it subtract random memory byte value from} \TT{5Ch} \RU{в}\EN{value in} \TT{AL}.
\index{x86!\Instructions!JP}
\TT{JP} \RU{это редкая инструкция, здесь она используется для проверки флага четности (PF),
который вычисляется по формуле в листинге}\EN{is rare instruction, here it used for checking parity flag (PF), 
which is generated by the formulae in the listing}.
\RU{Как следствие, выводимый символ определяется не каким-то конкретным битом из случайного байта в памяти,
а суммой бит, и это (надеемся) сделает результат более распределенным}\EN{As a consequence, the output character 
is determined not by some bit in random memory byte, but by sum of bits, 
this (hoperfully) makes result more distributed}.

\index{x86!\Instructions!SALC}
\index{x86!\Instructions!SETALC}
\index{NEC V20}
\RU{Можно сделать еще короче, если использовать недокументированную x86-инструкцию}\EN{It is possible to 
make this even shorter by using undocumented x86 instruction} \TT{SALC} (\ac{AKA} \TT{SETALC}) (``Set AL CF'').
\RU{Она появилась в}\EN{It was introduced in NEC V20} \ac{CPU} \RU{и выставляет}\EN{and sets} \TT{AL} \RU{в}\EN{to} 
\TT{0xFF} \RU{если}\EN{if} \TT{CF} \RU{это 1 или 0 если наоборот}\EN{is 1 or to 0 if otherwise}.
\RU{Так что этот код не запустится на}\EN{So this code will not run on} 8086/8088.

\lstinputlisting[caption=Peter Ferrie: 10 \RU{байт}\EN{bytes}]{examples/demos/10print/ferrie_10.lst.\LANG}

\RU{Так что можно избавиться и от условных переходов}\EN{So it is possible to get rid of conditional jumps at all}.
\EN{The }\ac{ASCII}\RU{-код обратного слэща}\EN{ code of backslash} (``\textbackslash{}'') 
\RU{это}\EN{is} \TT{0x5C} \AndENRU \TT{0x2F} \RU{для слэша}\EN{for slash} (``/'').
\RU{Так что нам нужно конвертировать один (псевдослучайный) бит из флага}\EN{So we need to convert one 
(pseudo-random) bit in} \TT{CF} \RU{в значение}\EN{flag to} \TT{0x5C} \OrENRU \TT{0x2F}\EN{ value}.

\RU{Это делается легко: применяя операцию ``И'' ко всем битам в}\EN{This is done easily: by \TT{AND}-ing all 
bits in} \TT{AL} (\RU{где все 8 бит либо выставлены, либо сброшены}\EN{where all 8 bits are set or cleared}) 
\RU{с}\EN{with} \TT{0x2D} \RU{мы имеем просто}\EN{we have just} 0 \OrENRU \TT{0x2D}.
\RU{Прибавляя значение}\EN{By adding} \TT{0x2F} \RU{к этому значению, мы получаем}\EN{to this value, we get} 
\TT{0x5C} \OrENRU \TT{0x2F}.
\RU{И просто выводим это на экран}\EN{Then just ouptut it to screen}.

\subsection{\Conclusion{}}

\index{DOSBox}
\RU{Также стоит отметить, что результат может быть разным в эмуляторе}\EN{It is also worth adding that result may 
be different in} DOSBox, \gls{Windows NT} \RU{и даже}\EN{and even} MS-DOS, 
\RU{из-за разных условий:
чип таймера может эмулироваться по-разному, изначальные значения регистров также могут быть разными}
\EN{due to different
conditions: timer chip may be emulated differently, initial register contents may be different as well}.
