\section{10 PRINT CHR\$(205.5+RND(1)); : GOTO 10}

\IFRU{Все примеры здесь для .COM-файлов под MS-DOS}{All examples here are MS-DOS .COM files}.
%FIXME -> about .COM files

\index{MS-DOS}
\IFRU{В}{In} \cite{10PRINT} \IFRU{можно прочитать об одном из простейших генераторов случайных лабиринтов}
{we can read about one of the most simplest possible random maze generators}.
\IFRU{Он просто бесконечно и случайно печатает символ слэша или обратный слэша, выдавая в итоге что-то вроде}
{It just prints slash or backslash character randomly and endlessly, resulting something like}:

\begin{figure}[H]
\centering
\includegraphics[scale=0.66]{examples/demos/10print/10print.png}
\end{figure}

\IFRU{Здесь несколько известных реализаций для 16-битного x86}
{There are some known implementations for 16-bit x86}.

\subsection{\IFRU{Версия 42-х байт от Trixter}{Trixter's 42 byte version}}

\newcommand{\FNURLTRIXTER}{\footnote{\url{http://trixter.oldskool.org/2012/12/17/maze-generation-in-thirteen-bytes/}}}

\IFRU{Листинг взят с его сайта}{The listing taken from his website}\FNURLTRIXTER, \IFRU{но комментарии --- мои}
{but comments are mine}.

\lstinputlisting{examples/demos/10print/10print_42_\LANG.lst}

\index{8253}
\IFRU{Псевдослучайное число на самом деле это время прошедшее со старта системы, получаемое из чипа таймера 8253, 
это значение
увеличивается на единицу 18.2 раза в секунду}{Pseudo-random value here is in fact the time 
passed from the system boot, taken from 8253 time chip, the value increases by one 18.2 times per second}.

\IFRU{Записывая ноль в порт}{By writing zero to port} \TT{43h}, 
\IFRU{мы имеем ввиду что команда это}{we mean the command is} "\IFRU{выбрать счетчик}{select counter} 0", 
"counter latch", 
"\IFRU{двоичный счетчик}{binary counter}" (\IFRU{а не значение \ac{BCD}}{not \ac{BCD} value}).

\index{x86!\Instructions!POPF}
\IFRU{Прерывания снова разрешаются при помощи инструкции}{Interrupts enabled back with} \TT{POPF}\IFRU{, которая
также возвращает флаг \TT{IF}}{ instruction, which restores \TT{IF} flag as well}.

\index{x86!\Instructions!IN}
\IFRU{Инструкцию \TT{IN} нельзя использовать с другими регистрами кроме}{It is not possible to 
use \TT{IN} instruction with other registers instead of} \TT{AL}, \IFRU{поэтому здесь перетасовка}
{hence that shuffling}.

\subsection{\IFRU{Моя попытка укоротить версию Trixter: 27 байт}
{My attempt to reduce Trixter's version: 27 bytes}}

\IFRU{Мы можем сказать что мы используем таймер не для того чтобы получить точное время, но псевдо-случайное число,
так что мы можем не тратить время (и код) на запрещение прерываний}{We can say that since we use timer not 
to get precise time value, but pseudo-random one, so we may not
spent time (and code) to disable interrupts}.
\IFRU{Еще можно сказать что так как мы берем бит из младшей 8-битной части, то мы можем считывать только её}
{Another thing we might say that we need only bit from a low 8-bit part, so let's read only it}.

\IFRU{Я немного укоротил код и вышло 27 байт}{I reduced the code slightly and I've got 27 bytes}:

\lstinputlisting{examples/demos/10print/10print_27_\LANG.lst}

\subsection{\IFRU{Использование случайного мусора в памяти как источника случайных чисел}
{Take a random memory garbage as a source of randomness}}

\IFRU{Так как это}{Since it is} MS-DOS, \IFRU{защиты памяти здесь нет вовсе, так что мы можем читать с какого
угодно адреса}{there are no memory protection at all, we can read from whatever address}.
\index{x86!\Instructions!LODSB}
\IFRU{И даже более того: простая инструкция}{Even more than that: simple} \TT{LODSB} 
\IFRU{будет читать байт по адресу}{instruction will read byte from} \TT{DS:SI}\IFRU{, но это не проблема
если правильные значения не установлены в регистры, пусть она читает 1) случайные байты; 2) из случайного
места в памяти!}{ address, but it's not a problem
if register values are not setted up, let it read 1) random bytes; 2) from random memory place!}

\IFRU{Так что на странице Trixter-а}{So it is suggested in Trixter webpage}\FNURLTRIXTER 
\IFRU{можно найти предложение использовать}{to use} \TT{LODSB} \IFRU{без всякой инициализации}{without any setup}.

\index{x86!\Instructions!SCASB}
\IFRU{Есть также предложение использовать инструкцию}{It is also suggested that} \TT{SCASB} 
\IFRU{вместо, потому что она выставляет флаги в соответствии с прочитанным значением}
{instruction can be used instead, because it sets flag according to the byte it read}.

\IFRU{Еще одна идея насчет минимизации кода это использовать прерывание DOS}
{Another idea to minimize code is to use} \TT{INT 29h} \IFRU{которое просто печатает символ на экране
из регистра}{DOS syscall, which just prints character stored in} \TT{AL}\EN{ register}.

\IFRU{Это то что сделали}{That is what} Peter Ferrie \AndENRU \HERMIT{}\EN{ did} (11 \AndENRU 
10 \IFRU{байт}{bytes})
\footnote{\url{http://pferrie.host22.com/misc/10print.htm}}:

\lstinputlisting[caption=\HERMIT: 11 \IFRU{байт}{bytes}]{examples/demos/10print/herm1t_11_\LANG.lst}

\index{x86!\Instructions!SCASB}
\TT{SCASB} \IFRU{также использует значение в регистре}{also use value in} \TT{AL}\IFRU{, она вычитает значение
слайчного байта в памяти из}
{ register, it subtract random memory byte value from} \TT{5Ch} \IFRU{в}{value in} \TT{AL}.
\index{x86!\Instructions!JP}
\TT{JP} \IFRU{это редкая инструкция, здесь она используется для проверки флага четности (PF),
который вычисляется по формуле в листинге}{is rare instruction, here it used for checking parity flag (PF), 
which is generated by the formulae in the listing}.
\IFRU{Как следствие, выводимый символ определяется не каким-то конкретным битом из случайного байта в памяти,
а суммой бит, и это (надеемся) сделает результат более распределенным}{As a consequence, the output character 
is determined not by some bit in random memory byte, but by sum of bits, 
this (hoperfully) makes result more distributed}.

\index{x86!\Instructions!SALC}
\index{x86!\Instructions!SETALC}
\index{NEC V20}
\IFRU{Можно сделать еще короче, если использовать недокументированную x86-инструкцию}{It is possible to 
make this even shorter by using undocumented x86 instruction} \TT{SALC} (\ac{AKA} \TT{SETALC}) (``Set AL CF'').
\IFRU{Она появилась в}{It was introduced in NEC V20} \ac{CPU} \IFRU{и выставляет}{and sets} \TT{AL} \IFRU{в}{to} 
\TT{0xFF} \IFRU{если}{if} \TT{CF} \IFRU{это 1 или 0 если наоборот}{is 1 or to 0 if otherwise}.
\IFRU{Так что этот код не запустится на}{So this code will not run on} 8086/8088.

\lstinputlisting[caption=Peter Ferrie: 10 \IFRU{байт}{bytes}]{examples/demos/10print/ferrie_10_\LANG.lst}

\IFRU{Так что можно избавиться и от условных переходов}{So it is possible to get rid of conditional jumps at all}.
\EN{The }\ac{ASCII}\IFRU{-код обратного слэща}{ code of backslash} (``\textbackslash{}'') 
\IFRU{это}{is} \TT{0x5C} \AndENRU \TT{0x2F} \IFRU{для слэша}{for slash} (``/'').
\IFRU{Так что нам нужно конвертировать один (псевдо-случайный) бит из флага}{So we need to convert one 
(pseudo-random) bit in} \TT{CF} \IFRU{в значение}{flag to} \TT{0x5C} \OrENRU \TT{0x2F}\EN{ value}.

\IFRU{Это делается легко: применяя операцию ``И'' ко всем битам в}{This is done easily: by \TT{AND}-ing all 
bits in} \TT{AL} (\IFRU{где все 8 бит либо выставлены, либо сброшены}{where all 8 bits are set or cleared}) 
\IFRU{с}{with} \TT{0x2D} \IFRU{мы имеем просто}{we have just} 0 \OrENRU \TT{0x2D}.
\IFRU{Прибавляя значение}{By adding} \TT{0x2F} \IFRU{к этому значению, мы получаем}{to this value, we get} 
\TT{0x5C} \OrENRU \TT{0x2F}.
\IFRU{И просто выводим это на экран}{Then just ouptut it to screen}.

\subsection{\IFRU{Вывод}{Conclusion}}

\index{DOSBox}
\IFRU{Также стоит отметить, что результат может быть разным в эмуляторе}{It is also worth adding that result may 
be different in} DOSBox, \gls{Windows NT} \IFRU{и даже}{and even} MS-DOS, 
\IFRU{из-за разных условий:
чип таймера может эмулироваться по-разному, изначальные значения регистров также могут быть разными}
{due to different
conditions: timer chip may be emulated differently, initial register contents may be different as well}.
