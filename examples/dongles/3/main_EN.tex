\subsection{Example \#3: MS-DOS}
\label{dongle_16bit_dos}

\myindex{MS-DOS}
Another very old software for MS-DOS from 1995 
also developed by a company that disappeared a long time ago.

\myindex{Intel!8086}
\myindex{Intel!80286}

In the pre-DOS extenders era, all the software for MS-DOS mostly relied on 16-bit 8086 or 80286 CPUs,
so the code was 16-bit en masse.

The 16-bit code is mostly same as you already saw in this book, but all registers
are 16-bit and there are less instructions available.

\label{IN_example}
\label{OUT_example}
\myindex{x86!\Instructions!IN}
\myindex{x86!\Instructions!OUT}

The MS-DOS environment has no system drivers, and any program can deal with the bare hardware via ports,
so here you can see the \TT{OUT}/\TT{IN} instructions, which are present in mostly in drivers in our times
(it is impossible to access ports directly in \gls{user mode} on all modern \ac{OS}es).

Given that, the MS-DOS program which works with a dongle has to access the LPT printer port directly.

So we can just search for such instructions. And yes, here they are:

\lstinputlisting[style=customasmx86]{examples/dongles/3/1.lst}

(All label names in this example were given by me).

\TT{out\_port()} is referenced only in one function:

\lstinputlisting[style=customasmx86]{examples/dongles/3/2.lst}

This is again a Sentinel Pro \q{hashing} dongle as in the previous example.
It is noticeably because text strings are passed here, too, and 16 bit values are returned and compared with others.

So that is how Sentinel Pro is accessed via ports.

The output port address is usually 0x378, i.e.,
the printer port, where the data to the old printers in pre-USB era was passed to.

The port is uni-directional, because when it was developed, no one imagined that someone
will need to transfer information from the printer
\footnote{If we consider Centronics only. The following IEEE 1284 standard allows the transfer of information from
the printer.}.

The only way to get information from the printer is a status register on port 0x379, which contains
such bits as \q{paper out}, \q{ack}, \q{busy}---thus the printer may signal to the host computer
if it is ready or not and if paper is present in it.

So the dongle returns information from one of these bits, one bit at each iteration.

\TT{\_in\_port\_2} contains the address of the status word (0x379) and 
\TT{\_in\_port\_1} contains the control register address (0x37A).

It seems that the dongle returns information via the \q{busy} flag at \TT{seg030:00B9}: 
each bit is stored in the \TT{DI} register, which is returned at the end of the function.

What do all these bytes sent to output port mean?
Hard to say. Perhaps, commands to the dongle.

But generally speaking, it is not necessary to know: it is easy to solve our task without that knowledge.

Here is the dongle checking routine:

\lstinputlisting[style=customasmx86]{examples/dongles/3/3.lst}

Since the routine can be called very frequently, e.g., before the execution of each important software feature, 
and accessing the dongle is generally slow (because of the slow printer port and also slow
\ac{MCU} in the dongle), they probably added a way to skip some dongle checks,
by checking the current time in the \TT{biostime()} function.

\myindex{\CStandardLibrary!rand()}
The \TT{get\_rand()} 
function uses the standard C function:

\lstinputlisting[style=customasmx86]{examples/dongles/3/4.lst}

So the text string is selected randomly, passed into the dongle, and then the result of the hashing 
is compared with the correct value.

The text strings seem to be constructed randomly as well, during software development.

And this is how the main dongle checking function is called:

\lstinputlisting[style=customasmx86]{examples/dongles/3/5.lst}

Bypassing the dongle is easy, just force the \TT{check\_dongle()} function to always return 0.

For example, by inserting this code at its beginning:

\begin{lstlisting}[style=customasmx86]
mov ax,0
retf
\end{lstlisting}

\myindex{\CStandardLibrary!strcpy()}

The observant reader might recall that the \TT{strcpy()} C function usually requires two pointers in its arguments,
but we see that 4 values are passed:

\begin{lstlisting}[style=customasmx86]
seg033:088F 1E                          push    ds
seg033:0890 68 22 44                    push    offset aTrupcRequiresA ; "This Software Requires a Software Lock\n"
seg033:0893 1E                          push    ds
seg033:0894 68 60 E9                    push    offset byte_6C7E0 ; dest
seg033:0897 9A 79 65 00+                call    _strcpy
seg033:089C 83 C4 08                    add     sp, 8
\end{lstlisting}

This is related to MS-DOS' memory model. You can read more about it here: 
\myref{8086_memory_model}.

So as you may see, \TT{strcpy()} and any other function that take pointer(s) in arguments
work with 16-bit pairs.

Let's get back to our example.
\TT{DS} is currently set to the data segment located in the executable,
that is where the text string is stored.

\myindex{x86!\Instructions!LES}

In the \TT{sent\_pro()} function, each byte of the string is loaded at\\
\TT{seg030:00EF}: the \TT{LES} instruction
loads the ES:BX pair simultaneously from the passed argument.

The \MOV at \TT{seg030:00F5} loads the byte from the memory at which the ES:BX pair points.

% TODO rewrite
%
%At \TT{seg030:00F2} only a second 16-bit part of address is \glslink{increment}{incremented}.
%
%This implies that the string passed to the function cannot be located on the boundary between two data segments.

