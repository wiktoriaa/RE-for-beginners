% TODO: OpenSSL tool, URLs, etc
\section{Случай с зашифрованной БД \#1}
\label{encrypted_DB1}

(Эта часть впервые появилась в моем блоге 26-Aug-2015.
Обсуждение: \url{https://news.ycombinator.com/item?id=10128684}.)

\subsection{Base64 и энтропия}

\myindex{XML}
Мне достался \ac{XML}-файл, содержащий некоторые зашифрованные данные.
Вероятно, что-то связанное с заказми и/или с информацией о клиентах.

\begin{lstlisting}
<?xml version = "1.0" encoding = "UTF-8"?>
<Orders>
	<Order>
		<OrderID>1</OrderID>
		<Data>yjmxhXUbhB/5MV45chPsXZWAJwIh1S0aD9lFn3XuJMSxJ3/E+UE3hsnH</Data>
	</Order>
	<Order>
		<OrderID>2</OrderID>
		<Data>0KGe/wnypFBjsy+U0C2P9fC5nDZP3XDZLMPCRaiBw9OjIk6Tu5U=</Data>
	</Order>
	<Order>
		<OrderID>3</OrderID>
		<Data>mqkXfdzvQKvEArdzh+zD9oETVGBFvcTBLs2ph1b5bYddExzp</Data>
	</Order>
	<Order>
		<OrderID>4</OrderID>
		<Data>FCx6JhIDqnESyT3HAepyE1BJ3cJd7wCk+APCRUeuNtZdpCvQ2MR/7kLXtfUHuA==</Data>
	</Order>
...
\end{lstlisting}

Файл доступен \href{https://raw.githubusercontent.com/dennis714/RE-for-beginners/master/examples/encrypted_DB1/encrypted.xml}{здесь}.

\myindex{base64}
Это явно данные закодированные в base64, потому что все строки состоят из латинских символов, цифр,
и символов плюс (+) и слэш (/).
Могут быть еще два выравнивающих символа (=), но они никогда не встречаются в середине строки.
Зная эти свойства base64, такие строки легко распозновать.

Попробуем декодировать эти блоки и вычислить их энтропии (\myref{entropy}) при помощи Wolfram Mathematica:

\begin{lstlisting}
In[]:= ListOfBase64Strings = 
  Map[First[#[[3]]] &, Cases[Import["encrypted.xml"], XMLElement["Data", _, _], Infinity]];

In[]:= BinaryStrings = 
  Map[ImportString[#, {"Base64", "String"}] &, ListOfBase64Strings];

In[]:= Entropies = Map[N[Entropy[2, #]] &, BinaryStrings];

In[]:= Variance[Entropies]
Out[]= 0.0238614
\end{lstlisting}

\myindex{Variance}
Разброс (variance) низкий.
Это означает, что значения энтропии не очень отличаются друг от друга.
Это видно на графике:

\begin{lstlisting}
In[]:= ListPlot[Entropies]
\end{lstlisting}

\begin{figure}[H]
\centering
\myincludegraphics{examples/encrypted_DB1/entropy.png}
\end{figure}

Большинство значений между 5.0 и 5.4.
Это свидетельство того что данные сжаты и/или зашифрованы.

Чтобы понять разброс (variance), подсчитаем энтропии всех строк в книге Конана Дойля \IT{The Hound of the Baskervilles}:

\begin{lstlisting}
In[]:= BaskervillesLines = Import["http://www.gutenberg.org/cache/epub/2852/pg2852.txt", "List"];

In[]:= EntropiesT = Map[N[Entropy[2, #]] &, BaskervillesLines];

In[]:= Variance[EntropiesT]
Out[]= 2.73883

In[]:= ListPlot[EntropiesT]
\end{lstlisting}

\begin{figure}[H]
\centering
\myincludegraphics{examples/encrypted_DB1/conan_doyle.png}
\end{figure}

Большинство значений находится вокруг 4, но есть также м\'{е}ньшие значения, и они повлияли на конечное значение разброса.

Вероятно, самые короткие строки имеют м\'{е}ньшую энтропию, попробуем короткую строку из книги Конан Дойля:

\begin{lstlisting}
In[]:= Entropy[2, "Yes, sir."] // N
Out[]= 2.9477
\end{lstlisting}

Попробуем еще м\'{е}ньшую:

\begin{lstlisting}
In[]:= Entropy[2, "Yes"] // N
Out[]= 1.58496

In[]:= Entropy[2, "No"] // N
Out[]= 1.
\end{lstlisting}

\subsection{Данные сжаты?}

ОК, наши данные сжаты и/или зашифрованы.
Сжаты ли? Почти все компрессоры данных помещают некоторый заголовок в начале, сигнатуру или что-то вроде этого.
Как видим, здесь ничего такого нет в начале каждого блока.
Все еще возможно что это какой-то самодельный компрессор, но они очень редки.
С другой стороны, самодельные криптоалгоритмы попадаются часто, потому что их куда легче заставить работать.
\myindex{memfrob()}
\myindex{ROT13}
Даже примитивные криптосистемы без ключей, как \IT{memfrob()}\footnote{\url{http://linux.die.net/man/3/memfrob}}
и ROT13 нормально работают без ошибок.
А чтобы написать свой компрессор с нуля, используя только фантазию и воображение, так что он будет работать без ошибок,
это серьезная задача.
Некоторые программисты реализуют ф-ции сжатия данных по учебникам, но это также редкость.
Наиболее популярные способы это:
\myindex{zlib}
1) просто взять опен-сорсную библиотеку вроде zlib;
2) скопипастить что-то откуда-то.
Опен-сорсные алгоритмы сжатия данных обычно добавляют какой-то заголовок, и точно так же делают алгоритмы с сайтов вроде
\url{http://www.codeproject.com/}.

\subsection{Данные зашифрованы?}

Основные алгоритмы шифрования обрабатывают данные блоками. DES --- по 8 байт, AES --- по 16 байт.
Если входной буфер не делится без остатка на длину блока, он дополняется нулями (или еще чем-то), так что зашифрованные
данные будут выровнены по размеру блока этого алгоритма шифрования.
Это не наш случай.

Используя Wolfram Mathematica, я проанализировал длины блоков:

\begin{lstlisting}
In[]:= Counts[Map[StringLength[#] &, BinaryStrings]]
Out[]= <|42 -> 1858, 38 -> 1235, 36 -> 699, 46 -> 1151, 40 -> 1784, 
 44 -> 1558, 50 -> 366, 34 -> 291, 32 -> 74, 56 -> 15, 48 -> 716, 
 30 -> 13, 52 -> 156, 54 -> 71, 60 -> 3, 58 -> 6, 28 -> 4|>
\end{lstlisting}

1858 блоков имеют длину 42 байта, 1235 блоков имеют длину 38 байт, итд.

Я сделал график:

\begin{lstlisting}
ListPlot[Counts[Map[StringLength[#] &, BinaryStrings]]]
\end{lstlisting}

\begin{figure}[H]
\centering
\myincludegraphics{examples/encrypted_DB1/lengths.png}
\end{figure}

Так что большинство блоков имеют размер между $\textasciitilde{}36$ и $\textasciitilde{}48$.
Вот еще что стоит отметить: длины всех блоков четные.
Нет ни одного блока с нечетной длиной.

Хотя, существуют потоковые шифры, которые работают на уровне байт, или даже на уровне бит.

\subsection{CryptoPP}
\myindex{CryptoPP}

Программа, при помощи которой можно листать зашифрованную базу написана на C\# и код на .NET сильно обфусцирован.
Тем не менее, имеется DLL с кодом для x86, который, после краткого рассмотрения,
имеет части из популярной опен-сорсной библиотеки CryptoPP!
(Я просто нашел внутри строки \q{CryptoPP}.)
Теперь легко найти все ф-ции внутри DLL, потому что библиотека CryptoPP опен-сорсная.

\myindex{AES}
Библиотека CryptoPP имеет множество ф-ций шифрования, включая AES (AKA Rijndael).
Современные x86-процессоры имеют AES-инструкции вроде \INS{AESENC}, \INS{AESDEC} и \INS{AESKEYGENASSIST}
\footnote{\url{https://en.wikipedia.org/wiki/AES_instruction_set}}.
Они не производят полного шифрования/дешифрования, но они делают б\'{о}льшую часть работы.
И новые версии CryptoPP используют их.
Например, здесь:
\href{https://github.com/mmoss/cryptopp/blob/2772f7b57182b31a41659b48d5f35a7b6cedd34d/src/rijndael.cpp#L1034}{1},
\href{https://github.com/mmoss/cryptopp/blob/2772f7b57182b31a41659b48d5f35a7b6cedd34d/src/rijndael.cpp#L1000}{2}.
\myindex{x86!\Instructions!AESENC}
\myindex{x86!\Instructions!AESDEC}
\myindex{tracer}
К моему удивлению, во время дешифрования, инструкция, \INS{AESENC} исполняется, а \INS{AESDEC} --- нет
(я это проверил при помощи моей утилиты tracer, но можно использовать любой отладчик).
Я проверил, поддерживает ли мой процессор AES-инструкции. Некоторые процессоры Intel i3 не поддерживают.
И если нет, библиотека CryptoPP применяет ф-ции AES реализованные старым способом
\footnote{\url{https://github.com/mmoss/cryptopp/blob/2772f7b57182b31a41659b48d5f35a7b6cedd34d/src/rijndael.cpp#L355}}.
Но мой процессор поддерживает их.
Почему \INS{AESDEC} не исполняется?
Почему программа использует шифрование AES чтобы дешифровать БД?

ОК, найти ф-цию шифрования блока это не проблема.
Она называется \IT{CryptoPP::Rijndael::Enc::ProcessAndXorBlock}:\\
\url{https://github.com/mmoss/cryptopp/blob/2772f7b57182b31a41659b48d5f35a7b6cedd34d/src/rijndael.cpp#L349},
и она может вызывать другую ф-цию: \IT{Rijndael::Enc::AdvancedProcessBlocks()}\\
\url{https://github.com/mmoss/cryptopp/blob/2772f7b57182b31a41659b48d5f35a7b6cedd34d/src/rijndael.cpp#L1179},
которая, в свою очередь, может вызывать две ф-ции (
\href{https://github.com/mmoss/cryptopp/blob/2772f7b57182b31a41659b48d5f35a7b6cedd34d/src/rijndael.cpp#L1000}{AESNI\_Enc\_Block}
and 
\href{https://github.com/mmoss/cryptopp/blob/2772f7b57182b31a41659b48d5f35a7b6cedd34d/src/rijndael.cpp#L1012}{AESNI\_Enc\_4\_Blocks}
)
которые имеют инструкции \INS{AESENC}.

Так что, судя по внутренностям CryptoPP, \\
\IT{CryptoPP::Rijndael::Enc::ProcessAndXorBlock()} шифрует один 16-байтный блок.
Попробуем установить брякпоинт на ней и посмотрим, что происходит во время дешифрования.
Я снова использую мою простую утилиту tracer.
Сейчас программа должна дешифровать первый блок.
О, и кстати, вот первый блок сконвертированный из кодировки base64 в шестнадцатеричный вид, будем держать его под рукой:

\lstinputlisting{examples/encrypted_DB1/1.lst}

А еще вот аргументы ф-ции из исходных файлов CryptoPP:

\begin{lstlisting}
size_t Rijndael::Enc::AdvancedProcessBlocks(const byte *inBlocks, const byte *xorBlocks, byte *outBlocks, size_t length, word32 flags);
\end{lstlisting}

Так что у него 5 аргументов. Возможные флаги это:

\begin{lstlisting}
enum {BT_InBlockIsCounter=1, BT_DontIncrementInOutPointers=2, BT_XorInput=4, BT_ReverseDirection=8, BT_AllowParallel=16} FlagsForAdvancedProcessBlocks;
\end{lstlisting}

ОК, запускаем tracer на ф-ции \IT{ProcessAndXorBlock()}:

\lstinputlisting{examples/encrypted_DB1/2.lst}

Тут мы можем увидить входы в ф-цию \IT{ProcessAndXorBlock()}, и выходы из нее.

Это вывод из ф-ции во время первого вызова:

\begin{lstlisting}
00000000: C7 39 4E 7B 33 1B D6 1F-B8 31 10 39 39 13 A5 5D ".9N{3....1.99..]"
\end{lstlisting}

Затем ф-ция \IT{ProcessAndXorBlock()} вызывается с блоком нулевого размера, но с флагом 8 (\IT{BT\_ReverseDirection}).

Второй вызов:

\begin{lstlisting}
00000000: 45 00 20 00 4A 00 4F 00-48 00 4E 00 53 00 00 00 "E. .J.O.H.N.S..."
\end{lstlisting}

Ох, тут есть знакомая нам строка!

Третий вызов:

\begin{lstlisting}
00000000: B1 27 7F E4 9F 01 E3 81-CF C6 12 FB B9 7C F1 BC ".'...........|.."
\end{lstlisting}

Первый вывод очень похож на первые 16 байт зашифрованного буфера.

Вывод первого вызова \IT{ProcessAndXorBlock()}:

\begin{lstlisting}
00000000: C7 39 4E 7B 33 1B D6 1F-B8 31 10 39 39 13 A5 5D ".9N{3....1.99..]"
\end{lstlisting}

Первые 16 байт зашифрованного буфера:

\begin{lstlisting}
00000000: CA 39 B1 85 75 1B 84 1F F9 31 5E 39 72 13 EC 5D  .9..u....1^9r..]
\end{lstlisting}

Тут слишком много одинаковых байт!
Как так получается, что результат шифрования AES может быть очень похож на шифрованный буфер в то время как это
не шифрование, а скорее дешифрование?

\subsection{Режим обратной связи по шифротексту}

\myindex{Режим обратной связи по шифротексту}
\myindex{XOR}
Ответ это \ac{CFB}:
в этом режиме, алгоритм AES используются не как алгоритм шифрования, а как устройство для генерации случайных данных
с криптографической стойкостью.
Само шифрование производится используя простую операцию XOR.

Вот алгоритм шифрования (иллюстрации взяты из Wikipedia):

\begin{figure}[H]
\centering
\myincludegraphics{examples/encrypted_DB1/601px-CFB_encryption.svg.png}
\end{figure}

И дешифрования:

\begin{figure}[H]
\centering
\myincludegraphics{examples/encrypted_DB1/601px-CFB_decryption.svg.png}
\label{fig:CFB_decryption}
\end{figure}

Посмотрим: операция шифрования в AES генерирует 16 байт (или 128 бит) \IT{случайных} данных,
которые можно использовать во время применения операции XOR, но кто заставляет нас использовать все 16 байт?
Если на последней итерации у нас 1 байт данных, давайте про-XOR-им 1 байт данных с 1 байтом сгенерированных
\IT{случайных} данных?
Это приводит к важному свойству режима \ac{CFB}: данные не нужно выравнивать, данные произвольного размера
могут быть зашифрованы и дешифрованы.

О, и вот почему все шифрованные блоки не выровнены.
И вот почему инструкция \INS{AESDEC} никогда не вызывается.

Давайте попробуем дешифровать первый блок вручную, используя Питон.
Режим \ac{CFB} также использует \ac{IV}, как \IT{seed} для \ac{CSPRNG}.
В нашем случае, \ac{IV} это блок, который шифруется на первой итерации:

\begin{lstlisting}
0038B920: 01 00 00 00 FF FF FF FF-79 C1 69 0B 67 C1 04 7D "........y.i.g..}"
\end{lstlisting}

О, и нам нужно также восстановить ключ шифрования.
\myindex{x86!\Instructions!AESKEYGENASSIST}
В DLL есть \INS{AESKEYGENASSIST}, и она вызывается, и используется в ф-ции \\
\IT{Rijndael::Base::UncheckedSetKey()}:\\
\url{https://github.com/mmoss/cryptopp/blob/2772f7b57182b31a41659b48d5f35a7b6cedd34d/src/rijndael.cpp#L198}
Её легко найти в IDA и установить брякпойнт. Посмотрим:

\begin{lstlisting}
... tracer.exe -l:filename.exe bpf=filename.exe!0x435c30,args:3,dump_args:0x10

Warning: no tracer.cfg file.
PID=2068|New process software.exe
no module registered with image base 0x77320000
no module registered with image base 0x76e20000
no module registered with image base 0x77320000
no module registered with image base 0x77220000
Warning: unknown (to us) INT3 breakpoint at ntdll.dll!LdrVerifyImageMatchesChecksum+0x96c (0x776c103b)
(0) software.exe!0x435c30(0x15e8000, 0x10, 0x14f808) (called from software.exe!.text+0x22fa1 (0x13d3fa1))
Argument 1/3 
015E8000: CD C5 7E AD 28 5F 6D E1-CE 8F CC 29 B1 21 88 8E "..~.(_m....).!.."
Argument 3/3 
0014F808: 38 82 58 01 C8 B9 46 00-01 D1 3C 01 00 F8 14 00 "8.X...F...<....."
Argument 3/3 +0x0: software.exe!.rdata+0x5238
Argument 3/3 +0x8: software.exe!.text+0x1c101
(0) software.exe!0x435c30() -> 0x13c2801
PID=2068|Process software.exe exited. ExitCode=0 (0x0)
\end{lstlisting}

Так вот это ключ: \IT{CD C5 7E AD 28 5F 6D E1-CE 8F CC 29 B1 21 88 8E}.

Во время ручного дешифрования мы получаем это:

\begin{lstlisting}
00000000: 0D 00 FF FE 46 00 52 00  41 00 4E 00 4B 00 49 00  ....F.R.A.N.K.I.
00000010: 45 00 20 00 4A 00 4F 00  48 00 4E 00 53 00 66 66  E. .J.O.H.N.S.ff
00000020: 66 66 66 9E 61 40 D4 07  06 01                    fff.a@....
\end{lstlisting}

Теперь это что-то читаемое!
И теперь мы видим, почему было так много одинаковых байт во время первой итерации дешифрования:
потому что в оригинальном тексте так много нулевых байт!

Дешифруем второй блок:

\begin{lstlisting}
00000000: 17 98 D0 84 3A E9 72 4F  DB 82 3F AD E9 3E 2A A8  ....:.rO..?..>*.
00000010: 41 00 52 00 52 00 4F 00  4E 00 CD CC CC CC CC CC  A.R.R.O.N.......
00000020: 1B 40 D4 07 06 01                                 .@....
\end{lstlisting}

Третий, четвертый и пятый:

\begin{lstlisting}
00000000: 5D 90 59 06 EF F4 96 B4  7C 33 A7 4A BE FF 66 AB  ].Y.....|3.J..f.
00000010: 49 00 47 00 47 00 53 00  00 00 00 00 00 C0 65 40  I.G.G.S.......e@
00000020: D4 07 06 01                                       ....
\end{lstlisting}

\begin{lstlisting}
00000000: D3 15 34 5D 21 18 7C 6E  AA F8 2D FE 38 F9 D7 4E  ..4]!.|n..-.8..N
00000010: 41 00 20 00 44 00 4F 00  48 00 45 00 52 00 54 00  A. .D.O.H.E.R.T.
00000020: 59 00 48 E1 7A 14 AE FF  68 40 D4 07 06 02        Y.H.z...h@....
\end{lstlisting}

\begin{lstlisting}
00000000: 1E 8B 90 0A 17 7B C5 52  31 6C 4E 2F DE 1B 27 19  .....{.R1lN...'.
00000010: 41 00 52 00 43 00 55 00  53 00 00 00 00 00 00 60  A.R.C.U.S.......
00000020: 66 40 D4 07 06 03                                 f@....
\end{lstlisting}

Все блоки, похоже, дешифруются корректно, но не первые 16 байт.

\subsection{Инициализирующий вектор}

Что влияет на первые 16 байт?

Вернемся снова к алгоритму дешифрования \ac{CFB}: \myref{fig:CFB_decryption}.

Мы видим что \ac{IV} может влиять на первую операцию дешифрования, но не на вторую,
потому что во время второй итерации используется шифротекст от первой итерации, и в случае дешифрования,
он такой же, не важно, какой был \ac{IV}!

Так что, вероятно, \ac{IV} каждый раз разный.
Используя мой tracer, я буду смотреть на первый вход во время дешифрования второго блока \ac{XML}-файла:

\begin{lstlisting}
0038B920: 02 00 00 00 FE FF FF FF-79 C1 69 0B 67 C1 04 7D "........y.i.g..}"
\end{lstlisting}

\dots third:

\begin{lstlisting}
0038B920: 03 00 00 00 FD FF FF FF-79 C1 69 0B 67 C1 04 7D "........y.i.g..}"
\end{lstlisting}

Похоже, первый и пятый байт каждый раз меняется.
Я в итоге разобрался, что первое 32-битное число это просто OrderID из \ac{XML}-файла,
и второе 32-битное число это тоже OrderID, но с отрицательным знаком. Остальные 8 байт не меняются.
И вот я расшифровал всю БД:
\url{https://raw.githubusercontent.com/dennis714/RE-for-beginners/master/examples/encrypted_DB1/decrypted.full.txt}.

Питоновский скрипт, который я использовал:
\url{https://github.com/dennis714/RE-for-beginners/blob/master/examples/encrypted_DB1/decrypt_blocks.py}.

Вероятно, автор хотел чтобы каждый блок шифровался немного иначе, так что он/она использовал OrderID как часть ключа.
А еще можно было бы делать разный ключ для AES вместо \ac{IV}.

Так что теперь мы знаем, что \ac{IV} влияет только на первый блок во время дешифрования в режиме \ac{CFB},
это его особенность.
Остальные блоки можно дешифровать не зная \ac{IV}, но используя ключ.

ОК, но почему режим \ac{CFB}? Очевидно, потому что самый первый пример на AES в CryptoPP wiki
использует режим \ac{CFB}:
\url{http://www.cryptopp.com/wiki/Advanced_Encryption_Standard#Encrypting_and_Decrypting_Using_AES}.
Вероятно, разработчик выбрал его из-за простоты:
пример может шифровать/дешифровать текстовые строки произвольной длины, без выравнивания.

Очень похоже что автор этой программы просто скопипастил пример из старницы в CryptoPP wiki.
Многие программисты так и делают.

Разница только в том что в примере в CryptoPP wiki \ac{IV} выбирается случайно, в то время как подобный индетерминизм
не был допустимым для автора программы, которую мы сейчас разбираем,
так что они решили инициализировать \ac{IV} используя ID заказа.

Теперь мы можем идти дальше, анализировать значение каждого байта дешифрованного блока.

\subsection{Структура буфера}

Возьмем первые 4 байта дешифрованных блоков:

\begin{lstlisting}
00000000: 0D 00 FF FE 46 00 52 00  41 00 4E 00 4B 00 49 00  ....F.R.A.N.K.I.
00000010: 45 00 20 00 4A 00 4F 00  48 00 4E 00 53 00 66 66  E. .J.O.H.N.S.ff
00000020: 66 66 66 9E 61 40 D4 07  06 01                    fff.a@....

00000000: 0B 00 FF FE 4C 00 4F 00  52 00 49 00 20 00 42 00  ....L.O.R.I. .B.
00000010: 41 00 52 00 52 00 4F 00  4E 00 CD CC CC CC CC CC  A.R.R.O.N.......
00000020: 1B 40 D4 07 06 01                                 .@....

00000000: 0A 00 FF FE 47 00 41 00  52 00 59 00 20 00 42 00  ....G.A.R.Y. .B.
00000010: 49 00 47 00 47 00 53 00  00 00 00 00 00 C0 65 40  I.G.G.S.......e@
00000020: D4 07 06 01                                       ....

00000000: 0F 00 FF FE 4D 00 45 00  4C 00 49 00 4E 00 44 00  ....M.E.L.I.N.D.
00000010: 41 00 20 00 44 00 4F 00  48 00 45 00 52 00 54 00  A. .D.O.H.E.R.T.
00000020: 59 00 48 E1 7A 14 AE FF  68 40 D4 07 06 02        Y.H.z...h@....
\end{lstlisting}

Легко увидеть строки закодированные в UTF-16, это имена и фамилии.
Первый байт (или 16-битное слово) похоже это просто длина строки, мы можем проверить это визуально.
\IT{FF FE} это, похоже, \ac{BOM} в Уникоде.

После каждой строки есть еще 12 байт.

Используя этот скрипт 
(\url{https://github.com/dennis714/RE-for-beginners/blob/master/examples/encrypted_DB1/dump_buffer_rest.py})
я получил случайную выборку из \IT{хвостов}:

\lstinputlisting{examples/encrypted_DB1/tails.lst}

Видим что байты 0x40 и 0x07 присутствуют в каждом \IT{хвосте}.
Самый последний байт всегда в пределах 1..0x1F (1..31), я проверил.
Предпоследний байт всегда в пределах 1..0xC (1..12).
Ух, это выглядит как дата!
Год может быть представлен как 16-битное значение, и может быть последние 4 байта это дата (16 бит для года, 8 бит для
месяца и еще 8 для дня)?
0x7DD это 2013, 0x7D5 это 2005, итд. Похоже нормально. Это дата.
Там есть еще 8 байт.
Судя по тому факту что БД называется \IT{orders} (заказы), может быть здесь присутствует сумма?
Я сделал попытку интерпретировать их как числа с плавающей точкой двойной точности в формате IEEE 754, и вывести все значения!

Некоторые:

\begin{lstlisting}
71.0
134.0
51.95
53.0
121.99
96.95
98.95
15.95
85.95
184.99
94.95
29.95
85.0
36.0
130.99
115.95
87.99
127.95
114.0
150.95
\end{lstlisting}

Похоже на правду!

Теперь мы можем вывест имена, суммы и даты.

\begin{lstlisting}
plain:
00000000: 0D 00 FF FE 46 00 52 00  41 00 4E 00 4B 00 49 00  ....F.R.A.N.K.I.
00000010: 45 00 20 00 4A 00 4F 00  48 00 4E 00 53 00 66 66  E. .J.O.H.N.S.ff
00000020: 66 66 66 9E 61 40 D4 07  06 01                    fff.a@....
OrderID= 1 name= FRANKIE JOHNS sum= 140.95 date= 2004 / 6 / 1

plain:
00000000: 0B 00 FF FE 4C 00 4F 00  52 00 49 00 20 00 42 00  ....L.O.R.I. .B.
00000010: 41 00 52 00 52 00 4F 00  4E 00 CD CC CC CC CC CC  A.R.R.O.N.......
00000020: 1B 40 D4 07 06 01                                 .@....
OrderID= 2 name= LORI BARRON sum= 6.95 date= 2004 / 6 / 1

plain:
00000000: 0A 00 FF FE 47 00 41 00  52 00 59 00 20 00 42 00  ....G.A.R.Y. .B.
00000010: 49 00 47 00 47 00 53 00  00 00 00 00 00 C0 65 40  I.G.G.S.......e@
00000020: D4 07 06 01                                       ....
OrderID= 3 name= GARY BIGGS sum= 174.0 date= 2004 / 6 / 1

plain:
00000000: 0F 00 FF FE 4D 00 45 00  4C 00 49 00 4E 00 44 00  ....M.E.L.I.N.D.
00000010: 41 00 20 00 44 00 4F 00  48 00 45 00 52 00 54 00  A. .D.O.H.E.R.T.
00000020: 59 00 48 E1 7A 14 AE FF  68 40 D4 07 06 02        Y.H.z...h@....
OrderID= 4 name= MELINDA DOHERTY sum= 199.99 date= 2004 / 6 / 2

plain:
00000000: 0B 00 FF FE 4C 00 45 00  4E 00 41 00 20 00 4D 00  ....L.E.N.A. .M.
00000010: 41 00 52 00 43 00 55 00  53 00 00 00 00 00 00 60  A.R.C.U.S.......
00000020: 66 40 D4 07 06 03                                 f@....
OrderID= 5 name= LENA MARCUS sum= 179.0 date= 2004 / 6 / 3
\end{lstlisting}

См. еще: \url{https://raw.githubusercontent.com/dennis714/RE-for-beginners/master/examples/encrypted_DB1/decrypted.full.with_data.txt}.
Или отфильтрованные: \url{https://github.com/dennis714/RE-for-beginners/blob/master/examples/encrypted_DB1/decrypted.short.txt}.
Похоже всё корректно.

Это что-то вроде сериализации в \ac{OOP}, т.е., запаковка значений с разными типами в бинарный буфер для хранения и/или
передачи.

\subsection{Шум в конце}

Остался только один вопрос, иногда \IT{хвост} длиннее:

\begin{lstlisting}
00000000: 0E 00 FF FE 54 00 48 00  45 00 52 00 45 00 53 00  ....T.H.E.R.E.S.
00000010: 45 00 20 00 54 00 55 00  54 00 54 00 4C 00 45 00  E. .T.U.T.T.L.E.
00000020: 66 66 66 66 66 1E 63 40  D4 07 07 1A 00 07 07 19  fffff.c@........
OrderID= 172 name= THERESE TUTTLE sum= 152.95 date= 2004 / 7 / 26
\end{lstlisting}

(Байты \IT{00 07 07 19} не используются и являются балластом.)

\begin{lstlisting}
00000000: 0C 00 FF FE 4D 00 45 00  4C 00 41 00 4E 00 49 00  ....M.E.L.A.N.I.
00000010: 45 00 20 00 4B 00 49 00  52 00 4B 00 00 00 00 00  E. .K.I.R.K.....
00000020: 00 20 64 40 D4 07 09 02  00 02                    . d@......
OrderID= 286 name= MELANIE KIRK sum= 161.0 date= 2004 / 9 / 2
\end{lstlisting}

(\IT{00 02} не используются.)

После близкого рассмотрения мы можем видеть, что шум в конце \IT{хвоста} просто остался от предыдущего
шифрования!

Вот два идущих подряд буфера:

\begin{lstlisting}
00000000: 10 00 FF FE 42 00 4F 00  4E 00 4E 00 49 00 45 00  ....B.O.N.N.I.E.
00000010: 20 00 47 00 4F 00 4C 00  44 00 53 00 54 00 45 00   .G.O.L.D.S.T.E.
00000020: 49 00 4E 00 9A 99 99 99  99 79 46 40 D4 07 07 19  I.N......yF@....
OrderID= 171 name= BONNIE GOLDSTEIN sum= 44.95 date= 2004 / 7 / 25

00000000: 0E 00 FF FE 54 00 48 00  45 00 52 00 45 00 53 00  ....T.H.E.R.E.S.
00000010: 45 00 20 00 54 00 55 00  54 00 54 00 4C 00 45 00  E. .T.U.T.T.L.E.
00000020: 66 66 66 66 66 1E 63 40  D4 07 07 1A 00 07 07 19  fffff.c@........
OrderID= 172 name= THERESE TUTTLE sum= 152.95 date= 2004 / 7 / 26
\end{lstlisting}

(Последние байты \IT{07 07 19} скопированы из предыдущего незашифрованного буфера.)

Еще два подряд идущих буфера:

\begin{lstlisting}
00000000: 0D 00 FF FE 4C 00 4F 00  52 00 45 00 4E 00 45 00  ....L.O.R.E.N.E.
00000010: 20 00 4F 00 54 00 4F 00  4F 00 4C 00 45 00 CD CC   .O.T.O.O.L.E...
00000020: CC CC CC 3C 5E 40 D4 07  09 02                    ...<^@....
OrderID= 285 name= LORENE OTOOLE sum= 120.95 date= 2004 / 9 / 2

00000000: 0C 00 FF FE 4D 00 45 00  4C 00 41 00 4E 00 49 00  ....M.E.L.A.N.I.
00000010: 45 00 20 00 4B 00 49 00  52 00 4B 00 00 00 00 00  E. .K.I.R.K.....
00000020: 00 20 64 40 D4 07 09 02  00 02                    . d@......
OrderID= 286 name= MELANIE KIRK sum= 161.0 date= 2004 / 9 / 2
\end{lstlisting}

Последний байт 02 был скопирован из предыдущего незашифрованного буфера.

Возможно, что буфер использующися для шифрования глобальный и/или не очищается перед каждым шифрованием.
Размер последнего буфера тоже как-то хаотично меняется, тем не менее, ошибка не была отловлена потому что она не влияет
на процесс дешифрования, который просто игнорирует шум в конце.
Эта распространенная ошибка.
\myindex{OpenSSL}
\myindex{Heartbleed}
Она была даже в OpenSSL (ошибка Heartbleed).

\subsection{Вывод}

Итог:
каждый практикующий реверс-инженер должен быть знаком с основными алгоритмами шифрования, а также с основными режимами
шифрования.
Некоторые книги об этом: \myref{crypto_books}.

\IT{Зашифрованное} содержимое БД было искусственно мною создано ради демонстрации.
Я использовал наиболее популярные имена и фамилии в США, отсюда: \url{http://stackoverflow.com/questions/1803628/raw-list-of-person-names},
и скомбинировал их случайным образом.
Даты и суммы были сгенерированы случайным образом.

Все файлы использованные в этой части здесь:
\url{https://github.com/dennis714/RE-for-beginners/tree/master/examples/encrypted_DB1}.

Тем не менее, многие особенности как здесь, я наблюдал в настоящем ПО.
Этот пример основан на них.

\subsection{Post Scriptum: перебор всех \ac{IV}}

Пример, который вы только что видели, был искусственно создан, но основан на настоящем ПО которое я разбирал.
Когда я над ним работал, я в начале заметил, что \ac{IV} генерируется используя некоторые 32-битное число,
и я не мог найти связь между этим числом и OrderID.
Так что я приготовился использовать полный перебор, который тут действительно возможен.

Это не проблема перебрать все 32-битные значения и попробовать каждое как основу для \ac{IV}.
Затем вы дешифруете первый 16-байтный блок и проверяете нулевые байты, которые всегда находятся на одних и тех же местах.

