\subsection{\IFRU{Передача параметров для функции}{Function arguments passing}}

\begin{lstlisting}
push arg3
push arg2
push arg1
call f
add esp, 4*3
\end{lstlisting}

\IFRU{Вызываемая функция получает свои параметры также через указатель стека.}
{Callee{\footnote{Function being called}} function get its arguments via stack ponter.}

\IFRU{См.также в соответствующем разделе о способах передачи аргументов через стек}
{See also section about calling conventions}~\ref{sec:callingconventions}.

\IFRU{Важно отметить, что, в общем, никто не заставляет программистов передавать параметры именно через стек,
это не является требованием к исполняемому коду.}
{It is important to note that no one oblige programmers to pass arguments through stack, it is not prerequisite.}

\IFRU{Вы можете делать это совершенно иначе, не используя стек.}
{One could implement any other method not using stack.}

\IFRU{К примеру, можно выделять в куче\footnote{heap в англоязычной литературе} место для аргументов, 
заполнять их и передавать в функцию указатель на это место через \EAX. И это вполне будет работать}
{For example, it is possible to allocate a place for arguments in heap, fill it and pass to a function 
via pointer to this pack in \EAX register. And this will work}
\footnote{\IFRU{Например, в книге Дональда Кнута ``Искусство программирования'', в разделе 1.4.1 
посвященном подпрограммам\cite[раздел 1.4.1]{Knuth:1998:ACP:521463}, 
мы можем прочитать о возможности располагать параметры для вызываемой подпрограммы после инструкции \JMP
передающей управление подпрограмме. Кнут описывает что это было особенно удобно для компьютеров System/360.}
{For example, in ``The Art of Computer Programming'' book by Donald Knuth, 
in section 1.4.1 dedicated to subroutines\cite[section 1.4.1]{Knuth:1998:ACP:521463},
we can read about one way to supply arguments to subroutine is simply to list them after the \JMP instruction
passing control to subroutine. Knuth writes that this method was particularly convenient on System/360.}}.

\IFRU{Однако, так традиционно сложилось, что в x86 и ARM передача аргументов происходит именно через стек.}
{However, it is convenient tradition in x86 and ARM to use stack for this.}
