\section{\RU{Защита от переполнения буфера}\EN{Buffer overflow protection methods}}
\label{subsec:BO_protection}

\RU{В наше время пытаются бороться с переполнением буфера невзирая на халатность программистов на \CCpp. 
В MSVC есть опции вроде}%
\EN{There are several methods to protect against this scourge, regardless of the \CCpp programmers' negligence.
MSVC has options like}\footnote{
\RU{описания защит, которые компилятор может вставлять в код}%
\EN{compiler-side buffer overflow protection methods}:
\href{http://go.yurichev.com/17133}{wikipedia.org/wiki/Buffer\_overflow\_protection}}:

\begin{lstlisting}
 /RTCs Stack Frame runtime checking
 /GZ Enable stack checks (/RTCs)
\end{lstlisting}

\index{x86!\Instructions!RET}
\index{Function prologue}
\index{Security cookie}
\RU{Одним из методов является вставка в прологе функции некоего случайного значения в область локальных переменных 
и проверка этого значения в эпилоге функции перед выходом. 
Если проверка не прошла, то не выполнять инструкцию \RET, а остановиться (или зависнуть). 
Процесс зависнет, но это лучше, чем удаленная атака на ваш компьютер.}
\EN{One of the methods is to write a random value between the local variables in stack at function prologue 
and to check it in function epilogue before the function exits.
If value is not the same, do not execute the last instruction \RET, but stop (or hang).
The process will halt, but that is much better than a remote attack to your host.}
    
\newcommand{\CANARYURL}{\RU{\href{http://go.yurichev.com/17135}{miningwiki.ru/wiki/Канарейка\_в\_шахте}}%
\EN{\href{http://go.yurichev.com/17134}{wikipedia.org/wiki/Domestic\_canary\#Miner.27s\_canary}}}

\index{Canary}
\RU{Это случайное значение иногда называют \q{канарейкой}%
\footnote{\q{canary} в англоязычной литературе}, 
по аналогии с шахтной канарейкой\footnote{\CANARYURL}.
Раньше использовали шахтеры, чтобы определять, есть ли в шахте опасный газ.
}
\EN{This random value is called a \q{canary} sometimes, it is related to the miners' canary\footnote{\CANARYURL},
they were used by miners in the past days in order to detect poisonous gases quickly.}
\RU{Канарейки очень к нему чувствительны и либо проявляли сильное беспокойство, либо гибли от газа.}
\EN{Canaries are very sensitive to mine gases, they become very agitated in case of danger, or even die.}

\RU{Если скомпилировать наш простейший пример работы с массивом}
\EN{If we compile our very simple array example}~(\myref{arrays_simple}) \InENRU \ac{MSVC}
\RU{с опцией RTC1 или RTCs}\EN{with RTC1 and RTCs option}, \RU{в конце нашей функции будет вызов 
функции}\EN{you can see a call to}
\TT{@\_RTC\_CheckStackVars@8}\RU{, проверяющей корректность \q{канарейки}.}
\EN{ a function at the end of the function that checks if the \q{canary} is correct.}

\RU{Посмотрим, как дела обстоят в GCC}\EN{Let's see how GCC handles this}. 
\RU{Возьмем пример из секции про}\EN{Let's take an} \TT{alloca()}~(\myref{alloca})\EN{ example}:

\lstinputlisting{patterns/02_stack/04_alloca/2_1.c}

\RU{По умолчанию, без дополнительных ключей, GCC 4.7.3 вставит в код проверку \q{канарейки}:}
\EN{By default, without any additional options, GCC 4.7.3 inserts a \q{canary} check into the code:}

\lstinputlisting[caption=GCC 4.7.3]{patterns/13_arrays/3_BO_protection/gcc_canary.asm.\LANG}

\index{x86!\Registers!GS}
\RU{Случайное значение находится в}\EN{The random value is located in} \TT{gs:20}. 
\RU{Оно записывается в стек, затем, в конце функции, значение в стеке
сравнивается с корректной \q{канарейкой} в}\EN{It gets written on the stack and then at the end of the function
the value in the stack is compared with the correct \q{canary} in} \TT{gs:20}. 
\RU{Если значения не равны, будет вызвана функция}\EN{If the values are not equal, the} 
\TT{\_\_stack\_chk\_fail} \RU{и в консоли мы увидим что-то вроде такого}
\EN{function is called and we can see in the console something like that} (Ubuntu 13.04 x86):

\begin{lstlisting}
*** buffer overflow detected ***: ./2_1 terminated
======= Backtrace: =========
/lib/i386-linux-gnu/libc.so.6(__fortify_fail+0x63)[0xb7699bc3]
/lib/i386-linux-gnu/libc.so.6(+0x10593a)[0xb769893a]
/lib/i386-linux-gnu/libc.so.6(+0x105008)[0xb7698008]
/lib/i386-linux-gnu/libc.so.6(_IO_default_xsputn+0x8c)[0xb7606e5c]
/lib/i386-linux-gnu/libc.so.6(_IO_vfprintf+0x165)[0xb75d7a45]
/lib/i386-linux-gnu/libc.so.6(__vsprintf_chk+0xc9)[0xb76980d9]
/lib/i386-linux-gnu/libc.so.6(__sprintf_chk+0x2f)[0xb7697fef]
./2_1[0x8048404]
/lib/i386-linux-gnu/libc.so.6(__libc_start_main+0xf5)[0xb75ac935]
======= Memory map: ========
08048000-08049000 r-xp 00000000 08:01 2097586    /home/dennis/2_1
08049000-0804a000 r--p 00000000 08:01 2097586    /home/dennis/2_1
0804a000-0804b000 rw-p 00001000 08:01 2097586    /home/dennis/2_1
094d1000-094f2000 rw-p 00000000 00:00 0          [heap]
b7560000-b757b000 r-xp 00000000 08:01 1048602    /lib/i386-linux-gnu/libgcc_s.so.1
b757b000-b757c000 r--p 0001a000 08:01 1048602    /lib/i386-linux-gnu/libgcc_s.so.1
b757c000-b757d000 rw-p 0001b000 08:01 1048602    /lib/i386-linux-gnu/libgcc_s.so.1
b7592000-b7593000 rw-p 00000000 00:00 0
b7593000-b7740000 r-xp 00000000 08:01 1050781    /lib/i386-linux-gnu/libc-2.17.so
b7740000-b7742000 r--p 001ad000 08:01 1050781    /lib/i386-linux-gnu/libc-2.17.so
b7742000-b7743000 rw-p 001af000 08:01 1050781    /lib/i386-linux-gnu/libc-2.17.so
b7743000-b7746000 rw-p 00000000 00:00 0
b775a000-b775d000 rw-p 00000000 00:00 0
b775d000-b775e000 r-xp 00000000 00:00 0          [vdso]
b775e000-b777e000 r-xp 00000000 08:01 1050794    /lib/i386-linux-gnu/ld-2.17.so
b777e000-b777f000 r--p 0001f000 08:01 1050794    /lib/i386-linux-gnu/ld-2.17.so
b777f000-b7780000 rw-p 00020000 08:01 1050794    /lib/i386-linux-gnu/ld-2.17.so
bff35000-bff56000 rw-p 00000000 00:00 0          [stack]
Aborted (core dumped)
\end{lstlisting}

\index{MS-DOS}
gs \RU{это так называемый сегментный регистр. Эти регистры широко использовались во времена MS-DOS 
и DOS-экстендеров.}\EN{is the so-called segment register. These registers were used widely in MS-DOS and DOS-extenders
times.}
\RU{Сейчас их функция немного изменилась.}\EN{Today, its function is different.}
\index{TLS}
\index{Windows!TIB}
\RU{Если говорить кратко, в Linux \TT{gs} всегда указывает на \ac{TLS}~(\myref{TLS})~--- там находится различная 
информация, специфичная для выполняющегося потока.}
\EN{To say it briefly, the \TT{gs} register in Linux always points to the
\ac{TLS}~(\myref{TLS})---some information specific to thread is stored there.}
\RU{Кстати, в win32 эту же роль играет сегментный регистр \TT{fs},
он всегда указывает на}\EN{By the way, in win32
the \TT{fs} register plays the same role, pointing to}
\ac{TIB} \footnote{\href{http://go.yurichev.com/17104}{wikipedia.org/wiki/Win32\_Thread\_Information\_Block}}. 

\RU{Больше информации можно почерпнуть из исходных кодов Linux (по крайней мере, в версии 3.11): 
в файле}\EN{More information can be found in the Linux kernel source code (at least in 3.11 version), in}
\IT{arch/x86/include/asm/stackprotector.h}\RU{ в комментариях описывается эта переменная.}
\EN{ this variable is described in the comments.}

\ifdefined\IncludeARM
\subsection{\OptimizingXcodeIV (\ThumbTwoMode)}

\RU{Возвращаясь к нашему простому примеру}
\EN{Let's get back to our simple array example} (\myref{arrays_simple}),
\RU{можно посмотреть, как LLVM добавит проверку \q{канарейки}:}
\EN{again, now we can see how LLVM checks the correctness of the \q{canary}:}

% TODO shorten the listing a bit? is full display of unrolled loop necessary?
\lstinputlisting{patterns/13_arrays/3_BO_protection/simple_Xcode_thumb_O3.asm.\LANG}

\index{Unrolled loop}
\RU{Во-первых, LLVM \q{развернул} цикл и все значения записываются в массив по одному, 
уже вычисленные, 
потому что LLVM посчитал что так будет быстрее.}
\EN{First of all, as we see, LLVM \q{unrolled} the loop and all values were written into an array one-by-one,
pre-calculated, as LLVM concluded it can work faster.}
\RU{Кстати, инструкции режима ARM позволяют сделать это ещё быстрее и это может быть вашим 
домашним заданием.}\EN{By the way, instructions in ARM mode may help to do this even faster, 
and finding this could be your homework.}

\RU{В конце функции мы видим сравнение \q{канареек}~--- той что лежит в локальном стеке и корректной, 
на которую ссылается регистр \Reg{8}.}
\EN{At the function end we see the comparison of the \q{canaries}---the one in the local stack and the correct one,
to which \Reg{8} points.}
\index{ARM!\Instructions!IT}
\RU{Если они равны, срабатывает блок из четырех инструкций при помощи \INS{ITTTT EQ}.
Это запись 0 в \Reg{0}, эпилог функции и выход из нее.}
\EN{If they are equal to each other, a 4-instruction block is triggered by \INS{ITTTT EQ},
which contains writing 0 in \Reg{0}, the function epilogue and exit.}
\RU{Если \q{канарейки} не равны, блок не срабатывает и происходит
переход на функцию}\EN{If the \q{canaries} are not equal, the block being skipped,
and the jump to} \TT{\_\_\_stack\_chk\_fail}\RU{, которая, вероятно, остановит работу программы.}
\EN{ function will occur, which, perhaps, will halt execution.}
% TODO1 illustrate this!

\fi
