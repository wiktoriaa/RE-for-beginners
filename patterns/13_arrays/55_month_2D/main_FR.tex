\subsection{Ensemble de chaînes comme un tableau à deux dimensions}

Retravaillons la fonction qui renvoie le nom d'un mois: \lstref{get_month1}.

Comme vous le voyez, au moins une opération de chargement en mémoire est nécessaire
pour préparer le pointeur sur la chaîne représentant le nom du mois.

Est-il possible de se passer de cette opération de chargement en mémoire?

En fait oui, si vous représentez la liste de chaînes comme un tableau à deux dimensions:

\lstinputlisting[style=customc]{patterns/13_arrays/55_month_2D/month2_FR.c}

Voici ce que nous obtenons:

\lstinputlisting[caption=MSVC 2013 x64 \Optimizing,style=customasmx86]{patterns/13_arrays/55_month_2D/MSVC2013_x64_Ox_FR.asm}

Il n'y a pas du tout d'accès à la mémoire.

Tout ce que fait cette fonction, c'est de calculer le point où le premier caractère
du nom du mois se trouve:
$pointeur\_sur\_la\_table + mois * 10$.

Il y a deux instructions \LEA, qui fonctionnent en fait comme plusieurs instructions
\MUL et \MOV.

La largeur du tableau est de 10 octets.

En effet, la chaîne la plus longue ici---\q{septembre}---fait 9 octets, plus l'indicateur
de fin de chaîne 0, ça fait 10 octets

Le reste du nom de chaque mois est complété par des zéros, afin d'occuper le même
espace (10 octets).

Donc, notre fonction fonctionne même plus vite, car toutes les chaînes débutent à
une adresse qui peut être facilement calculée.

GCC 4.9 \Optimizing fait encore plus court:

\begin{lstlisting}[caption=GCC 4.9 x64 \Optimizing,style=customasmx86]
	movsx	rdi, edi
	lea	rax, [rdi+rdi*4]
	lea	rax, month2[rax+rax]
	ret
\end{lstlisting}

\LEA est aussi utilisé ici pour la multiplication par 10.

Les compilateurs sans optimisations génèrent la multiplication différemment.

\lstinputlisting[caption=GCC 4.9 x64 \NonOptimizing,style=customasmx86]{patterns/13_arrays/55_month_2D/x64_GCC49_O0_FR.asm}

MSVC \NonOptimizing utilise simplement l'instruction \IMUL:
\myindex{x86!\Instructions!IMUL}

\lstinputlisting[caption=MSVC 2013 x64 \NonOptimizing,style=customasmx86]{patterns/13_arrays/55_month_2D/MSVC2013_x64_FR.asm}

\myindex{\CompilerAnomaly}
\label{MSVC2013_anomaly}

Mais une chose est est curieuse: pourquoi ajouter une multiplication par zéro et
ajouter zéro au résultat final?

Ceci ressemble à une bizarrerie du générateur de code du compilateur, qui n'a pas
été détectée par les tests du compilateur (le code résultant fonctionne correctement
après tout).
% класс!
%
Nous examinons volontairement de tels morceaux de code, afin que le lecteur prenne
conscience qu'il ne doit parfois pas se casser la tête sur des artefacts de compilateur.

\subsubsection{32-bit ARM}

Keil \Optimizing pour le mode Thumb utilise l'instruction de multiplication \INS{MULS}:

\lstinputlisting[caption=\OptimizingKeilVI (\ThumbMode),style=customasmARM]{patterns/13_arrays/55_month_2D/Keil_O3_thumb_FR.asm}

Keil \Optimizing pour mode ARM utilise des instructions d'addition et de décalage:

\lstinputlisting[caption=\OptimizingKeilVI (\ARMMode),style=customasmARM]{patterns/13_arrays/55_month_2D/Keil_O3_ARM_FR.asm}

\subsubsection{ARM64}

\lstinputlisting[caption=GCC 4.9 ARM64 \Optimizing,style=customasmARM]{patterns/13_arrays/55_month_2D/GCC49_ARM64_FR.asm}

\myindex{ARM!\Instructions!SXTW}
\myindex{ARM!\Instructions!ADRP/ADD pair}

\INS{SXTW} est utilisée pour étendre le signe, convertir l'entrée 32-bit en 64-bit
et stocker le résultat dans X0.

La paire \ADRP/\ADD est utilisée pour charger l'adresse de la table.

L'instruction \ADD a aussi un suffixe \LSL, qui aide avec les multiplications.

\subsubsection{MIPS}
\lstinputlisting[caption=GCC 4.4.5 \Optimizing (IDA),style=customasmMIPS]{patterns/13_arrays/55_month_2D/MIPS_O3_IDA_FR.lst}

\subsubsection{\Conclusion{}}

C'est une technique surannée de stocker des chaînes de texte.
Vous pouvez en trouver beaucoup dans \oracle, par exemple.
Il est difficile de dire si ça vaut la peine de le faire sur des ordinateurs modernes.
Néanmoins, c'est un bon exemple de tableaux, donc il a été ajouté à ce livre.

