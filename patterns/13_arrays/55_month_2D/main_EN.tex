\subsection{Pack of strings as a two-dimensional array}

Let's revisit the function that returns the name of a month: \lstref{get_month1}.

As you see, at least one memory load operation is needed to prepare a pointer to the string
that's the month's name.

Is it possible to get rid of this memory load operation?

In fact yes, if you represent the list of strings as a two-dimensional array:

\lstinputlisting[style=customc]{patterns/13_arrays/55_month_2D/month2_EN.c}

Here is what we've get:

\lstinputlisting[caption=\Optimizing MSVC 2013 x64,style=customasmx86]{patterns/13_arrays/55_month_2D/MSVC2013_x64_Ox_EN.asm}

There are no memory accesses at all.

All this function does is to calculate a point at which the first character of the name of the month is: 
$pointer\_to\_the\_table + month * 10$.

There are also two \LEA instructions, which effectively work as several \MUL and \MOV instructions.

The width of the array is 10 bytes. 

Indeed, the longest string here---\q{September}---is 9 bytes, and plus the terminating zero is 10 bytes.

The rest of the month names are padded by zero bytes, so they all occupy the same space (10 bytes).

Thus, our function works even faster, because all string start at an address which can be
calculated easily.

\Optimizing GCC 4.9 can do it even shorter:

\begin{lstlisting}[caption=\Optimizing GCC 4.9 x64,style=customasmx86]
	movsx	rdi, edi
	lea	rax, [rdi+rdi*4]
	lea	rax, month2[rax+rax]
	ret
\end{lstlisting}

\LEA is also used here for multiplication by 10.

Non-optimizing compilers do multiplication differently.

\lstinputlisting[caption=\NonOptimizing GCC 4.9 x64,style=customasmx86]{patterns/13_arrays/55_month_2D/x64_GCC49_O0_EN.asm}

\NonOptimizing MSVC just uses \IMUL instruction:
\myindex{x86!\Instructions!IMUL}

\lstinputlisting[caption=\NonOptimizing MSVC 2013 x64,style=customasmx86]{patterns/13_arrays/55_month_2D/MSVC2013_x64_EN.asm}

\myindex{\CompilerAnomaly}
\label{MSVC2013_anomaly}

But one thing is weird here: why add multiplication by zero and adding zero to the final result?

This looks like a compiler code generator quirk, which wasn't caught by the compiler's tests
(the resulting code works correctly, after all).
% класс!
%
We intentionally consider such pieces of code so the reader would understand, 
that sometimes one shouldn't puzzle over such compiler artifacts.

\subsubsection{32-bit ARM}

\Optimizing Keil 
for Thumb mode uses the multiplication instruction \INS{MULS}:

\lstinputlisting[caption=\OptimizingKeilVI (\ThumbMode),style=customasmARM]{patterns/13_arrays/55_month_2D/Keil_O3_thumb_EN.asm}

\Optimizing Keil for ARM mode uses add and shift operations:

\lstinputlisting[caption=\OptimizingKeilVI (\ARMMode),style=customasmARM]{patterns/13_arrays/55_month_2D/Keil_O3_ARM_EN.asm}

\subsubsection{ARM64}

\lstinputlisting[caption=\Optimizing GCC 4.9 ARM64,style=customasmARM]{patterns/13_arrays/55_month_2D/GCC49_ARM64_EN.asm}

\myindex{ARM!\Instructions!SXTW}
\myindex{ARM!\Instructions!ADRP/ADD pair}

\INS{SXTW} is used for sign-extension and promoting input 32-bit value into a 64-bit one and storing it in X0.

\ADRP/\ADD pair is used for loading the address of the table.

The \ADD instructions also has a \LSL suffix, which helps with multiplications.

\subsubsection{MIPS}
\lstinputlisting[caption=\Optimizing GCC 4.4.5 (IDA),style=customasmMIPS]{patterns/13_arrays/55_month_2D/MIPS_O3_IDA_EN.lst}

\subsubsection{\Conclusion{}}

This is a bit old-school technique to store text strings.
You may find a lot of it in \oracle, for example.
It's hard to say if it's worth doing on modern computers.
Nevertheless, it is a good example of arrays, so it was added to this book.

