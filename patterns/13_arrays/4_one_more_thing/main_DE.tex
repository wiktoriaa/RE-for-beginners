\subsection{Noch ein Wort zu Arrays}
Wir verstehen nun warum es nicht möglich ist etwas wie das Folgende in \CCpp Code zu schreiben:

\begin{lstlisting}[style=customc]
void f(int size)
{
    int a[size];
...
};
\end{lstlisting}
Das liegt daran, dass der Compiler die exakte Größe des Arrays zur Compilerzeit kennen muss, um Platz auf dem lokalen
Stack zu reservieren.

\myindex{\CLanguageElements!C99!variable length arrays}
\myindex{\CStandardLibrary!alloca()}
Wenn man ein Array beliebiger Größe benötigt, muss es über \TT{malloc()} angelegt werden und dann über den reservieren
Speicherblock als Arrays von Variablen des benötigten Typs angesprochen werden.

Oder man verwendet das C99 Standardfeature \InSqBrackets{\CNineNineStd 6.7.5/2}, dass intern wie
\IT{alloca()}~(\myref{alloca}) arbeitet.

Es ist auch möglich, C-Bibliotheken zu verwenden, die als Garbagecollector fungieren.
Des Weiteren gibt es auch Bibliotheken für C++, die intelligente Pointer unterstützen.