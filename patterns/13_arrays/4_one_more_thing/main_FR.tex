\subsection{Encore un mot sur les tableaux}


Maintenant nous comprenons pourquoi il est impossible d'écrire quelque chose comme
ceci en code \CCpp:

\begin{lstlisting}[style=customc]
void f(int size)
{
    int a[size];
...
};
\end{lstlisting}


C'est simplement parce que le compilateur doit connaître la taille exacte du tableau
pour lui allouer de l'espace sur la pile locale lors de l'étape de compilation.

\myindex{\CLanguageElements!C99!variable length arrays}
\myindex{\CStandardLibrary!alloca()}

% TODO: améliorer
Si vous avez besoin d'un tableau de taille arbitraire, il faut l'allouer en utilisant
\TT{malloc()}, puis en accèdant les blocs de mémoire allouée comme un tableau de
variables du type dont vous avez besoin.


Ou utiliser la caractéristique du standart C99 \InSqBrackets{\CNineNineStd 6.7.5/2},
et qui fonctionne comme \IT{alloca()}~(\myref{alloca}) en interne.


Il est aussi possible d'utiliser des bibliothèques de ramasse-miettes pour C.

Et il y a aussi des bibliothèques supportant les pointeurs intelligents pour C++.

