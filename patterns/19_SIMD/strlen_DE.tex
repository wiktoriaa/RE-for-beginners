\subsection{SIMD \strlen Implementierung}
\label{SIMD_strlen}

\newcommand{\URLMSDNSSE}{\href{http://go.yurichev.com/17262}{MSDN: MMX, SSE, und SSE2 Intrinsics}}
Man beachte, dass die \ac{SIMD} Befehle über spezielle Makros\footnote{\URLMSDNSSE} in den \CCpp Code eingefügt werden
können. Einige dieser Makros für MSVC befinden sich in der Datei \TT{intrin.h}.

\newcommand{\URLSTRLEN}{http://go.yurichev.com/17330}

\myindex{\CStandardLibrary!strlen()}
Es ist möglich die Funktion \strlen\footnote{strlen()~---Funktion der Standard-C-Bibliothek, mit der die Länge eines
Strings berechnet wird} mit SIMD Befehlen zu implementieren, sodass sie 2-2.5-mal schneller als die normale
Implementierung arbeitet. Diese Funktion lädt 16 Zeichen in ein XMM-Register und prüft jedes auf Null.\footnote{Dieses
Beispiel basiert auf Quellcode von: \url{\URLSTRLEN}.}.

\lstinputlisting[style=customc]{patterns/19_SIMD/18_3.c}

Kompilieren wir das Beispiel in MSVC 2010 mit der Option \Ox:

\lstinputlisting[caption=\Optimizing MSVC 2010,style=customasmx86]{patterns/19_SIMD/18_4_msvc_Ox_DE.asm}

Um die Funktionsweise zu verstehen müssen wir uns zunächst den Zweck der Funktion klarmachen. Sie berechnet die Länge
eines C-Strings; anders ausgedrückt: sie sucht nach dem Nullbyte und berechnet dann dessen Position relativ zum Anfang
des Strings.

Zunächst prüfen wir, ob der \TT{str} Pointer auf einer 16-Byte-Grenze liegt. Wenn nicht, rufen wir die normale
Implementierung von \strlen auf.

Danach laden wir die nächsten 16 Byte mit \MOVDQA in das Register \XMM{1}.

Der aufmerksame Leser fragt sich vielleicht, warum hier \MOVDQU nicht verwendet wird, da dieser Befehl die Daten auch
dann aus dem Speicher laden kann, wenn der Pointer nicht auf einer 16-Byte-Grenze liegt.

Man könnte es wie folgt lösen: wenn der Pointer entsprechend angeordnet ist, verwende \MOVDQA zum Laden, ansonsten den
langsameren Befehl \MOVDQU.

Aber an dieser Stelle lauert ein weiterer Fallstrick:

\myindex{Seite (Speicher)}
\newcommand{\URLPAGE}{\href{http://go.yurichev.com/17136}{wikipedia}}
In den Betriebssystemen der Reihe \gls{Windows NT} (aber nicht nur dort) wird Speicher in Seiten von 4KiB (4096 Byte)
reserviert. Jeder win32-Prozess hat 4GiB zur Verfügung, auch wenn tatsächlich nur einige Teile des Adressraumes wirklich
mit dem physischen Speicher verbunden sind. Wenn der Prozess auf einen nicht vorhandenen Speicherblock zugreift, wird
eine Exception ausgelöst. So arbeitet \ac{VM}\footnote{\URLPAGE}.

Eine Funktion, die 16 Byte auf einmal lädt, könnte also eine solche Grenze im Speicherblock übertreten.
Nehmen wir an, das Betriebssystem hat 8192 (0x2000) Byte ab der Adresse 0x008c000 reserviert.
Dieser Block umfasst also die Bytes von Adresse 0x08c0000 bis einschließlich 0x008c1fff.

Hinter diesen Block, d.h. ab Adresse 0x008c2000 befindet sich nichts, d.h. das Betriebssystem hat hier keinen weiteren
Speicher reservietrt. Jeder Versuch auf diesen Speicherbereich zuzugreifen, wird eine Exception auslösen.

Betrachten wir weiter das Beispiel, in dem ein Programm einen String enthält, der fast am Ende des Blocks 5 Zeichen
belegt. Das ist zunächst nichts Schlimmes.
%TODO
\begin{center}
  \begin{tabular}{ | l | l | }
    \hline
        0x008c1ff8 & 'h' \\
        0x008c1ff9 & 'e' \\
        0x008c1ffa & 'l' \\
        0x008c1ffb & 'l' \\
        0x008c1ffc & 'o' \\
        0x008c1ffd & '\textbackslash{}x00' \\
        0x008c1ffe & Zufallswert \\
        0x008c1fff & Zufallswert \\
    \hline
  \end{tabular}
\end{center}
Unter normalen Umständen ruft das Programm \strlen auf und übergibt einen Pointer auf den String \TT{'hello'}, der an
der Speicheradresse 0x000c1ff0 liegt.
\strlen liest ein Byte zur Zeit bis zur Adresse 0x000c1ffd, an der sich ein Nullbyte befindet, und hält dann an.

Wenn unser \strlen 16 Byte auf einmal von irgendeiner Startadresse aus liest, richtig angeordnet oder nicht, könnte
\MOVDQU versuchen 16 Byte auf einmal aus dem Adressraum 0x000c1ff0 bis 0x000c2000 zu lesen und eine Exception würde
ausgelöst. Dies gilt es natürlich zu vermeiden.

Wir arbeiten also nur mit Adressen, die auf einer 16-Byte-Grenze liegen, was uns in Kombination mit dem Wissen, dass die
Seitengröße des Betriebssystems normalerweise ebenfalls so angeordnet ist, eine gewisse Sicherheit daür bietet, dass
unsere Funktion nicht aus dem nicht reservierten Speicher zu lesen versucht.

Zurück zu unserer Funktion.

\myindex{x86!\Instructions!PXOR}
\verb|_mm_setzero_si128()|---ist ein Makro, das \TT{pxor xmm0, xmm0} erzeugt~---es löscht das \XMM{0} Register.

\verb|_mm_load_si128()|---ist ein Makro für \MOVDQA: es lädt 16 Bytes von der Adresse in das \XMM{1} Register.

\myindex{x86!\Instructions!PCMPEQB}
\verb|_mm_cmpeq_epi8()|---ist ein Makro für \PCMPEQB, ein Befehl, der zwei XMM-Register byteweise miteinader vergleicht.

Wenn ein Byte das gleiche ist wie im anderen Register, wird im Ergebnis hier \TT{0xff} stehen, ansonsten aber 0.

Ein Beispiel:

\begin{verbatim}
XMM1: 0x11223344556677880000000000000000
XMM0: 0x11ab3444007877881111111111111111
\end{verbatim}

Nach der Ausführung von \TT{pcmpeqb xmm1, xmm0}, enthält das \XMM{1} Register:

\begin{verbatim}
XMM1: 0xff0000ff0000ffff0000000000000000
\end{verbatim}
In unserem Fall vergleicht der Befehl jeden 16-Byte-Block mit einem Block aus 16 Nullbytes, der durch \TT{pxor xmm0,
xmm0} im \XMM{0} Register angelegt wurde.

\myindex{x86!\Instructions!PMOVMSKB}

Das nächste Makro ist \TT{\_mm\_movemask\_epi8()}~---das ist der Befehl \TT{PMOVMSKB}.

Es ist sehr nützlich im Zusammenspiel mit \PCMPEQB.

\TT{pmovmskb eax, xmm1}
Dieser Befehl setzt das erste Bit in \EAX auf 1, falls das MSB des ersten Bytes in \XMM{1} gleich 1 ist.
Mit anderen Worten: wenn das erste Byte im \XMM{1} Register \TT{0xff} ist, dann wird das erste Bit in \EAX ebenfalls auf
1 gesetzt.

Wenn das zweite Byte in \XMM{1} \TT{0xff} ist, wird das zweite Bit in \EAX auf 1 gesetzt.
Mit anderen Worten: der Befehl beantwortet die Frage welche Bytes in \XMM{1} gleich \TT{0xff} sind und gibt 16 Bits im
\EAX Register zurück.
Die anderen Bits im in \EAX werden gelöscht.

Vergessen wir aber nicht die Eigenart in unserem Algorithmus. Es könnte im Input 16 Byte wie die folgenden geben:

\begin{center}
\ifdefined\ebook
\begin{bytefield}[endianness=big,bitwidth=0.062\linewidth]{16}
\else
\begin{bytefield}[endianness=big,bitwidth=0.05\linewidth]{16}
\fi
\bitheader{15,14,13,12,11,10,9,3,2,1,0} \\
\bitbox{1}{\q{h}} & 
\bitbox{1}{\q{e}} & 
\bitbox{1}{\q{l}} & 
\bitbox{1}{\q{l}} & 
\bitbox{1}{\q{o}} & 
\bitbox{1}{0} & 
\bitbox{7}{\garbage{}} & 
\bitbox{1}{0} &
\bitbox{2}{\garbage{}} & 
\end{bytefield}
\end{center}

Das ist der String \TT{'hello'}, eine abschließende Null und ein paar Zufallswerte aus dem Speicher.

Wenn wir diese 16 Byte nach \XMM{1} laden und mit dem genullten \XMM{0} vergleichen, erhalten wir als Ergebnis etwas,
das so ähnlich ist wie der folgende Output\footnote{hier wird eine Ordnung vom \ac{MSB} zum \ac{LSB} verwendet}:

\begin{verbatim}
XMM1: 0x0000ff00000000000000ff0000000000
\end{verbatim}

Das bedeutet, dass der Befehl zwei Nullbytes gefunden hat. Dieses Ergebnis war zu erwarten.
 
\TT{PMOVMSKB} in our case will set \EAX to\\
\IT{0b0010000000100000}.
Offensichtlich muss unsere Funktion nur das erste Nullbit nehmen und den Rest ignorieren.

\myindex{x86!\Instructions!BSF}
\label{instruction_BSF}
Der nächste Befehl ist \TT{BSF} (\IT{Bit Scan Forward}). 
Dieser Befehl erkennt, dass das erste Bit auf 1 gesetzt ist und speichert dessen Position im ersten Operanden.

\begin{verbatim}
EAX=0b0010000000100000
\end{verbatim}
Nach der Ausführung von \TT{bsf eax, eax} enthält \EAX 5, was bedeutet, dass 1 an der 5. Stelle (von 0 aus gezählt)
gefunden wurde.

MSVC verfügt über ein Makro für diesen Befehl: \TT{\_BitScanForward}.

Der Rest ist einfach. Wenn ein Nullbyte gefunden wurde, wird dessen Position zum Zähler hinzuaddiert und wir erhalten
den Rückgabewert.

Das ist fast alles.

Hier ist bemerkenswert, dass der MSVC Compiler zur Optimierung zwei Schleifenrümpfe nebeneinander erzeugt hat.

SSE 4.2 (erschienen im Intel Core i7) verfügt über noch mehr Befehle, sodass diese Stringmanipulationen noch einfacher
bewerkstelligt werden können: \url{http://go.yurichev.com/17331}
