\subsection{Vectorisation}

\newcommand{\URLVEC}{\href{http://go.yurichev.com/17080}{Wikipédia: vectorisation}}

La vectorisation\footnote{\URLVEC}, c'est lorsque, par exemple, vous avez une boucle
qui prend une paire de tableaux en entrée et produit un tableau.
Le corps de la boucle prend les valeurs dans les tableaux en entrée, fait quelque
chose et met le résultat dans le tableau de sortie.
%It is important that there is only a single operation applied to each element.
%Il est important qu'il n'y ait qu'une seule opération appliquée à chaque élément.
La vectorisation est le fait de traiter plusieurs éléments simultanément.

La vectorisation n'est pas une nouvelle technologie: l'auteur de ce livre l'a vu
au moins sur la série du super-calculateur Cray Y-MP de 1988 lorsqu'il jouait avec
sa version \q{lite} le Cray Y-MP EL\footnote{À distance. Il est installé dans le
musée des super-calculateurs: \url{http://go.yurichev.com/17081}}.

% FIXME! add assembly listing!
Par exemple:

\begin{lstlisting}[style=customc]
for (i = 0; i < 1024; i++)
{
    C[i] = A[i]*B[i];
}
\end{lstlisting}

Ce morceau de code prend des éléments de A et de B, les multiplie et sauve le résultat
dans C.

\myindex{x86!\Instructions!PLMULLD}
\myindex{x86!\Instructions!PLMULHW}
\newcommand{\PMULLD}{\IT{PMULLD} (\IT{Multiply Packed Signed Dword Integers and Store Low Result})}
\newcommand{\PMULHW}{\TT{PMULHW} (\IT{Multiply Packed Signed Integers and Store High Result})}

Si chaque élément du tableau que nous avons est un \Tint 32-bit, alors il est possible
de charger 4 éléments de A dans un registre XMM 128-bit, 4 de B dans un autre registre
XMM, et en exécutant \PMULLD{} et \PMULHW{}, il est possible d'obtenir 4 \glslink{product}{produits}
64-bit en une fois.

Ainsi, le nombre d'exécution du corps de la boucle est $1024/4$ au lieu de 1024,
ce qui est 4 fois moins et, bien sûr, est plus rapide.

\newcommand{\URLINTELVEC}{\href{http://go.yurichev.com/17082}{Extrait: Vectorisation automatique efficace}}

\subsubsection{Exemple d'addition}

\myindex{Intel C++}

Certains compilateurs peuvent effectuer la vectorisation automatiquement dans des
cas simples, e.g., Intel C++\footnote{Sur la vectorisation automatique d'Intel C++:
\URLINTELVEC}.

Voici une fonction minuscule:

\begin{lstlisting}[style=customc]
int f (int sz, int *ar1, int *ar2, int *ar3)
{
	for (int i=0; i<sz; i++)
		ar3[i]=ar1[i]+ar2[i];

	return 0;
};
\end{lstlisting}

\myparagraph{Intel C++}

Compilons la avec Intel C++ 11.1.051 win32:

\begin{verbatim}
icl intel.cpp /QaxSSE2 /Faintel.asm /Ox
\end{verbatim}

Nous obtenons (dans \IDA):

\lstinputlisting[style=customasmx86]{patterns/19_SIMD/18_1_FR.asm}

Les instructions relatives à SSE2 sont:
\myindex{x86!\Instructions!MOVDQA}
\myindex{x86!\Instructions!MOVDQU}
\myindex{x86!\Instructions!PADDD}
\begin{itemize}
\item
\MOVDQU (\IT{Move Unaligned Double Quadword} déplacer double quadruple mot non
alignés)---charge juste 16 octets depuis la mémoire dans un registre XMM.

\item
\PADDD (\IT{Add Packed Integers} ajouter entier paqué)---ajoute 4 paires de nombres
32-bit et laisse le résultat dans le premier opérande.
À propos, aucune exception n'est levée en cas de débordement et aucun flag n'est mis,
seuls les 32-bit bas du résultat sont stockés.
Si un des opérandes de \PADDD est l'adresse d'une valeur en mémoire, alors l'adresse
doit être alignée sur une limite de 16 octets.
Si elle n'est pas alignée, une exception est levée\footnote{En savoir plus sur l'alignement
des données: \URLWPDA}.

\item
\MOVDQA (\IT{Move Aligned Double Quadword}) est la même chose que \MOVDQU, mais nécessite
que l'adresse de la valeur en mémoire soit alignée sur une limite de 16 octets. Si
elle n'est pas alignée, une exception est levée.
\MOVDQA fonctionne plus vite que \MOVDQU, mais nécessite la condition qui vient d'être
écrite.

\end{itemize}

Donc, ces instructions SSE2 sont exécutées seulement dans le cas où il y a plus
de 4 paires à traiter et que le pointeur \TT{ar3} est aligné sur une limite de 16
octets.

Ainsi, si \TT{ar2} est également aligné sur une limite de 16 octets, ce morceau de
code sera exécuté:

\begin{lstlisting}[style=customasmx86]
movdqu  xmm0, xmmword ptr [ebx+edi*4] ; ar1+i*4
paddd   xmm0, xmmword ptr [esi+edi*4] ; ar2+i*4
movdqa  xmmword ptr [eax+edi*4], xmm0 ; ar3+i*4
\end{lstlisting}

Autrement, la valeur de \TT{ar2} est chargée dans \XMM{0} avec \MOVDQU, qui ne nécessite
pas que le pointeur soit aligné, mais peut s'exécuter plus lentement.

\lstinputlisting[style=customasmx86]{patterns/19_SIMD/18_1_excerpt_FR.asm}

Dans tous les autres cas, le code non-SSE2 sera exécuté.

\myparagraph{GCC}

\newcommand{\URLGCCVEC}{\url{http://go.yurichev.com/17083}}

GCC peut aussi vectoriser dans des cas simples\footnote{Plus sur le support de la
vectorisation dans GCC: \URLGCCVEC}, si l'option \Othree est utilisée et le support
de SSE2 activé: \TT{-msse2}.

Ce que nous obtenons (GCC 4.4.1):

\lstinputlisting[style=customasmx86]{patterns/19_SIMD/18_2_gcc_O3.asm}

Presque le même, toutefois, pas aussi méticuleux qu'Intel C++.

\subsubsection{Exemple de copie de mémoire}
\label{vec_memcpy}

Revoyons le simple exemple memcpy()
(\myref{loop_memcpy}):

\lstinputlisting[style=customc]{memcpy.c}

Et ce que les optimisations de GCC 4.9.1 font:

\lstinputlisting[caption=GCC 4.9.1 x64 \Optimizing,style=customasmx86]{patterns/19_SIMD/memcpy_GCC49_x64_O3_FR.s}
