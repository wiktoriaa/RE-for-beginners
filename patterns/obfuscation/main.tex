\chapter{\IFRU{Обфускация}{Obfuscation}}

\IFRU{Обфускация это попытка спрятать код (или его значение) от reverse engineer-а}
{Obfuscation is an attempt to hide the code (or its meaning) from reverse engineer}.

\section{\IFRU{Текстовые строки}{Text strings}}

\IFRU{Как я указывал в}{As I revealed in} (\ref{sec:digging_strings}) \IFRU{текстовые строки могут быть крайне
полезны}{text strings may be utterly helpful}.
\IFRU{Знающие об этом программисты могут попытаться их спрятать так, чтобы их не было видно в \IDA{} или любом
шестнадцатеричном редакторе}
{Programmers who aware of this, may try to hide them resulting unableness to find the string in \IDA{} or any hex editor}.

\IFRU{Вот простейший метод}{Here is the simpliest method}.

\IFRU{Вот как строка может быть сконструирована}{That is how the string may be constructed}:

\begin{lstlisting}
mov     byte ptr [ebx], 'h'
mov     byte ptr [ebx+1], 'e'
mov     byte ptr [ebx+2], 'l'
mov     byte ptr [ebx+3], 'l'
mov     byte ptr [ebx+4], 'o'
mov     byte ptr [ebx+5], ' '
mov     byte ptr [ebx+6], 'w'
mov     byte ptr [ebx+7], 'o'
mov     byte ptr [ebx+8], 'r'
mov     byte ptr [ebx+9], 'l'
mov     byte ptr [ebx+10], 'd'
\end{lstlisting}

\IFRU{Строка также может сравниваться с другой}{The string is also can be compared with another like}:

\begin{lstlisting}
mov	ebx, offset username
cmp	byte ptr [ebx], 'j'
jnz	fail
cmp	byte ptr [ebx+1], 'o'
jnz	fail
cmp	byte ptr [ebx+2], 'h'
jnz	fail
cmp	byte ptr [ebx+3], 'n'
jnz	fail
jz	it_is_john
\end{lstlisting}

\IFRU{В обоих случаях, эти строки нельзя так просто нати в шестнадцатеричном редакторе}
{In both cases, it is impossible to find these strings straightforwardly in hex editor}.

\index{shellcode}
\IFRU{Кстати, точно также со строками можно работать в тех случаях, когда строку нельзя разместить в сегменте данных,
например, в \ac{PIC}, или в шелл-коде.}{By the way, it is a chance to work with the strings when it is impossible
to allocate it in data segment, for example, in \ac{PIC} or in shellcode.}

\IFRU{Еще один виденный мною метод с использованием ф-ции}{Another method I once saw is to use} \TT{sprintf()} 
\IFRU{для конструирования}{for constructing}:

\begin{lstlisting}
sprintf(buf, "%s%c%s%c%s", "hel",'l',"o w",'o',"rld");
\end{lstlisting}

\IFRU{Код выглядит ужасно, но как простейшая мера для анти-реверсинга, это может помочь}
{The code looks weird, but as a simpliest anti-reversing measure it may be helpul}.

\IFRU{Текстовые строки могут также присутствовать в зашифрованном виде, в таком случае,
их использование будет предварять вызов ф-ции для дешифровки}
{Text strings may also be present in encrypted form, then all string usage will precede string decrypting routine}.

\section{\IFRU{Исполняемый код}{Executable code}}

\subsection{\IFRU{Вставка мусора}{Inserting garbage}}

\IFRU{Обфускация исполняемого кода это вставка случайного мусора (между настоящим кодом), который исполняется, но не делает
ничего полезного}{Executable code obfuscation mean inserting random garbage code between real one,
which executes but not doing anything useful}.

\IFRU{Просто пример}{Simple example is}:

\begin{lstlisting}
add	eax, ebx
mul	ecx
\end{lstlisting}

\lstinputlisting[caption=obfuscated code]{patterns/obfuscation/1.asm.\LANG}

\IFRU{Здесь код-мусор использует регистры, которые не используются в настоящем коде}
{Here garbage code uses registers which are not used in the real code} (\TT{ESI} \AndENRU \TT{EDX}).
\IFRU{Впрочем, промежуточные результаты полученные при исполнении настоящего кода вполне могут использоваться
кодом-мусором для б\'{о}льшей путанницы}{However, intermediate results produced by the real code 
may be used by garbage instructions for extra mess}\EMDASH{}\IFRU{почему нет}{why not}?

\subsection{\IFRU{Замена инструкций на раздутые эквиваленты}{Replacing instructions to bloated equivalents}}

\begin{itemize}
\item \TT{MOV op1, op2} \IFRU{может быть заменена на пару}{can be replaced by} \TT{PUSH op2 / POP op1}\EN{ pair}.
\item \TT{JMP label} \IFRU{может быть заменена на пару}{can be replaced by} \TT{PUSH label / RET}\EN{ pair}. 
\IDA{} \IFRU{не покажет ссылок на эту метку}{will not show references to the label}.
\item \TT{CALL label} \IFRU{может быть заменена на тройку}{can be replaced by}
\TT{PUSH label\_after\_CALL\_instruction / PUSH label / RET}\EN{ triplet}.
\item \TT{PUSH op} \IFRU{также можно заменить на пару}{may also be replaced by} 
\TT{SUB ESP, 4 (\OrENRU 8) / MOV [ESP], op}\EN{ pair}.
\end{itemize}

\subsection{\IFRU{Всегда исполняющийся/никогда не исполняющийся код}{Always executed/never executed code}}

\IFRU{Если разработчик уверен что в}{If the developer is sure that} ESI \IFRU{всегда будет 0 в этом месте}
{at the point is always 0}:

\lstinputlisting{patterns/obfuscation/2.asm.\LANG}

Reverse engineer\IFRU{-у понадобится какое-то время чтобы с этим разобраться}{ need some time to get into it}.

\index{opaque predicate}
\IFRU{Это также называется}{This is also called} \IT{opaque predicate}.

\IFRU{Еще один пример}{Another example} (\IFRU{и снова разработчик уверен что}
{and again, developer is sure that} ESI\EMDASH{}\IFRU{всегда ноль}{is always zero}):

\lstinputlisting{patterns/obfuscation/3.asm.\LANG}

\subsection{\IFRU{Сделать побольше путанницы}{Making a lot of mess}}

\begin{lstlisting}
instruction 1
instruction 2
instruction 3
\end{lstlisting}

\IFRU{Можно заменить на}{Can be replaced to}:

\begin{lstlisting}
begin:		jmp	ins1_label

ins2_label:	instruction 2
		jmp	ins3_label

ins3_label:	instruction 3
		jmp	exit:

ins1_label:	instruction 1
		jmp	ins2_label
exit:
\end{lstlisting}

\subsection{\IFRU{Использование косвенных указателей}{Using indirect pointers}}

\begin{lstlisting}
dummy_data1	db	100h dup (0)
message1	db	'hello world',0

dummy_data2	db	200h dup (0)
message2	db	'another message',0

func		proc
		...
		mov	eax, offset dummy_data1 ; PE or ELF reloc here
		add	eax, 100h
		push	eax
		call	dump_string
		...
		mov	eax, offset dummy_data2 ; PE or ELF reloc here
		add	eax, 200h
		push	eax
		call	dump_string
		...
func		endp
\end{lstlisting}

\IDA{} \IFRU{покажет ссылки на}{will show references only to} \TT{dummy\_data1} \AndENRU \TT{dummy\_data2}, 
\IFRU{но не на сами текстовые строки}{but not to the text strings}.

\IFRU{К глобальным переменным и даже ф-циям можно обращаться так же}
{Global variables and even functions may be accessed like that}.

\section{\IFRU{Виртуальная машина / псевдо-код}{Virtual machine / pseudo-code}}

\IFRU{Программист может также создать свой собственный}
{Programmer may construct his/her own} \ac{PL} \OrENRU \ac{ISA} \IFRU{и интерпретатор для него}{and interpreter for it}.
(\IFRU{Как версии Visual Basic перед 5.0}{Like pre-5.0 Visual Basic}, .NET, Java machine).
Reverse engineer\IFRU{-у придется потратить какое-то время для понимания деталей всех инструкций в}
{will have to spend some time to understand meaning and details of all} \ac{ISA}\EN{ instructions}.
\IFRU{Ему также возможно придется писать что-то вроде дизассемблера/декомпилятора}
{Probably, he/she will also need to write a disassembler/decompiler of some sort}.

\section{\IFRU{Еще кое-что}{Other thing to mention}}

\IFRU{Моя попытка (хотя и слабая) пропатчить компилятор Tiny C чтобы он выдавал обфусцированный код}
{My own (yet weak) attempt to patch Tiny C compiler to produce obfuscated code}: \url{http://blog.yurichev.com/node/58}.

\IFRU{Использование инструкции}{Using} \MOV \IFRU{для сложных вещей}
{instruction for really complicated things}: \cite{MOV_is_TM}.

