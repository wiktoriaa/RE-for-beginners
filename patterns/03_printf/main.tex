\chapter{\PrintfSeveralArgumentsSectionName}

\RU{Попробуем теперь немного расширить пример \IT{\HelloWorldSectionName}~(\ref{sec:helloworld}),
написав в теле функции \main:}
\EN{Now let's extend the \IT{\HelloWorldSectionName}~(\ref{sec:helloworld}) example, replacing \printf in
the \main function body by this:}

\lstinputlisting{patterns/03_printf/1.c}

\subsection{x86: \IFRU{3 аргумента}{3 arguments}}

\subsubsection{MSVC}

\IFRU{Компилируем при помощи MSVC 2010 Express, и в итоге получим:}
{Let's compile it by MSVC 2010 Express and we got:}

\begin{lstlisting}
$SG3830	DB	'a=%d; b=%d; c=%d', 00H

...

	push	3
	push	2
	push	1
	push	OFFSET $SG3830
	call	_printf
	add	esp, 16					; 00000010H
\end{lstlisting}

\IFRU{Все почти то же, за исключением того, что теперь видно, что аргументы для \printf заталкиваются в стек в обратном порядке: самый первый аргумент заталкивается последним.}
{Almost the same, but now we can see the \printf arguments are pushing into stack in reverse order: and the first argument is pushing in as the last one.}

\IFRU{Кстати, вспомним что переменные типа \Tint в 32-битной системе, как известно, имеет ширину 32 бита, это 4 байта}
{By the way, variables of \Tint type in 32-bit environment has 32-bit width that is 4 bytes}.

\IFRU{Итак, у нас всего 4 аргумента. $4*4 = 16$ ~--- именно 16 байт занимают в стеке указатель на строку плюс еще 3 числа типа \Tint.}
{So, we got here 4 arguments. $4*4 = 16$~---they occupy exactly 16 bytes in the stack: 32-bit pointer to string and 3 number of \Tint type.}

\index{x86!\Instructions!ADD}
\index{x86!\Registers!ESP}
\index{cdecl}
\IFRU{Когда при помощи инструкции \TT{``ADD ESP, X''} корректируется \glslink{stack pointer}{указатель стека} \ESP 
после вызова какой-либо функции, зачастую можно сделать вывод о том, сколько аргументов 
у вызываемой функции было, разделив X на 4.}
{When \gls{stack pointer} (the \ESP register) is corrected by \TT{``ADD ESP, X''}
instruction after a function 
call, often, the number of function arguments could be deduced here: just divide X by 4.}

\IFRU{Конечно, это относится только к cdecl-методу передачи аргументов через стек.}
{Of course, this is related only to \IT{cdecl} calling convention.}

\IFRU{См. также в соответствующем разделе о способах передачи аргументов через стек}
{See also section about calling conventions}~(\ref{sec:callingconventions}).

\IFRU{Иногда бывает так, что подряд идут несколько вызовов разных функций, 
но стек корректируется только один раз, после последнего вызова:}
{It is also possible for compiler to merge several \TT{``ADD ESP, X''} instructions into one, after last call:}

\begin{lstlisting}
push a1
push a2
call ...
...
push a1
call ...
...
push a1
push a2
push a3
call ...
add esp, 24
\end{lstlisting}

\subsubsection{MSVC \AndENRU \olly}
\index{\olly}

\IFRU{Попробуем этот же пример в}{Now let's try to load this example in} \olly.
\IFRU{Это один из наиболее популярных win32-отладчиков user-режима}{It is one of the most 
popular user-land win32 debugger}.
\IFRU{Мы можем компилировать наш пример в}{We can try to compile our example in} MSVC 2012 
\IFRU{с опцией}{with} \TT{/MD} \IFRU{что означает, линковать с библиотекой}{option, meaning, to link 
against} \TT{MSVCR*.DLL},
\IFRU{чтобы импортируемые ф-ции были хорошо видны в отладчике}{so we will able to see imported 
functions clearly in debugger}.

\IFRU{Затем загружаем исполняемый файл в}{Then load executable in} \olly.
\IFRU{Самый первый брякпойнт в}{The very first breakpoint is in} \TT{ntdll.dll}, \IFRU{нажмите}{press} 
F9 (\IFRU{запустить}{run}).
\IFRU{Второй брякпойнт в}{The second breakpoint is in} \ac{CRT}-\IFRU{коде}{code}.
\IFRU{Теперь мы должны найти ф-цию}{Now we should find} \main\EN{ function}.

\IFRU{Найдите этот код скроллируя окно кода до самого верха (MSVC располагает ф-цию \main в самом начале
секции кода)}{Find this code by scrolling the code to the very bottom (MSVC allocates \main function at
the very beginning of the code section)}: 
\figname \ref{fig:printf3_olly_1}.

\IFRU{Кликните на инструкции}{Click on} \TT{PUSH EBP}\IFRU{, нажмите}{ instruction, press} F2 
(\IFRU{установка брякпойнта}{set breakpoint}) \IFRU{и нажмите}{and press} F9 (\IFRU{запустить}{run}).
\IFRU{Нам нужно произвести все эти манипуляции, чтобы пропустить \ac{CRT}-код, потому что нам он пока
не интересен}{We need to do these manupulations in order to skip \ac{CRT}-code, because, we don't really 
interesting in it yet}.

\IFRU{Нажмите}{Press} F8 (\stepover) 6 \IFRU{раз, т.е., пропустить
6 инструкций}{times, i.e., skip 6 instructions}: \figname \ref{fig:printf3_olly_2}.

\IFRU{Теперь}{Now the} \PC \IFRU{указывает на инструкцию}{points to the}
\TT{CALL printf}\EN{ instruction}.
\olly, \IFRU{как и другие отладчики, подсвечивает регистры со значениями, которые изменились}
{like other debuggers, highlights value of registers which were changed}.
\IFRU{Так что, каждый раз, когда мы нажимаем}{So each time you press F8}, \EIP 
\IFRU{изменяется и его значение подсвечивается красным}{is changing and its value looking red}.
\ESP \IFRU{также меняется, потому что значения заталкиваются в стек}{is changing as well, 
because values are pushed into the stack}.

\IFRU{Где находятся эти значения в стеке}{Where are the values in the stack}?
\IFRU{Посмотрите на правое/нижнее окно в отладчике}{Take a look into right/bottom window of debugger}:

\begin{figure}[H]
\centering
\includegraphics[scale=0.66]{patterns/03_printf/olly3_stack.png}
\caption{\olly: \IFRU{стек, после того как значения там сохранены}{stack after values pushed}
(\IFRU{я сделал здесь округлую красную пометку в графическом редакторе}{I made round red mark 
here in graphics editor})}
\end{figure}

\IFRU{Так что здесь видно 3 столбца: адрес в стеке, значение в стеке и еще дополнительный комментарий
от \olly}{So we can see there 3 columns: address in the stack, 
value in the stack and some additional \olly comments}. 
\olly \IFRU{понимает}{understands} \printf\IFRU{-строки}{-like strings}, 
\IFRU{так что он показывает здесь и строку и 3 значения \IT{привязанных} к ней}{so it reports the 
string here and 3 values \IT{attached} to it}.

\IFRU{Нажмите}{Press} F8 (\stepover).

\IFRU{В коносил мы видим вывод}{In the console we'll see the output}:

\begin{figure}[H]
\centering
\includegraphics[scale=0.66]{patterns/03_printf/olly3_console.png}
\caption{\RU{Ф-ция }\printf \IFRU{исполнилась}{function executed}}
\end{figure}

\IFRU{Посмотрим, как изменились регистры и состояние стека}{Let's see how registers and stack state 
are changed}: \figname \ref{fig:printf3_olly_3}.

\RU{Регистр }\EAX \IFRU{теперь содержит}{register now contains} \TT{0xD} (13).
That's correct, \printf returns number of characters printed.
\RU{Значение }\EIP \IFRU{изменилось: действительно, теперь здесь адрес инструкции после}
{value is changed: indeed, now there is address of the instruction after} \TT{CALL printf}.
\RU{Значения регистров }\ECX \AndENRU \EDX \IFRU{также изменились}{values are changed as well}.
\IFRU{Очевидно, внутренности ф-ции \printf используют их для каких-то своих нужд}{Apparently, 
\printf function's hidden machinery used them for its own needs}.

\IFRU{Очень важный момент в том что значение \ESP не изменилось. И состояние стека также!}
{A very important thing is that \ESP value is not changed. And stack state too!}
\IFRU{Мы ясно видим здесь и строку формата и соответствующие ей 3 значения, они все еще здесь}
{We clearly see that format string and corresponding 3 values are still there}.
\IFRU{Действительно, по соглашению вызовов \IT{cdecl}, вызывающая ф-ция не очищает аргументы из стека}
{Indeed, that's \IT{cdecl} calling convention, calling function doesn't clear arguments in stack}.
\IFRU{Это должна делать вызывающая ф-ция}{It's caller's duty to do so}.

\IFRU{Нажмите}{Press} F8 \IFRU{снова, чтобы исполнилась инструкция}{again to execute} 
\TT{ADD ESP, 10}\EN{ instruction}: \figname \ref{fig:printf3_olly_4}.

\ESP \IFRU{изменился, но значения все еще в стеке}{is changed, but values are still in the stack}!
\IFRU{Конечно, никому не нужно заполнять эти значения нулями или что-то в этом роде}{Yes, 
of course, no one needs to fill these values by zero or something like that}.
\IFRU{Потому что всё что выше указателя стека}{Because, everything above stack pointer} (\SP) 
\IFRU{это}{is} \IT{\IFRU{шум}{noise}} \OrENRU \IT{\IFRU{мусор}{garbage}}, \IFRU{это всё не имеет
особой ценности}{it has no value at all}.
\IFRU{Было бы очень затратно по времени очищать ненужные элементы стека, к тому же, никому это и не 
нужно}{It would be time consuming to clear unused stack entries, besides, no one really needs to}.

\begin{figure}[H]
\centering
\includegraphics[scale=0.66]{patterns/03_printf/olly3_1.png}
\caption{\olly: \IFRU{самое начало ф-ции}{the very start of the} \main\EN{ function}}
\label{fig:printf3_olly_1}
\end{figure}

\begin{figure}[H]
\centering
\includegraphics[scale=0.66]{patterns/03_printf/olly3_2.png}
\caption{\olly: \IFRU{перед исполнением}{before} \printf\EN{ execution}}
\label{fig:printf3_olly_2}
\end{figure}

\begin{figure}[H]
\centering
\includegraphics[scale=0.66]{patterns/03_printf/olly3_3.png}
\caption{\olly: \IFRU{после исполнения}{after} \printf\EN{ execution}}
\label{fig:printf3_olly_3}
\end{figure}

\begin{figure}[H]
\centering
\includegraphics[scale=0.66]{patterns/03_printf/olly3_4.png}
\caption{\olly: \IFRU{после исполнения инструкции}{after} \TT{ADD ESP, 10}\EN{ instruction execution}}
\label{fig:printf3_olly_4}
\end{figure}

\subsubsection{GCC}

\IFRU{Скомпилируем то же самое в Linux при помощи GCC 4.4.1 и посмотрим в \IDA что вышло:}
{Now let's compile the same in Linux by GCC 4.4.1 and take a look in \IDA what we got:}

\begin{lstlisting}
main            proc near

var_10          = dword ptr -10h
var_C           = dword ptr -0Ch
var_8           = dword ptr -8
var_4           = dword ptr -4

                push    ebp
                mov     ebp, esp
                and     esp, 0FFFFFFF0h
                sub     esp, 10h
                mov     eax, offset aADBDCD ; "a=%d; b=%d; c=%d"
                mov     [esp+10h+var_4], 3
                mov     [esp+10h+var_8], 2
                mov     [esp+10h+var_C], 1
                mov     [esp+10h+var_10], eax
                call    _printf
                mov     eax, 0
                leave
                retn
main            endp
\end{lstlisting}

\IFRU{Можно сказать, что этот короткий код, созданный GCC, отличается от кода MSVC только способом помещения 
значений в стек.
Здесь GCC снова работает со стеком напрямую без \PUSH/\POP.}
{It can be said, the difference between code by MSVC and GCC is only in method of placing arguments on the stack.
Here GCC working directly with stack without \PUSH/\POP.}

\section{x64: \IFRU{8 аргументов}{8 arguments}}

\index{x86-64}
\label{example_printf8_x64}
\IFRU{Для того, чтобы посмотреть, как остальные аргументы будут передаваться через стек, 
изменим пример еще раз, 
увеличив количество передаваемых аргументов до 9 (строка формата \printf и 8 переменных типа \Tint)}
{To see how other arguments will be passed via the stack, let's change our example again by increasing the number of arguments
to be passed to 9 (\printf format string + 8 \Tint variables)}:

\lstinputlisting{patterns/03_printf/2.c}

\subsection{MSVC}

\IFRU{Как уже было сказано раннее, первые 4 аргумента в Win64 передаются в регистрах}
{As we saw before, the first 4 arguments are passed in the} \RCX, \RDX, \Reg{8}, \Reg{9}
\IFRU{, а остальные --- через стек}{ registers in Win64, while all the rest---via the stack}.
\IFRU{Здесь мы это и видим}{That is what we see here}.
\IFRU{Впрочем, инструкция \PUSH не используется, вместо нее, при помощи \MOV, значения сразу записываются в стек}
{However, the \MOV instruction, instead of \PUSH, is used for preparing the stack, so the values are written
to the stack in a straightforward manner}.

\lstinputlisting[caption=MSVC 2012 x64]{patterns/03_printf/2_MSVC_x64.asm}

\subsection{GCC}

\IFRU{В *NIX-системах для x86-64 ситуация похожая, вот только первые 6 аргументов передаются через}
{In *NIX OS-es, it's the same story for x86-64, except that the first 6 arguments are passed in the} \RDI, \RSI,
\RDX, \RCX, \Reg{8}, \Reg{9}\EN{ registers}.
\IFRU{Остальные --- через стек}{All the rest---via the stack}.
\IFRU{GCC генерирует код записывающий указатель на строку в \EDI вместо \RDI --- 
это мы уже рассмотрели чуть раньше}{GCC generates the code writing string pointer into \EDI instead if \RDI{}---we
saw this thing before}: \ref{hw_EDI_instead_of_RDI}.

\IFRU{Почему перед вызовом \printf очищается регистр \EAX, мы уже рассмотрели раннее}
{We also saw before the \EAX register being cleared before a \printf call}: \ref{SysVABI_input_EAX}.

\lstinputlisting[caption=GCC 4.4.6 -O3 x64]{patterns/03_printf/2_GCC_x64_\LANG.s}

\subsection{GCC + GDB}
\index{GDB}

\IFRU{Попробуем этот пример в}{Let's try this example in} \ac{GDB}.

\begin{lstlisting}
$ gcc -g 2.c -o 2
\end{lstlisting}

\begin{lstlisting}
$ gdb 2
GNU gdb (GDB) 7.6.1-ubuntu
Copyright (C) 2013 Free Software Foundation, Inc.
License GPLv3+: GNU GPL version 3 or later <http://gnu.org/licenses/gpl.html>
This is free software: you are free to change and redistribute it.
There is NO WARRANTY, to the extent permitted by law.  Type "show copying"
and "show warranty" for details.
This GDB was configured as "x86_64-linux-gnu".
For bug reporting instructions, please see:
<http://www.gnu.org/software/gdb/bugs/>...
Reading symbols from /home/dennis/polygon/2...done.
\end{lstlisting}

\begin{lstlisting}[caption=\IFRU{ставим брякпойнт на \printf, запускаем}{let's set breakpoint to \printf, and run}]
(gdb) b printf
Breakpoint 1 at 0x400410
(gdb) run
Starting program: /home/dennis/polygon/2 

Breakpoint 1, __printf (format=0x400628 "a=%d; b=%d; c=%d; d=%d; e=%d; f=%d; g=%d; h=%d\n") at printf.c:29
29	printf.c: No such file or directory.
\end{lstlisting}

\IFRU{В регистрах}{Registers} \RSI/\RDX/\RCX/\Reg{8}/\Reg{9} 
\IFRU{всё предсказуемо}{has the values which are should be there}.
\RU{А }\RIP \IFRU{содержит адрес самой первой инструкции ф-ции}{has an address of the very first instruction
of the} \printf\EN{ function}.

\begin{lstlisting}
(gdb) info registers
rax            0x0	0
rbx            0x0	0
rcx            0x3	3
rdx            0x2	2
rsi            0x1	1
rdi            0x400628	4195880
rbp            0x7fffffffdf60	0x7fffffffdf60
rsp            0x7fffffffdf38	0x7fffffffdf38
r8             0x4	4
r9             0x5	5
r10            0x7fffffffdce0	140737488346336
r11            0x7ffff7a65f60	140737348263776
r12            0x400440	4195392
r13            0x7fffffffe040	140737488347200
r14            0x0	0
r15            0x0	0
rip            0x7ffff7a65f60	0x7ffff7a65f60 <__printf>
...
\end{lstlisting}

\begin{lstlisting}[caption=\IFRU{смотрим на строку формата}{let's inspect format string}]
(gdb) x/s $rdi
0x400628:	"a=%d; b=%d; c=%d; d=%d; e=%d; f=%d; g=%d; h=%d\n"
\end{lstlisting}

\IFRU{Дампим стек на этот раз с командой x/g}{Let's dump stack with x/g command this time}\EMDASH{}g 
\IFRU{означает}{means} \IT{giant words}, \IFRU{т.е., 64-битные слова}{i.e., 64-bit words}.

\begin{lstlisting}
(gdb) x/10g $rsp
0x7fffffffdf38:	0x0000000000400576	0x0000000000000006
0x7fffffffdf48:	0x0000000000000007	0x00007fff00000008
0x7fffffffdf58:	0x0000000000000000	0x0000000000000000
0x7fffffffdf68:	0x00007ffff7a33de5	0x0000000000000000
0x7fffffffdf78:	0x00007fffffffe048	0x0000000100000000
\end{lstlisting}

\IFRU{Самый первый элемент стека, как и в прошлый раз, это}{The very first stack element, 
just like in previous case, is} \ac{RA}.
\IFRU{Через стек также передаются 3 значения}{3 values are also passed in stack}: 6, 7, 8.
\IFRU{Видно, что 8 передается с не очищенной старшей 32-битной частью}{We also see that 8 is passed
with high 32-bits not cleared}: \TT{0x00007fff00000008}.
\IFRU{Это нормально, ведь передаются числа типа \Tint, а они 32-битные}{That's OK, because, values has
\Tint type, which is 32-bit type}.
\IFRU{Так что в старшей части регистра или памяти стека остался ``случайный мусор''}{So, high register
or stack element part may contain ``random garbage''}.

\IFRU{\ac{GDB} показывает всю ф-цию \main, если попытаться посмотреть, куда возвратится управление после
исполнения \printf}{If to take a look, where control flow will return after \printf execution,
\ac{GDB} will show the whole \main function}:

\begin{lstlisting}
(gdb) set disassembly-flavor intel
(gdb) disas 0x0000000000400576
Dump of assembler code for function main:
   0x000000000040052d <+0>:	push   rbp
   0x000000000040052e <+1>:	mov    rbp,rsp
   0x0000000000400531 <+4>:	sub    rsp,0x20
   0x0000000000400535 <+8>:	mov    DWORD PTR [rsp+0x10],0x8
   0x000000000040053d <+16>:	mov    DWORD PTR [rsp+0x8],0x7
   0x0000000000400545 <+24>:	mov    DWORD PTR [rsp],0x6
   0x000000000040054c <+31>:	mov    r9d,0x5
   0x0000000000400552 <+37>:	mov    r8d,0x4
   0x0000000000400558 <+43>:	mov    ecx,0x3
   0x000000000040055d <+48>:	mov    edx,0x2
   0x0000000000400562 <+53>:	mov    esi,0x1
   0x0000000000400567 <+58>:	mov    edi,0x400628
   0x000000000040056c <+63>:	mov    eax,0x0
   0x0000000000400571 <+68>:	call   0x400410 <printf@plt>
   0x0000000000400576 <+73>:	mov    eax,0x0
   0x000000000040057b <+78>:	leave  
   0x000000000040057c <+79>:	ret    
End of assembler dump.
\end{lstlisting}

\IFRU{Заканчиваем исполнение \printf, исполняем инструкцию обнуляющую \EAX, 
удостоверяемся что в регистре \EAX именно ноль}{Let's finish \printf execution, execute the instruction
zeroing \EAX, take a notice that \EAX register has exactly zero}.
\RIP \IFRU{указывает сейчас на инструкцию}{now points to the} \TT{LEAVE}\IFRU{, т.е., предпоследнюю в ф-ции \main}
{ instruction, i.e., penultimate in \main function}.

\begin{lstlisting}
(gdb) finish
Run till exit from #0  __printf (format=0x400628 "a=%d; b=%d; c=%d; d=%d; e=%d; f=%d; g=%d; h=%d\n") at printf.c:29
a=1; b=2; c=3; d=4; e=5; f=6; g=7; h=8
main () at 2.c:6
6		return 0;
Value returned is $1 = 39
(gdb) next
7	};
(gdb) info registers
rax            0x0	0
rbx            0x0	0
rcx            0x26	38
rdx            0x7ffff7dd59f0	140737351866864
rsi            0x7fffffd9	2147483609
rdi            0x0	0
rbp            0x7fffffffdf60	0x7fffffffdf60
rsp            0x7fffffffdf40	0x7fffffffdf40
r8             0x7ffff7dd26a0	140737351853728
r9             0x7ffff7a60134	140737348239668
r10            0x7fffffffd5b0	140737488344496
r11            0x7ffff7a95900	140737348458752
r12            0x400440	4195392
r13            0x7fffffffe040	140737488347200
r14            0x0	0
r15            0x0	0
rip            0x40057b	0x40057b <main+78>
...
\end{lstlisting}


\section{ARM: \IFRU{3 аргумента}{3 arguments}}

\IFRU{В ARM традиционно принята такая схема передачи аргументов в функцию: 
4 первых аргумента через регистры \Reg{0}-\Reg{3}; а остальные ~--- через стек}
{Traditionally, ARM's scheme for passing arguments (calling convention) is as follows:
the first 4 arguments are passed in the \Reg{0}-\Reg{3} registers; the remaining arguments, via the stack}.
\IFRU{Это немного похоже на то, как аргументы передаются в}{This resembles the arguments passing scheme in} 
fastcall~(\ref{fastcall}) \OrENRU win64~(\ref{sec:callingconventions_win64}).

\subsection{\NonOptimizingKeil + \ARMMode}

\begin{lstlisting}[caption=\NonOptimizingKeil + \ARMMode]
.text:00000014             printf_main1
.text:00000014 10 40 2D E9                 STMFD   SP!, {R4,LR}
.text:00000018 03 30 A0 E3                 MOV     R3, #3
.text:0000001C 02 20 A0 E3                 MOV     R2, #2
.text:00000020 01 10 A0 E3                 MOV     R1, #1
.text:00000024 1D 0E 8F E2                 ADR     R0, aADBDCD     ; "a=%d; b=%d; c=%d\n"
.text:00000028 0D 19 00 EB                 BL      __2printf
.text:0000002C 10 80 BD E8                 LDMFD   SP!, {R4,PC}
\end{lstlisting}

\IFRU{Итак, первые 4 аргумента передаются через регистры \Reg{0}-\Reg{3}, по порядку: 
указатель на формат-строку для \printf
в \Reg{0}, затем $1$ в \Reg{1}, $2$ в \Reg{2} и $3$ в \Reg{3}}
{So, the first 4 arguments are passed via the \Reg{0}-\Reg{3} registers in this order:
a pointer to the \printf format string in 
\Reg{0}, then $1$ in \Reg{1}, $2$ in \Reg{2} and $3$ in \Reg{3}}.

\IFRU{Пока что здесь нет ничего необычного}{There is nothing unusual so far}.

\subsection{\OptimizingKeil + \ARMMode}
\label{ARM_B_to_printf}

\begin{lstlisting}[caption=\OptimizingKeil + \ARMMode]
.text:00000014                             EXPORT printf_main1
.text:00000014             printf_main1
.text:00000014 03 30 A0 E3                 MOV     R3, #3
.text:00000018 02 20 A0 E3                 MOV     R2, #2
.text:0000001C 01 10 A0 E3                 MOV     R1, #1
.text:00000020 1E 0E 8F E2                 ADR     R0, aADBDCD     ; "a=%d; b=%d; c=%d\n"
.text:00000024 CB 18 00 EA                 B       __2printf
\end{lstlisting}

\index{ARM!\Registers!Link Register}
\index{ARM!\Instructions!B}
\index{Function epilogue}
\IFRU{Это соптимизированная версия (\Othree) для режима ARM, и здесь мы видим последнюю инструкцию: 
\TT{B} вместо привычной нам \TT{BL}}{This is optimized (\Othree) version for ARM mode and here we see \TT{B} as
the last instruction instead of the familiar \TT{BL}}.
\IFRU{Отличия между этой соптимизированной версией и предыдущей, скомпилированной без оптимизации, 
еще и в том, 
что здесь нет пролога и эпилога функции (инструкций, сохраняющих состояние регистров \TT{\Reg{0}} и \LR)}
{Another difference between this optimized version and the previous one (compiled without optimization)
is also in the
fact that there is no function prologue and epilogue (instructions that save \TT{\Reg{0}} and \LR registers values)}.
\index{x86!\Instructions!JMP}
\IFRU{Инструкция \TT{B} просто переходит на другой адрес, без манипуляций с регистром \LR, то есть
это аналог \JMP в x86}
{The \TT{B} instruction just jumps to another address, without any manipulation of the \LR register,
that is, it is analogous to \JMP in x86}.
\IFRU{Почему это работает нормально? Потому что этот код эквивалентен предыдущему.}
{Why does it work? Because this code is, in fact, effectively equivalent to the previous.}
\IFRU{Основных причин две: 1) стек не модифицируется, как и \glslink{stack pointer}{указатель стека} \SP; 2) вызов функции \printf последний, 
после него ничего не происходит}{There are two main reasons: 1) neither the stack nor \SP, the \gls{stack pointer}, is modified;
2) the call to \printf is the last instruction, so there is nothing going on after it}.
\IFRU{Функция \printf, отработав, просто вернет управление по адресу, записанному в \LR}{After finishing, the \printf
function will just return control to the address stored in \LR}.
\IFRU{Но в \LR находится адрес места, откуда была вызвана наша функция}
{But the address of the point from where our function
was called is now in \LR}!
\IFRU{А следовательно, управление из \printf вернется сразу туда}
{Consequently, control from \printf will be returned to that point}.
\IFRU{Следовательно, нет нужды сохранять \LR, потому что нет нужны модифицировать \LR}
{As a consequence, we do not need to save \LR since we do not need to modify \LR}.
\IFRU{А нет нужды модифицировать \LR, потому что нет иных вызовов функций, кроме \printf, к тому же, после этого вызова не нужно ничего здесь больше делать}
{We do not need to modify \LR since there are no other function calls except \printf. Furthermore,
after this call we do not to do anything}!
\IFRU{Поэтому такая оптимизация возможна}
{That's why this optimization is possible}.

\IFRU{Еще один похожий пример описан в секции}{Another similar example was described in} 
``\SwitchCaseDefaultSectionName'' 
\IFRU{, здесь}{section, here}~(\ref{jump_to_last_printf}).

\subsection{\OptimizingKeil + \ThumbMode}

\begin{lstlisting}[caption=\OptimizingKeil + \ThumbMode]
.text:0000000C             printf_main1
.text:0000000C 10 B5                       PUSH    {R4,LR}
.text:0000000E 03 23                       MOVS    R3, #3
.text:00000010 02 22                       MOVS    R2, #2
.text:00000012 01 21                       MOVS    R1, #1
.text:00000014 A4 A0                       ADR     R0, aADBDCD     ; "a=%d; b=%d; c=%d\n"
.text:00000016 06 F0 EB F8                 BL      __2printf
.text:0000001A 10 BD                       POP     {R4,PC}
\end{lstlisting}

\IFRU{Здесь нет особых отличий от неоптимизированного варианта для режима ARM}
{There is no significant difference from the non-optimized code for ARM mode}.



\subsection{ARM: \IFRU{8 аргументов}{8 arguments}}

\IFRU{Снова воспользуемся примером с 9-ю аргументами из предыдущей секции}{Let's use again the example
with 9 arguments from the previous section}: \ref{example_printf8_x64}.

\begin{lstlisting}
void printf_main2()
{
	printf("a=%d; b=%d; c=%d; d=%d; e=%d; f=%d; g=%d; h=%d\n", 1, 2, 3, 4, 5, 6, 7, 8);
};
\end{lstlisting}

\subsubsection{\OptimizingKeil: \ARMMode}

\begin{lstlisting}
.text:00000028             printf_main2
.text:00000028
.text:00000028             var_18          = -0x18
.text:00000028             var_14          = -0x14
.text:00000028             var_4           = -4
.text:00000028
.text:00000028 04 E0 2D E5                 STR     LR, [SP,#var_4]!
.text:0000002C 14 D0 4D E2                 SUB     SP, SP, #0x14
.text:00000030 08 30 A0 E3                 MOV     R3, #8
.text:00000034 07 20 A0 E3                 MOV     R2, #7
.text:00000038 06 10 A0 E3                 MOV     R1, #6
.text:0000003C 05 00 A0 E3                 MOV     R0, #5
.text:00000040 04 C0 8D E2                 ADD     R12, SP, #0x18+var_14
.text:00000044 0F 00 8C E8                 STMIA   R12, {R0-R3}
.text:00000048 04 00 A0 E3                 MOV     R0, #4
.text:0000004C 00 00 8D E5                 STR     R0, [SP,#0x18+var_18]
.text:00000050 03 30 A0 E3                 MOV     R3, #3
.text:00000054 02 20 A0 E3                 MOV     R2, #2
.text:00000058 01 10 A0 E3                 MOV     R1, #1
.text:0000005C 6E 0F 8F E2                 ADR     R0, aADBDCDDDEDFDGD ; "a=%d; b=%d; c=%d; d=%d; e=%d; f=%d; g=%"...
.text:00000060 BC 18 00 EB                 BL      __2printf
.text:00000064 14 D0 8D E2                 ADD     SP, SP, #0x14
.text:00000068 04 F0 9D E4                 LDR     PC, [SP+4+var_4],#4
\end{lstlisting}

\IFRU{Этот код можно условно разделить на несколько частей}{This code can be divided into several parts}:

\begin{itemize}
\index{Function prologue}
\item \IFRU{Пролог функции}{Function prologue}:

\index{ARM!\Instructions!STR}
\IFRU{Самая первая инструкция}{The very first} \TT{``STR LR, [SP,\#var\_4]!''} 
\IFRU{сохраняет в стеке \LR, ведь нам придется использовать этот регистр для вызова \printf}
{instruction saves the \LR on the stack, because we will use this register for \printf call}.

\index{ARM!\Instructions!SUB}
\IFRU{Вторая инструкция}{The second} \TT{``SUB SP, SP, \#0x14''}
\IFRU{уменьшает \glslink{stack pointer}{указатель стека} \SP, но, на самом деле, эта процедура нужна для выделения в локальном стеке места размером \TT{0x14} ($20$) байт}
{instruction decreasing
\SP \gls{stack pointer}, but in fact, 
this procedure is needed for allocating a space of size \TT{0x14} ($20$) bytes in the stack}.
\IFRU{Действительно, нам нужно передать 5 32-битных значений через стек в \printf, каждое значение занимает 4 байта, а $5*4=20$ ~--- как раз}
{Indeed, we need to pass 5 32-bit values via stack to the \printf function, and each one occupy 4 bytes, that is $5*4=20$~---exactly}.
\IFRU{Остальные 4 32-битных значения будут переданы через регистры}{Other 4 32-bit values will be passed in
registers}.

\item \IFRU{Передача 5, 6, 7 и 8 через стек}{Passing 5, 6, 7 and 8 via stack}:

\IFRU{Затем значения 5, 6, 7 и 8 записываются в регистры \Reg{0}, \Reg{1}, \Reg{2} и \Reg{3} соответственно}
{Then values 5, 6, 7 and 8
are written to the \Reg{0}, \Reg{1}, \Reg{2} and \Reg{3} registers respectively}.
\IFRU{Затем инструкция}{Then} \TT{``ADD R12, SP, \#0x18+var\_14''} 
\IFRU{записывает в регистр \TT{R12} адрес места в стеке, куда будут помещены эти 4 значения}
{instruction writes an address of the point in the stack, where these 4 variables will be written, into the \TT{R12} register}.
\index{IDA!var\_?}
\IT{var\_14} \IFRU{~--- это макрос ассемблера}{is an assembly macro}, \IFRU{равный}{equal to} $-0x14$, 
\IFRU{такие макросы создает \IDA, чтобы удобнее было показывать, как код обращается к стеку}
{such macros are created by \IDA in order to demonstrate simply how code accessing stack}.
\IFRU{Макросы \IT{var\_?}, создаваемые \IDA, отражают локальные переменные в стеке}{\IT{var\_?} macros created
by \IDA reflecting local variables in the stack}.
\IFRU{Так что в \TT{R12} будет записано $SP+4$}{So, $SP+4$ will be written into the \TT{R12} register}.
\index{ARM!\Instructions!STMIA}
\IFRU{Следующая инструкция}{The next} \TT{``STMIA R12, {R0-R3}''} 
\IFRU{записывает содержимое регистров \Reg{0}-\Reg{3} по адресу в памяти, на который указывает \TT{R12}}
{instruction
writes \Reg{0}-\Reg{3} registers contents at the point in memory to which \TT{R12} pointing}.
\IFRU{Инструкция }\TT{STMIA} \IFRU{означает}{instruction meaning} \IT{Store Multiple Increment After}. 
\IT{Increment After} \IFRU{означает, что \TT{R12} будет увеличиваться на 4 после записи каждого значения регистра}
{meaning the \TT{R12} will be increasing by $4$ after each register value write}.

\item \IFRU{Передача $4$ через стек}{Passing $4$ via stack}:
\IFRU{$4$ записывается в \Reg{0}, затем это значение при помощи инструкции}{$4$ is stored in the \Reg{0} and then,
this value, with the help of} \TT{``STR R0, [SP,\#0x18+var\_18]''} \IFRU{попадает в стек}{instruction, is saved
on the stack}.
\IT{var\_18} \IFRU{равен}{is} $-0x18$, \IFRU{смещение будет $0$}{offset will be $0$}, 
\IFRU{так что, значение из регистра \Reg{0} ($4$) запишется туда, куда указывает \SP}
{so, value from the \Reg{0} register ($4$) will be written to a point the \SP pointing to}.

\item \IFRU{Передача 1, 2 и 3 через регистры}{Passing 1, 2 and 3 via registers}:

\IFRU{Значения для первых трех чисел (a, b, c) (1, 2, 3 соответственно) передаются в регистрах R1, R2 и R3 перед самим вызовом \printf}
{Values of first 3 numbers (a, b, c) (1, 2, 3 respectively) are passing in R1, R2 and R3
registers right before \printf call}, \IFRU{а остальные 5 значений передаются через стек, и вот как}{and other
5 values are passed via stack and this is how}:

\item \IFRU{Вызов \printf}{\printf call}:

\index{Function epilogue}
\item \IFRU{Эпилог функции}{Function epilogue}:

\RU{Инструкция }\TT{``ADD SP, SP, \#0x14''} \IFRU{возвращает \SP на прежнее место, 
аннулируя таким образом всё, что было записано в стеке}
{instruction returns the \SP pointer back to former point,
thus annulling what was written to stack}.
\IFRU{Конечно, то что было записано в стек, там пока и останется, но всё это будет многократно 
перезаписано во время исполнения последующих функций}
{Of course, what was written on the stack will stay there, but it all will be
rewritten while execution of following functions}.

\index{ARM!\Instructions!LDR}
\RU{Инструкция }\TT{``LDR PC, [SP+4+var\_4],\#4''} \IFRU{загружает в \PC сохраненное значение \LR из стека,
обеспечивая таким образом выход из функции}
{instruction loads saved \LR value from the stack into the \PC register, providing
exit from the function}.

\end{itemize}

\subsubsection{\OptimizingKeil: \ThumbMode}

\begin{lstlisting}
.text:0000001C             printf_main2
.text:0000001C
.text:0000001C             var_18          = -0x18
.text:0000001C             var_14          = -0x14
.text:0000001C             var_8           = -8
.text:0000001C
.text:0000001C 00 B5                       PUSH    {LR}
.text:0000001E 08 23                       MOVS    R3, #8
.text:00000020 85 B0                       SUB     SP, SP, #0x14
.text:00000022 04 93                       STR     R3, [SP,#0x18+var_8]
.text:00000024 07 22                       MOVS    R2, #7
.text:00000026 06 21                       MOVS    R1, #6
.text:00000028 05 20                       MOVS    R0, #5
.text:0000002A 01 AB                       ADD     R3, SP, #0x18+var_14
.text:0000002C 07 C3                       STMIA   R3!, {R0-R2}
.text:0000002E 04 20                       MOVS    R0, #4
.text:00000030 00 90                       STR     R0, [SP,#0x18+var_18]
.text:00000032 03 23                       MOVS    R3, #3
.text:00000034 02 22                       MOVS    R2, #2
.text:00000036 01 21                       MOVS    R1, #1
.text:00000038 A0 A0                       ADR     R0, aADBDCDDDEDFDGD ; "a=%d; b=%d; c=%d; d=%d; e=%d; f=%d; g=%"...
.text:0000003A 06 F0 D9 F8                 BL      __2printf
.text:0000003E
.text:0000003E             loc_3E                                  ; CODE XREF: example13_f+16
.text:0000003E 05 B0                       ADD     SP, SP, #0x14
.text:00000040 00 BD                       POP     {PC}
\end{lstlisting}

\IFRU{Это почти то же самое, что и в предыдущем примере, только код для thumb и значения помещаются в 
стек немного иначе: в начале $8$ за первый раз, затем $5$, $6$, $7$ за второй раз и $4$ за третий раз}{Almost 
same as
in previous example, however, this is thumb code and values are packed into stack differently: 
$8$ for the first time, then $5$, $6$, $7$ for the second and $4$ for the third}.

\subsubsection{\OptimizingXcode: \ARMMode}

\begin{lstlisting}
__text:0000290C             _printf_main2
__text:0000290C
__text:0000290C             var_1C          = -0x1C
__text:0000290C             var_C           = -0xC
__text:0000290C
__text:0000290C 80 40 2D E9                 STMFD           SP!, {R7,LR}
__text:00002910 0D 70 A0 E1                 MOV             R7, SP
__text:00002914 14 D0 4D E2                 SUB             SP, SP, #0x14
__text:00002918 70 05 01 E3                 MOV             R0, #0x1570
__text:0000291C 07 C0 A0 E3                 MOV             R12, #7
__text:00002920 00 00 40 E3                 MOVT            R0, #0
__text:00002924 04 20 A0 E3                 MOV             R2, #4
__text:00002928 00 00 8F E0                 ADD             R0, PC, R0
__text:0000292C 06 30 A0 E3                 MOV             R3, #6
__text:00002930 05 10 A0 E3                 MOV             R1, #5
__text:00002934 00 20 8D E5                 STR             R2, [SP,#0x1C+var_1C]
__text:00002938 0A 10 8D E9                 STMFA           SP, {R1,R3,R12}
__text:0000293C 08 90 A0 E3                 MOV             R9, #8
__text:00002940 01 10 A0 E3                 MOV             R1, #1
__text:00002944 02 20 A0 E3                 MOV             R2, #2
__text:00002948 03 30 A0 E3                 MOV             R3, #3
__text:0000294C 10 90 8D E5                 STR             R9, [SP,#0x1C+var_C]
__text:00002950 A4 05 00 EB                 BL              _printf
__text:00002954 07 D0 A0 E1                 MOV             SP, R7
__text:00002958 80 80 BD E8                 LDMFD           SP!, {R7,PC}
\end{lstlisting}

\index{ARM!\Instructions!STMFA}
\index{ARM!\Instructions!STMIB}
\IFRU{Почти то же самое, что мы уже видели, за исключением того, что}
{Almost the same what we already figured out, with the
exception of} \TT{STMFA} (Store Multiple Full Ascending) 
\IFRU{~--- это синоним инструкции}{instruction, it is synonym to} 
\TT{STMIB} (Store Multiple Increment Before) \EN{instruction}. 
\IFRU{Эта инструкция увеличивает \SP и только затем записывает в память значение очередного регистра, 
но не наоборот}{This
instruction increasing value in the \SP register and only then writing next register value into memory, but not vice versa}.

\IFRU{Второе, что бросается в глаза, это то, что инструкции как будто бы расположены случайно}{Another thing
we easily spot is the instructions are ostensibly located randomly}.
\IFRU{Например, значение в регистре \Reg{0} подготавливается в трех местах, по адресам \TT{0x2918}, \TT{0x2920} 
и \TT{0x2928}, 
когда это можно было бы сделать в одном месте}{For instance, value in the \Reg{0} register is prepared in three
places, at addresses \TT{0x2918}, \TT{0x2920} and \TT{0x2928}, when it would be possible to do it in one single point}.
\IFRU{Однако, у оптимизирующего компилятора могут быть свои доводы о том, как лучше составлять инструкции 
друг с другом для лучшей эффективности исполнения}
{However, optimizing compiler has its own reasons about how
to place instructions better}.
\IFRU{Процессор обычно пытается исполнять одновременно идущие друг за другом инструкции}
{Usually, processor attempts to simultaneously execute instructions located side-by-side}.
\IFRU{К примеру, инструкции}{For example, instructions like} \TT{``MOVT R0, \#0''} \AndENRU 
\TT{``ADD R0, PC, R0''} \IFRU{не могут быть исполнены одновременно, потому что обе инструкции модифицируют 
регистр \Reg{0}}{cannot be executed simultaneously since they both modifying the \Reg{0} register}. 
\IFRU{А вот инструкции}{On the other hand,} \TT{``MOVT R0, \#0''} \AndENRU \TT{``MOV R2, \#4''} 
\IFRU{легко можно исполнить одновременно, 
потому что эффекты от их исполнения никак не конфликтуют друг с другом}{instructions can be executed
simultaneously since effects of their execution are not conflicting with each other}.
\IFRU{Вероятно, компилятор старается генерировать код именно таким образом там, где это возможно}
{Presumably, compiler tries to generate code in such a way, where it is possible, of course}.
 
\subsubsection{\OptimizingXcode: \ThumbTwoMode}

\begin{lstlisting}
__text:00002BA0                   _printf_main2
__text:00002BA0
__text:00002BA0                   var_1C          = -0x1C
__text:00002BA0                   var_18          = -0x18
__text:00002BA0                   var_C           = -0xC
__text:00002BA0
__text:00002BA0 80 B5                             PUSH            {R7,LR}
__text:00002BA2 6F 46                             MOV             R7, SP
__text:00002BA4 85 B0                             SUB             SP, SP, #0x14
__text:00002BA6 41 F2 D8 20                       MOVW            R0, #0x12D8
__text:00002BAA 4F F0 07 0C                       MOV.W           R12, #7
__text:00002BAE C0 F2 00 00                       MOVT.W          R0, #0
__text:00002BB2 04 22                             MOVS            R2, #4
__text:00002BB4 78 44                             ADD             R0, PC  ; char *
__text:00002BB6 06 23                             MOVS            R3, #6
__text:00002BB8 05 21                             MOVS            R1, #5
__text:00002BBA 0D F1 04 0E                       ADD.W           LR, SP, #0x1C+var_18
__text:00002BBE 00 92                             STR             R2, [SP,#0x1C+var_1C]
__text:00002BC0 4F F0 08 09                       MOV.W           R9, #8
__text:00002BC4 8E E8 0A 10                       STMIA.W         LR, {R1,R3,R12}
__text:00002BC8 01 21                             MOVS            R1, #1
__text:00002BCA 02 22                             MOVS            R2, #2
__text:00002BCC 03 23                             MOVS            R3, #3
__text:00002BCE CD F8 10 90                       STR.W           R9, [SP,#0x1C+var_C]
__text:00002BD2 01 F0 0A EA                       BLX             _printf
__text:00002BD6 05 B0                             ADD             SP, SP, #0x14
__text:00002BD8 80 BD                             POP             {R7,PC}
\end{lstlisting}

\IFRU{Почти то же самое, что и в предыдущем примере,
лишь за тем исключением, что здесь используются thumb-инструкции}
{Almost the same as in previous example,
with the exception the thumb-instructions are used here instead}.



\section{\RU{Кстати}\EN{By the way}}

\index{fastcall}
\RU{Кстати, разница между способом передачи параметров принятая в x86, x64, fastcall и ARM неплохо иллюстрирует тот важный момент, что процессору, в общем, все равно, как будут 
передаваться параметры функций. Можно создать гипотетический компилятор, который будет передавать их при 
помощи указателя на структуру с параметрами, не пользуясь стеком вообще.}
\EN{By the way, this difference between passing arguments in x86, x64, 
fastcall and ARM is a good illustration of the fact that the CPU is not aware of how arguments are passed to functions. 
It is also possible to create a hypothetical compiler that is able to pass arguments 
via a special structure not using stack at all.}

