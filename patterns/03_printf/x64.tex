\subsection{x64: \IFRU{8 аргументов}{8 arguments}}

\index{x86-64}
\label{example_printf8_x64}
\IFRU{Для того, чтобы посмотреть, как остальные аргументы будут передаваться через стек, 
изменим пример еще раз, 
увеличив количество передаваемых аргументов до 9 (строка формата \printf и 8 переменных типа \Tint)}
{To see,
how other arguments will be passed via stack, let's change our example again by increasing number of arguments
to be passed to 9 (\printf format string + 8 \Tint variables)}:

\lstinputlisting{patterns/03_printf/2.c}

\subsubsection{MSVC}

\IFRU{Как уже было сказано раннее, первые 4 аргумента в Win64 передаются в регистрах}
{As we saw before, the first 4 arguments are passed in the} \RCX, \RDX, \Reg{8}, \Reg{9}
\IFRU{, а остальные --- через стек}{ registers in Win64, while all the rest---via stack}.
\IFRU{Здесь мы это и видим}{That is what we see here}.
\IFRU{Впрочем, инструкция \PUSH не используется, вместо нее, при помощи \MOV, значения сразу записываются в стек}
{However, \MOV instruction used here for stack preparing instead of \PUSH, so the values are written
to the stack straightforwardly}.

\lstinputlisting[caption=MSVC 2012 x64]{patterns/03_printf/2_MSVC_x64.asm}

\subsubsection{GCC}

\IFRU{В *NIX-системах для x86-64 ситуация похожая, вот только первые 6 аргументов передаются через}
{The very same story for the *NIX OS-es for x86-64, but first 6 arguments are passed in the} \RDI, \RSI,
\RDX, \RCX, \Reg{8}, \Reg{9}\EN{ registers}.
\IFRU{Остальные --- через стек}{All the rest---via stack}.
\IFRU{GCC генерирует код записывающий указатель на строку в \EDI вместо \RDI --- 
это мы уже рассмотрели чуть раньше}{GCC generates the code writing string pointer into \EDI instead if \RDI{}---we
saw this thing shortly before}: \ref{hw_EDI_instead_of_RDI}.

\IFRU{Почему перед вызовом \printf очищается регистр \EAX, мы уже рассмотрели раннее}
{Why \EAX register is cleared before \printf call, we saw before}: \ref{SysVABI_input_EAX}.

\lstinputlisting[caption=GCC 4.4.6 -O3 x64]{patterns/03_printf/2_GCC_x64_\IFRU{ru}{en}.s}

