\subsubsection{ARM}

\myparagraph{\NonOptimizingKeilVI (\ARMMode)}

\lstinputlisting[label=Keil_number_sign,style=customasmARM]{patterns/09_loops/simple/ARM/Keil_ARM_O0.asm}

Le compteur de boucle $i$ est stocké dans le registre \Reg{4}.
L'instruction \INS{MOV R4, \#2} initialise $i$.
Les instructions \INS{MOV R0, R4} et \INS{BL printing\_function} composent le corps
de la boucle, la première instruction préparant l'argument pour la fonction \ttf
et la seconde l'appellant.
\myindex{ARM!\Instructions!ADD}
L'instruction \INS{ADD R4, R4, \#1} ajoute 1 à la variable $i$ à chaque itération.
\myindex{ARM!\Instructions!CMP}
\myindex{ARM!\Instructions!BLT}
\INS{CMP R4, \#0xA} compare $i$ avec \TT{0xA} (10). 
L'instruction suivante, \INS{BLT} (\IT{Branch Less Than}) saute si $i$ est inférieur
à 10.
Autrement, 0 est écrit dans \Reg{0} (puisque notre fonction renvoie 0) et l'exécution
de la fonction se termine.

\myparagraph{\OptimizingKeilVI (\ThumbMode)}

\lstinputlisting[style=customasmARM]{patterns/09_loops/simple/ARM/Keil_thumb_O3.asm}

Pratiquement la même chose.

\myparagraph{\OptimizingXcodeIV (\ThumbTwoMode)}
\label{ARM_unrolled_loops}

\lstinputlisting[style=customasmARM]{patterns/09_loops/simple/ARM/xcode_thumb_O3.asm}

En fait, ceci était dans ma fonction \ttf:

\begin{lstlisting}[style=customc]
void printing_function(int i)
{
    printf ("%d\n", i);
};
\end{lstlisting}

\myindex{Unrolled loop}
\myindex{Inline code}
Donc, non seulement LLVM \IT{déroule} la boucle, mais aussi \IT{inline} ma très
simple fonction et insère son corps 8 fois au lieu de l'appeler.

Ceci est possible lorsque la fonction est très simple (comme la mienne) et lorsqu'elle
n'est pas trop appelée (comme ici).

\myparagraph{ARM64: GCC 4.9.1 \Optimizing}

\lstinputlisting[caption=\Optimizing GCC 4.9.1,style=customasmARM]{patterns/09_loops/simple/ARM/ARM64_GCC491_O3_FR.s}

\myparagraph{ARM64: GCC 4.9.1 \NonOptimizing}

\lstinputlisting[caption=\NonOptimizing GCC 4.9.1 -fno-inline,style=customasmARM]{patterns/09_loops/simple/ARM/ARM64_GCC491_O3_FR.s}
