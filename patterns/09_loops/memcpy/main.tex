\ifx\RUSSIAN\undefined
\section{Memory copying routine}
\label{loop_memcpy}

Real-world memory copy routines may copy blocks by 4 or 8 bytes at each iteration, use \ac{SIMD}, 
vectorization, etc.
But for the sake of simplicity, this example is simplest possible.

\lstinputlisting{memcpy.c}

\subsection{Straight-forward implementation}

\lstinputlisting[caption=GCC 4.9 x64 optimized for size (-Os)]{patterns/09_loops/memcpy/memcpy_GCC49_x64_Os.s}

\lstinputlisting[caption=GCC 4.9 ARM64 optimized for size (-Os)]{patterns/09_loops/memcpy/memcpy_GCC49_ARM64_Os.s}

\lstinputlisting[caption=\OptimizingKeilVI (\ThumbMode)]{patterns/09_loops/memcpy/memcpy_Keil_Thumb_O3.s}

\subsection{ARM in ARM mode}

Keil in ARM mode take full advantage of conditional suffixes:

\lstinputlisting[caption=\OptimizingKeilVI (\ARMMode)]{patterns/09_loops/memcpy/memcpy_Keil_ARM_O3.s}

That's why there are only one branch instruction instead of 2.

\subsection{Vectorization}

Optimizing GCC can do much more on this example: \ref{vec_memcpy}.

\fi