\subsection{x86}

\subsubsection{MSVC}

\RU{Компилируем}\EN{Let's compile} (MSVC 2010):

\lstinputlisting[caption=MSVC 2010]{patterns/14_bitfields/4_popcnt/shifts_MSVC_\LANG.asm}

\myparagraph{\olly}

\RU{Загрузим этот пример в}\EN{Let's load this example into} \olly. 
\RU{Входное значения для ф-ции пусть будет}\EN{Let's input value be} \TT{0x12345678}.\\
\\
\RU{Для}\EN{For} $i=1$, \RU{мы видим как}\EN{we see how} $i$ \RU{загружается в}\EN{is loaded into} \ECX: 
\figref{fig:shifts_olly1_1}.
\EDX \RU{содержит}\EN{is} $1$. \RU{Сейчас будет исполнена }\TT{SHL}\EN{ is to be executed now}.\\
\\
\TT{SHL} \RU{исполнилась}\EN{was executed}:
\figref{fig:shifts_olly1_2}.
\EDX \RU{содержит}\EN{contain} $1 \ll 1$ (\OrENRU $2$). \RU{Это битовая маска}\EN{This is a bit mask}.\\
\\
\ANDIns \RU{устанавливает}\EN{sets} \ZF \RU{в}\EN{to} $1$, 
\RU{что означает, что входное значение}{which is meaning that input value} (\TT{0x12345678}) 
\RU{умножается\footnote{Логическое ``И''} с}\EN{ ANDed with} $2$ \RU{давая в результате}\EN{resulting} $0$:
\figref{fig:shifts_olly1_3}.
\RU{Так что, нет во входном значении соответстующего бита}\EN{So, no corresponding bit in input value}.
\RU{Участок кода, увеличивающий счетчик бит на единицу не будет исполнен: инструкция \JZ \textit{обойдет} его}
\EN{The piece of code which \glslink{increment}{increments} counter will not be executed: 
\JZ instruction will \textit{bypass} it}.\\
\\
\RU{Я немного потрассировал далее и}\EN{Now I traced some time further and} $i$ \RU{теперь}\EN{is now} $4$.
\TT{SHL} \RU{исполнилась}\EN{is to be executed now}: \figref{fig:shifts_olly4_1}.\\
\\
\EDX=$1 \ll 4$ (\OrENRU \TT{0x10} \OrENRU $16$): \figref{fig:shifts_olly4_2}.
\RU{Это следующая битовая маска}\EN{This is next bit mask}.\\
\\
\ANDIns \RU{исполнилась}\EN{is executed}: \figref{fig:shifts_olly4_3}.
\ZF \RU{сейчас}\EN{is} $0$ \RU{потому что этот бит присутствует во входном значении}
\EN{because there are this bit in input value}.
\RU{Действительно}\EN{Indeed}, \TT{0x12345678 \& 0x10 = 0x10}. 
\RU{Этот бит считается: переход не сработает и счетчик бит будет увеличен на единицу}\EN{This bit counts: 
jump will not trigger and bits counter will be \glslink{increment}{incremented} now}.\\
\\
\RU{Кстати, ф-ция возвращает}\EN{By the way, function returns} $13$. 
\RU{Это количество установленных бит в значении}\EN{This is total bits set in} \TT{0x12345678}\EN{ value}.

\begin{figure}[H]
\centering
\includegraphics[scale=\FigScale]{patterns/14_bitfields/4_popcnt/olly1_1.png}
\caption{\olly: $i=1$, $i$ \RU{загружено в}\EN{is loaded into} \ECX}
\label{fig:shifts_olly1_1}
\end{figure}

\begin{figure}[H]
\centering
\includegraphics[scale=\FigScale]{patterns/14_bitfields/4_popcnt/olly1_2.png}
\caption{\olly: $i=1$, \EDX=$1 \ll 1=2$}
\label{fig:shifts_olly1_2}
\end{figure}

\begin{figure}[H]
\centering
\includegraphics[scale=\FigScale]{patterns/14_bitfields/4_popcnt/olly1_3.png}
\caption{\olly: $i=1$, \RU{есть ли этот бит во входном значении? Нет.}
\EN{are there that bit in the input value? No.} (\ZF=1)}
\label{fig:shifts_olly1_3}
\end{figure}

\begin{figure}[H]
\centering
\includegraphics[scale=\FigScale]{patterns/14_bitfields/4_popcnt/olly4_1.png}
\caption{\olly: $i=4$, $i$ \RU{загружено в}\EN{is loaded into} \ECX}
\label{fig:shifts_olly4_1}
\end{figure}

\begin{figure}[H]
\centering
\includegraphics[scale=\FigScale]{patterns/14_bitfields/4_popcnt/olly4_2.png}
\caption{\olly: $i=4$, \EDX=$1 \ll 4=0x10$}
\label{fig:shifts_olly4_2}
\end{figure}

\begin{figure}[H]
\centering
\includegraphics[scale=\FigScale]{patterns/14_bitfields/4_popcnt/olly4_3.png}
\caption{\olly: $i=4$, \RU{есть ли этот бит во входном значении? Да.}
\EN{are there that bit in the input value? Yes.} (\ZF=0)}
\label{fig:shifts_olly4_3}
\end{figure}

\subsubsection{GCC}

\RU{Скомпилируем то же и в}\EN{Let's compile it in} GCC 4.4.1:

\lstinputlisting[caption=GCC 4.4.1]{patterns/14_bitfields/4_popcnt/shifts_gcc.asm}
