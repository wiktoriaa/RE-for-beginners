\subsection{MIPS}

\subsubsection{\NonOptimizing GCC}

\lstinputlisting[caption=\NonOptimizing GCC 4.4.5 (IDA)]{patterns/14_bitfields/4_popcnt/MIPS_O0_IDA.lst.\LANG}

\index{MIPS!\Instructions!SLL}
\index{MIPS!\Instructions!SLLV}
\RU{Это многословно: все локальные переменные расположены в локальном стеке и перезагружаются каждый раз,
когда нужны.}
\EN{That verbose: all local variables are located in local stack and reloaded each time it needed.}
\RU{Инструкция SLLV это ``Shift Word Left Logical Variable'', она отличается от SLL только тем что
количество бит для сдвига кодируется в SLL (и, следовательно, фиксировано), а SLL берет количество из регистра.}
\EN{SLLV instruction is ``Shift Word Left Logical Variable'', it's different from SLL only in that sense
that shift amount is encoded in SLL instruction (and is fixed, as a consequence), 
but SLLV takes shift amount value from register.}

\subsubsection{\Optimizing GCC}

\RU{Это более сжато}\EN{That is terser}.
\RU{Здесь две инструкции сдвигов вместо одной.}\EN{There are two shifting instructions instead of single.}
\RU{Почему}\EN{Why}?
\RU{Можно заменить первую инструкцию SLLV на инструкцию безусловного перехода, передав управление прямо
на вторую SLLV.}
\EN{It's possible to replace first SLLV instruction with unconditional branch instruction, 
jumping right to the second SLLV.}
\RU{Но это еще одна инструкция перехода в ф-ции, а от них избавляться всегда выгодно}
\EN{But this is another branching instruction in function, and it's always favorable to get rid of them}: 
\ref{branch_predictors}.

\lstinputlisting[caption=\Optimizing GCC 4.4.5 (IDA)]{patterns/14_bitfields/4_popcnt/MIPS_O3_IDA.lst.\LANG}
