\subsection{MIPS}

\subsubsection{\NonOptimizing GCC}

\lstinputlisting[caption=\NonOptimizing GCC 4.4.5 (IDA)]{patterns/14_bitfields/4_popcnt/MIPS_O0_IDA.lst.\LANG}

\index{MIPS!\Instructions!SLL}
\index{MIPS!\Instructions!SLLV}
\RU{Это многословно: все локальные переменные расположены в локальном стеке и перезагружаются каждый раз,
когда нужны.}
\EN{That is verbose: all local variables are located in the local stack and reloaded each time they're needed.}
\RU{Инструкция \SLLV это \q{Shift Word Left Logical Variable}, она отличается от \SLL только тем что
количество бит для сдвига кодируется в \SLL (и, следовательно, фиксировано), а \SLL берет количество из регистра.}
\EN{The \SLLV instruction is \q{Shift Word Left Logical Variable}, it differs from \SLL only in that
the shift amount is encoded in the \SLL instruction (and is fixed, as a consequence), 
but \SLLV takes shift amount from a register.}

\subsubsection{\Optimizing GCC}

\RU{Это более сжато}\EN{That is terser}.
\RU{Здесь две инструкции сдвигов вместо одной.}\EN{There are two shift instructions instead of one.}
\RU{Почему}\EN{Why}?
\RU{Можно заменить первую инструкцию \SLLV на инструкцию безусловного перехода, передав управление прямо
на вторую \SLLV.}
\EN{It's possible to replace the first \SLLV instruction with an unconditional branch instruction that 
jumps right to the second \SLLV.}
\RU{Но это ещё одна инструкция перехода в функции, а от них избавляться всегда выгодно}%
\EN{But this is another branching instruction in the function, and it's always favorable to get rid of them}: 
\myref{branch_predictors}.

\lstinputlisting[caption=\Optimizing GCC 4.4.5 (IDA)]{patterns/14_bitfields/4_popcnt/MIPS_O3_IDA.lst.\LANG}
