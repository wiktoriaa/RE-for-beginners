\clearpage
\myparagraph{\olly}
\index{\olly}

\RU{Загрузим этот пример в}\EN{Let's load this example into} \olly. 
\RU{Входное значение для функции пусть будет}\EN{Let the input value be} \TT{0x12345678}.\\
\\
\RU{Для}\EN{For} $i=1$, \RU{мы видим, как}\EN{we see how} $i$ \RU{загружается в}\EN{is loaded into} \ECX: 

\begin{figure}[H]
\centering
\includegraphics[scale=\FigScale]{patterns/14_bitfields/4_popcnt/olly1_1.png}
\caption{\olly: $i=1$, $i$ \RU{загружено в}\EN{is loaded into} \ECX}
\label{fig:shifts_olly1_1}
\end{figure}

\EDX \RU{содержит}\EN{is} 1. \RU{Сейчас будет исполнена }\SHL\EN{ is to be executed now}.

\clearpage
\SHL \RU{исполнилась}\EN{was executed}:

\begin{figure}[H]
\centering
\includegraphics[scale=\FigScale]{patterns/14_bitfields/4_popcnt/olly1_2.png}
\caption{\olly: $i=1$, \EDX=$1 \ll 1=2$}
\label{fig:shifts_olly1_2}
\end{figure}

\EDX \RU{содержит}\EN{contain} $1 \ll 1$ (\OrENRU 2). \RU{Это битовая маска}\EN{This is a bit mask}.

\clearpage
\AND \RU{устанавливает}\EN{sets} \ZF \RU{в}\EN{to} 1, 
\RU{что означает, что входное значение}\EN{which implies that the input value} (\TT{0x12345678}) 
\RU{умножается\footnote{Логическое \q{И}} с}\EN{ ANDed with} 2 \RU{давая в результате}\EN{results in} 0:

\begin{figure}[H]
\centering
\includegraphics[scale=\FigScale]{patterns/14_bitfields/4_popcnt/olly1_3.png}
\caption{\olly: $i=1$, \RU{есть ли этот бит во входном значении? Нет.}
\EN{is there that bit in the input value? No.} (\ZF=1)}
\label{fig:shifts_olly1_3}
\end{figure}

\RU{Так что во входном значении соответствующего бита нет}\EN{So, there is no corresponding bit in the input value}.
\RU{Участок кода, увеличивающий счетчик бит на единицу, не будет исполнен: инструкция \JZ \emph{обойдет} его}
\EN{The piece of code, which \glslink{increment}{increments} the counter is not to be executed: 
the \JZ instruction \emph{bypassing} it}.

\clearpage
\RU{Немного потрассируем далее и $i$ теперь 4.}%
\EN{Let's trace a bit further and $i$ is now 4.}
\RU{\SHL исполнилась:}%
\EN{\SHL is to be executed now:}

\begin{figure}[H]
\centering
\includegraphics[scale=\FigScale]{patterns/14_bitfields/4_popcnt/olly4_1.png}
\caption{\olly: $i=4$, $i$ \RU{загружено в}\EN{is loaded into} \ECX}
\label{fig:shifts_olly4_1}
\end{figure}

\clearpage
\EDX=$1 \ll 4$ (\OrENRU \TT{0x10} \OrENRU 16): 

\begin{figure}[H]
\centering
\includegraphics[scale=\FigScale]{patterns/14_bitfields/4_popcnt/olly4_2.png}
\caption{\olly: $i=4$, \EDX=$1 \ll 4=0x10$}
\label{fig:shifts_olly4_2}
\end{figure}

\RU{Это ещё одна битовая маска}\EN{This is another bit mask}.

\clearpage
\AND \RU{исполнилась}\EN{is executed}:

\begin{figure}[H]
\centering
\includegraphics[scale=\FigScale]{patterns/14_bitfields/4_popcnt/olly4_3.png}
\caption{\olly: $i=4$, \RU{есть ли этот бит во входном значении? Да.}
\EN{is there that bit in the input value? Yes.} (\ZF=0)}
\label{fig:shifts_olly4_3}
\end{figure}

\ZF \RU{сейчас}\EN{is} 0 \RU{потому что этот бит присутствует во входном значении}
\EN{because this bit is present in the input value}.
\RU{Действительно}\EN{Indeed}, \TT{0x12345678 \& 0x10 = 0x10}. 
\RU{Этот бит считается: переход не сработает и счетчик бит будет увеличен на единицу.}
\EN{This bit counts: the jump is not triggering and the bit counter 
\glslink{increment}{incrementing}.}\\
\\
\RU{Функция возвращает}\EN{The function returns} 13. 
\RU{Это количество установленных бит в значении}\EN{This is total number of bits set in} \TT{0x12345678}.
