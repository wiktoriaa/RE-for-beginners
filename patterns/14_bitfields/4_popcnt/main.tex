\section{\RU{Подсчет выставленных бит}\EN{Counting bits set to 1}}

\RU{Вот этот несложный пример иллюстрирует функцию, считающую количество бит-единиц во входном значении.}
\EN{Here is a simple example of a function that calculates the number of bits set in the input value.}

\RU{Эта операция также называется}\EN{This operation is also called} \q{population count}%
\footnote{\RU{современные x86-процессоры (поддерживающие SSE4) даже имеют инструкцию POPCNT для этого}
\EN{modern x86 CPUs (supporting SSE4) even have a POPCNT instruction for it}}.

\lstinputlisting{patterns/14_bitfields/4_popcnt/shifts.c}

\RU{В этом цикле счетчик итераций $i$ считает от 0 до 31, а $1 \ll i$ будет от 1 до \TT{0x80000000}. 
Описывая это словами, можно сказать 
\IT{сдвинуть единицу на $n$ бит влево}.
Т.е. в некотором смысле, выражение $1 \ll i$ последовательно выдает все возможные позиции бит в 32-битном числе. 
Освободившийся бит справа всегда обнуляется.}
\EN{In this loop, the iteration count value $i$ is counting from 0 to 31, 
so the $1 \ll i$ statement is counting from 1 to \TT{0x80000000}.
Describing this operation in natural language, we would say \IT{shift 1 by n bits left}.
In other words, $1 \ll i$ statement consequently produces all possible bit positions in a 32-bit number.
The freed bit at right is always cleared.}

\RU{Вот таблица всех возможных значений}\EN{Here is a table of all possible} $1 \ll i$ 
\RU{для}\EN{for} $i=0 \ldots 31$:

\begin{center}
\begin{tabular}{ | l | l | l | l | }
\hline
\cellcolor{blue!25} \RU{Выражение в }\CCpp\EN{ expression} & 
\cellcolor{blue!25} \RU{Степень двойки}\EN{Power of two} & 
\cellcolor{blue!25} \RU{Десятичная форма}\EN{Decimal form} & 
\cellcolor{blue!25} \RU{Шестнадцатеричная форма}\EN{Hexadecimal form} \\
\hline
$1 \ll 0$ & 1 & 1 & 1 \\
\hline
$1 \ll 1$ & $2^{1}$ & 2 & 2 \\
\hline
$1 \ll 2$ & $2^{2}$ & 4 & 4 \\
\hline
$1 \ll 3$ & $2^{3}$ & 8 & 8 \\
\hline
$1 \ll 4$ & $2^{4}$ & 16 & 0x10 \\
\hline
$1 \ll 5$ & $2^{5}$ & 32 & 0x20 \\
\hline
$1 \ll 6$ & $2^{6}$ & 64 & 0x40 \\
\hline
$1 \ll 7$ & $2^{7}$ & 128 & 0x80 \\
\hline
$1 \ll 8$ & $2^{8}$ & 256 & 0x100 \\
\hline
$1 \ll 9$ & $2^{9}$ & 512 & 0x200 \\
\hline
$1 \ll 10$ & $2^{10}$ & 1024 & 0x400 \\
\hline
$1 \ll 11$ & $2^{11}$ & 2048 & 0x800 \\
\hline
$1 \ll 12$ & $2^{12}$ & 4096 & 0x1000 \\
\hline
$1 \ll 13$ & $2^{13}$ & 8192 & 0x2000 \\
\hline
$1 \ll 14$ & $2^{14}$ & 16384 & 0x4000 \\
\hline
$1 \ll 15$ & $2^{15}$ & 32768 & 0x8000 \\
\hline
$1 \ll 16$ & $2^{16}$ & 65536 & 0x10000 \\
\hline
$1 \ll 17$ & $2^{17}$ & 131072 & 0x20000 \\
\hline
$1 \ll 18$ & $2^{18}$ & 262144 & 0x40000 \\
\hline
$1 \ll 19$ & $2^{19}$ & 524288 & 0x80000 \\
\hline
$1 \ll 20$ & $2^{20}$ & 1048576 & 0x100000 \\
\hline
$1 \ll 21$ & $2^{21}$ & 2097152 & 0x200000 \\
\hline
$1 \ll 22$ & $2^{22}$ & 4194304 & 0x400000 \\
\hline
$1 \ll 23$ & $2^{23}$ & 8388608 & 0x800000 \\
\hline
$1 \ll 24$ & $2^{24}$ & 16777216 & 0x1000000 \\
\hline
$1 \ll 25$ & $2^{25}$ & 33554432 & 0x2000000 \\
\hline
$1 \ll 26$ & $2^{26}$ & 67108864 & 0x4000000 \\
\hline
$1 \ll 27$ & $2^{27}$ & 134217728 & 0x8000000 \\
\hline
$1 \ll 28$ & $2^{28}$ & 268435456 & 0x10000000 \\
\hline
$1 \ll 29$ & $2^{29}$ & 536870912 & 0x20000000 \\
\hline
$1 \ll 30$ & $2^{30}$ & 1073741824 & 0x40000000 \\
\hline
$1 \ll 31$ & $2^{31}$ & 2147483648 & 0x80000000 \\
\hline
\end{tabular}
\end{center}

\RU{Это числа-константы (битовые маски), которые крайне часто попадаются в практике reverse engineer-а, 
и их нужно уметь распознавать.}
\EN{These constant numbers (bit masks) very often appear in code and a practicing reverse engineer 
must be able to spot them quickly.}
\RU{Числа в десятичном виде заучивать, пожалуй, незачем, а числа в шестнадцатеричном
виде их легко запомнить.}
\EN{You probably haven't to memorize the decimal numbers, but the hexadecimal ones are very easy to remember.}

\RU{Эти константы очень часто используются для определения отдельных бит как флагов.}
\EN{These constants are very often used for mapping flags to specific bits.}
\RU{Например, это из файла}\EN{For example, here is excerpt from} \TT{ssl\_private.h} \RU{из исходников}
\EN{from} Apache 2.4.6\EN{ source code}:

\begin{lstlisting}
/**
 * Define the SSL options
 */
#define SSL_OPT_NONE           (0)
#define SSL_OPT_RELSET         (1<<0)
#define SSL_OPT_STDENVVARS     (1<<1)
#define SSL_OPT_EXPORTCERTDATA (1<<3)
#define SSL_OPT_FAKEBASICAUTH  (1<<4)
#define SSL_OPT_STRICTREQUIRE  (1<<5)
#define SSL_OPT_OPTRENEGOTIATE (1<<6)
#define SSL_OPT_LEGACYDNFORMAT (1<<7)
\end{lstlisting}

\RU{Вернемся назад к нашему примеру}\EN{Let's get back to our example}.

\RU{Макрос \TT{IS\_SET} проверяет наличие этого бита в $a$.}
\EN{The \TT{IS\_SET} macro checks bit presence in $a$.}
\index{x86!\Instructions!AND}
\RU{Макрос \TT{IS\_SET} на самом деле это операция логического И (\IT{AND}) 
и она возвращает 0 если бита там нет, 
либо эту же битовую маску, если бит там есть. 
В \CCpp, конструкция \TT{if()} срабатывает, если выражение внутри её не ноль, пусть хоть 123456, 
поэтому все будет работать.}
\EN{The \TT{IS\_SET} macro is in fact the logical AND operation (\IT{AND}) 
and it returns 0 if the specific bit is absent there,
or the bit mask, if the bit is present.
\IT{The if()} operator in \CCpp triggers if the expression in it is not zero, it might be even 123456, that is why
it always works correctly.}

% subsections
\subsection{x86}

\subsubsection{MSVC}

\RU{Компилируем}\EN{Let's compile} (MSVC 2010):

\lstinputlisting[caption=MSVC 2010]{patterns/14_bitfields/4_popcnt/shifts_MSVC_\LANG.asm}

\myparagraph{\olly}

\RU{Загрузим этот пример в}\EN{Let's load this example into} \olly. 
\RU{Входное значения для ф-ции пусть будет}\EN{Let's input value be} \TT{0x12345678}.\\
\\
\RU{Для}\EN{For} $i=1$, \RU{мы видим как}\EN{we see how} $i$ \RU{загружается в}\EN{is loaded into} \ECX: 
\figref{fig:shifts_olly1_1}.
\EDX \RU{содержит}\EN{is} $1$. \RU{Сейчас будет исполнена }\TT{SHL}\EN{ is to be executed now}.\\
\\
\TT{SHL} \RU{исполнилась}\EN{was executed}:
\figref{fig:shifts_olly1_2}.
\EDX \RU{содержит}\EN{contain} $1 \ll 1$ (\OrENRU $2$). \RU{Это битовая маска}\EN{This is a bit mask}.\\
\\
\ANDIns \RU{устанавливает}\EN{sets} \ZF \RU{в}\EN{to} $1$, 
\RU{что означает, что входное значение}{which is meaning that input value} (\TT{0x12345678}) 
\RU{умножается\footnote{Логическое ``И''} с}\EN{ ANDed with} $2$ \RU{давая в результате}\EN{resulting} $0$:
\figref{fig:shifts_olly1_3}.
\RU{Так что, нет во входном значении соответстующего бита}\EN{So, no corresponding bit in input value}.
\RU{Участок кода, увеличивающий счетчик бит на единицу не будет исполнен: инструкция \JZ \textit{обойдет} его}
\EN{The piece of code which \glslink{increment}{increments} counter will not be executed: 
\JZ instruction will \textit{bypass} it}.\\
\\
\RU{Я немного потрассировал далее и}\EN{Now I traced some time further and} $i$ \RU{теперь}\EN{is now} $4$.
\TT{SHL} \RU{исполнилась}\EN{is to be executed now}: \figref{fig:shifts_olly4_1}.\\
\\
\EDX=$1 \ll 4$ (\OrENRU \TT{0x10} \OrENRU $16$): \figref{fig:shifts_olly4_2}.
\RU{Это следующая битовая маска}\EN{This is next bit mask}.\\
\\
\ANDIns \RU{исполнилась}\EN{is executed}: \figref{fig:shifts_olly4_3}.
\ZF \RU{сейчас}\EN{is} $0$ \RU{потому что этот бит присутствует во входном значении}
\EN{because there are this bit in input value}.
\RU{Действительно}\EN{Indeed}, \TT{0x12345678 \& 0x10 = 0x10}. 
\RU{Этот бит считается: переход не сработает и счетчик бит будет увеличен на единицу}\EN{This bit counts: 
jump will not trigger and bits counter will be \glslink{increment}{incremented} now}.\\
\\
\RU{Кстати, ф-ция возвращает}\EN{By the way, function returns} $13$. 
\RU{Это количество установленных бит в значении}\EN{This is total bits set in} \TT{0x12345678}\EN{ value}.

\begin{figure}[H]
\centering
\includegraphics[scale=\FigScale]{patterns/14_bitfields/4_popcnt/olly1_1.png}
\caption{\olly: $i=1$, $i$ \RU{загружено в}\EN{is loaded into} \ECX}
\label{fig:shifts_olly1_1}
\end{figure}

\begin{figure}[H]
\centering
\includegraphics[scale=\FigScale]{patterns/14_bitfields/4_popcnt/olly1_2.png}
\caption{\olly: $i=1$, \EDX=$1 \ll 1=2$}
\label{fig:shifts_olly1_2}
\end{figure}

\begin{figure}[H]
\centering
\includegraphics[scale=\FigScale]{patterns/14_bitfields/4_popcnt/olly1_3.png}
\caption{\olly: $i=1$, \RU{есть ли этот бит во входном значении? Нет.}
\EN{are there that bit in the input value? No.} (\ZF=1)}
\label{fig:shifts_olly1_3}
\end{figure}

\begin{figure}[H]
\centering
\includegraphics[scale=\FigScale]{patterns/14_bitfields/4_popcnt/olly4_1.png}
\caption{\olly: $i=4$, $i$ \RU{загружено в}\EN{is loaded into} \ECX}
\label{fig:shifts_olly4_1}
\end{figure}

\begin{figure}[H]
\centering
\includegraphics[scale=\FigScale]{patterns/14_bitfields/4_popcnt/olly4_2.png}
\caption{\olly: $i=4$, \EDX=$1 \ll 4=0x10$}
\label{fig:shifts_olly4_2}
\end{figure}

\begin{figure}[H]
\centering
\includegraphics[scale=\FigScale]{patterns/14_bitfields/4_popcnt/olly4_3.png}
\caption{\olly: $i=4$, \RU{есть ли этот бит во входном значении? Да.}
\EN{are there that bit in the input value? Yes.} (\ZF=0)}
\label{fig:shifts_olly4_3}
\end{figure}

\subsubsection{GCC}

\RU{Скомпилируем то же и в}\EN{Let's compile it in} GCC 4.4.1:

\lstinputlisting[caption=GCC 4.4.1]{patterns/14_bitfields/4_popcnt/shifts_gcc.asm}

\subsection{x64}
\label{subsec:popcnt}

\RU{Немного изменим пример, расширив его до 64-х бит}\EN{Let's modify the example slightly to extend it to 64-bit}:

\lstinputlisting[label=popcnt_x64_example]{patterns/14_bitfields/4_popcnt/shifts64.c}

\ifdefined\IncludeGCC
\subsubsection{\NonOptimizing GCC 4.8.2}

\RU{Пока всё просто}\EN{So far so easy}.

\lstinputlisting[caption=\NonOptimizing GCC 4.8.2]{patterns/14_bitfields/4_popcnt/shifts64_GCC_O0.s.\LANG}

\subsubsection{\Optimizing GCC 4.8.2}

\lstinputlisting[caption=\Optimizing GCC 4.8.2,numbers=left,label=shifts64_GCC_O3]{patterns/14_bitfields/4_popcnt/shifts64_GCC_O3.s.\LANG}

\RU{Код более лаконичный, но содержит одну необычную вещь}\EN{This code is terser, but has a quirk}.
\RU{Во всех примерах, что мы пока видели, инкремент значения переменной \q{rt} происходит после сравнения 
определенного бита с единицей, но здесь \q{rt} увеличивается на 1 до этого (строка 6), записывая новое значение
в регистр \EDX.}
\EN{In all examples that we see so far, we were incrementing the \q{rt} value after comparing a specific bit,
but the code here increments \q{rt} before (line 6), writing the new value into register \EDX .}
\RU{Затем, если последний бит был 1, инструкция}\EN{Thus, if the last bit is 1, the} \CMOVNE%
\footnote{Conditional MOVe if Not Equal\RU{ (\MOV если не равно)}}\EN{ instruction}
(\RU{которая синонимична}\EN{which is a synonym for} \CMOVNZ%
\footnote{Conditional MOVe if Not Zero\RU{ (\MOV если не ноль)}}) \IT{\RU{фиксирует}\EN{commits}} 
\RU{новое значение}\EN{the new value of} \q{rt}
\RU{копируя значение из}\EN{by moving} \EDX (\q{\RU{предполагаемое значение rt}\EN{proposed rt value}}) 
\RU{в}\EN{into} \EAX 
(\q{\RU{текущее}\EN{current} rt} \RU{которое будет возвращено в конце функции}\EN{to be returned at the end}).
\RU{Следовательно, инкремент происходит на каждом шаге цикла, т.е. 64 раза, вне всякой связи с входным
значением.}
\EN{Hence, the incrementing is done at each step of loop, i.e., 64 times, without any relation to the input value.}

\RU{Преимущество этого кода в том, что он содержит только один условный переход (в конце цикла) вместо
двух (пропускающий инкремент \q{rt} и ещё одного в конце цикла).}
\EN{The advantage of this code is that it contain only one conditional jump (at the end of the loop) instead of 
two jumps (skipping the \q{rt} value increment and at the end of loop).}
\RU{И это может работать быстрее на современных CPU с предсказателем переходов}%
\EN{And that might work faster on the modern CPUs with branch predictors}: \myref{branch_predictors}.

\label{FATRET}
\index{x86!\Instructions!FATRET}
\RU{Последняя инструкция это}\EN{The last instruction is} \INS{REP RET} (\EN{opcode}\RU{опкод} \TT{F3 C3}) 
\RU{которая также называется}\EN{which is also called} \INS{FATRET} \RU{в}\EN{by} MSVC.
\RU{Это оптимизированная версия \RET, рекомендуемая AMD для вставки в конце функции, если \RET идет
сразу после условного перехода}\EN{This is somewhat optimized version of \RET, 
which is recommended by AMD to be placed at the end of function, if \RET goes right after conditional jump}: 
\cite[p.15]{AMDOptimization}
\footnote{\RU{Больше об этом}\EN{More information on it}: \url{http://go.yurichev.com/17328}}.
\fi

\subsubsection{\Optimizing MSVC 2010}

\lstinputlisting[caption=MSVC 2010]{patterns/14_bitfields/4_popcnt/MSVC_2010_x64_Ox.asm.\LANG}

\index{x86!\Instructions!ROL}
\RU{Здесь используется инструкция}\EN{Here the} \ROL \RU{вместо}\EN{instruction is used instead of} 
\SHL, \RU{которая на самом деле}\EN{which is in fact} \q{rotate left}\RU{ (прокручивать влево)} 
\RU{вместо}\EN{instead of} \q{shift left}\RU{ (сдвиг влево)},
\RU{но здесь, в этом примере, она работает так же как и}
\EN{but in this example it works just as} \TT{SHL}.

\ifx\LITE\undefined
\RU{Об этих \q{прокручивающих} инструкциях больше читайте здесь}\EN{You can read more about the rotate instruction 
here}: \myref{ROL_ROR}.
\fi

\Reg{8} \RU{здесь считает от 64 до 0}\EN{here is counting from 64 to 0}. 
\RU{Это как бы инвертированная переменная $i$}\EN{It's just like an inverted $i$}.

\RU{Вот таблица некоторых регистров в процессе исполнения}\EN{Here is a table of some registers during the execution}:

\begin{center}
\begin{tabular}{ | l | l | }
\hline
\cellcolor{blue!25} RDX & \cellcolor{blue!25} R8 \\
\hline
0x0000000000000001 & 64 \\
\hline
0x0000000000000002 & 63 \\
\hline
0x0000000000000004 & 62 \\
\hline
0x0000000000000008 & 61 \\
\hline
... & ... \\
\hline
0x4000000000000000 & 2 \\
\hline
0x8000000000000000 & 1 \\
\hline
\end{tabular}
\end{center}

\ifx\LITE\undefined
\index{x86!\Instructions!FATRET}
\RU{В конце видим инструкцию}\EN{At the end we see the} \INS{FATRET}\RU{, которая была описана здесь}\EN{ instruction, 
which was explained here}: \myref{FATRET}.
\fi

\subsubsection{\Optimizing MSVC 2012}

\lstinputlisting[caption=MSVC 2012]{patterns/14_bitfields/4_popcnt/MSVC_2012_x64_Ox.asm.\LANG}

\index{\CompilerAnomaly}
\label{MSVC2012_anomaly}
\Optimizing MSVC 2012 \RU{делает почти то же самое что и оптимизирующий}\EN{does almost the same job as 
optimizing} MSVC 2010, \RU{но почему-то он генерирует 2 идентичных тела цикла и счетчик цикла теперь 32
вместо 64}\EN{but somehow, it generates two identical loop bodies and the loop count is now 32 instead of 64}.
\RU{Честно говоря, нельзя сказать, почему. Какой-то трюк с оптимизацией? Может быть, телу цикла лучше быть
немного длиннее?}
\EN{To be honest, it's not possible to say why. Some optimization trick? Maybe it's better for the loop body to be slightly 
longer?}
\RU{Так или иначе, такой код здесь уместен, чтобы показать, что результат компилятора
иногда может быть очень странный и нелогичный, но прекрасно работающий, конечно же.}
\EN{Anyway, such code is relevant here to show that sometimes the compiler output may be really weird and 
illogical, but perfectly working.}

\ifdefined\IncludeARM
\subsection{ARM + \OptimizingXcodeIV + \ARMMode}

\begin{lstlisting}[caption=\OptimizingXcodeIV + \ARMMode,label=ARM_leaf_example4]
                MOV             R1, R0
                MOV             R0, #0
                MOV             R2, #1
                MOV             R3, R0
loc_2E54
                TST             R1, R2,LSL R3 ; set flags according to R1 & (R2<<R3)
                ADD             R3, R3, #1    ; R3++
                ADDNE           R0, R0, #1    ; if ZF flag is cleared by TST, R0++
                CMP             R3, #32
                BNE             loc_2E54
                BX              LR
\end{lstlisting}

\index{ARM!\Instructions!TST}
\TT{TST} \RU{это то же что и}\EN{is the same things as} \TEST \InENRU x86.

\index{ARM!Optional operators!LSL}
\index{ARM!Optional operators!LSR}
\index{ARM!Optional operators!ASR}
\index{ARM!Optional operators!ROR}
\index{ARM!Optional operators!RRX}
\index{ARM!\Instructions!MOV}
\index{ARM!\Instructions!TST}
\index{ARM!\Instructions!CMP}
\index{ARM!\Instructions!ADD}
\index{ARM!\Instructions!SUB}
\index{ARM!\Instructions!RSB}
\RU{Как я уже указывал}\EN{As I mentioned before}~(\ref{shifts_in_ARM_mode}),
\RU{в режиме ARM нет отдельной инструкции для сдвигов.}
\EN{there are no separate shifting instructions in ARM mode.}
\RU{Однако, модификаторами}\EN{However, there are modifiers} 
LSL (\IT{Logical Shift Left}), 
LSR (\IT{Logical Shift Right}), 
ASR (\IT{Arithmetic Shift Right}), 
ROR (\IT{Rotate Right}) \AndENRU 
RRX (\IT{Rotate Right with Extend}) \RU{можно дополнять некоторые инструкции, такие как}
\EN{, which may be added to such instructions as} \MOV, \TT{TST},
\CMP, \ADD, \SUB, \TT{RSB}\footnote{\DataProcessingInstructionsFootNote}.

\RU{Эти модификаторы указывают, как сдвигать второй операнд, и на сколько.}
\EN{These modificators are defines, how to shift second operand and by how many bits.}

\index{ARM!Optional operators!LSL}
\RU{Таким образом, инструкция }\EN{Thus} \TT{``TST R1, R2,LSL R3''} 
\RU{здесь работает как}\EN{instruction works here as} $R1 \land (R2 \ll R3)$.

\subsection{ARM + \OptimizingXcodeIV + \ThumbTwoMode}

\index{ARM!\Instructions!LSL.W}
\index{ARM!\Instructions!LSL}
\RU{Почти такое же}\EN{Almost the same}, 
\RU{только здесь применяется пара инструкций}\EN{but here are two} 
\TT{LSL.W}/\TT{TST} 
\RU{вместо одной}\EN{instructions are used instead of single} 
\TT{TST},
\RU{ведь в режиме thumb нельзя добавлять указывать модификатор}\EN{because, in thumb mode, it is not
possible to define} \TT{LSL} \RU{прямо в}\EN{modifier right in} \TT{TST}.

\begin{lstlisting}[label=ARM_leaf_example5]
                MOV             R1, R0
                MOVS            R0, #0
                MOV.W           R9, #1
                MOVS            R3, #0
loc_2F7A
                LSL.W           R2, R9, R3
                TST             R2, R1
                ADD.W           R3, R3, #1
                IT NE
                ADDNE           R0, #1
                CMP             R3, #32
                BNE             loc_2F7A
                BX              LR
\end{lstlisting}

\subsection{ARM64 + \Optimizing GCC 4.9}

\RU{Я взял 64-битный пример, который уже использовал}\EN{I took 64-bit example I already used}: 
\ref{popcnt_x64_example}.

\lstinputlisting[caption=\Optimizing GCC (Linaro) 4.8]{patterns/14_bitfields/4_popcnt/ARM64_GCC_O3.s}

\RU{Результат очень похож на тот, что GCC сгенерировал для x64}\EN{The result is very similar to what GCC 
generated for x64}: \ref{shifts64_GCC_O3}.

\RU{Инструкция }\TT{CSEL} \RU{это}\EN{instruction is} ``Conditional SELect''\RU{ (выбор при условии)}, 
\RU{она просто выбирает одну из переменных, в зависимости от флагов выставленных}\EN{it just choose one 
variable of two depending on flags set by} \TT{TST} \RU{и копирует значение в регистр}\EN{and copy value 
into} \RegW{2}\RU{, содержаий переменную ``rt''}\EN{ register, which holds ``rt'' variable}.

\subsection{ARM64 + \NonOptimizing GCC 4.9}

\RU{И снова, я использовал 64-битный пример, который я уже использовал раннее}\EN{Again, I use 64-bit 
example I already used}: \ref{popcnt_x64_example}.

\RU{Код более многословный, как обычно}\EN{The code is more verbose, as usual}.

\lstinputlisting[caption=\NonOptimizing GCC (Linaro) 4.8]{patterns/14_bitfields/4_popcnt/ARM64_GCC_O0.s}

\fi
\ifdefined\IncludeMIPS
\ifx\RUSSIAN\undefined
\subsection{MIPS}

\subsubsection{\NonOptimizing GCC}

\lstinputlisting[caption=\NonOptimizing GCC 4.4.5 (IDA)]{patterns/14_bitfields/4_popcnt/MIPS_O0_IDA.lst}

\index{MIPS!\Instructions!SLL}
\index{MIPS!\Instructions!SLLV}
That verbose: all local variables are located in local stack and reloaded each time.
SLLV instruction is ``Shift Word Left Logical Variable'', it's different from SLL only in that sense
that shift amount is encoded in SLL instruction, but SLLV takes shift amount value from register.

\subsubsection{\Optimizing GCC}

That is more terse.
There are two shifting instructions instead of single.
Why?
It's possible to replace first SLLV instruction with unconditional branch instruction, 
jumping right to the second SLLV.
But this is another branching instruction in function, and it's always favorable to get rid of them: 
\ref{branch_predictors}.

\lstinputlisting[caption=\Optimizing GCC 4.4.5 (IDA)]{patterns/14_bitfields/4_popcnt/MIPS_O3_IDA.lst}

\fi

\fi
