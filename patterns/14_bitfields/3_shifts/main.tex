\section{\ShiftsSectionName}

\RU{Битовые сдвиги в \CCpp реализованы при помощи операторов $\ll$ и $\gg$.}
\EN{Bit shifts in \CCpp are implemented via $\ll$ and $\gg$ operators.}

\RU{В x86 есть инструкции}\EN{x86 \ac{ISA} has} SHL (SHift Left) \AndENRU SHR (SHift Right) 
\RU{для этого}\EN{instructions for this}.

\subsection{\RU{Деление и умножение при помощи сдвигов}\EN{Division and multiplication using shifts}}

\RU{Инструкции сдвига также активно применяются при делении или умножении 
на числа-степени двойки: $2^{n}$ (т.е., $1$, $2$, $4$, $8$, и т.д.).}
\EN{Shift instructions are often used in division and multiplications by power of two numbers:
$2^{n}$ (e.g., $1$, $2$, $4$, $8$, etc).}

\subsubsection{\RU{Умножение}\EN{Multiplication}}

\begin{lstlisting}
unsigned int f(unsigned int a)
{
	return a*4;
};
\end{lstlisting}

\begin{lstlisting}[caption=\NonOptimizing MSVC 2010]
_a$ = 8		; size = 4
_f	PROC
	push	ebp
	mov	ebp, esp
	mov	eax, DWORD PTR _a$[ebp]
	shl	eax, 2
	pop	ebp
	ret	0
_f	ENDP
\end{lstlisting}

\RU{Умножить на $4$ это просто сдвинуть число на 2 бита влево, 
вставив 2 нулевых бита справа (как два самых младших бита). 
Это как умножить $3$ на $100$ ~--- нужно просто дописать два нуля справа.}
\EN{Multiplication by $4$ is just shifting the number to the left by 2 bits,
while inserting 2 zero bits at right (as the last two bits).
It is just like to multiply $3$ by $100$~---we need just to add two zeroes at the right.}

\RU{Вот как работает инструкция сдвига влево}\EN{That's how shift left instruction works}:

\index{x86!\Instructions!SHL}
\begin{center}
	\begin{tikzpicture}[scale=0.7, every node/.style={scale=0.7}]
	\edef\bitsize{1cm}
	\tikzstyle{byte}=[draw,minimum size=\bitsize]	
	\tikzstyle{every path}=[thick]

	\node [draw,rectangle,minimum size=\bitsize] (a1) {7};
	\node [draw,rectangle,minimum size=\bitsize] (a2) [right of=a1] {6};
	\node [draw,rectangle,minimum size=\bitsize] (a3) [right of=a2] {5};
	\node [draw,rectangle,minimum size=\bitsize] (a4) [right of=a3] {4};
	\node [draw,rectangle,minimum size=\bitsize] (a5) [right of=a4] {3};
	\node [draw,rectangle,minimum size=\bitsize] (a6) [right of=a5] {2};
	\node [draw,rectangle,minimum size=\bitsize] (a7) [right of=a6] {1};
	\node [draw,rectangle,minimum size=\bitsize] (a8) [right of=a7] {0};

	\node (empty) [below of=a1] {};

	\node [draw,rectangle,minimum size=\bitsize] (b1) [below of=empty] {7};
	\node [draw,rectangle,minimum size=\bitsize] (b2) [right of=b1] {6};
	\node [draw,rectangle,minimum size=\bitsize] (b3) [right of=b2] {5};
	\node [draw,rectangle,minimum size=\bitsize] (b4) [right of=b3] {4};
	\node [draw,rectangle,minimum size=\bitsize] (b5) [right of=b4] {3};
	\node [draw,rectangle,minimum size=\bitsize] (b6) [right of=b5] {2};
	\node [draw,rectangle,minimum size=\bitsize] (b7) [right of=b6] {1};
	\node [draw,rectangle,minimum size=\bitsize] (b8) [right of=b7] {0};
	
	\node [shape=rectangle,draw,minimum size=\bitsize] (d) [left=of b1] {CF};
	\node [shape=rectangle,draw,minimum size=\bitsize] (c) [right=of b8] {0};
	
	\draw [->] (c.west) -- (b8.east);

	\draw [->] (a2.south) -- (b1.north);
	\draw [->] (a3.south) -- (b2.north);
	\draw [->] (a4.south) -- (b3.north);
	\draw [->] (a5.south) -- (b4.north);
	\draw [->] (a6.south) -- (b5.north);
	\draw [->] (a7.south) -- (b6.north);
	\draw [->] (a8.south) -- (b7.north);
	
	\draw [->] (a1.south) -- (d.north);

	\end{tikzpicture}
\end{center}


\RU{Добавленные биты справа --- всегда нули}\EN{Added bits at right---always zeroes}.

\RU{Умножение на 4 в}\EN{Multiplication by 4 in} ARM:

\begin{lstlisting}[caption=\NonOptimizingKeil + \ARMMode]
f PROC
        LSL      r0,r0,#2
        BX       lr
        ENDP
\end{lstlisting}

\subsubsection{\RU{Деление}\EN{Division}}

\RU{Например}\EN{For example}:

\begin{lstlisting}
unsigned int f(unsigned int a)
{
	return a/4;
};
\end{lstlisting}

\RU{Имеем в итоге}\EN{We got} (MSVC 2010):

\begin{lstlisting}[caption=MSVC 2010]
_a$ = 8							; size = 4
_f	PROC
	mov	eax, DWORD PTR _a$[esp-4]
	shr	eax, 2
	ret	0
_f	ENDP
\end{lstlisting}

\label{SHR}
\index{x86!\Instructions!SHR}
\RU{Инструкция \SHR (\IT{SHift Right}) в данном примере сдвигает число на 2 бита вправо. 
При этом, освободившиеся два бита слева (т.е., самые 
старшие разряды), выставляются в нули. А самые младшие 2 бита выкидываются. 
Фактически, эти два выкинутых бита ~--- остаток от деления.}
\EN{\SHR (\IT{SHift Right}) instruction in this example is shifting a number by 2 bits right.
Two freed bits at left (e.g., two most significant bits) are set to zero.
Two least significant bits are dropped.
In fact, these two dropped bits~---division operation remainder.}

\index{x86!\Instructions!SHR}
\RU{Инструкция \SHR работает так же, как и \SHL, только в другую сторону.}
\EN{\SHR instruction works just like as \SHL but in other direction.}

\input{shift_right}

\label{division_by_shifting}
\RU{Для того, чтобы это проще понять, представьте себе десятичную систему счисления и число $23$. 
$23$ можно разделить на $10$ просто откинув последний разряд ($3$ ~--- это остаток от деления). 
После этой операции останется $2$ как \glslink{quotient}{частное}.}
\EN{It can be easily understood if to imagine decimal numeral system and number $23$.
$23$ can be easily divided by $10$ just by dropping last digit ($3$~---is division remainder). 
$2$ is leaving after operation as a \gls{quotient}.}

\RU{Деление на 4 в}\EN{Division by 4 in} ARM:

\begin{lstlisting}[caption=\NonOptimizingKeil + \ARMMode]
f PROC
        LSR      r0,r0,#2
        BX       lr
        ENDP
\end{lstlisting}

\subsection{\RU{Подсчет выставленных бит}\EN{Counting bits set to 1}}

\RU{Вот этот несложный пример иллюстрирует функцию, считающую количество бит-единиц во входной переменной.}
\EN{Here is a simple example of function, calculating number of $1$ bits in input variable.}

\RU{Эта ф-ция также называется}\EN{This function is also called} ``population count''
\footnote{\RU{современные x86-процессоры (поддерживающие SSE4) даже имеют инструкцию POPCNT для этого}
\EN{modern x86 CPUs (supporting SSE4) even have POPCNT instruction for it}}.

\lstinputlisting{patterns/14_bitfields/3_shifts/shifts.c}

\RU{В этом цикле, счетчик итераций \IT{i} считает от $0$ до $31$, а $1 \ll i$ будет от $1$ до \TT{0x80000000}. 
Описывая это словами, можно сказать 
\IT{сдвинуть единицу на $n$ бит влево}.
Т.е., в некотором смысле, выражение $1 \ll i$ последовательно выдаст все возможные позиции бит в 32-битном числе. 
Кстати, освободившийся бит справа всегда обнуляется.}
\EN{In this loop, iteration count value \IT{i} counting from $0$ to $31$, $1 \ll i$ statement will be counting 
from $1$ to \TT{0x80000000}.
Describing this operation in natural language, we would say \IT{shift $1$ by n bits left}.
In other words, $1 \ll i$ statement will consequently produce all possible bit positions in 32-bit number.
By the way, freed bit at right is always cleared.}

\RU{Вот таблица всех возможных значений}\EN{Here is a table of all possible} $1 \ll i$ 
\RU{для}\EN{for} $i=0 \ldots 31$:

\begin{center}
\begin{tabular}{ | l | l | l | l | }
\hline
\cellcolor{blue!25} \RU{Выражение в }\CCpp\EN{ expression} & 
\cellcolor{blue!25} \RU{Степень двойки}\EN{Power of two} & 
\cellcolor{blue!25} \RU{Десятичная форма}\EN{Decimal form} & 
\cellcolor{blue!25} \RU{Шестнадцатеричная форма}\EN{Hexadecimal form} \\
\hline
$1 \ll 0$ & 1 & 1 & 1 \\
\hline
$1 \ll 1$ & $2^{1}$ & 2 & 2 \\
\hline
$1 \ll 2$ & $2^{2}$ & 4 & 4 \\
\hline
$1 \ll 3$ & $2^{3}$ & 8 & 8 \\
\hline
$1 \ll 4$ & $2^{4}$ & 16 & 0x10 \\
\hline
$1 \ll 5$ & $2^{5}$ & 32 & 0x20 \\
\hline
$1 \ll 6$ & $2^{6}$ & 64 & 0x40 \\
\hline
$1 \ll 7$ & $2^{7}$ & 128 & 0x80 \\
\hline
$1 \ll 8$ & $2^{8}$ & 256 & 0x100 \\
\hline
$1 \ll 9$ & $2^{9}$ & 512 & 0x200 \\
\hline
$1 \ll 10$ & $2^{10}$ & 1024 & 0x400 \\
\hline
$1 \ll 11$ & $2^{11}$ & 2048 & 0x800 \\
\hline
$1 \ll 12$ & $2^{12}$ & 4096 & 0x1000 \\
\hline
$1 \ll 13$ & $2^{13}$ & 8192 & 0x2000 \\
\hline
$1 \ll 14$ & $2^{14}$ & 16384 & 0x4000 \\
\hline
$1 \ll 15$ & $2^{15}$ & 32768 & 0x8000 \\
\hline
$1 \ll 16$ & $2^{16}$ & 65536 & 0x10000 \\
\hline
$1 \ll 17$ & $2^{17}$ & 131072 & 0x20000 \\
\hline
$1 \ll 18$ & $2^{18}$ & 262144 & 0x40000 \\
\hline
$1 \ll 19$ & $2^{19}$ & 524288 & 0x80000 \\
\hline
$1 \ll 20$ & $2^{20}$ & 1048576 & 0x100000 \\
\hline
$1 \ll 21$ & $2^{21}$ & 2097152 & 0x200000 \\
\hline
$1 \ll 22$ & $2^{22}$ & 4194304 & 0x400000 \\
\hline
$1 \ll 23$ & $2^{23}$ & 8388608 & 0x800000 \\
\hline
$1 \ll 24$ & $2^{24}$ & 16777216 & 0x1000000 \\
\hline
$1 \ll 25$ & $2^{25}$ & 33554432 & 0x2000000 \\
\hline
$1 \ll 26$ & $2^{26}$ & 67108864 & 0x4000000 \\
\hline
$1 \ll 27$ & $2^{27}$ & 134217728 & 0x8000000 \\
\hline
$1 \ll 28$ & $2^{28}$ & 268435456 & 0x10000000 \\
\hline
$1 \ll 29$ & $2^{29}$ & 536870912 & 0x20000000 \\
\hline
$1 \ll 30$ & $2^{30}$ & 1073741824 & 0x40000000 \\
\hline
$1 \ll 31$ & $2^{31}$ & 2147483648 & 0x80000000 \\
\hline
\end{tabular}
\end{center}

\RU{Это числа-константы (битовые маски), которые крайне часто попадаются в практике reverse engineer-а, и их нужно
уметь распозновать.}
\EN{These constant numbers (bit masks) are very often appears in code and practicing reverse engineer should quickly
to spot them.}
\RU{Числа в десятичном виде заучивать, пожалуй, незачем, а числа в шестнадцатиричном
виде итак легко запомнить.}
\EN{You probably shouldn't memorize decimal numbers, but hexadecimal ones are very easy to remember.}

\RU{Эти константы очень часто используются для определения отдельных бит как флагов.}
\EN{These constants are very often used for mapping flags to specific bits.}
\RU{Например, это из файла}\EN{For example, here is excerpt from} \TT{ssl\_private.h} \RU{из исходников}
\EN{file from} Apache 2.4.6\EN{ source code}:

\begin{lstlisting}
/**
 * Define the SSL options
 */
#define SSL_OPT_NONE           (0)
#define SSL_OPT_RELSET         (1<<0)
#define SSL_OPT_STDENVVARS     (1<<1)
#define SSL_OPT_EXPORTCERTDATA (1<<3)
#define SSL_OPT_FAKEBASICAUTH  (1<<4)
#define SSL_OPT_STRICTREQUIRE  (1<<5)
#define SSL_OPT_OPTRENEGOTIATE (1<<6)
#define SSL_OPT_LEGACYDNFORMAT (1<<7)
\end{lstlisting}

\RU{Вернемся назад к нашему примеру}\EN{Let's back to our example}.

\RU{Макрос \TT{IS\_SET} проверяет наличие этого бита в \TT{a}.}
\EN{\TT{IS\_SET} macro is checking bit presence in the \TT{a}.}

\index{x86!\Instructions!AND}
\RU{Макрос \TT{IS\_SET} на самом деле это операция логического И (\IT{AND}) 
и она возвращает $0$ если бита там нет, 
либо эту же битовую маску, если бит там есть. 
В \CCpp, конструкция \TT{if()} срабатывает, если выражение внутри её не ноль, пусть хоть $123456$, 
поэтому все будет работать.}
\EN{The \TT{IS\_SET} macro is in fact logical and operation (\IT{AND}) 
and it returns $0$ if specific bit is absent there,
or bit mask, if the bit is present.
\IT{if()} operator triggered in \CCpp if expression in it is not a zero, it might be even $123456$, that is why
it always working correctly.}

% subsubsections
\subsection{x86: \IFRU{3 аргумента}{3 arguments}}

\subsubsection{MSVC}

\IFRU{Компилируем при помощи MSVC 2010 Express, и в итоге получим:}
{Let's compile it by MSVC 2010 Express and we got:}

\begin{lstlisting}
$SG3830	DB	'a=%d; b=%d; c=%d', 00H

...

	push	3
	push	2
	push	1
	push	OFFSET $SG3830
	call	_printf
	add	esp, 16					; 00000010H
\end{lstlisting}

\IFRU{Все почти то же, за исключением того, что теперь видно, что аргументы для \printf заталкиваются в стек в обратном порядке: самый первый аргумент заталкивается последним.}
{Almost the same, but now we can see the \printf arguments are pushing into stack in reverse order: and the first argument is pushing in as the last one.}

\IFRU{Кстати, вспомним что переменные типа \Tint в 32-битной системе, как известно, имеет ширину 32 бита, это 4 байта}
{By the way, variables of \Tint type in 32-bit environment has 32-bit width that is 4 bytes}.

\IFRU{Итак, у нас всего 4 аргумента. $4*4 = 16$ ~--- именно 16 байт занимают в стеке указатель на строку плюс еще 3 числа типа \Tint.}
{So, we got here 4 arguments. $4*4 = 16$~---they occupy exactly 16 bytes in the stack: 32-bit pointer to string and 3 number of \Tint type.}

\index{x86!\Instructions!ADD}
\index{x86!\Registers!ESP}
\index{cdecl}
\IFRU{Когда при помощи инструкции \TT{``ADD ESP, X''} корректируется \glslink{stack pointer}{указатель стека} \ESP 
после вызова какой-либо функции, зачастую можно сделать вывод о том, сколько аргументов 
у вызываемой функции было, разделив X на 4.}
{When \gls{stack pointer} (the \ESP register) is corrected by \TT{``ADD ESP, X''}
instruction after a function 
call, often, the number of function arguments could be deduced here: just divide X by 4.}

\IFRU{Конечно, это относится только к cdecl-методу передачи аргументов через стек.}
{Of course, this is related only to \IT{cdecl} calling convention.}

\IFRU{См. также в соответствующем разделе о способах передачи аргументов через стек}
{See also section about calling conventions}~(\ref{sec:callingconventions}).

\IFRU{Иногда бывает так, что подряд идут несколько вызовов разных функций, 
но стек корректируется только один раз, после последнего вызова:}
{It is also possible for compiler to merge several \TT{``ADD ESP, X''} instructions into one, after last call:}

\begin{lstlisting}
push a1
push a2
call ...
...
push a1
call ...
...
push a1
push a2
push a3
call ...
add esp, 24
\end{lstlisting}

\subsubsection{MSVC \AndENRU \olly}
\index{\olly}

\IFRU{Попробуем этот же пример в}{Now let's try to load this example in} \olly.
\IFRU{Это один из наиболее популярных win32-отладчиков user-режима}{It is one of the most 
popular user-land win32 debugger}.
\IFRU{Мы можем компилировать наш пример в}{We can try to compile our example in} MSVC 2012 
\IFRU{с опцией}{with} \TT{/MD} \IFRU{что означает, линковать с библиотекой}{option, meaning, to link 
against} \TT{MSVCR*.DLL},
\IFRU{чтобы импортируемые ф-ции были хорошо видны в отладчике}{so we will able to see imported 
functions clearly in debugger}.

\IFRU{Затем загружаем исполняемый файл в}{Then load executable in} \olly.
\IFRU{Самый первый брякпойнт в}{The very first breakpoint is in} \TT{ntdll.dll}, \IFRU{нажмите}{press} 
F9 (\IFRU{запустить}{run}).
\IFRU{Второй брякпойнт в}{The second breakpoint is in} \ac{CRT}-\IFRU{коде}{code}.
\IFRU{Теперь мы должны найти ф-цию}{Now we should find} \main\EN{ function}.

\IFRU{Найдите этот код скроллируя окно кода до самого верха (MSVC располагает ф-цию \main в самом начале
секции кода)}{Find this code by scrolling the code to the very bottom (MSVC allocates \main function at
the very beginning of the code section)}: 
\figname \ref{fig:printf3_olly_1}.

\IFRU{Кликните на инструкции}{Click on} \TT{PUSH EBP}\IFRU{, нажмите}{ instruction, press} F2 
(\IFRU{установка брякпойнта}{set breakpoint}) \IFRU{и нажмите}{and press} F9 (\IFRU{запустить}{run}).
\IFRU{Нам нужно произвести все эти манипуляции, чтобы пропустить \ac{CRT}-код, потому что нам он пока
не интересен}{We need to do these manupulations in order to skip \ac{CRT}-code, because, we don't really 
interesting in it yet}.

\IFRU{Нажмите}{Press} F8 (\stepover) 6 \IFRU{раз, т.е., пропустить
6 инструкций}{times, i.e., skip 6 instructions}: \figname \ref{fig:printf3_olly_2}.

\IFRU{Теперь}{Now the} \PC \IFRU{указывает на инструкцию}{points to the}
\TT{CALL printf}\EN{ instruction}.
\olly, \IFRU{как и другие отладчики, подсвечивает регистры со значениями, которые изменились}
{like other debuggers, highlights value of registers which were changed}.
\IFRU{Так что, каждый раз, когда мы нажимаем}{So each time you press F8}, \EIP 
\IFRU{изменяется и его значение подсвечивается красным}{is changing and its value looking red}.
\ESP \IFRU{также меняется, потому что значения заталкиваются в стек}{is changing as well, 
because values are pushed into the stack}.

\IFRU{Где находятся эти значения в стеке}{Where are the values in the stack}?
\IFRU{Посмотрите на правое/нижнее окно в отладчике}{Take a look into right/bottom window of debugger}:

\begin{figure}[H]
\centering
\includegraphics[scale=0.66]{patterns/03_printf/olly3_stack.png}
\caption{\olly: \IFRU{стек, после того как значения там сохранены}{stack after values pushed}
(\IFRU{я сделал здесь округлую красную пометку в графическом редакторе}{I made round red mark 
here in graphics editor})}
\end{figure}

\IFRU{Так что здесь видно 3 столбца: адрес в стеке, значение в стеке и еще дополнительный комментарий
от \olly}{So we can see there 3 columns: address in the stack, 
value in the stack and some additional \olly comments}. 
\olly \IFRU{понимает}{understands} \printf\IFRU{-строки}{-like strings}, 
\IFRU{так что он показывает здесь и строку и 3 значения \IT{привязанных} к ней}{so it reports the 
string here and 3 values \IT{attached} to it}.

\IFRU{Нажмите}{Press} F8 (\stepover).

\IFRU{В коносил мы видим вывод}{In the console we'll see the output}:

\begin{figure}[H]
\centering
\includegraphics[scale=0.66]{patterns/03_printf/olly3_console.png}
\caption{\RU{Ф-ция }\printf \IFRU{исполнилась}{function executed}}
\end{figure}

\IFRU{Посмотрим, как изменились регистры и состояние стека}{Let's see how registers and stack state 
are changed}: \figname \ref{fig:printf3_olly_3}.

\RU{Регистр }\EAX \IFRU{теперь содержит}{register now contains} \TT{0xD} (13).
That's correct, \printf returns number of characters printed.
\RU{Значение }\EIP \IFRU{изменилось: действительно, теперь здесь адрес инструкции после}
{value is changed: indeed, now there is address of the instruction after} \TT{CALL printf}.
\RU{Значения регистров }\ECX \AndENRU \EDX \IFRU{также изменились}{values are changed as well}.
\IFRU{Очевидно, внутренности ф-ции \printf используют их для каких-то своих нужд}{Apparently, 
\printf function's hidden machinery used them for its own needs}.

\IFRU{Очень важный момент в том что значение \ESP не изменилось. И состояние стека также!}
{A very important thing is that \ESP value is not changed. And stack state too!}
\IFRU{Мы ясно видим здесь и строку формата и соответствующие ей 3 значения, они все еще здесь}
{We clearly see that format string and corresponding 3 values are still there}.
\IFRU{Действительно, по соглашению вызовов \IT{cdecl}, вызывающая ф-ция не очищает аргументы из стека}
{Indeed, that's \IT{cdecl} calling convention, calling function doesn't clear arguments in stack}.
\IFRU{Это должна делать вызывающая ф-ция}{It's caller's duty to do so}.

\IFRU{Нажмите}{Press} F8 \IFRU{снова, чтобы исполнилась инструкция}{again to execute} 
\TT{ADD ESP, 10}\EN{ instruction}: \figname \ref{fig:printf3_olly_4}.

\ESP \IFRU{изменился, но значения все еще в стеке}{is changed, but values are still in the stack}!
\IFRU{Конечно, никому не нужно заполнять эти значения нулями или что-то в этом роде}{Yes, 
of course, no one needs to fill these values by zero or something like that}.
\IFRU{Потому что всё что выше указателя стека}{Because, everything above stack pointer} (\SP) 
\IFRU{это}{is} \IT{\IFRU{шум}{noise}} \OrENRU \IT{\IFRU{мусор}{garbage}}, \IFRU{это всё не имеет
особой ценности}{it has no value at all}.
\IFRU{Было бы очень затратно по времени очищать ненужные элементы стека, к тому же, никому это и не 
нужно}{It would be time consuming to clear unused stack entries, besides, no one really needs to}.

\begin{figure}[H]
\centering
\includegraphics[scale=0.66]{patterns/03_printf/olly3_1.png}
\caption{\olly: \IFRU{самое начало ф-ции}{the very start of the} \main\EN{ function}}
\label{fig:printf3_olly_1}
\end{figure}

\begin{figure}[H]
\centering
\includegraphics[scale=0.66]{patterns/03_printf/olly3_2.png}
\caption{\olly: \IFRU{перед исполнением}{before} \printf\EN{ execution}}
\label{fig:printf3_olly_2}
\end{figure}

\begin{figure}[H]
\centering
\includegraphics[scale=0.66]{patterns/03_printf/olly3_3.png}
\caption{\olly: \IFRU{после исполнения}{after} \printf\EN{ execution}}
\label{fig:printf3_olly_3}
\end{figure}

\begin{figure}[H]
\centering
\includegraphics[scale=0.66]{patterns/03_printf/olly3_4.png}
\caption{\olly: \IFRU{после исполнения инструкции}{after} \TT{ADD ESP, 10}\EN{ instruction execution}}
\label{fig:printf3_olly_4}
\end{figure}

\subsubsection{GCC}

\IFRU{Скомпилируем то же самое в Linux при помощи GCC 4.4.1 и посмотрим в \IDA что вышло:}
{Now let's compile the same in Linux by GCC 4.4.1 and take a look in \IDA what we got:}

\begin{lstlisting}
main            proc near

var_10          = dword ptr -10h
var_C           = dword ptr -0Ch
var_8           = dword ptr -8
var_4           = dword ptr -4

                push    ebp
                mov     ebp, esp
                and     esp, 0FFFFFFF0h
                sub     esp, 10h
                mov     eax, offset aADBDCD ; "a=%d; b=%d; c=%d"
                mov     [esp+10h+var_4], 3
                mov     [esp+10h+var_8], 2
                mov     [esp+10h+var_C], 1
                mov     [esp+10h+var_10], eax
                call    _printf
                mov     eax, 0
                leave
                retn
main            endp
\end{lstlisting}

\IFRU{Можно сказать, что этот короткий код, созданный GCC, отличается от кода MSVC только способом помещения 
значений в стек.
Здесь GCC снова работает со стеком напрямую без \PUSH/\POP.}
{It can be said, the difference between code by MSVC and GCC is only in method of placing arguments on the stack.
Here GCC working directly with stack without \PUSH/\POP.}

\section{ARM}

\subsection{\NonOptimizingXcode + \ARMMode}

\lstinputlisting[caption=\NonOptimizingXcode + \ARMMode]{patterns/10_strlen/xcode_ARM_O0_en.asm}

\IFRU{Неоптимизирующий LLVM генерирует слишком много кода, зато на этом примере можно посмотреть, 
как функции работают с локальными переменными в стеке.}
{Non-optimizing LLVM generates too much code, however, here we can see how function works with 
local variables in the stack.}
\IFRU{В нашей функции только локальных переменных две, это два указателя}
{There are only two local variables in our function},
\IT{eos} \AndENRU \IT{str}.

\IFRU{В этом листинге}{In this listing}, \IFRU{сгенерированном при помощи}{generated by} \IDA, 
\IFRU{я переименовал}{I renamed} \IT{var\_8} \AndENRU \IT{var\_4} \IFRU{в}{into} \IT{eos} 
\AndENRU \IT{str} \IFRU{вручную}{manually}.

\IFRU{Итак, первые несколько инструкций просто сохраняют входное значение в переменных}{So, 
first instructions are just saves input value in} \IT{str} \AndENRU \IT{eos}.

\IFRU{Начиная с метки}{Loop body is beginning at} \IT{loc\_2CB8}\IFRU{, начинается тело цикла}{ label}.

\IFRU{Первые три инструкции в теле цикла}{First three instruction in loop body} (\TT{LDR}, \ADD, \TT{STR}) 
\IFRU{загружают значение}{loads} \IT{eos} \IFRU{в}{value into} \Reg{0}, 
\IFRU{затем происходит инкремент значения и оно сохраняется назад в локальной переменной \IT{eos} расположенной 
в стеке.}{then value is \glslink{increment}{incremented} and it is saved back into \IT{eos} local variable located in the stack.}

\index{ARM!\Instructions!LDRSB}
\IFRU{Следующая инструкция}{The next} \TT{``LDRSB R0, [R0]''} (\IT{Load Register Signed Byte}) 
\IFRU{загружает байт из памяти по адресу \Reg{0}, расширяет его до 32-бит считая его знаковым (signed) 
и сохраняет в \Reg{0}}{instruction loading byte from memory at \Reg{0} address and sign-extends it to 32-bit}.
\index{x86!\Instructions!MOVSX}
\IFRU{Это немного похоже на инструкцию}{This is similar to} \MOVSX \IFRU{в}{instruction in} x86.
\IFRU{Компилятор считает этот байт знаковым (signed), потому что тип \Tchar по стандарту Си ~--- знаковый.}
{The compiler treating this byte as signed since \Tchar type in C standard is signed.}
\IFRU{Об это я уже немного писал}{I already wrote about it}~(\ref{MOVSX}) \IFRU{в этой же секции, 
но посвященной x86}{in this section, but related to x86}.

\index{x86!8086}
\index{8080}
\index{ARM}
\IFRU{Следует также заметить, что, в ARM нет возможности использовать 8-битную или 16-битную часть 
регистра, как это возможно в x86.}
{It is should be noted, it is impossible in ARM to use 8-bit part or 16-bit part 
of 32-bit register separately of the whole register,
as it is in x86.}
\IFRU{Вероятно, это связано с тем что за x86 тянется длинный шлейф совместимости со своими предками, 
такими как
16-битный 8086 и даже 8-битный 8080, а ARM разрабатывался с чистого листа как 32-битный RISC-процессор.}
{Apparently, it is because x86 has a huge history of compatibility with its ancestors like 16-bit 8086 
and even 8-bit 8080,
but ARM was developed from scratch as 32-bit RISC-processor.}
\IFRU{Следовательно, чтобы работать с отдельными байтами на ARM, так или иначе, придется использовать 
32-битные регистры.}
{Consequently, in order to process separate bytes in ARM, one have to use 32-bit registers anyway.}

\IFRU{Итак}{So}, \TT{LDRSB} \IFRU{загружает символ из строки в \Reg{0}, по одному}
{loads symbol from string into \Reg{0}, one by one}.
\IFRU{Следующие инструкции}{Next} \CMP \AndENRU \ac{BEQ} \IFRU{проверяют, является ли этот символ $0$.}
{instructions checks, if loaded symbol is $0$.}
\IFRU{Если не $0$, то происходит переход на начало тела цикла.}{If not $0$, control passing to loop body
begin.}
\IFRU{А если $0$, выходим из цикла.}{And if $0$, loop is finishing.}

\IFRU{В конце функции вычисляется разница между}{At the end of function, a difference between} 
\IT{eos} \AndENRU \IT{str}\IFRU{, вычитается еще единица и вычисленное 
значение возвращается через \Reg{0}.}{ is calculated, 1 is also subtracting, and resulting value is returned
via \Reg{0}.}

N.B. \IFRU{В этой функции не сохранялись регистры}{Registers was not saved in this function}.
\index{ARM!\Registers!scratch registers}
\IFRU{Это потому что, по стандарту, регистры \Reg{0}-\Reg{3} называются также ``scratch registers'',
они предназначены для передачи аргументов, 
их значения не нужно восстанавливать при выходе из функции, потому что они больше не нужны в вызывающей функции.
Таким образом, их можно использовать как захочется}
{That's because by ARM calling convention, \Reg{0}-\Reg{3} registers are ``scratch registers'', 
they are intended for arguments passing,
its values may not be restored upon function exit since calling function will not use them anymore.
Consequently, they may be used for anything we want.}
\IFRU{А так как никакие больше регистры не используются, то и сохранять нечего.}
{Other registers are not used here, so that is why we have nothing to save on the stack.}
\IFRU{Поэтому, управление можно вернуть назад вызывающей функции 
простым переходом (\TT{BX}), по адресу в регистре \LR.}
{Thus, control may be returned back to calling function by simple jump (\TT{BX}),
to address in the \LR register.}

%\subsection{\NonOptimizingXcode + режим thumb}
%Практически, точно такой же код.

\subsection{\OptimizingXcode + \ThumbMode}

\lstinputlisting[caption=\OptimizingXcode + \ThumbMode]{patterns/10_strlen/xcode_thumb_O3.asm}

\IFRU{Оптимизирующий LLVM решил, что под переменные \IT{eos} и \IT{str} выделять место в стеке не обязательно}
{As optimizing LLVM concludes, space on the stack for \IT{eos} and \IT{str} may not be allocated},
\IFRU{и эти переменные можно хранить прямо в регистрах.}
{and these variables may always be stored right in registers.}
\IFRU{Перед началом тела цикла}{Before loop body beginning}, \IT{str} \IFRU{будет находиться в}{will always be in} 
\Reg{0}, \IFRU{а}{and} \IT{eos}\EMDASH\InENRU \Reg{1}.

\index{ARM!\Instructions!LDRB.W}
\index{ARM!\IFRU{Режимы адресации}{Adressing modes}}
\RU{Инструкция }\TT{``LDRB.W R2, [R1],\#1''} \IFRU{загружает в \Reg{2} байт из памяти по адресу \Reg{1}, 
расширяя его как знаковый (signed), до 32-битного
значения, но не только это.}
{instruction loads byte from memory at the address \Reg{1} into \Reg{2}, sign-extending it to 32-bit value, but not
only that.}
\TT{\#1} \IFRU{в конце инструкции называется}{at the instruction's end calling} ``Post-indexed addressing'', 
\IFRU{это значит, что после загрузки байта, к \Reg{1} добавится единица.}{this means, $1$ is to be added
to the \Reg{1} after byte load.}
\IFRU{Это очень удобно для работы с массивами.}
{That's convenient when accessing arrays.}

\index{PDP-11}
\index{\CLanguageElements!\PostIncrement}
\index{\CLanguageElements!\PostDecrement}
\index{\CLanguageElements!\PreIncrement}
\index{\CLanguageElements!\PreDecrement}
\IFRU{Такого режима адресации в x86 нет, но он есть в некоторых других процессорах, даже на PDP-11.}
{There is no such addressing mode in x86, but it is present in some other processors, even on PDP-11.}
\IFRU{Существует байка, что режимы пре-инкремента, пост-инкремента, 
пре-декремента и пост-декремента адреса в PDP-11}
{There is a legend the pre-increment, post-increment, pre-decrement and post-decrement modes in PDP-11},
\IFRU{были ``виновны'' в появлении таких конструкций языка Си (который разрабатывался на PDP-11) как}
{were ``guilty'' in appearance such C language (which developed on PDP-11) constructs as}
*ptr++, *++ptr, *ptr-{}-, *-{}-ptr. 
\IFRU{Кстати, это является труднозапоминаемой особенностью в Си.}
{By the way, this is one of hard to memorize C feature.}
\IFRU{Дела обстоят так:}{This is how it is:}

\begin{center}
\begin{tabular}{ | l | l | l | l | }
\hline
\headercolor{} \IFRU{термин в Си}{C term} & 
\headercolor{} \IFRU{термин в ARM}{ARM term} & 
\headercolor{} \IFRU{выражение Си}{C statement} & 
\headercolor{} \IFRU{как это работает}{how it works} \\
\hline
\PostIncrement & 
post-indexed addressing & 
\TT{*ptr++} & 
\IFRU{использовать значение \TT{*ptr}}{use \TT{*ptr} value}, \\
& & & \IFRU{затем инкремент указателя \TT{ptr}}{then \gls{increment} \TT{ptr} pointer} \\
\hline
\PostDecrement & 
post-indexed addressing & 
\TT{*ptr-{}-} & 
\IFRU{использовать значение \TT{*ptr}}{use \TT{*ptr} value}, \\
& & & \IFRU{затем \glslink{decrement}{декремент} указателя \TT{ptr}}{then \gls{decrement} \TT{ptr} pointer} \\
\hline
\PreIncrement & 
pre-indexed addressing & 
\TT{*++ptr} & 
\IFRU{инкремент указателя \TT{ptr}}{\gls{increment} \TT{ptr} pointer}, \\
& & & \IFRU{затем использовать значение \TT{*ptr}}{then use \TT{*ptr} value} \\
\hline
\PreDecrement & 
post-indexed addressing & 
\TT{*-{}-ptr} & 
\IFRU{\glslink{decrement}{декремент} указателя \TT{ptr}}{\gls{decrement} \TT{ptr} pointer}, \\
& & & \IFRU{затем использовать значение \TT{*ptr}}{then use \TT{*ptr} value} \\
\hline
\end{tabular}
\end{center}

\IFRU{Деннис Ритчи (один из создателей ЯП Си) указывал, что, это, вероятно, придумал Кен Томпсон 
(еще один создатель Си),
потому что подобная возможность процессора имелась еще в PDP-7}
{Dennis Ritchie (one of C language creators) mentioned that it is, probably, was invented by Ken Thompson
(another C creator) because this processor feature was present in PDP-7}
\cite{Ritchie:1986}\cite{Ritchie:1993:DCL:155360.155580}.
\IFRU{Таким образом, компиляторы с ЯП Си на тот процессор, где это есть, могут использовать это.}
{Thus, C language compilers may use it, if it is present in target processor.}

\IFRU{Далее в теле цикла можно увидеть \CMP и \ac{BNE}, они продолжают работу цикла до тех пор, 
пока не будет встречен $0$.}
{Then one may spot \CMP and \ac{BNE} in loop body, these instructions continue operation until
$0$ will be met in string.}

\index{ARM!\Instructions!MVNS}
\index{x86!\Instructions!NOT}
\RU{После конца цикла }\TT{MVNS}\footnote{MoVe Not} 
\IFRU{(инвертирование всех бит, аналог \NOT на x86)}
{(inverting all bits, \NOT in x86 analogue)}
\IFRU{и \ADD вычисляют}{instructions and \ADD computes} $eos - str - 1$.
\IFRU{На самом деле, эти две инструкции вычисляют}
{In fact, these two instructions computes}
$R0 = ~str + eos$, 
\IFRU{что эквивалентно тому, что было в исходном коде, а почему это так, я уже описывал чуть раньше, здесь}
{which is effectively equivalent to what was in source code, and why it is so, I already described here}
~(\ref{strlen_NOT_ADD}).

\IFRU{Вероятно, LLVM, как и GCC, посчитал что такой код будет короче, или быстрее.}
{Apparently, LLVM, just like GCC, concludes this code will be shorter, or faster.}

%\subsection{\OptimizingXcode + \ARMMode}
%Практически, точно такой же код.

\subsection{\OptimizingKeil{} + \ARMMode}

\lstinputlisting[caption=\OptimizingKeil + \ARMMode]{patterns/10_strlen/Keil_ARM_O3.asm}

\index{ARM!\Instructions!SUBEQ}
\IFRU{Практически то же самое что мы уже видели, за тем исключением что выражение}
{Almost the same what we saw before, with the exception the}
$str - eos - 1$ 
\IFRU{может быть вычислено не в самом конце функции, а прямо в теле цикла.}
{expression may be computed not at the function's end, but right in loop body.}
\RU{Суффикс }\TT{-EQ}\IFRU{, как мы помним, означает что инструкция будет выполнена только
если операнды в исполненной перед этим инструкции \CMP были равны.}
{suffix, as we may recall, means the instruction will be executed only if operands in executed before
\CMP were equal to each other.}
\IFRU{Таким образом}{Thus}, \IFRU{если в \Reg{0} будет $0$}{if $0$ will be in the \Reg{0} register},
\IFRU{обе инструкции}{both} \TT{SUBEQ} \IFRU{исполнятся и результат останется в \Reg{0}.}
{instructions are to be executed and result is leaving in the \Reg{0} register.}


