\subsection{\ShiftsSectionName}
Bitverschiebungen sind in \CCpp mit den Befehlen $\ll$ und $\gg$ implementiert.
Die x86 \ac{ISA} verwendet die Befehle SHL (SHift Left) und SHR (SHift Right) zu
diesem Zweck.
Schiebebefehle werden oft bei der Division und Multiplikation mit Potenzen von 2
$2^n$ (d.h. 1,2,4,8, etc.) verwendet:
\myref{subsec:mult_using_shifts},
\myref{division_by_shifting}.

% FIXME: rework this
Schiebebefehle sind auch wichtig, da sie oft für die Isolation einzelnes Bits
oder für die Konstruktion von Werten aus mehreren einzelnen Bits verwendet
werden.
