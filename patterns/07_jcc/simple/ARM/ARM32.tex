\subsubsection{32-\RU{битный}\EN{bit} ARM}
\label{subsec:jcc_ARM}

\myparagraph{\OptimizingKeilVI (\ARMMode)}

\lstinputlisting[caption=\OptimizingKeilVI (\ARMMode)]{patterns/07_jcc/simple/ARM/ARM_O3_signed.asm}

\index{ARM!Condition codes}
% FIXME \ref -> which instructions?
\RU{Многие инструкции в режиме ARM могут быть исполнены только при некоторых выставленных флагах.}
\EN{Many instructions in ARM mode could be executed only when specific flags are set.}
\RU{Это нередко используется для сравнения чисел.}
\EN{E.g. this is often used when comparing numbers.}

\index{ARM!\Instructions!ADD}
\index{ARM!\Instructions!ADDAL}
\RU{К примеру, инструкция \ADD на самом деле называется \TT{ADDAL} внутри, \TT{AL} означает 
\IT{Always}, то есть, исполнять всегда.}
\EN{For instance, the \ADD instruction is in fact named \TT{ADDAL} internally, where \TT{AL} stands for
\IT{Always}, i.e., execute always.}
\RU{Предикаты кодируются в 4-х старших битах инструкции 32-битных ARM-инструкций}
\EN{The predicates are encoded in 4 high bits of the 32-bit ARM instructions} (\IT{condition field}).
\index{ARM!\Instructions!B}
\RU{Инструкция безусловного перехода \TT{B} на самом деле условная и кодируется так же, 
как и прочие инструкции условных переходов, но имеет \TT{AL} в \IT{condition field}, 
то есть исполняется всегда (\IT{execute ALways}), игнорируя флаги.}
\EN{The \TT{B} instruction for unconditional jumping is in fact conditional and encoded just like any other
conditional jump, but has \TT{AL} in the \IT{condition field}, and it implies \IT{execute ALways}, 
ignoring flags.}

\index{ARM!\Instructions!ADR}
\index{ARM!\Instructions!ADRcc}
\index{ARM!\Instructions!CMP}
\RU{Инструкция \TT{ADRGT} работает так же, как и \TT{ADR}, но исполняется только в случае,
если предыдущая инструкция \CMP,
сравнивая два числа, обнаруживает, что одно из них больше второго (\IT{Greater Than}).}
\EN{The \TT{ADRGT} instruction works just like \TT{ADR} but executes only in case the previous \CMP
instruction founds one of the numbers greater than the another, while comparing the two (\IT{Greater Than}).}

\index{ARM!\Instructions!BL}
\index{ARM!\Instructions!BLcc}
\RU{Следующая инструкция \TT{BLGT} ведет себя так же, как и \TT{BL} и сработает, только если 
результат сравнения был такой же}
\EN{The next \TT{BLGT} instruction behaves exactly as \TT{BL} 
and is triggered only if the result of the comparison was the same} (\IT{Greater Than}). 
\TT{ADRGT} \RU{записывает в \Reg{0} указатель на строку}\EN{writes a pointer to the string} 
\TT{a>b\textbackslash{}n}
\RU{, а \TT{BLGT} вызывает}\EN{ into \Reg{0} and \TT{BLGT} calls} \printf.
\RU{Следовательно, эти инструкции с суффиксом \TT{-GT} исполнятся только в том случае, если значение
в \Reg{0} (там $a$) было больше, чем значение в \Reg{4} (там $b$).}
\EN{therefore, these instructions with suffix \TT{-GT} are to execute only in case the value in \Reg{0} (which is $a$) is bigger than the value in \Reg{4} (which is $b$).}

\index{ARM!\Instructions!ADRcc}
\index{ARM!\Instructions!BLcc}
\RU{Далее мы увидим инструкции \TT{ADREQ} и \TT{BLEQ}.}
\EN{Moving forward we see the \TT{ADREQ} and \TT{BLEQ} instructions.}
\RU{Они работают так же, как и \TT{ADR} и \TT{BL}, но исполнятся только если значения при последнем сравнении были равны.}
\EN{They behave just like \TT{ADR} and \TT{BL}, but are to be executed only if operands were equal to each
other during the last comparison.}
\RU{Перед ними расположен ещё один \CMP, потому что вызов \printf мог испортить состояние флагов.}
\EN{Another \CMP is located before them (because the \printf execution may have tampered the flags).}

\index{ARM!\Instructions!LDMccFD}
\index{ARM!\Instructions!LDMFD}
\RU{Далее мы увидим \TT{LDMGEFD}. Эта инструкция работает так же, как и \TT{LDMFD}\footnote{\ac{LDMFD}}, 
но сработает только если в результате сравнения одно из значений было больше или равно второму}
\EN{Then we see \TT{LDMGEFD}, this instruction works just like \TT{LDMFD}\footnote{\ac{LDMFD}},
but is triggered only when one of the values is greater or equal than the other}
(\IT{Greater or Equal}).


\RU{Смысл инструкции}\EN{The } \TT{LDMGEFD SP!, \{R4-R6,PC\}} 
\RU{в том, что это как бы эпилог функции, но он сработает только если $a>=b$, только тогда работа 
функции закончится.}
\EN{instruction acts like a function epilogue, but it will be triggered only if $a>=b$, and only then the function execution will finish.}

\index{Function epilogue}
\RU{Но если это не так, то есть $a<b$, то исполнение дойдет до следующей инструкции 
\TT{LDMFD SP!, \{R4-R6,LR\}}. Это ещё один эпилог функции. Эта инструкция восстанавливает состояние регистров
\TT{R4-R6}, но и \ac{LR} вместо \ac{PC}, таким образом, пока что, не делая возврата из функции.}
\EN{But if that condition is not satisfied, i.e., $a<b$, then the control flow will continue to the next \TT{\q{LDMFD SP!, \{R4-R6,LR\}}} instruction, which is one more function epilogue. This instruction restores not only the \TT{R4-R6} registers state, but also \ac{LR} instead of \ac{PC}, thus, it does not returns from the function.}

\RU{Последние две инструкции вызывают}\EN{The last two instructions call} \printf 
\RU{со строкой}\EN{with the string} <<a<b\textbackslash{}n>> \RU{в качестве единственного аргумента}\EN{as 
a sole argument}.
\RU{Безусловный переход на \printf вместо возврата из функции мы уже рассматривали в секции}
\EN{We already examined an unconditional jump to the \printf function instead of function return in} <<\PrintfSeveralArgumentsSectionName>>\EN{ section}~(\myref{ARM_B_to_printf}).

\index{ARM!\Instructions!ADRcc}
\index{ARM!\Instructions!BLcc}
\index{ARM!\Instructions!LDMccFD}
\RU{Функция }\TT{f\_unsigned} \RU{точно такая же, но там используются инструкции}\EN{is similar,
only the} \TT{ADRHI}, \TT{BLHI}, \AndENRU \TT{LDMCSFD}\RU{. Эти предикаты}\EN{ instructions are
used there, these predicates}
(\IT{HI = Unsigned higher, CS = Carry Set (greater than or equal)}) 
\RU{аналогичны рассмотренным, но служат для работы с беззнаковыми значениями.}
\EN{are analogous to those examined before, but for unsigned values.}

\RU{В функции \main ничего нового для нас нет:}
\EN{There is not much new in the \main function for us:}
\lstinputlisting[caption=\main]{patterns/07_jcc/simple/ARM/ARM_O3_main.asm}

\RU{Так, в режиме ARM можно обойтись без условных переходов.}
\EN{That is how you can get rid of conditional jumps in ARM mode.}

\index{\RU{Конвейер RISC}\EN{RISC pipeline}}
\RU{Почему это хорошо?}\EN{Why is this so good?} \RU{Читайте здесь}\EN{Read here}: \myref{branch_predictors}.

\index{x86!\Instructions!CMOVcc}
\RU{В x86 нет аналогичной возможности, если не считать инструкцию \TT{CMOVcc}, это то же что и \MOV, 
но она срабатывает
только при определенных выставленных флагах, обычно выставленных при помощи \CMP во время сравнения.}
\EN{There is no such feature in x86, except the \TT{CMOVcc} instruction, it is the same as \MOV,
but triggered only when specific flags are set, usually set by \CMP.}

\myparagraph{\OptimizingKeilVI (\ThumbMode)}

\lstinputlisting[caption=\OptimizingKeilVI (\ThumbMode)]{patterns/07_jcc/simple/ARM/ARM_thumb_signed.asm}

\index{ARM!\Instructions!BLE}
\index{ARM!\Instructions!BNE}
\index{ARM!\Instructions!BGE}
\index{ARM!\Instructions!BLS}
\index{ARM!\Instructions!BCS}
\index{ARM!\Instructions!B}
\index{ARM!\ThumbMode}
\RU{В режиме Thumb только инструкции \TT{B} могут быть дополнены условием исполнения (\IT{condition code}), 
так что код для режима Thumb выглядит привычнее.}
\EN{Only \TT{B} instructions in Thumb mode may be supplemented by \IT{condition codes}, so the Thumb code 
looks more ordinary.}

\TT{BLE} \RU{это обычный переход с условием}\EN{is a normal conditional jump} \IT{Less than or Equal}, 
\TT{BNE}\EMDASH\IT{Not Equal}, 
\TT{BGE}\EMDASH\IT{Greater than or Equal}.

\RU{Функция }\TT{f\_unsigned} \RU{точно такая же, но для работы с беззнаковыми величинами 
там используются инструкции}\EN{is similar, only other instructions are used while dealing 
with unsigned values:} \TT{BLS} 
(\IT{Unsigned lower or same}) \AndENRU \TT{BCS} (\IT{Carry Set (Greater than or equal)}).
