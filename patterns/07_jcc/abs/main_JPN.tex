\subsection{絶対値の計算}
\label{sec:abs}

簡単な関数の例。

\lstinputlisting[style=customc]{abs.c}

\subsubsection{\Optimizing MSVC}

これは普通、どのようにコードが生成されるのかを示したものです。

\lstinputlisting[caption=\Optimizing MSVC 2012 x64,style=customasmx86]{patterns/07_jcc/abs/abs_MSVC2012_Ox_x64_JPN.asm}

GCC 4.9はほとんど同じです。

\subsubsection{\OptimizingKeilVI: \ThumbMode}

\lstinputlisting[caption=\OptimizingKeilVI: \ThumbMode,style=customasmARM]{patterns/07_jcc/abs/abs_Keil_thumb_O3_JPN.s}

\myindex{ARM!\Instructions!RSB}

ARMにはネゲート命令がないため、Keilコンパイラは\q{逆引き命令}を使用します。これは逆のオペランドで減算するだけです。

\subsubsection{\OptimizingKeilVI: \ARMMode}

ARMモードでは、いくつかの命令に条件コードを追加することができます。そのため、Keilコンパイラは次のように処理します。

\lstinputlisting[caption=\OptimizingKeilVI: \ARMMode,style=customasmARM]{patterns/07_jcc/abs/abs_Keil_ARM_O3_JPN.s}

今度は条件付きジャンプはありません。これは良いですね。:\myref{branch_predictors}

\subsubsection{\NonOptimizing GCC 4.9 (ARM64)}

\myindex{ARM!\Instructions!XOR}

ARM64には、否定するための命令\INS{NEG}があります。

\lstinputlisting[caption=\Optimizing GCC 4.9 (ARM64),style=customasmARM]{patterns/07_jcc/abs/abs_GCC49_ARM64_O0_JPN.s}

\subsubsection{MIPS}

\lstinputlisting[caption=\Optimizing GCC 4.4.5 (IDA),style=customasmMIPS]{patterns/07_jcc/abs/MIPS_O3_IDA_JPN.lst}

\myindex{MIPS!\Instructions!BLTZ}
ここでは\INS{BLTZ}(\q{Branch if Less Than Zero})という新しい命令があります。
\myindex{MIPS!\Instructions!SUBU}
\myindex{MIPS!\Pseudoinstructions!NEGU}

\INS{NEGU}擬似命令もあります。これはゼロからの減算だけです。 
\INS{SUBU}と\INS{NEGU}の両方の \q{U}接尾辞は、整数オーバーフローの場合に発生する例外がないことを意味します。

\subsubsection{Branchless version?}

このコードを分岐がないバージョンにすることもできます。 これについては、後述の\myref{chap:branchless_abs}を参照してください。
