\section{Условные переходы}
\label{sec:Jcc}
\myindex{\CLanguageElements!if}

% sections
\subsection{\RU{Простой пример}\EN{Simple example}\DEph{}
\FR{Exemple simple}\ITA{Esempio semplice}
}

\lstinputlisting[style=customc]{patterns/07_jcc/simple/ex.c}

% subsections
\subsection{x86}

\subsubsection{x86 + MSVC}

\RU{Имеем в итоге функцию \TT{f\_signed()}:}\EN{Here is how the \TT{f\_signed()} function looks like:}

\lstinputlisting[caption=\NonOptimizing MSVC 2010]{patterns/07_jcc/simple/signed_MSVC.asm}

\index{x86!\Instructions!JLE}
\RU{Первая инструкция \JLE значит}
\EN{The first instruction, \JLE, stands for} \IT{Jump if Less or Equal}. 
\RU{Если второй операнд больше первого или 
равен ему, произойдет переход туда, где будет следующая проверка.}
\EN{In other words, if the second operand is 
larger or equal to the first one, the control flow will be passed to the specified in the instruction address or label.}
\RU{А если это условие не срабатывает (то есть второй операнд меньше первого), то перехода не будет, 
и сработает первый \printf.}
\EN{If this condition does not trigger because the second operand is smaller than the first one, the control flow would not be altered and the first \printf would be executed.}
\index{x86!\Instructions!JNE}
\RU{Вторая проверка это}\EN{The second check is} \JNE: \IT{Jump if Not Equal}.
\RU{Переход не произойдет, если операнды равны}\EN{The control flow will not change if the operands are 
equal}.

\index{x86!\Instructions!JGE}
\RU{Третья проверка}\EN{The third check is} \JGE: \IT{Jump if Greater or Equal}\EMDASH{}\RU{переход 
если первый операнд больше второго или равен ему}\EN{jump if the first operand is larger than 
the second or if they are equal}.
\RU{Кстати, если все три условных перехода сработают, ни один \printf не вызовется. 
Но без внешнего вмешательства это невозможно.}
\EN{So, if all three conditional jumps are triggered, none of the \printf calls would be executed whatsoever. 
This is impossible without special intervention.}

\EN{Now let's take a look at the \TT{f\_unsigned()} function.}
\EN{The}\RU{Функция} \TT{f\_unsigned()} \RU{точно такая же, за тем исключением, что используются инструкции 
\JBE и \JAE вместо \JLE и \JGE:}
\EN{function is the same as \TT{f\_signed()}, with the exception that the \JBE and \JAE instructions
are used instead of \JLE and \JGE, as follows:}

\lstinputlisting[caption=GCC]{patterns/07_jcc/simple/unsigned_MSVC.asm}

\index{x86!\Instructions!JBE}
\index{x86!\Instructions!JAE}
\RU{Здесь всё то же самое, только инструкции условных переходов немного другие:}
\EN{As already mentioned, the branch instructions are different:}
\JBE\EMDASH{}\IT{Jump if Below or Equal} \AndENRU \JAE\EMDASH{}\IT{Jump if Above or Equal}.
\RU{Эти инструкции}\EN{These instructions} (\JA/\JAE/\JB/\JBE) 
\RU{отличаются от}\EN{differ from} \JG/\JGE/\JL/\JLE \RU{тем, что работают с беззнаковыми переменными.}
\EN{in the fact that they work with unsigned numbers.}

\index{x86!\Instructions!JA}
\index{x86!\Instructions!JB}
\index{x86!\Instructions!JG}
\index{x86!\Instructions!JL}
\index{Signed numbers}
\RU{Отступление: смотрите также секцию о представлении знака в числах}
\EN{See also the section about signed number representations}~(\myref{sec:signednumbers}).
\RU{Таким образом, увидев где используется \JG/\JL вместо \JA/\JB и наоборот, 
можно сказать почти уверенно насчет того, 
является ли тип переменной знаковым (signed) или беззнаковым (unsigned).}
\EN{That is why if we see \JG/\JL in use instead of \JA/\JB or vice-versa, 
we can be almost sure that the variables are signed or unsigned, respectively.}

\RU{Далее функция \main, где ничего нового для нас нет:}
\EN{Here is also the \main function, where there is nothing much new to us:}

\lstinputlisting[caption=\main]{patterns/07_jcc/simple/main_MSVC.asm}

\ifdefined\IncludeOlly
\clearpage
\subsubsection{x86 + MSVC + \olly}
\index{\olly}
\index{x86!\Registers!\Flags}

\RU{Если попробовать этот пример в \olly, можно увидеть, как выставляются флаги}\EN{We
can see how flags are set by running this example in \olly}.
\RU{Начнем с функции}\EN{Let's begin with} \TT{f\_unsigned()}\RU{, которая работает с беззнаковыми числами.}
\EN{, which works with unsigned numbers.}
\RU{В целом в каждой функции \CMP исполняется три раза, но для одних и тех же аргументов, 
так что флаги все три раза будут одинаковы.}
\EN{\CMP is executed thrice here, but for the same arguments, 
so the flags are the same each time.}

\RU{Результат первого сравнения}\EN{Result of the first comparison}:

\begin{figure}[H]
\centering
\includegraphics[scale=\FigScale]{patterns/07_jcc/simple/olly_unsigned1.png}
\caption{\olly: \TT{f\_unsigned()}: \RU{первый условный переход}\EN{first conditional jump}}
\label{fig:jcc_olly_unsigned_1}
\end{figure}

\RU{Итак, флаги}\EN{So, the flags are}: C=1, P=1, A=1, Z=0, S=1, T=0, D=0, O=0.
\RU{Для краткости, в \olly флаги называются только одной буквой.}
\EN{They are named with one character for brevity in \olly.}

\olly \RU{подсказывает, что первый переход}\EN{gives a hint that the} (\JBE) 
\RU{сейчас сработает}\EN{jump is to be triggered now}.
\RU{Действительно, если заглянуть в}\EN{Indeed, if we take a look into} \cite{Intel}, 
\RU{прочитаем там, что}
\EN{we can read there that} \JBE \RU{срабатывает в случаях если}\EN{is triggering if} 
CF=1 \OrENRU ZF=1.
\RU{Условие здесь выполняется, так что переход срабатывает}\EN{The condition is true here, so the jump is triggered}.

\clearpage
\RU{Следующий переход}\EN{The next conditional jump}:

\begin{figure}[H]
\centering
\includegraphics[scale=\FigScale]{patterns/07_jcc/simple/olly_unsigned2.png}
\caption{\olly: \TT{f\_unsigned()}: \RU{второй условный переход}\EN{second conditional jump}}
\label{fig:jcc_olly_unsigned_2}
\end{figure}

\olly \RU{подсказывает, что}\EN{gives a hint that} \JNZ \RU{сработает}\EN{is to be triggered now}.
\RU{Действительно}\EN{Indeed}, \JNZ \RU{срабатывает если}\EN{triggering if} ZF=0 (zero flag).

\clearpage
\RU{Третий переход,}\EN{The third conditional jump,} \JNB:

\begin{figure}[H]
\centering
\includegraphics[scale=\FigScale]{patterns/07_jcc/simple/olly_unsigned3.png}
\caption{\olly: \TT{f\_unsigned()}: \RU{третий условный переход}\EN{third conditional jump}}
\label{fig:jcc_olly_unsigned_3}
\end{figure}

\RU{В}\EN{In} \cite{Intel} \RU{мы можем найти, что}\EN{we can see that} \JNB \RU{срабатывает если}
\EN{triggers if} CF=0 (carry flag).
\RU{В нашем случае это не так, переход не срабатывает, и исполняется третий по счету}
\EN{That is not true in our case, so the third} \printf\EN{ will execute}.


\clearpage
\RU{Теперь можно попробовать в \olly функцию}\EN{Now let's review the} \TT{f\_signed()}%
\RU{, работающую с знаковыми величинами}\EN{function, which works with signed values, in \olly}.

\RU{Флаги выставляются точно так же}\EN{Flags are set in the same way}: 
C=1, P=1, A=1, Z=0, S=1, T=0, D=0, O=0.

\RU{Первый переход}\EN{The first conditional jump} \JLE \RU{сработает}\EN{is to be triggered}:

\begin{figure}[H]
\centering
\includegraphics[scale=\FigScale]{patterns/07_jcc/simple/olly_signed1.png}
\caption{\olly: \TT{f\_signed()}: \RU{первый условный переход}\EN{first conditional jump}}
\label{fig:jcc_olly_signed_1}
\end{figure}

\RU{В}\EN{In} \cite{Intel} \RU{мы можем прочитать, что эта инструкция срабатывает если}\EN{we find
that this instruction is triggered if} 
ZF=1 \OrENRU SF$\neq$OF.
\RU{В нашем случае }SF$\neq$OF\RU{, так что переход срабатывает}\EN{ in our case, so the jump triggers}.

\clearpage
\RU{Второй переход}\EN{The second} \JNZ \RU{сработает}\EN{conditional jump triggering}: 
\RU{он срабатывает если}\EN{if} ZF=0 (zero flag):


\begin{figure}[H]
\centering
\includegraphics[scale=\FigScale]{patterns/07_jcc/simple/olly_signed2.png}
\caption{\olly: \TT{f\_signed()}: \RU{второй условный переход}\EN{second conditional jump}}
\label{fig:jcc_olly_signed_2}
\end{figure}

\clearpage
\RU{Третий переход}\EN{The third conditional jump} \JGE 
\RU{не сработает, потому что он срабатывает, только если}\EN{will not trigger because it would only do so if} SF=OF, 
\RU{что в нашем случае не так}\EN{and that is not true in our case}:

\begin{figure}[H]
\centering
\includegraphics[scale=\FigScale]{patterns/07_jcc/simple/olly_signed3.png}
\caption{\olly: \TT{f\_signed()}: \RU{третий условный переход}\EN{third conditional jump}}
\label{fig:jcc_olly_signed_3}
\end{figure}

\fi

\clearpage
\subsubsection{x86 + MSVC + Hiew}
\index{Hiew}

\RU{Можем попробовать модифицировать исполняемый файл так,}\EN{We can try to patch the executable file in a way} 
\RU{чтобы функция}\EN{that the} \TT{f\_unsigned()} \RU{всегда показывала}\EN{function would always print} \q{a==b}, 
\RU{при любых входящих значениях}\EN{no matter the input values}.
\RU{Вот как она выглядит в}\EN{Here is how it looks in} Hiew:

\begin{figure}[H]
\centering
\includegraphics[scale=\FigScale]{patterns/07_jcc/simple/hiew_unsigned1.png}
\caption{Hiew: \RU{функция }\TT{f\_unsigned()}\EN{ function}}
\label{fig:jcc_hiew_1}
\end{figure}

\RU{Собственно, задач три}\EN{Essentially, we need to accomplish three tasks}:
\begin{itemize}
\item \RU{заставить первый переход срабатывать всегда}\EN{force the first jump to always trigger};
\item \RU{заставить второй переход не срабатывать никогда}\EN{force the second jump to never trigger};
\item \RU{заставить третий переход срабатывать всегда}\EN{force the third jump to always trigger}.
\end{itemize}

\RU{Так мы направим путь исполнения кода (code flow) во второй}\EN{Thus we can direct the code flow
to always pass through the second} \printf,
\RU{и он всегда будет срабатывать и выводить на консоль}\EN{and output} \q{a==b}.

\RU{Для этого нужно изменить три инструкции (или байта)}\EN{Three instructions (or bytes) has to be patched}:

\begin{itemize}
\item \RU{Первый переход теперь будет}\EN{The first jump becomes} \JMP, \RU{но смещение перехода 
(\gls{jump offset}) останется прежним}\EN{but the \gls{jump offset} would remain the same}.

\item \RU{Второй переход может быть и будет срабатывать иногда, но в любом случае он будет совершать переход
только на следующую инструкцию, потому что мы выставляем смещение перехода (\gls{jump offset}) в 0.}
\EN{The second jump might be triggered sometimes, but in any case it will jump to the next
instruction, because, we set the \gls{jump offset} to 0.}
\RU{В этих инструкциях смещение перехода просто прибавляется к адресу следующей инструкции.}
\EN{In these instructions the \gls{jump offset} is added to the address for the next instruction.}
\RU{Когда смещение 0, переход будет на следующую инструкцию.}\EN{So if the offset is 0,
the jump will transfer the control to the next instruction.}

\item \RU{Третий переход конвертируем в \JMP точно так же, как и первый, он будет срабатывать всегда.}
\EN{The third jump we replace with \JMP just as we do with the first one, so it will always trigger.}

\end{itemize}

\clearpage
\RU{Что и делаем}\EN{Here is the modified code}:

\begin{figure}[H]
\centering
\includegraphics[scale=\FigScale]{patterns/07_jcc/simple/hiew_unsigned2.png}
\caption{Hiew: \RU{модифицируем функцию}\EN{let's modify the} \TT{f\_unsigned()}\EN{ function}}
\label{fig:jcc_hiew_2}
\end{figure}

\RU{Если забыть про какой-то из переходов, то тогда будет срабатывать несколько вызовов \printf, 
а нам ведь нужно чтобы исполнялся только один.}
\EN{If we miss to change any of these jumps, then several \printf calls may execute, while we want to execute only one.}

\ifdefined\IncludeGCC
\subsubsection{\NonOptimizing GCC}

\index{puts() \RU{вместо}\EN{instead of} printf()}
\NonOptimizing GCC 4.4.1 \RU{производит почти такой же код, за исключением}
\EN{produces almost the same code, but with} \puts~(\myref{puts}) \RU{вместо}\EN{instead of} \printf.

\subsubsection{\Optimizing GCC}

\RU{Наблюдательный читатель может спросить, зачем исполнять \CMP так много раз,
если флаги всегда одни и те же}\EN{An observant reader may ask, why execute \CMP several times, 
if the flags has the same values after each execution}?
\RU{По видимому, оптимизирующий MSVC не может этого делать, но GCC 4.8.1 делает больше оптимизаций:}
\EN{Perhaps optimizing MSVC can not do this, but optimizing GCC 4.8.1 can go deeper:}

\lstinputlisting[caption=GCC 4.8.1 f\_signed()]{patterns/07_jcc/simple/GCC_O3_signed.asm}

% should be here instead of 'switch' section?
\RU{Мы здесь также видим}\EN{We also see} \TT{JMP puts} \RU{вместо}\EN{here instead of} \TT{CALL puts / RETN}.
\RU{Этот прием описан немного позже}%
\EN{This kind of trick will have explained later}: \myref{JMP_instead_of_RET}.

\RU{Нужно сказать, что x86-код такого типа редок}\EN{This type of x86 code 
is somewhat rare}.
MSVC 2012\RU{, как видно, не может генерировать подобное}\EN{ as it seems, can't generate such code}.
\RU{С другой стороны, программисты на ассемблере прекрасно осведомлены о том, что инструкции}\EN{On the other hand, assembly language programmers are fully aware of the fact that} \TT{Jcc} \RU{можно располагать последовательно.}
\EN{instructions can be stacked.}
\RU{Так что если вы видите это где-то, имеется немалая вероятность, что этот фрагмент кода был написан вручную.}
\EN{So if you see such stacking somewhere, it is highly probable that the code was hand-written.}

\EN{The}\RU{Функция} \TT{f\_unsigned()} \RU{получилась не настолько эстетически короткой}\EN{function is not that 
\ae{}sthetically short}:

\lstinputlisting[caption=GCC 4.8.1 f\_unsigned()]{patterns/07_jcc/simple/GCC_O3_unsigned.asm.\LANG}

\RU{Тем не менее, здесь 2 инструкции \TT{CMP} вместо трех.}
\EN{Nevertheless, there are two \TT{CMP} instructions instead of three.}
\RU{Так что, алгоритмы оптимизации GCC 4.8.1, наверное, ещё пока не идеальны.}
\EN{So optimization algorithms of GCC 4.8.1 are probably not perfect yet.} 
\fi

\subsubsection{ARM}

% subsubsections here
\EN{\subsubsection{ARM}

\myparagraph{\OptimizingXcodeIV (\ARMMode)}

\lstinputlisting[caption=\OptimizingXcodeIV (\ARMMode),style=customasmARM]{patterns/12_FPU/3_comparison/ARM/Xcode_ARM_EN.lst}

\myindex{ARM!\Registers!APSR}
\myindex{ARM!\Registers!FPSCR}
A very simple case.
The input values are placed into the \GTT{D17} and \GTT{D16} registers and then compared using the \INS{VCMPE} instruction.

Just like in the x86 coprocessor, the ARM coprocessor has its own status and flags register (\ac{FPSCR}),
since there is a necessity to store coprocessor-specific flags.
% TODO -> расписать регистр по битам
\myindex{ARM!\Instructions!VMRS}
And just like in x86, there are no conditional jump instruction in ARM, 
that can check bits in the status register of the coprocessor. 
So there is \INS{VMRS}, which copies 4 bits (N, Z, C, V) from the coprocessor status word into bits of the \IT{general} status register (\ac{APSR}).

\myindex{ARM!\Instructions!VMOVGT}
\INS{VMOVGT} is the analog of the \INS{MOVGT}, 
instruction for D-registers, it executes if one operand is greater than the other while comparing (\IT{GT---Greater Than}). 

If it gets executed, the value of $a$ is to be written into \GTT{D16} (that is currently stored in \GTT{D17}).
Otherwise the value of $b$ stays in the \GTT{D16} register.

\myindex{ARM!\Instructions!VMOV}

The penultimate instruction \INS{VMOV} prepares the value in the \GTT{D16} register for returning it via the \Reg{0} and \Reg{1}
register pair.

\myparagraph{\OptimizingXcodeIV (\ThumbTwoMode)}

\begin{lstlisting}[caption=\OptimizingXcodeIV (\ThumbTwoMode),style=customasmARM]
VMOV            D16, R2, R3 ; b
VMOV            D17, R0, R1 ; a
VCMPE.F64       D17, D16
VMRS            APSR_nzcv, FPSCR
IT GT 
VMOVGT.F64      D16, D17
VMOV            R0, R1, D16
BX              LR
\end{lstlisting}

Almost the same as in the previous example, however slightly different.
As we already know, many instructions in ARM mode can be supplemented by condition predicate.
But there is no such thing in Thumb mode. 
There is no space in the 16-bit instructions for 4 more bits in which conditions can be encoded.

\myindex{ARM!\ThumbTwoMode}

However, Thumb-2 was extended to make it possible to specify predicates to old Thumb instructions.
Here, in the \IDA-generated listing, we see the \INS{VMOVGT} instruction, as in previous example.

In fact, the usual \INS{VMOV} is encoded there, but \IDA adds the \GTT{-GT} suffix to it, 
since there is a \INS{IT GT} instruction placed right before it.

\label{ARM_Thumb_IT}
\myindex{ARM!\Instructions!IT}
\myindex{ARM!if-then block}
The \INS{IT} instruction defines a so-called \IT{if-then block}. 

After the instruction it is possible to place up to 4 instructions, 
each of them has a predicate suffix.
In our example, \INS{IT GT} implies that the next instruction is to be executed, if the \IT{GT} (\IT{Greater Than}) condition is true.

\myindex{Angry Birds}
Here is a more complex code fragment, by the way, from Angry Birds (for iOS):

\begin{lstlisting}[caption=Angry Birds Classic,style=customasmARM]
...
ITE NE
VMOVNE          R2, R3, D16
VMOVEQ          R2, R3, D17
BLX             _objc_msgSend ; not suffixed
...
\end{lstlisting}

\INS{ITE} stands for \IT{if-then-else} 

and it encodes suffixes for the next two instructions.

The first instruction executes if the condition encoded in \INS{ITE} (\IT{NE, not equal}) is true at, and the second---if the condition is not true.
(The inverse condition of \GTT{NE} is \GTT{EQ} (\IT{equal})).

The instruction followed after the second \INS{VMOV} (or \INS{VMOVEQ}) is a normal one, not suffixed (\INS{BLX}).

\myindex{Angry Birds}
One more that's slightly harder, which is also from Angry Birds:

\begin{lstlisting}[caption=Angry Birds Classic,style=customasmARM]
...
ITTTT EQ
MOVEQ           R0, R4
ADDEQ           SP, SP, #0x20
POPEQ.W         {R8,R10}
POPEQ           {R4-R7,PC}
BLX             ___stack_chk_fail ; not suffixed
...
\end{lstlisting}

Four \q{T} symbols in the instruction mnemonic mean that the four subsequent instructions are to be executed if the condition is true.

That's why \IDA adds the \GTT{-EQ} suffix to each one of them. 

And if there was, for example, \INS{ITEEE EQ} (\IT{if-then-else-else-else}), 
then the suffixes would have been set as follows:

\begin{lstlisting}
-EQ
-NE
-NE
-NE
\end{lstlisting}

\myindex{Angry Birds}
Another fragment from Angry Birds:

\begin{lstlisting}[caption=Angry Birds Classic,style=customasmARM]
...
CMP.W           R0, #0xFFFFFFFF
ITTE LE
SUBLE.W         R10, R0, #1
NEGLE           R0, R0
MOVGT           R10, R0
MOVS            R6, #0         ; not suffixed
CBZ             R0, loc_1E7E32 ; not suffixed
...
\end{lstlisting}

\INS{ITTE} (\IT{if-then-then-else}) 

implies that the 1st and 2nd instructions are to be executed if the \GTT{LE} (\IT{Less or Equal})
condition is true, and the 3rd---if the inverse condition (\GTT{GT}---\IT{Greater Than}) 
is true.

Compilers usually don't generate all possible combinations.
\myindex{Angry Birds}

For example, in the mentioned Angry Birds game (\IT{classic} version for iOS)
only these variants of the \INS{IT} instruction are used: 
\INS{IT}, \INS{ITE}, \INS{ITT}, \INS{ITTE}, \INS{ITTT}, \INS{ITTTT}.
\myindex{\GrepUsage}
How to learn this?
In \IDA It is possible to produce listing files, so it was created with an option to show 4 bytes for each opcode.
Then, knowing the high part of the 16-bit opcode (\INS{IT} is \GTT{0xBF}), we do the following using \GTT{grep}:

\begin{lstlisting}
cat AngryBirdsClassic.lst | grep " BF" | grep "IT" > results.lst
\end{lstlisting}

\myindex{ARM!\ThumbTwoMode}

By the way, if you program in ARM assembly language manually for Thumb-2 mode, 
and you add conditional suffixes,
the assembler will add the \INS{IT} instructions automatically with the required flags where it is necessary.

\myparagraph{\NonOptimizingXcodeIV (\ARMMode)}

\begin{lstlisting}[caption=\NonOptimizingXcodeIV (\ARMMode),style=customasmARM]
b               = -0x20
a               = -0x18
val_to_return   = -0x10
saved_R7        = -4

                STR             R7, [SP,#saved_R7]!
                MOV             R7, SP
                SUB             SP, SP, #0x1C
                BIC             SP, SP, #7
                VMOV            D16, R2, R3
                VMOV            D17, R0, R1
                VSTR            D17, [SP,#0x20+a]
                VSTR            D16, [SP,#0x20+b]
                VLDR            D16, [SP,#0x20+a]
                VLDR            D17, [SP,#0x20+b]
                VCMPE.F64       D16, D17
                VMRS            APSR_nzcv, FPSCR
                BLE             loc_2E08
                VLDR            D16, [SP,#0x20+a]
                VSTR            D16, [SP,#0x20+val_to_return]
                B               loc_2E10

loc_2E08
                VLDR            D16, [SP,#0x20+b]
                VSTR            D16, [SP,#0x20+val_to_return]

loc_2E10
                VLDR            D16, [SP,#0x20+val_to_return]
                VMOV            R0, R1, D16
                MOV             SP, R7
                LDR             R7, [SP+0x20+b],#4
                BX              LR
\end{lstlisting}

Almost the same as we already saw, 
but there is too much redundant code because the $a$ and $b$ variables are stored in the local stack, as well
as the return value.

\myparagraph{\OptimizingKeilVI (\ThumbMode)}

\begin{lstlisting}[caption=\OptimizingKeilVI (\ThumbMode),style=customasmARM]
                PUSH    {R3-R7,LR}
                MOVS    R4, R2
                MOVS    R5, R3
                MOVS    R6, R0
                MOVS    R7, R1
                BL      __aeabi_cdrcmple
                BCS     loc_1C0
                MOVS    R0, R6
                MOVS    R1, R7
                POP     {R3-R7,PC}

loc_1C0
                MOVS    R0, R4
                MOVS    R1, R5
                POP     {R3-R7,PC}
\end{lstlisting}


Keil doesn't generate FPU-instructions since it cannot rely on them being
supported on the target CPU, and it cannot be done by straightforward bitwise comparing.
%TODO1: why?
So it calls an external library function to do the comparison: \GTT{\_\_aeabi\_cdrcmple}. 
\myindex{ARM!\Instructions!BCS}

N.B. The result of the comparison is to be left in the flags by this function, so the following
\INS{BCS} (\IT{Carry set---Greater than or equal})
instruction can work without any additional code.

}
\RU{\myparagraph{32-битный ARM}
\label{subsec:jcc_ARM}

\mysubparagraph{\OptimizingKeilVI (\ARMMode)}

\lstinputlisting[caption=\OptimizingKeilVI (\ARMMode),style=customasmARM]{patterns/07_jcc/simple/ARM/ARM_O3_signed.asm}

\myindex{ARM!Condition codes}
% FIXME \ref -> which instructions?
Многие инструкции в режиме ARM могут быть исполнены только при некоторых выставленных флагах.

Это нередко используется для сравнения чисел.

\myindex{ARM!\Instructions!ADD}
\myindex{ARM!\Instructions!ADDAL}
К примеру, инструкция \ADD на самом деле называется \TT{ADDAL} внутри, \TT{AL} означает \IT{Always}, то есть, исполнять всегда.
Предикаты кодируются в 4-х старших битах инструкции 32-битных ARM-инструкций (\IT{condition field}).
\myindex{ARM!\Instructions!B}
Инструкция безусловного перехода \TT{B} на самом деле условная и кодируется так же, 
как и прочие инструкции условных переходов, но имеет \TT{AL} в \IT{condition field}, 
то есть исполняется всегда (\IT{execute ALways}), игнорируя флаги.

\myindex{ARM!\Instructions!ADR}
\myindex{ARM!\Instructions!ADRcc}
\myindex{ARM!\Instructions!CMP}
Инструкция \TT{ADRGT} работает так же, как и \TT{ADR}, но исполняется только в случае,
если предыдущая инструкция \CMP,
сравнивая два числа, обнаруживает, что одно из них больше второго (\IT{Greater Than}).

\myindex{ARM!\Instructions!BL}
\myindex{ARM!\Instructions!BLcc}
Следующая инструкция \TT{BLGT} ведет себя так же, как и \TT{BL} и сработает, только если 
результат сравнения ``больше чем'' (\IT{Greater Than}).
\TT{ADRGT} записывает в \Reg{0} указатель на строку \TT{a>b\textbackslash{}n}, а \TT{BLGT} вызывает \printf.
Следовательно, эти инструкции с суффиксом \TT{-GT} исполнятся только в том случае, если значение
в \Reg{0} (там $a$) было больше, чем значение в \Reg{4} (там $b$).

\myindex{ARM!\Instructions!ADRcc}
\myindex{ARM!\Instructions!BLcc}
Далее мы увидим инструкции \TT{ADREQ} и \TT{BLEQ}.
Они работают так же, как и \TT{ADR} и \TT{BL}, но исполнятся только если значения при последнем сравнении были равны.
Перед ними расположен ещё один \CMP, потому что вызов \printf мог испортить состояние флагов.

\myindex{ARM!\Instructions!LDMccFD}
\myindex{ARM!\Instructions!LDMFD}
Далее мы увидим \TT{LDMGEFD}. Эта инструкция работает так же, как и \TT{LDMFD}\footnote{\ac{LDMFD}}, 
но сработает только если в результате сравнения одно из значений было больше или равно второму (\IT{Greater or Equal}).
Смысл инструкции \TT{LDMGEFD SP!, \{R4-R6,PC\}} 
в том, что это как бы эпилог функции, но он сработает только если $a>=b$, только тогда работа 
функции закончится.

\myindex{Function epilogue}
Но если это не так, то есть $a<b$, то исполнение дойдет до следующей инструкции 
\TT{LDMFD SP!, \{R4-R6,LR\}}. Это ещё один эпилог функции. Эта инструкция восстанавливает состояние регистров
\TT{R4-R6}, но и \ac{LR} вместо \ac{PC}, таким образом, пока что, не делая возврата из функции.

Последние две инструкции вызывают \printf 
со строкой <<a<b\textbackslash{}n>> в качестве единственного аргумента.
Безусловный переход на \printf вместо возврата из функции мы уже рассматривали в секции
 <<\PrintfSeveralArgumentsSectionName>>~(\myref{ARM_B_to_printf}).

\myindex{ARM!\Instructions!ADRcc}
\myindex{ARM!\Instructions!BLcc}
\myindex{ARM!\Instructions!LDMccFD}
Функция \TT{f\_unsigned} точно такая же, но там используются инструкции \TT{ADRHI}, \TT{BLHI}, и \TT{LDMCSFD}. Эти предикаты
(\IT{HI = Unsigned higher, CS = Carry Set (greater than or equal)})
аналогичны рассмотренным, но служат для работы с беззнаковыми значениями.

В функции \main ничего нового для нас нет:

\lstinputlisting[caption=\main,style=customasmARM]{patterns/07_jcc/simple/ARM/ARM_O3_main.asm}

Так, в режиме ARM можно обойтись без условных переходов.

\myindex{Конвейер RISC}
Почему это хорошо? Читайте здесь: \myref{branch_predictors}.

\myindex{x86!\Instructions!CMOVcc}
В x86 нет аналогичной возможности, если не считать инструкцию \TT{CMOVcc}, это то же что и \MOV, 
но она срабатывает только при определенных выставленных флагах, обычно выставленных при помощи \CMP во время сравнения.

\mysubparagraph{\OptimizingKeilVI (\ThumbMode)}

\lstinputlisting[caption=\OptimizingKeilVI (\ThumbMode),style=customasmARM]{patterns/07_jcc/simple/ARM/ARM_thumb_signed.asm}

\myindex{ARM!\Instructions!BLE}
\myindex{ARM!\Instructions!BNE}
\myindex{ARM!\Instructions!BGE}
\myindex{ARM!\Instructions!BLS}
\myindex{ARM!\Instructions!BCS}
\myindex{ARM!\Instructions!B}
\myindex{ARM!\ThumbMode}
В режиме Thumb только инструкции \TT{B} могут быть дополнены условием исполнения (\IT{condition code}), 
так что код для режима Thumb выглядит привычнее.

\TT{BLE} это обычный переход с условием \IT{Less than or Equal}, 
\TT{BNE} --- \IT{Not Equal}, 
\TT{BGE} --- \IT{Greater than or Equal}.

Функция \TT{f\_unsigned} точно такая же, но для работы с беззнаковыми величинами 
там используются инструкции \TT{BLS} 
(\IT{Unsigned lower or same}) и \TT{BCS} (\IT{Carry Set (Greater than or equal)}).
}
\DE{\myparagraph{32-bit ARM}
\label{subsec:jcc_ARM}

\mysubparagraph{\OptimizingKeilVI (\ARMMode)}

\lstinputlisting[caption=\OptimizingKeilVI (\ARMMode),style=customasmARM]{patterns/07_jcc/simple/ARM/ARM_O3_signed.asm}

\myindex{ARM!Condition codes}
% FIXME \ref -> which instructions?
Viele Befehle im ARM mode können nur ausgeführt werden, wenn spezielle Flags gesetzt sind.
Dies ist beispielsweise oft beim Vergleich von Zahlen der Fall.

\myindex{ARM!\Instructions!ADD}
\myindex{ARM!\Instructions!ADDAL}
Der \ADD Befehl zum Beispiel heißt hier intern \TT{ADDAL}, wobei \TT{AL} für \IT{Always} (dt. immer) steht, d.h. er wird
immer ausgeführt.
Die Prädikate werden in den 4 höchstwertigsten Bits des 32-Bit-ARM-Befehls kodiert, dem \IT{condition field}.

\myindex{ARM!\Instructions!B}
Der Befehl \TT{B} für einen unbedingten Sprung ist tatsächlich doch bedingt und genau wie jeder andere bedingte Sprung
kodiert, nut dass er \TT{AL} im \IT{condition field} hat und dadurch die Flags ignoriert und immer ausgeführt wird.

\myindex{ARM!\Instructions!ADR}
\myindex{ARM!\Instructions!ADRcc}
\myindex{ARM!\Instructions!CMP}
Der Befehl \TT{ADRGT} arbeitet wie \TT{ADR}, wird aber nur ausgeführt, wenn das vorangehende \CMP ergeben hat, dass eine
der beiden Eingabezahlen größer war als die andere. 

\myindex{ARM!\Instructions!BL}
\myindex{ARM!\Instructions!BLcc}
% ToBeUpdated
Der folgende \TT{BLGT} Befehl verhält sich genau wie \TT{BL} und wird nur dann ausgeführt, wenn das Ergebnis des
Vergleichs das gleiche war (d.h. größer als).
\TT{ADRGT} schreibt einen Pointer auf den String \TT{a>b\textbackslash{}n} nach \Reg{0} und \TT{BLGT} ruft \printf auf.
Das heißt, Befehl mit dem Suffix \TT{-GT} werden nur ausgeführt, wenn der Wert in \Reg{0} (das ist $a$) größer ist als
der Wert in \Reg{4} (das ist $b$).

\myindex{ARM!\Instructions!ADRcc}
\myindex{ARM!\Instructions!BLcc}
Im Folgenden finden wir die Befehle \TT{ADREQ} und \TT{BLEQ}.
Sie verhalten sich wie \TT{ADR} und \TT{BL}, werden aber nur ausgeführt, wenn die beiden Operanden des letzten
Vergleichs gleich waren.
Ein weiteres \CMP befindet sich davor (denn die Ausführung von \printf könnte die Flags verändert haben).

\myindex{ARM!\Instructions!LDMccFD}
\myindex{ARM!\Instructions!LDMFD}
Dann finden wir \TT{LDMGEFD}; dieser Befehl arbeitet genau wie \TT{LDMFD}\footnote{\ac{LDMFD}}, wird aber nur
ausgeführt, wenn einer der Werte größer gleich dem anderen ist. 
Der Befehl \TT{LDMGEFD SP!, \{R4-R6,PC\}} fungiert als Funktionsepilog, wird aber nur ausgeführt, ewnn $a>=b$ und nur
dann wird die Funktionsausführung beendet.
\myindex{Function epilogue}
Wenn aber diese Bedingung nicht erfüllt ist, d.h. $a<b$, wird der Control Flow zum nächsten \\
\TT{\q{LDMFD SP!, \{R4-R6,LR\}}} springen, der ebenfalls einen Funktionsepilog darstellt. Dieser Befehl stellt nicht nur
den Zustand der \TT{R4-R6} Register wieder her, sondern auch \ac{LR} anstatt \ac{PC}, dadurch gibt er nichts aus der
Funktion zurück.
Die letzten beiden Befehle rufen \printf mit dem String <<a<b\textbackslash{}n>> als einzigem Argument auf.
Wir haben bereits einen unbedingten Sprung zur Funktion \printf anstelle einer Funktionsrückgabe im Abschnitt
<<\PrintfSeveralArgumentsSectionName>>~(\myref{ARM_B_to_printf}) untersucht.

\myindex{ARM!\Instructions!ADRcc}
\myindex{ARM!\Instructions!BLcc}
\myindex{ARM!\Instructions!LDMccFD}
\TT{f\_unsigned} ist ähnlich, nur die Befehle \TT{ADRHI}, \TT{BLHI} und \TT{LDMCSFD} werden hier verwendet.
Deren Prädikaten (\IT{HI = Unsigned higher, CS = Carry Set (greater than or equal)}) sind analog zu den eben
betrachteten, nur eben für vorzeichenlose Werte. 

In der Funktion \main finden wir nicht viel Neues:

\lstinputlisting[caption=\main,style=customasmARM]{patterns/07_jcc/simple/ARM/ARM_O3_main.asm}
Auf diese Weise kann man bedingte Sprünge im ARM mode entfernen.


\myindex{RISC pipeline}
Für eine Begründung warum dies vorteilhaft ist, siehe: \myref{branch_predictors}.

\myindex{x86!\Instructions!CMOVcc}
In x86 gibt es kein solches Feature, außer dem \TT{CMOVcc} Befehl, der genau wie \MOV funktioniert, aber nur ausgeführt
wird, wenn spezielle Flags - normalerweise durch \CMP - gesetzt sind.


\mysubparagraph{\OptimizingKeilVI (\ThumbMode)}

\lstinputlisting[caption=\OptimizingKeilVI (\ThumbMode),style=customasmARM]{patterns/07_jcc/simple/ARM/ARM_thumb_signed.asm}

\myindex{ARM!\Instructions!BLE}
\myindex{ARM!\Instructions!BNE}
\myindex{ARM!\Instructions!BGE}
\myindex{ARM!\Instructions!BLS}
\myindex{ARM!\Instructions!BCS}
\myindex{ARM!\Instructions!B}
\myindex{ARM!\ThumbMode}
Nur der \TT{B} Befehl im Thumb mode kann mit condition codes versehen werden, sodass der Thumb Code gewöhnlicher
aussieht.


\TT{BLE} ist ein normaler bedingter Sprung \IT{Less than or Equal}, 
\TT{BNE}---\IT{Not Equal}, 
\TT{BGE}---\IT{Greater than or Equal}.

\TT{f\_unsigned} ist ähnlich, nur dass andere Befehle verwendet werden, wenn mit vorzeichenlosen Werten umgegangen wird:
\TT{BLS} (\IT{Unsigned lower or same}) und \TT{BCS} (\IT{Carry Set (Greater than or equal)}).
}
\FR{\myparagraph{ARM 32-bit}
\label{subsec:jcc_ARM}

\mysubparagraph{\OptimizingKeilVI (\ARMMode)}

\lstinputlisting[caption=\OptimizingKeilVI (\ARMMode),style=customasmARM]{patterns/07_jcc/simple/ARM/ARM_O3_signed.asm}

\myindex{ARM!Condition codes}
% FIXME \ref -> which instructions?

Beaucoup d'instructions en mode ARM ne peuvent être exécutées que lorsque certains
flags sont mis.
E.g, ceci est souvent utilisé lorsque l'on compare les nombres

\myindex{ARM!\Instructions!ADD}
\myindex{ARM!\Instructions!ADDAL}

Par exemple, l'instruction \ADD est en fait appelée \TT{ADDAL} en interne, oú \TT{AL}
signifie \IT{Always}, i.e., toujours exécuter.
Les préficats sont encodés dans les 4 bits du haut des instructions ARM 32-bit. (\IT{condition field}).
\myindex{ARM!\Instructions!B}
L'instruction de saut inconditionnel \TT{B} est en fait conditionnelle et encodée
comme toutes les autres instructions de saut conditionnel, mais a \TT{AL} dans le
\IT{champ de condition}, et \IT{s'exécute toujours} (ALways), ignorants les flags.

\myindex{ARM!\Instructions!ADR}
\myindex{ARM!\Instructions!ADRcc}
\myindex{ARM!\Instructions!CMP}

L'instruction \TT{ADRGT} fonctionne comme \TT{ADR} mais ne s'exécute que dans le
cas oú l'instruction \CMP précédente a trouvé un des nombres plus grand que l'autre,
en comparant les deux (\IT{Greater Than}).

\myindex{ARM!\Instructions!BL}
\myindex{ARM!\Instructions!BLcc}

L'instruction \TT{BLGT} se comporte exactement comme \TT{BL} et n'est effectuée
que si le résultat de la comparaison était \IT{Greater Than} (plus grand).
\TT{ADRGT} écrit un pointeur sur la chaîne \TT{a>b\textbackslash{}n} dans \Reg{0}
et \TT{BLGT} appelle \printf.
Donc, les instructions suffixées par \TT{-GT} ne sont exécutées que si la valeur
dans \Reg{0} (qui est $a$) est plus grande que la valeur dans \Reg{4} (qui est $b$).

\myindex{ARM!\Instructions!ADRcc}
\myindex{ARM!\Instructions!BLcc}

En avançant, nous voyons les instructions \TT{ADREQ} et \TT{BLEQ}.
Elles se comportent comme \TT{ADR} et \TT{BL}, mais ne sont exécutées que si les
opérandes étaient égales lors de la dernière comparaison.
Un autre \CMP se trouve avant elles (car l'exécution de \printf pourrait avoir
modifiée les flags).

\myindex{ARM!\Instructions!LDMccFD}
\myindex{ARM!\Instructions!LDMFD}

Ensuite nous voyons \TT{LDMGEFD}, cette instruction fonctionne comme \TT{LDMFD}\footnote{\ac{LDMFD}},
mais n'est exécutée que si l'une des valeurs est supérieure ou égale à l'autre
(\IT{Greater or Equal}).
L'instruction \TT{LDMGEFD SP!, \{R4-R6,PC\}} se comporte comme une fonction épilogue,
mais elle ne sera exécutée que si $a>=b$, et seulement lorsque l'exécution de la
fonction se terminera.

\myindex{Function epilogue}

Mais si cette condition n'est pas satisfaite, i.e., $a<b$, alors le flux d'exécution
continue à l'instruction suivante, \TT{\q{LDMFD SP!, \{R4-R6,LR\}}}, qui est aussi
un épilogue de la fonction. Cette instruction ne restaure pas seulement l'état des
registres \TT{R4-R6}, mais aussi \ac{LR} au lieu de \ac{PC}, donc il ne retourne
pas de la fonction.
Les deux dernières instructions appellent \printf avec la chaîne <<a<b\textbackslash{}n>>
comme unique argument.
Nous avons déjà examiné un saut inconditionnel à la fonction \printf au lieu
d'un appel avec retour dans <<\PrintfSeveralArgumentsSectionName>> section~(\myref{ARM_B_to_printf}).

\myindex{ARM!\Instructions!ADRcc}
\myindex{ARM!\Instructions!BLcc}
\myindex{ARM!\Instructions!LDMccFD}
\TT{f\_unsigned} est très similaire, à part les instructions \TT{ADRHI}, \TT{BLHI},
et \TT{LDMCSFD} utilisées ici, ces prédicats (\IT{HI = Unsigned higher, CS = Carry
Set (greater than or equal)}) sont analogues à ceux examinés avant, mais pour des
valeurs non signées.

Il n'y a pas grand chose de nouveau pour nous dans la fonction \main:

\lstinputlisting[caption=\main,style=customasmARM]{patterns/07_jcc/simple/ARM/ARM_O3_main.asm}

C'est ainsi que vous pouvez vous débarrasser des sauts conditionnels en mode ARM.

\myindex{RISC pipeline}
Pourquoi est-ce que c'est si utile? Lire ici: \myref{branch_predictors}.

\myindex{x86!\Instructions!CMOVcc}

Il n'y a pas de telle caractéristique en x86, exceptée l'instruction \TT{CMOVcc},
qui est comme un \MOV, mais effectuée seulement lorsque certains flags sont mis,
en général mis par \CMP.

\mysubparagraph{\OptimizingKeilVI (\ThumbMode)}

\lstinputlisting[caption=\OptimizingKeilVI (\ThumbMode),style=customasmARM]{patterns/07_jcc/simple/ARM/ARM_thumb_signed.asm}

\myindex{ARM!\Instructions!BLE}
\myindex{ARM!\Instructions!BNE}
\myindex{ARM!\Instructions!BGE}
\myindex{ARM!\Instructions!BLS}
\myindex{ARM!\Instructions!BCS}
\myindex{ARM!\Instructions!B}
\myindex{ARM!\ThumbMode}

En mode Thumb, seules les instructions \TT{B} peuvent être complètées par un
\IT{condition codes}, (code de condition) donc le code Thumb paraît plus ordinaire.

\TT{BLE} est un saut conditionnel normal \IT{Less than or Equal} (inférieur ou égal),
\TT{BNE}---\IT{Not Equal} (non égal),
\TT{BGE}---\IT{Greater than or Equal} (supérieur ou égal).

\TT{f\_unsigned} est similaire, seules d'autres instructions sont utilisées
pour travailler avec des valeurs non-signées: \TT{BLS}
(\IT{Unsigned lower or same} non signée, inférieur ou égal) et \TT{BCS} (\IT{Carry
Set (Greater than or equal)} supérieur ou égal).
}
\ITA{\myparagraph{32-bit ARM}
\label{subsec:jcc_ARM}

\mysubparagraph{\OptimizingKeilVI (\ARMMode)}

\lstinputlisting[caption=\OptimizingKeilVI (\ARMMode),style=customasmARM]{patterns/07_jcc/simple/ARM/ARM_O3_signed.asm}

\myindex{ARM!Condition codes}
% FIXME \ref -> which instructions?

In modalità ARM molte istruzioni possono essere eseguite solo quando specifici flag sono settati.
Es. sono spesso usate quando si confrontano numeri.

\myindex{ARM!\Instructions!ADD}
\myindex{ARM!\Instructions!ADDAL}

Ad esempio, l'istruzione \ADD è infatti chiamata internamente is in fact named \TT{ADDAL}, il suffisso \TT{AL} sta per
\IT{Always}, ad indicare che viene eseguita sempre.
I predicati sono codificati nei 4 bit alti dell'istruzione ARM a 32-bit (\IT{condition field}).
\myindex{ARM!\Instructions!B}
L'istruzione \TT{B} per effettuare un salto non condizionale è in realtà condizionale ed è codificata proprio come ogni altro
jump condizionale, ma ha \TT{AL} (\IT{execute ALways}) nel \IT{condition field}, e ciò implica che venga sempre eseguito, ignorando i flag.

\myindex{ARM!\Instructions!ADR}
\myindex{ARM!\Instructions!ADRcc}
\myindex{ARM!\Instructions!CMP}

L'istruzione \TT{ADRGT} funziona come \TT{ADR}, ma viene eseguita soltanto nel caso in cui la precedente istruzione \CMP
trovi uno dei due numeri a confronto più grande dell'altro, (\IT{Greater Than}).

\myindex{ARM!\Instructions!BL}
\myindex{ARM!\Instructions!BLcc}

La successiva istruzione \TT{BLGT} si comporta esattamente come \TT{BL} 
ed il salto viene innescato solo se il risultato del confronto è (\IT{Greater Than}). 
\TT{ADRGT} scrive un putatore alla stringa \TT{a>b\textbackslash{}n} nel registro \Reg{0} e \TT{BLGT} chiama \printf.
Le istruzioni aventi il suffisso \TT{-GT} in questo caso sono quindi eseguite solo se il valore in \Reg{0} (ovvero $a$) è maggiore del valore 
in \Reg{4} (ovvero $b$).

\myindex{ARM!\Instructions!ADRcc}
\myindex{ARM!\Instructions!BLcc}

Andando avanti vediamo le istruzioni \TT{ADREQ} e \TT{BLEQ} instructions.
Si comporano come \TT{ADR} e \TT{BL}, ma vengono eseguite solo se gli operandi erano uguali al momento dell'ultimo confronto.
Un altra \CMP si trova subito prima di loro (poiché l'esecuzione di \printf potrebbe aver alterato i flag).

\myindex{ARM!\Instructions!LDMccFD}
\myindex{ARM!\Instructions!LDMFD}

Ancora più avanti vediamo \TT{LDMGEFD}, questa istruzione funziona come \TT{LDMFD}\footnote{\ac{LDMFD}},
ma viene eseguita solo quando uno dei valori e maggiore di o uguale all'altro (\IT{Greater or Equal}).
L'istruzione \TT{LDMGEFD SP!, \{R4-R6,PC\}} si comporta come un epilogo di funzione, ma viene eseguita solo se $a>=b$, e solo in tal caso avrà termine l'esecuzione della funzione.

\myindex{Function epilogue}

Nel caso in cui questa condizione non venga soddisfatta, ovvero se $a<b$, il flusso continuerà alla successiva istruzione \\
\TT{\q{LDMFD SP!, \{R4-R6,LR\}}} , un altro epilogo di funzione. Questa istruzione non ripristina soltanto lo stato dei registri \TT{R4-R6} , ma anche \ac{LR} invece di \ac{PC}, non ritornando così dalla funzione.
Le due ultime istruzioni chiamano \printf con la stringa <<a<b\textbackslash{}n>> come unico argomento.
Abbiamo già visto un salto diretto non condizionale alla funzione \printf senza altro codice di uscita/ritorno dalla funzione nella sezione <<\PrintfSeveralArgumentsSectionName>> section~(\myref{ARM_B_to_printf}).

\myindex{ARM!\Instructions!ADRcc}
\myindex{ARM!\Instructions!BLcc}
\myindex{ARM!\Instructions!LDMccFD}
\TT{f\_unsigned} è simile, e vengono utilizzate le funzioni \TT{ADRHI}, \TT{BLHI}, e \TT{LDMCSFD}. Questi predicati (\IT{HI = Unsigned higher, CS = Carry Set (maggiore di o uguale a)}) sono analoghi a quelli visti in precedenza, e operano su valori di tipo unsigned.

Nella funzione \main non c'è nulla di nuovo:

\lstinputlisting[caption=\main,style=customasmARM]{patterns/07_jcc/simple/ARM/ARM_O3_main.asm}

In questo modo ci si può sbarazzare dei salti condizionali in modalità ARM.
\myindex{RISC pipeline}
Perchè è bene? Leggi qui: \myref{branch_predictors}.

\myindex{x86!\Instructions!CMOVcc}

Non esiste una funzionalità simile in x86, eccetto per l'istruzione \TT{CMOVcc} , che è uguale a \MOV ma viene eseguita solo
se specifici flag sono settati, solitamente da \CMP.

\mysubparagraph{\OptimizingKeilVI (\ThumbMode)}

\lstinputlisting[caption=\OptimizingKeilVI (\ThumbMode),style=customasmARM]{patterns/07_jcc/simple/ARM/ARM_thumb_signed.asm}

\myindex{ARM!\Instructions!BLE}
\myindex{ARM!\Instructions!BNE}
\myindex{ARM!\Instructions!BGE}
\myindex{ARM!\Instructions!BLS}
\myindex{ARM!\Instructions!BCS}
\myindex{ARM!\Instructions!B}
\myindex{ARM!\ThumbMode}

Solo le istruzioni \TT{B} in modalità Thumb possono essere supplementate da \IT{condition codes}, pertanto il codice Thumb 
ha un aspetto più ordinario.

\TT{BLE} è un normale jump condizionale \IT{Less than or Equal}, 
\TT{BNE}---\IT{Not Equal}, 
\TT{BGE}---\IT{Greater than or Equal}.

\TT{f\_unsigned} è simile, con la differenza che vengono usate altre istruzioni per operare con valori
di tipo unsigned: \TT{BLS} 
(\IT{Unsigned lower or same}) e \TT{BCS} (\IT{Carry Set (Greater than or equal)}).
}

\EN{\myparagraph{ARM64: \Optimizing GCC (Linaro) 4.9}

\lstinputlisting[caption=f\_signed(),style=customasmARM]{patterns/07_jcc/simple/ARM/ARM64_GCC_O3_signed_EN.lst}

\lstinputlisting[caption=f\_unsigned(),style=customasmARM]{patterns/07_jcc/simple/ARM/ARM64_GCC_O3_unsigned_EN.lst}

The comments were added by the author of this book.
What is striking is that the compiler is not aware that some conditions are not possible at all,
so there is dead code at some places, which can never be executed.

\mysubparagraph{\Exercise}

Try to optimize these functions manually for size, removing redundant instructions, without adding new ones.

}
\RU{\myparagraph{ARM64: \Optimizing GCC (Linaro) 4.9}

\lstinputlisting[caption=f\_signed(),style=customasmARM]{patterns/07_jcc/simple/ARM/ARM64_GCC_O3_signed_RU.lst}

\lstinputlisting[caption=f\_unsigned(),style=customasmARM]{patterns/07_jcc/simple/ARM/ARM64_GCC_O3_unsigned_RU.lst}

Комментарии добавлены автором этой книги.
В глаза бросается то, что компилятор не в курсе, что некоторые ситуации невозможны,
поэтому кое-где в функциях остается код, который никогда не исполнится.

\mysubparagraph{\Exercise}

Попробуйте вручную оптимизировать функции по размеру, убрав избыточные инструкции и не добавляя новых.
}
\DE{\myparagraph{ARM64: \Optimizing GCC (Linaro) 4.9}

\lstinputlisting[caption=f\_signed(),style=customasmARM]{patterns/07_jcc/simple/ARM/ARM64_GCC_O3_signed_DE.lst}

\lstinputlisting[caption=f\_unsigned(),style=customasmARM]{patterns/07_jcc/simple/ARM/ARM64_GCC_O3_unsigned_DE.lst}
Die Kommentare stammen vom Autor. 
Erstaunlich ist hier, dass der Compiler nicht bemerkt, dass einige Bedingungen unmöglich zu erfüllen sind, sodass Dead
Code vorliegt, der nie ausgeführt werden kann.

\mysubparagraph{\Exercise}
Versuchen Sie die Funktionen manuell hinsichtlich Größe und Entfernen redundanter Befehle zu optimieren.
}
\FR{\myparagraph{ARM64: GCC (Linaro) 4.9 \Optimizing}

\lstinputlisting[caption=f\_signed(),style=customasmARM]{patterns/07_jcc/simple/ARM/ARM64_GCC_O3_signed_FR.lst}

\lstinputlisting[caption=f\_unsigned(),style=customasmARM]{patterns/07_jcc/simple/ARM/ARM64_GCC_O3_unsigned_FR.lst}

Les commentaires ont été ajoutés par l'auteur de ce livre.
Ce qui frappe ici, c'est que le compilateur n'est pas au courant que certaines conditions
ne sont pas possible du tout, donc il y a du code mort par endroit, qui ne sera jamais
exécuté.

\mysubparagraph{\Exercise}

Essayez d'optimiser manuellement la taille de ces fonctions, en supprimant les instructions
redondantes, sans en ajouter de nouvelles.

}
\ITA{\myparagraph{ARM64: \Optimizing GCC (Linaro) 4.9}

\lstinputlisting[caption=f\_signed(),style=customasmARM]{patterns/07_jcc/simple/ARM/ARM64_GCC_O3_signed_ITA.lst}

\lstinputlisting[caption=f\_unsigned(),style=customasmARM]{patterns/07_jcc/simple/ARM/ARM64_GCC_O3_unsigned_ITA.lst}

I commenti nel codice sono stati inseriti dall'autore di questo libro.
E' impressionante notare come il compilatore non si sia reso conto che alcuni condizioni sono del tutto impossibili, e per questo motivo
si trovano delle parti con codice "morto" (dead code), che non può mai essere eseguito.

\mysubparagraph{\Exercise}

Prova ad ottimizzare manualmente queste funzioni per ottenere una versione più compatta, rimuovendo istruzioni ridondanti e senza aggiungerne di nuove.

}

\subsection{MIPS}

\RU{Одна отличительная особенность MIPS это отсутствие регистра флагов.}
\EN{One distinctive MIPS feature is absence of flags.}
\RU{Очевидно, так было сделано для упрощения анализа зависимости данных (data dependency).}
\EN{Apparently, it was done so to simplify data dependency analysis.}

\index{x86!\Instructions!SETcc}
\index{MIPS!\Instructions!SLT}
\index{MIPS!\Instructions!SLTU}
\RU{Так что здесь есть инструкция похожая на SETcc в x86: SLT (``Set on Less Than'' --- установить если
меньше чем, знаковая версия) и SLTU (беззнаковая версия).}
\EN{So there are instruction similar to SETcc in x86: SLT (``Set on Less Than'': signed version) and 
SLTU (unsigned version).}
\RU{Эта инструкция устанавливает регистр-получатель в 1 если условие верно или в 0 в противном случае.}
\EN{This instruction sets destination register value to 1 if condition is true or to 0 if otherwise.}

\index{MIPS!\Instructions!BEQ}
\index{MIPS!\Instructions!BNE}
\RU{Затем регистр-получатель проверяется используя инструкцию 
BEQ (``Branch on Equal'' --- переход если равно) или BNE (``Branch on Not Equal'' --- переход если не равно) 
и может произойти переход.}
\EN{Destination register is then checked using BEQ (``Branch on Equal'') or BNE (``Branch on Not Equal'') 
instructions and jump may occur.}

\RU{Так что эта пара инструкций должна использоваться в MIPS для сравнения и перехода.}
\EN{So, this instruction pair should be used in MIPS for comparison and branch.}

\RU{Начнем с знаковой версии нашей ф-ции:}
\EN{Let's first start with signed version of our function:}

\lstinputlisting[caption=\NonOptimizing GCC 4.4.5 (IDA)]{patterns/07_jcc/simple/O0_MIPS_signed_IDA.lst}

``SLT REG0, REG0, REG1'' \RU{сокращается в IDA до более короткой формы}\EN{is reduced by IDA to its 
shorter form}: ``SLT REG0, REG1''.
\index{MIPS!\Pseudoinstructions!BEQZ}
\RU{Мы также видим здесь псевдоинструкцию BEQZ (``Branch if Equal to Zero'' --- переход если равно нулю), 
которая, на самом деле, ``BEQ REG, \$ZERO, LABEL''.}
\EN{We also see there BEQZ pseudoinstruction (``Branch if Equal to Zero''), 
which is in fact ``BEQ REG, \$ZERO, LABEL''.}

\index{MIPS!\Instructions!SLTU}
\RU{Беззнаковая версия точно такая же, только здесь используется SLTU (беззнаковая версия, 
отсюда ``U'' в названии) вместо SLT:}
\EN{Unsigned version is just the same, but SLTU (unsigned version, hence ``U'' in name) is used instead of SLT:}

\lstinputlisting[caption=\NonOptimizing GCC 4.4.5 (IDA)]{patterns/07_jcc/simple/O0_MIPS_unsigned_IDA.lst}


\EN{\subsection{Calculating absolute value}
\label{sec:abs}

A simple function:

\lstinputlisting[style=customc]{abs.c}

\subsubsection{\Optimizing MSVC}

This is how the code is usually generated:

\lstinputlisting[caption=\Optimizing MSVC 2012 x64,style=customasmx86]{patterns/07_jcc/abs/abs_MSVC2012_Ox_x64_EN.asm}

GCC 4.9 does mostly the same.

\subsubsection{\OptimizingKeilVI: \ThumbMode}

\lstinputlisting[caption=\OptimizingKeilVI: \ThumbMode,style=customasmARM]{patterns/07_jcc/abs/abs_Keil_thumb_O3_EN.s}

\myindex{ARM!\Instructions!RSB}

ARM lacks a negate instruction, so the Keil compiler uses the \q{Reverse Subtract} instruction, which just subtracts with reversed operands.

\subsubsection{\OptimizingKeilVI: \ARMMode}

It is possible to add condition codes to some instructions in ARM mode, so that is what the Keil compiler does:

\lstinputlisting[caption=\OptimizingKeilVI: \ARMMode,style=customasmARM]{patterns/07_jcc/abs/abs_Keil_ARM_O3_EN.s}

Now there are no conditional jumps and this is good: \myref{branch_predictors}.

\subsubsection{\NonOptimizing GCC 4.9 (ARM64)}

\myindex{ARM!\Instructions!XOR}

ARM64 has instruction \INS{NEG} for negating:

\lstinputlisting[caption=\Optimizing GCC 4.9 (ARM64),style=customasmARM]{patterns/07_jcc/abs/abs_GCC49_ARM64_O0_EN.s}

\subsubsection{MIPS}

\lstinputlisting[caption=\Optimizing GCC 4.4.5 (IDA),style=customasmMIPS]{patterns/07_jcc/abs/MIPS_O3_IDA_EN.lst}

\myindex{MIPS!\Instructions!BLTZ}
Here we see a new instruction: \INS{BLTZ} (\q{Branch if Less Than Zero}).
\myindex{MIPS!\Instructions!SUBU}
\myindex{MIPS!\Pseudoinstructions!NEGU}

There is also the \INS{NEGU} pseudo instruction, which just does subtraction from zero.
The \q{U} suffix in both \INS{SUBU} and \INS{NEGU} implies that no exception to be raised in case of integer overflow.

\subsubsection{Branchless version?}

You could have also a branchless version of this code. This we will review later: \myref{chap:branchless_abs}.
}
\RU{\subsection{Вычисление абсолютной величины}
\label{sec:abs}

Это простая функция:

\lstinputlisting[style=customc]{abs.c}

\subsubsection{\Optimizing MSVC}

Обычный способ генерации кода:

\lstinputlisting[caption=\Optimizing MSVC 2012 x64,style=customasmx86]{patterns/07_jcc/abs/abs_MSVC2012_Ox_x64_RU.asm}

GCC 4.9 делает почти то же самое.

\subsubsection{\OptimizingKeilVI: \ThumbMode}

\lstinputlisting[caption=\OptimizingKeilVI: \ThumbMode,style=customasmARM]{patterns/07_jcc/abs/abs_Keil_thumb_O3_RU.s}

\myindex{ARM!\Instructions!RSB}
В ARM нет инструкции для изменения знака, так что компилятор Keil использует инструкцию \q{Reverse Subtract},
которая просто вычитает, но с операндами, переставленными наоборот.

\subsubsection{\OptimizingKeilVI: \ARMMode}

В режиме ARM можно добавлять коды условий к некоторым инструкций, что компилятор Keil и сделал:

\lstinputlisting[caption=\OptimizingKeilVI: \ARMMode,style=customasmARM]{patterns/07_jcc/abs/abs_Keil_ARM_O3_RU.s}

Теперь здесь нет условных переходов и это хорошо:
 
\myref{branch_predictors}.

\subsubsection{\NonOptimizing GCC 4.9 (ARM64)}

\myindex{ARM!\Instructions!XOR}
В ARM64 есть инструкция \INS{NEG} для смены знака:

\lstinputlisting[caption=\Optimizing GCC 4.9 (ARM64),style=customasmARM]{patterns/07_jcc/abs/abs_GCC49_ARM64_O0_RU.s}

\subsubsection{MIPS}

\lstinputlisting[caption=\Optimizing GCC 4.4.5 (IDA),style=customasmMIPS]{patterns/07_jcc/abs/MIPS_O3_IDA_RU.lst}

\myindex{MIPS!\Instructions!BLTZ}
Видим здесь новую инструкцию: \INS{BLTZ} (\q{Branch if Less Than Zero}).
\myindex{MIPS!\Instructions!SUBU}
\myindex{MIPS!\Pseudoinstructions!NEGU}
Тут есть также псевдоинструкция \INS{NEGU}, которая на самом деле вычитает из нуля.
Суффикс \q{U} в обоих инструкциях \INS{SUBU} и \INS{NEGU} означает, что при целочисленном переполнении исключение не
сработает.

\subsubsection{Версия без переходов?}

Возможна также версия и без переходов, мы рассмотрим её позже: \myref{chap:branchless_abs}.
}
\DE{\subsection{Betrag berechnen}
\label{sec:abs}

Eine einfache Funktion:

\lstinputlisting[style=customc]{abs.c}

\subsubsection{\Optimizing MSVC}

Normalerweise wird folgender Code erzeugt:

\lstinputlisting[caption=\Optimizing MSVC 2012 x64,style=customasmx86]{patterns/07_jcc/abs/abs_MSVC2012_Ox_x64_DE.asm}

GCC 4.9 macht ungefähr das gleiche.

\subsubsection{\OptimizingKeilVI: \ThumbMode}

\lstinputlisting[caption=\OptimizingKeilVI: \ThumbMode,style=customasmARM]{patterns/07_jcc/abs/abs_Keil_thumb_O3_DE.s}

\myindex{ARM!\Instructions!RSB}
ARM fehlt ein Befehl zur Negation, sodass der Keil Compiler den \q{Reverse
Subtract} Befehl verwendet, der mit umgekehrten Operanden subtrahiert.

\subsubsection{\OptimizingKeilVI: \ARMMode}
Es ist im ARM mode möglich, einigen Befehlen condition codes hinzuzufügen und genau das tut der Keil Compiler:


\lstinputlisting[caption=\OptimizingKeilVI: \ARMMode,style=customasmARM]{patterns/07_jcc/abs/abs_Keil_ARM_O3_DE.s}
Jetzt sind keine bedingten Sprünge mehr übrig und das ist vorteilhaft: \myref{branch_predictors}.


\subsubsection{\NonOptimizing GCC 4.9 (ARM64)}

\myindex{ARM!\Instructions!XOR}

ARM64 kennt den Befehl \INS{NEG} zum Negieren:

\lstinputlisting[caption=\Optimizing GCC 4.9 (ARM64),style=customasmARM]{patterns/07_jcc/abs/abs_GCC49_ARM64_O0_DE.s}

\subsubsection{MIPS}

\lstinputlisting[caption=\Optimizing GCC 4.4.5 (IDA),style=customasmMIPS]{patterns/07_jcc/abs/MIPS_O3_IDA_DE.lst}

\myindex{MIPS!\Instructions!BLTZ}
Hier finden wir einen neuen Befehl: \INS{BLTZ} (\q{Branch if Less Than Zero}).
\myindex{MIPS!\Instructions!SUBU}
\myindex{MIPS!\Pseudoinstructions!NEGU}
Es gibt zusätzlich noch den \INS{NEGU} Pseudo-Befehl, der eine Subtraktion von Null durchführt. Der Suffix \q{U} bei
\INS{SUBU} und \INS{NEGU} zeigt an, dass keine Exception für den Fall eines Integer Overflows geworfen wird.


\subsubsection{Verzweigungslose Version?}
Man kann auch eine verzweigungslose Version dieses Codes erzeugen. Dies werden wir später betrachten:
\myref{chap:branchless_abs}. 
}
\FR{\subsection{Calcul de valeur absolue}
\label{sec:abs}

Une fonction simple:

\lstinputlisting[style=customc]{abs.c}

\subsubsection{MSVC \Optimizing}

Ceci est le code généré habituellement:

\lstinputlisting[caption=MSVC 2012 x64 \Optimizing,style=customasmx86]{patterns/07_jcc/abs/abs_MSVC2012_Ox_x64_FR.asm}

GCC 4.9 génère en gros le même code:

\subsubsection{\OptimizingKeilVI: \ThumbMode}

\lstinputlisting[caption=\OptimizingKeilVI: \ThumbMode,style=customasmARM]{patterns/07_jcc/abs/abs_Keil_thumb_O3_FR.s}

\myindex{ARM!\Instructions!RSB}

Il manque une instruction de négation en ARM, donc le compilateur Keil utilise l'instruction
\q{Reverse Subtract}, qui soustrait la valeur du registre de l'opérande.

\subsubsection{\OptimizingKeilVI: \ARMMode}

Il est possible d'ajouter un code de condition à certaines instructions en mode
ARM, c'est donc ce que fait le compilateur Keil:

\lstinputlisting[caption=\OptimizingKeilVI: \ARMMode,style=customasmARM]{patterns/07_jcc/abs/abs_Keil_ARM_O3_FR.s}

Maintenant, il n'y a plus de saut conditionnel et c'est mieux: \myref{branch_predictors}.

\subsubsection{GCC 4.9 \NonOptimizing (ARM64)}

\myindex{ARM!\Instructions!XOR}

ARM64 possède l'instruction \INS{NEG} pour effectuer la négation:

\lstinputlisting[caption=GCC 4.9 \Optimizing (ARM64),style=customasmARM]{patterns/07_jcc/abs/abs_GCC49_ARM64_O0_FR.s}

\subsubsection{MIPS}

\lstinputlisting[caption=GCC 4.4.5 \Optimizing (IDA),style=customasmMIPS]{patterns/07_jcc/abs/MIPS_O3_IDA_FR.lst}

\myindex{MIPS!\Instructions!BLTZ}
Nous voyons ici une nouvelle instruction: \INS{BLTZ} (\q{Branch if Less Than Zero}
branchement si plus petit que zéro).
\myindex{MIPS!\Instructions!SUBU}
\myindex{MIPS!\Pseudoinstructions!NEGU}

Il y a aussi la pseudo-instruction \INS{NEGU}, qui effectue une soustraction à zéro.
Le suffixe \q{U} dans les deux instructions \INS{SUBU} et \INS{NEGU} indique qu'aucune
exception ne sera levée en cas de débordement de la taille d'un entier.

\subsubsection{Version sans branchement?}

Vous pouvez aussi avoir une version sans branchement de ce code. Ceci sera revu plus
tard: \myref{chap:branchless_abs}.

}

\EN{\subsection{Ternary conditional operator}
\label{chap:cond}

The ternary conditional operator in \CCpp has the following syntax:

\begin{lstlisting}
expression ? expression : expression
\end{lstlisting}

Here is an example:

\lstinputlisting[style=customc]{patterns/07_jcc/cond_operator/cond.c}

\subsubsection{x86}

Old and non-optimizing compilers generate assembly code just as if an \TT{if/else} statement was used:

\lstinputlisting[caption=\NonOptimizing MSVC 2008,style=customasmx86]{patterns/07_jcc/cond_operator/MSVC2008_EN.asm}

\lstinputlisting[caption=\Optimizing MSVC 2008,style=customasmx86]{patterns/07_jcc/cond_operator/MSVC2008_Ox_EN.asm}

Newer compilers are more concise:

\lstinputlisting[caption=\Optimizing MSVC 2012 x64,style=customasmx86]{patterns/07_jcc/cond_operator/MSVC2012_Ox_x64_EN.asm}

\myindex{x86!\Instructions!CMOVcc}
\Optimizing GCC 4.8 for x86 also uses the \TT{CMOVcc} instruction, while the non-optimizing GCC 4.8 uses conditional jumps.

\subsubsection{ARM}

\myindex{x86!\Instructions!ADRcc}
\Optimizing Keil for ARM mode also uses the conditional instructions \TT{ADRcc}:

\lstinputlisting[label=cond_Keil_ARM_O3,caption=\OptimizingKeilVI (\ARMMode),style=customasmARM]{patterns/07_jcc/cond_operator/Keil_ARM_O3_EN.s}

Without manual intervention, the two instructions \TT{ADREQ} and \TT{ADRNE} cannot be executed in the same run.

\Optimizing Keil for Thumb mode needs to use conditional jump instructions, since there are no load instructions
that support conditional flags:

\lstinputlisting[caption=\OptimizingKeilVI (\ThumbMode),style=customasmARM]{patterns/07_jcc/cond_operator/Keil_thumb_O3_EN.s}

\subsubsection{ARM64}

\Optimizing GCC (Linaro) 4.9 for ARM64 also uses conditional jumps:

\lstinputlisting[label=cond_ARM64,caption=\Optimizing GCC (Linaro) 4.9,style=customasmARM]{patterns/07_jcc/cond_operator/ARM64_GCC_O3_EN.s}

That is because ARM64 does not have a simple load instruction with conditional flags,
like \TT{ADRcc} in 32-bit ARM mode or \INS{CMOVcc} in x86.

\myindex{ARM!\Instructions!CSEL}
It has, however, \q{Conditional SELect} instruction (\TT{CSEL})\InSqBrackets{\ARMSixFourRef p390, C5.5},
but GCC 4.9 does not seem to be smart enough to use it in such piece of code.

\subsubsection{MIPS}

Unfortunately, GCC 4.4.5 for MIPS is not very smart, either:

\lstinputlisting[caption=\Optimizing GCC 4.4.5 (\assemblyOutput),style=customasmMIPS]{patterns/07_jcc/cond_operator/MIPS_O3_EN.s}

\subsubsection{Let's rewrite it in an \TT{if/else} way}

\lstinputlisting[style=customc]{patterns/07_jcc/cond_operator/cond2.c}

\myindex{x86!\Instructions!CMOVcc}

Interestingly, optimizing GCC 4.8 for x86 was also able to use \TT{CMOVcc} in this case:

\lstinputlisting[caption=\Optimizing GCC 4.8,style=customasmx86]{patterns/07_jcc/cond_operator/cond2_GCC_O3_EN.s}

\Optimizing Keil in ARM mode generates code identical to \lstref{cond_Keil_ARM_O3}.

But the optimizing MSVC 2012 is not that good (yet).

\subsubsection{\Conclusion{}}

Why optimizing compilers try to get rid of conditional jumps? Read here about it: \myref{branch_predictors}.
}
\RU{\subsection{Тернарный условный оператор}
\label{chap:cond}

Тернарный условный оператор (ternary conditional operator) в \CCpp это:

\begin{lstlisting}
expression ? expression : expression
\end{lstlisting}

И вот пример:

\lstinputlisting[style=customc]{patterns/07_jcc/cond_operator/cond.c}

\subsubsection{x86}

Старые и неоптимизирующие компиляторы генерируют код так, как если бы выражение \TT{if/else} было использовано
вместо него:

\lstinputlisting[caption=\NonOptimizing MSVC 2008,style=customasmx86]{patterns/07_jcc/cond_operator/MSVC2008_RU.asm}

\lstinputlisting[caption=\Optimizing MSVC 2008,style=customasmx86]{patterns/07_jcc/cond_operator/MSVC2008_Ox_RU.asm}

Новые компиляторы могут быть более краткими:

\lstinputlisting[caption=\Optimizing MSVC 2012 x64,style=customasmx86]{patterns/07_jcc/cond_operator/MSVC2012_Ox_x64_RU.asm}

\myindex{x86!\Instructions!CMOVcc}
\Optimizing GCC 4.8 для x86 также использует инструкцию \TT{CMOVcc},
тогда как неоптимизирующий GCC 4.8 использует условные переходы.

\subsubsection{ARM}

\myindex{x86!\Instructions!ADRcc}
\Optimizing Keil для режима ARM также использует инструкцию \INS{ADRcc}, срабатывающую при некотором
условии:

\lstinputlisting[label=cond_Keil_ARM_O3,caption=\OptimizingKeilVI (\ARMMode),style=customasmARM]{patterns/07_jcc/cond_operator/Keil_ARM_O3_RU.s}

Без внешнего вмешательства инструкции \TT{ADREQ} и \TT{ADRNE} никогда не исполнятся одновременно.
\Optimizing Keil для режима Thumb вынужден использовать инструкции условного перехода, потому
что тут нет инструкции загрузки значения, поддерживающей флаги условия:

\lstinputlisting[caption=\OptimizingKeilVI (\ThumbMode),style=customasmARM]{patterns/07_jcc/cond_operator/Keil_thumb_O3_RU.s}

\subsubsection{ARM64}

\Optimizing GCC (Linaro) 4.9 для ARM64 также использует условные переходы:

\lstinputlisting[label=cond_ARM64,caption=\Optimizing GCC (Linaro) 4.9,style=customasmARM]{patterns/07_jcc/cond_operator/ARM64_GCC_O3_RU.s}

Это потому что в ARM64 нет простой инструкции загрузки с флагами условия, как \TT{ADRcc} в 32-битном 
режиме ARM или \TT{CMOVcc} в x86.

\myindex{ARM!\Instructions!CSEL}
Но с другой стороны, там есть инструкция \TT{CSEL} (\q{Conditional SELect})
\InSqBrackets{\ARMSixFourRef p390, C5.5},
но GCC 4.9 наверное, пока не так
хорош, чтобы генерировать её в таком фрагменте кода

\subsubsection{MIPS}

GCC 4.4.5 для MIPS тоже не так хорош, к сожалению:

\lstinputlisting[caption=\Optimizing GCC 4.4.5 (\assemblyOutput),style=customasmMIPS]{patterns/07_jcc/cond_operator/MIPS_O3_RU.s}

\subsubsection{Перепишем, используя обычный \TT{if/else}}

\lstinputlisting[style=customc]{patterns/07_jcc/cond_operator/cond2.c}

\myindex{x86!\Instructions!CMOVcc}
Интересно, оптимизирующий GCC 4.8 для x86 также может генерировать \TT{CMOVcc} в этом случае:

\lstinputlisting[caption=\Optimizing GCC 4.8,style=customasmx86]{patterns/07_jcc/cond_operator/cond2_GCC_O3_RU.s}

\Optimizing Keil в режиме ARM генерирует код идентичный этому: \lstref{cond_Keil_ARM_O3}.

Но оптимизирующий MSVC 2012 пока не так хорош.

\subsubsection{\Conclusion{}}

Почему оптимизирующие компиляторы стараются избавиться от условных переходов? Читайте больше об этом здесь:
 \myref{branch_predictors}.
}


\EN{\section{Getting minimal and maximal values}

\subsection{32-bit}

\lstinputlisting{patterns/07_jcc/minmax/minmax.c}

\lstinputlisting[caption=\NonOptimizing MSVC 2013]{patterns/07_jcc/minmax/minmax_MSVC_2013.asm.\LANG}

\myindex{x86!\Instructions!Jcc}

These two functions differ only in the conditional jump instruction: 
\INS{JGE} (\q{Jump if Greater or Equal}) is used in the first one
and \INS{JLE} (\q{Jump if Less or Equal}) in the second.

\myindex{\CompilerAnomaly}
\label{MSVC_double_JMP_anomaly}

There is one unneeded \JMP instruction in each function, which MSVC probably left by mistake.

\subsubsection{Branchless}

ARM for Thumb mode reminds us of x86 code:

\lstinputlisting[caption=\OptimizingKeilVI (\ThumbMode)]{patterns/07_jcc/minmax/minmax_Keil_Thumb_O3.s.\LANG}

\myindex{ARM!\Instructions!Bcc}

The functions differ in the branching instruction: \INS{BGT} and \INS{BLT}.
It's possible to use conditional suffixes in ARM mode, so the code is shorter.

\myindex{ARM!\Instructions!MOVcc}
\INS{MOVcc} is to be executed only if the condition is met:

\lstinputlisting[caption=\OptimizingKeilVI (\ARMMode)]{patterns/07_jcc/minmax/minmax_Keil_ARM_O3.s.\LANG}

\myindex{x86!\Instructions!CMOVcc}
\Optimizing GCC 4.8.1 and optimizing MSVC 2013 can use \INS{CMOVcc} instruction, which is analogous to \INS{MOVcc} in ARM:

\lstinputlisting[caption=\Optimizing MSVC 2013]{patterns/07_jcc/minmax/minmax_GCC481_O3.s.\LANG}

\subsection{64-bit}

\lstinputlisting{patterns/07_jcc/minmax/minmax64.c}

There is some unneeded value shuffling, but the code is comprehensible:

\lstinputlisting[caption=\NonOptimizing GCC 4.9.1 ARM64]{patterns/07_jcc/minmax/minmax64_GCC_49_ARM64_O0.s}

\subsubsection{Branchless}

No need to load function arguments from the stack, as they are already in the registers:

\lstinputlisting[caption=\Optimizing GCC 4.9.1 x64]{patterns/07_jcc/minmax/minmax64_GCC_49_x64_O3.s.\LANG}

MSVC 2013 does almost the same.

\myindex{ARM!\Instructions!CSEL}

ARM64 has the \INS{CSEL} instruction, which works just as \INS{MOVcc} in ARM or \INS{CMOVcc} in x86, just the name is different:
\q{Conditional SELect}.

\lstinputlisting[caption=\Optimizing GCC 4.9.1 ARM64]{patterns/07_jcc/minmax/minmax64_GCC_49_ARM64_O3.s.\LANG}

\ifdefined\IncludeMIPS
\subsection{MIPS}

Unfortunately, GCC 4.4.5 for MIPS is not that good:

\lstinputlisting[caption=\Optimizing GCC 4.4.5 (IDA)]{patterns/07_jcc/minmax/minmax_MIPS_O3_IDA.lst.\LANG}

Do not forget about the \IT{branch delay slots}: the first \INS{MOVE} is executed \IT{before} \INS{BEQZ}, 
the second \INS{MOVE} is executed only if the branch wasn't taken.

\fi % MIPS

}
\RU{\subsection{Поиск минимального и максимального значения}

\subsubsection{32-bit}

\lstinputlisting[style=customc]{patterns/07_jcc/minmax/minmax.c}

\lstinputlisting[caption=\NonOptimizing MSVC 2013,style=customasmx86]{patterns/07_jcc/minmax/minmax_MSVC_2013_RU.asm}

\myindex{x86!\Instructions!Jcc}
Эти две функции отличаются друг от друга только инструкцией условного перехода:
\INS{JGE} (\q{Jump if Greater or Equal}~--- переход если больше или равно) используется в первой
и \INS{JLE} (\q{Jump if Less or Equal}~--- переход если меньше или равно) во второй.

\myindex{\CompilerAnomaly}
\label{MSVC_double_JMP_anomaly}
Здесь есть ненужная инструкция \JMP в каждой функции, которую MSVC, наверное, оставил по ошибке.

\myparagraph{Без переходов}

ARM в режиме Thumb напоминает нам x86-код:

\lstinputlisting[caption=\OptimizingKeilVI (\ThumbMode),style=customasmARM]{patterns/07_jcc/minmax/minmax_Keil_Thumb_O3_RU.s}

\myindex{ARM!\Instructions!Bcc}
Функции отличаются только инструкцией перехода: \INS{BGT} и \INS{BLT}.
А в режиме ARM можно использовать условные суффиксы, так что код более плотный.
\INS{MOVcc} будет исполнена только если условие верно:

\myindex{ARM!\Instructions!MOVcc}

\lstinputlisting[caption=\OptimizingKeilVI (\ARMMode),style=customasmARM]{patterns/07_jcc/minmax/minmax_Keil_ARM_O3_RU.s}

\myindex{x86!\Instructions!CMOVcc}
\Optimizing GCC 4.8.1 и оптимизирующий MSVC 2013 
могут использовать инструкцию \INS{CMOVcc}, которая аналогична \INS{MOVcc} в ARM:

\lstinputlisting[caption=\Optimizing MSVC 2013,style=customasmx86]{patterns/07_jcc/minmax/minmax_GCC481_O3_RU.s}

\subsubsection{64-bit}

\lstinputlisting[style=customc]{patterns/07_jcc/minmax/minmax64.c}

Тут есть ненужные перетасовки значений, но код в целом понятен:

\lstinputlisting[caption=\NonOptimizing GCC 4.9.1 ARM64,style=customasmARM]{patterns/07_jcc/minmax/minmax64_GCC_49_ARM64_O0.s}

\myparagraph{Без переходов}

Нет нужды загружать аргументы функции из стека, они уже в регистрах:

\lstinputlisting[caption=\Optimizing GCC 4.9.1 x64,style=customasmx86]{patterns/07_jcc/minmax/minmax64_GCC_49_x64_O3_RU.s}

MSVC 2013 делает то же самое.

\myindex{ARM!\Instructions!CSEL}
В ARM64 есть инструкция \INS{CSEL}, которая работает точно также, как и \INS{MOVcc} в ARM и \INS{CMOVcc} в x86,
но название другое: \q{Conditional SELect}.

\lstinputlisting[caption=\Optimizing GCC 4.9.1 ARM64,style=customasmARM]{patterns/07_jcc/minmax/minmax64_GCC_49_ARM64_O3_RU.s}

\subsubsection{MIPS}

А GCC 4.4.5 для MIPS не так хорош, к сожалению:

\lstinputlisting[caption=\Optimizing GCC 4.4.5 (IDA),style=customasmMIPS]{patterns/07_jcc/minmax/minmax_MIPS_O3_IDA_RU.lst}

Не забывайте о \IT{branch delay slots}: первая \INS{MOVE} исполняется \IT{перед} \INS{BEQZ},
вторая \INS{MOVE} исполняется только если переход не произошел.

}



\subsection{\Conclusion{}}

\subsubsection{x86}

Примерный скелет условных переходов:

\begin{lstlisting}[caption=x86,style=customasmx86]
CMP register, register/value
Jcc true ; §cc=код условия§
false:
... код, исполняющийся, если сравнение ложно ...
JMP exit 
true:
... код, исполняющийся, если сравнение истинно ...
exit:
\end{lstlisting}

\subsubsection{ARM}

\begin{lstlisting}[caption=ARM,style=customasmARM]
CMP register, register/value
Bcc true ; §cc=код условия§
false:
... код, исполняющийся, если сравнение ложно ...
JMP exit 
true:
... код, исполняющийся, если сравнение истинно ...
exit:
\end{lstlisting}

\subsubsection{MIPS}

\begin{lstlisting}[caption=Проверка на ноль,style=customasmMIPS]
BEQZ REG, label
...
\end{lstlisting}

\begin{lstlisting}[caption=Меньше ли нуля? (используя псевдоинструкцию),style=customasmMIPS]
BLTZ REG, label
...
\end{lstlisting}

\begin{lstlisting}[caption=Проверка на равенство,style=customasmMIPS]
BEQ REG1, REG2, label
...
\end{lstlisting}

\begin{lstlisting}[caption=Проверка на неравенство,style=customasmMIPS]
BNE REG1, REG2, label
...
\end{lstlisting}

\begin{lstlisting}[caption=Проверка на меньше (знаковое),style=customasmMIPS]
SLT REG1, REG2, REG3
BEQ REG1, label
...
\end{lstlisting}

\begin{lstlisting}[caption=Проверка на меньше (беззнаковое),style=customasmMIPS]
SLTU REG1, REG2, REG3
BEQ REG1, label
...
\end{lstlisting}

\subsubsection{Без инструкций перехода}

\myindex{ARM!\Instructions!MOVcc}
\myindex{x86!\Instructions!CMOVcc}
\myindex{ARM!\Instructions!CSEL}

Если тело условного выражения очень короткое, может быть
использована инструкция условного копирования: \INS{MOVcc} в ARM (в режиме ARM), \INS{CSEL} в ARM64, \INS{CMOVcc} в x86.

\myparagraph{ARM}

В режиме ARM можно использовать условные суффиксы для некоторых инструкций:

\begin{lstlisting}[caption=ARM (\ARMMode),style=customasmARM]
CMP register, register/value
instr1_cc ; инструкция, которая будет исполнена, если условие истинно
instr2_cc ; еще инструкция, которая будет исполнена, если условие истинно
... и т.д....
\end{lstlisting}

Нет никаких ограничений на количество инструкций с условными суффиксами до тех пор,
пока флаги CPU не были модифицированы одной из таких инструкций.

% FIXME: list of such instructions or \myref{} to it

\myindex{ARM!\Instructions!IT}
В режиме Thumb есть инструкция \INS{IT}, позволяющая дополнить следующие 4 инструкции суффиксами, задающими
условие.

Читайте больше об этом: \myref{ARM_Thumb_IT}.

\begin{lstlisting}[caption=ARM (\ThumbMode),style=customasmARM]
CMP register, register/value
ITEEE EQ ; выставить такие суффиксы: if-then-else-else-else
instr1   ; инструкция будет исполнена, если истинно
instr2   ; инструкция будет исполнена, если ложно
instr3   ; инструкция будет исполнена, если ложно
instr4   ; инструкция будет исполнена, если ложно
\end{lstlisting}

\subsection{\Exercise}

(ARM64) Попробуйте переписать код в \lstref{cond_ARM64} 
убрав все инструкции условного перехода, и используйте инструкцию \INS{CSEL}.

