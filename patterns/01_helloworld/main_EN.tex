\mysection{\HelloWorldSectionName}
\label{sec:helloworld}

Let's use the famous example from the book [\KRBook]:

\lstinputlisting[caption=\CCpp Code,style=customc]{patterns/01_helloworld/hw.c}

\subsection{x86}

\subsection{MSVC}

\RU{Компилируем в}\EN{Let's compile it in}\NL{We compileren het in} MSVC 2010:

\begin{lstlisting}
cl 1.cpp /Fa1.asm
\end{lstlisting}

\RU{(Ключ /Fa означает сгенерировать листинг на ассемблере)}%
\EN{(/Fa option instructs the compiler to generate assembly listing file)}%
\NL{(/Fa optie zorgt ervoor dat de compiler het bestand met assembly listing genereert)}%

\begin{lstlisting}[caption=MSVC 2010]
CONST	SEGMENT
$SG3830	DB	'hello, world', 0AH, 00H
CONST	ENDS
PUBLIC	_main
EXTRN	_printf:PROC
; Function compile flags: /Odtp
_TEXT	SEGMENT
_main	PROC
	push	ebp
	mov	ebp, esp
	push	OFFSET $SG3830
	call	_printf
	add	esp, 4
	xor	eax, eax
	pop	ebp
	ret	0
_main	ENDP
_TEXT	ENDS
\end{lstlisting}

\ifx\LITE\undefined
\RU{MSVC выдает листинги в синтаксисе Intel.}\EN{MSVC produces assembly listings in Intel-syntax.}\NLph{}
\RU{Разница между синтаксисом Intel и AT\&T будет рассмотрена немного позже:}
\EN{The difference between Intel-syntax and AT\&T-syntax will be discussed in} 
\NL{Het verschil tussen Intel-syntax en AT\&T-syntax zal besproken worden in:}\myref{ATT_syntax}.
\fi

\RU{Компилятор сгенерировал файл \TT{1.obj}, который впоследствии будет слинкован линкером в \TT{1.exe}.}%
\EN{The compiler generated the file, \TT{1.obj}, which is to be linked into \TT{1.exe}.}%
\NL{De compiler heeft het bestand, \TT{1.obj} gegenereerd, hetwelk gelinkt wordt tot \TT{1.exe}.}%
\RU{В нашем случае этот файл состоит из двух сегментов: \TT{CONST} (для данных-констант) и \TT{\_TEXT} (для кода).}%
\EN{In our case, the file contains two segments: \TT{CONST} (for data constants) and \TT{\_TEXT} (for code).}%
\NL{In ons geval bevat het bestand twee segmenten: \TT{CONST} (voor data constanten) en \TT{\_TEXT}(voor code).}%

\index{\CLanguageElements!const}
\label{string_is_const_char}
\RU{Строка \TT{hello, world} в \CCpp имеет тип \TT{const char[]} \cite[p176, 7.3.2]{TCPPPL}, 
однако не имеет имени.}%
\EN{The string \TT{hello, world} in \CCpp has type \TT{const char[]} \cite[p176, 7.3.2]{TCPPPL},
but it does not have its own name.}%
\NL{De string \TT{hello, world} in \CCpp is van het type \TT{const char[]} \cite[p176, 7.3.2]{TCPPPL},
maar heeft geen eigen naam.}%
\RU{Но компилятору нужно как-то с ней работать, поэтому он дает ей внутреннее имя \TT{\$SG3830}.}%
\EN{The compiler needs to deal with the string somehow so it defines the internal name \TT{\$SG3830} for it.}%
\NL{De compiler moet een manier hebben om met de string om te kunnen, en definieert er daarom de interne naam \TT{\$SG3830} voor.}%

\RU{Поэтому пример можно было бы переписать вот так}\EN{That is why the example may be rewritten as follows}\NL{Daarom kan het voorbeeld herschreven worden als volgt}:

\lstinputlisting{patterns/01_helloworld/hw_2.c}

\RU{Вернемся к листингу на ассемблере. Как видно, строка заканчивается нулевым байтом~--- это требования стандарта \CCpp для строк.}%
\EN{Let's go back to the assembly listing. As we can see, the string is terminated by a zero byte, which is standard for \CCpp strings.}%
\NL{Laten we terug gaan naar de assembly listing. Zoals je kan zien, wordt de string beeindigd door een nul-byte. Dit is standaard voor \CCpp strings.}%
\RU{Больше о строках в Си}\EN{More about C strings}\NL{Meer over C strings}: \myref{C_strings}.

\RU{В сегменте кода \TT{\_TEXT} находится пока только одна функция}%
\EN{In the code segment, \TT{\_TEXT}, there is only one function so far}%
\NL{In het code segment, \TT{\_TEXT}, is er slechts een functie tot nu toe}: \main.
\RU{Функция \main, как и практически все функции, начинается с пролога и заканчивается эпилогом}%
\EN{The function \main starts with prologue code and ends with epilogue code (like almost any function)}%
\NL{De functie \main begint met een proloog code en eindigt met een epiloog code (zoals bijna elke functie)}%
\footnote{\RU{Об этом смотрите подробнее в разделе о прологе и эпилоге функции}%
\EN{You can read more about it in the section about function prologues and epilogues}%
\NL{Je kan hier meer over lezen in de sectie over functieprologen en epilogen}%
~(\myref{sec:prologepilog}).}.

\index{x86!\Instructions!CALL}
\RU{Далее следует вызов функции \printf}
\EN{After the function prologue we see the call to the \printf function}
\NL{Na de functie proloog zien we de call naar de \printf functie}: \TT{CALL \_printf}. 
\index{x86!\Instructions!PUSH}
\RU{Перед этим вызовом адрес строки (или указатель на неё) с нашим приветствием при помощи инструкции \PUSH помещается в стек.}
\EN{Before the call the string address (or a pointer to it) containing our greeting is placed on the stack with the help of the \PUSH instruction.}
\NL{Voor de call wordt het adres van de string (of een pointer ernaar) die onze begroeting bevat, op de stack geplaatsd met de hulp van de \PUSH instructie.}

\RU{После того, как функция \printf возвращает управление в функцию \main, адрес строки (или указатель на неё) всё ещё лежит в стеке.}%
\EN{When the \printf function returns the control to the \main function, the string address (or a pointer to it) is still on the stack.}%
\NL{Wanneer de \printf functie de controle teruggeeft aan de \main functie, staat het string adres (of de pointer ernaar) nog steeds op de stack.}%
\RU{Так как он больше не нужен, то \glslink{stack pointer}{указатель стека} (регистр \ESP) корректируется.}%
\EN{Since we do not need it anymore, the \gls{stack pointer} (the \ESP register) needs to be corrected.}%
\NL{Aangezien we dit niet meer nodig hebben, moet de \gls{stack pointer} (het \ESP register) gecorrigeerd worden.}%

\index{x86!\Instructions!ADD}
\TT{ADD ESP, 4} \RU{означает прибавить 4 к значению в регистре \ESP.}
\EN{means add 4 to the \ESP register value.}
\NL{betekent dat er 4 wordt opgeteld bij de \ESP registerwaarde.}
\RU{Почему 4? Так как это 32-битный код, для передачи адреса нужно 4 байта. В x64-коде это 8 байт.}
\EN{Why 4? Since this is a 32-bit program, we need exactly 4 bytes for address passing through the stack. If it was x64 code we would need 8 bytes.}
\NL{Waarom 4? Aangezien dit een 32-bit programma is, hebben we exact 4 bytes nodig om een adres door te geven via de stack. als het x64 code was, zouden we 8 bytes nodig gehad hebben.}
\TT{ADD ESP, 4} \RU{эквивалентно \TT{POP регистр}, но без использования какого-либо регистра\footnote{Флаги
процессора, впрочем, модифицируются}.}
\EN{is effectively equivalent to \TT{POP register} but without using any register\footnote{CPU flags, however, are modified}.}
\NL{is een effectief equivalent voor \TT{POP register} maar zonder gebruik van een register\footnote{CPU flags worden echter wel aangepast}.}

\index{Intel C++}
\index{\oracle}
\index{x86!\Instructions!POP}

\RU{Некоторые компиляторы, например, Intel C++ Compiler, в этой же ситуации могут вместо 
\ADD сгенерировать \TT{POP ECX} (подобное можно встретить, например, в коде \oracle{}, им скомпилированном),
что почти то же самое, только портится значение в регистре \ECX.}
\EN{For the same purpose, some compilers (like the Intel C++ Compiler) may emit \TT{POP ECX} 
instead of \ADD (e.g., such a pattern can be observed in the \oracle{} code as it is compiled with the Intel C++ compiler).
This instruction has almost the same effect but the \ECX register contents will be overwritten.}
\NL{Met dezelfde reden zullen sommige compilers (zoals de Intel C++ Compiler) gebruik maken van \TT{POP ECX}
in plaats van \ADD (een dergelijk patroon kan waargenomen worden in de \oracle{} code aangezien deze gecompileerd is met de Intel C++ compiler).
Deze instructie heeft bijna hetzelfde effect, maar de inhoud van het \ECX register zal overschreven worden.}
\RU{Возможно, компилятор применяет \TT{POP ECX}, потому что эта инструкция короче (1 байт у \TT{POP} против 3 у \TT{ADD}).}
\EN{The Intel C++ compiler probably uses \TT{POP ECX} since this instruction's opcode is shorter than 
\TT{ADD ESP, x} (1 byte for \TT{POP} against 3 for \TT{ADD}).}
\NL{De Intel C++ Compiler gebruikt waarschijnlijk \TT{POP ECX} aangezien de opcode van deze instructie
korter is dan \TT{ADD ESP, x} (1 byte voor \TT{POP} tegen 3 voor \TT{ADD}).}

\RU{Вот пример использования \TT{POP} вместо \TT{ADD} из \oracle{}:}
\EN{Here is an example of using \TT{POP} instead of \TT{ADD} from \oracle{}:}
\NL{Hier is een voorbeeld van het gebruik van \TT{POP} in plaats van \TT{ADD} van \oracle{}:}

\begin{lstlisting}[caption=\oracle 10.2 Linux (\RU{файл }app.o\EN{ file}\NL{ bestand})]
.text:0800029A                 push    ebx
.text:0800029B                 call    qksfroChild
.text:080002A0                 pop     ecx
\end{lstlisting}

%\RU{О стеке можно прочитать в соответствующем разделе}
%\EN{Read more about the stack in section}
%\NL{Lees meer over de stack in de sectie} ~(\myref{sec:stack}).
\index{\CLanguageElements!return}
\RU{После вызова \printf в оригинальном коде на \CCpp указано \TT{return 0}~--- вернуть 0 
в качестве результата функции \main.}
\EN{After calling \printf, the original \CCpp code contains the statement \TT{return 0}~---return 0 as the result of the \main function.}
\NL{Na \printf aan te roepen, bevat de originele \CCpp code het statement \TT{return 0}~---return 0 als resultaat van de \main functie.}
\index{x86!\Instructions!XOR}
\RU{В сгенерированном коде это обеспечивается инструкцией \INS{XOR EAX, EAX}.}
\EN{In the generated code this is implemented by the instruction \INS{XOR EAX, EAX}.}
\NL{In de gegenereerde code wordt dit geimplementeerd door de instructie \INS{XOR EAX, EAX}.}
\index{x86!\Instructions!MOV}
\RU{\XOR, как легко догадаться~--- \q{исключающее ИЛИ}}%
\EN{\XOR is in fact just \q{eXclusive OR}}%
\NL{\XOR is feitelijk simpelweg \q{eXclusive OR}}%
\footnote{\href{http://go.yurichev.com/17118}{wikipedia}}
\RU{, но компиляторы часто используют его вместо простого}
\EN{but the compilers often use it instead of}
\NL{maar de compilers gebruiken het vaak in plaats van}
\TT{MOV EAX, 0}\EMDASH{}\RU{снова потому, что опкод короче (2 байта у \TT{XOR} против 5 у \TT{MOV}).}
\EN{again because it is a slightly shorter opcode (2 bytes for \TT{XOR} against 5 for \TT{MOV}).}
\NL{wederom omdat de opcode hiervoor iets korter is (2 bytes voor \TT{XOR} tegenover 5 voor \TT{MOV}).}

\index{x86!\Instructions!SUB}
\RU{Некоторые компиляторы генерируют}\EN{Some compilers emit}\NL{Sommige compilers gebruiken}
\INS{SUB EAX, EAX},
\RU{что значит \IT{отнять значение в} \EAX \IT{от значения в }\EAX,
что в любом случае даст 0 в результате.}
\EN{which means \IT{SUBtract the value in the} \EAX \IT{from the value in} \EAX,
which, in any case, results in zero.}
\NL{wat staat voor \IT{verminder de waarde in} \EAX \IT{met de waarde in} \EAX,
wat in elke situatie resulteert in nul.}

\index{x86!\Instructions!RET}
\RU{Самая последняя инструкция \RET возвращает управление в вызывающую функцию.
Обычно это код \CCpp \ac{CRT}, который, в свою очередь, 
вернёт управление операционной системе.}
\EN{The last instruction \RET returns the control to the \gls{caller}.
Usually, this is \CCpp \ac{CRT} code, which, in turn, returns control to the \ac{OS}.}
\NL{De laatste instructie \RET geeft de controle terug aan de \gls{caller}.
Gewoonlijk is dit \CCpp \ac{CRT} code, die op zijn beurt de controle teruggeeft aan het \ac{OS}.}


\EN{\subsubsection{GCC}

% The text states that GCC uses Intel syntax, but the footnote sounds like in needs to be activated
% Maybe edit the text to: GCC can produce Intel syntax (like MSVC), and the footnote to: Use the \TT{-S -masm=intel}.} to activate this
Now let's try to compile the same \CCpp code in the GCC 4.4.1 compiler in Linux: \TT{gcc 1.c -o 1}.
Next, with the assistance of the \IDA disassembler, let's see how the \main function was created.
\IDA, like MSVC, uses Intel-syntax\footnote{We could also have GCC produce assembly listings in Intel-syntax by applying the options \TT{-S -masm=intel}.}.

\begin{lstlisting}[caption=code in \IDA,style=customasmx86]
main            proc near

var_10          = dword ptr -10h

                push    ebp
                mov     ebp, esp
                and     esp, 0FFFFFFF0h
                sub     esp, 10h
                mov     eax, offset aHelloWorld ; "hello, world\n"
                mov     [esp+10h+var_10], eax
                call    _printf
                mov     eax, 0
                leave
                retn
main            endp
\end{lstlisting}

\myindex{Function prologue}
\myindex{x86!\Instructions!AND}
The result is almost the same.
The address of the \TT{hello, world} string (stored in the data segment) is loaded in the \EAX register first, and then saved onto the stack. \\
In addition, the function prologue has \INS{AND ESP, 0FFFFFFF0h}~---this
instruction aligns the \ESP register value on a 16-byte boundary.
This results in all values in the stack being aligned the same way (The CPU performs better if the values it is dealing with are located in memory at addresses aligned
on a 4-byte or 16-byte boundary)\footnote{\URLWPDA}.

\myindex{x86!\Instructions!SUB}
\INS{SUB ESP, 10h} allocates 16 bytes on the stack. Although, as we can see hereafter, only 4 are necessary here.

This is because the size of the allocated stack is also aligned on a 16-byte boundary.

% TODO1: rewrite.
\myindex{x86!\Instructions!PUSH}
The string address (or a pointer to the string) is then stored directly onto the stack without using the \PUSH instruction.
\IT{var\_10}~---is a local variable and is also an argument for \printf{}.
Read about it below.

Then the \printf function is called.

Unlike MSVC, when GCC is compiling without optimization turned on, it emits \TT{MOV EAX, 0} instead of a shorter opcode.

\myindex{x86!\Instructions!LEAVE}
The last instruction, \LEAVE~---is the equivalent of the \TT{MOV ESP, EBP} and \TT{POP EBP} instruction pair~---in other words, this instruction sets the \gls{stack pointer} (\ESP) back and restores the \EBP register to its initial state.
This is necessary since we modified these register values (\ESP and \EBP) at the beginning of the function (by executing \INS{MOV EBP, ESP} / \INS{AND ESP, \ldots}).

\subsubsection{GCC: \ATTSyntax}
\label{ATT_syntax}

Let's see how this can be represented in assembly language AT\&T syntax.
This syntax is much more popular in the UNIX-world.

\begin{lstlisting}[caption=let's compile in GCC 4.7.3]
gcc -S 1_1.c
\end{lstlisting}

We get this:

\lstinputlisting[caption=GCC 4.7.3,style=customasmx86]{patterns/01_helloworld/GCC.s}

The listing contains many macros (the parts that begin with a dot). These are not interesting for us at the moment.

For now, for the sake of simplicity, we can ignore them (except the \IT{.string} macro which
encodes a null-terminated character sequence just like a C-string). Then we'll see this
\footnote{This GCC option can be used to eliminate \q{unnecessary} macros: \IT{-fno-asynchronous-unwind-tables}}:

\lstinputlisting[caption=GCC 4.7.3,style=customasmx86]{patterns/01_helloworld/GCC_refined.s}

\myindex{\ATTSyntax}
\myindex{\IntelSyntax}
Some of the major differences between Intel and AT\&T syntax are:

\begin{itemize}

\item
Source and destination operands are written in opposite order.

In Intel-syntax: <instruction> <destination operand> <source operand>.

In AT\&T syntax: <instruction> <source operand> <destination operand>.

\myindex{\CStandardLibrary!memcpy()}
\myindex{\CStandardLibrary!strcpy()}
Here is an easy way to memorize the difference:
when you deal with Intel-syntax, you can imagine that there is an equality sign ($=$) between operands
and when you deal with AT\&T-syntax imagine there is a right arrow ($\rightarrow$)
\footnote{By the way, in some C standard functions (e.g., memcpy(), strcpy()) the arguments
are listed in the same way as in Intel-syntax: first the pointer to the destination memory block, and then
the pointer to the source memory block.}.

\item
AT\&T: Before register names, a percent sign must be written (\%) and before numbers a dollar sign (\$).
Parentheses are used instead of brackets.

\item
AT\&T: A suffix is added to instructions to define the operand size:

\begin{itemize}
\item q --- quad (64 bits)
\item l --- long (32 bits)
\item w --- word (16 bits)
\item b --- byte (8 bits)
\end{itemize}

% TODO1 simple example may be? \RU{Например mov\textbf{l}, movb, movw представляют различые версии инсструкция mov} \EN {For example: movl, movb, movw are variations of the mov instruction}

\end{itemize}

To go back to the compiled result: it is almost identical to what was displayed by \IDA.
There is one subtle difference: \TT{0FFFFFFF0h} is presented as \TT{\$-16}.
It's the same thing: \TT{16} in the decimal system is \TT{0x10} in hexadecimal.
\TT{-0x10} is equal to \TT{0xFFFFFFF0} (for a 32-bit data type).

\myindex{x86!\Instructions!MOV}
One more thing: the return value is set to 0 by using the usual \MOV, not \XOR.
\MOV just loads a value to a register.
Its name is a misnomer (as the data is not moved but rather copied). In other architectures, this instruction is named \q{LOAD} or \q{STORE} or something similar.
}
\RU{\subsubsection{GCC}

Теперь скомпилируем то же самое компилятором GCC 4.4.1 в Linux: \TT{gcc 1.c -o 1}.
Затем при помощи \IDA посмотрим как скомпилировалась функция \main.
\IDA, как и MSVC, показывает код в синтаксисе Intel\footnote{Мы также можем заставить GCC генерировать листинги в этом формате при помощи ключей \TT{-S -masm=intel}.}.

\begin{lstlisting}[caption=код в \IDA,style=customasmx86]
main            proc near

var_10          = dword ptr -10h

                push    ebp
                mov     ebp, esp
                and     esp, 0FFFFFFF0h
                sub     esp, 10h
                mov     eax, offset aHelloWorld ; "hello, world\n"
                mov     [esp+10h+var_10], eax
                call    _printf
                mov     eax, 0
                leave
                retn
main            endp
\end{lstlisting}

\myindex{Function prologue}
\myindex{x86!\Instructions!AND}
Почти то же самое. 
Адрес строки \TT{hello, world}, лежащей в сегменте данных, вначале сохраняется в \EAX, затем записывается в стек.
А ещё в прологе функции мы видим \TT{AND ESP, 0FFFFFFF0h}~--- 
эта инструкция выравнивает значение в \ESP по 16-байтной границе, делая все значения 
в стеке также выровненными по этой границе (процессор более эффективно работает с переменными, расположенными
в памяти по адресам кратным 4 или 16)\footnote{\URLWPDA}.

\myindex{x86!\Instructions!SUB}
\INS{SUB ESP, 10h} выделяет в стеке 16 байт. Хотя, как будет видно далее, здесь достаточно только 4.

Это происходит потому, что количество выделяемого места в локальном стеке тоже выровнено по 16-байтной границе.

% TODO1: rewrite.
\myindex{x86!\Instructions!PUSH}
Адрес строки (или указатель на строку) затем записывается прямо в стек без помощи инструкции \PUSH.
\IT{var\_10} одновременно и локальная переменная и аргумент для \printf{}. Подробнее об этом будет ниже.

Затем вызывается \printf.

В отличие от MSVC, GCC в компиляции без включенной оптимизации генерирует \TT{MOV EAX, 0} вместо более короткого опкода.

\myindex{x86!\Instructions!LEAVE}
Последняя инструкция \LEAVE~--- это аналог команд \TT{MOV ESP, EBP} и \TT{POP EBP}~--- то есть возврат \glslink{stack pointer}{указателя стека} и регистра \EBP в первоначальное состояние.
Это необходимо, т.к. в начале функции мы модифицировали регистры \ESP и \EBP{}\\
(при помощи \INS{MOV EBP, ESP} / \INS{AND ESP, \ldots}).

\subsubsection{GCC: \ATTSyntax}
\label{ATT_syntax}

Попробуем посмотреть, как выглядит то же самое в синтаксисе AT\&T языка ассемблера.
Этот синтаксис больше распространен в UNIX-мире.

\begin{lstlisting}[caption=компилируем в GCC 4.7.3]
gcc -S 1_1.c
\end{lstlisting}

Получим такой файл:

\lstinputlisting[caption=GCC 4.7.3,style=customasmx86]{patterns/01_helloworld/GCC.s}

Здесь много макросов (начинающихся с точки). Они нам пока не интересны.

Пока что, ради упрощения, мы можем 
их игнорировать (кроме макроса \IT{.string}, при помощи которого кодируется последовательность символов, 
оканчивающихся нулем~--- такие же строки как в Си). И тогда получится следующее
\footnote{Кстати, для уменьшения генерации \q{лишних} макросов, можно использовать такой ключ GCC: \IT{-fno-asynchronous-unwind-tables}}:

\lstinputlisting[caption=GCC 4.7.3,style=customasmx86]{patterns/01_helloworld/GCC_refined.s}

\myindex{\ATTSyntax}
\myindex{\IntelSyntax}
Основные отличия синтаксиса Intel и AT\&T следующие:

\begin{itemize}

\item
Операнды записываются наоборот.

В Intel-синтаксисе: \\
<инструкция> <операнд назначения> <операнд-источник>.

В AT\&T-синтаксисе: \\
<инструкция> <операнд-источник> <операнд назначения>.

\myindex{\CStandardLibrary!memcpy()}
\myindex{\CStandardLibrary!strcpy()}
Чтобы легче понимать разницу, можно запомнить следующее:
когда вы работаете с синтаксисом Intel~--- можете в уме ставить знак равенства ($=$) между операндами,
а когда с синтаксисом AT\&T~--- мысленно ставьте стрелку направо ($\rightarrow$)
\footnote{Кстати, в некоторых стандартных функциях библиотеки Си (например, memcpy(), strcpy()) также применяется 
расстановка аргументов как в синтаксисе Intel: вначале указатель в памяти на блок назначения, 
затем указатель на блок-источник.}.

\item
AT\&T: Перед именами регистров ставится символ процента (\%), а перед числами символ доллара (\$).
Вместо квадратных скобок используются круглые.

\item
AT\&T: К каждой инструкции добавляется специальный символ, определяющий тип данных:

\begin{itemize}
\item q --- quad (64 бита)
\item l --- long (32 бита)
\item w --- word (16 бит)
\item b --- byte (8 бит)
\end{itemize}

% TODO1 simple example may be? \RU{Например mov\textbf{l}, movb, movw представляют различые версии инсструкция mov} \EN {For example: movl, movb, movw are variations of the mov instruciton}

\end{itemize}

Возвращаясь к результату компиляции: он идентичен тому, который мы посмотрели в \IDA.
Одна мелочь: \TT{0FFFFFFF0h} записывается как \TT{\$-16}.
Это то же самое: \TT{16} в десятичной системе это \TT{0x10} в шестнадцатеричной.
\TT{-0x10} будет как раз \TT{0xFFFFFFF0} (в рамках 32-битных чисел).

\myindex{x86!\Instructions!MOV}
Возвращаемый результат устанавливается в 0 обычной инструкцией \MOV, а не \XOR.
\MOV просто загружает значение в регистр.
Её название не очень удачное (данные не перемещаются, а копируются). В других архитектурах подобная инструкция обычно носит название \q{LOAD} или \q{STORE} или что-то в этом роде.

}
\NL{\subsubsection{GCC}

Nu zullen we dezelfde \CCpp code compileren in de GCC 4.4.1 compiler in Linux: \TT{gcc 1.c -o 1}.
Vervolgens, met de assistentie van de \IDA disassembler, zullen we kijken hoe de \main functie gemaakt is.
\IDA, maakt net als MSVC gebruik van de Intel-syntax\footnote{We hadden GCC ook assembly listings kunnen laten gereren in Intel-syntax door gebruik te maken van de opties \TT{-S -masm=intel}.}.

\begin{lstlisting}[caption=code in \IDA,style=customasmx86]
main            proc near

var_10          = dword ptr -10h

                push    ebp
                mov     ebp, esp
                and     esp, 0FFFFFFF0h
                sub     esp, 10h
                mov     eax, offset aHelloWorld ; "hello, world\n"
                mov     [esp+10h+var_10], eax
                call    _printf
                mov     eax, 0
                leave
                retn
main            endp
\end{lstlisting}

\myindex{Function prologue}
\myindex{x86!\Instructions!AND}
Het resultaat is bijna hetzelfde.
Het adres van de \TT{hello, world} string (opgeslagen in het data segment) wordt eerst ingeladen in het \EAX register en wordt daarna opgeslagen op de stack.
Daarbovenop vind je in de functie proloog hetvolgende terug: \TT{AND ESP, 0FFFFFFF0h}~---
deze instructie lijnt de \ESP registerwaarde uit op een 16-byte begrenzing.
Dit resulteert in het feit dat alle waarden op de stack op dezelfde manier uitgelijnd worden.
De CPU presteert beter als de waarden die hij moet behandelen gelokaliseerd zijn in het geheugen op adressen die gealigneerd zijn op een 4-byte of 16-byte begrenzing.\footnote{URLWPDA}.

\myindex{x86!\Instructions!SUB}
\INS{SUB ESP, 10h} reserveert 16 bytes op de stack. Zoals we hierna echter kunnen zijn, zijn er in dit geval slechts 4 nodig.

Dit komt doordat de grootte van de gereserveerde stack ook uitgelijnd is op een 16-byte begrenzing.

% TODO1: rewrite.
\myindex{x86!\Instructions!PUSH}
Het string adres (of een pointer naar de string) wordt dan rechtstreeks op de stack geplaatst zonder gebruik te maken van de \PUSH instructie.
\IT{var\_10}~---is een lokale variabele en is ook een argument voor \printf{}.
Lees er hieronder meer over.

\NLph{}

In tegenstelling tot MSVC, wanneer GCC compileert zonder optimizatie, maakt het gebruik van \TT{MOV EAX, 0} in plaats van kortere opcodes.

\myindex{x86!\Instructions!LEAVE}
De laatste instructie, \LEAVE~---is het equivalent van het \TT{MOV ESP, EBP} en \TT{POP EBP} instructiepaar.
Met andere woorden, deze instructie zet de \gls{stack pointer} (\ESP) terug, en herstelt het \EBP register
terug tot zijn oorspronkelijke staat.
Dit is nodig aangezien we deze registerwaarden hebben gewijzigd (\ESP en \EBP) in het begin van de functie (door het uitvoeren van \INS{MOV EBP, ESP} / \INS{AND ESP, \ldots}).

\subsubsection{GCC: \ATTSyntax}
\label{ATT_syntax}

Laat ons eens kijken hoe dit kan weergegeven worden in assembly in de AT\&T syntax.
Deze syntax is veel populairder in de UNIX-wereld.

\begin{lstlisting}[caption=\NLph{} GCC 4.7.3]
gcc -S 1_1.c
\end{lstlisting}

We krijgen dit resultaat:

\lstinputlisting[caption=GCC 4.7.3,style=customasmx86]{patterns/01_helloworld/GCC.s}

De lijst bevat vele macros (die beginnen met een punt). Maar deze zijn niet interessant voor ons momenteel.

Voorlopig, om het simpel te houden, kunnen we deze negeren (buiten de \IT{.string} macro, dewelke
een null-terminated karakter reeks encodeert net als een C-string). Daarna zien we dit
\footnote{Deze GCC optie kan gebruikt worden om alle \q{onnodige} macros te elimineren: \IT{-fno-asynchronous-unwind-tables}}:

\lstinputlisting[caption=GCC 4.7.3,style=customasmx86]{patterns/01_helloworld/GCC_refined.s}

\myindex{\ATTSyntax}
\myindex{\IntelSyntax}
Sommige grote verschillen tussen de Intel en AT\&T syntax zijn:

\begin{itemize}

\item
\NLph{}

In Intel-syntax: <instructie> <doel> <bron>.

In AT\&T syntax: <instructie> <bron> <doel>.

\myindex{\CStandardLibrary!memcpy()}
\myindex{\CStandardLibrary!strcpy()}
Een gemakkelijke manier om dit verschil te onthouden is: 
Wanneer je met Intel-syntax te doen krijgt, kan je je inbeelden dat er een gelijkheidsteken ($=$) staat tussen de operands
en met AT\&T-syntax beeld je je in dat er een pijl naar rechts staat ($\rightarrow$)
\footnote{Trouwens, in sommige C standaard functies (bv. memcpy(), strcpy()) worden
de argumenten opgelijst op dezelfde manier als in Intel-syntax: eerst een pointer naar het bestemmings geheugen block, 
gevolgd door een pointer naar de bron.}.

\item
AT\&T: Voor registernamen moet een percentteken geschreven worden (\%) en voor cijfers een dollarteken (\$).
Ronde haakjes worden gebruikt in plaats van haakjes.

\item
AT\&T: Een suffix wordt toegevoegd aan de instructies om de operand grootte te bepalen:

\begin{itemize}
\item q --- quad (64 bits)
\item l --- long (32 bits)
\item w --- word (16 bits)
\item b --- byte (8 bits)
\end{itemize}

\end{itemize}

Laten we even terugblikken op het gecompileerde resultaat: dit is identiek als wat we gezien hebben in \IDA.
Met een klein verschil: \TT{0FFFFFFF0h} wordt weergegeven als \TT{\$-16}.
Dit is hetzelfde: \TT{16} in het decimaalsysteem is \TT{0x10} in hexadecimal.
\TT{-0x10} is gelijk aan \TT{0xFFFFFFF0} (voor een 32-bit data type).

\myindex{x86!\Instructions!MOV}
Nog een ding: de return value wordt best op 0 gezet door gebruik te maken van \MOV, niet van \XOR.
\MOV laadt gewoon een waarde in het register.
De naam is een foute noemer (data wordt niet verplaatst, maar eerder gekopieerd). In andere architecturen wordt deze instructie \q{LOAD} of \q{STORE} of iets soortgelijks genoemd.

}
\ITA{\subsubsection{GCC}

Proviamo adesso a compilare lo stesso codice \CCpp con il compilatore GCC 4.4.1 su Linux: \TT{gcc 1.c -o 1}.
Successivamente, con l'aiuto del disassembler \IDA, vediamo come è stata creata la funzione \main .
\IDA, come MSVC, utilizza la sintassi Intel\footnote{Possiamo anche fare in modo che GCC produca un listato assembly con la sintassi Intel tramite l'opzione \TT{-S -masm=intel}.}.

\begin{lstlisting}[caption=codice in \IDA,style=customasmx86]
main            proc near

var_10          = dword ptr -10h

                push    ebp
                mov     ebp, esp
                and     esp, 0FFFFFFF0h
                sub     esp, 10h
                mov     eax, offset aHelloWorld ; "hello, world\n"
                mov     [esp+10h+var_10], eax
                call    _printf
                mov     eax, 0
                leave
                retn
main            endp
\end{lstlisting}

\myindex{Function prologue}
\myindex{x86!\Instructions!AND}
Il risultato è pressoché lo stesso.
L'indirizzo della stringa \TT{hello, world} (memorizzato nel data segment) è caricato prima nel registro \EAX e successivamente salvato sullo stack.
Inoltre, il prologo della funzione contiene \TT{AND ESP, 0FFFFFFF0h}~---questa 
istruzione allinea il valore del registro \ESP a 16-byte.
Ciò fa sì che tutti i valori sullo stack siano allineati allo stesso modo (la CPU è più efficiente se i valori che tratta sono collocati in memoria ad indirizzi allineati a, ovvero multipli di, 4 o 16 byte)\footnote{\URLWPDA}.

\myindex{x86!\Instructions!SUB}
\INS{SUB ESP, 10h} alloca 16 byte sullo stack. Tuttavia, come vedremo a breve, solo 4 sono necessari in questo caso.

Ciò è dovuto al fatto che la dimensione dello stack allocato è anch'essa allineata a 16 byte.

% TODO1: rewrite.
\myindex{x86!\Instructions!PUSH}
L'indirizzo della stringa (o un puntatore alla stringa) è quindi memorizzato direttamente sullo stack senza utilizzare l'istruzione \PUSH .
\IT{var\_10}~--- è una variabile locale ed è anche un argomento di \printf{}.
Maggiori dettagli in seguito.

Infine viene chiamata la funzione \printf.

Diversamente da MSVC, quando GCC compila senza ottimizzazione emette \TT{MOV EAX, 0} invece di un opcode più breve.

\myindex{x86!\Instructions!LEAVE}
L'ultima istruzione, \LEAVE~---è l'equivalente della coppia di istruzioni \TT{MOV ESP, EBP} e \TT{POP EBP} ~---in altre parole, questa istruzione riporta indietro lo \gls{stack pointer} (\ESP) e ripristina il registro \EBP al suo stato iniziale.
Ciò è necessario poiché abbiamo modificato i valori di questi registri (\ESP and \EBP) all'inizio della funzione ( eseguendo \INS{MOV EBP, ESP} / \INS{AND ESP, \ldots}).

\subsubsection{GCC: \ATTSyntax}
\label{ATT_syntax}

Vediamo come tutto questo può essere rappresentato nella sintassi assembly AT\&T.
Questa sintassi è molto più popolare nel mondo UNIX.

\begin{lstlisting}[caption=compiliamo in GCC 4.7.3]
gcc -S 1_1.c
\end{lstlisting}

Otteniamo questo:

\lstinputlisting[caption=GCC 4.7.3,style=customasmx86]{patterns/01_helloworld/GCC.s}

Il listato contiene molte macro (iniziano con il punto). Per il momento non ci interessano.

Per il momento, e solo per una questione di semplificazione, possiamo ignorarle (fatta eccezione per la macro \IT{.string} che codifica una sequenza di caratteri che termina con il null-byte (zero) proprio come una stringa C). Consideriamo soltanto questo
\footnote{Questa opzione di GCC può essere usata per eliminare le macro \q{superflue}: \IT{-fno-asynchronous-unwind-tables}}:

\lstinputlisting[caption=GCC 4.7.3,style=customasmx86]{patterns/01_helloworld/GCC_refined.s}

\myindex{\ATTSyntax}
\myindex{\IntelSyntax}
Alcune delle differenze maggiori tra la sintassi Intel e quella AT\&T sono:

\begin{itemize}

\item
\ITAph{}

Sintassi Intel: <istruzione> <operando di destinazione> <operando di origine>.

Sintassi AT\&T: <istruzione> <operando di origine> <operando di destinazione>.

\myindex{\CStandardLibrary!memcpy()}
\myindex{\CStandardLibrary!strcpy()}
Ecco un modo facile per memorizzare la differenza:
quando si tratta di sintassi Intel immagina che ci sia un segno di uguaglianza ($=$) tra i due operandi, quando si tratta di sintassi AT\&T immagina una freccia da sinistra a destra ($\rightarrow$)
\footnote{A proposito, in alcune funzioni standard C(es., memcpy(), strcpy()) gli argomenti sono elencati nello stesso modo della sintassi Intel: prima il puntatore al blocco di memoria di destinazione, e poi il puntatore al blocco di memoria di origine.}.

\item
AT\&T: Il simbolo di percentuale (\%) deve essere scritto prima del nome di un registro, e il dollaro (\$) prima dei numeri.

\item
AT\&T: All'istruzione si aggiunge un suffisso che definisce le dimensioni dell'operando:

\begin{itemize}
\item q --- quad (64 bit)
\item l --- long (32 bit)
\item w --- word (16 bit)
\item b --- byte (8 bit)
\end{itemize}

\end{itemize}

Torniamo al risultato compilato: è identico a quello che abbiamo visto in \IDA.
Con una piccola differenza: \TT{0FFFFFFF0h} è presentato come \TT{\$-16}.
E' la stessa cosa: \TT{16} nel sistema decimale è \TT{0x10} in esadecimale.
\TT{-0x10} è uguale a \TT{0xFFFFFFF0} (per un tipo di dato a 32-bit).

\myindex{x86!\Instructions!MOV}
Ancora una cosa: il valore di ritorno viene settato a 0 usando \MOV, non \XOR.
\MOV semplicemente carica un valore in un registro.
Il suo nome è fuorviante (il dato non viene spostato, bensì copiato). In altre architectures questa istruzione è chiamata \q{LOAD} o \q{STORE} o qualcosa di simile.

}
\DE{\subsubsection{GCC}

Als nächstes wird der gleiche \CCpp-Code mit GCC 4.4.1 unter Linux kompiliert: \TT{gcc 1.c -o 1}.
Mithilfe des \IDA-Disassemblers wird untersucht, wie die \main-Funktion erzeugt wurde.
\IDA nutzt, genau wie MSVX den Intel-Syntax\footnote{GCC kann Assembler-Ausgaben im Intel-Syntax erzeugen mit der Options \TT{-S -masm=intel}.}.

\begin{lstlisting}[caption=Code in \IDA,style=customasmx86]
main            proc near

var_10          = dword ptr -10h

                push    ebp
                mov     ebp, esp
                and     esp, 0FFFFFFF0h
                sub     esp, 10h
                mov     eax, offset aHelloWorld ; "hello, world\n"
                mov     [esp+10h+var_10], eax
                call    _printf
                mov     eax, 0
                leave
                retn
main            endp
\end{lstlisting}

\myindex{Function prologue}
\myindex{x86!\Instructions!AND}
Das Ergebnis ist fast das gleiche.
Die Adresse der \TT{hello, world}-Zeichenkette (im Daten-Segment) wird zunächst in das \EAX-Register geladen und anschließend auf dem Stack gesichert.\\
Zusätzlich beinhaltet der Funktions-Prolog \INS{AND ESP, 0FFFFFFF0h}~---diese
Anweisung richtet den \ESP-Register-Wert an eine 16-Byte-Grenze aus.
Dies führt dazu, dass alle Werte im Stack auf die gleiche Weise ausgerichtet sind.
Die CPU kann Anweisungen schneller ausführen, wenn die zu verarbeitenden Daten auf einer an 4- oder 16-Byte-Grenzen ausgerichteten Adresse liegen\footnote{\URLWPDA}.

\myindex{x86!\Instructions!SUB}
\INS{SUB ESP, 10h} reserviert 16 Byte auf dem Stack, auch wenn - wie später gezeigt wird - nur 4 Byte benötigt werden.

Der Grund liegt darin, dass auch die Größe des Stacks an eine 16-Byte-Grenze ausgerichtet ist.

% TODO1: rewrite.
\myindex{x86!\Instructions!PUSH}
Die Adresse der Zeichenkette (oder ein Zeiger darauf) wird anschließend direkt ohne die \PUSH-Anweisung auf dem Stack gespeichert.
IT{var\_10}~---ist eine lokale Variable und ein Argument für \printf{}.
Mehr dazu später.

Anschließend wird die \printf-Funktion aufgerufen.

Anders als MSVC erzeugt GCC ohne Optimierung Die Anweisung \TT{MOV EAX, 0} anstatt des kürzeren OpCodes.

\myindex{x86!\Instructions!LEAVE}
Die letzte Anweisung \LEAVE ist ein Äquivalent zu der Kombination aus \TT{MOV ESP, EBP} und \TT{POP EBP}.
Mit anderen Worten: diese Anweisung setzt den \gls{stack pointer} (\ESP) zurück und stellt die initalen Werte des \EBP-Registers wieder her.
Dies ist notwendig weil die Registerwerte (\ESP und \EBP) zu Beginn der Funktion (durch \INS{MOV EBP, ESP} / \INS{AND ESP, \ldots}).

\subsubsection{GCC: \ATTSyntax}
\label{ATT_syntax}

Im nächsten Beispiel ist sichtbar, wie dies im AT\%T-Syntax dargestellt werden kann.
Dieser Syntax ist sehr viel populärer in der UNIX-Welt.

\begin{lstlisting}[caption=Das Beispiel kompiliert mit GCC 4.7.3]
gcc -S 1_1.c
\end{lstlisting}

Das Ergebnis ist wie folgt:

\lstinputlisting[caption=GCC 4.7.3,style=customasmx86]{patterns/01_helloworld/GCC.s}

Der Quellcode beinhaltet Makros (beginnend mit einem Punkt), die hier aber nicht von Belang sind.

An dieser Stelle werden aus Gründen der Übersichtlichkeit alle Makros au0er \IT{.string}
ignoriert. Letzeres kodiert eine Null-terminierte Zeichenkette, die einem C-String entspricht.

Die resultierende Ausgabe ist diese
\footnote{Um die \q{unnötigen} Makros zu unterdrücen kann die GCC-Option \IT{-fno-asynchronous-unwind-tables} genutzt werden}:

\lstinputlisting[caption=GCC 4.7.3,style=customasmx86]{patterns/01_helloworld/GCC_refined.s}

\myindex{\ATTSyntax}
\myindex{\IntelSyntax}
Einige der Hauptunterschiede zwischen Intel und AT\&T-Syntax sin:

\begin{itemize}

\item
Quell- und Zieloperanden sind in umgekehrter Reihenfolge angegeben.

Im Intel-Syntax: <Anweisung> <Ziel-Operand> <Quell-Operand>.

Im AT\&T-Syntax: <Anweisung> <Quell-Operand> <Ziel-Operand>.

\myindex{\CStandardLibrary!memcpy()}
\myindex{\CStandardLibrary!strcpy()}
Hier ist eine einfache Möglichkeit um sich den Unterschied zu merken:
Beim Umgang mit dem Intel-Syntax, kann man sich ein Gleichheitszeichen ($=$) zwischen den Operanden vorstellen
und beim AT\&T-Syntax einen Pfeil nach rechts ($\rightarrow$)
\footnote{Einige C-Standard-Funktionen (z.B. memcpy(), strcpy()) sind die Parameter ebenfalls wie im
Intel-Syntax aufgelistet: erst der Zeiger zum Ziel, dann der Zeiger auf die Speicher-Quelle)}.

\item
AT\&T: Vor einem Register-Namen muss ein Prozentzeichen (\%) und vor Zahlen ein Dollarzeichen (\$) stehen.
Statt eckigen werden runde Klammern genutzt.

\item
AT\&T: An eine Anweisung ist ein Suffix angehängt, der die Operandengröße angibt:

\begin{itemize}
\item q --- quad (64 bits)
\item l --- long (32 bits)
\item w --- word (16 bits)
\item b --- byte (8 bits)
\end{itemize}

% TODO1 simple example may be? \RU{Например mov\textbf{l}, movb, movw представляют различые версии инсструкция mov} \EN {For example: movl, movb, movw are variations of the mov instruciton} \DE {Zum Beispiel sind movl, movb und movw Variationen der mov-Anweisung}

\end{itemize}

Nochmals zu dem kompilierten Ergebnis: Dieses ist identisch mit der Anzeige in \IDA,
jedoch mit einem kleinen Unterschied: \TT{0FFFFFFF0h} wird als \TT{\$-16} angezeigt.
Der eigentliche Wert ist der selbe: \TT{16} im Dezimalsystem ist \TT{0x10} im Hexadezimalsystem.
Für 32-Bit-Datentypen ist \TT{-0x10} identisch mit \TT{0xFFFFFFF0}.

\myindex{x86!\Instructions!MOV}
Eine weitere Sache: der Rückgabewert ist mittels \MOV auf Null gesetzt, nicht mit \XOR.
\MOV läd lediglich einen Wert in ein Register.
Der Name ist irreführend, da die Daten nicht verschoben, sondern kopiert werden.
In anderen Architekturen ist wird dieser Befehl \q{LOAD} oder \q{STORE} oder ähnlich genannt.
}

\subsubsection{String patching (Win32)}

We can easily find ``hello, world'' string in executable file using Hiew:

\begin{figure}[H]
\centering
\myincludegraphics{patterns/01_helloworld/hola_edit1.png}
\caption{Hiew}
\label{}
\end{figure}

And we can try to translate our message to Spanish language:

\begin{figure}[H]
\centering
\myincludegraphics{patterns/01_helloworld/hola_edit2.png}
\caption{Hiew}
\label{}
\end{figure}

Spanish text is one byte shorter than English, so we also add 0x0A byte at the end (\TT{\textbackslash{}n}) and zero byte.

It works.

What if we want to insert longer message?
There are some zero bytes after original English text.
Hard to say if they can be overwritten: they may be used somewhere in \ac{CRT} code, or maybe not.
Anyway, you can overwrite them only if you really know what you are doing.

\subsubsection{String patching (Linux x64)}

\myindex{\radare}
Let's try to patch Linux x64 executable using \radare{}:

\lstinputlisting[caption=\radare{} session]{patterns/01_helloworld/radare.lst}

What I do here: I search for \q{hello} string using \TT{/} command, 
then I set \IT{cursor} (or \IT{seek} in \radare{} terms) to that address.
Then I want to be sure that this is really that place: \TT{px} dumps bytes there.
\TT{oo+} switches \radare{} to \IT{read-write} mode.
\TT{w} writes ASCII string at the current \IT{seek}.
Note \TT{\textbackslash{}00} at the end---this is zero byte.
\TT{q} quits.

\subsubsection{Software \IT{localization} of MS-DOS era}

The way I described was a common way to translate MS-DOS software to Russian language back to 1980's and 1990's.
Russian words and sentences are usually slightly longer than its English counterparts, so that is why \IT{localized}
software has a lot of weird acronyms and hardly readable abbreviations.

Perhaps, this also happened to other languages during that era.



\subsection{x86-64}
\EN{\subsubsection{MSVC: x86-64}

\myindex{x86-64}
Let's also try 64-bit MSVC:

\lstinputlisting[caption=MSVC 2012 x64,style=customasmx86]{patterns/01_helloworld/MSVC_x64.asm}

\myindex{fastcall}

In x86-64, all registers were extended to 64-bit and now their names have an \TT{R-} prefix.
In order to use the stack less often (in other words, to access external memory/cache less often), there exists
a popular way to pass function arguments via registers (\IT{fastcall}) \myref{fastcall}.
I.e., a part of the function arguments is passed in registers, the rest---via the stack.
In Win64, 4 function arguments are passed in the \RCX, \RDX, \Reg{8}, \Reg{9} registers.
That is what we see here: a pointer to the string for \printf is now passed not in the stack, but in the \RCX register.
The pointers are 64-bit now, so they are passed in the 64-bit registers (which have the \TT{R-} prefix).
However, for backward compatibility, it is still possible to access the 32-bit parts, using the \TT{E-} prefix.
This is how the \RAX/\EAX/\AX/\AL register looks like in x86-64:

\RegTableOne{RAX}{EAX}{AX}{AH}{AL}

The \main function returns an \Tint{}-typed value, which is, in \CCpp, for better backward compatibility
and portability, still 32-bit, so that is why the \EAX register is cleared at the function end (i.e., the 32-bit
part of the register) instead of \RAX{}.
There are also 40 bytes allocated in the local stack.
This is called the \q{shadow space}, about which we are going to talk later: \myref{shadow_space}.
}
\ITA{\subsubsection{MSVC: x86-64}

\myindex{x86-64}
Proviamo anche con MSVC a 64-bit:

\lstinputlisting[caption=MSVC 2012 x64,style=customasmx86]{patterns/01_helloworld/MSVC_x64.asm}

\myindex{fastcall}

In x86-64, tutti i registri sono stati estesi a 64-bit ed il loro nome ha il prefisso \TT{R-}.
Per usare lo stack meno spesso (in altre parole, per accedere meno spesso alla memoria esterna/cache), esiste un metodo molto diffuso per passare gli argomenti delle funzioni tramite i registri (\IT{fastcall})
\myref{fastcall}.
Ovvero, una parte degli argomenti è passata attraverso i registri, il resto ---attraverso lo stack.
In Win64, 4 argpmenti di funzione sono passati nei registri \RCX, \RDX, \Reg{8}, \Reg{9}.
Questo è ciò che vediamo qui: un puntatore alla stringa per \printf è adesso passato nel registro \RCX anziché tramite lo stack.
I puntatori adesso sono a 64-bit , quindi sono passati nei registri a 64-bit (aventi il prefisso \TT{R-}).
E' comunque possibile, per retrocompatibilità, accedere alle parti a 32-bit parts, usando il prefisso \TT{E-}.
I registri \RAX/\EAX/\AX/\AL in x86-64 appaiono così:

\RegTableOne{RAX}{EAX}{AX}{AH}{AL}

La funzione \main restituisce un valore di tipo \Tint{}, che in \CCpp, per migliore retrocompatibilità e portabilità, resta ancora a 32-bit, motivo per cui il registro \EAX viene svuotato invece di \RAX{} alla fine della funzione (i.e., la parte a 32-bit
del registro).
Ci sono anche 40 byte allocati nello stack locale.
Questo spazio è detto \q{shadow space}, di cui parleremo più avanti: \myref{shadow_space}.

}
\NL{\subsubsection{MSVC: x86-64}

\myindex{x86-64}
Laat ons ook eens kijken naar 64-bit MSVC:

\lstinputlisting[caption=MSVC 2012 x64,style=customasmx86]{patterns/01_helloworld/MSVC_x64.asm}

\myindex{fastcall}

In x86-64 zijn alle registers uitgebreid tot 64-bit, en hebben hun namen een \TT{R-} prefix gekregen.
Om de stack minder te gebruiken (met andere woorden, om het externe geheugen/cache minder vaak te benaderen), bestaat
er een populaire manier om functies parameters door te geven via registers (\IT{fastcall}) \myref{fastcall}.
Bv., een deel van de parameters wordt doorgegeven via het register, de rest --- via de stack.
In Win64, worden 4 functie parameters doorgegeven via de \RCX, \RDX, \Reg{8}, \Reg{9} registers.
Dat is wat we hier zien: een pointer naar de string voor \printf wordt doorgegeven, niet via de stack, maar via het \RCX register.
De pointers zijn 64-bit nu, dus worden ze doorgegeven in de 64-bit registers (dewelke de \TT{R-} prefix hebben).
Voor backward compatibility is het echter nog steeds mogelijk om de 32-bit gedeelten aan te spreken, door gebruik te maken van de \TT{E-} prefix.
Dit is hoe de \RAX/\EAX/\AX/\AL registers eruit zien in x86-64:

\RegTableOne{RAX}{EAX}{AX}{AH}{AL}

De \main functie geeft een \Tint{}-typed waarde terug, hetwelk, in \CCpp, voor betere backward compatibiliteit
en portabiliteit, nog steeds 32-bit is. Daarom wordt het \EAX register ook leeggemaakt bij het einde van de functie
(het 32-bit gedeelte van het register) in plaats van \RAX{}.
Er zijn ook 40 bytes gealloceerd op de lokale stack.
Dit wordt de \q{shadow space} genoemd, waarover we het later nog gaan hebben: \myref{shadow_space}.

}
\RU{\subsubsection{MSVC: x86-64}

\myindex{x86-64}
Попробуем также 64-битный MSVC:

\lstinputlisting[caption=MSVC 2012 x64,style=customasmx86]{patterns/01_helloworld/MSVC_x64.asm}

\myindex{fastcall}

В x86-64 все регистры были расширены до 64-х бит и теперь имеют префикс \TT{R-}.
Чтобы поменьше задействовать стек (иными словами, поменьше обращаться кэшу и внешней памяти), уже давно имелся
довольно популярный метод передачи аргументов функции через регистры (\IT{fastcall}) \myref{fastcall}.
Т.е. часть аргументов функции передается через регистры и часть ---через стек.
В Win64 первые 4 аргумента функции передаются через регистры \RCX, \RDX, \Reg{8}, \Reg{9}.
Это мы здесь и видим: указатель на строку в \printf теперь передается не через стек, а через регистр \RCX.
Указатели теперь 64-битные, так что они передаются через 64-битные части регистров (имеющие префикс \TT{R-}).
Но для обратной совместимости можно обращаться и к нижним 32 битам регистров используя префикс \TT{E-}.
Вот как выглядит регистр \RAX/\EAX/\AX/\AL в x86-64:

\RegTableOne{RAX}{EAX}{AX}{AH}{AL}

Функция \main возвращает значение типа \Tint, который в \CCpp, надо полагать, для лучшей совместимости и переносимости,
оставили 32-битным. Вот почему в конце функции \main обнуляется не \RAX, а \EAX, т.е. 32-битная часть регистра.
Также видно, что 40 байт выделяются в локальном стеке.
Это \q{shadow space} которое мы будем рассматривать позже: \myref{shadow_space}.
}
\PTBR{\subsubsection{MSVC: x86-64}

\myindex{x86-64}
Vamos tentar também o MSVC 64-bits:

\lstinputlisting[caption=MSVC 2012 x64,style=customasmx86]{patterns/01_helloworld/MSVC_x64.asm}

\myindex{fastcall}

No x86-64, todos os registradores foram extendidos para 64-bits e agora seus nomes contém um \TT{R-} no prefixo.
A fim de diminuir a frequência com que a stack (pilha) é usada (em outras palavras, para acessar memória externa/cache menos vezes),
existe uma maneira popular de passar argumentos para funções através dos registradores (\IT{fastcall}) \myref{fastcall}.
Por exemplo, uma parte dos argumentos da função é passada nos registradores, o resto pela stack.
No Win64, 4 argumentos de funções são passados através dos registradores \RCX, \RDX, \Reg{8}, \Reg{9}.
Que é o que nós vemos, um ponteiro para a string para o printf() não é passado pela stack, mas no registrador \RCX.
Os ponteiros são 64-bits agora, então, eles são passados através dos registradores de 64-bits (que tem prefixo \TT{R-}).
Entretanto, para compatibilidade, ainda é possível acessar partes de 32-bits, usando o prefixo \TT{E-}.
É assim que os registradores \RAX/\EAX/\AX/\AL se parecem no x86-64:

\RegTableOne{RAX}{EAX}{AX}{AH}{AL}

A função \main retorna um valor do tipo inteiro, que em \CCpp é melhor para compatibilidade com versões anteriores e portabilidade,
de 32-bits, por isso o registrador \EAX é limpo no final da função (a parte de 32-bits do registrador) ao invés de \RAX.
Há também 40 bytes alocados na pilha local.
Que é chamado de ``shadow space'', o qual falaremos mais tarde: \myref{shadow_space}.

}


\EN{\subsubsection{GCC: x86-64}

\myindex{x86-64}
Let's also try GCC in 64-bit Linux:

\lstinputlisting[caption=GCC 4.4.6 x64,style=customasmx86]{patterns/01_helloworld/GCC_x64_EN.s}

% I think I got the intent right on the following line - Renaissance
Linux, *BSD and \MacOSX also use a method to pass function arguments in registers. \SysVABI{}.

The first 6 arguments are passed in the \RDI, \RSI, \RDX, \RCX, \Reg{8}, and \Reg{9}  registers, and the rest---via the stack.

So the pointer to the string is passed in \EDI (the 32-bit part of the register).
Why doesn't it use the 64-bit part, \RDI?

It is important to keep in mind that all \MOV instructions in 64-bit mode that write something into the lower 32-bit register part also clear the higher 32-bits (as stated in Intel manuals: \myref{x86_manuals}).\\
I.e., the \INS{MOV EAX, 011223344h} writes a value into \RAX correctly, since the higher bits will be cleared.

If we open the compiled object file (.o), we can also see all the instructions' opcodes
\footnote{This must be enabled in \textbf{Options $\rightarrow$ Disassembly $\rightarrow$ Number of opcode bytes}}:

\lstinputlisting[caption=GCC 4.4.6 x64,style=customasmx86]{patterns/01_helloworld/GCC_x64.lst}

\label{hw_EDI_instead_of_RDI}
As we can see, the instruction that writes into \EDI at \TT{0x4004D4} occupies 5 bytes.
The same instruction writing a 64-bit value into \RDI occupies 7 bytes.
Apparently, GCC is trying to save some space.
Besides, it can be sure that the data segment containing the string will not be allocated at the addresses higher than 4\gls{GiB}.

\label{SysVABI_input_EAX}
% There isn't an ABI acronym in acronyms.tex - I figure the intent is to Application Binary Interface,
% so I put it in there (in the EN section, commented out)
We also see that the \EAX register has been cleared before the \printf function call.
This is done because according to \ac{ABI} standard mentioned above,
the number of used vector registers is to be passed in \EAX in *NIX systems on x86-64.
}
\RU{\subsubsection{GCC: x86-64}

\myindex{x86-64}
Попробуем GCC в 64-битном Linux:

\lstinputlisting[caption=GCC 4.4.6 x64,style=customasmx86]{patterns/01_helloworld/GCC_x64_RU.s}

В Linux, *BSD и \MacOSX для x86-64 также принят способ передачи аргументов функции через регистры \SysVABI.

6 первых аргументов передаются через регистры \RDI, \RSI, \RDX, \RCX, \Reg{8}, \Reg{9}, а остальные --- через стек.

Так что указатель на строку передается через \EDI (32-битную часть регистра).
Но почему не через 64-битную часть, \RDI?

Важно запомнить, что в 64-битном режиме все инструкции \MOV, записывающие что-либо в младшую 32-битную часть регистра, обнуляют старшие 32-бита (это можно найти в документации от Intel: \myref{x86_manuals}).
То есть, инструкция \INS{MOV EAX, 011223344h} корректно запишет это значение в \RAX, старшие биты сбросятся в ноль.

Если посмотреть в \IDA скомпилированный объектный файл (.o), увидим также опкоды всех инструкций
\footnote{Это нужно задать в \textbf{Options $\rightarrow$ Disassembly $\rightarrow$ Number of opcode bytes}}:

\lstinputlisting[caption=GCC 4.4.6 x64,style=customasmx86]{patterns/01_helloworld/GCC_x64.lst}

\label{hw_EDI_instead_of_RDI}
Как видно, инструкция, записывающая в \EDI по адресу \TT{0x4004D4}, занимает 5 байт.
Та же инструкция, записывающая 64-битное значение в \RDI, занимает 7 байт.
Возможно, GCC решил немного сэкономить.
К тому же, вероятно, он уверен, что сегмент данных, где хранится строка, никогда не будет расположен в адресах выше 4\gls{GiB}.

\label{SysVABI_input_EAX}
Здесь мы также видим обнуление регистра \EAX перед вызовом \printf.
Это делается потому что по упомянутому выше стандарту передачи аргументов в *NIX для x86-64 в \EAX передается количество задействованных векторных регистров.

}
\NL{\subsubsection{GCC: x86-64}

\myindex{x86-64}
Laat ons ook eens kijken naar GCC in 64-bit Linux:

% TODO translate:
\lstinputlisting[caption=GCC 4.4.6 x64,style=customasmx86]{patterns/01_helloworld/GCC_x64_EN.s}

Een methode om functieargumenten door te geven in registers wordt ook gebruikt in Linux, *BSD en \MacOSX \SysVABI.

De eerste 6 argumenten worden doorgegeven in de \RDI, \RSI, \RDX, \RCX, \Reg{8}, \Reg{9} registers, en de rest --- via de stack.

De pointer naar de string wordt dus doorgegeven via \EDI (het 32-bit gedeelte van het register).
Maar waarom gebruikt men niet het 64-bit gedeelte, \RDI?

Het is belangrijk te onthouden dat alle \MOV instructies in 64-bit modus, die iets schrijven in het onderste 32-bit gedeelte van het register, ook het bovenste 32-bit gedeelte leegmaken.
\INS{MOV EAX, 011223344h} schrijft een waarde correct weg in \RAX, aangezien de bovenste bits zullen worden leeggemaakt.

Als we het gecompileerde object-bestand (.o) openen, kunnen we ook de opcodes zien van alle instructies
\footnote{Dit moet ook geactiveerd worden in \textbf{Options $\rightarrow$ Disassembly $\rightarrow$ Number of opcode bytes}}:

\lstinputlisting[caption=GCC 4.4.6 x64,style=customasmx86]{patterns/01_helloworld/GCC_x64.lst}

\label{hw_EDI_instead_of_RDI}
Zoals je kan zien, bezet de instructie die in \EDI schrijft op \TT{0x4004D4} 5 bytes.
Dezelfde instructie die een 64-bit waarde in \RDI schrijft, bezet 7 bytes.
Blijkbaar probeert GCC wat plaats te besparen.
Daarnaast kunnen we met zekerheid zeggen dat het data segment dat de string bevat, niet zal gealloceerd worden op de adressen hoger dan 4\gls{GiB}.

\label{SysVABI_input_EAX}
We zien ook dat het \EAX register leeggemaakt is voor de \printf functie call.
Dit wordt gedaan omdat het aantal gebruikte vector registers wordt doorgegeven in \EAX in *NIX systemen op x86-64.

}
\ITA{\subsubsection{GCC: x86-64}

\myindex{x86-64}
\ITAph{}:

% TODO: translate:
\lstinputlisting[caption=GCC 4.4.6 x64,,style=customasmx86]{patterns/01_helloworld/GCC_x64_EN.s}

Un metodo per passare argomenti di funzione nei registri usato anche in Linux, *BSD and \MacOSX è \SysVABI.

I primi 6 argomenti sono passati nei registri \RDI, \RSI, \RDX, \RCX, \Reg{8}, \Reg{9}  , ed il resto---tramite lo stack.

Quindi il puntatore alla stringa viene passato in \EDI (la parte a 32-bit del registro).
Ma perchè no nusare la parte a 64-bit \RDI?

E' importante ricordare che tutte le istruzioni \MOV in modalità 64-bit che scrivono qualcosa nella parte bassa a 32-bit di un registro, azzera anche la parte alta a 32-bits.
Ad esempio, \INS{MOV EAX, 011223344h} scrive un valore in \RAX correttamente, poichè i bit della parte alta saranno azzerati.

Se apriamo il file oggetto compilato (.o), possiamo anche vedere gli opcode di tutte le istruzioni
\footnote{Deve essere abilitato in \textbf{Options $\rightarrow$ Disassembly $\rightarrow$ Number of opcode bytes}}:

\lstinputlisting[caption=GCC 4.4.6 x64,style=customasmx86]{patterns/01_helloworld/GCC_x64.lst}

\label{hw_EDI_instead_of_RDI}
Come possiamo notare, l'istruzione che scrive dentro \EDI a \TT{0x4004D4} occupa 5 byte.
La stessa istruzione che scrive un valore a 64-bit dentro \RDI occupa 7 bytes.
Apparentemente, GCC sta cercando di risparmiare un po' di spazio.
Inoltre, può essere sicuro che il segmento dati contenente la stringa non sarà allocato ad indirizzi maggiori di 4\gls{GiB}.

\label{SysVABI_input_EAX}
Notiamo anche che il registro \EAX è stato azzerato prima della chiamata alla funzione \printf .
Ciò avviene perché il numbero dei registri vettore usati viene passato in \EAX nei sistemi *NIX x86-64.

}


\subsubsection{Address patching (Win64)}

If our example was compiled in MSVC 2013 using \TT{\textbackslash{}MD} switch
(meaning a smaller executable due to \TT{MSVCR*.DLL} file linkage), the \main function comes first, and can be easily found:

\begin{figure}[H]
\centering
\myincludegraphics{patterns/01_helloworld/hiew_incr1.png}
\caption{Hiew}
\label{}
\end{figure}

As an experiment, we can \gls{increment} address by 1:

\begin{figure}[H]
\centering
\myincludegraphics{patterns/01_helloworld/hiew_incr2.png}
\caption{Hiew}
\label{}
\end{figure}

Hiew shows \q{ello, world}.
And when we run the patched executable, this very string is printed.

\subsubsection{Pick another string from binary image (Linux x64)}

The binary file I've got when I compile our example using GCC 5.4.0 on Linux x64 box has many other text strings.
They are mostly imported function names and library names.

Run objdump to get the contents of all sections of the compiled file:

\begin{lstlisting}[basicstyle=\ttfamily, mathescape]
$\$$ objdump -s a.out

a.out:     file format elf64-x86-64

Contents of section .interp:
 400238 2f6c6962 36342f6c 642d6c69 6e75782d  /lib64/ld-linux-
 400248 7838362d 36342e73 6f2e3200           x86-64.so.2.
Contents of section .note.ABI-tag:
 400254 04000000 10000000 01000000 474e5500  ............GNU.
 400264 00000000 02000000 06000000 20000000  ............ ...
Contents of section .note.gnu.build-id:
 400274 04000000 14000000 03000000 474e5500  ............GNU.
 400284 fe461178 5bb710b4 bbf2aca8 5ec1ec10  .F.x[.......^...
 400294 cf3f7ae4                             .?z.

...
\end{lstlisting}

It's not a problem to pass address of the text string \q{/lib64/ld-linux-x86-64.so.2} to \TT{printf()}:

\begin{lstlisting}[style=customc]
#include <stdio.h>

int main()
{
    printf(0x400238);
    return 0;
}
\end{lstlisting}

It's hard to believe, but this code prints the aforementioned string.

If you would change the address to \TT{0x400260}, the \q{GNU} string would be printed.
This address is true for my specific GCC version, GNU toolset, etc.
On your system, the executable may be slightly different, and all addresses will also be different.
Also, adding/removing code to/from this source code will probably shift all addresses back or forward.


\section{GCC\EMDASH{}\EN{one more thing}\RU{ещё кое-что}}
\label{use_parts_of_C_strings}

\RU{Тот факт, что \IT{анонимная} Си-строка имеет тип}\EN{The fact that an \IT{anonymous} C-string has} 
\IT{const}\EN{ type} (\myref{string_is_const_char}), 
\RU{и тот факт, что выделенные в сегменте констант Си-строки гаратировано неизменяемые (immutable), 
ведет к интересному следствию}\EN{and
that C-strings allocated in constants segment are guaranteed to be immutable, has an interesting consequence}:
\RU{компилятор может использовать определенную часть строки}\EN{the compiler may use a specific part of the string}.

\RU{Вот простой пример}\EN{Let's try this example}:

\begin{lstlisting}
#include <stdio.h>

int f1()
{
	printf ("world\n");
}

int f2()
{
	printf ("hello world\n");
}

int main()
{
	f1();
	f2();
}
\end{lstlisting}

\RU{Среднестатистический компилятор с \CCpp (включая MSVC) выделит место для двух строк, 
но вот что делает GCC 4.8.1}%
\EN{Common \CCpp{}-compilers (including MSVC) allocate two strings, 
but let's see what GCC 4.8.1 does}:

\begin{lstlisting}[caption=GCC 4.8.1 + \RU{листинг в }IDA\EN{ listing}]
f1              proc near

s               = dword ptr -1Ch

                sub     esp, 1Ch
                mov     [esp+1Ch+s], offset s ; "world\n"
                call    _puts
                add     esp, 1Ch
                retn
f1              endp

f2              proc near

s               = dword ptr -1Ch

                sub     esp, 1Ch
                mov     [esp+1Ch+s], offset aHello ; "hello "
                call    _puts
                add     esp, 1Ch
                retn
f2              endp

aHello          db 'hello '
s               db 'world',0xa,0
\end{lstlisting}

\RU{Действительно, когда мы выводим строку}\EN{Indeed: when we print the \q{hello world} string}, 
\RU{эти два слова расположены в памяти впритык друг к другу и \puts, вызываясь из функции f2(), вообще не знает,
что эти строки разделены}\EN{these two words are positioned in memory adjacently and \puts called from f2() 
function is not aware that this string is divided}. \RU{Они и не разделены на самом деле, они разделены
только \q{виртуально}, в нашем листинге}\EN{In fact, it's not divided; it's divided only \q{virtually}, in this
listing}.

\RU{Когда}\EN{When} \puts \RU{вызывается из f1(), он использует строку}\EN{is called from f1(), it uses the} 
\q{world} \RU{плюс нулевой байт}\EN{string plus a zero byte}. \puts \RU{не знает, что там ещё есть какая-то строка
перед этой}\EN{is not aware that there is something before this string}!

\RU{Этот трюк часто используется (по крайней мере в GCC) и может сэкономить немного памяти.}
\EN{This clever trick is often used by at least GCC and can save some memory.}

\section{ARM}
\label{sec:hw_ARM}

\index{\idevices}
\index{Raspberry Pi}
\index{Xcode}
\index{LLVM}
\index{Keil}
\RU{Для экспериментов с процессором ARM я использовал несколько компиляторов:}
\EN{For my experiments with ARM processors I used several compilers:} 

\begin{itemize}
\item \RU{Популярный в embedded-среде}\EN{Popular in the embedded area} Keil Release 6/2013.

\item Apple Xcode 4.6.3 \EN{IDE} (\RU{с компилятором}\EN{with} LLVM-GCC 4.2 \EN{compiler}
\footnote{\EN{It is indeed so: Apple Xcode 4.6.3 uses open-source GCC as front-end compiler and LLVM 
code generator}\RU{Это действительно так: Apple Xcode 4.6.3 использует опен-сорсный GCC как компилятор
переднего плана и коде-генератор LLVM}}.

%\item GCC 4.8.1 (Linaro) (\RU{для}\EN{for} ARM64).
%
\item GCC 4.9 (Linaro) (\RU{для}\EN{for} ARM64), 
\RU{доступный как исполняемые файлы для win32 на}\EN{available as win32-executables at} 
\url{http://www.linaro.org/projects/armv8/}.

\end{itemize}

\RU{Везде в этой книге, кроме как если указано иное, идет речь о 32-битном ARM.}
\EN{32-bit ARM code is used in all cases in this book, if not mentioned otherwise.}
\RU{Когда речь идет о 64-битном ARM, он называется здесь ARM64.}
\EN{If we talk about 64-bit ARM here, it will be called ARM64.}

% subsections
\EN{\subsubsection{\NonOptimizingKeilVI (\ARMMode)}

Let's start by compiling our example in Keil:

\begin{lstlisting}
armcc.exe --arm --c90 -O0 1.c 
\end{lstlisting}

\myindex{\IntelSyntax}
The \IT{armcc} compiler produces assembly listings in Intel-syntax, but it has high-level ARM-processor related macros
\footnote{e.g. ARM mode lacks \PUSH/\POP instructions}, 
but it is more important for us to see the instructions \q{as is} so let's see the compiled result in \IDA.

\begin{lstlisting}[caption=\NonOptimizingKeilVI (\ARMMode) \IDA,style=customasmARM]
.text:00000000             main
.text:00000000 10 40 2D E9    STMFD   SP!, {R4,LR}
.text:00000004 1E 0E 8F E2    ADR     R0, aHelloWorld ; "hello, world"
.text:00000008 15 19 00 EB    BL      __2printf
.text:0000000C 00 00 A0 E3    MOV     R0, #0
.text:00000010 10 80 BD E8    LDMFD   SP!, {R4,PC}

.text:000001EC 68 65 6C 6C+aHelloWorld  DCB "hello, world",0    ; DATA XREF: main+4
\end{lstlisting}

In the example, we can easily see each instruction has a size of 4 bytes.
Indeed, we compiled our code for ARM mode, not for Thumb.

\myindex{ARM!\Instructions!STMFD}
\myindex{ARM!\Instructions!POP}
The very first instruction, \INS{STMFD SP!, \{R4,LR\}}\footnote{\ac{STMFD}}, 
works as an x86 \PUSH instruction, writing the values of two registers (\Reg{4} and \ac{LR}) into the stack.

Indeed, in the output listing from the \IT{armcc} compiler, for the sake of simplification, 
actually shows the \INS{PUSH \{r4,lr\}} instruction.
But that is not quite precise. The \PUSH instruction is only available in Thumb mode.
So, to make things less confusing, we're doing this in \IDA.

This instruction first \glspl{decrement} the \ac{SP} so it points to the place in the stack
that is free for new entries, then it saves the values of the \Reg{4} and \ac{LR} registers at the address
stored in the modified \ac{SP}.

This instruction (like the \PUSH instruction in Thumb mode) is able to save several register values at once which can be very useful.
By the way, this has no equivalent in x86.
It can also be noted that the \TT{STMFD} instruction is a generalization 
of the \PUSH instruction (extending its features), since it can work with any register, not just with \ac{SP}.
In other words, \TT{STMFD} may be used for storing a set of registers at the specified memory address.

\myindex{\PICcode}
\myindex{ARM!\Instructions!ADR}
The \INS{ADR R0, aHelloWorld}
instruction adds or subtracts the value in the \ac{PC} register to the offset where the \TT{hello, world} string is located.
How is the \TT{PC} register used here, one might ask?
This is called \q{\PICcode}\footnote{Read more about it in relevant section~(\myref{sec:PIC})}.

Such code can be executed at a non-fixed address in memory.
In other words, this is \ac{PC}-relative addressing.
The \INS{ADR} instruction takes into account the difference between the address of this instruction and the address where the string is located.
This difference (offset) is always to be the same, no matter at what address our code is loaded by the \ac{OS}.
That's why all we need is to add the address of the current instruction (from \ac{PC}) in order to get the absolute memory address of our C-string.

\myindex{ARM!\Registers!Link Register}
\myindex{ARM!\Instructions!BL}
\INS{BL \_\_2printf}\footnote{Branch with Link} instruction calls the \printf function. 
Here's how this instruction works: 

\begin{itemize}
\item store the address following the \INS{BL} instruction (\TT{0xC}) into the \ac{LR};
\item then pass the control to \printf by writing its address into the \ac{PC} register.
\end{itemize}

When \printf finishes its execution it must have information about where it needs to return the control to.
That's why each function passes control to the address stored in the \ac{LR} register.

That is a difference between \q{pure} \ac{RISC}-processors like ARM and \ac{CISC}-processors like x86,
where the return address is usually stored on the stack.
Read more about this in next section~(\myref{sec:stack}).

By the way, an absolute 32-bit address or offset cannot be encoded in the 32-bit \TT{BL} instruction because
it only has space for 24 bits.
As we may recall, all ARM-mode instructions have a size of 4 bytes (32 bits).
Hence, they can only be located on 4-byte boundary addresses.
This implies that the last 2 bits of the instruction address (which are always zero bits) may be omitted.
In summary, we have 26 bits for offset encoding. This is enough to encode $current\_PC \pm{} \approx{}32M$.

\myindex{ARM!\Instructions!MOV}
Next, the \INS{MOV R0, \#0}\footnote{Meaning MOVe} instruction just writes 0 into the \Reg{0} register.
That's because our C-function returns 0 and the return value is to be placed in the \Reg{0} register.

\myindex{ARM!\Registers!Link Register}
\myindex{ARM!\Instructions!LDMFD}
\myindex{ARM!\Instructions!POP}
The last instruction \INS{LDMFD SP!, {R4,PC}}\footnote{\ac{LDMFD} is an inverse instruction of \ac{STMFD}}.
It loads values from the stack (or any other memory place) in order to save them into \Reg{4} and \ac{PC}, and \glslink{increment}{increments} the \gls{stack pointer} \ac{SP}.
It works like \POP here.\\
N.B. The very first instruction \TT{STMFD} saved the \Reg{4} and \ac{LR} registers pair on the stack, but \Reg{4} and \ac{PC} are \IT{restored} during the \TT{LDMFD} execution.

As we already know, the address of the place where each function must return control to is usually saved in the \ac{LR} register.
The very first instruction saves its value in the stack because the same register will be used by our
\main function when calling \printf.
In the function's end, this value can be written directly to the \ac{PC} register, thus passing control to where our function has been called.

Since \main is usually the primary function in \CCpp,
the control will be returned to the \ac{OS} loader or to a point in a \ac{CRT},
or something like that.

All that allows omitting the \INS{BX LR} instruction at the end of the function.

\myindex{ARM!DCB}
\TT{DCB} is an assembly language directive defining an array of bytes or ASCII strings, akin to the DB directive 
in the x86-assembly language.

}
\RU{\subsubsection{\NonOptimizingKeilVI (\ARMMode)}

Для начала скомпилируем наш пример в Keil:

\begin{lstlisting}
armcc.exe --arm --c90 -O0 1.c 
\end{lstlisting}

\myindex{\IntelSyntax}
Компилятор \IT{armcc} генерирует листинг на ассемблере в формате Intel.
Этот листинг содержит некоторые высокоуровневые макросы, связанные с ARM
\footnote{например, он показывает инструкции \PUSH/\POP, отсутствующие в режиме ARM},
а нам важнее увидеть инструкции \q{как есть}, так что посмотрим скомпилированный результат в \IDA.

\begin{lstlisting}[caption=\NonOptimizingKeilVI (\ARMMode) \IDA,style=customasmARM]
.text:00000000             main
.text:00000000 10 40 2D E9    STMFD   SP!, {R4,LR}
.text:00000004 1E 0E 8F E2    ADR     R0, aHelloWorld ; "hello, world"
.text:00000008 15 19 00 EB    BL      __2printf
.text:0000000C 00 00 A0 E3    MOV     R0, #0
.text:00000010 10 80 BD E8    LDMFD   SP!, {R4,PC}

.text:000001EC 68 65 6C 6C+aHelloWorld  DCB "hello, world",0    ; DATA XREF: main+4
\end{lstlisting}

В вышеприведённом примере можно легко увидеть, что каждая инструкция имеет размер 4 байта.
Действительно, ведь мы же компилировали наш код для режима ARM, а не Thumb.

\myindex{ARM!\Instructions!STMFD}
\myindex{ARM!\Instructions!POP}
Самая первая инструкция, \INS{STMFD SP!, \{R4,LR\}}\footnote{\ac{STMFD}},
работает как инструкция \PUSH в x86, записывая значения двух регистров (\Reg{4} и \ac{LR}) в стек.
Действительно, в выдаваемом листинге на ассемблере компилятор \IT{armcc} для упрощения указывает здесь инструкцию
\INS{PUSH \{r4,lr\}}.
Но это не совсем точно, инструкция \PUSH доступна только в режиме Thumb, поэтому,
во избежание путаницы, я предложил работать в \IDA.

Итак, эта инструкция уменьшает \ac{SP}, чтобы он указывал на место в стеке, свободное для записи
новых значений, затем записывает значения регистров \Reg{4} и \ac{LR} 
по адресу в памяти, на который указывает измененный регистр \ac{SP}.

Эта инструкция, как и инструкция \PUSH в режиме Thumb, может сохранить в стеке одновременно несколько значений регистров, что может быть очень удобно.
Кстати, такого в x86 нет.
Также следует заметить, что \TT{STMFD}~--- генерализация инструкции \PUSH (то есть расширяет её возможности), потому что может работать с любым регистром, а не только с \ac{SP}.
Другими словами, \TT{STMFD} можно использовать для записи набора регистров в указанном месте памяти.

\myindex{\PICcode}
\myindex{ARM!\Instructions!ADR}
Инструкция \INS{ADR R0, aHelloWorld} прибавляет или отнимает значение регистра \ac{PC} к смещению, где хранится строка
\TT{hello, world}.
Причем здесь \ac{PC}, можно спросить? Притом, что это так называемый \q{\PICcode}
\footnote{Читайте больше об этом в соответствующем разделе ~(\myref{sec:PIC})}.
Он предназначен для исполнения будучи не привязанным к каким-либо адресам в памяти.
Другими словами, это относительная от \ac{PC} адресация.
В опкоде инструкции \TT{ADR} указывается разница между адресом этой инструкции и местом, где хранится строка.
Эта разница всегда будет постоянной, вне зависимости от того, куда был загружен \ac{OS} наш код.
Поэтому всё, что нужно~--- это прибавить адрес текущей инструкции (из \ac{PC}), чтобы получить текущий абсолютный адрес нашей Си-строки.

\myindex{ARM!\Registers!Link Register}
\myindex{ARM!\Instructions!BL}
Инструкция \INS{BL \_\_2printf}\footnote{Branch with Link} вызывает функцию \printf.
Работа этой инструкции состоит из двух фаз:

\begin{itemize}
\item записать адрес после инструкции \INS{BL} (\TT{0xC}) в регистр \ac{LR};
\item передать управление в \printf, записав адрес этой функции в регистр \ac{PC}.
\end{itemize}

Ведь когда функция \printf закончит работу, нужно знать, куда вернуть управление, поэтому закончив работу, всякая функция передает управление по адресу, записанному в регистре \ac{LR}.

В этом разница между \q{чистыми} \ac{RISC}-процессорами вроде ARM и \ac{CISC}-процессорами как x86,
где адрес возврата обычно записывается в стек ~(\myref{sec:stack}).

Кстати, 32-битный абсолютный адрес (либо смещение) невозможно закодировать в 32-битной инструкции \INS{BL}, в ней есть место только для 24-х бит.
Поскольку все инструкции в режиме ARM имеют длину 4 байта (32 бита) и инструкции могут находится только по адресам кратным 4, то последние 2 бита (всегда нулевых) можно не кодировать.
В итоге имеем 26 бит, при помощи которых можно закодировать $current\_PC \pm{} \approx{}32M$.

\myindex{ARM!\Instructions!MOV}
Следующая инструкция \INS{MOV R0, \#0}\footnote{Означает MOVe}
просто записывает 0 в регистр \Reg{0}.
Ведь наша Си-функция возвращает 0, а возвращаемое значение всякая функция оставляет в \Reg{0}.

\myindex{ARM!\Registers!Link Register}
\myindex{ARM!\Instructions!LDMFD}
\myindex{ARM!\Instructions!POP}
Последняя инструкция \INS{LDMFD SP!, {R4,PC}}\footnote{\ac{LDMFD}~--- это инструкция, обратная \ac{STMFD}}.
Она загружает из стека (или любого другого места в памяти) значения для сохранения их в \Reg{4} и \ac{PC}, увеличивая \glslink{stack pointer}{указатель стека} \ac{SP}.
Здесь она работает как аналог \POP.\\
N.B. Самая первая инструкция \TT{STMFD} сохранила в стеке \Reg{4} и \ac{LR}, а \IT{восстанавливаются} во время исполнения \TT{LDMFD} регистры \Reg{4} и \ac{PC}.

Как мы уже знаем, в регистре \ac{LR} обычно сохраняется адрес места, куда нужно всякой функции вернуть управление.
Самая первая инструкция сохраняет это значение в стеке, потому что наша функция \main позже будет сама пользоваться этим регистром в момент вызова \printf.
А затем, в конце функции, это значение можно сразу записать прямо в \ac{PC}, таким образом, передав управление туда, откуда была вызвана наша функция.

Так как функция \main обычно самая главная в \CCpp, управление будет возвращено в загрузчик \ac{OS}, либо куда-то в \ac{CRT} 
или что-то в этом роде.

Всё это позволяет избавиться от инструкции \INS{BX LR} в самом конце функции.

\myindex{ARM!DCB}
\TT{DCB}~--- директива ассемблера, описывающая массивы байт или ASCII-строк, аналог директивы DB в x86-ассемблере.

}
\ITA{\subsubsection{\NonOptimizingKeilVI (\ARMMode)}

Iniziamo a compilare il nostro esempio in Keil:

\begin{lstlisting}
armcc.exe --arm --c90 -O0 1.c 
\end{lstlisting}

\myindex{\IntelSyntax}
Il compilatore \IT{armcc} produce un listato assembly con sintassi Intel,
e utilizza macro di alto livello legate al processore ARM
\footnote{ad esempio, l' ARM mode 'e privo delle istruzioni \PUSH/\POP},
tuttavia e' piu' importante per noi vedere le istruzioni \q{cosi' come sono}, quindi guardiamo il risultato compilato con \IDA.

\begin{lstlisting}[caption=\NonOptimizingKeilVI (\ARMMode) \IDA,style=customasmARM]
.text:00000000             main
.text:00000000 10 40 2D E9    STMFD   SP!, {R4,LR}
.text:00000004 1E 0E 8F E2    ADR     R0, aHelloWorld ; "hello, world"
.text:00000008 15 19 00 EB    BL      __2printf
.text:0000000C 00 00 A0 E3    MOV     R0, #0
.text:00000010 10 80 BD E8    LDMFD   SP!, {R4,PC}

.text:000001EC 68 65 6C 6C+aHelloWorld  DCB "hello, world",0    ; DATA XREF: main+4
\end{lstlisting}

Nell'esempio possiamo facilmente vedere che ogni istruzione ha lunghezza pari a 4 byte.
Difatti abbiamo compilato il codice per la modalita' ARM e non Thumb.

\myindex{ARM!\Instructions!STMFD}
\myindex{ARM!\Instructions!POP}
La prima istruzione, \TT{STMFD SP!, \{R4,LR\}}\footnote{\ac{STMFD}},
funzione come l' istruzione \PUSH in x86, scrivendo i valori di due registri (\Reg{4} \ITAph{} \ac{LR}) nello stack.
Infatti il listato di output prodotto dal compilatore \IT{armcc}, per semplificazione, mostra l'istruzione \INS{PUSH \{r4,lr\}}.
Ma cio' non e' del tutto esatto. L'istruzione\PUSH e' disponibile solo in modalita' Thumb. Utilizziamo quindi \IDA per non fare confusione.

Questa istruzione dapprima \glslink{decrement}{decrementa} il valore di \ac{SP} cosi' da farlo puntare alla porzione dello stack che' e' libera di ospitare nuovi dati, quindi salva il valore dei registri \Reg{4} e \ac{LR} all'indirizzo memorizzato nel registro \ac{SP} appena modificato.

Questa istruzione (esattamente come \PUSH in Thumb mode) e' in grado di salvare il valore di piu' registri contemporaneamente, cosa che e' puo' risultare molto utile. 
A proposito, non ha un equivalente in x86.
Si puo' notare anche che l'istruzione \TT{STMFD} e' una generalizzazione dell'istruzione \PUSH (che estende le sue funzionalita'), poiche' puo' funzionare con qualunque registro, e non solo \ac{SP}.
In altre parole, \TT{STMFD} puo' essere usata per memorizzare un insieme di registri all'indirizzo di memoria specificato.

\myindex{\PICcode}
\myindex{ARM!\Instructions!ADR}
L'istruzione \INS{ADR R0, aHelloWorld}
aggiunge o sottrae il valore nel registro \ac{PC} all'offset dove e' memorizzata la stringa \TT{hello, world}.
Ci si potrebbe chiedere, come e' utilizzato qui il registro \TT{PC}?
Cio' e' detto \q{\PICcode}
\footnote{Maggiori informazioni sono fornite nella sezione~(\myref{sec:PIC})}.
Questo tipo di codice puo' essere eseguito a indirizzi non fissi (variabili)in memoria.
In altre parole, e' un indirizzamento relativo a \ac{PC} (\ac{PC}-relative addressing).
L'istruzione \TT{ADR} tiene conto della differenza tra l'indirizzo di questa istruzione e l'indirizzo dove si trova la stringa.
Questa differenza (offset) sara' sempre la stessa, a prescindere dall'indirizzo in cui nostro codice sara' caricato dall'\ac{OS}.
Cio' spiega perche' bisogna soltanto aggiungere l'indirizzo dell'istruzione corrente (from \ac{PC}) per ottenere l'indirizzo assoluto in memoria della nostra stringa C.

\myindex{ARM!\Registers!Link Register}
\myindex{ARM!\Instructions!BL}
L'istruzione \INS{BL \_\_2printf}\footnote{Branch with Link} chiama la funzione \printf. 
Questa istruzione funziona cosi': 
\begin{itemize}
\item memorizza l'indirizzo successivo all'istruzione \INS{BL} (\TT{0xC}) nel registro \ac{LR};
\item quindi passa il controllo a \printf scrivendo il suo indirizzo nel registro \ac{PC}.
\end{itemize}

Quando la funzione \printf termina la sua esecuzione, deve sapere a chi restituire il controllo (dove ritornare). Per questo motivo ogni funzione passa il controllo all'indirizzo memorizzato nel registro \ac{LR}.

Questa e' una differenza tra processori \ac{RISC} \q{puri} come ARM e processori simili a \ac{CISC} come x86, nei quali il return address e' solitamente memorizzato nello stack
\footnote{Maggiori informazioni si trovano nella prossima sezione~(\myref{sec:stack})}.

A proposito, un indirizzo assoluto o un offset a 32-bit non puo' essere codificato nell'istruzione a 32-bit \TT{BL} poiche' ha solo spazio per 24 bit.
Come potremmo ricordare, tutte le istruzioni in ARM-mode hanno dimensione fissa di 4 byte (32 bit).
Dunque possono essere collocate solo su indirizzi allineati a 4-byte.
Cio' implica che gli ultimi 2 bits dell'indirizzo dell'istruzione (che sono sempre zero) possono essere omessi.
Abbiamo in definitiva 26 bit per la codifica dell'offset (offset encoding). E cio e' sufficiente per codificare $current\_PC \pm{} \approx{}32M$.

\myindex{ARM!\Instructions!MOV}
L'istruzione successiva, \INS{MOV R0, \#0}\footnote{\ITAph{} MOVe} scrive soltanto 0 nel registro \Reg{0}.
Questo succede perche' la nostra funzione C restituisce 0, ed il valore di ritorno deve essere memorizzato nel registro \Reg{0}.

\myindex{ARM!\Registers!Link Register}
\myindex{ARM!\Instructions!LDMFD}
\myindex{ARM!\Instructions!POP}
L'ultima istruzione \INS{LDMFD SP!, {R4,PC}}\footnote{\ac{LDMFD} e' l'istruzione inversa rispetto a  \ac{STMFD}}.
Carica i valori dallo stack (o qualunque altra zona di memoria) per salvarli nei registri \Reg{4} e \ac{PC}, e \glslink{increment}{incrementa} lo \gls{stack pointer} \ac{SP}.
In questo caso funziona come \POP.\\
N.B. La prima istruzione \TT{STMFD} aveva salvato la coppia di registri \Reg{4} e \ac{LR} sullo stack, ma \Reg{4} e \ac{PC} vengono \IT{ripristinati} durante l'esecuzione di \TT{LDMFD}.

Come gia' sappiamo, l'indirizzo del posto a cui ogni funzione devere restituire il controllo e' solitamente salvato nel registro \ac{LR}.
La prima istruzione salva il suo valore nello stack perche' lo stesso registro sara' usato dalla nostra funzione \main per la chiamata a \printf.
Al termine della funzione, questo valore puo' essere scritto direttamente nel registro \ac{PC}, passando di fatti il controllo al punto in cui la nostra funzione era stata chiamata.

Dal momento che \main e' solitamente la funzione principale in \CCpp,
il controllo sara' restituito al loader dell' \ac{OS} oppure ad un punto in una \ac{CRT},
o qualcosa del genere.

Tutto cio' consente di omettere l'istruzione \TT{BX LR} alla fine della funzione.

\myindex{ARM!DCB}
\TT{DCB} e' una direttiva assembly che definisce un array di byte o una stringa ASCII, analoga alla direttiva DB in linguaggio assembly x86.

}


\subsection{\NonOptimizingKeilVI (\ThumbMode)}

\RU{Скомпилируем тот же пример в Keil для режима Thumb}\EN{Let's compile the same example using Keil in Thumb mode}:

\begin{lstlisting}
armcc.exe --thumb --c90 -O0 1.c 
\end{lstlisting}

\RU{Получим (в \IDA)}\EN{We are getting (in \IDA)}:

\begin{lstlisting}[caption=\NonOptimizingKeilVI (\ThumbMode) + \IDA]
.text:00000000             main
.text:00000000 10 B5          PUSH    {R4,LR}
.text:00000002 C0 A0          ADR     R0, aHelloWorld ; "hello, world"
.text:00000004 06 F0 2E F9    BL      __2printf
.text:00000008 00 20          MOVS    R0, #0
.text:0000000A 10 BD          POP     {R4,PC}

.text:00000304 68 65 6C 6C+aHelloWorld  DCB "hello, world",0    ; DATA XREF: main+2
\end{lstlisting}

\RU{Сразу бросаются в глаза двухбайтные (16-битные) опкоды\EMDASH{}это, как уже было отмечено, Thumb.}%
\EN{We can easily spot the 2-byte (16-bit) opcodes. This is, as was already noted, Thumb.}
\index{ARM!\Instructions!BL}
\RU{Кроме инструкции \TT{BL}.}\EN{The \TT{BL} instruction, however, }
\RU{Но на самом деле она состоит из двух 16-битных инструкций}%
\EN{consists of two 16-bit instructions}.
\RU{Это потому что в одном 16-битном опкоде слишком мало места для задания смещения, по которому находится функция \printf}%
\EN{This is because it is impossible to load an offset for the \printf function while using the small space in one 16-bit opcode}.
\RU{Так что первая 16-битная инструкция загружает старшие 10 бит смещения, а вторая~--- младшие 11 бит смещения}%
\EN{Therefore, the first 16-bit instruction loads the higher 10 bits of the offset and the second instruction loads 
the lower 11 bits of the offset}.
% TODO:
% BL has space for 11 bits, so if we don't encode the lowest bit,
% then we should get 11 bits for the upper half, and 12 bits for the lower half.
% And the highest bit encodes the sign, so the destination has to be within
% \pm 4M of current_PC.
% This may be less if adding the lower half does not carry over,
% but I'm not sure --all my programs have 0 for the upper half,
% and don't carry over for the lower half.
% It would be interesting to check where __2printf is located relative to 0x8
% (I think the program counter is the next instruction on a multiple of 4
% for THUMB).
% The lower 11 bytes of the BL instructions and the even bit are
% 000 0000 0110 | 001 0010 1110 0 = 000 0000 0110 0010 0101 1100 = 0x00625c,
% so __2printf should be at 0x006264.
% But if we only have 10 and 11 bits, then the offset would be:
% 00 0000 0110 | 01 0010 1110 0 = 0 0000 0011 0010 0101 1100 = 0x00325c,
% so __2printf should be at 0x003264.
% In this case, though, the new program counter can only be 1M away,
% because of the highest bit is used for the sign.
\RU{Как уже было упомянуто, все инструкции в Thumb-режиме имеют длину 2 байта (или 16 бит)}%
\EN{As was noted, all instructions in Thumb mode have a size of 2 bytes (or 16 bits)}.
\RU{Поэтому невозможна такая ситуация, когда Thumb-инструкция начинается по нечетному адресу.}
\EN{This implies it is impossible for a Thumb-instruction to be at an odd address whatsoever.}
\RU{Учитывая сказанное, последний бит адреса можно не кодировать}%
\EN{Given the above, the last address bit may be omitted while encoding instructions}.
\RU{Таким образом, в Thumb-инструкции \TT{BL} можно закодировать адрес}
\EN{In summary, the \TT{BL} Thumb-instruction can encode an address in} $current\_PC \pm{}\approx{}2M$.

\index{ARM!\Instructions!PUSH}
\index{ARM!\Instructions!POP}
\RU{Остальные инструкции в функции (\PUSH и \POP) здесь работают почти так же, как и описанные \TT{STMFD}/\TT{LDMFD}, только регистр \ac{SP} здесь не указывается явно}%
\EN{As for the other instructions in the function: \PUSH and \POP work here just like the described \TT{STMFD}/\TT{LDMFD} only the \ac{SP} register is not mentioned explicitly here}.
\TT{ADR} \RU{работает так же, как и в предыдущем примере}\EN{works just like in the previous example}.
\TT{MOVS} \RU{записывает 0 в регистр \Reg{0} для возврата нуля}%
\EN{writes 0 into the \Reg{0} register in order to return zero}.

\subsection{\OptimizingXcodeIV (\ARMMode)}

Xcode 4.6.3 \RU{без включенной оптимизации выдает слишком много лишнего кода, поэтому включим оптимизацию компилятора (ключ \Othree), потому что там меньше инструкций.}
\EN{without optimization turned on produces a lot of redundant code so we'll study optimized output, where the instruction count is as small as possible, setting the compiler switch \Othree.}

\begin{lstlisting}[caption=\OptimizingXcodeIV (\ARMMode)]
__text:000028C4             _hello_world
__text:000028C4 80 40 2D E9   STMFD           SP!, {R7,LR}
__text:000028C8 86 06 01 E3   MOV             R0, #0x1686
__text:000028CC 0D 70 A0 E1   MOV             R7, SP
__text:000028D0 00 00 40 E3   MOVT            R0, #0
__text:000028D4 00 00 8F E0   ADD             R0, PC, R0
__text:000028D8 C3 05 00 EB   BL              _puts
__text:000028DC 00 00 A0 E3   MOV             R0, #0
__text:000028E0 80 80 BD E8   LDMFD           SP!, {R7,PC}

__cstring:00003F62 48 65 6C 6C+aHelloWorld_0  DCB "Hello world!",0
\end{lstlisting}

\RU{Инструкции}\EN{The instructions} \TT{STMFD} \AndENRU \TT{LDMFD} \RU{нам уже знакомы}\EN{are already familiar to us}.

\index{ARM!\Instructions!MOV}
\RU{Инструкция \MOV просто записывает число \TT{0x1686} в регистр \Reg{0}~--- это смещение, указывающее на строку \q{Hello world!}}%
\EN{The \MOV instruction just writes the number \TT{0x1686} into the \Reg{0} register.
This is the offset pointing to the \q{Hello world!} string}.

\RU{Регистр \TT{R7} (по стандарту, принятому в \cite{IOSABI}) это frame pointer, о нем будет рассказано позже.}
\EN{The \TT{R7} register (as it is standardized in \cite{IOSABI}) is a frame pointer. More on that below.}

\index{ARM!\Instructions!MOVT}
\RU{Инструкция}\EN{The} \TT{MOVT R0, \#0} (MOVe Top) \RU{записывает 0 в старшие 16 бит регистра}%
\EN{instruction writes 0 into higher 16 bits of the register}.
\RU{Дело в том, что обычная инструкция \MOV в режиме ARM может записывать какое-либо значение только в младшие 16 бит регистра, ведь в ней нельзя закодировать больше}%
\EN{The issue here is that the generic \MOV instruction in ARM mode may write only the lower 16 bits of the register}.
\RU{Помните, что в режиме ARM опкоды всех инструкций ограничены длиной в 32 бита. Конечно, это ограничение не касается перемещений данных между регистрами.}
\EN{Remember, all instruction opcodes in ARM mode are limited in size to 32 bits. Of course, this limitation is not related to moving data between registers.}
\RU{Поэтому для записи в старшие биты (с 16-го по 31-й включительно) существует дополнительная команда \TT{MOVT}}%
\EN{That's why an additional instruction \TT{MOVT} exists for writing into the higher bits (from 16 to 31 inclusive)}.
\RU{Впрочем, здесь её использование избыточно, потому что инструкция \TT{MOV R0, \#0x1686} выше и так обнулила старшую часть регистра}%
\EN{Its usage here, however, is redundant because the \TT{MOV R0, \#0x1686} instruction above cleared the higher part of the register}.
\RU{Возможно, это недочет компилятора}\EN{This is probably a shortcoming of the compiler}.
% TODO:
% I think, more specifically, the string is not put in the text section,
% ie. the compiler is actually not using position-independent code,
% as mentioned in the next paragraph.
% MOVT is used because the assembly code is generated before the relocation,
% so the location of the string is not yet known,
% and the high bits may still be needed.

\index{ARM!\Instructions!ADD}
\RU{Инструкция}\EN{The} \TT{ADD R0, PC, R0} \RU{прибавляет \ac{PC} к \Reg{0} для вычисления действительного адреса строки \q{Hello world!}. Как нам уже известно, это \q{\PICcode}, поэтому такая корректива необходима}%
\EN{instruction adds the value in the \ac{PC} to the value in the \Reg{0}, to calculate the absolute address of the \q{Hello world!} string. 
As we already know, it is \q{\PICcode} so this correction is essential here}.

\RU{Инструкция \TT{BL} вызывает \puts вместо \printf}%
\EN{The \TT{BL} instruction calls the \puts function instead of \printf}.

\label{puts}
\index{\CStandardLibrary!puts()}
\index{puts() \RU{вместо}\EN{instead of} printf()}
\RU{Компилятор заменил вызов \printf на \puts. 
Действительно, \printf с одним аргументом это почти аналог \puts.}
\EN{GCC replaced the first \printf call with \puts.
Indeed: \printf with a sole argument is almost analogous to \puts.} 
\RU{\IT{Почти}, если принять условие, что в строке не будет управляющих символов \printf, 
начинающихся со знака процента. Тогда эффект от работы этих двух функций будет разным}%
\EN{\IT{Almost}, because the two functions are producing the same result only in case the 
string does not contain printf format identifiers starting with \IT{\%}. 
In case it does, the effect of these two functions would be different}%
\footnote{
\RU{Также нужно заметить, что \puts не требует символа перевода строки `\textbackslash{}n' в конце строки,
поэтому его здесь нет.}
\EN{It has also to be noted the \puts does not require a `\textbackslash{}n' new line symbol 
at the end of a string, so we do not see it here.}}.

\RU{Зачем компилятор заменил один вызов на другой? Наверное потому что \puts работает быстрее}%
\EN{Why did the compiler replace the \printf with \puts? Probably because \puts is faster}%
\footnote{\href{http://go.yurichev.com/17063}{ciselant.de/projects/gcc\_printf/gcc\_printf.html}}. 
\RU{Видимо потому что \puts проталкивает символы в \gls{stdout} не сравнивая каждый со знаком процента.}
\EN{Because it just passes characters to \gls{stdout} without comparing every one of them with the \IT{\%} symbol.}

\RU{Далее уже знакомая инструкция}\EN{Next, we see the familiar} 
\TT{MOV R0, \#0}\RU{, служащая для установки в 0 возвращаемого значения функции}%
\EN{instruction intended to set the \Reg{0} register to 0}.

\subsection{\OptimizingXcodeIV (\ThumbTwoMode)}

\RU{По умолчанию}\EN{By default} Xcode 4.6.3 
\RU{генерирует код для режима Thumb-2 примерно в такой манере}%
\EN{generates code for Thumb-2 in this manner}:

\begin{lstlisting}[caption=\OptimizingXcodeIV (\ThumbTwoMode)]
__text:00002B6C                   _hello_world
__text:00002B6C 80 B5          PUSH            {R7,LR}
__text:00002B6E 41 F2 D8 30    MOVW            R0, #0x13D8
__text:00002B72 6F 46          MOV             R7, SP
__text:00002B74 C0 F2 00 00    MOVT.W          R0, #0
__text:00002B78 78 44          ADD             R0, PC
__text:00002B7A 01 F0 38 EA    BLX             _puts
__text:00002B7E 00 20          MOVS            R0, #0
__text:00002B80 80 BD          POP             {R7,PC}

...

__cstring:00003E70 48 65 6C 6C 6F 20+aHelloWorld  DCB "Hello world!",0xA,0
\end{lstlisting}

% Q: If you subtract 0x13D8 from 0x3E70,
% you actually get a location that is not in this function, or in _puts.
% How is PC-relative addressing done in THUMB2?
% A: it's not Thumb-related. there are just mess with two different segments. TODO: rework this listing.

\index{\ThumbTwoMode}
\index{ARM!\Instructions!BL}
\index{ARM!\Instructions!BLX}
\RU{Инструкции \TT{BL} и \TT{BLX} в Thumb, как мы помним, кодируются как пара 16-битных инструкций, 
а в Thumb-2 эти \IT{суррогатные} опкоды расширены так, что новые инструкции кодируются здесь как 
32-битные инструкции}%
\EN{The \TT{BL} and \TT{BLX} instructions in Thumb mode, as we recall, are encoded as a pair
of 16-bit instructions.
In Thumb-2 these \IT{surrogate} opcodes are extended in such a way so that new instructions
may be encoded here as 32-bit instructions}.
\RU{Это можно заметить по тому что опкоды Thumb-2 инструкций всегда начинаются с \TT{0xFx} либо с \TT{0xEx}}%
\EN{That is obvious considering that the opcodes of the Thumb-2 instructions always begin with \TT{0xFx} or \TT{0xEx}}.
\RU{Но в листинге \IDA байты опкода переставлены местами.
Это из-за того, что в процессоре ARM инструкции кодируются так:
в начале последний байт, потом первый (для Thumb и Thumb-2 режима), либо, 
(для инструкций в режиме ARM) в начале четвертый байт, затем третий, второй и первый 
(т.е. другой \gls{endianness})}%
\EN{But in the \IDA listing
the opcode bytes are swapped because for ARM processor the instructions are encoded as follows: 
last byte comes first and after that comes the first one (for Thumb and Thumb-2 modes) 
or for instructions in ARM mode the fourth byte comes first, then the third,
then the second and finally the first (due to different \gls{endianness})}.

\RU{Вот так байты следуют в листингах IDA:}
\EN{So that is how bytes are located in IDA listings:}
\begin{itemize}
\item \RU{для режимов ARM и ARM64}\EN{for ARM and ARM64 modes}: 4-3-2-1;
\item \RU{для режима Thumb}\EN{for Thumb mode}: 2-1;
\item \RU{для пары 16-битных инструкций в режиме Thumb-2}\EN{for 16-bit instructions pair in Thumb-2 mode}: 2-1-4-3.
\end{itemize}

\index{ARM!\Instructions!MOVW}
\index{ARM!\Instructions!MOVT.W}
\index{ARM!\Instructions!BLX}
\RU{Так что мы видим здесь что инструкции \TT{MOVW}, \TT{MOVT.W} и \TT{BLX} начинаются с}
\EN{So as we can see, the \TT{MOVW}, \TT{MOVT.W} and \TT{BLX} instructions begin with} \TT{0xFx}.

\RU{Одна из Thumb-2 инструкций это}\EN{One of the Thumb-2 instructions is}
\TT{MOVW R0, \#0x13D8}\RU{~--- она записывает 16-битное число в младшую часть регистра \Reg{0}, очищая старшие биты.}
\EN{~---it stores a 16-bit value into the lower part of the \Reg{0} register, clearing the higher bits.}

\RU{Ещё}\EN{Also,} \TT{MOVT.W R0, \#0}\RU{~--- эта инструкция работает так же, как и}
\EN{~works just like} 
\TT{MOVT} \RU{из предыдущего примера, но она работает в}\EN{from the previous example only it works in} Thumb-2.

\index{ARM!\RU{переключение режимов}\EN{mode switching}}
\index{ARM!\Instructions!BLX}
\RU{Помимо прочих отличий, здесь используется инструкция}
\EN{Among the other differences, the} \TT{BLX} 
\RU{вместо}\EN{instruction is used in this case instead of the} \TT{BL}.
\RU{Отличие в том, что помимо сохранения адреса возврата в регистре \ac{LR} и передаче управления 
в функцию \puts, происходит смена режима процессора с Thumb/Thumb-2 на режим ARM (либо назад).}
\EN{The difference is that, besides saving the \ac{RA} in the \ac{LR} register and passing control 
to the \puts function, the processor is also switching from Thumb/Thumb-2 mode to ARM mode (or back).}
\RU{Здесь это нужно потому, что инструкция, куда ведет переход, выглядит так (она закодирована в режиме ARM)}%
\EN{This instruction is placed here since the instruction to which control is passed looks like (it is encoded in ARM mode)}:

\begin{lstlisting}
__symbolstub1:00003FEC _puts           ; CODE XREF: _hello_world+E
__symbolstub1:00003FEC 44 F0 9F E5     LDR  PC, =__imp__puts
\end{lstlisting}

\EN{This is essentially a jump to the place where the address of \puts is written in the imports' section.}
\RU{Это просто переход на место, где записан адрес \puts в секции импортов.}

\RU{Итак, внимательный читатель может задать справедливый вопрос: почему бы не вызывать \puts сразу в 
том же месте кода, где он нужен?}
\EN{So, the observant reader may ask: why not call \puts right at the point in the code where it is needed?}

\RU{Но это не очень выгодно из-за экономии места и вот почему}%
\EN{Because it is not very space-efficient}.

\index{\RU{Динамически подгружаемые библиотеки}\EN{Dynamically loaded libraries}}
\RU{Практически любая программа использует внешние динамические библиотеки (будь то DLL в Windows, .so в *NIX 
либо .dylib в \MacOSX)}\EN{Almost any program uses external dynamic libraries (like DLL in Windows, .so in *NIX or .dylib in \MacOSX)}.
\RU{В динамических библиотеках находятся часто используемые библиотечные функции, в том числе стандартная функция Си \puts}%
\EN{The dynamic libraries contain frequently used library functions, including the standard C-function \puts}.

\index{Relocation}
\RU{В исполняемом бинарном файле}\EN{In an executable binary file} 
(Windows PE .exe, ELF \RU{либо}\EN{or} Mach-O) \RU{имеется секция импортов, список символов (функций либо глобальных переменных) импортируемых из внешних модулей, а также названия самих модулей}%
\EN{an import section is present.
This is a list of symbols (functions or global variables) imported from external modules along with the names of the modules themselves}.

\RU{Загрузчик \ac{OS} загружает необходимые модули и, перебирая импортируемые символы в основном модуле, проставляет правильные адреса каждого символа}%
\EN{The \ac{OS} loader loads all modules it needs and, while enumerating import symbols in the primary module, determines the correct addresses of each symbol}.

\RU{В нашем случае,}\EN{In our case,} \IT{\_\_imp\_\_puts} 
\RU{это 32-битная переменная, куда загрузчик \ac{OS} запишет правильный адрес этой же функции во внешней библиотеке}%
\EN{is a 32-bit variable used by the \ac{OS} loader to store the correct address of the function in an external library}. 
\RU{Так что инструкция \TT{LDR} просто берет 32-битное значение из этой переменной, и, записывая его в регистр \ac{PC}, просто передает туда управление}%
\EN{Then the \TT{LDR} instruction just reads the 32-bit value from this variable and writes it into the \ac{PC} register, passing control to it}.

\RU{Чтобы уменьшить время работы загрузчика \ac{OS}, 
нужно чтобы ему пришлось записать адрес каждого символа только один раз, 
в соответствующее, выделенное для них, место.}
\EN{So, in order to reduce the time the \ac{OS} loader needs for completing this procedure, 
it is good idea to write the address of each symbol only once, to a dedicated place.}

\index{thunk-\RU{функции}\EN{functions}}
\RU{К тому же, как мы уже убедились, нельзя одной инструкцией загрузить в регистр 32-битное число без обращений к памяти}%
\EN{Besides, as we have already figured out, it is impossible to load a 32-bit value into a register 
while using only one instruction without a memory access}.
\RU{Так что наиболее оптимально выделить отдельную функцию, работающую в режиме ARM, 
чья единственная цель~--- передавать управление дальше, в динамическую библиотеку.}
\EN{Therefore, the optimal solution is to allocate a separate function working in ARM mode with the sole 
goal of passing control to the dynamic library}
\RU{И затем ссылаться на эту короткую функцию из одной инструкции (так называемую \glslink{thunk function}{thunk-функцию}) из Thumb-кода}%
\EN{and then to jump to this short one-instruction function (the so-called \gls{thunk function}) from the Thumb-code}.

\index{ARM!\Instructions!BL}
\RU{Кстати, в предыдущем примере (скомпилированном для режима ARM), переход при помощи инструкции \TT{BL} ведет 
на такую же \glslink{thunk function}{thunk-функцию}, однако режим процессора не переключается (отсюда отсутствие \q{X} в мнемонике инструкции)}%
\EN{By the way, in the previous example (compiled for ARM mode) the control is passed by the \TT{BL} to the 
same \gls{thunk function}.
The processor mode, however, is not being switched (hence the absence of an \q{X} in the instruction mnemonic)}.

\subsubsection{\EN{More about thunk-functions}\RU{Еще о thunk-функциях}}
\index{thunk-\RU{функции}\EN{functions}}

\RU{Thunk-функции трудновато понять, вероятно, из-за путаницы в терминах.}
\EN{Thunk-functions are hard to understand, apparently, because of a misnomer.}

\RU{Проще всего представлять их как адаптеры-переходники из одного типа разъемов в другой.}
\EN{The simplest way to understand it as adaptors or convertors of one type of jack to another.}
\RU{Например, адаптер позволяющее вставить в американскую розетку британскую вилку, или наоборот.}
\EN{For example, an adaptor allowing the insertion of a British power plug into an American wall socket, or vice-versa.} 

\EN{Thunk functions are also sometimes called \IT{wrappers}.}
\RU{Thunk-функции также иногда называются \IT{wrapper-ами}. \IT{Wrap} в английском языке это \IT{обертывать}, \IT{завертывать}.}

\RU{Вот еще несколько описаний этих функций:}
\EN{Here are a couple more descriptions of these functions:}

\begin{framed}
\begin{quotation}
“A piece of coding which provides an address:”, according to P. Z. Ingerman, 
who invented thunks in 1961 as a way of binding actual parameters to their formal 
definitions in Algol-60 procedure calls. If a procedure is called with an expression 
in the place of a formal parameter, the compiler generates a thunk which computes 
the expression and leaves the address of the result in some standard location.

\dots

Microsoft and IBM have both defined, in their Intel-based systems, a “16-bit environment” 
(with bletcherous segment registers and 64K address limits) and a “32-bit environment” 
(with flat addressing and semi-real memory management). The two environments can both be 
running on the same computer and OS (thanks to what is called, in the Microsoft world, 
WOW which stands for Windows On Windows). MS and IBM have both decided that the process 
of getting from 16- to 32-bit and vice versa is called a “thunk”; for Windows 95, 
there is even a tool, THUNK.EXE, called a “thunk compiler”.
\end{quotation}
\end{framed}
% TODO FIXME move to bibliography and quote properly above the quote
( \href{http://go.yurichev.com/17362}{The Jargon File} )

\subsection{ARM64}

\subsubsection{GCC}

\RU{Компилируем пример в}\EN{Let's compile the example using} GCC 4.8.1 \InENRU ARM64:

\lstinputlisting[numbers=left,label=hw_ARM64_GCC,caption=\NonOptimizing GCC 4.8.1 + objdump]
{patterns/01_helloworld/ARM/hw.lst}

\RU{В ARM64 нет режима thumb и thumb-2, только ARM, так что тут только 32-битные инструкции.}
\EN{There are no thumb and thumb-2 modes in ARM64, only ARM, so there are 32-bit instructions only.}
\RU{Регистров тут в 2 раза больше}\EN{Registers count is doubled}: \ref{ARM64_GPRs}.
\RU{64-битные регистры теперь имеют префикс}\EN{64-bit registers has} 
\TT{X-}\EN{ prefixes, while its 32-bit parts}\RU{, а их 32-битные части}\EMDASH{}\TT{W-}.

\RU{Инструкция }\TT{STP}\EN{ instruction} (\IT{Store Pair}) 
\RU{сохраняет в стеке сразу два регистра}\EN{saves two registers in stack simultaneously}: \RegX{29} \InENRU \RegX{30}.
\RU{Конечно, эта инструкция может сохранять эту пару где угодно в памяти, но здесь указан регистр \ac{SP}, так что,
пара сохраняется именно в стеке.}
\EN{Of course, this instruction is able to save this pair at random place of memory, 
but \ac{SP} register is specified here, so the pair is saved in stack.}
\RU{Регистры в ARM64 64-битные, каждый это 8 байт, так что для хранения двух регистров нужно именно 16 байт.}
\EN{ARM64 registers are 64-bit ones, each contain 8 bytes, so one need 16 bytes for saving two registers.}

\RU{Восклицательный знак после операнда означает, что в начале от \ac{SP} будет отнято 16, и только затем
значения из пары регистров будут записаны в стек.}
\EN{Exclamation mark after operand mean that 16 will be subtracted from \ac{SP} first, and only then
values from registers pair will be written into the stack.}
\RU{Это называется}\EN{This is also called} \IT{pre-index}.
\RU{Больше о разнице между}\EN{About difference between} \IT{post-index} \AndENRU \IT{pre-index}, 
\RU{описано здесь}\EN{read here}: \ref{ARM_postindex_vs_preindex}.

\RU{Таким образом, в терминах более знакомого всем процессора x86, первая инструкция это просто аналог 
пары инструкций}
\EN{Hence, in terms of more familiar x86, the first instruction is just analogous to pair of}
\TT{PUSH X29} \AndENRU \TT{PUSH X30}.
\RegX{29} \EN{is used as \ac{FP} in ARM64}\RU{в ARM64 используется как \ac{FP}}, \EN{and}\RU{а} \RegX{30} 
\EN{as}\RU{как} \ac{LR}, \RU{поэтому они сохраняются в прологе ф-ции и
восстанавливаются в эпилоге}\EN{so that's why they are saved in function prologue and restored in function
epilogue}.

\EN{The second instruction saves}\RU{Вторая инструкция записывает} \ac{SP} \InENRU \RegX{29} (\OrENRU \ac{FP}).
\RU{Это нужно для установки стекового фрейма ф-ции}\EN{This is needed for function stack frame setup}.

\RU{Инструкции }\TT{ADRP} \AndENRU \ADD \EN{instructions are needed for forming address of the 
string}\RU{нужны для формирования адреса строки} ``Hello!'' \EN{in the \RegX{0} register}\RU{в регистре \RegX{0}}, 
\RU{ведь первый аргумент ф-ции передается через этот регистр}\EN{because first function argument is passed
in this register}.
\RU{Но в ARM нет инструкций, при помощи которых можно записать в регистр длинное число}\EN{But there are
no instructions in ARM helping to write large number into register} 
(\RU{потому что сама длина инструкции ограничена 4-ю байтами, больше об этом здесь}\EN{because instruction
length is limited by 4 bytes, read more about it here}: \ref{ARM_big_constants_loading}).
\RU{Так что нужно использовать несколько инструкций}\EN{So several instructions should be used}.
\RU{Первая инструкция}\EN{The first instruction} (\TT{ADRP}) \EN{writes address of 4Kb page where string is
located into \RegX{0}}\RU{записывает в \RegX{0} адрес 4-килобайтной страницы где находится строка}, 
\EN{and the second one}\RU{а вторая} (\ADD) \RU{просто прибавляет к этому адресу остаток}\EN{just adds
reminder to the address}.
\EN{Read more about}\RU{Читайте больше об этом}: \ref{ARM64_relocs}.

\TT{0x400000 + 0x648 = 0x400648}, \EN{and we see our ``Hello!'' C-string in the \TT{.rodata} data segment at this
address}\RU{и мы видим что в секции данных \TT{.rodata} по этому адресу как раз находится наша
Си-строка ``Hello!''}.

\RU{Затем, при помощи инструкции \TT{BL} вызывается \puts, это уже рассматривалось раннее: \ref{puts}.}
\EN{\puts is called then using \TT{BL} instruction, this was already discussed before: \ref{puts}.}

\RU{Инструкция }\MOV \EN{instruction writes $0$ into}\RU{записывает $0$ в} \RegW{0}. 
\RegW{0} \RU{это младшие 32 бита регистра}\EN{is low 32 bits of} \RegX{0}\EN{ register}:

\begin{center}
\begin{tabular}{ | l | l | }
\hline
\RU{Старшие 32 бита}\EN{High 32-bit part} & \RU{младщие 32 бита}\EN{low 32-bit part} \\
\hline
\multicolumn{2}{ | c | }{X0} \\
\hline
\multicolumn{1}{ | c | }{} & \multicolumn{1}{ c | }{W0} \\
\hline
\end{tabular}
\end{center}


\RU{А результат ф-ции возвращается через \RegX{0}, и \main возвращает $0$, 
так что вот так готовится возвращаемый результат.}
\EN{Function result is returning via \RegX{0} and \main returning $0$, so that's how returning
result is prepared.}
\RU{Почему именно 32-битная часть}\EN{But why 32-bit part}?
\RU{Потому в ARM64, как и в x86-64, тип \Tint оставили 32-битным, для лучшей совместимости.}
\EN{Because \Tint in ARM64, just like in x86-64, is still 32-bit, for better compatibility.}
\RU{Следовательно, раз уж ф-ция возвращает 32-битный \Tint, то нужно заполнить только 32 младших бита 
регистра \RegX{0}.}
\EN{So if function returning 32-bit \Tint, only 32 lowest bits of \RegX{0} register should be filled.}

\RU{Для того, чтобы удостовериться в этом, я немного отредактировал свой пример и перекомпилировал его.}
\EN{In order to get sure about it, I changed by example slightly and recompiled it.}
\RU{Теперь}\EN{Now} \main \RU{возвращает 64-битное значение}\EN{returns 64-bit value}:

\begin{lstlisting}[caption=\main \RU{возвращающая значение типа}\EN{returning a value of} \TT{uint64\_t}\EN{ type}]
#include <stdio.h>
#include <stdint.h>

uint64_t main()
{
        printf ("Hello!\n");
        return 0;
};
\end{lstlisting}

\RU{Результат точно такой же, только \MOV в той строке теперь выглядит так:}
\EN{Result is very same, but that's how \MOV at that line is now looks like:}

\begin{lstlisting}[caption=\NonOptimizing GCC 4.8.1 + objdump]
  4005a4:       d2800000        mov     x0, #0x0                        // #0
\end{lstlisting}

\RU{Далее, при помощи инструкции \TT{LDP} (\IT{Load Pair}), восстанавливаются регистры \RegX{29} и \RegX{30}.}
\EN{\TT{LDP} (\IT{Load Pair}) then restores \RegX{29} and \RegX{30} registers.}
\RU{Восклицательного знака после инструкции нет: это означает, что в начале значения достаются из стека,
и только потом \ac{SP} увеличивается на 16.}
\EN{There are no exclamation mark after instruction: this mean, the value is first loaded from the stack,
only then \ac{SP} value is increased by 16.}
\RU{Это называется}\EN{This is called} \IT{post-index}.

\RU{В ARM64 есть новая инструкция}\EN{New instruction appeared in ARM64}: \RET. 
\RU{Она работает так же как и}\EN{It works just as} \TT{BX LR}, \RU{но там добавлен специальный бит,
подсказывающий процессору, что это именно выход из ф-ции, а не просто переход, чтобы процессор
мог более оптимально исполнять эту инструкцию}\EN{but a special \IT{hint} bit is added, showing to the \ac{CPU}
that this is return from the function, not just another branch instruction, so it can execute it more optimally}.

\RU{Из-за простоты этой ф-ции, оптимизирующий GCC генерирует точно такой же код.}
\EN{Due to simplicity of the function, optimizing GCC generates the very same code.}

\subsubsection{Xcode}

\OptimizingXcodeV \RU{делает почти такой же код}\EN{doing the same}.
\RU{Вот что показывает}\EN{Here is what} otool\footnote{\RU{аналог}\EN{analogous to} objdump} 
\RU{с ключом}\EN{with the} \TT{-tv}\EN{ key shows}:

\begin{lstlisting}[caption=\OptimizingXcodeV + otool]
0000000000000000		stp	fp, lr, [sp, #-16]!
0000000000000004		add	fp, sp, 0
0000000000000008		adrp	x0, 0 ; 0x0
000000000000000c		add	x0, x0, 0
0000000000000010		bl	0x10
0000000000000014		movz	w0, #0
0000000000000018		ldp	fp, lr, [sp], #16
000000000000001c		ret	lr
\end{lstlisting}

\RU{Код такой же}\EN{The code is the same}.
\TT{MOVZ} \RU{в данном случае это синоним}\EN{here is synonymous to} \MOV.
\RU{Вот что немного отличается: otool показывает регистр}\EN{Here is difference: otool shows} 
\RegX{29}\EN{ register} \RU{как}\EN{as} \ac{FP}, \RU{а}\EN{and} \RegX{30} \RU{как}\EN{as} \ac{LR}.
\RU{Это действительно синонимы для этих регистров.}
\EN{These are indeed nicknames for these registers.}

\RU{\RET otool показывает как \TT{RET LR}, это немного избыточный результат дизассемблера.}
\EN{otool also shows \RET as \TT{RET LR}, this is somewhat redundant disassembler result.}


\section{MIPS}

\subsection{\RU{О \q{глобальном указателе} (\q{global pointer})}\EN{A word about the \q{global pointer}}}
\label{MIPS_GP}

\index{MIPS!\GlobalPointer}
\RU{\q{Глобальный указатель} (\q{global pointer})~--- это важная концепция в MIPS.}
\EN{One important MIPS concept is the \q{global pointer}.}
\RU{Как мы уже возможно знаем, каждая инструкция в MIPS имеет размер 32 бита, поэтому невозможно
закодировать 32-битный адрес внутри одной инструкции. Вместо этого нужно использовать пару инструкций
(как это сделал GCC для загрузки адреса текстовой строки в нашем примере).}
\EN{As we may already know, each MIPS instruction has a size of 32 bits, so it's impossible to embed a 32-bit
address into one instruction: a pair has to be used for this 
(like GCC did in our example for the text string address loading).}

\RU{С другой стороны, используя только одну инструкцию, 
возможно загружать данные по адресам в пределах $register-32768...register+32767$, потому что 16 бит
знакового смещения можно закодировать в одной инструкции).}
\EN{It's possible, however, to load data from the address in the range of $register-32768...register+32767$ using one
single instruction (because 16 bits of signed offset could be encoded in a single instruction).}
\RU{Так мы можем выделить какой-то регистр для этих целей и ещё выделить буфер в 64KiB для самых 
частоиспользуемых данных.}
\EN{So we can allocate some register for this purpose and also allocate a 64KiB area of most used data.}
\RU{Выделенный регистр называется \q{глобальный указатель} (\q{global pointer}) и он указывает на середину
области 64KiB.}
\EN{This allocated register is called a \q{global pointer} and it points to the middle of the 64KiB area.}
\RU{Эта область обычно содержит глобальные переменные и адреса импортированных функций вроде \printf,
потому что разработчики GCC решили, что получение адреса функции должно быть как можно более быстрой операцией,
исполняющейся за одну инструкцию вместо двух.}
\EN{This area usually contains global variables and addresses of imported functions like \printf, 
because the GCC developers decided that getting the address of some function must be as fast as a single instruction
execution instead of two.}
\RU{В ELF-файле эта 64KiB-область находится частично в секции .sbss (\q{small \ac{BSS}}) для неинициализированных
данных и в секции .sdata (\q{small data}) для инициализированных данных.}
\EN{In an ELF file this 64KiB area is located partly in sections .sbss (\q{small \ac{BSS}}) for uninitialized data and 
.sdata (\q{small data}) for initialized data.}

\RU{Это значит что программист может выбирать, к чему нужен как можно более быстрый доступ, и затем расположить
это в секциях .sdata/.sbss.}
\EN{This implies that the programmer may choose what data he/she wants to be accessed fast and place it into 
.sdata/.sbss.}

\RU{Некоторые программисты \q{старой школы} могут вспомнить модель памяти в MS-DOS \myref{8086_memory_model} 
или в менеджерах памяти вроде XMS/EMS, где вся память делилась на блоки по 64KiB.}
\EN{Some old-school programmers may recall the MS-DOS memory model \myref{8086_memory_model} 
or the MS-DOS memory managers like XMS/EMS where all memory was divided in 64KiB blocks.}

\index{PowerPC}
\RU{Эта концепция применяется не только в MIPS. По крайней мере PowerPC также использует эту технику.}
\EN{This concept is not unique to MIPS. At least PowerPC uses this technique as well.}

\subsection{\Optimizing GCC}

\EN{Lets consider the following example, which illustrates the \q{global pointer} concept.}
\RU{Рассмотрим следующий пример, иллюстрирующий концепцию \q{глобального указателя}.}

\lstinputlisting[caption=\Optimizing GCC 4.4.5 (\assemblyOutput),numbers=left]{patterns/01_helloworld/MIPS/hw_O3.s.\LANG}

\RU{Как видно, регистр \$GP в прологе функции выставляется в середину этой области.}
\EN{As we see, the \$GP register is set in the function prologue to point to the middle of this area.}
\RU{Регистр \ac{RA} сохраняется в локальном стеке.}
\EN{The \ac{RA} register is also saved in the local stack.}
\RU{Здесь также используется \puts вместо \printf.}
\EN{\puts is also used here instead of \printf.}
\index{MIPS!\Instructions!LW}
\RU{Адрес функции \puts загружается в \$25 инструкцией LW (\q{Load Word}).}
\EN{The address of the \puts function is loaded into \$25 using LW the instruction (\q{Load Word}).}
\index{MIPS!\Instructions!LUI}
\index{MIPS!\Instructions!ADDIU}
\RU{Затем адрес текстовой строки загружается в \$4 парой инструкций LUI (\q{Load Upper Immediate}) и
ADDIU (\q{Add Immediate Unsigned Word}).}
\EN{Then the address of the text string is loaded to \$4 using LUI (\q{Load Upper Immediate}) and 
ADDIU (\q{Add Immediate Unsigned Word}) instruction pair.}
\RU{LUI устанавливает старшие 16 бит регистра (поэтому в имени инструкции присутствует \q{upper}) и ADDIU
прибавляет младшие 16 бит к адресу.}
\EN{LUI sets the high 16 bits of the register (hence \q{upper} word in instruction name) and ADDIU adds
the lower 16 bits of the address.}
\RU{ADDIU следует за JALR (помните о \IT{branch delay slots}?).}
\EN{ADDIU follows JALR (remember \IT{branch delay slots}?).}
\RU{Регистр \$4 также называется \$A0, который используется для передачи первого аргумента функции}%
\EN{The register \$4 is also called \$A0, which is used for passing the first function argument}%
\footnote{\RU{Таблица регистров в MIPS доступна в приложении}\EN{The MIPS registers table %
is available in appendix} \myref{MIPS_registers_ref}}.

\index{MIPS!\Instructions!JALR}
\RU{JALR (\q{Jump and Link Register}) делает переход по адресу в регистре \$25 (там адрес \puts) 
при этом сохраняя адрес следующей инструкции (LW) в \ac{RA}.}
\EN{JALR (\q{Jump and Link Register}) jumps to the address stored in the \$25 register (address of \puts) 
while saving the address of the next instruction (LW) in \ac{RA}.}
\RU{Это так же как и в ARM}\EN{This is very similar to ARM}.
\RU{И ещё одна важная вещь: адрес сохраняемый в \ac{RA} это адрес не следующей инструкции (потому что
это \IT{delay slot} и исполняется перед инструкцией перехода),
а инструкции после неё (после \IT{delay slot}).}
\EN{Oh, and one important thing is that the address saved in \ac{RA} is not the address of the next instruction (because
it's in a \IT{delay slot} and is executed before the jump instruction),
but the address of the instruction after the next one (after the \IT{delay slot}).}
\RU{Таким образом во время исполнения \TT{JALR} в \ac{RA} записывается $PC + 8$. В нашем случае это адрес
инструкции LW следующей после ADDIU.}
\EN{Hence, $PC + 8$ is written to \ac{RA} during the execution of \TT{JALR}, in our case, this is the address of the LW
instruction next to ADDIU.}

\RU{LW (\q{Load Word}) в строке 20 восстанавливает \ac{RA} из локального стека 
(эта инструкция скорее часть эпилога функции).}
\EN{LW (\q{Load Word}) at line 20 restores \ac{RA} from the local stack 
(this instruction is actually part of the function epilogue).}

\index{MIPS!\Pseudoinstructions!MOVE}
\RU{MOVE в строке 22 копирует значение из регистра \$0 (\$ZERO) в \$2 (\$V0).}
\EN{MOVE at line 22 copies the value from the \$0 (\$ZERO) register to \$2 (\$V0).}
\label{MIPS_zero_register}
\RU{В MIPS есть \IT{константный} регистр, всегда содержащий ноль.}
\EN{MIPS has a \IT{constant} register, which always holds zero.}
\RU{Должно быть, разработчики MIPS решили что 0 это самая востребованная константа в программировании,
так что пусть будет использоваться регистр \$0, всякий раз, когда будет нужен 0.}
\EN{Apparently, the MIPS developers came up with the idea that zero is in fact the busiest constant in the computer programming,
so let's just use the \$0 register every time zero is needed.}
\RU{Другой интересный факт: в MIPS нет инструкции, копирующей значения из регистра в регистр.}
\EN{Another interesting fact is that MIPS lacks an instruction that transfers data between registers.}
\RU{На самом деле}\EN{In fact}, \TT{MOVE DST, SRC} \RU{это}\EN{is} \TT{ADD DST, SRC, \$ZERO} ($DST=SRC+0$), 
\RU{которая делает тоже самое}\EN{which does the same}.
\RU{Очевидно, разработчики MIPS хотели сделать как можно более компактную таблицу опкодов.}
\EN{Apparently, the MIPS developers wanted to have a compact opcode table.}
\RU{Это не значит, что сложение происходит во время каждой инструкции MOVE.}
\EN{This does not mean an actual addition happens at each MOVE instruction.}
\RU{Скорее всего, эти псевдоинструкции оптимизируются в \ac{CPU} и \ac{ALU} никогда не используется.}
\EN{Most likely, the \ac{CPU} optimizes these pseudoinstructions and the \ac{ALU} is never used.}

\index{MIPS!\Instructions!J}
\RU{J в строке 24 делает переход по адресу в \ac{RA}, и это работает как выход из функции.}
\EN{J at line 24 jumps to the address in \ac{RA}, which is effectively performing a return from the function.}
\RU{ADDIU после J на самом деле исполняется перед J (помните о \IT{branch delay slots}?) 
и это часть эпилога функции.}
\EN{ADDIU after J is in fact executed before J (remember \IT{branch delay slots}?) 
and is part of the function epilogue.}

\RU{Вот листинг сгенерированный \IDA. Каждый регистр имеет свой псевдоним:}
\EN{Here is also a listing generated by \IDA. Each register here has its own pseudoname:}

\lstinputlisting[caption=\Optimizing GCC 4.4.5 (\IDA),numbers=left]{patterns/01_helloworld/MIPS/hw_O3_IDA.lst.\LANG}

\RU{Инструкция в строке 15 сохраняет GP в локальном стеке. Эта инструкция мистическим образом отсутствует
в листинге от GCC, может быть из-за ошибки в самом GCC\footnote{Очевидно, функция вывода листингов не так критична
для пользователей GCC, поэтому там вполне могут быть неисправленные ошибки.}.}
\EN{The instruction at line 15 saves the GP value into the local stack, and this instruction is missing mysteriously from the GCC output listing, maybe by a GCC error\footnote{Apparently, functions generating listings 
are not so critical to GCC users, so some unfixed errors may still exist.}.}
\RU{Значение GP должно быть сохранено, потому что всякая функция может работать со своим собственным окном данных
размером 64KiB.}
\EN{The GP value has to be saved indeed, because each function can use its own 64KiB data window.}

\RU{Регистр, содержащий адрес функции \puts называется \$T9, потому что регистры с префиксом T- называются
\q{temporaries} и их содержимое можно не сохранять.}
\EN{The register containing the \puts address is called \$T9, because registers prefixed with T- are called
\q{temporaries} and their contents may not be preserved.}

\subsection{\NonOptimizing GCC}

\NonOptimizing GCC \RU{более многословный}\EN{is more verbose}.

\lstinputlisting[caption=\NonOptimizing GCC 4.4.5 (\assemblyOutput),numbers=left]{patterns/01_helloworld/MIPS/hw_O0.s.\LANG}

\RU{Мы видим, что регистр FP используется как указатель на фрейм стека.}
\EN{We see here that register FP is used as a pointer to the stack frame.}
\RU{Мы также видим 3 \ac{NOP}-а.}\EN{We also see 3 \ac{NOP}s.}
\RU{Второй и третий следуют за инструкциями перехода.}
\EN{The second and third of which follow the branch instructions.}

\RU{Вероятно, компилятор GCC всегда добавляет \ac{NOP}-ы (из-за \IT{branch delay slots})
после инструкций переходов и затем, если включена оптимизация, от них может избавляться.}%
\EN{Perhaps, the GCC compiler always adds \ac{NOP}s (because of \IT{branch delay slots}) after branch
instructions and then, if optimization is turned on, maybe eliminates them.}
\RU{Так что они остались здесь}\EN{So in this case they are left here}.

\RU{Вот также листинг от \IDA:}
\EN{Here is also \IDA listing:}

\lstinputlisting[caption=\NonOptimizing GCC 4.4.5 (\IDA),numbers=left]{patterns/01_helloworld/MIPS/hw_O0_IDA.lst.\LANG}

\index{MIPS!\Pseudoinstructions!LA}
\RU{Интересно что \IDA распознала пару инструкций LUI/ADDIU и собрала их в одну псевдоинструкцию 
LA (\q{Load Address}) в строке 15.}
\EN{Interestingly, \IDA recognized the LUI/ADDIU instructions pair and coalesced them into one 
LA (\q{Load Address}) pseudoinstruction at line 15.}
\RU{Мы также видим, что размер этой псевдоинструкции 8 байт!}
\EN{We may also see that this pseudoinstruction has a size of 8 bytes!}
\RU{Это псевдоинструкция (или \IT{макрос}), потому что это не настоящая инструкция MIPS, а скорее
просто удобное имя для пары инструкций.}
\EN{This is a pseudoinstruction (or \IT{macro}) because it's not a real MIPS instruction, but rather
a handy name for an instruction pair.}

\index{MIPS!\Pseudoinstructions!NOP}
\index{MIPS!\Instructions!OR}
\RU{Ещё кое что: \IDA не распознала \ac{NOP}-инструкции в строках 22, 26 и 41.}
\EN{Another thing is that \IDA doesn't recognize \ac{NOP} instructions, so here they are at lines 22, 26 and 41.}
\RU{Это}\EN{It is} \TT{OR \$AT, \$ZERO}.
\RU{По своей сути это инструкция, применяющая операцию ИЛИ к содержимому регистра \$AT с нулем, что,
конечно же, холостая операция.}
\EN{Essentially, this instruction applies the OR operation to the contents of the \$AT register
with zero, which is, of course, an idle instruction.}
\RU{MIPS, как и многие другие \ac{ISA}, не имеет отдельной \ac{NOP}-инструкции.}
\EN{MIPS, like many other \ac{ISA}s, doesn't have a separate \ac{NOP} instruction.}

\subsection{\RU{Роль стекового фрейма в этом примере}\EN{Role of the stack frame in this example}}

\RU{Адрес текстовой строки передается в регистре.}
\EN{The address of the text string is passed in the register.}
\RU{Так зачем устанавливать локальный стек?}\EN{Why setup a local stack anyway?}
\RU{Причина в том, что значения регистров \ac{RA} и GP должны быть сохранены где-то
(потому что вызывается \printf) и для этого используется локальный стек.}
\EN{The reason for this lies in the fact that the values of registers \ac{RA} and GP have to be saved somewhere 
(because \printf is called), and the local stack is used for this purpose.}
\RU{Если бы это была \gls{leaf function}, тогда можно было бы избавиться от пролога и эпилога функции. Например:}
\EN{If this was a \gls{leaf function}, it would have been possible to get rid of the function prologue and epilogue,
for example:} \myref{MIPS_leaf_function_ex1}.

\subsection{\Optimizing GCC: \RU{загрузим в}\EN{load it into} GDB}

\index{GDB}
\lstinputlisting[caption=\RU{пример сессии в GDB}\EN{sample GDB session}]{patterns/01_helloworld/MIPS/O3_GDB.txt}


\subsection{\Conclusion{}}

The main difference between x86/ARM and x64/ARM64 code is that the pointer to the string is now 64-bits in length.
Indeed, modern \ac{CPU}s are now 64-bit due to both the reduced cost of memory and the greater demand for it by modern applications. 
We can add much more memory to our computers than 32-bit pointers are able to address.
As such, all pointers are now 64-bit.

% sections
\section{\Exercises}

\begin{itemize}
	\item \url{http://challenges.re/48}
	\item \url{http://challenges.re/49}
\end{itemize}


