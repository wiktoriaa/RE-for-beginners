\subsection{MSVC}

\RU{Компилируем в}\EN{Let's compile it in}\NL{We compileren het in} MSVC 2010:

\begin{lstlisting}
cl 1.cpp /Fa1.asm
\end{lstlisting}

\RU{(Ключ /Fa означает сгенерировать листинг на ассемблере)}%
\EN{(/Fa option instructs the compiler to generate assembly listing file)}%
\NL{(/Fa optie zorgt ervoor dat de compiler het bestand met assembly listing genereert)}%

\begin{lstlisting}[caption=MSVC 2010]
CONST	SEGMENT
$SG3830	DB	'hello, world', 0AH, 00H
CONST	ENDS
PUBLIC	_main
EXTRN	_printf:PROC
; Function compile flags: /Odtp
_TEXT	SEGMENT
_main	PROC
	push	ebp
	mov	ebp, esp
	push	OFFSET $SG3830
	call	_printf
	add	esp, 4
	xor	eax, eax
	pop	ebp
	ret	0
_main	ENDP
_TEXT	ENDS
\end{lstlisting}

\ifx\LITE\undefined
\RU{MSVC выдает листинги в синтаксисе Intel.}\EN{MSVC produces assembly listings in Intel-syntax.}\NLph{}
\RU{Разница между синтаксисом Intel и AT\&T будет рассмотрена немного позже:}
\EN{The difference between Intel-syntax and AT\&T-syntax will be discussed in} 
\NL{Het verschil tussen Intel-syntax en AT\&T-syntax zal besproken worden in:}\myref{ATT_syntax}.
\fi

\RU{Компилятор сгенерировал файл \TT{1.obj}, который впоследствии будет слинкован линкером в \TT{1.exe}.}%
\EN{The compiler generated the file, \TT{1.obj}, which is to be linked into \TT{1.exe}.}%
\NL{De compiler heeft het bestand, \TT{1.obj} gegenereerd, hetwelk gelinkt wordt tot \TT{1.exe}.}%
\RU{В нашем случае этот файл состоит из двух сегментов: \TT{CONST} (для данных-констант) и \TT{\_TEXT} (для кода).}%
\EN{In our case, the file contains two segments: \TT{CONST} (for data constants) and \TT{\_TEXT} (for code).}%
\NL{In ons geval bevat het bestand twee segmenten: \TT{CONST} (voor data constanten) en \TT{\_TEXT}(voor code).}%

\index{\CLanguageElements!const}
\label{string_is_const_char}
\RU{Строка \TT{hello, world} в \CCpp имеет тип \TT{const char[]} \cite[p176, 7.3.2]{TCPPPL}, 
однако не имеет имени.}%
\EN{The string \TT{hello, world} in \CCpp has type \TT{const char[]} \cite[p176, 7.3.2]{TCPPPL},
but it does not have its own name.}%
\NL{De string \TT{hello, world} in \CCpp is van het type \TT{const char[]} \cite[p176, 7.3.2]{TCPPPL},
maar heeft geen eigen naam.}%
\RU{Но компилятору нужно как-то с ней работать, поэтому он дает ей внутреннее имя \TT{\$SG3830}.}%
\EN{The compiler needs to deal with the string somehow so it defines the internal name \TT{\$SG3830} for it.}%
\NL{De compiler moet een manier hebben om met de string om te kunnen, en definieert er daarom de interne naam \TT{\$SG3830} voor.}%

\RU{Поэтому пример можно было бы переписать вот так}\EN{That is why the example may be rewritten as follows}\NL{Daarom kan het voorbeeld herschreven worden als volgt}:

\lstinputlisting{patterns/01_helloworld/hw_2.c}

\RU{Вернемся к листингу на ассемблере. Как видно, строка заканчивается нулевым байтом~--- это требования стандарта \CCpp для строк.}%
\EN{Let's go back to the assembly listing. As we can see, the string is terminated by a zero byte, which is standard for \CCpp strings.}%
\NL{Laten we terug gaan naar de assembly listing. Zoals je kan zien, wordt de string beeindigd door een nul-byte. Dit is standaard voor \CCpp strings.}%
\RU{Больше о строках в Си}\EN{More about C strings}\NL{Meer over C strings}: \myref{C_strings}.

\RU{В сегменте кода \TT{\_TEXT} находится пока только одна функция}%
\EN{In the code segment, \TT{\_TEXT}, there is only one function so far}%
\NL{In het code segment, \TT{\_TEXT}, is er slechts een functie tot nu toe}: \main.
\RU{Функция \main, как и практически все функции, начинается с пролога и заканчивается эпилогом}%
\EN{The function \main starts with prologue code and ends with epilogue code (like almost any function)}%
\NL{De functie \main begint met een proloog code en eindigt met een epiloog code (zoals bijna elke functie)}%
\footnote{\RU{Об этом смотрите подробнее в разделе о прологе и эпилоге функции}%
\EN{You can read more about it in the section about function prologues and epilogues}%
\NL{Je kan hier meer over lezen in de sectie over functieprologen en epilogen}%
~(\myref{sec:prologepilog}).}.

\index{x86!\Instructions!CALL}
\RU{Далее следует вызов функции \printf}
\EN{After the function prologue we see the call to the \printf function}
\NL{Na de functie proloog zien we de call naar de \printf functie}: \TT{CALL \_printf}. 
\index{x86!\Instructions!PUSH}
\RU{Перед этим вызовом адрес строки (или указатель на неё) с нашим приветствием при помощи инструкции \PUSH помещается в стек.}
\EN{Before the call the string address (or a pointer to it) containing our greeting is placed on the stack with the help of the \PUSH instruction.}
\NL{Voor de call wordt het adres van de string (of een pointer ernaar) die onze begroeting bevat, op de stack geplaatsd met de hulp van de \PUSH instructie.}

\RU{После того, как функция \printf возвращает управление в функцию \main, адрес строки (или указатель на неё) всё ещё лежит в стеке.}%
\EN{When the \printf function returns the control to the \main function, the string address (or a pointer to it) is still on the stack.}%
\NL{Wanneer de \printf functie de controle teruggeeft aan de \main functie, staat het string adres (of de pointer ernaar) nog steeds op de stack.}%
\RU{Так как он больше не нужен, то \glslink{stack pointer}{указатель стека} (регистр \ESP) корректируется.}%
\EN{Since we do not need it anymore, the \gls{stack pointer} (the \ESP register) needs to be corrected.}%
\NL{Aangezien we dit niet meer nodig hebben, moet de \gls{stack pointer} (het \ESP register) gecorrigeerd worden.}%

\index{x86!\Instructions!ADD}
\TT{ADD ESP, 4} \RU{означает прибавить 4 к значению в регистре \ESP.}
\EN{means add 4 to the \ESP register value.}
\NL{betekent dat er 4 wordt opgeteld bij de \ESP registerwaarde.}
\RU{Почему 4? Так как это 32-битный код, для передачи адреса нужно 4 байта. В x64-коде это 8 байт.}
\EN{Why 4? Since this is a 32-bit program, we need exactly 4 bytes for address passing through the stack. If it was x64 code we would need 8 bytes.}
\NL{Waarom 4? Aangezien dit een 32-bit programma is, hebben we exact 4 bytes nodig om een adres door te geven via de stack. als het x64 code was, zouden we 8 bytes nodig gehad hebben.}
\TT{ADD ESP, 4} \RU{эквивалентно \TT{POP регистр}, но без использования какого-либо регистра\footnote{Флаги
процессора, впрочем, модифицируются}.}
\EN{is effectively equivalent to \TT{POP register} but without using any register\footnote{CPU flags, however, are modified}.}
\NL{is een effectief equivalent voor \TT{POP register} maar zonder gebruik van een register\footnote{CPU flags worden echter wel aangepast}.}

\index{Intel C++}
\index{\oracle}
\index{x86!\Instructions!POP}

\RU{Некоторые компиляторы, например, Intel C++ Compiler, в этой же ситуации могут вместо 
\ADD сгенерировать \TT{POP ECX} (подобное можно встретить, например, в коде \oracle{}, им скомпилированном),
что почти то же самое, только портится значение в регистре \ECX.}
\EN{For the same purpose, some compilers (like the Intel C++ Compiler) may emit \TT{POP ECX} 
instead of \ADD (e.g., such a pattern can be observed in the \oracle{} code as it is compiled with the Intel C++ compiler).
This instruction has almost the same effect but the \ECX register contents will be overwritten.}
\NL{Met dezelfde reden zullen sommige compilers (zoals de Intel C++ Compiler) gebruik maken van \TT{POP ECX}
in plaats van \ADD (een dergelijk patroon kan waargenomen worden in de \oracle{} code aangezien deze gecompileerd is met de Intel C++ compiler).
Deze instructie heeft bijna hetzelfde effect, maar de inhoud van het \ECX register zal overschreven worden.}
\RU{Возможно, компилятор применяет \TT{POP ECX}, потому что эта инструкция короче (1 байт у \TT{POP} против 3 у \TT{ADD}).}
\EN{The Intel C++ compiler probably uses \TT{POP ECX} since this instruction's opcode is shorter than 
\TT{ADD ESP, x} (1 byte for \TT{POP} against 3 for \TT{ADD}).}
\NL{De Intel C++ Compiler gebruikt waarschijnlijk \TT{POP ECX} aangezien de opcode van deze instructie
korter is dan \TT{ADD ESP, x} (1 byte voor \TT{POP} tegen 3 voor \TT{ADD}).}

\RU{Вот пример использования \TT{POP} вместо \TT{ADD} из \oracle{}:}
\EN{Here is an example of using \TT{POP} instead of \TT{ADD} from \oracle{}:}
\NL{Hier is een voorbeeld van het gebruik van \TT{POP} in plaats van \TT{ADD} van \oracle{}:}

\begin{lstlisting}[caption=\oracle 10.2 Linux (\RU{файл }app.o\EN{ file}\NL{ bestand})]
.text:0800029A                 push    ebx
.text:0800029B                 call    qksfroChild
.text:080002A0                 pop     ecx
\end{lstlisting}

%\RU{О стеке можно прочитать в соответствующем разделе}
%\EN{Read more about the stack in section}
%\NL{Lees meer over de stack in de sectie} ~(\myref{sec:stack}).
\index{\CLanguageElements!return}
\RU{После вызова \printf в оригинальном коде на \CCpp указано \TT{return 0}~--- вернуть 0 
в качестве результата функции \main.}
\EN{After calling \printf, the original \CCpp code contains the statement \TT{return 0}~---return 0 as the result of the \main function.}
\NL{Na \printf aan te roepen, bevat de originele \CCpp code het statement \TT{return 0}~---return 0 als resultaat van de \main functie.}
\index{x86!\Instructions!XOR}
\RU{В сгенерированном коде это обеспечивается инструкцией \INS{XOR EAX, EAX}.}
\EN{In the generated code this is implemented by the instruction \INS{XOR EAX, EAX}.}
\NL{In de gegenereerde code wordt dit geimplementeerd door de instructie \INS{XOR EAX, EAX}.}
\index{x86!\Instructions!MOV}
\RU{\XOR, как легко догадаться~--- \q{исключающее ИЛИ}}%
\EN{\XOR is in fact just \q{eXclusive OR}}%
\NL{\XOR is feitelijk simpelweg \q{eXclusive OR}}%
\footnote{\href{http://go.yurichev.com/17118}{wikipedia}}
\RU{, но компиляторы часто используют его вместо простого}
\EN{but the compilers often use it instead of}
\NL{maar de compilers gebruiken het vaak in plaats van}
\TT{MOV EAX, 0}\EMDASH{}\RU{снова потому, что опкод короче (2 байта у \TT{XOR} против 5 у \TT{MOV}).}
\EN{again because it is a slightly shorter opcode (2 bytes for \TT{XOR} against 5 for \TT{MOV}).}
\NL{wederom omdat de opcode hiervoor iets korter is (2 bytes voor \TT{XOR} tegenover 5 voor \TT{MOV}).}

\index{x86!\Instructions!SUB}
\RU{Некоторые компиляторы генерируют}\EN{Some compilers emit}\NL{Sommige compilers gebruiken}
\INS{SUB EAX, EAX},
\RU{что значит \IT{отнять значение в} \EAX \IT{от значения в }\EAX,
что в любом случае даст 0 в результате.}
\EN{which means \IT{SUBtract the value in the} \EAX \IT{from the value in} \EAX,
which, in any case, results in zero.}
\NL{wat staat voor \IT{verminder de waarde in} \EAX \IT{met de waarde in} \EAX,
wat in elke situatie resulteert in nul.}

\index{x86!\Instructions!RET}
\RU{Самая последняя инструкция \RET возвращает управление в вызывающую функцию.
Обычно это код \CCpp \ac{CRT}, который, в свою очередь, 
вернёт управление операционной системе.}
\EN{The last instruction \RET returns the control to the \gls{caller}.
Usually, this is \CCpp \ac{CRT} code, which, in turn, returns control to the \ac{OS}.}
\NL{De laatste instructie \RET geeft de controle terug aan de \gls{caller}.
Gewoonlijk is dit \CCpp \ac{CRT} code, die op zijn beurt de controle teruggeeft aan het \ac{OS}.}

