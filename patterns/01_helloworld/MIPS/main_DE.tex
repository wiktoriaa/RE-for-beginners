\subsection{MIPS}

\subsubsection{ein Wort über \q{globale Zeiger}}
\label{MIPS_GP}

\myindex{MIPS!\GlobalPointer}

Ein wichtiges Konzept bei MIPS ist der \q{globale Zeiger}.
Wie bereits bekannt, besteht besteht jede MIPS-Anweisung aus 32 Bit, so dass es
nicht möglich ist eine 32-Bit-Anweisung darin unterzubringen: ein Anweisungspaar
wird verwendet (wie GCC dies in dem Beispiel zum Laden der Zeichenkettenadresse
getan hat).

Es ist jedoch möglich, Daten aus dem Adressbereich $register-32768...register+32767$
mit nur einer Anweisung zuladen, weil ein 16 Bit vorzeichenbehafteter Offset
in einer einzelnen Anweisung kodiert werden kann.
Es können also einige Register zu diesen Zweck alloziert werden und 64KiB-Bereiche
für die am häufigsten genutzten Daten
Dieses allozierte Register wird \q{globaler Zeiger} genannt und zeigt in die Mitte
des 64KiB-Bereichs.

Dieser Bereich enthält in der Regel globale Variablen und Adressen von importierten
Funktionen wie \printf, weil die GCC-Entwickler entschieden, dass das Laden einiger
Funktionsadressen so schnell sein sollte wie eine einzelne Anweisung anstatt zwei.
In einer ELF-Datei ist dieser 64KiB-Bereich teils in der Sektion .sbss (\q{small \ac{BSS}})
für uninitialiserte Daten und teil in .sdata (\q{small data}) für initialisierte
Daten zu finden.

Dies impliziert, dass der Programmierer entscheiden kann, auf welche Daten ein schneller
Zugriff (durch das Platzieren in .sdata/.sbss) möglich sein soll.
Einige Programmierer \q{der alten Schule} erinnern sich vielleicht an das MS-DOS
Speichermodell \myref{8086_memory_model} oder MS-DOS Speicherverwaltungen wie XMS/EMS
bei denen der komplette Speicher in 64KiB-Blöcke unterteilt war.

\myindex{PowerPC}

Dieses Konzept ist nicht nur bei MIPS vorhanden. Zumindest der PowerPC nutzt es ebenfalls.

\subsubsection{\Optimizing GCC}

Nachfolgen ein Beispiel welches das Konzept der \q{globalen Zeiger} veranschaulichen soll.

\lstinputlisting[caption=\Optimizing GCC 4.4.5 (\assemblyOutput),numbers=left,style=customasmMIPS]{patterns/01_helloworld/MIPS/hw_O3_DE.s}

Wie zu sehen wird das \$GP-Register im Funktionsprolog so gesetzt, dass auf die Mitte
dieses Bereichs gezeigt wird.
Das \ac{RA}-Register wird ebenfalls auf dem lokalen Stack gesichert.
Anstelle von \printf wird wieder \puts aufgerufen.
\myindex{MIPS!\Instructions!LW}
Die Adresse der Funktion \puts wird mit der \INS{LW}-Anweisung (\q{Load Word}) in \$25 geladen.
\myindex{MIPS!\Instructions!LUI}
\myindex{MIPS!\Instructions!ADDIU}
Anschließend wird die Adressse der Zeichenkette mit dem Anweisungspaar \INS{LUI} (\q{Load Upper Immediate})
und \INS{ADDIU} (\q{Add Immediate Unsigned Word}) in \$4 geladen.
\INS{LUI} setzt die  oberen 16 Bit des Registers (deswegen \q{upper} im Anweisungsnamen)
und \INS{ADDIU} addiert die unteren 16 Bit der Adresse.

\INS{ADDIU} folgt \INS{JALR} (zur Erinnerung: \IT{branch delay slots}).
Das Register \$4 wird auch \$A0 genannt und für das Übergeben des ersten Funktionsarguments
genutzt\footnote{Die MIPS-Register-Tabelle ist im Anhang verfügbar \myref{MIPS_registers_ref}}.

\myindex{MIPS!\Instructions!JALR}

\INS{JALR} (\q{Jump and Link Register}) springt zu der Adresse die im Register \$25
gespeichert ist (Adresse von \puts) und speichert die Adresse der übernächsten Anweisung
(LW) in \ac{RA}. Dies ist sehr ähnlich zu ARM.
Eine wichtige Sache ist, dass die Adresse in \ac{RA} nicht die Adresse der nächsten
Anweisung ist (da dies ein \IT{delay slot} ist und vor der Sprunganweisung ausgeführt wird),
sondern die Adresse der darauf folgenden Anweisung (nach dem \IT{delay slot}).
Da in diesem Fall während der Ausführung von \TT{JALR} der Wert $PC + 8$ in \ac{RA}
geschrieben wird, ist dies die Adresse der \INS{LW}-Anweisung nach \INS{ADDIU}.

\INS{LW} (\q{Load Word}) in Zeile 20 stellt \ac{RA} wieder vom lokalen Stack her.
Diese Anweisung ist tatsächlich ein Teil des Funktionsepilogs.

\myindex{MIPS!\Pseudoinstructions!MOVE}

\INS{MOVE} in Zeile 22 kopiert der Wert vom \$0 (\$ZERO)-Register in \$2 (\$V0).
\label{MIPS_zero_register}

MIPS besitzt ein \IT{konstantes} Register, welches immer eine Null beinhaltet.
Anscheinend hatten die MIPS-Entwickler die Idee, dass eine Null die beliebteste
Konstante in der Programmierung ist, also wird in Zukunft immer das \$0-Register
genutzt wenn eine Null benötigt wird.

Eine weitere interessante Tatsache in MIPS ist das Fehlen einer Anweisung zum Transferieren
von Daten zwischen zwei Registern.
Die Anweisung \TT{MOVE DST, SRC} entspricht jedoch \TT{ADD DST, SRC, \$ZERO} ($DST=SRC+0$),
und bewirkt genau das gleiche.
Anscheinend wollten die MIPS-Entwickler eine kompakte Opcode-Tabelle haben.
Das bedeutet nicht, dass dies bei jeder \INS{MOVE}-Anweisung passiert.
Sehr wahrscheinlich optimiert die \ac{CPU} diese Pseudo-Anweisung und die \ac{ALU}
wird niemals genutzt.

\myindex{MIPS!\Instructions!J}

\INS{J} in Zeile 24 springt zu der Adresse in \ac{RA}, was im Endeffekt einem Sprung aus einer Funktion entspricht.
\INS{ADDIU} nach \INS{J} wird tatsächlich bevor \INS{J} ausgeführt (siehe \IT{branch delay slots}) und ist
ein Teil des Funktions-Epilogs.
Hier ist die Ausgabe, die \IDA generiert. Jedes Register hat einen eigenen Pseudo-Namen:

\lstinputlisting[caption=\Optimizing GCC 4.4.5 (\IDA),numbers=left,style=customasmMIPS]{patterns/01_helloworld/MIPS/hw_O3_IDA_DE.lst}

Die Anweisung in Zeile 15 speichert den GP-Wert auf dem lokalen Stack. Diese Anweisung fällt seltsamerweise
beim GCC, was vielleicht auf einen Fehler des Compilers hinweist. \footnote{Anscheinend sind Funktionen die
Listings erzeugen nicht so kritisch für GCC-Nutzer, so dass vielleicht noch unbehobene Fehler existieren.}.
Der GP-Wert muss auch gespeichert werden weil jede Funktion ihren eigenen 64KiB-Datenbereich nutzen kann.
Das Register mit der Adresse von \puts wird \$T9, da Register mit Präfix T- temporäre Register sind deren
Inhalte nicht erhalten werden müssen.

\subsubsection{\NonOptimizing GCC}

\NonOptimizing GCC ist ausführlicher.

\lstinputlisting[caption=\NonOptimizing GCC 4.4.5 (\assemblyOutput),numbers=left,style=customasmMIPS]{patterns/01_helloworld/MIPS/hw_O0_DE.s}

Es ist zu sehen, dass das FP-Register als Zeiger zum Stack Frame genutzt wird.
Außerdem sind im Listing drei \ac{NOP}-Anweisungen.
Die zweite und dritte welche der Sprunganweisung folgt.
Möglicherweise fügt der GCC-Compiler immer \ac{NOP}-Anweisungen nach einer Sprung hinzu
(wegen der \IT{branch delay slots}) und entfernt diese wenn die Optimierung eingeschaltet ist.
In diesem Fall bleiben sie also bestehen.

Nachfolgend das \IDA-Listing:

\lstinputlisting[caption=\NonOptimizing GCC 4.4.5 (\IDA),numbers=left,style=customasmMIPS]{patterns/01_helloworld/MIPS/hw_O0_IDA_DE.lst}

\myindex{MIPS!\Pseudoinstructions!LA}

Interessanterweise kennt \IDA das Anweisungspaar \INS{LUI}/\INS{ADDIU} und fasst diese zu einer einzigen
Pseudoanweisung \INS{LA} (\q{Load Address}) zusammen (Zeile 15).
Es ist auch zu sehen, dass diese Pseudoanweisung eine Größe von 8 Byte hat!
Dies ist eine Pseudoanweisung (oder \IT{Makro}) weil es sich hier nicht um eine echte MIPS-Anweisung
handelt, sondern eher um einen handlichen Namen für ein Anweisungspaar.

\myindex{MIPS!\Pseudoinstructions!NOP}
\myindex{MIPS!\Instructions!OR}

Eine weitere Sache ist, dass \IDA keine \ac{NOP}-Anweisung kennt.
Also ist in den Zeilen 22, 26 und 41 \TT{OR \$AT, \$ZERO}.
Im Wesentlichen führt diese Anweisung eine ODER-Operation auf die Inhalte des \$AT-Register aus,
welche 0 ist. Dies entspricht natürlich einer Idle-Anweisung.
MIPS hat wie viele andere \ac{ISA} keine separate \ac{NOP}-Anweisung.

\subsubsection{Aufgabe des Stack Frames in diesem Beispiel}

Die Adresse dieser Zeichenkette ist in einem Register übergeben.
Warum wird dennoch der lokale Stack vorbereitet?
Der Grund dafür liegt in der Tatsache, dass die Werte der Register \ac{RA} und GP
wegen des Aufrugs von \printf irgendwo gesichert werden müssen und hier eben der
lokale Stack dafür genutzt wird.
Wenn dies eine \gls{leaf function} wäre, bestünde die Möglichkeit den Funktionsepilog
und -prolog wegzulassen, wie hier: \myref{MIPS_leaf_function_ex1}.

\subsubsection{\Optimizing GCC: in GDB laden}

\myindex{GDB}
\lstinputlisting[caption=sample GDB session]{patterns/01_helloworld/MIPS/O3_GDB.txt}
