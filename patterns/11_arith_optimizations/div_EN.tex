\subsection{Division}

\subsubsection{Division using shifts}
\label{division_by_shifting}

Example of division by 4:

\begin{lstlisting}[style=customc]
unsigned int f(unsigned int a)
{
	return a/4;
};
\end{lstlisting}

We get (MSVC 2010):

\begin{lstlisting}[caption=MSVC 2010,style=customasmx86]
_a$ = 8		; size = 4
_f	PROC
	mov	eax, DWORD PTR _a$[esp-4]
	shr	eax, 2
	ret	0
_f	ENDP
\end{lstlisting}

\label{SHR}
\myindex{x86!\Instructions!SHR}

The \SHR (\IT{SHift Right}) instruction in this example is shifting a number by 2 bits to the right.
The two freed bits at left (e.g., two most significant bits) are set to zero.
The two least significant bits are dropped.
In fact, these two dropped bits are the division operation remainder.

\myindex{x86!\Instructions!SHR}

The \SHR instruction works just like \SHL, but in the other direction.

\input{shift_right}

It is easy to understand if you imagine the number 23 in the decimal numeral system.
23 can be easily divided by 10 just by dropping last digit (3---division remainder). 
2 is left after the operation as a \gls{quotient}.

So the remainder is dropped, but that's OK, we work on integer values anyway, 
these are not a \glslink{real number}{real numbers}!

Division by 4 in ARM:

\begin{lstlisting}[caption=\NonOptimizingKeilVI (\ARMMode),style=customasmARM]
f PROC
        LSR      r0,r0,#2
        BX       lr
        ENDP
\end{lstlisting}

Division by 4 in MIPS:

\begin{lstlisting}[caption=\Optimizing GCC 4.4.5 (IDA),style=customasmMIPS]
        jr      $ra
        srl     $v0, $a0, 2 ; branch delay slot
\end{lstlisting}

\myindex{MIPS!\Instructions!SRL}
The SRL instruction is \q{Shift Right Logical}.
