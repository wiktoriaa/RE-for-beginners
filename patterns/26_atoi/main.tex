\ifdefined\RUSSIAN
\else

\chapter{String to number conversion (atoi())}

\index{\CStandardLibrary!atoi()}
Let's try to reimplement atoi() standard C function.

\section{Simple example}

Here is simplest possible way to read number represented in \ac{ASCII} form.
It's not error-prone: character other than digit will lead to incorrect result.

\lstinputlisting{patterns/26_atoi/atoi.c}

So what the algorithm does is just reads digits from left to right. 
A zero \ac{ASCII} character is subtracted from each digit. 
Digits from 0 to 9 are came subsequently in the \ac{ASCII} table, so, 
we may not even know exact '0' character value. 
All we need to know is that '0' minus '0' is 0, '9' minus '0' is 9 and so on.
Subtracting '0' from each character resulting a number from 0 to 9 inclusive.
Any other character will lead to incorrect result, of course!
Each digit is to be added to the final result (in ``rt'' variable), but final result
is also multiplied by 10 at each digit.
In other word, result is shifted left by one position in decimal form on each iteration.
The very last digit is added, but not shifted.

\subsection{\Optimizing MSVC 2013 x64}

\lstinputlisting[caption=\Optimizing MSVC 2013 x64]{patterns/26_atoi/atoi.asm.MSVC2013.x64.Ox}

Character is loaded in two places: first character and all subsequent characters.
This is done for loop regrouping.
\index{x86!\Instructions!LEA}
There are no instruction for multiplication by 10, two LEA instruction do this instead.
\index{x86!\Instructions!ADD}
\index{x86!\Instructions!SUB}
MSVC sometimes use ADD instruction with negative constant instead of SUB.
This is the case.
I truly don't know why this is better then SUB.
But MSVC does this often.

\subsection{\Optimizing GCC 4.9.1 x64}

\Optimizing GCC 4.9.1 is more concise, but there are one redundant RET instruction at the end.
One would be enough.

\lstinputlisting[caption=\Optimizing GCC 4.9.1 x64]{patterns/26_atoi/atoi.s.GCC491.O3.x64}

\ifdefined\IncludeARM
\subsection{\OptimizingKeilVI (\ARMMode)}

\lstinputlisting[caption=\OptimizingKeilVI (\ARMMode)]{patterns/26_atoi/atoi.s.ARM.O3}

\subsection{\OptimizingKeilVI (\ThumbMode)}

\lstinputlisting[caption=\OptimizingKeilVI (\ThumbMode)]{patterns/26_atoi/atoi.s.thumb.O3}

Interestingly, from school mathematics we may remember that order of addition and 
subtraction operations doesn't matter.
That's our case: first, $rt*10 - '0'$ expression is computed, then the input character value is added
to it.
Indeed, result is the same, but compiler probably did some regrouping.

\subsection{\Optimizing GCC 4.9.1 ARM64}

ARM64 compiler can use pre-increment instruction suffix:

\lstinputlisting[caption=\Optimizing GCC 4.9.1 ARM64]{patterns/26_atoi/atoi.s.GCC49.ARM64.O3}
\fi

\section{Slightly advanced example}

My new code snippet is more advanced, now it checks for minus sign at the very first character
and reports about error if non-digit was found in the input string:

\lstinputlisting{patterns/26_atoi/atoi2.c}

\subsection{\Optimizing GCC 4.9.1 x64}

\lstinputlisting[caption=\Optimizing GCC 4.9.1 x64]{patterns/26_atoi/atoi2.s.GCC491.O3.x64}

\index{x86!\Instructions!NEG}
If minus sign was encountered at the string begin, 
NEG instruction will be executed at the end.
It just negates a number.

There are also one thing to mention. 
How common programmer would check if the character is not a digit?
Just how I wrote in the source code:

\begin{lstlisting}
if (*s<'0' || *s>'9')
    ...
\end{lstlisting}

There are two comparison operations.
What is interesting is that we can replace both operations by single one: 
just subtract '0' from character value,
treat result as unsigned value (this is important) and check if it's greater than 9.

As an example, let's say, user input contain dot character (``.'') which have \ac{ASCII} code 46.
$46-48=-2$ if to treat result as signed number.
Indeed, dot character is located two places earlier than ``0'' character in the \ac{ASCII} table.
But it is \TT{0xFFFFFFFE} ($4294967294$) if to treat result as unsigned value, and that's definitely bigger than 9!

Compilers does this often, so it's important to recognize these tricks.

\Optimizing MSVC 2013 x64 does the same tricks.

\ifdefined\IncludeARM
\subsection{\OptimizingKeilVI (\ARMMode)}

\lstinputlisting[caption=\OptimizingKeilVI (\ARMMode),numbers=left]{patterns/26_atoi/atoi2.s.ARM.O3}

There are no NEG instruction in 32-bit ARM, so ``Reverse Subtraction'' operation (line 31) 
is used here.
It is triggered if the result of CMP instruction (at line 29) was ``Not Equal'' (hence -NE suffix).
\index{ARM!\Instructions!RSB}
So what RSBNE does is subtract resulting value from 0.
It works just like regular subtraction operation, but swaps operands.
Subtracting any number from 0 resulting negation: $0-x=-x$.

Thumb code is mostly the same.

\index{ARM!\Instructions!NEG}
GCC 4.9 for ARM64 can use NEG instruction, available in ARM64.
\fi
\fi
