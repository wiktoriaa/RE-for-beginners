\section{Computing arithmetic mean}

Let's imagine, I need to calculate arithmetic mean, and by some weird reason, 
I need to specify all values in function arguments.
This is just sum of all values divided by its number.

But it's impossible to get number of arguments in variadic function in \CCpp, so I'll denote 
value of $-1$ as a terminator.

There are standard stdarg.h file which define macros for dealing such type of arguments.
printf() and scanf() use it as well.

\lstinputlisting{patterns/25_variadic_functions/arith.c}

The very first argument should be treated just like usual argument.
\index{\CStandardLibrary!va\_arg}
All other arguments are loaded using \TT{va\_arg} macro and summed.

So what is inside?

\subsection{cdecl calling conventions}

\lstinputlisting[caption=\Optimizing MSVC 6.0]{patterns/25_variadic_functions/arith_MSVC60_Ox.asm}

The arguments, as we may see, is pushed in \main one-by-one.
The first argument is pushed first. Terminator value ($-1$) pushed last.

\TT{arith\_mean()} function takes first argument and stores it to $sum$ variable.
Then, it sets \EDX register to the address of the second argument, takes value from it, adds to $sum$,
and do this in infinite loop, until $-1$ is met.

When it's met, sum is divided by number of all values which were not $-1$ and the \gls{quotient} is returned.

So, in other words, I would say, the function treat stack fragment as an array of integer values of infinite
length.
Now we can understand why cdecl calling convention forcing to pass the first argument last. 
Because otherwise, it would not be possible to find the last value, or, for printf-like functions, it would not be
possible to find address of the format-string.

\subsection{Register-based calling conventions}

Observant reader may ask, what about calling convetions where first arguments are passed in registers?
Let's see:

\lstinputlisting[caption=\Optimizing MSVC 2012 x64]{patterns/25_variadic_functions/arith_MSVC_2012_Ox_x64.asm}

We see that first 4 arguments are passed in registers and two more --- in stack.
\TT{arith\_mean()} function first place these 4 arguments into \IT{Shadow Space} and then treat
\IT{Shadow Space} and stack behind it as a single continuous array!

What about GCC? Things are slightly clumsy here, because now the function is divided by two parts:
first part is saving registers into ``red zone'', workout that space, and the second part working out
the rest of stack:

\lstinputlisting[caption=\Optimizing GCC 4.9.1 x64]{patterns/25_variadic_functions/arith_GCC491_x64_O3.s}
