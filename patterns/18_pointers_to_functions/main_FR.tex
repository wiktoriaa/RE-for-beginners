\newcommand{\comp}{\TT{comp()}\xspace}
\mysection{Pointeurs sur des fonctions}
\label{sec:pointerstofunctions}

\myindex{\CLanguageElements!\Pointers}

Un pointeur sur une fonction, comme tout autre pointeur, est juste l'adresse de début
de la fonction dans son segment de code.

\myindex{Callbacks}
Ils sont souvent utilisés pour appeler des fonctions callback\footnote{\href{http://go.yurichev.com/17071}{Wikipédia}}
(de rappel).

Des exemples bien connus sont:

\begin{itemize}
\item \qsort\footnote{\href{http://go.yurichev.com/17072}{Wikipédia}},
{\TT{atexit()}}\footnote{\url{http://go.yurichev.com/17073}} de la bibliothèque C standard; 

\item signaux des OS *NIX\footnote{\href{http://go.yurichev.com/17074}{Wikipédia}};

\item démarrage de thread: \TT{CreateThread()} (win32), \TT{pthread\_create()} (POSIX);

\item beaucoup de fonctions win32, comme \TT{EnumChildWindows()}\footnote{\href{http://go.yurichev.com/17075}{MSDN}}.

\item dans de nombreux endroits du noyau Linux, par exemple les fonctions des drivers
du système de fichier sont appelées via callbacks: \url{http://go.yurichev.com/17076}

\item Les fonctions des plugins GCC sont aussi appelées via callbacks: 
\url{http://go.yurichev.com/17077}

\item Un autre exemple de pointeurs de fonction est une table dans le gestionnaire
de fenêtres Linux \q{dwm} qui définit les raccourcis. Chaque raccourci a une fonction
correspondante à appeler si une touche spécifiques est pressée: \href{http://go.yurichev.com/17078}{GitHub}
Comme on le voit, une telle table est plus pratique à gérer qu'une grande déclaration
switch().
\end{itemize}

\myindex{\CStandardLibrary!qsort()}

Donc, la fonction \qsort est une implémentation du tri rapide dans la bibliothèque
standard \CCpp.
La fonction est capable de trier n'importe quoi, tout type de données, tant que vous
avez une fonction pour comparer deux éléments et que \qsort est capable de l'appeler.

La fonction de comparaison peut être définie comme:

\begin{lstlisting}
int (*compare)(const void *, const void *)
\end{lstlisting}

Utilisons l'exemple suivant:

\lstinputlisting[numbers=left,label=qsort_c_src,style=customc]{patterns/18_pointers_to_functions/17_1.c}

\subsection{MSVC}

Compilons le dans MSVC 2010 (certaines parties ont été omises, dans un but de concision)
avec l'option \TT{\Ox}:

\lstinputlisting[caption=MSVC 2010 \Optimizing: /GS- /MD,style=customasmx86]{patterns/18_pointers_to_functions/17_2_msvc_Ox.asm}

Rien de surprenant jusqu'ici.
Comme quatrième argument, l'adresse du label \TT{\_comp} est passée, qui est juste
l'endroit où se trouve \comp, ou, en d'autres mots, l'adresse de la première instruction
de cette fonction.

Comment est-ce que \qsort l'appelle?

\myindex{Windows!MSVCR80.DLL}

Regardons cette fonction, située dans MSVCR80.DLL (un module DLL de MSVC avec des
fonctions de la bibliothèque C standard):

\lstinputlisting[caption=MSVCR80.DLL,style=customasmx86]{patterns/18_pointers_to_functions/17_3_MSVCR.lst}

\TT{comp}---est le quatrième argument de la fonction.
Ici, le contrôle est passé à l'adresse dans l'argument \TT{comp}.
Avant cela, deux arguments sont préparés pour \comp. Son résultat est testé après
son exécution.

C'est pourquoi il est dangereux d'utiliser des pointeurs sur des fonctions.
Tout d'abord, si vous appelez \qsort avec un pointeur de fonction incorrect, \qsort
peut passer le contrôle du flux à un point incorrect, le processus peut planter et
ce bug sera difficile à trouver.

La seconde raison est que les types de la fonction de callback doivent être strictement
conforme, appeler la mauvaise fonction avec de mauvais arguments du mauvais type peut
conduire à de sérieux problèmes, toutefois, le plantage du processus n'est pas un
problème ici~---le problème est de comment déterminer la raison du plantage~---car
le compilateur peut être silencieux sur le problème potentiel lors de la compilation.

\clearpage
\subsubsection{MSVC + \olly}
\myindex{\olly}

Chargeons notre exemple dans \olly et mettons un point d'arrêt sur \comp.
Nous voyons comment les valeurs sont comparées lors du premier appel de \comp:

\begin{figure}[H]
\centering
\myincludegraphics{patterns/18_pointers_to_functions/olly1.png}
\caption{\olly: premier appel de \comp}
\label{fig:qsort_olly1}
\end{figure}

\olly montre les valeurs comparées dans la fenêtre sous celle du code, par commodité.
Nous voyons que \ac{SP} pointe sur \ac{RA}, où se trouve la fonction \qsort (dans
\TT{MSVCR100.DLL}).

\clearpage
En traçant (F8) jusqu'à l'instruction \TT{RETN} et appuyant sur F8 une fois de plus,
nous retournons à la fonction \qsort:

\begin{figure}[H]
\centering
\myincludegraphics{patterns/18_pointers_to_functions/olly2.png}
\caption{\olly: le code dans \qsort juste après l'appel de \comp}
\label{fig:qsort_olly2}
\end{figure}

Ça a été un appel à la fonction de comparaison.

\clearpage
Voici aussi une copie d'écran au moment du second appel à\comp{}---maintenant les
valeurs qui doivent être comparées sont différentes:

\begin{figure}[H]
\centering
\myincludegraphics{patterns/18_pointers_to_functions/olly3.png}
\caption{\olly: second appel de \comp}
\label{fig:qsort_olly3}
\end{figure}


\subsubsection{MSVC + tracer}
\myindex{tracer}

Regardons quelles sont le paires comparées.
Ces 10 nombres vont être triés:
1892, 45, 200, -98, 4087, 5, -12345, 1087, 88, -100000.

Nous avons l'adresse de la première instruction \CMP dans \comp, c'est \TT{0x0040100C}
et nous y avons mis un point d'arrêt:

\begin{lstlisting}
tracer.exe -l:17_1.exe bpx=17_1.exe!0x0040100C
\end{lstlisting}

Maintenant nous avons des informations sur les registres au point d'arrêt:

\begin{lstlisting}
PID=4336|New process 17_1.exe
(0) 17_1.exe!0x40100c
EAX=0x00000764 EBX=0x0051f7c8 ECX=0x00000005 EDX=0x00000000
ESI=0x0051f7d8 EDI=0x0051f7b4 EBP=0x0051f794 ESP=0x0051f67c
EIP=0x0028100c
FLAGS=IF
(0) 17_1.exe!0x40100c
EAX=0x00000005 EBX=0x0051f7c8 ECX=0xfffe7960 EDX=0x00000000
ESI=0x0051f7d8 EDI=0x0051f7b4 EBP=0x0051f794 ESP=0x0051f67c
EIP=0x0028100c
FLAGS=PF ZF IF
(0) 17_1.exe!0x40100c
EAX=0x00000764 EBX=0x0051f7c8 ECX=0x00000005 EDX=0x00000000
ESI=0x0051f7d8 EDI=0x0051f7b4 EBP=0x0051f794 ESP=0x0051f67c
EIP=0x0028100c
FLAGS=CF PF ZF IF
...
\end{lstlisting}

Filtrons sur \TT{EAX} et \TT{ECX} et nous obtenons:

\begin{lstlisting}
EAX=0x00000764 ECX=0x00000005
EAX=0x00000005 ECX=0xfffe7960
EAX=0x00000764 ECX=0x00000005
EAX=0x0000002d ECX=0x00000005
EAX=0x00000058 ECX=0x00000005
EAX=0x0000043f ECX=0x00000005
EAX=0xffffcfc7 ECX=0x00000005
EAX=0x000000c8 ECX=0x00000005
EAX=0xffffff9e ECX=0x00000005
EAX=0x00000ff7 ECX=0x00000005
EAX=0x00000ff7 ECX=0x00000005
EAX=0xffffff9e ECX=0x00000005
EAX=0xffffff9e ECX=0x00000005
EAX=0xffffcfc7 ECX=0xfffe7960
EAX=0x00000005 ECX=0xffffcfc7
EAX=0xffffff9e ECX=0x00000005
EAX=0xffffcfc7 ECX=0xfffe7960
EAX=0xffffff9e ECX=0xffffcfc7
EAX=0xffffcfc7 ECX=0xfffe7960
EAX=0x000000c8 ECX=0x00000ff7
EAX=0x0000002d ECX=0x00000ff7
EAX=0x0000043f ECX=0x00000ff7
EAX=0x00000058 ECX=0x00000ff7
EAX=0x00000764 ECX=0x00000ff7
EAX=0x000000c8 ECX=0x00000764
EAX=0x0000002d ECX=0x00000764
EAX=0x0000043f ECX=0x00000764
EAX=0x00000058 ECX=0x00000764
EAX=0x000000c8 ECX=0x00000058
EAX=0x0000002d ECX=0x000000c8
EAX=0x0000043f ECX=0x000000c8
EAX=0x000000c8 ECX=0x00000058
EAX=0x0000002d ECX=0x000000c8
EAX=0x0000002d ECX=0x00000058
\end{lstlisting}

Il y a 34 paires.
C'est pourquoi, l'algorithme de tri rapide a besoin de 34 opérations de comparaison
pour trier ces 10 nombres.

\clearpage
\subsubsection{MSVC + tracer (couverture du code)}
\myindex{tracer}

Nous pouvons aussi utiliser la capacité du tracer pour collecter tous les registres
possible et les montrer dans \IDA.

Exécutons pas à pas toutes les instructions dans \comp:

\begin{lstlisting}
tracer.exe -l:17_1.exe bpf=17_1.exe!0x00401000,trace:cc
\end{lstlisting}

Nous obtenons un scipt .idc pour charger dans \IDA et chargeons le:

\begin{figure}[H]
\centering
\myincludegraphics{patterns/18_pointers_to_functions/tracer_cc.png}
\caption{tracer et IDA. N.B.: 
certaines valeurs sont coupées à droite}
\label{fig:qsort_tracer_cc}
\end{figure}

\IDA a donné un nom à la fonction (PtFuncCompare)---car \IDA voit que le pointeur sur
cette fonction est passé à \qsort.

Nous voyons que les pointeurs $a$ et $b$ pointent sur différents emplacements dans
le tableau, mais le pas entre eux est 4, puisque des valeurs 32-bit sont stockées
dans le tableau.

Nous voyons que les instructions en \TT{0x401010} et \TT{0x401012} ne sont jamais
exécutées (donc elles sont laissées en blanc): en effet, \comp ne renvoie jamais
0, car il n'y a pas d'éléments égaux dans le tableau.

\subsection{GCC}

Il n'y a pas beaucoup de différence:

\begin{lstlisting}[caption=GCC,style=customasmx86]
                lea     eax, [esp+40h+var_28]
                mov     [esp+40h+var_40], eax
                mov     [esp+40h+var_28], 764h
                mov     [esp+40h+var_24], 2Dh
                mov     [esp+40h+var_20], 0C8h
                mov     [esp+40h+var_1C], 0FFFFFF9Eh
                mov     [esp+40h+var_18], 0FF7h
                mov     [esp+40h+var_14], 5
                mov     [esp+40h+var_10], 0FFFFCFC7h
                mov     [esp+40h+var_C], 43Fh
                mov     [esp+40h+var_8], 58h
                mov     [esp+40h+var_4], 0FFFE7960h
                mov     [esp+40h+var_34], offset comp
                mov     [esp+40h+var_38], 4
                mov     [esp+40h+var_3C], 0Ah
                call    _qsort
\end{lstlisting}

Fonction \comp:

\begin{lstlisting}[style=customasmx86]
                public comp
comp            proc near

arg_0           = dword ptr  8
arg_4           = dword ptr  0Ch

                push    ebp
                mov     ebp, esp
                mov     eax, [ebp+arg_4]
                mov     ecx, [ebp+arg_0]
                mov     edx, [eax]
                xor     eax, eax
                cmp     [ecx], edx
                jnz     short loc_8048458
                pop     ebp
                retn
loc_8048458:
                setnl   al
                movzx   eax, al
                lea     eax, [eax+eax-1]
                pop     ebp
                retn
comp            endp
\end{lstlisting}

\myindex{Linux!libc.so.6}

L'implémentation de \qsort se trouve dans \TT{libc.so.6} et c'est en fait juste un
wrapper\footnote{un concept similaire à une \glslink{thunk function}{fonction thunk}}
pour \TT{qsort\_r()}.

À son tour, elle appelle \TT{quicksort()}, où notre fonction défini est appelée via
le pointeur passé:

\begin{lstlisting}[caption=(fihier libc.so.6{,} glibc version---2.10.1),style=customasmx86]
...
.text:0002DDF6                 mov     edx, [ebp+arg_10]
.text:0002DDF9                 mov     [esp+4], esi
.text:0002DDFD                 mov     [esp], edi
.text:0002DE00                 mov     [esp+8], edx
.text:0002DE04                 call    [ebp+arg_C]
...
\end{lstlisting}

\subsubsection{GCC + GDB (avec code source)}
\myindex{GDB}

Évidemment, nous avons le code source C de notre exemple (\myref{qsort_c_src}), donc
nous pouvons mettre un point d'arrêt ($b$) sur le numéro de ligne (11---la ligne
où la première comparaison se produit.
Nous devons aussi compiler l'exemple avec les informations de débogage incluses (\TT{-g}),
donc la table avec les adresses et les numéros de ligne correspondants est présente.

Nous pouvons aussi afficher les valeurs en utilisant les noms de variable (\TT{p}):
les informations de débogage nous disent aussi quel registre et/ou élément de la
pile locale contient quelle variable.

\myindex{Glibc}
Nous pouvons aussi voir la pile (\TT{bt}) et y trouver qu'il y a une fonction intermédiaire
\TT{msort\_with\_tmp()} utilisée dans la Glibc.

\lstinputlisting[caption=session GDB]{patterns/18_pointers_to_functions/GDB_source.txt}

\subsubsection{GCC + GDB (pas de code source)}
\myindex{GDB}

Mais souvent, il n'y a pas de code source du tout, donc nous pouvons désassembler
la fonction \comp (\TT{disas}), trouver la toute première instruction \CMP et placer
un point d'arrêt ($b$) à cette adresse.

À chaque point d'arrêt, nous allons afficher le contenu de tous les registres\\
(\TT{info registers}).
Le contenu de la pile est aussi disponible (\TT{bt}),

mais partiellement: il n'y a pas l'information du numéro de ligne pour \comp.

\lstinputlisting[caption=session GDB]{patterns/18_pointers_to_functions/GDB_no_source.txt}

\subsection{Danger des pointeurs sur des fonctions}

Comme nous pouvons le voir, la fonction \qsort attend un pointeur sur une fonction
qui prend deux arguments de type \IT{void*} et renvoie un entier:
Si vous avez plusieurs fonctions de comparaison dans votre code (une qui compare
les chaînes, une autre---les entiers, etc.), il est très facile de les mélanger les
unes avec les autres.
Vous pouvez essayer de trier un tableau de chaîne en utilisant une fonction qui compare
les entiers, et le compilateur ne vous avertira pas de ce bogue.

