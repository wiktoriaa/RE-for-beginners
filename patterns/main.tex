\chapter{\IFRU{Образцы кода}{Code patterns}}

\IFRU
{Когда я учил Си, а затем Си++, я просто писал небольшие фрагменты кода, компилировал и смотрел что 
получилось на ассемблере. Так было намного проще понять. Я делал это такое количество раз, 
что связь между кодом на \CCpp и тем, что генерирует компилятор, вбилась мне в подсознание достаточно 
глубоко, поэтому я могу, глядя на код на ассемблере, сразу понимать, в общих чертах, что там было написано 
на Си. Возможно это поможет кому-то ещё, попробую описать некоторые примеры.}
{When I first learned C and then C++, I wrote small pieces of code, compiled them, 
and saw what 
was produced in the assembly language. This was easy for me. I did it many times and the relation 
between the \CCpp code and what the compiler produced was imprinted in my mind so deep that 
I can quickly understand what was in the original C code when I look at produced x86 code. 
Perhaps this technique may be helpful for someone else so I will try to describe some examples here.}

\section{\IFRU{Краткое введенение в CPU}{Short introduction to the CPU}}

\IFRU{}{The} \ac{CPU} \IFRU{это собственно устройство исполняющее все программы}{is the unit which executes all of the programs}.

\IFRU{Немного терминологии}{Short glossary}:

\begin{description}
\item[\IFRU{Инструкция}{Instruction}]: \IFRU{примитивная команда}{a primitive command to the} \ac{CPU}.
\IFRU{Простейшие примеры: перемещение между регистрами, работа с памятью, примитивные арифметические операции}
{Simplest examples: moving data between registers, working with memory, arithmetic primitives}.
\IFRU{Как правило, каждый}{As a rule, each} \ac{CPU} \IFRU{имеет свой набор инструкций}{has its own instruction set architecture} 
(\ac{ISA}).

\item[\IFRU{Машинный код}{Machine code}]: \IFRU{код понимаемый}{code for the} \ac{CPU}. 
\IFRU{Каждая инструкция обычно кодируется несколькими байтами}{Each instruction is usually encoded by several bytes}.

\item[\IFRU{Язык ассемблера}{Assembly language}]: 
\IFRU{машинный код плюс некоторые расширения призванные облегчить труд программиста: макросы, итд}
{mnemonic code and some extensions like macros which are intended to make a programmer's life easier}.

\item[\IFRU{Регистр CPU}{CPU register}]: 
\IFRU{Каждый}{Each} \ac{CPU} \IFRU{имеет некоторый фиксированный набор регистров общего назначения}{has a fixed set
of general purpose registers} (\ac{GPR}).
$\approx 8$ \InENRU x86, $\approx 16$ \InENRU x86-64, $\approx 16$ \InENRU ARM.
\IFRU{Проще всего понимать регистр как временную переменную без типа}
{The easiest way to understand a register is to think of it as an untyped temporary variable}.
\IFRU{Можно представить что вы пишете на \ac{PL} высокого уровня и у вас только 8 переменных шириной 32 бита}
{Imagine you are working with a high-level \ac{PL} and you have only 8 32-bit variables}.
\IFRU{Можно сделать очень много используя только их}{A lot of things can be done using only these}!
\end{description}

\IFRU{Откуда взялась разница между машинным кодом и \ac{PL} высокого уровня}
{What is the difference between machine code and a \ac{PL}}?
\IFRU{Человеку проще писать на \ac{PL} высокого уровня вроде \CCpp, Java, Python, 
а \ac{CPU} проще работать с абстракциями куда более низкого уровня}
{It is much easier for humans to use a high-level \ac{PL} like \CCpp, Java, Python, etc., 
but it is easier for a \ac{CPU} to use a much lower level of abstraction}.
\IFRU{Возможно, можно было бы придумать \ac{CPU} исполняющий код \ac{PL} высокого уровня, но он был бы значительно сложнее}
{Perhaps, it would be possible to invent a \ac{CPU} which can execute high-level \ac{PL} code, but it would be much more complex}.
\IFRU{И наоборот, человеку очень неудобно писать на ассемблере из-за его низкоуровневости, к тому же, крайне трудно обойтись
без мелких ошибок}
{On the contrary, it is very inconvenient for humans to use assembly language due to its low-levelness. Besides, it is very hard
to do it without making a huge amount of annoying mistakes}.
\IFRU{Программа переводящая код из \ac{PL} высокого уровня в ассемблер называется \IT{компилятором}
\footnote{В более старой русскоязычной литературе также часто встречается термин ``транслятор''.}}
{The program which converts high-level \ac{PL} code into assembly is called a \IT{compiler}}.

% sections here:

\chapter{\HelloWorldSectionName}
\label{sec:helloworld}

\RU{Продолжим, используя знаменитый пример из книги}
\EN{Let's use the famous example from the book}
\NL{We bekijken het beroemde voorbeeld uit het boek}
``The C programming Language''\cite{Kernighan:1988:CPL:576122}:

\lstinputlisting{patterns/01_helloworld/hw.c}

\section{x86}

\subsection{MSVC}

\RU{Компилируем в}\EN{Let's compile it in}\NL{We compileren het in} MSVC 2010:

\begin{lstlisting}
cl 1.cpp /Fa1.asm
\end{lstlisting}

\RU{(Ключ /Fa означает сгенерировать листинг на ассемблере)}%
\EN{(/Fa option instructs the compiler to generate assembly listing file)}%
\NL{(/Fa optie zorgt ervoor dat de compiler het bestand met assembly listing genereert)}%

\begin{lstlisting}[caption=MSVC 2010]
CONST	SEGMENT
$SG3830	DB	'hello, world', 0AH, 00H
CONST	ENDS
PUBLIC	_main
EXTRN	_printf:PROC
; Function compile flags: /Odtp
_TEXT	SEGMENT
_main	PROC
	push	ebp
	mov	ebp, esp
	push	OFFSET $SG3830
	call	_printf
	add	esp, 4
	xor	eax, eax
	pop	ebp
	ret	0
_main	ENDP
_TEXT	ENDS
\end{lstlisting}

\ifx\LITE\undefined
\RU{MSVC выдает листинги в синтаксисе Intel.}\EN{MSVC produces assembly listings in Intel-syntax.}\NLph{}
\RU{Разница между синтаксисом Intel и AT\&T будет рассмотрена немного позже:}
\EN{The difference between Intel-syntax and AT\&T-syntax will be discussed in} 
\NL{Het verschil tussen Intel-syntax en AT\&T-syntax zal besproken worden in:}\myref{ATT_syntax}.
\fi

\RU{Компилятор сгенерировал файл \TT{1.obj}, который впоследствии будет слинкован линкером в \TT{1.exe}.}%
\EN{The compiler generated the file, \TT{1.obj}, which is to be linked into \TT{1.exe}.}%
\NL{De compiler heeft het bestand, \TT{1.obj} gegenereerd, hetwelk gelinkt wordt tot \TT{1.exe}.}%
\RU{В нашем случае этот файл состоит из двух сегментов: \TT{CONST} (для данных-констант) и \TT{\_TEXT} (для кода).}%
\EN{In our case, the file contains two segments: \TT{CONST} (for data constants) and \TT{\_TEXT} (for code).}%
\NL{In ons geval bevat het bestand twee segmenten: \TT{CONST} (voor data constanten) en \TT{\_TEXT}(voor code).}%

\index{\CLanguageElements!const}
\label{string_is_const_char}
\RU{Строка \TT{hello, world} в \CCpp имеет тип \TT{const char[]} \cite[p176, 7.3.2]{TCPPPL}, 
однако не имеет имени.}%
\EN{The string \TT{hello, world} in \CCpp has type \TT{const char[]} \cite[p176, 7.3.2]{TCPPPL},
but it does not have its own name.}%
\NL{De string \TT{hello, world} in \CCpp is van het type \TT{const char[]} \cite[p176, 7.3.2]{TCPPPL},
maar heeft geen eigen naam.}%
\RU{Но компилятору нужно как-то с ней работать, поэтому он дает ей внутреннее имя \TT{\$SG3830}.}%
\EN{The compiler needs to deal with the string somehow so it defines the internal name \TT{\$SG3830} for it.}%
\NL{De compiler moet een manier hebben om met de string om te kunnen, en definieert er daarom de interne naam \TT{\$SG3830} voor.}%

\RU{Поэтому пример можно было бы переписать вот так}\EN{That is why the example may be rewritten as follows}\NL{Daarom kan het voorbeeld herschreven worden als volgt}:

\lstinputlisting{patterns/01_helloworld/hw_2.c}

\RU{Вернемся к листингу на ассемблере. Как видно, строка заканчивается нулевым байтом~--- это требования стандарта \CCpp для строк.}%
\EN{Let's go back to the assembly listing. As we can see, the string is terminated by a zero byte, which is standard for \CCpp strings.}%
\NL{Laten we terug gaan naar de assembly listing. Zoals je kan zien, wordt de string beeindigd door een nul-byte. Dit is standaard voor \CCpp strings.}%
\RU{Больше о строках в Си}\EN{More about C strings}\NL{Meer over C strings}: \myref{C_strings}.

\RU{В сегменте кода \TT{\_TEXT} находится пока только одна функция}%
\EN{In the code segment, \TT{\_TEXT}, there is only one function so far}%
\NL{In het code segment, \TT{\_TEXT}, is er slechts een functie tot nu toe}: \main.
\RU{Функция \main, как и практически все функции, начинается с пролога и заканчивается эпилогом}%
\EN{The function \main starts with prologue code and ends with epilogue code (like almost any function)}%
\NL{De functie \main begint met een proloog code en eindigt met een epiloog code (zoals bijna elke functie)}%
\footnote{\RU{Об этом смотрите подробнее в разделе о прологе и эпилоге функции}%
\EN{You can read more about it in the section about function prologues and epilogues}%
\NL{Je kan hier meer over lezen in de sectie over functieprologen en epilogen}%
~(\myref{sec:prologepilog}).}.

\index{x86!\Instructions!CALL}
\RU{Далее следует вызов функции \printf}
\EN{After the function prologue we see the call to the \printf function}
\NL{Na de functie proloog zien we de call naar de \printf functie}: \TT{CALL \_printf}. 
\index{x86!\Instructions!PUSH}
\RU{Перед этим вызовом адрес строки (или указатель на неё) с нашим приветствием при помощи инструкции \PUSH помещается в стек.}
\EN{Before the call the string address (or a pointer to it) containing our greeting is placed on the stack with the help of the \PUSH instruction.}
\NL{Voor de call wordt het adres van de string (of een pointer ernaar) die onze begroeting bevat, op de stack geplaatsd met de hulp van de \PUSH instructie.}

\RU{После того, как функция \printf возвращает управление в функцию \main, адрес строки (или указатель на неё) всё ещё лежит в стеке.}%
\EN{When the \printf function returns the control to the \main function, the string address (or a pointer to it) is still on the stack.}%
\NL{Wanneer de \printf functie de controle teruggeeft aan de \main functie, staat het string adres (of de pointer ernaar) nog steeds op de stack.}%
\RU{Так как он больше не нужен, то \glslink{stack pointer}{указатель стека} (регистр \ESP) корректируется.}%
\EN{Since we do not need it anymore, the \gls{stack pointer} (the \ESP register) needs to be corrected.}%
\NL{Aangezien we dit niet meer nodig hebben, moet de \gls{stack pointer} (het \ESP register) gecorrigeerd worden.}%

\index{x86!\Instructions!ADD}
\TT{ADD ESP, 4} \RU{означает прибавить 4 к значению в регистре \ESP.}
\EN{means add 4 to the \ESP register value.}
\NL{betekent dat er 4 wordt opgeteld bij de \ESP registerwaarde.}
\RU{Почему 4? Так как это 32-битный код, для передачи адреса нужно 4 байта. В x64-коде это 8 байт.}
\EN{Why 4? Since this is a 32-bit program, we need exactly 4 bytes for address passing through the stack. If it was x64 code we would need 8 bytes.}
\NL{Waarom 4? Aangezien dit een 32-bit programma is, hebben we exact 4 bytes nodig om een adres door te geven via de stack. als het x64 code was, zouden we 8 bytes nodig gehad hebben.}
\TT{ADD ESP, 4} \RU{эквивалентно \TT{POP регистр}, но без использования какого-либо регистра\footnote{Флаги
процессора, впрочем, модифицируются}.}
\EN{is effectively equivalent to \TT{POP register} but without using any register\footnote{CPU flags, however, are modified}.}
\NL{is een effectief equivalent voor \TT{POP register} maar zonder gebruik van een register\footnote{CPU flags worden echter wel aangepast}.}

\index{Intel C++}
\index{\oracle}
\index{x86!\Instructions!POP}

\RU{Некоторые компиляторы, например, Intel C++ Compiler, в этой же ситуации могут вместо 
\ADD сгенерировать \TT{POP ECX} (подобное можно встретить, например, в коде \oracle{}, им скомпилированном),
что почти то же самое, только портится значение в регистре \ECX.}
\EN{For the same purpose, some compilers (like the Intel C++ Compiler) may emit \TT{POP ECX} 
instead of \ADD (e.g., such a pattern can be observed in the \oracle{} code as it is compiled with the Intel C++ compiler).
This instruction has almost the same effect but the \ECX register contents will be overwritten.}
\NL{Met dezelfde reden zullen sommige compilers (zoals de Intel C++ Compiler) gebruik maken van \TT{POP ECX}
in plaats van \ADD (een dergelijk patroon kan waargenomen worden in de \oracle{} code aangezien deze gecompileerd is met de Intel C++ compiler).
Deze instructie heeft bijna hetzelfde effect, maar de inhoud van het \ECX register zal overschreven worden.}
\RU{Возможно, компилятор применяет \TT{POP ECX}, потому что эта инструкция короче (1 байт у \TT{POP} против 3 у \TT{ADD}).}
\EN{The Intel C++ compiler probably uses \TT{POP ECX} since this instruction's opcode is shorter than 
\TT{ADD ESP, x} (1 byte for \TT{POP} against 3 for \TT{ADD}).}
\NL{De Intel C++ Compiler gebruikt waarschijnlijk \TT{POP ECX} aangezien de opcode van deze instructie
korter is dan \TT{ADD ESP, x} (1 byte voor \TT{POP} tegen 3 voor \TT{ADD}).}

\RU{Вот пример использования \TT{POP} вместо \TT{ADD} из \oracle{}:}
\EN{Here is an example of using \TT{POP} instead of \TT{ADD} from \oracle{}:}
\NL{Hier is een voorbeeld van het gebruik van \TT{POP} in plaats van \TT{ADD} van \oracle{}:}

\begin{lstlisting}[caption=\oracle 10.2 Linux (\RU{файл }app.o\EN{ file}\NL{ bestand})]
.text:0800029A                 push    ebx
.text:0800029B                 call    qksfroChild
.text:080002A0                 pop     ecx
\end{lstlisting}

%\RU{О стеке можно прочитать в соответствующем разделе}
%\EN{Read more about the stack in section}
%\NL{Lees meer over de stack in de sectie} ~(\myref{sec:stack}).
\index{\CLanguageElements!return}
\RU{После вызова \printf в оригинальном коде на \CCpp указано \TT{return 0}~--- вернуть 0 
в качестве результата функции \main.}
\EN{After calling \printf, the original \CCpp code contains the statement \TT{return 0}~---return 0 as the result of the \main function.}
\NL{Na \printf aan te roepen, bevat de originele \CCpp code het statement \TT{return 0}~---return 0 als resultaat van de \main functie.}
\index{x86!\Instructions!XOR}
\RU{В сгенерированном коде это обеспечивается инструкцией \INS{XOR EAX, EAX}.}
\EN{In the generated code this is implemented by the instruction \INS{XOR EAX, EAX}.}
\NL{In de gegenereerde code wordt dit geimplementeerd door de instructie \INS{XOR EAX, EAX}.}
\index{x86!\Instructions!MOV}
\RU{\XOR, как легко догадаться~--- \q{исключающее ИЛИ}}%
\EN{\XOR is in fact just \q{eXclusive OR}}%
\NL{\XOR is feitelijk simpelweg \q{eXclusive OR}}%
\footnote{\href{http://go.yurichev.com/17118}{wikipedia}}
\RU{, но компиляторы часто используют его вместо простого}
\EN{but the compilers often use it instead of}
\NL{maar de compilers gebruiken het vaak in plaats van}
\TT{MOV EAX, 0}\EMDASH{}\RU{снова потому, что опкод короче (2 байта у \TT{XOR} против 5 у \TT{MOV}).}
\EN{again because it is a slightly shorter opcode (2 bytes for \TT{XOR} against 5 for \TT{MOV}).}
\NL{wederom omdat de opcode hiervoor iets korter is (2 bytes voor \TT{XOR} tegenover 5 voor \TT{MOV}).}

\index{x86!\Instructions!SUB}
\RU{Некоторые компиляторы генерируют}\EN{Some compilers emit}\NL{Sommige compilers gebruiken}
\INS{SUB EAX, EAX},
\RU{что значит \IT{отнять значение в} \EAX \IT{от значения в }\EAX,
что в любом случае даст 0 в результате.}
\EN{which means \IT{SUBtract the value in the} \EAX \IT{from the value in} \EAX,
which, in any case, results in zero.}
\NL{wat staat voor \IT{verminder de waarde in} \EAX \IT{met de waarde in} \EAX,
wat in elke situatie resulteert in nul.}

\index{x86!\Instructions!RET}
\RU{Самая последняя инструкция \RET возвращает управление в вызывающую функцию.
Обычно это код \CCpp \ac{CRT}, который, в свою очередь, 
вернёт управление операционной системе.}
\EN{The last instruction \RET returns the control to the \gls{caller}.
Usually, this is \CCpp \ac{CRT} code, which, in turn, returns control to the \ac{OS}.}
\NL{De laatste instructie \RET geeft de controle terug aan de \gls{caller}.
Gewoonlijk is dit \CCpp \ac{CRT} code, die op zijn beurt de controle teruggeeft aan het \ac{OS}.}


\ifdefined\IncludeGCC
\EN{\subsubsection{GCC}

% The text states that GCC uses Intel syntax, but the footnote sounds like in needs to be activated
% Maybe edit the text to: GCC can produce Intel syntax (like MSVC), and the footnote to: Use the \TT{-S -masm=intel}.} to activate this
Now let's try to compile the same \CCpp code in the GCC 4.4.1 compiler in Linux: \TT{gcc 1.c -o 1}.
Next, with the assistance of the \IDA disassembler, let's see how the \main function was created.
\IDA, like MSVC, uses Intel-syntax\footnote{We could also have GCC produce assembly listings in Intel-syntax by applying the options \TT{-S -masm=intel}.}.

\begin{lstlisting}[caption=code in \IDA,style=customasmx86]
main            proc near

var_10          = dword ptr -10h

                push    ebp
                mov     ebp, esp
                and     esp, 0FFFFFFF0h
                sub     esp, 10h
                mov     eax, offset aHelloWorld ; "hello, world\n"
                mov     [esp+10h+var_10], eax
                call    _printf
                mov     eax, 0
                leave
                retn
main            endp
\end{lstlisting}

\myindex{Function prologue}
\myindex{x86!\Instructions!AND}
The result is almost the same.
The address of the \TT{hello, world} string (stored in the data segment) is loaded in the \EAX register first, and then saved onto the stack. \\
In addition, the function prologue has \INS{AND ESP, 0FFFFFFF0h}~---this
instruction aligns the \ESP register value on a 16-byte boundary.
This results in all values in the stack being aligned the same way (The CPU performs better if the values it is dealing with are located in memory at addresses aligned
on a 4-byte or 16-byte boundary)\footnote{\URLWPDA}.

\myindex{x86!\Instructions!SUB}
\INS{SUB ESP, 10h} allocates 16 bytes on the stack. Although, as we can see hereafter, only 4 are necessary here.

This is because the size of the allocated stack is also aligned on a 16-byte boundary.

% TODO1: rewrite.
\myindex{x86!\Instructions!PUSH}
The string address (or a pointer to the string) is then stored directly onto the stack without using the \PUSH instruction.
\IT{var\_10}~---is a local variable and is also an argument for \printf{}.
Read about it below.

Then the \printf function is called.

Unlike MSVC, when GCC is compiling without optimization turned on, it emits \TT{MOV EAX, 0} instead of a shorter opcode.

\myindex{x86!\Instructions!LEAVE}
The last instruction, \LEAVE~---is the equivalent of the \TT{MOV ESP, EBP} and \TT{POP EBP} instruction pair~---in other words, this instruction sets the \gls{stack pointer} (\ESP) back and restores the \EBP register to its initial state.
This is necessary since we modified these register values (\ESP and \EBP) at the beginning of the function (by executing \INS{MOV EBP, ESP} / \INS{AND ESP, \ldots}).

\subsubsection{GCC: \ATTSyntax}
\label{ATT_syntax}

Let's see how this can be represented in assembly language AT\&T syntax.
This syntax is much more popular in the UNIX-world.

\begin{lstlisting}[caption=let's compile in GCC 4.7.3]
gcc -S 1_1.c
\end{lstlisting}

We get this:

\lstinputlisting[caption=GCC 4.7.3,style=customasmx86]{patterns/01_helloworld/GCC.s}

The listing contains many macros (the parts that begin with a dot). These are not interesting for us at the moment.

For now, for the sake of simplicity, we can ignore them (except the \IT{.string} macro which
encodes a null-terminated character sequence just like a C-string). Then we'll see this
\footnote{This GCC option can be used to eliminate \q{unnecessary} macros: \IT{-fno-asynchronous-unwind-tables}}:

\lstinputlisting[caption=GCC 4.7.3,style=customasmx86]{patterns/01_helloworld/GCC_refined.s}

\myindex{\ATTSyntax}
\myindex{\IntelSyntax}
Some of the major differences between Intel and AT\&T syntax are:

\begin{itemize}

\item
Source and destination operands are written in opposite order.

In Intel-syntax: <instruction> <destination operand> <source operand>.

In AT\&T syntax: <instruction> <source operand> <destination operand>.

\myindex{\CStandardLibrary!memcpy()}
\myindex{\CStandardLibrary!strcpy()}
Here is an easy way to memorize the difference:
when you deal with Intel-syntax, you can imagine that there is an equality sign ($=$) between operands
and when you deal with AT\&T-syntax imagine there is a right arrow ($\rightarrow$)
\footnote{By the way, in some C standard functions (e.g., memcpy(), strcpy()) the arguments
are listed in the same way as in Intel-syntax: first the pointer to the destination memory block, and then
the pointer to the source memory block.}.

\item
AT\&T: Before register names, a percent sign must be written (\%) and before numbers a dollar sign (\$).
Parentheses are used instead of brackets.

\item
AT\&T: A suffix is added to instructions to define the operand size:

\begin{itemize}
\item q --- quad (64 bits)
\item l --- long (32 bits)
\item w --- word (16 bits)
\item b --- byte (8 bits)
\end{itemize}

% TODO1 simple example may be? \RU{Например mov\textbf{l}, movb, movw представляют различые версии инсструкция mov} \EN {For example: movl, movb, movw are variations of the mov instruction}

\end{itemize}

To go back to the compiled result: it is almost identical to what was displayed by \IDA.
There is one subtle difference: \TT{0FFFFFFF0h} is presented as \TT{\$-16}.
It's the same thing: \TT{16} in the decimal system is \TT{0x10} in hexadecimal.
\TT{-0x10} is equal to \TT{0xFFFFFFF0} (for a 32-bit data type).

\myindex{x86!\Instructions!MOV}
One more thing: the return value is set to 0 by using the usual \MOV, not \XOR.
\MOV just loads a value to a register.
Its name is a misnomer (as the data is not moved but rather copied). In other architectures, this instruction is named \q{LOAD} or \q{STORE} or something similar.
}
\RU{\subsubsection{GCC}

Теперь скомпилируем то же самое компилятором GCC 4.4.1 в Linux: \TT{gcc 1.c -o 1}.
Затем при помощи \IDA посмотрим как скомпилировалась функция \main.
\IDA, как и MSVC, показывает код в синтаксисе Intel\footnote{Мы также можем заставить GCC генерировать листинги в этом формате при помощи ключей \TT{-S -masm=intel}.}.

\begin{lstlisting}[caption=код в \IDA,style=customasmx86]
main            proc near

var_10          = dword ptr -10h

                push    ebp
                mov     ebp, esp
                and     esp, 0FFFFFFF0h
                sub     esp, 10h
                mov     eax, offset aHelloWorld ; "hello, world\n"
                mov     [esp+10h+var_10], eax
                call    _printf
                mov     eax, 0
                leave
                retn
main            endp
\end{lstlisting}

\myindex{Function prologue}
\myindex{x86!\Instructions!AND}
Почти то же самое. 
Адрес строки \TT{hello, world}, лежащей в сегменте данных, вначале сохраняется в \EAX, затем записывается в стек.
А ещё в прологе функции мы видим \TT{AND ESP, 0FFFFFFF0h}~--- 
эта инструкция выравнивает значение в \ESP по 16-байтной границе, делая все значения 
в стеке также выровненными по этой границе (процессор более эффективно работает с переменными, расположенными
в памяти по адресам кратным 4 или 16)\footnote{\URLWPDA}.

\myindex{x86!\Instructions!SUB}
\INS{SUB ESP, 10h} выделяет в стеке 16 байт. Хотя, как будет видно далее, здесь достаточно только 4.

Это происходит потому, что количество выделяемого места в локальном стеке тоже выровнено по 16-байтной границе.

% TODO1: rewrite.
\myindex{x86!\Instructions!PUSH}
Адрес строки (или указатель на строку) затем записывается прямо в стек без помощи инструкции \PUSH.
\IT{var\_10} одновременно и локальная переменная и аргумент для \printf{}. Подробнее об этом будет ниже.

Затем вызывается \printf.

В отличие от MSVC, GCC в компиляции без включенной оптимизации генерирует \TT{MOV EAX, 0} вместо более короткого опкода.

\myindex{x86!\Instructions!LEAVE}
Последняя инструкция \LEAVE~--- это аналог команд \TT{MOV ESP, EBP} и \TT{POP EBP}~--- то есть возврат \glslink{stack pointer}{указателя стека} и регистра \EBP в первоначальное состояние.
Это необходимо, т.к. в начале функции мы модифицировали регистры \ESP и \EBP{}\\
(при помощи \INS{MOV EBP, ESP} / \INS{AND ESP, \ldots}).

\subsubsection{GCC: \ATTSyntax}
\label{ATT_syntax}

Попробуем посмотреть, как выглядит то же самое в синтаксисе AT\&T языка ассемблера.
Этот синтаксис больше распространен в UNIX-мире.

\begin{lstlisting}[caption=компилируем в GCC 4.7.3]
gcc -S 1_1.c
\end{lstlisting}

Получим такой файл:

\lstinputlisting[caption=GCC 4.7.3,style=customasmx86]{patterns/01_helloworld/GCC.s}

Здесь много макросов (начинающихся с точки). Они нам пока не интересны.

Пока что, ради упрощения, мы можем 
их игнорировать (кроме макроса \IT{.string}, при помощи которого кодируется последовательность символов, 
оканчивающихся нулем~--- такие же строки как в Си). И тогда получится следующее
\footnote{Кстати, для уменьшения генерации \q{лишних} макросов, можно использовать такой ключ GCC: \IT{-fno-asynchronous-unwind-tables}}:

\lstinputlisting[caption=GCC 4.7.3,style=customasmx86]{patterns/01_helloworld/GCC_refined.s}

\myindex{\ATTSyntax}
\myindex{\IntelSyntax}
Основные отличия синтаксиса Intel и AT\&T следующие:

\begin{itemize}

\item
Операнды записываются наоборот.

В Intel-синтаксисе: \\
<инструкция> <операнд назначения> <операнд-источник>.

В AT\&T-синтаксисе: \\
<инструкция> <операнд-источник> <операнд назначения>.

\myindex{\CStandardLibrary!memcpy()}
\myindex{\CStandardLibrary!strcpy()}
Чтобы легче понимать разницу, можно запомнить следующее:
когда вы работаете с синтаксисом Intel~--- можете в уме ставить знак равенства ($=$) между операндами,
а когда с синтаксисом AT\&T~--- мысленно ставьте стрелку направо ($\rightarrow$)
\footnote{Кстати, в некоторых стандартных функциях библиотеки Си (например, memcpy(), strcpy()) также применяется 
расстановка аргументов как в синтаксисе Intel: вначале указатель в памяти на блок назначения, 
затем указатель на блок-источник.}.

\item
AT\&T: Перед именами регистров ставится символ процента (\%), а перед числами символ доллара (\$).
Вместо квадратных скобок используются круглые.

\item
AT\&T: К каждой инструкции добавляется специальный символ, определяющий тип данных:

\begin{itemize}
\item q --- quad (64 бита)
\item l --- long (32 бита)
\item w --- word (16 бит)
\item b --- byte (8 бит)
\end{itemize}

% TODO1 simple example may be? \RU{Например mov\textbf{l}, movb, movw представляют различые версии инсструкция mov} \EN {For example: movl, movb, movw are variations of the mov instruciton}

\end{itemize}

Возвращаясь к результату компиляции: он идентичен тому, который мы посмотрели в \IDA.
Одна мелочь: \TT{0FFFFFFF0h} записывается как \TT{\$-16}.
Это то же самое: \TT{16} в десятичной системе это \TT{0x10} в шестнадцатеричной.
\TT{-0x10} будет как раз \TT{0xFFFFFFF0} (в рамках 32-битных чисел).

\myindex{x86!\Instructions!MOV}
Возвращаемый результат устанавливается в 0 обычной инструкцией \MOV, а не \XOR.
\MOV просто загружает значение в регистр.
Её название не очень удачное (данные не перемещаются, а копируются). В других архитектурах подобная инструкция обычно носит название \q{LOAD} или \q{STORE} или что-то в этом роде.

}
\NL{\subsubsection{GCC}

Nu zullen we dezelfde \CCpp code compileren in de GCC 4.4.1 compiler in Linux: \TT{gcc 1.c -o 1}.
Vervolgens, met de assistentie van de \IDA disassembler, zullen we kijken hoe de \main functie gemaakt is.
\IDA, maakt net als MSVC gebruik van de Intel-syntax\footnote{We hadden GCC ook assembly listings kunnen laten gereren in Intel-syntax door gebruik te maken van de opties \TT{-S -masm=intel}.}.

\begin{lstlisting}[caption=code in \IDA,style=customasmx86]
main            proc near

var_10          = dword ptr -10h

                push    ebp
                mov     ebp, esp
                and     esp, 0FFFFFFF0h
                sub     esp, 10h
                mov     eax, offset aHelloWorld ; "hello, world\n"
                mov     [esp+10h+var_10], eax
                call    _printf
                mov     eax, 0
                leave
                retn
main            endp
\end{lstlisting}

\myindex{Function prologue}
\myindex{x86!\Instructions!AND}
Het resultaat is bijna hetzelfde.
Het adres van de \TT{hello, world} string (opgeslagen in het data segment) wordt eerst ingeladen in het \EAX register en wordt daarna opgeslagen op de stack.
Daarbovenop vind je in de functie proloog hetvolgende terug: \TT{AND ESP, 0FFFFFFF0h}~---
deze instructie lijnt de \ESP registerwaarde uit op een 16-byte begrenzing.
Dit resulteert in het feit dat alle waarden op de stack op dezelfde manier uitgelijnd worden.
De CPU presteert beter als de waarden die hij moet behandelen gelokaliseerd zijn in het geheugen op adressen die gealigneerd zijn op een 4-byte of 16-byte begrenzing.\footnote{URLWPDA}.

\myindex{x86!\Instructions!SUB}
\INS{SUB ESP, 10h} reserveert 16 bytes op de stack. Zoals we hierna echter kunnen zijn, zijn er in dit geval slechts 4 nodig.

Dit komt doordat de grootte van de gereserveerde stack ook uitgelijnd is op een 16-byte begrenzing.

% TODO1: rewrite.
\myindex{x86!\Instructions!PUSH}
Het string adres (of een pointer naar de string) wordt dan rechtstreeks op de stack geplaatst zonder gebruik te maken van de \PUSH instructie.
\IT{var\_10}~---is een lokale variabele en is ook een argument voor \printf{}.
Lees er hieronder meer over.

\NLph{}

In tegenstelling tot MSVC, wanneer GCC compileert zonder optimizatie, maakt het gebruik van \TT{MOV EAX, 0} in plaats van kortere opcodes.

\myindex{x86!\Instructions!LEAVE}
De laatste instructie, \LEAVE~---is het equivalent van het \TT{MOV ESP, EBP} en \TT{POP EBP} instructiepaar.
Met andere woorden, deze instructie zet de \gls{stack pointer} (\ESP) terug, en herstelt het \EBP register
terug tot zijn oorspronkelijke staat.
Dit is nodig aangezien we deze registerwaarden hebben gewijzigd (\ESP en \EBP) in het begin van de functie (door het uitvoeren van \INS{MOV EBP, ESP} / \INS{AND ESP, \ldots}).

\subsubsection{GCC: \ATTSyntax}
\label{ATT_syntax}

Laat ons eens kijken hoe dit kan weergegeven worden in assembly in de AT\&T syntax.
Deze syntax is veel populairder in de UNIX-wereld.

\begin{lstlisting}[caption=\NLph{} GCC 4.7.3]
gcc -S 1_1.c
\end{lstlisting}

We krijgen dit resultaat:

\lstinputlisting[caption=GCC 4.7.3,style=customasmx86]{patterns/01_helloworld/GCC.s}

De lijst bevat vele macros (die beginnen met een punt). Maar deze zijn niet interessant voor ons momenteel.

Voorlopig, om het simpel te houden, kunnen we deze negeren (buiten de \IT{.string} macro, dewelke
een null-terminated karakter reeks encodeert net als een C-string). Daarna zien we dit
\footnote{Deze GCC optie kan gebruikt worden om alle \q{onnodige} macros te elimineren: \IT{-fno-asynchronous-unwind-tables}}:

\lstinputlisting[caption=GCC 4.7.3,style=customasmx86]{patterns/01_helloworld/GCC_refined.s}

\myindex{\ATTSyntax}
\myindex{\IntelSyntax}
Sommige grote verschillen tussen de Intel en AT\&T syntax zijn:

\begin{itemize}

\item
\NLph{}

In Intel-syntax: <instructie> <doel> <bron>.

In AT\&T syntax: <instructie> <bron> <doel>.

\myindex{\CStandardLibrary!memcpy()}
\myindex{\CStandardLibrary!strcpy()}
Een gemakkelijke manier om dit verschil te onthouden is: 
Wanneer je met Intel-syntax te doen krijgt, kan je je inbeelden dat er een gelijkheidsteken ($=$) staat tussen de operands
en met AT\&T-syntax beeld je je in dat er een pijl naar rechts staat ($\rightarrow$)
\footnote{Trouwens, in sommige C standaard functies (bv. memcpy(), strcpy()) worden
de argumenten opgelijst op dezelfde manier als in Intel-syntax: eerst een pointer naar het bestemmings geheugen block, 
gevolgd door een pointer naar de bron.}.

\item
AT\&T: Voor registernamen moet een percentteken geschreven worden (\%) en voor cijfers een dollarteken (\$).
Ronde haakjes worden gebruikt in plaats van haakjes.

\item
AT\&T: Een suffix wordt toegevoegd aan de instructies om de operand grootte te bepalen:

\begin{itemize}
\item q --- quad (64 bits)
\item l --- long (32 bits)
\item w --- word (16 bits)
\item b --- byte (8 bits)
\end{itemize}

\end{itemize}

Laten we even terugblikken op het gecompileerde resultaat: dit is identiek als wat we gezien hebben in \IDA.
Met een klein verschil: \TT{0FFFFFFF0h} wordt weergegeven als \TT{\$-16}.
Dit is hetzelfde: \TT{16} in het decimaalsysteem is \TT{0x10} in hexadecimal.
\TT{-0x10} is gelijk aan \TT{0xFFFFFFF0} (voor een 32-bit data type).

\myindex{x86!\Instructions!MOV}
Nog een ding: de return value wordt best op 0 gezet door gebruik te maken van \MOV, niet van \XOR.
\MOV laadt gewoon een waarde in het register.
De naam is een foute noemer (data wordt niet verplaatst, maar eerder gekopieerd). In andere architecturen wordt deze instructie \q{LOAD} of \q{STORE} of iets soortgelijks genoemd.

}
\ITA{\subsubsection{GCC}

Proviamo adesso a compilare lo stesso codice \CCpp con il compilatore GCC 4.4.1 su Linux: \TT{gcc 1.c -o 1}.
Successivamente, con l'aiuto del disassembler \IDA, vediamo come è stata creata la funzione \main .
\IDA, come MSVC, utilizza la sintassi Intel\footnote{Possiamo anche fare in modo che GCC produca un listato assembly con la sintassi Intel tramite l'opzione \TT{-S -masm=intel}.}.

\begin{lstlisting}[caption=codice in \IDA,style=customasmx86]
main            proc near

var_10          = dword ptr -10h

                push    ebp
                mov     ebp, esp
                and     esp, 0FFFFFFF0h
                sub     esp, 10h
                mov     eax, offset aHelloWorld ; "hello, world\n"
                mov     [esp+10h+var_10], eax
                call    _printf
                mov     eax, 0
                leave
                retn
main            endp
\end{lstlisting}

\myindex{Function prologue}
\myindex{x86!\Instructions!AND}
Il risultato è pressoché lo stesso.
L'indirizzo della stringa \TT{hello, world} (memorizzato nel data segment) è caricato prima nel registro \EAX e successivamente salvato sullo stack.
Inoltre, il prologo della funzione contiene \TT{AND ESP, 0FFFFFFF0h}~---questa 
istruzione allinea il valore del registro \ESP a 16-byte.
Ciò fa sì che tutti i valori sullo stack siano allineati allo stesso modo (la CPU è più efficiente se i valori che tratta sono collocati in memoria ad indirizzi allineati a, ovvero multipli di, 4 o 16 byte)\footnote{\URLWPDA}.

\myindex{x86!\Instructions!SUB}
\INS{SUB ESP, 10h} alloca 16 byte sullo stack. Tuttavia, come vedremo a breve, solo 4 sono necessari in questo caso.

Ciò è dovuto al fatto che la dimensione dello stack allocato è anch'essa allineata a 16 byte.

% TODO1: rewrite.
\myindex{x86!\Instructions!PUSH}
L'indirizzo della stringa (o un puntatore alla stringa) è quindi memorizzato direttamente sullo stack senza utilizzare l'istruzione \PUSH .
\IT{var\_10}~--- è una variabile locale ed è anche un argomento di \printf{}.
Maggiori dettagli in seguito.

Infine viene chiamata la funzione \printf.

Diversamente da MSVC, quando GCC compila senza ottimizzazione emette \TT{MOV EAX, 0} invece di un opcode più breve.

\myindex{x86!\Instructions!LEAVE}
L'ultima istruzione, \LEAVE~---è l'equivalente della coppia di istruzioni \TT{MOV ESP, EBP} e \TT{POP EBP} ~---in altre parole, questa istruzione riporta indietro lo \gls{stack pointer} (\ESP) e ripristina il registro \EBP al suo stato iniziale.
Ciò è necessario poiché abbiamo modificato i valori di questi registri (\ESP and \EBP) all'inizio della funzione ( eseguendo \INS{MOV EBP, ESP} / \INS{AND ESP, \ldots}).

\subsubsection{GCC: \ATTSyntax}
\label{ATT_syntax}

Vediamo come tutto questo può essere rappresentato nella sintassi assembly AT\&T.
Questa sintassi è molto più popolare nel mondo UNIX.

\begin{lstlisting}[caption=compiliamo in GCC 4.7.3]
gcc -S 1_1.c
\end{lstlisting}

Otteniamo questo:

\lstinputlisting[caption=GCC 4.7.3,style=customasmx86]{patterns/01_helloworld/GCC.s}

Il listato contiene molte macro (iniziano con il punto). Per il momento non ci interessano.

Per il momento, e solo per una questione di semplificazione, possiamo ignorarle (fatta eccezione per la macro \IT{.string} che codifica una sequenza di caratteri che termina con il null-byte (zero) proprio come una stringa C). Consideriamo soltanto questo
\footnote{Questa opzione di GCC può essere usata per eliminare le macro \q{superflue}: \IT{-fno-asynchronous-unwind-tables}}:

\lstinputlisting[caption=GCC 4.7.3,style=customasmx86]{patterns/01_helloworld/GCC_refined.s}

\myindex{\ATTSyntax}
\myindex{\IntelSyntax}
Alcune delle differenze maggiori tra la sintassi Intel e quella AT\&T sono:

\begin{itemize}

\item
\ITAph{}

Sintassi Intel: <istruzione> <operando di destinazione> <operando di origine>.

Sintassi AT\&T: <istruzione> <operando di origine> <operando di destinazione>.

\myindex{\CStandardLibrary!memcpy()}
\myindex{\CStandardLibrary!strcpy()}
Ecco un modo facile per memorizzare la differenza:
quando si tratta di sintassi Intel immagina che ci sia un segno di uguaglianza ($=$) tra i due operandi, quando si tratta di sintassi AT\&T immagina una freccia da sinistra a destra ($\rightarrow$)
\footnote{A proposito, in alcune funzioni standard C(es., memcpy(), strcpy()) gli argomenti sono elencati nello stesso modo della sintassi Intel: prima il puntatore al blocco di memoria di destinazione, e poi il puntatore al blocco di memoria di origine.}.

\item
AT\&T: Il simbolo di percentuale (\%) deve essere scritto prima del nome di un registro, e il dollaro (\$) prima dei numeri.

\item
AT\&T: All'istruzione si aggiunge un suffisso che definisce le dimensioni dell'operando:

\begin{itemize}
\item q --- quad (64 bit)
\item l --- long (32 bit)
\item w --- word (16 bit)
\item b --- byte (8 bit)
\end{itemize}

\end{itemize}

Torniamo al risultato compilato: è identico a quello che abbiamo visto in \IDA.
Con una piccola differenza: \TT{0FFFFFFF0h} è presentato come \TT{\$-16}.
E' la stessa cosa: \TT{16} nel sistema decimale è \TT{0x10} in esadecimale.
\TT{-0x10} è uguale a \TT{0xFFFFFFF0} (per un tipo di dato a 32-bit).

\myindex{x86!\Instructions!MOV}
Ancora una cosa: il valore di ritorno viene settato a 0 usando \MOV, non \XOR.
\MOV semplicemente carica un valore in un registro.
Il suo nome è fuorviante (il dato non viene spostato, bensì copiato). In altre architectures questa istruzione è chiamata \q{LOAD} o \q{STORE} o qualcosa di simile.

}
\DE{\subsubsection{GCC}

Als nächstes wird der gleiche \CCpp-Code mit GCC 4.4.1 unter Linux kompiliert: \TT{gcc 1.c -o 1}.
Mithilfe des \IDA-Disassemblers wird untersucht, wie die \main-Funktion erzeugt wurde.
\IDA nutzt, genau wie MSVX den Intel-Syntax\footnote{GCC kann Assembler-Ausgaben im Intel-Syntax erzeugen mit der Options \TT{-S -masm=intel}.}.

\begin{lstlisting}[caption=Code in \IDA,style=customasmx86]
main            proc near

var_10          = dword ptr -10h

                push    ebp
                mov     ebp, esp
                and     esp, 0FFFFFFF0h
                sub     esp, 10h
                mov     eax, offset aHelloWorld ; "hello, world\n"
                mov     [esp+10h+var_10], eax
                call    _printf
                mov     eax, 0
                leave
                retn
main            endp
\end{lstlisting}

\myindex{Function prologue}
\myindex{x86!\Instructions!AND}
Das Ergebnis ist fast das gleiche.
Die Adresse der \TT{hello, world}-Zeichenkette (im Daten-Segment) wird zunächst in das \EAX-Register geladen und anschließend auf dem Stack gesichert.\\
Zusätzlich beinhaltet der Funktions-Prolog \INS{AND ESP, 0FFFFFFF0h}~---diese
Anweisung richtet den \ESP-Register-Wert an eine 16-Byte-Grenze aus.
Dies führt dazu, dass alle Werte im Stack auf die gleiche Weise ausgerichtet sind.
Die CPU kann Anweisungen schneller ausführen, wenn die zu verarbeitenden Daten auf einer an 4- oder 16-Byte-Grenzen ausgerichteten Adresse liegen\footnote{\URLWPDA}.

\myindex{x86!\Instructions!SUB}
\INS{SUB ESP, 10h} reserviert 16 Byte auf dem Stack, auch wenn - wie später gezeigt wird - nur 4 Byte benötigt werden.

Der Grund liegt darin, dass auch die Größe des Stacks an eine 16-Byte-Grenze ausgerichtet ist.

% TODO1: rewrite.
\myindex{x86!\Instructions!PUSH}
Die Adresse der Zeichenkette (oder ein Zeiger darauf) wird anschließend direkt ohne die \PUSH-Anweisung auf dem Stack gespeichert.
IT{var\_10}~---ist eine lokale Variable und ein Argument für \printf{}.
Mehr dazu später.

Anschließend wird die \printf-Funktion aufgerufen.

Anders als MSVC erzeugt GCC ohne Optimierung Die Anweisung \TT{MOV EAX, 0} anstatt des kürzeren OpCodes.

\myindex{x86!\Instructions!LEAVE}
Die letzte Anweisung \LEAVE ist ein Äquivalent zu der Kombination aus \TT{MOV ESP, EBP} und \TT{POP EBP}.
Mit anderen Worten: diese Anweisung setzt den \gls{stack pointer} (\ESP) zurück und stellt die initalen Werte des \EBP-Registers wieder her.
Dies ist notwendig weil die Registerwerte (\ESP und \EBP) zu Beginn der Funktion (durch \INS{MOV EBP, ESP} / \INS{AND ESP, \ldots}).

\subsubsection{GCC: \ATTSyntax}
\label{ATT_syntax}

Im nächsten Beispiel ist sichtbar, wie dies im AT\%T-Syntax dargestellt werden kann.
Dieser Syntax ist sehr viel populärer in der UNIX-Welt.

\begin{lstlisting}[caption=Das Beispiel kompiliert mit GCC 4.7.3]
gcc -S 1_1.c
\end{lstlisting}

Das Ergebnis ist wie folgt:

\lstinputlisting[caption=GCC 4.7.3,style=customasmx86]{patterns/01_helloworld/GCC.s}

Der Quellcode beinhaltet Makros (beginnend mit einem Punkt), die hier aber nicht von Belang sind.

An dieser Stelle werden aus Gründen der Übersichtlichkeit alle Makros au0er \IT{.string}
ignoriert. Letzeres kodiert eine Null-terminierte Zeichenkette, die einem C-String entspricht.

Die resultierende Ausgabe ist diese
\footnote{Um die \q{unnötigen} Makros zu unterdrücen kann die GCC-Option \IT{-fno-asynchronous-unwind-tables} genutzt werden}:

\lstinputlisting[caption=GCC 4.7.3,style=customasmx86]{patterns/01_helloworld/GCC_refined.s}

\myindex{\ATTSyntax}
\myindex{\IntelSyntax}
Einige der Hauptunterschiede zwischen Intel und AT\&T-Syntax sin:

\begin{itemize}

\item
Quell- und Zieloperanden sind in umgekehrter Reihenfolge angegeben.

Im Intel-Syntax: <Anweisung> <Ziel-Operand> <Quell-Operand>.

Im AT\&T-Syntax: <Anweisung> <Quell-Operand> <Ziel-Operand>.

\myindex{\CStandardLibrary!memcpy()}
\myindex{\CStandardLibrary!strcpy()}
Hier ist eine einfache Möglichkeit um sich den Unterschied zu merken:
Beim Umgang mit dem Intel-Syntax, kann man sich ein Gleichheitszeichen ($=$) zwischen den Operanden vorstellen
und beim AT\&T-Syntax einen Pfeil nach rechts ($\rightarrow$)
\footnote{Einige C-Standard-Funktionen (z.B. memcpy(), strcpy()) sind die Parameter ebenfalls wie im
Intel-Syntax aufgelistet: erst der Zeiger zum Ziel, dann der Zeiger auf die Speicher-Quelle)}.

\item
AT\&T: Vor einem Register-Namen muss ein Prozentzeichen (\%) und vor Zahlen ein Dollarzeichen (\$) stehen.
Statt eckigen werden runde Klammern genutzt.

\item
AT\&T: An eine Anweisung ist ein Suffix angehängt, der die Operandengröße angibt:

\begin{itemize}
\item q --- quad (64 bits)
\item l --- long (32 bits)
\item w --- word (16 bits)
\item b --- byte (8 bits)
\end{itemize}

% TODO1 simple example may be? \RU{Например mov\textbf{l}, movb, movw представляют различые версии инсструкция mov} \EN {For example: movl, movb, movw are variations of the mov instruciton} \DE {Zum Beispiel sind movl, movb und movw Variationen der mov-Anweisung}

\end{itemize}

Nochmals zu dem kompilierten Ergebnis: Dieses ist identisch mit der Anzeige in \IDA,
jedoch mit einem kleinen Unterschied: \TT{0FFFFFFF0h} wird als \TT{\$-16} angezeigt.
Der eigentliche Wert ist der selbe: \TT{16} im Dezimalsystem ist \TT{0x10} im Hexadezimalsystem.
Für 32-Bit-Datentypen ist \TT{-0x10} identisch mit \TT{0xFFFFFFF0}.

\myindex{x86!\Instructions!MOV}
Eine weitere Sache: der Rückgabewert ist mittels \MOV auf Null gesetzt, nicht mit \XOR.
\MOV läd lediglich einen Wert in ein Register.
Der Name ist irreführend, da die Daten nicht verschoben, sondern kopiert werden.
In anderen Architekturen ist wird dieser Befehl \q{LOAD} oder \q{STORE} oder ähnlich genannt.
}

\fi

\section{x86-64}
\EN{\subsubsection{MSVC: x86-64}

\myindex{x86-64}
Let's also try 64-bit MSVC:

\lstinputlisting[caption=MSVC 2012 x64,style=customasmx86]{patterns/01_helloworld/MSVC_x64.asm}

\myindex{fastcall}

In x86-64, all registers were extended to 64-bit and now their names have an \TT{R-} prefix.
In order to use the stack less often (in other words, to access external memory/cache less often), there exists
a popular way to pass function arguments via registers (\IT{fastcall}) \myref{fastcall}.
I.e., a part of the function arguments is passed in registers, the rest---via the stack.
In Win64, 4 function arguments are passed in the \RCX, \RDX, \Reg{8}, \Reg{9} registers.
That is what we see here: a pointer to the string for \printf is now passed not in the stack, but in the \RCX register.
The pointers are 64-bit now, so they are passed in the 64-bit registers (which have the \TT{R-} prefix).
However, for backward compatibility, it is still possible to access the 32-bit parts, using the \TT{E-} prefix.
This is how the \RAX/\EAX/\AX/\AL register looks like in x86-64:

\RegTableOne{RAX}{EAX}{AX}{AH}{AL}

The \main function returns an \Tint{}-typed value, which is, in \CCpp, for better backward compatibility
and portability, still 32-bit, so that is why the \EAX register is cleared at the function end (i.e., the 32-bit
part of the register) instead of \RAX{}.
There are also 40 bytes allocated in the local stack.
This is called the \q{shadow space}, about which we are going to talk later: \myref{shadow_space}.
}
\ITA{\subsubsection{MSVC: x86-64}

\myindex{x86-64}
Proviamo anche con MSVC a 64-bit:

\lstinputlisting[caption=MSVC 2012 x64,style=customasmx86]{patterns/01_helloworld/MSVC_x64.asm}

\myindex{fastcall}

In x86-64, tutti i registri sono stati estesi a 64-bit ed il loro nome ha il prefisso \TT{R-}.
Per usare lo stack meno spesso (in altre parole, per accedere meno spesso alla memoria esterna/cache), esiste un metodo molto diffuso per passare gli argomenti delle funzioni tramite i registri (\IT{fastcall})
\myref{fastcall}.
Ovvero, una parte degli argomenti è passata attraverso i registri, il resto ---attraverso lo stack.
In Win64, 4 argpmenti di funzione sono passati nei registri \RCX, \RDX, \Reg{8}, \Reg{9}.
Questo è ciò che vediamo qui: un puntatore alla stringa per \printf è adesso passato nel registro \RCX anziché tramite lo stack.
I puntatori adesso sono a 64-bit , quindi sono passati nei registri a 64-bit (aventi il prefisso \TT{R-}).
E' comunque possibile, per retrocompatibilità, accedere alle parti a 32-bit parts, usando il prefisso \TT{E-}.
I registri \RAX/\EAX/\AX/\AL in x86-64 appaiono così:

\RegTableOne{RAX}{EAX}{AX}{AH}{AL}

La funzione \main restituisce un valore di tipo \Tint{}, che in \CCpp, per migliore retrocompatibilità e portabilità, resta ancora a 32-bit, motivo per cui il registro \EAX viene svuotato invece di \RAX{} alla fine della funzione (i.e., la parte a 32-bit
del registro).
Ci sono anche 40 byte allocati nello stack locale.
Questo spazio è detto \q{shadow space}, di cui parleremo più avanti: \myref{shadow_space}.

}
\NL{\subsubsection{MSVC: x86-64}

\myindex{x86-64}
Laat ons ook eens kijken naar 64-bit MSVC:

\lstinputlisting[caption=MSVC 2012 x64,style=customasmx86]{patterns/01_helloworld/MSVC_x64.asm}

\myindex{fastcall}

In x86-64 zijn alle registers uitgebreid tot 64-bit, en hebben hun namen een \TT{R-} prefix gekregen.
Om de stack minder te gebruiken (met andere woorden, om het externe geheugen/cache minder vaak te benaderen), bestaat
er een populaire manier om functies parameters door te geven via registers (\IT{fastcall}) \myref{fastcall}.
Bv., een deel van de parameters wordt doorgegeven via het register, de rest --- via de stack.
In Win64, worden 4 functie parameters doorgegeven via de \RCX, \RDX, \Reg{8}, \Reg{9} registers.
Dat is wat we hier zien: een pointer naar de string voor \printf wordt doorgegeven, niet via de stack, maar via het \RCX register.
De pointers zijn 64-bit nu, dus worden ze doorgegeven in de 64-bit registers (dewelke de \TT{R-} prefix hebben).
Voor backward compatibility is het echter nog steeds mogelijk om de 32-bit gedeelten aan te spreken, door gebruik te maken van de \TT{E-} prefix.
Dit is hoe de \RAX/\EAX/\AX/\AL registers eruit zien in x86-64:

\RegTableOne{RAX}{EAX}{AX}{AH}{AL}

De \main functie geeft een \Tint{}-typed waarde terug, hetwelk, in \CCpp, voor betere backward compatibiliteit
en portabiliteit, nog steeds 32-bit is. Daarom wordt het \EAX register ook leeggemaakt bij het einde van de functie
(het 32-bit gedeelte van het register) in plaats van \RAX{}.
Er zijn ook 40 bytes gealloceerd op de lokale stack.
Dit wordt de \q{shadow space} genoemd, waarover we het later nog gaan hebben: \myref{shadow_space}.

}
\RU{\subsubsection{MSVC: x86-64}

\myindex{x86-64}
Попробуем также 64-битный MSVC:

\lstinputlisting[caption=MSVC 2012 x64,style=customasmx86]{patterns/01_helloworld/MSVC_x64.asm}

\myindex{fastcall}

В x86-64 все регистры были расширены до 64-х бит и теперь имеют префикс \TT{R-}.
Чтобы поменьше задействовать стек (иными словами, поменьше обращаться кэшу и внешней памяти), уже давно имелся
довольно популярный метод передачи аргументов функции через регистры (\IT{fastcall}) \myref{fastcall}.
Т.е. часть аргументов функции передается через регистры и часть ---через стек.
В Win64 первые 4 аргумента функции передаются через регистры \RCX, \RDX, \Reg{8}, \Reg{9}.
Это мы здесь и видим: указатель на строку в \printf теперь передается не через стек, а через регистр \RCX.
Указатели теперь 64-битные, так что они передаются через 64-битные части регистров (имеющие префикс \TT{R-}).
Но для обратной совместимости можно обращаться и к нижним 32 битам регистров используя префикс \TT{E-}.
Вот как выглядит регистр \RAX/\EAX/\AX/\AL в x86-64:

\RegTableOne{RAX}{EAX}{AX}{AH}{AL}

Функция \main возвращает значение типа \Tint, который в \CCpp, надо полагать, для лучшей совместимости и переносимости,
оставили 32-битным. Вот почему в конце функции \main обнуляется не \RAX, а \EAX, т.е. 32-битная часть регистра.
Также видно, что 40 байт выделяются в локальном стеке.
Это \q{shadow space} которое мы будем рассматривать позже: \myref{shadow_space}.
}
\PTBR{\subsubsection{MSVC: x86-64}

\myindex{x86-64}
Vamos tentar também o MSVC 64-bits:

\lstinputlisting[caption=MSVC 2012 x64,style=customasmx86]{patterns/01_helloworld/MSVC_x64.asm}

\myindex{fastcall}

No x86-64, todos os registradores foram extendidos para 64-bits e agora seus nomes contém um \TT{R-} no prefixo.
A fim de diminuir a frequência com que a stack (pilha) é usada (em outras palavras, para acessar memória externa/cache menos vezes),
existe uma maneira popular de passar argumentos para funções através dos registradores (\IT{fastcall}) \myref{fastcall}.
Por exemplo, uma parte dos argumentos da função é passada nos registradores, o resto pela stack.
No Win64, 4 argumentos de funções são passados através dos registradores \RCX, \RDX, \Reg{8}, \Reg{9}.
Que é o que nós vemos, um ponteiro para a string para o printf() não é passado pela stack, mas no registrador \RCX.
Os ponteiros são 64-bits agora, então, eles são passados através dos registradores de 64-bits (que tem prefixo \TT{R-}).
Entretanto, para compatibilidade, ainda é possível acessar partes de 32-bits, usando o prefixo \TT{E-}.
É assim que os registradores \RAX/\EAX/\AX/\AL se parecem no x86-64:

\RegTableOne{RAX}{EAX}{AX}{AH}{AL}

A função \main retorna um valor do tipo inteiro, que em \CCpp é melhor para compatibilidade com versões anteriores e portabilidade,
de 32-bits, por isso o registrador \EAX é limpo no final da função (a parte de 32-bits do registrador) ao invés de \RAX.
Há também 40 bytes alocados na pilha local.
Que é chamado de ``shadow space'', o qual falaremos mais tarde: \myref{shadow_space}.

}


\ifdefined\IncludeGCC
\EN{\subsubsection{GCC: x86-64}

\myindex{x86-64}
Let's also try GCC in 64-bit Linux:

\lstinputlisting[caption=GCC 4.4.6 x64,style=customasmx86]{patterns/01_helloworld/GCC_x64_EN.s}

% I think I got the intent right on the following line - Renaissance
Linux, *BSD and \MacOSX also use a method to pass function arguments in registers. \SysVABI{}.

The first 6 arguments are passed in the \RDI, \RSI, \RDX, \RCX, \Reg{8}, and \Reg{9}  registers, and the rest---via the stack.

So the pointer to the string is passed in \EDI (the 32-bit part of the register).
Why doesn't it use the 64-bit part, \RDI?

It is important to keep in mind that all \MOV instructions in 64-bit mode that write something into the lower 32-bit register part also clear the higher 32-bits (as stated in Intel manuals: \myref{x86_manuals}).\\
I.e., the \INS{MOV EAX, 011223344h} writes a value into \RAX correctly, since the higher bits will be cleared.

If we open the compiled object file (.o), we can also see all the instructions' opcodes
\footnote{This must be enabled in \textbf{Options $\rightarrow$ Disassembly $\rightarrow$ Number of opcode bytes}}:

\lstinputlisting[caption=GCC 4.4.6 x64,style=customasmx86]{patterns/01_helloworld/GCC_x64.lst}

\label{hw_EDI_instead_of_RDI}
As we can see, the instruction that writes into \EDI at \TT{0x4004D4} occupies 5 bytes.
The same instruction writing a 64-bit value into \RDI occupies 7 bytes.
Apparently, GCC is trying to save some space.
Besides, it can be sure that the data segment containing the string will not be allocated at the addresses higher than 4\gls{GiB}.

\label{SysVABI_input_EAX}
% There isn't an ABI acronym in acronyms.tex - I figure the intent is to Application Binary Interface,
% so I put it in there (in the EN section, commented out)
We also see that the \EAX register has been cleared before the \printf function call.
This is done because according to \ac{ABI} standard mentioned above,
the number of used vector registers is to be passed in \EAX in *NIX systems on x86-64.
}
\RU{\subsubsection{GCC: x86-64}

\myindex{x86-64}
Попробуем GCC в 64-битном Linux:

\lstinputlisting[caption=GCC 4.4.6 x64,style=customasmx86]{patterns/01_helloworld/GCC_x64_RU.s}

В Linux, *BSD и \MacOSX для x86-64 также принят способ передачи аргументов функции через регистры \SysVABI.

6 первых аргументов передаются через регистры \RDI, \RSI, \RDX, \RCX, \Reg{8}, \Reg{9}, а остальные --- через стек.

Так что указатель на строку передается через \EDI (32-битную часть регистра).
Но почему не через 64-битную часть, \RDI?

Важно запомнить, что в 64-битном режиме все инструкции \MOV, записывающие что-либо в младшую 32-битную часть регистра, обнуляют старшие 32-бита (это можно найти в документации от Intel: \myref{x86_manuals}).
То есть, инструкция \INS{MOV EAX, 011223344h} корректно запишет это значение в \RAX, старшие биты сбросятся в ноль.

Если посмотреть в \IDA скомпилированный объектный файл (.o), увидим также опкоды всех инструкций
\footnote{Это нужно задать в \textbf{Options $\rightarrow$ Disassembly $\rightarrow$ Number of opcode bytes}}:

\lstinputlisting[caption=GCC 4.4.6 x64,style=customasmx86]{patterns/01_helloworld/GCC_x64.lst}

\label{hw_EDI_instead_of_RDI}
Как видно, инструкция, записывающая в \EDI по адресу \TT{0x4004D4}, занимает 5 байт.
Та же инструкция, записывающая 64-битное значение в \RDI, занимает 7 байт.
Возможно, GCC решил немного сэкономить.
К тому же, вероятно, он уверен, что сегмент данных, где хранится строка, никогда не будет расположен в адресах выше 4\gls{GiB}.

\label{SysVABI_input_EAX}
Здесь мы также видим обнуление регистра \EAX перед вызовом \printf.
Это делается потому что по упомянутому выше стандарту передачи аргументов в *NIX для x86-64 в \EAX передается количество задействованных векторных регистров.

}
\NL{\subsubsection{GCC: x86-64}

\myindex{x86-64}
Laat ons ook eens kijken naar GCC in 64-bit Linux:

% TODO translate:
\lstinputlisting[caption=GCC 4.4.6 x64,style=customasmx86]{patterns/01_helloworld/GCC_x64_EN.s}

Een methode om functieargumenten door te geven in registers wordt ook gebruikt in Linux, *BSD en \MacOSX \SysVABI.

De eerste 6 argumenten worden doorgegeven in de \RDI, \RSI, \RDX, \RCX, \Reg{8}, \Reg{9} registers, en de rest --- via de stack.

De pointer naar de string wordt dus doorgegeven via \EDI (het 32-bit gedeelte van het register).
Maar waarom gebruikt men niet het 64-bit gedeelte, \RDI?

Het is belangrijk te onthouden dat alle \MOV instructies in 64-bit modus, die iets schrijven in het onderste 32-bit gedeelte van het register, ook het bovenste 32-bit gedeelte leegmaken.
\INS{MOV EAX, 011223344h} schrijft een waarde correct weg in \RAX, aangezien de bovenste bits zullen worden leeggemaakt.

Als we het gecompileerde object-bestand (.o) openen, kunnen we ook de opcodes zien van alle instructies
\footnote{Dit moet ook geactiveerd worden in \textbf{Options $\rightarrow$ Disassembly $\rightarrow$ Number of opcode bytes}}:

\lstinputlisting[caption=GCC 4.4.6 x64,style=customasmx86]{patterns/01_helloworld/GCC_x64.lst}

\label{hw_EDI_instead_of_RDI}
Zoals je kan zien, bezet de instructie die in \EDI schrijft op \TT{0x4004D4} 5 bytes.
Dezelfde instructie die een 64-bit waarde in \RDI schrijft, bezet 7 bytes.
Blijkbaar probeert GCC wat plaats te besparen.
Daarnaast kunnen we met zekerheid zeggen dat het data segment dat de string bevat, niet zal gealloceerd worden op de adressen hoger dan 4\gls{GiB}.

\label{SysVABI_input_EAX}
We zien ook dat het \EAX register leeggemaakt is voor de \printf functie call.
Dit wordt gedaan omdat het aantal gebruikte vector registers wordt doorgegeven in \EAX in *NIX systemen op x86-64.

}
\ITA{\subsubsection{GCC: x86-64}

\myindex{x86-64}
\ITAph{}:

% TODO: translate:
\lstinputlisting[caption=GCC 4.4.6 x64,,style=customasmx86]{patterns/01_helloworld/GCC_x64_EN.s}

Un metodo per passare argomenti di funzione nei registri usato anche in Linux, *BSD and \MacOSX è \SysVABI.

I primi 6 argomenti sono passati nei registri \RDI, \RSI, \RDX, \RCX, \Reg{8}, \Reg{9}  , ed il resto---tramite lo stack.

Quindi il puntatore alla stringa viene passato in \EDI (la parte a 32-bit del registro).
Ma perchè no nusare la parte a 64-bit \RDI?

E' importante ricordare che tutte le istruzioni \MOV in modalità 64-bit che scrivono qualcosa nella parte bassa a 32-bit di un registro, azzera anche la parte alta a 32-bits.
Ad esempio, \INS{MOV EAX, 011223344h} scrive un valore in \RAX correttamente, poichè i bit della parte alta saranno azzerati.

Se apriamo il file oggetto compilato (.o), possiamo anche vedere gli opcode di tutte le istruzioni
\footnote{Deve essere abilitato in \textbf{Options $\rightarrow$ Disassembly $\rightarrow$ Number of opcode bytes}}:

\lstinputlisting[caption=GCC 4.4.6 x64,style=customasmx86]{patterns/01_helloworld/GCC_x64.lst}

\label{hw_EDI_instead_of_RDI}
Come possiamo notare, l'istruzione che scrive dentro \EDI a \TT{0x4004D4} occupa 5 byte.
La stessa istruzione che scrive un valore a 64-bit dentro \RDI occupa 7 bytes.
Apparentemente, GCC sta cercando di risparmiare un po' di spazio.
Inoltre, può essere sicuro che il segmento dati contenente la stringa non sarà allocato ad indirizzi maggiori di 4\gls{GiB}.

\label{SysVABI_input_EAX}
Notiamo anche che il registro \EAX è stato azzerato prima della chiamata alla funzione \printf .
Ciò avviene perché il numbero dei registri vettore usati viene passato in \EAX nei sistemi *NIX x86-64.

}


\fi

\ifdefined\IncludeGCC
\section{GCC\EMDASH{}\EN{one more thing}\RU{ещё кое-что}}
\label{use_parts_of_C_strings}

\RU{Тот факт, что \IT{анонимная} Си-строка имеет тип}\EN{The fact that an \IT{anonymous} C-string has} 
\IT{const}\EN{ type} (\myref{string_is_const_char}), 
\RU{и тот факт, что выделенные в сегменте констант Си-строки гаратировано неизменяемые (immutable), 
ведет к интересному следствию}\EN{and
that C-strings allocated in constants segment are guaranteed to be immutable, has an interesting consequence}:
\RU{компилятор может использовать определенную часть строки}\EN{the compiler may use a specific part of the string}.

\RU{Вот простой пример}\EN{Let's try this example}:

\begin{lstlisting}
#include <stdio.h>

int f1()
{
	printf ("world\n");
}

int f2()
{
	printf ("hello world\n");
}

int main()
{
	f1();
	f2();
}
\end{lstlisting}

\RU{Среднестатистический компилятор с \CCpp (включая MSVC) выделит место для двух строк, 
но вот что делает GCC 4.8.1}%
\EN{Common \CCpp{}-compilers (including MSVC) allocate two strings, 
but let's see what GCC 4.8.1 does}:

\begin{lstlisting}[caption=GCC 4.8.1 + \RU{листинг в }IDA\EN{ listing}]
f1              proc near

s               = dword ptr -1Ch

                sub     esp, 1Ch
                mov     [esp+1Ch+s], offset s ; "world\n"
                call    _puts
                add     esp, 1Ch
                retn
f1              endp

f2              proc near

s               = dword ptr -1Ch

                sub     esp, 1Ch
                mov     [esp+1Ch+s], offset aHello ; "hello "
                call    _puts
                add     esp, 1Ch
                retn
f2              endp

aHello          db 'hello '
s               db 'world',0xa,0
\end{lstlisting}

\RU{Действительно, когда мы выводим строку}\EN{Indeed: when we print the \q{hello world} string}, 
\RU{эти два слова расположены в памяти впритык друг к другу и \puts, вызываясь из функции f2(), вообще не знает,
что эти строки разделены}\EN{these two words are positioned in memory adjacently and \puts called from f2() 
function is not aware that this string is divided}. \RU{Они и не разделены на самом деле, они разделены
только \q{виртуально}, в нашем листинге}\EN{In fact, it's not divided; it's divided only \q{virtually}, in this
listing}.

\RU{Когда}\EN{When} \puts \RU{вызывается из f1(), он использует строку}\EN{is called from f1(), it uses the} 
\q{world} \RU{плюс нулевой байт}\EN{string plus a zero byte}. \puts \RU{не знает, что там ещё есть какая-то строка
перед этой}\EN{is not aware that there is something before this string}!

\RU{Этот трюк часто используется (по крайней мере в GCC) и может сэкономить немного памяти.}
\EN{This clever trick is often used by at least GCC and can save some memory.}

\fi
\ifdefined\IncludeARM
\section{ARM}
\label{sec:hw_ARM}

\index{\idevices}
\index{Raspberry Pi}
\index{Xcode}
\index{LLVM}
\index{Keil}
\RU{Для экспериментов с процессором ARM я использовал несколько компиляторов:}
\EN{For my experiments with ARM processors I used several compilers:} 

\begin{itemize}
\item \RU{Популярный в embedded-среде}\EN{Popular in the embedded area} Keil Release 6/2013.

\item Apple Xcode 4.6.3 \EN{IDE} (\RU{с компилятором}\EN{with} LLVM-GCC 4.2 \EN{compiler}
\footnote{\EN{It is indeed so: Apple Xcode 4.6.3 uses open-source GCC as front-end compiler and LLVM 
code generator}\RU{Это действительно так: Apple Xcode 4.6.3 использует опен-сорсный GCC как компилятор
переднего плана и коде-генератор LLVM}}.

%\item GCC 4.8.1 (Linaro) (\RU{для}\EN{for} ARM64).
%
\item GCC 4.9 (Linaro) (\RU{для}\EN{for} ARM64), 
\RU{доступный как исполняемые файлы для win32 на}\EN{available as win32-executables at} 
\url{http://www.linaro.org/projects/armv8/}.

\end{itemize}

\RU{Везде в этой книге, кроме как если указано иное, идет речь о 32-битном ARM.}
\EN{32-bit ARM code is used in all cases in this book, if not mentioned otherwise.}
\RU{Когда речь идет о 64-битном ARM, он называется здесь ARM64.}
\EN{If we talk about 64-bit ARM here, it will be called ARM64.}

% subsections
\EN{\input{patterns/01_helloworld/ARM/keil_ARM_EN}}
\RU{\input{patterns/01_helloworld/ARM/keil_ARM_RU}}
\ITA{\input{patterns/01_helloworld/ARM/keil_ARM_ITA}}


\subsection{\NonOptimizingKeilVI (\ThumbMode)}

\RU{Скомпилируем тот же пример в Keil для режима Thumb}\EN{Let's compile the same example using Keil in Thumb mode}:

\begin{lstlisting}
armcc.exe --thumb --c90 -O0 1.c 
\end{lstlisting}

\RU{Получим (в \IDA)}\EN{We are getting (in \IDA)}:

\begin{lstlisting}[caption=\NonOptimizingKeilVI (\ThumbMode) + \IDA]
.text:00000000             main
.text:00000000 10 B5          PUSH    {R4,LR}
.text:00000002 C0 A0          ADR     R0, aHelloWorld ; "hello, world"
.text:00000004 06 F0 2E F9    BL      __2printf
.text:00000008 00 20          MOVS    R0, #0
.text:0000000A 10 BD          POP     {R4,PC}

.text:00000304 68 65 6C 6C+aHelloWorld  DCB "hello, world",0    ; DATA XREF: main+2
\end{lstlisting}

\RU{Сразу бросаются в глаза двухбайтные (16-битные) опкоды\EMDASH{}это, как уже было отмечено, Thumb.}%
\EN{We can easily spot the 2-byte (16-bit) opcodes. This is, as was already noted, Thumb.}
\index{ARM!\Instructions!BL}
\RU{Кроме инструкции \TT{BL}.}\EN{The \TT{BL} instruction, however, }
\RU{Но на самом деле она состоит из двух 16-битных инструкций}%
\EN{consists of two 16-bit instructions}.
\RU{Это потому что в одном 16-битном опкоде слишком мало места для задания смещения, по которому находится функция \printf}%
\EN{This is because it is impossible to load an offset for the \printf function while using the small space in one 16-bit opcode}.
\RU{Так что первая 16-битная инструкция загружает старшие 10 бит смещения, а вторая~--- младшие 11 бит смещения}%
\EN{Therefore, the first 16-bit instruction loads the higher 10 bits of the offset and the second instruction loads 
the lower 11 bits of the offset}.
% TODO:
% BL has space for 11 bits, so if we don't encode the lowest bit,
% then we should get 11 bits for the upper half, and 12 bits for the lower half.
% And the highest bit encodes the sign, so the destination has to be within
% \pm 4M of current_PC.
% This may be less if adding the lower half does not carry over,
% but I'm not sure --all my programs have 0 for the upper half,
% and don't carry over for the lower half.
% It would be interesting to check where __2printf is located relative to 0x8
% (I think the program counter is the next instruction on a multiple of 4
% for THUMB).
% The lower 11 bytes of the BL instructions and the even bit are
% 000 0000 0110 | 001 0010 1110 0 = 000 0000 0110 0010 0101 1100 = 0x00625c,
% so __2printf should be at 0x006264.
% But if we only have 10 and 11 bits, then the offset would be:
% 00 0000 0110 | 01 0010 1110 0 = 0 0000 0011 0010 0101 1100 = 0x00325c,
% so __2printf should be at 0x003264.
% In this case, though, the new program counter can only be 1M away,
% because of the highest bit is used for the sign.
\RU{Как уже было упомянуто, все инструкции в Thumb-режиме имеют длину 2 байта (или 16 бит)}%
\EN{As was noted, all instructions in Thumb mode have a size of 2 bytes (or 16 bits)}.
\RU{Поэтому невозможна такая ситуация, когда Thumb-инструкция начинается по нечетному адресу.}
\EN{This implies it is impossible for a Thumb-instruction to be at an odd address whatsoever.}
\RU{Учитывая сказанное, последний бит адреса можно не кодировать}%
\EN{Given the above, the last address bit may be omitted while encoding instructions}.
\RU{Таким образом, в Thumb-инструкции \TT{BL} можно закодировать адрес}
\EN{In summary, the \TT{BL} Thumb-instruction can encode an address in} $current\_PC \pm{}\approx{}2M$.

\index{ARM!\Instructions!PUSH}
\index{ARM!\Instructions!POP}
\RU{Остальные инструкции в функции (\PUSH и \POP) здесь работают почти так же, как и описанные \TT{STMFD}/\TT{LDMFD}, только регистр \ac{SP} здесь не указывается явно}%
\EN{As for the other instructions in the function: \PUSH and \POP work here just like the described \TT{STMFD}/\TT{LDMFD} only the \ac{SP} register is not mentioned explicitly here}.
\TT{ADR} \RU{работает так же, как и в предыдущем примере}\EN{works just like in the previous example}.
\TT{MOVS} \RU{записывает 0 в регистр \Reg{0} для возврата нуля}%
\EN{writes 0 into the \Reg{0} register in order to return zero}.

\subsection{\OptimizingXcodeIV (\ARMMode)}

Xcode 4.6.3 \RU{без включенной оптимизации выдает слишком много лишнего кода, поэтому включим оптимизацию компилятора (ключ \Othree), потому что там меньше инструкций.}
\EN{without optimization turned on produces a lot of redundant code so we'll study optimized output, where the instruction count is as small as possible, setting the compiler switch \Othree.}

\begin{lstlisting}[caption=\OptimizingXcodeIV (\ARMMode)]
__text:000028C4             _hello_world
__text:000028C4 80 40 2D E9   STMFD           SP!, {R7,LR}
__text:000028C8 86 06 01 E3   MOV             R0, #0x1686
__text:000028CC 0D 70 A0 E1   MOV             R7, SP
__text:000028D0 00 00 40 E3   MOVT            R0, #0
__text:000028D4 00 00 8F E0   ADD             R0, PC, R0
__text:000028D8 C3 05 00 EB   BL              _puts
__text:000028DC 00 00 A0 E3   MOV             R0, #0
__text:000028E0 80 80 BD E8   LDMFD           SP!, {R7,PC}

__cstring:00003F62 48 65 6C 6C+aHelloWorld_0  DCB "Hello world!",0
\end{lstlisting}

\RU{Инструкции}\EN{The instructions} \TT{STMFD} \AndENRU \TT{LDMFD} \RU{нам уже знакомы}\EN{are already familiar to us}.

\index{ARM!\Instructions!MOV}
\RU{Инструкция \MOV просто записывает число \TT{0x1686} в регистр \Reg{0}~--- это смещение, указывающее на строку \q{Hello world!}}%
\EN{The \MOV instruction just writes the number \TT{0x1686} into the \Reg{0} register.
This is the offset pointing to the \q{Hello world!} string}.

\RU{Регистр \TT{R7} (по стандарту, принятому в \cite{IOSABI}) это frame pointer, о нем будет рассказано позже.}
\EN{The \TT{R7} register (as it is standardized in \cite{IOSABI}) is a frame pointer. More on that below.}

\index{ARM!\Instructions!MOVT}
\RU{Инструкция}\EN{The} \TT{MOVT R0, \#0} (MOVe Top) \RU{записывает 0 в старшие 16 бит регистра}%
\EN{instruction writes 0 into higher 16 bits of the register}.
\RU{Дело в том, что обычная инструкция \MOV в режиме ARM может записывать какое-либо значение только в младшие 16 бит регистра, ведь в ней нельзя закодировать больше}%
\EN{The issue here is that the generic \MOV instruction in ARM mode may write only the lower 16 bits of the register}.
\RU{Помните, что в режиме ARM опкоды всех инструкций ограничены длиной в 32 бита. Конечно, это ограничение не касается перемещений данных между регистрами.}
\EN{Remember, all instruction opcodes in ARM mode are limited in size to 32 bits. Of course, this limitation is not related to moving data between registers.}
\RU{Поэтому для записи в старшие биты (с 16-го по 31-й включительно) существует дополнительная команда \TT{MOVT}}%
\EN{That's why an additional instruction \TT{MOVT} exists for writing into the higher bits (from 16 to 31 inclusive)}.
\RU{Впрочем, здесь её использование избыточно, потому что инструкция \TT{MOV R0, \#0x1686} выше и так обнулила старшую часть регистра}%
\EN{Its usage here, however, is redundant because the \TT{MOV R0, \#0x1686} instruction above cleared the higher part of the register}.
\RU{Возможно, это недочет компилятора}\EN{This is probably a shortcoming of the compiler}.
% TODO:
% I think, more specifically, the string is not put in the text section,
% ie. the compiler is actually not using position-independent code,
% as mentioned in the next paragraph.
% MOVT is used because the assembly code is generated before the relocation,
% so the location of the string is not yet known,
% and the high bits may still be needed.

\index{ARM!\Instructions!ADD}
\RU{Инструкция}\EN{The} \TT{ADD R0, PC, R0} \RU{прибавляет \ac{PC} к \Reg{0} для вычисления действительного адреса строки \q{Hello world!}. Как нам уже известно, это \q{\PICcode}, поэтому такая корректива необходима}%
\EN{instruction adds the value in the \ac{PC} to the value in the \Reg{0}, to calculate the absolute address of the \q{Hello world!} string. 
As we already know, it is \q{\PICcode} so this correction is essential here}.

\RU{Инструкция \TT{BL} вызывает \puts вместо \printf}%
\EN{The \TT{BL} instruction calls the \puts function instead of \printf}.

\label{puts}
\index{\CStandardLibrary!puts()}
\index{puts() \RU{вместо}\EN{instead of} printf()}
\RU{Компилятор заменил вызов \printf на \puts. 
Действительно, \printf с одним аргументом это почти аналог \puts.}
\EN{GCC replaced the first \printf call with \puts.
Indeed: \printf with a sole argument is almost analogous to \puts.} 
\RU{\IT{Почти}, если принять условие, что в строке не будет управляющих символов \printf, 
начинающихся со знака процента. Тогда эффект от работы этих двух функций будет разным}%
\EN{\IT{Almost}, because the two functions are producing the same result only in case the 
string does not contain printf format identifiers starting with \IT{\%}. 
In case it does, the effect of these two functions would be different}%
\footnote{
\RU{Также нужно заметить, что \puts не требует символа перевода строки `\textbackslash{}n' в конце строки,
поэтому его здесь нет.}
\EN{It has also to be noted the \puts does not require a `\textbackslash{}n' new line symbol 
at the end of a string, so we do not see it here.}}.

\RU{Зачем компилятор заменил один вызов на другой? Наверное потому что \puts работает быстрее}%
\EN{Why did the compiler replace the \printf with \puts? Probably because \puts is faster}%
\footnote{\href{http://go.yurichev.com/17063}{ciselant.de/projects/gcc\_printf/gcc\_printf.html}}. 
\RU{Видимо потому что \puts проталкивает символы в \gls{stdout} не сравнивая каждый со знаком процента.}
\EN{Because it just passes characters to \gls{stdout} without comparing every one of them with the \IT{\%} symbol.}

\RU{Далее уже знакомая инструкция}\EN{Next, we see the familiar} 
\TT{MOV R0, \#0}\RU{, служащая для установки в 0 возвращаемого значения функции}%
\EN{instruction intended to set the \Reg{0} register to 0}.

\subsection{\OptimizingXcodeIV (\ThumbTwoMode)}

\RU{По умолчанию}\EN{By default} Xcode 4.6.3 
\RU{генерирует код для режима Thumb-2 примерно в такой манере}%
\EN{generates code for Thumb-2 in this manner}:

\begin{lstlisting}[caption=\OptimizingXcodeIV (\ThumbTwoMode)]
__text:00002B6C                   _hello_world
__text:00002B6C 80 B5          PUSH            {R7,LR}
__text:00002B6E 41 F2 D8 30    MOVW            R0, #0x13D8
__text:00002B72 6F 46          MOV             R7, SP
__text:00002B74 C0 F2 00 00    MOVT.W          R0, #0
__text:00002B78 78 44          ADD             R0, PC
__text:00002B7A 01 F0 38 EA    BLX             _puts
__text:00002B7E 00 20          MOVS            R0, #0
__text:00002B80 80 BD          POP             {R7,PC}

...

__cstring:00003E70 48 65 6C 6C 6F 20+aHelloWorld  DCB "Hello world!",0xA,0
\end{lstlisting}

% Q: If you subtract 0x13D8 from 0x3E70,
% you actually get a location that is not in this function, or in _puts.
% How is PC-relative addressing done in THUMB2?
% A: it's not Thumb-related. there are just mess with two different segments. TODO: rework this listing.

\index{\ThumbTwoMode}
\index{ARM!\Instructions!BL}
\index{ARM!\Instructions!BLX}
\RU{Инструкции \TT{BL} и \TT{BLX} в Thumb, как мы помним, кодируются как пара 16-битных инструкций, 
а в Thumb-2 эти \IT{суррогатные} опкоды расширены так, что новые инструкции кодируются здесь как 
32-битные инструкции}%
\EN{The \TT{BL} and \TT{BLX} instructions in Thumb mode, as we recall, are encoded as a pair
of 16-bit instructions.
In Thumb-2 these \IT{surrogate} opcodes are extended in such a way so that new instructions
may be encoded here as 32-bit instructions}.
\RU{Это можно заметить по тому что опкоды Thumb-2 инструкций всегда начинаются с \TT{0xFx} либо с \TT{0xEx}}%
\EN{That is obvious considering that the opcodes of the Thumb-2 instructions always begin with \TT{0xFx} or \TT{0xEx}}.
\RU{Но в листинге \IDA байты опкода переставлены местами.
Это из-за того, что в процессоре ARM инструкции кодируются так:
в начале последний байт, потом первый (для Thumb и Thumb-2 режима), либо, 
(для инструкций в режиме ARM) в начале четвертый байт, затем третий, второй и первый 
(т.е. другой \gls{endianness})}%
\EN{But in the \IDA listing
the opcode bytes are swapped because for ARM processor the instructions are encoded as follows: 
last byte comes first and after that comes the first one (for Thumb and Thumb-2 modes) 
or for instructions in ARM mode the fourth byte comes first, then the third,
then the second and finally the first (due to different \gls{endianness})}.

\RU{Вот так байты следуют в листингах IDA:}
\EN{So that is how bytes are located in IDA listings:}
\begin{itemize}
\item \RU{для режимов ARM и ARM64}\EN{for ARM and ARM64 modes}: 4-3-2-1;
\item \RU{для режима Thumb}\EN{for Thumb mode}: 2-1;
\item \RU{для пары 16-битных инструкций в режиме Thumb-2}\EN{for 16-bit instructions pair in Thumb-2 mode}: 2-1-4-3.
\end{itemize}

\index{ARM!\Instructions!MOVW}
\index{ARM!\Instructions!MOVT.W}
\index{ARM!\Instructions!BLX}
\RU{Так что мы видим здесь что инструкции \TT{MOVW}, \TT{MOVT.W} и \TT{BLX} начинаются с}
\EN{So as we can see, the \TT{MOVW}, \TT{MOVT.W} and \TT{BLX} instructions begin with} \TT{0xFx}.

\RU{Одна из Thumb-2 инструкций это}\EN{One of the Thumb-2 instructions is}
\TT{MOVW R0, \#0x13D8}\RU{~--- она записывает 16-битное число в младшую часть регистра \Reg{0}, очищая старшие биты.}
\EN{~---it stores a 16-bit value into the lower part of the \Reg{0} register, clearing the higher bits.}

\RU{Ещё}\EN{Also,} \TT{MOVT.W R0, \#0}\RU{~--- эта инструкция работает так же, как и}
\EN{~works just like} 
\TT{MOVT} \RU{из предыдущего примера, но она работает в}\EN{from the previous example only it works in} Thumb-2.

\index{ARM!\RU{переключение режимов}\EN{mode switching}}
\index{ARM!\Instructions!BLX}
\RU{Помимо прочих отличий, здесь используется инструкция}
\EN{Among the other differences, the} \TT{BLX} 
\RU{вместо}\EN{instruction is used in this case instead of the} \TT{BL}.
\RU{Отличие в том, что помимо сохранения адреса возврата в регистре \ac{LR} и передаче управления 
в функцию \puts, происходит смена режима процессора с Thumb/Thumb-2 на режим ARM (либо назад).}
\EN{The difference is that, besides saving the \ac{RA} in the \ac{LR} register and passing control 
to the \puts function, the processor is also switching from Thumb/Thumb-2 mode to ARM mode (or back).}
\RU{Здесь это нужно потому, что инструкция, куда ведет переход, выглядит так (она закодирована в режиме ARM)}%
\EN{This instruction is placed here since the instruction to which control is passed looks like (it is encoded in ARM mode)}:

\begin{lstlisting}
__symbolstub1:00003FEC _puts           ; CODE XREF: _hello_world+E
__symbolstub1:00003FEC 44 F0 9F E5     LDR  PC, =__imp__puts
\end{lstlisting}

\EN{This is essentially a jump to the place where the address of \puts is written in the imports' section.}
\RU{Это просто переход на место, где записан адрес \puts в секции импортов.}

\RU{Итак, внимательный читатель может задать справедливый вопрос: почему бы не вызывать \puts сразу в 
том же месте кода, где он нужен?}
\EN{So, the observant reader may ask: why not call \puts right at the point in the code where it is needed?}

\RU{Но это не очень выгодно из-за экономии места и вот почему}%
\EN{Because it is not very space-efficient}.

\index{\RU{Динамически подгружаемые библиотеки}\EN{Dynamically loaded libraries}}
\RU{Практически любая программа использует внешние динамические библиотеки (будь то DLL в Windows, .so в *NIX 
либо .dylib в \MacOSX)}\EN{Almost any program uses external dynamic libraries (like DLL in Windows, .so in *NIX or .dylib in \MacOSX)}.
\RU{В динамических библиотеках находятся часто используемые библиотечные функции, в том числе стандартная функция Си \puts}%
\EN{The dynamic libraries contain frequently used library functions, including the standard C-function \puts}.

\index{Relocation}
\RU{В исполняемом бинарном файле}\EN{In an executable binary file} 
(Windows PE .exe, ELF \RU{либо}\EN{or} Mach-O) \RU{имеется секция импортов, список символов (функций либо глобальных переменных) импортируемых из внешних модулей, а также названия самих модулей}%
\EN{an import section is present.
This is a list of symbols (functions or global variables) imported from external modules along with the names of the modules themselves}.

\RU{Загрузчик \ac{OS} загружает необходимые модули и, перебирая импортируемые символы в основном модуле, проставляет правильные адреса каждого символа}%
\EN{The \ac{OS} loader loads all modules it needs and, while enumerating import symbols in the primary module, determines the correct addresses of each symbol}.

\RU{В нашем случае,}\EN{In our case,} \IT{\_\_imp\_\_puts} 
\RU{это 32-битная переменная, куда загрузчик \ac{OS} запишет правильный адрес этой же функции во внешней библиотеке}%
\EN{is a 32-bit variable used by the \ac{OS} loader to store the correct address of the function in an external library}. 
\RU{Так что инструкция \TT{LDR} просто берет 32-битное значение из этой переменной, и, записывая его в регистр \ac{PC}, просто передает туда управление}%
\EN{Then the \TT{LDR} instruction just reads the 32-bit value from this variable and writes it into the \ac{PC} register, passing control to it}.

\RU{Чтобы уменьшить время работы загрузчика \ac{OS}, 
нужно чтобы ему пришлось записать адрес каждого символа только один раз, 
в соответствующее, выделенное для них, место.}
\EN{So, in order to reduce the time the \ac{OS} loader needs for completing this procedure, 
it is good idea to write the address of each symbol only once, to a dedicated place.}

\index{thunk-\RU{функции}\EN{functions}}
\RU{К тому же, как мы уже убедились, нельзя одной инструкцией загрузить в регистр 32-битное число без обращений к памяти}%
\EN{Besides, as we have already figured out, it is impossible to load a 32-bit value into a register 
while using only one instruction without a memory access}.
\RU{Так что наиболее оптимально выделить отдельную функцию, работающую в режиме ARM, 
чья единственная цель~--- передавать управление дальше, в динамическую библиотеку.}
\EN{Therefore, the optimal solution is to allocate a separate function working in ARM mode with the sole 
goal of passing control to the dynamic library}
\RU{И затем ссылаться на эту короткую функцию из одной инструкции (так называемую \glslink{thunk function}{thunk-функцию}) из Thumb-кода}%
\EN{and then to jump to this short one-instruction function (the so-called \gls{thunk function}) from the Thumb-code}.

\index{ARM!\Instructions!BL}
\RU{Кстати, в предыдущем примере (скомпилированном для режима ARM), переход при помощи инструкции \TT{BL} ведет 
на такую же \glslink{thunk function}{thunk-функцию}, однако режим процессора не переключается (отсюда отсутствие \q{X} в мнемонике инструкции)}%
\EN{By the way, in the previous example (compiled for ARM mode) the control is passed by the \TT{BL} to the 
same \gls{thunk function}.
The processor mode, however, is not being switched (hence the absence of an \q{X} in the instruction mnemonic)}.

\subsubsection{\EN{More about thunk-functions}\RU{Еще о thunk-функциях}}
\index{thunk-\RU{функции}\EN{functions}}

\RU{Thunk-функции трудновато понять, вероятно, из-за путаницы в терминах.}
\EN{Thunk-functions are hard to understand, apparently, because of a misnomer.}

\RU{Проще всего представлять их как адаптеры-переходники из одного типа разъемов в другой.}
\EN{The simplest way to understand it as adaptors or convertors of one type of jack to another.}
\RU{Например, адаптер позволяющее вставить в американскую розетку британскую вилку, или наоборот.}
\EN{For example, an adaptor allowing the insertion of a British power plug into an American wall socket, or vice-versa.} 

\EN{Thunk functions are also sometimes called \IT{wrappers}.}
\RU{Thunk-функции также иногда называются \IT{wrapper-ами}. \IT{Wrap} в английском языке это \IT{обертывать}, \IT{завертывать}.}

\RU{Вот еще несколько описаний этих функций:}
\EN{Here are a couple more descriptions of these functions:}

\begin{framed}
\begin{quotation}
“A piece of coding which provides an address:”, according to P. Z. Ingerman, 
who invented thunks in 1961 as a way of binding actual parameters to their formal 
definitions in Algol-60 procedure calls. If a procedure is called with an expression 
in the place of a formal parameter, the compiler generates a thunk which computes 
the expression and leaves the address of the result in some standard location.

\dots

Microsoft and IBM have both defined, in their Intel-based systems, a “16-bit environment” 
(with bletcherous segment registers and 64K address limits) and a “32-bit environment” 
(with flat addressing and semi-real memory management). The two environments can both be 
running on the same computer and OS (thanks to what is called, in the Microsoft world, 
WOW which stands for Windows On Windows). MS and IBM have both decided that the process 
of getting from 16- to 32-bit and vice versa is called a “thunk”; for Windows 95, 
there is even a tool, THUNK.EXE, called a “thunk compiler”.
\end{quotation}
\end{framed}
% TODO FIXME move to bibliography and quote properly above the quote
( \href{http://go.yurichev.com/17362}{The Jargon File} )

\subsection{ARM64}

\subsubsection{GCC}

\RU{Компилируем пример в}\EN{Let's compile the example using} GCC 4.8.1 \InENRU ARM64:

\lstinputlisting[numbers=left,label=hw_ARM64_GCC,caption=\NonOptimizing GCC 4.8.1 + objdump]
{patterns/01_helloworld/ARM/hw.lst}

\RU{В ARM64 нет режима thumb и thumb-2, только ARM, так что тут только 32-битные инструкции.}
\EN{There are no thumb and thumb-2 modes in ARM64, only ARM, so there are 32-bit instructions only.}
\RU{Регистров тут в 2 раза больше}\EN{Registers count is doubled}: \ref{ARM64_GPRs}.
\RU{64-битные регистры теперь имеют префикс}\EN{64-bit registers has} 
\TT{X-}\EN{ prefixes, while its 32-bit parts}\RU{, а их 32-битные части}\EMDASH{}\TT{W-}.

\RU{Инструкция }\TT{STP}\EN{ instruction} (\IT{Store Pair}) 
\RU{сохраняет в стеке сразу два регистра}\EN{saves two registers in stack simultaneously}: \RegX{29} \InENRU \RegX{30}.
\RU{Конечно, эта инструкция может сохранять эту пару где угодно в памяти, но здесь указан регистр \ac{SP}, так что,
пара сохраняется именно в стеке.}
\EN{Of course, this instruction is able to save this pair at random place of memory, 
but \ac{SP} register is specified here, so the pair is saved in stack.}
\RU{Регистры в ARM64 64-битные, каждый это 8 байт, так что для хранения двух регистров нужно именно 16 байт.}
\EN{ARM64 registers are 64-bit ones, each contain 8 bytes, so one need 16 bytes for saving two registers.}

\RU{Восклицательный знак после операнда означает, что в начале от \ac{SP} будет отнято 16, и только затем
значения из пары регистров будут записаны в стек.}
\EN{Exclamation mark after operand mean that 16 will be subtracted from \ac{SP} first, and only then
values from registers pair will be written into the stack.}
\RU{Это называется}\EN{This is also called} \IT{pre-index}.
\RU{Больше о разнице между}\EN{About difference between} \IT{post-index} \AndENRU \IT{pre-index}, 
\RU{описано здесь}\EN{read here}: \ref{ARM_postindex_vs_preindex}.

\RU{Таким образом, в терминах более знакомого всем процессора x86, первая инструкция это просто аналог 
пары инструкций}
\EN{Hence, in terms of more familiar x86, the first instruction is just analogous to pair of}
\TT{PUSH X29} \AndENRU \TT{PUSH X30}.
\RegX{29} \EN{is used as \ac{FP} in ARM64}\RU{в ARM64 используется как \ac{FP}}, \EN{and}\RU{а} \RegX{30} 
\EN{as}\RU{как} \ac{LR}, \RU{поэтому они сохраняются в прологе ф-ции и
восстанавливаются в эпилоге}\EN{so that's why they are saved in function prologue and restored in function
epilogue}.

\EN{The second instruction saves}\RU{Вторая инструкция записывает} \ac{SP} \InENRU \RegX{29} (\OrENRU \ac{FP}).
\RU{Это нужно для установки стекового фрейма ф-ции}\EN{This is needed for function stack frame setup}.

\RU{Инструкции }\TT{ADRP} \AndENRU \ADD \EN{instructions are needed for forming address of the 
string}\RU{нужны для формирования адреса строки} ``Hello!'' \EN{in the \RegX{0} register}\RU{в регистре \RegX{0}}, 
\RU{ведь первый аргумент ф-ции передается через этот регистр}\EN{because first function argument is passed
in this register}.
\RU{Но в ARM нет инструкций, при помощи которых можно записать в регистр длинное число}\EN{But there are
no instructions in ARM helping to write large number into register} 
(\RU{потому что сама длина инструкции ограничена 4-ю байтами, больше об этом здесь}\EN{because instruction
length is limited by 4 bytes, read more about it here}: \ref{ARM_big_constants_loading}).
\RU{Так что нужно использовать несколько инструкций}\EN{So several instructions should be used}.
\RU{Первая инструкция}\EN{The first instruction} (\TT{ADRP}) \EN{writes address of 4Kb page where string is
located into \RegX{0}}\RU{записывает в \RegX{0} адрес 4-килобайтной страницы где находится строка}, 
\EN{and the second one}\RU{а вторая} (\ADD) \RU{просто прибавляет к этому адресу остаток}\EN{just adds
reminder to the address}.
\EN{Read more about}\RU{Читайте больше об этом}: \ref{ARM64_relocs}.

\TT{0x400000 + 0x648 = 0x400648}, \EN{and we see our ``Hello!'' C-string in the \TT{.rodata} data segment at this
address}\RU{и мы видим что в секции данных \TT{.rodata} по этому адресу как раз находится наша
Си-строка ``Hello!''}.

\RU{Затем, при помощи инструкции \TT{BL} вызывается \puts, это уже рассматривалось раннее: \ref{puts}.}
\EN{\puts is called then using \TT{BL} instruction, this was already discussed before: \ref{puts}.}

\RU{Инструкция }\MOV \EN{instruction writes $0$ into}\RU{записывает $0$ в} \RegW{0}. 
\RegW{0} \RU{это младшие 32 бита регистра}\EN{is low 32 bits of} \RegX{0}\EN{ register}:

\input{ARM_X0_register}

\RU{А результат ф-ции возвращается через \RegX{0}, и \main возвращает $0$, 
так что вот так готовится возвращаемый результат.}
\EN{Function result is returning via \RegX{0} and \main returning $0$, so that's how returning
result is prepared.}
\RU{Почему именно 32-битная часть}\EN{But why 32-bit part}?
\RU{Потому в ARM64, как и в x86-64, тип \Tint оставили 32-битным, для лучшей совместимости.}
\EN{Because \Tint in ARM64, just like in x86-64, is still 32-bit, for better compatibility.}
\RU{Следовательно, раз уж ф-ция возвращает 32-битный \Tint, то нужно заполнить только 32 младших бита 
регистра \RegX{0}.}
\EN{So if function returning 32-bit \Tint, only 32 lowest bits of \RegX{0} register should be filled.}

\RU{Для того, чтобы удостовериться в этом, я немного отредактировал свой пример и перекомпилировал его.}
\EN{In order to get sure about it, I changed by example slightly and recompiled it.}
\RU{Теперь}\EN{Now} \main \RU{возвращает 64-битное значение}\EN{returns 64-bit value}:

\begin{lstlisting}[caption=\main \RU{возвращающая значение типа}\EN{returning a value of} \TT{uint64\_t}\EN{ type}]
#include <stdio.h>
#include <stdint.h>

uint64_t main()
{
        printf ("Hello!\n");
        return 0;
};
\end{lstlisting}

\RU{Результат точно такой же, только \MOV в той строке теперь выглядит так:}
\EN{Result is very same, but that's how \MOV at that line is now looks like:}

\begin{lstlisting}[caption=\NonOptimizing GCC 4.8.1 + objdump]
  4005a4:       d2800000        mov     x0, #0x0                        // #0
\end{lstlisting}

\RU{Далее, при помощи инструкции \TT{LDP} (\IT{Load Pair}), восстанавливаются регистры \RegX{29} и \RegX{30}.}
\EN{\TT{LDP} (\IT{Load Pair}) then restores \RegX{29} and \RegX{30} registers.}
\RU{Восклицательного знака после инструкции нет: это означает, что в начале значения достаются из стека,
и только потом \ac{SP} увеличивается на 16.}
\EN{There are no exclamation mark after instruction: this mean, the value is first loaded from the stack,
only then \ac{SP} value is increased by 16.}
\RU{Это называется}\EN{This is called} \IT{post-index}.

\RU{В ARM64 есть новая инструкция}\EN{New instruction appeared in ARM64}: \RET. 
\RU{Она работает так же как и}\EN{It works just as} \TT{BX LR}, \RU{но там добавлен специальный бит,
подсказывающий процессору, что это именно выход из ф-ции, а не просто переход, чтобы процессор
мог более оптимально исполнять эту инструкцию}\EN{but a special \IT{hint} bit is added, showing to the \ac{CPU}
that this is return from the function, not just another branch instruction, so it can execute it more optimally}.

\RU{Из-за простоты этой ф-ции, оптимизирующий GCC генерирует точно такой же код.}
\EN{Due to simplicity of the function, optimizing GCC generates the very same code.}

\subsubsection{Xcode}

\OptimizingXcodeV \RU{делает почти такой же код}\EN{doing the same}.
\RU{Вот что показывает}\EN{Here is what} otool\footnote{\RU{аналог}\EN{analogous to} objdump} 
\RU{с ключом}\EN{with the} \TT{-tv}\EN{ key shows}:

\begin{lstlisting}[caption=\OptimizingXcodeV + otool]
0000000000000000		stp	fp, lr, [sp, #-16]!
0000000000000004		add	fp, sp, 0
0000000000000008		adrp	x0, 0 ; 0x0
000000000000000c		add	x0, x0, 0
0000000000000010		bl	0x10
0000000000000014		movz	w0, #0
0000000000000018		ldp	fp, lr, [sp], #16
000000000000001c		ret	lr
\end{lstlisting}

\RU{Код такой же}\EN{The code is the same}.
\TT{MOVZ} \RU{в данном случае это синоним}\EN{here is synonymous to} \MOV.
\RU{Вот что немного отличается: otool показывает регистр}\EN{Here is difference: otool shows} 
\RegX{29}\EN{ register} \RU{как}\EN{as} \ac{FP}, \RU{а}\EN{and} \RegX{30} \RU{как}\EN{as} \ac{LR}.
\RU{Это действительно синонимы для этих регистров.}
\EN{These are indeed nicknames for these registers.}

\RU{\RET otool показывает как \TT{RET LR}, это немного избыточный результат дизассемблера.}
\EN{otool also shows \RET as \TT{RET LR}, this is somewhat redundant disassembler result.}


\fi
\ifdefined\IncludeMIPS
\section{MIPS}

\subsection{\RU{О \q{глобальном указателе} (\q{global pointer})}\EN{A word about the \q{global pointer}}}
\label{MIPS_GP}

\index{MIPS!\GlobalPointer}
\RU{\q{Глобальный указатель} (\q{global pointer})~--- это важная концепция в MIPS.}
\EN{One important MIPS concept is the \q{global pointer}.}
\RU{Как мы уже возможно знаем, каждая инструкция в MIPS имеет размер 32 бита, поэтому невозможно
закодировать 32-битный адрес внутри одной инструкции. Вместо этого нужно использовать пару инструкций
(как это сделал GCC для загрузки адреса текстовой строки в нашем примере).}
\EN{As we may already know, each MIPS instruction has a size of 32 bits, so it's impossible to embed a 32-bit
address into one instruction: a pair has to be used for this 
(like GCC did in our example for the text string address loading).}

\RU{С другой стороны, используя только одну инструкцию, 
возможно загружать данные по адресам в пределах $register-32768...register+32767$, потому что 16 бит
знакового смещения можно закодировать в одной инструкции).}
\EN{It's possible, however, to load data from the address in the range of $register-32768...register+32767$ using one
single instruction (because 16 bits of signed offset could be encoded in a single instruction).}
\RU{Так мы можем выделить какой-то регистр для этих целей и ещё выделить буфер в 64KiB для самых 
частоиспользуемых данных.}
\EN{So we can allocate some register for this purpose and also allocate a 64KiB area of most used data.}
\RU{Выделенный регистр называется \q{глобальный указатель} (\q{global pointer}) и он указывает на середину
области 64KiB.}
\EN{This allocated register is called a \q{global pointer} and it points to the middle of the 64KiB area.}
\RU{Эта область обычно содержит глобальные переменные и адреса импортированных функций вроде \printf,
потому что разработчики GCC решили, что получение адреса функции должно быть как можно более быстрой операцией,
исполняющейся за одну инструкцию вместо двух.}
\EN{This area usually contains global variables and addresses of imported functions like \printf, 
because the GCC developers decided that getting the address of some function must be as fast as a single instruction
execution instead of two.}
\RU{В ELF-файле эта 64KiB-область находится частично в секции .sbss (\q{small \ac{BSS}}) для неинициализированных
данных и в секции .sdata (\q{small data}) для инициализированных данных.}
\EN{In an ELF file this 64KiB area is located partly in sections .sbss (\q{small \ac{BSS}}) for uninitialized data and 
.sdata (\q{small data}) for initialized data.}

\RU{Это значит что программист может выбирать, к чему нужен как можно более быстрый доступ, и затем расположить
это в секциях .sdata/.sbss.}
\EN{This implies that the programmer may choose what data he/she wants to be accessed fast and place it into 
.sdata/.sbss.}

\RU{Некоторые программисты \q{старой школы} могут вспомнить модель памяти в MS-DOS \myref{8086_memory_model} 
или в менеджерах памяти вроде XMS/EMS, где вся память делилась на блоки по 64KiB.}
\EN{Some old-school programmers may recall the MS-DOS memory model \myref{8086_memory_model} 
or the MS-DOS memory managers like XMS/EMS where all memory was divided in 64KiB blocks.}

\index{PowerPC}
\RU{Эта концепция применяется не только в MIPS. По крайней мере PowerPC также использует эту технику.}
\EN{This concept is not unique to MIPS. At least PowerPC uses this technique as well.}

\subsection{\Optimizing GCC}

\EN{Lets consider the following example, which illustrates the \q{global pointer} concept.}
\RU{Рассмотрим следующий пример, иллюстрирующий концепцию \q{глобального указателя}.}

\lstinputlisting[caption=\Optimizing GCC 4.4.5 (\assemblyOutput),numbers=left]{patterns/01_helloworld/MIPS/hw_O3.s.\LANG}

\RU{Как видно, регистр \$GP в прологе функции выставляется в середину этой области.}
\EN{As we see, the \$GP register is set in the function prologue to point to the middle of this area.}
\RU{Регистр \ac{RA} сохраняется в локальном стеке.}
\EN{The \ac{RA} register is also saved in the local stack.}
\RU{Здесь также используется \puts вместо \printf.}
\EN{\puts is also used here instead of \printf.}
\index{MIPS!\Instructions!LW}
\RU{Адрес функции \puts загружается в \$25 инструкцией LW (\q{Load Word}).}
\EN{The address of the \puts function is loaded into \$25 using LW the instruction (\q{Load Word}).}
\index{MIPS!\Instructions!LUI}
\index{MIPS!\Instructions!ADDIU}
\RU{Затем адрес текстовой строки загружается в \$4 парой инструкций LUI (\q{Load Upper Immediate}) и
ADDIU (\q{Add Immediate Unsigned Word}).}
\EN{Then the address of the text string is loaded to \$4 using LUI (\q{Load Upper Immediate}) and 
ADDIU (\q{Add Immediate Unsigned Word}) instruction pair.}
\RU{LUI устанавливает старшие 16 бит регистра (поэтому в имени инструкции присутствует \q{upper}) и ADDIU
прибавляет младшие 16 бит к адресу.}
\EN{LUI sets the high 16 bits of the register (hence \q{upper} word in instruction name) and ADDIU adds
the lower 16 bits of the address.}
\RU{ADDIU следует за JALR (помните о \IT{branch delay slots}?).}
\EN{ADDIU follows JALR (remember \IT{branch delay slots}?).}
\RU{Регистр \$4 также называется \$A0, который используется для передачи первого аргумента функции}%
\EN{The register \$4 is also called \$A0, which is used for passing the first function argument}%
\footnote{\RU{Таблица регистров в MIPS доступна в приложении}\EN{The MIPS registers table %
is available in appendix} \myref{MIPS_registers_ref}}.

\index{MIPS!\Instructions!JALR}
\RU{JALR (\q{Jump and Link Register}) делает переход по адресу в регистре \$25 (там адрес \puts) 
при этом сохраняя адрес следующей инструкции (LW) в \ac{RA}.}
\EN{JALR (\q{Jump and Link Register}) jumps to the address stored in the \$25 register (address of \puts) 
while saving the address of the next instruction (LW) in \ac{RA}.}
\RU{Это так же как и в ARM}\EN{This is very similar to ARM}.
\RU{И ещё одна важная вещь: адрес сохраняемый в \ac{RA} это адрес не следующей инструкции (потому что
это \IT{delay slot} и исполняется перед инструкцией перехода),
а инструкции после неё (после \IT{delay slot}).}
\EN{Oh, and one important thing is that the address saved in \ac{RA} is not the address of the next instruction (because
it's in a \IT{delay slot} and is executed before the jump instruction),
but the address of the instruction after the next one (after the \IT{delay slot}).}
\RU{Таким образом во время исполнения \TT{JALR} в \ac{RA} записывается $PC + 8$. В нашем случае это адрес
инструкции LW следующей после ADDIU.}
\EN{Hence, $PC + 8$ is written to \ac{RA} during the execution of \TT{JALR}, in our case, this is the address of the LW
instruction next to ADDIU.}

\RU{LW (\q{Load Word}) в строке 20 восстанавливает \ac{RA} из локального стека 
(эта инструкция скорее часть эпилога функции).}
\EN{LW (\q{Load Word}) at line 20 restores \ac{RA} from the local stack 
(this instruction is actually part of the function epilogue).}

\index{MIPS!\Pseudoinstructions!MOVE}
\RU{MOVE в строке 22 копирует значение из регистра \$0 (\$ZERO) в \$2 (\$V0).}
\EN{MOVE at line 22 copies the value from the \$0 (\$ZERO) register to \$2 (\$V0).}
\label{MIPS_zero_register}
\RU{В MIPS есть \IT{константный} регистр, всегда содержащий ноль.}
\EN{MIPS has a \IT{constant} register, which always holds zero.}
\RU{Должно быть, разработчики MIPS решили что 0 это самая востребованная константа в программировании,
так что пусть будет использоваться регистр \$0, всякий раз, когда будет нужен 0.}
\EN{Apparently, the MIPS developers came up with the idea that zero is in fact the busiest constant in the computer programming,
so let's just use the \$0 register every time zero is needed.}
\RU{Другой интересный факт: в MIPS нет инструкции, копирующей значения из регистра в регистр.}
\EN{Another interesting fact is that MIPS lacks an instruction that transfers data between registers.}
\RU{На самом деле}\EN{In fact}, \TT{MOVE DST, SRC} \RU{это}\EN{is} \TT{ADD DST, SRC, \$ZERO} ($DST=SRC+0$), 
\RU{которая делает тоже самое}\EN{which does the same}.
\RU{Очевидно, разработчики MIPS хотели сделать как можно более компактную таблицу опкодов.}
\EN{Apparently, the MIPS developers wanted to have a compact opcode table.}
\RU{Это не значит, что сложение происходит во время каждой инструкции MOVE.}
\EN{This does not mean an actual addition happens at each MOVE instruction.}
\RU{Скорее всего, эти псевдоинструкции оптимизируются в \ac{CPU} и \ac{ALU} никогда не используется.}
\EN{Most likely, the \ac{CPU} optimizes these pseudoinstructions and the \ac{ALU} is never used.}

\index{MIPS!\Instructions!J}
\RU{J в строке 24 делает переход по адресу в \ac{RA}, и это работает как выход из функции.}
\EN{J at line 24 jumps to the address in \ac{RA}, which is effectively performing a return from the function.}
\RU{ADDIU после J на самом деле исполняется перед J (помните о \IT{branch delay slots}?) 
и это часть эпилога функции.}
\EN{ADDIU after J is in fact executed before J (remember \IT{branch delay slots}?) 
and is part of the function epilogue.}

\RU{Вот листинг сгенерированный \IDA. Каждый регистр имеет свой псевдоним:}
\EN{Here is also a listing generated by \IDA. Each register here has its own pseudoname:}

\lstinputlisting[caption=\Optimizing GCC 4.4.5 (\IDA),numbers=left]{patterns/01_helloworld/MIPS/hw_O3_IDA.lst.\LANG}

\RU{Инструкция в строке 15 сохраняет GP в локальном стеке. Эта инструкция мистическим образом отсутствует
в листинге от GCC, может быть из-за ошибки в самом GCC\footnote{Очевидно, функция вывода листингов не так критична
для пользователей GCC, поэтому там вполне могут быть неисправленные ошибки.}.}
\EN{The instruction at line 15 saves the GP value into the local stack, and this instruction is missing mysteriously from the GCC output listing, maybe by a GCC error\footnote{Apparently, functions generating listings 
are not so critical to GCC users, so some unfixed errors may still exist.}.}
\RU{Значение GP должно быть сохранено, потому что всякая функция может работать со своим собственным окном данных
размером 64KiB.}
\EN{The GP value has to be saved indeed, because each function can use its own 64KiB data window.}

\RU{Регистр, содержащий адрес функции \puts называется \$T9, потому что регистры с префиксом T- называются
\q{temporaries} и их содержимое можно не сохранять.}
\EN{The register containing the \puts address is called \$T9, because registers prefixed with T- are called
\q{temporaries} and their contents may not be preserved.}

\subsection{\NonOptimizing GCC}

\NonOptimizing GCC \RU{более многословный}\EN{is more verbose}.

\lstinputlisting[caption=\NonOptimizing GCC 4.4.5 (\assemblyOutput),numbers=left]{patterns/01_helloworld/MIPS/hw_O0.s.\LANG}

\RU{Мы видим, что регистр FP используется как указатель на фрейм стека.}
\EN{We see here that register FP is used as a pointer to the stack frame.}
\RU{Мы также видим 3 \ac{NOP}-а.}\EN{We also see 3 \ac{NOP}s.}
\RU{Второй и третий следуют за инструкциями перехода.}
\EN{The second and third of which follow the branch instructions.}

\RU{Вероятно, компилятор GCC всегда добавляет \ac{NOP}-ы (из-за \IT{branch delay slots})
после инструкций переходов и затем, если включена оптимизация, от них может избавляться.}%
\EN{Perhaps, the GCC compiler always adds \ac{NOP}s (because of \IT{branch delay slots}) after branch
instructions and then, if optimization is turned on, maybe eliminates them.}
\RU{Так что они остались здесь}\EN{So in this case they are left here}.

\RU{Вот также листинг от \IDA:}
\EN{Here is also \IDA listing:}

\lstinputlisting[caption=\NonOptimizing GCC 4.4.5 (\IDA),numbers=left]{patterns/01_helloworld/MIPS/hw_O0_IDA.lst.\LANG}

\index{MIPS!\Pseudoinstructions!LA}
\RU{Интересно что \IDA распознала пару инструкций LUI/ADDIU и собрала их в одну псевдоинструкцию 
LA (\q{Load Address}) в строке 15.}
\EN{Interestingly, \IDA recognized the LUI/ADDIU instructions pair and coalesced them into one 
LA (\q{Load Address}) pseudoinstruction at line 15.}
\RU{Мы также видим, что размер этой псевдоинструкции 8 байт!}
\EN{We may also see that this pseudoinstruction has a size of 8 bytes!}
\RU{Это псевдоинструкция (или \IT{макрос}), потому что это не настоящая инструкция MIPS, а скорее
просто удобное имя для пары инструкций.}
\EN{This is a pseudoinstruction (or \IT{macro}) because it's not a real MIPS instruction, but rather
a handy name for an instruction pair.}

\index{MIPS!\Pseudoinstructions!NOP}
\index{MIPS!\Instructions!OR}
\RU{Ещё кое что: \IDA не распознала \ac{NOP}-инструкции в строках 22, 26 и 41.}
\EN{Another thing is that \IDA doesn't recognize \ac{NOP} instructions, so here they are at lines 22, 26 and 41.}
\RU{Это}\EN{It is} \TT{OR \$AT, \$ZERO}.
\RU{По своей сути это инструкция, применяющая операцию ИЛИ к содержимому регистра \$AT с нулем, что,
конечно же, холостая операция.}
\EN{Essentially, this instruction applies the OR operation to the contents of the \$AT register
with zero, which is, of course, an idle instruction.}
\RU{MIPS, как и многие другие \ac{ISA}, не имеет отдельной \ac{NOP}-инструкции.}
\EN{MIPS, like many other \ac{ISA}s, doesn't have a separate \ac{NOP} instruction.}

\subsection{\RU{Роль стекового фрейма в этом примере}\EN{Role of the stack frame in this example}}

\RU{Адрес текстовой строки передается в регистре.}
\EN{The address of the text string is passed in the register.}
\RU{Так зачем устанавливать локальный стек?}\EN{Why setup a local stack anyway?}
\RU{Причина в том, что значения регистров \ac{RA} и GP должны быть сохранены где-то
(потому что вызывается \printf) и для этого используется локальный стек.}
\EN{The reason for this lies in the fact that the values of registers \ac{RA} and GP have to be saved somewhere 
(because \printf is called), and the local stack is used for this purpose.}
\RU{Если бы это была \gls{leaf function}, тогда можно было бы избавиться от пролога и эпилога функции. Например:}
\EN{If this was a \gls{leaf function}, it would have been possible to get rid of the function prologue and epilogue,
for example:} \myref{MIPS_leaf_function_ex1}.

\subsection{\Optimizing GCC: \RU{загрузим в}\EN{load it into} GDB}

\index{GDB}
\lstinputlisting[caption=\RU{пример сессии в GDB}\EN{sample GDB session}]{patterns/01_helloworld/MIPS/O3_GDB.txt}

\fi

\section{\Conclusion{}}

\ifdefined\RUSSIAN
Основная разница между кодом x86/ARM и x64/ARM64 в том, что указатель на строку теперь 64-битный.
Действительно, ведь для того современные \ac{CPU} и стали 64-битными, потому что подешевела память,
её теперь можно поставить в компьютер намного больше, и чтобы её адресовать, 32-х бит уже
недостаточно.
Поэтому все указатели теперь 64-битные.
\fi

\ifdefined\ENGLISH
The main difference between x86/ARM and x64/ARM64 code is that the pointer to the string is now 64-bits in length.
Indeed, modern \ac{CPU}s are now 64-bit due to both the reduced cost of memory and the greater demand for it by modern applications. 
We can add much more memory to our computers than 32-bit pointers are able to address.
As such, all pointers are now 64-bit.
\fi

\ifdefined\DUTCH
Het grootste verschil tussen x86/ARM en x64/ARM64 code is dat de pointer naar de string nu 64-bits in lengte is.
De meeste moderne \ac{CPU}s zijn tegenwoordig 64-bit wegens zowel de verminderde gebruik van geheugen, als de grote vraag ervoor door moderne applicaties.
We kunnen hierdoor veel meer geheugen aan onze computers toevoegen dan dat 32-bit pointers kunnen aanspreken.
Bijgevolg zijn alle pointers nu 64-bit.
\fi

% sections
\ifdefined\IncludeExercises
\section{\Exercises}

\begin{itemize}
	\item \url{http://challenges.re/48}
	\item \url{http://challenges.re/49}
\end{itemize}


\fi

\EN{\section{\Stack}
\label{sec:stack}
\myindex{\Stack}

The stack is one of the most fundamental data structures in computer science
\footnote{\href{http://go.yurichev.com/17119}{wikipedia.org/wiki/Call\_stack}}.
\ac{AKA} \ac{LIFO}.

Technically, it is just a block of memory in process memory along with the \ESP or \RSP register in x86 or x64, or the \ac{SP} register in ARM, as a pointer within that block.

\myindex{ARM!\Instructions!PUSH}
\myindex{ARM!\Instructions!POP}
\myindex{x86!\Instructions!PUSH}
\myindex{x86!\Instructions!POP}
The most frequently used stack access instructions are \PUSH and \POP (in both x86 and ARM Thumb-mode). 
\PUSH subtracts from \ESP/\RSP/\ac{SP} 4 in 32-bit mode (or 8 in 64-bit mode) and then writes the contents of its sole operand to the memory address pointed by \ESP/\RSP/\ac{SP}.

\POP is the reverse operation: retrieve the data from the memory location that \ac{SP} points to, 
load it into the instruction operand (often a register) and then add 4 (or 8) to the \gls{stack pointer}.

After stack allocation, the \gls{stack pointer} points at the bottom of the stack.
\PUSH decreases the \gls{stack pointer} and \POP increases it.
The bottom of the stack is actually at the beginning of the memory allocated for the stack block. It seems strange, but that's the way it is.

ARM supports both descending and ascending stacks.

\myindex{ARM!\Instructions!STMFD}
\myindex{ARM!\Instructions!LDMFD}
\myindex{ARM!\Instructions!STMED}
\myindex{ARM!\Instructions!LDMED}
\myindex{ARM!\Instructions!STMFA}
\myindex{ARM!\Instructions!LDMFA}
\myindex{ARM!\Instructions!STMEA}
\myindex{ARM!\Instructions!LDMEA}

For example the \ac{STMFD}/\ac{LDMFD}, \ac{STMED}/\ac{LDMED} instructions are intended to deal with a descending stack (grows downwards, starting with a high address and progressing to a lower one).
The \ac{STMFA}/\ac{LDMFA}, \ac{STMEA}/\ac{LDMEA} instructions are intended to deal with an ascending stack (grows upwards, starting from a low address and progressing to a higher one).

% It might be worth mentioning that STMED and STMEA write first,
% and then move the pointer,
% and that LDMED and LDMEA move the pointer first, and then read.
% In other words, ARM not only lets the stack grow in a non-standard direction,
% but also in a non-standard order.
% Maybe this can be in the glossary, which would explain why E stands for "empty".

\subsection{Why does the stack grow backwards?}
\label{stack_grow_backwards}

Intuitively, we might think that the stack grows upwards, i.e. towards higher addresses, like any other data structure.

The reason that the stack grows backward is probably historical.
When the computers were big and occupied a whole room, it was easy to divide memory into two parts, one for the \gls{heap} and one for the stack.
Of course, it was unknown how big the \gls{heap} and the stack would be during program execution, so this solution was the simplest possible.

\begin{center}
	\begin{tikzpicture}
	\tikzstyle{every path}=[thick]

	\node [rectangle,draw,minimum width=6cm, minimum height=2cm] (memory) {};
	\node [] [right=0.2cm of memory.west] (heap) {\MLHeap};
	\node [] [left=0.2cm of memory.east] (stack) {\MLStack};

	\node [] (center1) [right=2cm of memory.west] {};
	\node [] (center2) [left=2cm of memory.east] {};

	\draw [->] (heap) -- (center1);
	\draw [->] (stack) -- (center2);

	\node [] [above left=1.1cm and 0.2cm of heap] (t1) {\MLStartOfHeap};
	\node [] [above right=1.1cm and 0.2cm of stack] (t2) {\MLStartOfStack};

	\draw [->] (t1) -- (memory.west);
	\draw [->] (t2) -- (memory.east);

	\end{tikzpicture}
\end{center}


In \RitchieThompsonUNIX we can read:

\begin{framed}
\begin{quotation}
The user-core part of an image is divided into three logical segments. The program text segment begins at location 0 in the virtual address space. During execution, this segment is write-protected and a single copy of it is shared among all processes executing the same program. At the first 8K byte boundary above the program text segment in the virtual address space begins a nonshared, writable data segment, the size of which may be extended by a system call. Starting at the highest address in the virtual address space is a stack segment, which automatically grows downward as the hardware's stack pointer fluctuates.
\end{quotation}
\end{framed}

This reminds us how some students write two lecture notes using only one notebook:
notes for the first lecture are written as usual, 
and notes for the second one are written from the end of notebook, by flipping it.
Notes may meet each other somewhere in between, in case of lack of free space.

% I think if we want to expand on this analogy,
% one might remember that the line number increases as as you go down a page.
% So when you decrease the address when pushing to the stack, visually,
% the stack does grow upwards.
% Of course, the problem is that in most human languages,
% just as with computers,
% we write downwards, so this direction is what makes buffer overflows so messy.

\subsection{What is the stack used for?}

% subsections
\EN{\input{patterns/02_stack/01_saving_ret_addr_EN}}
\RU{\input{patterns/02_stack/01_saving_ret_addr_RU}}
\DE{\input{patterns/02_stack/01_saving_ret_addr_DE}}
\FR{\input{patterns/02_stack/01_saving_ret_addr_FR}}
\PTBR{\input{patterns/02_stack/01_saving_ret_addr_PTBR}}
\ITA{\input{patterns/02_stack/01_saving_ret_addr_ITA}}

\subsection{\RU{Передача параметров функции}\EN{Passing function arguments}}

\RU{Самый распространенный способ передачи параметров в x86 называется}
\EN{The most popular way to pass parameters in x86 is called} \q{cdecl}:

\begin{lstlisting}
push arg3
push arg2
push arg1
call f
add esp, 12 ; 4*3=12
\end{lstlisting}

\RU{Вызываемая функция получает свои параметры также через указатель стека.}
\EN{\Gls{callee} functions get their arguments via the stack pointer.}

\RU{Следовательно, так расположены значения в стеке перед исполнением самой первой инструкции
функции \ttf{}:}
\EN{Therefore, this is how the argument values are located in the stack before the execution
of the \ttf{} function's very first instruction:}

\begin{center}
\begin{tabular}{ | l | l | }
\hline
ESP & \RU{адрес возврата}\EN{return address} \\
\hline
ESP+4 & \argument \#1, \MarkedInIDAAs{} \TT{arg\_0} \\
\hline
ESP+8 & \argument \#2, \MarkedInIDAAs{} \TT{arg\_4} \\
\hline
ESP+0xC & \argument \#3, \MarkedInIDAAs{} \TT{arg\_8} \\
\hline
\dots & \dots \\
\hline
\end{tabular}
\end{center}

\ifx\LITE\undefined
\RU{См. также в соответствующем разделе о других способах передачи аргументов через стек}
\EN{For more information on other calling conventions see also section}~(\myref{sec:callingconventions}).
\fi
\RU{Важно отметить, что, в общем, никто не заставляет программистов передавать параметры именно через стек,
это не является требованием к исполняемому коду.}
\EN{It is worth noting that nothing obliges programmers to pass arguments through the stack. It is not a requirement.}
\RU{Вы можете делать это совершенно иначе, не используя стек вообще.}
\EN{One could implement any other method without using the stack at all.}

\RU{К примеру, можно выделять в \glslink{heap}{куче} место для аргументов, 
заполнять их и передавать в функцию указатель на это место через \EAX. И это вполне будет работать}%
\EN{For example, it is possible to allocate a space for arguments in the \gls{heap}, fill it and pass it to a function 
via a pointer to this block in the \EAX register. This will work}%
\footnote{\RU{Например, в книге Дональда Кнута \q{Искусство программирования}, в разделе 1.4.1 
посвященном подпрограммам \cite[раздел 1.4.1]{Knuth:1998:ACP:521463}, 
мы можем прочитать о возможности располагать параметры для вызываемой подпрограммы после инструкции \JMP,
передающей управление подпрограмме. Кнут описывает, что это было особенно удобно для компьютеров IBM System/360.}%
\EN{For example, in the \q{The Art of Computer Programming} book by Donald Knuth, 
in section 1.4.1 dedicated to subroutines \cite[section 1.4.1]{Knuth:1998:ACP:521463},
we could read that one way to supply arguments to a subroutine is simply to list them after the \JMP instruction
passing control to subroutine. Knuth explains that this method was particularly convenient on IBM System/360.}}.
\RU{Однако традиционно сложилось, что в x86 и ARM передача аргументов происходит именно через стек.}
% I am unsure about what this comment means.
% My guess is that the arguments are put in the memory position after
% the jump instruction, so you could say:
% "one way to supply arguments to a subroutine is simply to list them in memory
% after the \JMP instruction that passes control to the subroutine."
% Right now, "after" also sounds like it refers to the time after
% the jump happens, which I think is too late.
\EN{However, it is a convenient custom in x86 and ARM to use the stack for this purpose.} \\
\\
\RU{Кстати, вызываемая функция не имеет информации о количестве переданных ей аргументов.}
\EN{By the way, the \gls{callee} function does not have any information about how many arguments were passed.}
\RU{Функции Си с переменным количеством аргументов (как \printf) определяют их количество по 
спецификаторам строки формата (начинающиеся со знака \%).}
\EN{C functions with a variable number of arguments (like \printf) determine their number using format string  specifiers (which begin with the \% symbol).}
\RU{Если написать что-то вроде}\EN{If we write something like} 

\begin{lstlisting}
printf("%d %d %d", 1234);
\end{lstlisting}

\printf \RU{выведет 1234, затем ещё два случайных числа, которые волею случая оказались в стеке рядом.}
\EN{will print 1234, and then two random numbers, which were lying next to it in the stack.}\\
\\
\RU{Вот почему не так уж и важно, как объявлять функцию \main}
\EN{That's why it is not very important how we declare the \main function}: \RU{как}\EN{as} \main, 
\TT{main(int argc, char *argv[])} 
\RU{либо}\EN{or} \TT{main(int argc, char *argv[], char *envp[])}.

\RU{В реальности, \ac{CRT}-код вызывает \main примерно так:}
\EN{In fact, the \ac{CRT}-code is calling \main roughly as:}

\begin{lstlisting}
push envp
push argv
push argc
call main
...
\end{lstlisting}

\RU{Если вы объявляете \main без аргументов, они, тем не менее, присутствуют в стеке, но не используются.}
\EN{If you declare \main as \main without arguments, they are, nevertheless, still present in the stack, but
are not used.}
\RU{Если вы объявите \main как}\EN{If you declare \main as} \TT{main(int argc, char *argv[])}, 
\RU{вы можете использовать два первых аргумента, а третий останется для вашей функции \q{невидимым}.}
\EN{you will be able to use first two arguments, and the third will remain \q{invisible} for your function.}
\RU{Более того, можно даже объявить}\EN{Even more, it is possible to declare} \TT{main(int argc)}, 
\RU{и это будет работать}\EN{and it will work}.


\EN{\subsubsection{Local variable storage}

A function could allocate space in the stack for its local variables just by decreasing 
the \gls{stack pointer} towards the stack bottom.

% I think here, "stack bottom" means the lowest address in the stack space,
% but the reader might also think it means towards the top of the stack space,
% like in a pop, so you might change "towards the stack bottom" to
% "towards the lowest address of the stack", or just take it out,
% since "decreasing" also suggests that.

Hence, it's very fast, no matter how many local variables are defined.
It is also not a requirement to store local variables in the stack.
You could store local variables wherever you like, 
but traditionally this is how it's done.

}
\RU{\subsubsection{Хранение локальных переменных}

Функция может выделить для себя некоторое место в стеке для локальных переменных, просто отодвинув 
\glslink{stack pointer}{указатель стека} глубже к концу стека.

% I think here, "stack bottom" means the lowest address in the stack space,
% but the reader might also think it means towards the top of the stack space,
% like in a pop, so you might change "towards the stack bottom" to
% "towards the lowest address of the stack", or just take it out,
% since "decreasing" also suggests that.

Это очень быстро вне зависимости от количества локальных переменных.
Хранить локальные переменные в стеке не является необходимым требованием. 
Вы можете хранить локальные переменные где угодно. 
Но по традиции всё сложилось так.

}
\PTBR{\subsubsection{Armazenamento de variáveis locais}

Uma função poderia alocar espaço na pilha para suas variáveis locais simplesmente decrementando o ponteiro da pilha.

% I think here, "stack bottom" means the lowest address in the stack space,
% but the reader might also think it means towards the top of the stack space,
% like in a pop, so you might change "towards the stack bottom" to
% "towards the lowest address of the stack", or just take it out,
% since "decreasing" also suggests that.

Consequentemente, é muito rápido, não importando quantas variáveis locais serão definidas.
Também não é um requisito armazenar variáveis locais na pilha.
Você pode armazenar variáveis locais onde você quiser, mas, tradicionalmente, é assim que é feito.

}
\EN{\input{patterns/02_stack/04_alloca/main_EN}}
\FR{\input{patterns/02_stack/04_alloca/main_FR}}
\RU{\input{patterns/02_stack/04_alloca/main_RU}}
\PTBR{\input{patterns/02_stack/04_alloca/main_PTBR}}
\ITA{\input{patterns/02_stack/04_alloca/main_ITA}}
\DE{\input{patterns/02_stack/04_alloca/main_DE}}

\subsection{(Windows) SEH}
\index{Windows!Structured Exception Handling}

\RU{В стеке хранятся записи \ac{SEH} для функции (если они присутствуют)}%
\EN{\ac{SEH} records are also stored on the stack (if they are present).}.

\ifx\LITE\undefined
\RU{Читайте больше о нем здесь}\EN{Read more about it}: (\myref{sec:SEH}).
\fi

\subsection{\RU{Защита от переполнений буфера}\EN{Buffer overflow protection}\PTBR{Proteção contra estouro de buffer}}

\RU{Здесь больше об этом}\EN{More about it here}\PTBR{Mais sobre aqui}~(\myref{subsec:bufferoverflow}).



\subsubsection{Automatic deallocation of data in stack}

Perhaps the reason for storing local variables and SEH records in the stack is that they are freed automatically upon function exit,
using just one instruction to correct the stack pointer (it is often \ADD).
Function arguments, as we could say, are also deallocated automatically at the end of function.
In contrast, everything stored in the \IT{heap} must be deallocated explicitly.

% sections
\EN{\input{patterns/02_stack/07_layout_EN}}
\RU{\subsection{Разметка типичного стека}

Разметка типичного стека в 32-битной среде
перед исполнением самой первой инструкции функции выглядит так:

\input{patterns/02_stack/stack_layout}

% I think this only applies to RISC architectures
% that don't have a POP instruction that only lets you read one value
% (ie. ARM and MIPS).
% In x86, the return address is saved before entering the function,
% and the function does not have the chance to save the frame pointer.
% Also, you should mention that this is how the stack looks like
% right after the function prologue,
% which some readers might think is the first instruction,
% but is needed to save the frame pointer.

}
\PTBR{\subsection{Um modelo típico de pilha}

Um modelo típico de pilha em um ambiente 32-bits no início de uma função,
antes da execução da primeira instrução, se parece com isso:

\input{patterns/02_stack/stack_layout}

% I think this only applies to RISC architectures
% that don't have a POP instruction that only lets you read one value
% (ie. ARM and MIPS).
% In x86, the return address is saved before entering the function,
% and the function does not have the chance to save the frame pointer.
% Also, you should mention that this is how the stack looks like
% right after the function prologue,
% which some readers might think is the first instruction,
% but is needed to save the frame pointer.
}
\section{\RU{Мусор в стеке}\EN{Noise in stack}}

\RU{Часто в этой книге говорится о \q{шуме} или \q{мусоре} в стеке или памяти.}
\EN{Often in this book \q{noise} or \q{garbage} values in the stack or memory are mentioned.}
\RU{Откуда он берется}\EN{Where do they come from}?
\RU{Это то, что осталось там после исполнения предыдущих функций.}
\EN{These are what was left in there after other functions' executions.}
\RU{Короткий пример}\EN{Short example}:

\lstinputlisting{patterns/02_stack/08_noise/st.c}

\RU{Компилируем}\EN{Compiling}\dots

\lstinputlisting[caption=\NonOptimizing MSVC 2010]{patterns/02_stack/08_noise/st.asm}

\RU{Компилятор поворчит немного}\EN{The compiler will grumble a little bit}\dots

\begin{lstlisting}
c:\Polygon\c>cl st.c /Fast.asm /MD
Microsoft (R) 32-bit C/C++ Optimizing Compiler Version 16.00.40219.01 for 80x86
Copyright (C) Microsoft Corporation.  All rights reserved.

st.c
c:\polygon\c\st.c(11) : warning C4700: uninitialized local variable 'c' used
c:\polygon\c\st.c(11) : warning C4700: uninitialized local variable 'b' used
c:\polygon\c\st.c(11) : warning C4700: uninitialized local variable 'a' used
Microsoft (R) Incremental Linker Version 10.00.40219.01
Copyright (C) Microsoft Corporation.  All rights reserved.

/out:st.exe
st.obj
\end{lstlisting}

\RU{Но когда мы запускаем}\EN{But when we run the compiled program}\dots

\begin{lstlisting}
c:\Polygon\c>st
1, 2, 3
\end{lstlisting}

\RU{Ох. Вот это странно. Мы ведь не устанавливали значения никаких переменных в}\EN{Oh, 
what a weird thing! We did not set any variables in} \TT{f2()}. 
\RU{Эти значения --- это \q{привидения}, которые всё ещё в стеке.}
\EN{These are \q{ghosts} values, which are still in the stack.}

\clearpage
\RU{Загрузим пример в}\EN{Let's load the example into} \olly:

\begin{figure}[H]
\centering
\includegraphics[scale=\FigScale]{patterns/02_stack/08_noise/olly1.png}
\caption{\olly: \TT{f1()}}
\label{fig:stack_noise_olly1}
\end{figure}

\RU{Когда}\EN{When} \TT{f1()} \RU{заполняет переменные}\EN{assigns the variables} $a$, $b$ \AndENRU $c$ 
\RU{они сохраняются по адресу}\EN{, their values are stored at the address} \TT{0x1FF860} 
\RU{\etc{}.}\EN{and so on.}

\clearpage
\RU{А когда исполняется}\EN{And when} \TT{f2()}\EN{ executes}:

\begin{figure}[H]
\centering
\includegraphics[scale=\FigScale]{patterns/02_stack/08_noise/olly2.png}
\caption{\olly: \TT{f2()}}
\label{fig:stack_noise_olly2}
\end{figure}

... $a$, $b$ \AndENRU $c$ \RU{в функции}\EN{of} \TT{f2()} \RU{находятся по тем же адресам!}
\EN{are located at the same addresses!}
\RU{Пока никто не перезаписал их, так что они здесь в нетронутом виде.}
\EN{No one has overwritten the values yet, so at that point they are still untouched.}

\RU{Для создания такой странной ситуации несколько функций должны исполняться друг за другом
и \ac{SP} должен быть одинаковым при входе в функции, т.е. у функций должно быть равное количество
аргументов). Тогда локальные переменные будут расположены в том же месте стека.}
\EN{So, for this weird situation to occur, several functions have to be called one after another and
\ac{SP} has to be the same at each function entry (i.e., they have the same number
of arguments). Then the local variables will be located at the same positions in the stack.}

\RU{Подводя итоги, все значения в стеке (да и памяти вообще) это значения оставшиеся от 
исполнения предыдущих функций.}
\EN{Summarizing, all values in the stack (and memory cells in general) 
have values left there from previous function executions.}
\RU{Строго говоря, они не случайны, они скорее непредсказуемы.}
\EN{They are not random in the strict sense, but rather have unpredictable values.}

\RU{А как иначе}\EN{Is there another option}?
\RU{Можно было бы очищать части стека перед исполнением каждой функции,
но это слишком много лишней (и ненужной) работы.}
\EN{It probably would be possible to clear portions of the stack before each function execution,
but that's too much extra (and unnecessary) work.}

\subsection{MSVC 2013}

\EN{The example was compiled by}\RU{Этот пример был скомпилирован в} MSVC 2010.
\EN{But the reader of this book made attempt to compile this example in MSVC 2013, ran it, and got all 3 numbers reversed:}%
\RU{Но один читатель этой книги сделал попытку скомпилировать пример в MSVC 2013, запустил и увидел 3 числа в обратном порядке:}

\begin{lstlisting}
c:\Polygon\c>st
3, 2, 1
\end{lstlisting}

\EN{Why?}\RU{Почему?}

\EN{I also compiled this example in MSVC 2013 and saw this:}%
\RU{Я также попробовал скомпилировать этот пример в MSVC 2013 и увидел это:}

\begin{lstlisting}[caption=MSVC 2013]
_a$ = -12						; size = 4
_b$ = -8						; size = 4
_c$ = -4						; size = 4
_f2	PROC

...

_f2	ENDP

_c$ = -12						; size = 4
_b$ = -8						; size = 4
_a$ = -4						; size = 4
_f1	PROC

...

_f1	ENDP
\end{lstlisting}

\EN{Unlike MSVC 2010, MSVC 2013 allocated a/b/c variables in function \TT{f2()} in reverse order.}%
\RU{В отличии от MSVC 2010, MSVC 2013 разместил переменные a/b/c в функции \TT{f2()} в обратном порядке.}
\EN{And this is completely correct, because \CCpp standards has no rule, in which order local variables must be allocated in the local stack, if at all.}%
\RU{И это полностью корректно, потому что в стандартах \CCpp нет правила, в каком порядке локальные переменные должны быть размещены в локальном стеке,
если вообще.}
\EN{The reason of difference is because MSVC 2010 has one way to do it, and MSVC 2013 has probably something changed inside of compiler guts, so it behaves
slightly different.}%
\RU{Разница есть из-за того что MSVC 2010 делает это одним способом, а в MSVC 2013, вероятно, что-то немного изменили во внутренностях компилятора,
так что он ведет себя слегка иначе.}


\section{\Exercises}

\subsection{\Exercise \#1}
\label{exercise_stack_1}

\RU{Если это скомпилировать в MSVC и запустить, появится три числа. Откуда они берутся? 
Откуда они берутся если скомпилировать в MSVC с оптимизациями (\Ox)?}
\EN{If to compile this piece of code in MSVC and run, a three number will be printed. 
Where they are came from?
Where they are came from if to compile it in MSVC with optimization (\Ox)?}
\RU{Почему в GCC ситуация совсем иная}\EN{Why the situation is completely different in GCC}?

\begin{lstlisting}
#include <stdio.h>

int main()
{
	printf ("%d, %d, %d\n");

	return 0;
};
\end{lstlisting}

\RU{Ответ}\EN{Answer}: \ref{exercise_solutions_stack_1}.

}
\FR{\section{\Stack}
\label{sec:stack}
\myindex{\Stack}

La pile est une des structures de données les plus fondamentales en informatique.
\footnote{\href{http://go.yurichev.com/17119}{wikipedia.org/wiki/Call\_stack}}.
\ac{AKA} \ac{LIFO}.

Techniquement, il s'agit d'un bloc de mémoire présent dans l'espace d'adressage
d'un processus et qui est utilisé par le registre \ESP ou \RSP en x86 ou x64,
ou par le registre \ac{SP} en ARM comme un pointeur dans ce bloc mémoire. 

\myindex{ARM!\Instructions!PUSH}
\myindex{ARM!\Instructions!POP}
\myindex{x86!\Instructions!PUSH}
\myindex{x86!\Instructions!POP}
Les instructions d'accès à la pile sont \PUSH et \POP (en x86 ainsi qu'en ARM Thumb-mode).
\PUSH soustrait à \ESP/\RSP/\ac{SP} 4 en mode 32-bit (ou 8 en mode 64-bit) et écrit
ensuite le contenu de l'opérande associée à l'adresse mémoire pointée par \ESP/\RSP/\ac{SP}.

\POP est l'operation inverse: elle récupére la donnée depuis l'adresse mémoire pointée par \ac{SP},
l'écrit dans l'opérande associée (souvent un registre) puis ajoute 4 (ou 8) au \glslink{stack pointer}{pointeur de pile}.

Après une allocation sur la pile, le \glslink{stack pointer}{pointeur de pile} pointe sur le bas de la pile.
\PUSH décrémente le \gls{stack pointer} et \POP l'incrémente.

Le bas de la pile représente en réalité le début de la mémoire allouée pour
 le bloc de pile. Cela semble étrange, mais c'est comme ça.

ARM supporte à la fois les piles ascendantes et descendantes.

\myindex{ARM!\Instructions!STMFD}
\myindex{ARM!\Instructions!LDMFD}
\myindex{ARM!\Instructions!STMED}
\myindex{ARM!\Instructions!LDMED}
\myindex{ARM!\Instructions!STMFA}
\myindex{ARM!\Instructions!LDMFA}
\myindex{ARM!\Instructions!STMEA}
\myindex{ARM!\Instructions!LDMEA}

Par exemple les instructions \ac{STMFD}/\ac{LDMFD}, \ac{STMED}/\ac{LDMED} sont utilisées pour gérer les piles
descendantes (qui grandissent vers le bas en commençant avec une adresse haute et évoluent vers une plus basse).

Les instructions \ac{STMFA}/\ac{LDMFA}, \ac{STMEA}/\ac{LDMEA} sont utilisées pour gérer les piles montantes
(qui grandissent vers les adresses hautes de l'adresse space, en commençant avec une adresse située en bas de l'adresse space)

% It might be worth mentioning that STMED and STMEA write first,
% and then move the pointer,
% and that LDMED and LDMEA move the pointer first, and then read.
% In other words, ARM not only lets the stack grow in a non-standard direction,
% but also in a non-standard order.
% Maybe this can be in the glossary, which would explain why E stands for "empty".

\subsection{Pourquoi la pile grandit en descendant ?}
\label{stack_grow_backwards}

Intuitivement, on pourrait penser que la pile grandit vers le haut, i.e. vers des
adresses plus élevées, comme n'importe qu'elle autre structure de données.

La raison pour laquelle la pile grandit vers le bas est probablement historique.
Dans le passé, les ordinateurs étaient énormes et occupaient des piéces entières,
il était facile de diviser la mémoire en deux parties, une pour le \gls{heap} et
une pour la pile.
Evidemment, on ignorait quelle serait la taille du \gls{heap} et de la pile durant
l'éxécution du progamme, donc cette solution était la plus simple possible.

\begin{center}
	\begin{tikzpicture}
	\tikzstyle{every path}=[thick]

	\node [rectangle,draw,minimum width=6cm, minimum height=2cm] (memory) {};
	\node [] [right=0.2cm of memory.west] (heap) {\MLHeap};
	\node [] [left=0.2cm of memory.east] (stack) {\MLStack};

	\node [] (center1) [right=2cm of memory.west] {};
	\node [] (center2) [left=2cm of memory.east] {};

	\draw [->] (heap) -- (center1);
	\draw [->] (stack) -- (center2);

	\node [] [above left=1.1cm and 0.2cm of heap] (t1) {\MLStartOfHeap};
	\node [] [above right=1.1cm and 0.2cm of stack] (t2) {\MLStartOfStack};

	\draw [->] (t1) -- (memory.west);
	\draw [->] (t2) -- (memory.east);

	\end{tikzpicture}
\end{center}


Dans \RitchieThompsonUNIX on peut lire:

\begin{framed}
\begin{quotation}
The user-core part of an image is divided into three logical segments. The program text segment begins at location 0 in the virtual address space. During execution, this segment is write-protected and a single copy of it is shared among all processes executing the same program. At the first 8K byte boundary above the program text segment in the virtual address space begins a nonshared, writable data segment, the size of which may be extended by a system call. Starting at the highest address in the virtual address space is a pile segment, which automatically grows downward as the hardware's pile pointer fluctuates.
\end{quotation}
\end{framed}

Cela nous rappelle comment certains étudiants prennent des notes pour deux cours différents dans
un seul et même cahier en prenant un cours d'un côté du cahier, et l'autre cours de l'autre côté.
Les notes de cours finissent par se rencontrer à un moment dans le cahier quand il n'y a plus de place.

% I think if we want to expand on this analogy,
% one might remember that the line number increases as as you go down a page.
% So when you decrease the address when pushing to the stack, visually,
% the stack does grow upwards.
% Of course, the problem is that in most human languages,
% just as with computers,
% we write downwards, so this direction is what makes buffer overflows so messy.

\subsection{Quel est le rôle de la pile ?}

% subsections
\subsubsection{Sauvegarder l'adresse de retour de la fonction}

\myparagraph{x86}

\myindex{x86!\Instructions!CALL}
Lorsque l'on appelle une fonction avec une instruction \CALL, l'adresse du point
exactement après cette dernière est sauvegardée sur la pile et un saut inconditionnel
à l'adresse de l'opérande \CALL est exécuté.

\myindex{x86!\Instructions!PUSH}
\myindex{x86!\Instructions!JMP}
L'instruction \CALL est équivalente à la\\
paire d'instructions \INS{PUSH address\_after\_call / JMP operand}.

\myindex{x86!\Instructions!RET}
\myindex{x86!\Instructions!POP}
\RET va chercher une valeur sur la pile et y saute~---ce qui est équivalent à
la paire d'instructions \TT{POP tmp / JMP tmp}.

\myindex{\Stack!\MLStackOverflow}
\myindex{\Recursion}
Déborder de la pile est très facile. Il suffit de lancer une récursion éternelle:

\begin{lstlisting}[style=customc]
void f()
{
	f();
};
\end{lstlisting}

MSVC 2008 signale le problème:

\begin{lstlisting}
c:\tmp6>cl ss.cpp /Fass.asm
Microsoft (R) 32-bit C/C++ Optimizing Compiler Version 15.00.21022.08 for 80x86
Copyright (C) Microsoft Corporation.  All rights reserved.

ss.cpp
c:\tmp6\ss.cpp(4) : warning C4717: 'f' : recursive on all control paths, function will cause runtime stack overflow
\end{lstlisting}

\dots mais génère tout de même le code correspondant:

\begin{lstlisting}[style=customasmx86]
?f@@YAXXZ PROC			; f
; File c:\tmp6\ss.cpp
; Line 2
	push	ebp
	mov	ebp, esp
; Line 3
	call	?f@@YAXXZ	; f
; Line 4
	pop	ebp
	ret	0
?f@@YAXXZ ENDP			; f
\end{lstlisting}

\dots Si nous utilisons l'option d'optimisation du compilateur (option \TT{\Ox})
le code optimisé ne va pas déborder de la pile et au lieu de cela va fonctionner
\IT{correctemment}\footnote{ironique ici}:

\begin{lstlisting}[style=customasmx86]
?f@@YAXXZ PROC			; f
; File c:\tmp6\ss.cpp
; Line 2
$LL3@f:
; Line 3
	jmp	SHORT $LL3@f
?f@@YAXXZ ENDP			; f
\end{lstlisting}

GCC 4.4.1 génère un code similaire dans les deux cas, sans, toutefois émettre
d'avertissement à propos de ce problème.

\myparagraph{ARM}

\myindex{ARM!\Registers!Link Register}
Les programmes ARM utilisent également la pile pour sauver les adresses de retour,
mais différemment.
Comme mentionné dans \q{\HelloWorldSectionName}~(\myref{sec:hw_ARM}),
\ac{RA} est sauvegardé dans \ac{LR} (\gls{link register}).
Si l'on a toutefois besoin d'appeler une autre fonction et d'utiliser le registre
\ac{LR} une fois de plus, sa valeur doit être sauvegardée.
\myindex{Function prologue}
Usuellement, cela se fait dans le prologue de la fonction.

\myindex{ARM!\Instructions!PUSH}
\myindex{ARM!\Instructions!POP}
Souvent, nous voyons des instructions comme \INS{PUSH {R4-R7,LR}} en même temps
que cette instruction dans le prologue \INS{POP {R4-R7,PC}}---ces registres qui
sont utilisés dans la fonction sont sauvegardés sur la pile, \ac{LR} inclus.

\myindex{ARM!Fonction leaf} % FIXME traduire avec feuille ?
Néanmoins, si une fonction n'appelle jamais d'autre fonction, dans la terminologie
\ac{RISC} elle est appelée \IT{\glslink{leaf function}{fonction leaf}}\footnote{\href{http://go.yurichev.com/17064}{infocenter.arm.com/help/index.jsp?topic=/com.arm.doc.faqs/ka13785.html}}.
Ceci a comme conséquence que les fonctions leaf ne sauvegardent pas le registre
\ac{LR} (car elles ne le modifient pas).
Si une telle fonction est petite et utilise un petit nombre de registres, elle
peut ne pas utiliser du tout la pile.
Ainsi, il est possible d'appeler des fonctions leaf sans utiliser la pile.
Ce qui peut être plus rapide sur des vieilles machines x86 car la mémoire externe
n'est pas utilisée pour la pile
\footnote{Il y a quelques temps, sur PDP-11 et VAX, l'instruction CALL (appel d'autres fonctions) était coûteux; jusqu'à 50\%
du temps d'exécution pouvait être passé à ça, il était donc considèré qu'avoir un grand nombre de petite fonction était un \gls{anti-pattern} \InSqBrackets{\TAOUP Chapter 4, Part II}.}.
Cela peut être utile pour des situations où la mémoire pour la pile n'est pas
encore allouée ou disponible.

Quelques exemples de fonctions leaf:
\myref{ARM_leaf_example1}, \myref{ARM_leaf_example2},
\myref{ARM_leaf_example3}, \myref{ARM_leaf_example4}, \myref{ARM_leaf_example5},
\myref{ARM_leaf_example6}, \myref{ARM_leaf_example7}, \myref{ARM_leaf_example10}.


\subsubsection{Passage des arguments de fonction}

Le moyen le plus utilisé pour passer des arguments en x86 est appelé \q{cdecl}:

\begin{lstlisting}[style=customasmx86]
push arg3
push arg2
push arg1
call f
add esp, 12 ; 4*3=12
\end{lstlisting}

La fonction \glslink{callee}{appelée} reçoit ses arguments par la pile.

Voici donc comment sont stockés les arguments sur la pile avant l'exécution
de la première instruction de la fonction \ttf{}:

\begin{center}
\begin{tabular}{ | l | l | }
\hline
ESP & return address \\
\hline
ESP+4 & \argument \#1, \MarkedInIDAAs{} \TT{arg\_0} \\
\hline
ESP+8 & \argument \#2, \MarkedInIDAAs{} \TT{arg\_4} \\
\hline
ESP+0xC & \argument \#3, \MarkedInIDAAs{} \TT{arg\_8} \\
\hline
\dots & \dots \\
\hline
\end{tabular}
\end{center}

Pour plus d'information sur les conventions d'appel, voir cette section~(\myref{sec:callingconventions}).

\par
A propos, la fonction \glslink{callee}{appelée} n'a aucune d'information sur le
nombre d'arguments qui ont été passés.
Les fonctions C avec un nombre variable d'arguments (comme \printf) déterminent
leur nombre en utilisant les spécificateurs de la chaîne de format (qui commencent
pas le symbole \%).

Si nous écrivons quelque comme:

\begin{lstlisting}
printf("%d %d %d", 1234);
\end{lstlisting}

\printf va afficher 1234, et deux autres nombres aléatoires\footnote{Pas aléatoire
dans le sens strict du terme, mais plutôt imprévisibles: \myref{noise_in_stack}},
qui sont situés à côté dans la pile.

\par
C'est pourquoi la façon dont la fonction \main est déclarée n'est pas très importante:
comme \main, \\\TT{main(int argc, char *argv[])} ou \TT{main(int argc, char *argv[], char *envp[])}.

En fait, le code-\ac{CRT} appelle \main, schématiquement, de cette façon:
	
\begin{lstlisting}[style=customasmx86]
push envp
push argv
push argc
call main
...
\end{lstlisting}

Si vous déclarez \main comme \main sans argument, ils sont néanmoins toujours présents
sur la pile, mais ne sont pas utilisés.
Si vous déclarez \main as comme \TT{main(int argc, char *argv[])},
vous pourrez utiliser les deux premiers arguments, et le troisième restera \q{invisible}
pour votre fonction.
Il est même possible de déclarer \main comme \TT{main(int argc)}, cela fonctionnera.

\myparagraph{Autres façons de passer les arguments}

Il est à noter que rien n'oblige les programmeurs à passer les arguments à travers
la pile. Ce n'est pas une exigence.
On peut implémenter n'importe quelle autre méthode sans utiliser du tout la pile.

Une méthode répandue chez les débutants en assembleur est de passer les arguments
par des variables globales, comme:

\lstinputlisting[caption=Code assembleur,style=customasmx86]{patterns/02_stack/global_args.asm}

Mais cette méthode a un inconvénient évident: la fonction \IT{do\_something()}
ne peut pas s'appeler elle-même récursivement (ou par une autre fonction),
car il faudrait écraser ses propres arguments.
La même histoire avec les variables locales: si vous les stocker dans des variables
globales, la fonction ne peut pas s'appeler elle-même.
Et ce n'est pas thread-safe.
\footnote{Correctemment implémenter, chaque thread aurait sa propre pile avec ses propres arguments/variables.}.
Une méthode qui stocke ces informations sur la pile rend cela plus facile---elle
peut contenir autant d'arguments de fonctions et/ou de valeurs, que la pile a d'espace.

\InSqBrackets{\TAOCPvolI{}, 189} mentionne un schéma encore plus étrange, particulièrement
pratique sur les IBM System/360.

\myindex{MS-DOS}
\myindex{x86!\Instructions!INT}

MS-DOS a une manière de passer tout les arguments de fonctions via des registres,
par exemple, c'est un morceau de code pour un ancien MS-DOS 16-bit qui affiche
``Hello, world!'':

\begin{lstlisting}[style=customasmx86]
mov  dx, msg      ; address of message
mov  ah, 9        ; 9 means "print string" function
int  21h          ; DOS "syscall"

mov  ah, 4ch      ; "terminate program" function
int  21h          ; DOS "syscall"

msg  db 'Hello, World!\$'
\end{lstlisting}

\myindex{fastcall}
C'est presque similaire à la méthode \myref{fastcall}.
Et c'est aussi très similaire aux appels systèmes sous Linux (\myref{linux_syscall}) et Windows.

\myindex{x86!\Flags!CF}
Si une fonction MS-DOS devait renvoyer une valeur booléenne (i.e., un simple bit,
souvent pour indiquer un état d'erreur), le flag \TT{CF} était souvent utilisé.

Par exemple:

\begin{lstlisting}[style=customasmx86]
mov ah, 3ch       ; create file
lea dx, filename
mov cl, 1
int 21h
jc  error
mov file_handle, ax
...
error:
...
\end{lstlisting}

En cas d'erreur, le flag \TT{CF} est mis. Sinon, le handle du fichier nouvellement
créer est retourné via \TT{AX}.

Cette méthode est encore utilisée par les programmeurs en langage d'assemblage.
Dans le code source de Windows Research Kernel (qui est très similaire à Windows
2003) nous pouvons trouver quelque chose comme ça (file \IT{base/ntos/ke/i386/cpu.asm}):

\begin{lstlisting}[style=customasmx86]
        public  Get386Stepping
Get386Stepping  proc

        call    MultiplyTest            ; Perform multiplication test
        jnc     short G3s00             ; if nc, muttest is ok
        mov     ax, 0
        ret
G3s00:
        call    Check386B0              ; Check for B0 stepping
        jnc     short G3s05             ; if nc, it's B1/later
        mov     ax, 100h                ; It is B0/earlier stepping
        ret

G3s05:
        call    Check386D1              ; Check for D1 stepping
        jc      short G3s10             ; if c, it is NOT D1
        mov     ax, 301h                ; It is D1/later stepping
        ret

G3s10:
        mov     ax, 101h                ; assume it is B1 stepping
        ret

	...

MultiplyTest    proc

        xor     cx,cx                   ; 64K times is a nice round number
mlt00:  push    cx
        call    Multiply                ; does this chip's multiply work?
        pop     cx
        jc      short mltx              ; if c, No, exit
        loop    mlt00                   ; if nc, YEs, loop to try again
        clc
mltx:
        ret

MultiplyTest    endp
\end{lstlisting}


\subsubsection{Stockage des variables locales}

Une fonction peut allouer de l'espace sur la pile pour ses variables locales
simplement en décrémentant le \glslink{stack pointer}{pointeur de pile} vers le
bas de la pile.

% I think here, "stack bottom" means the lowest address in the stack space,
% but the reader might also think it means towards the top of the stack space,
% like in a pop, so you might change "towards the stack bottom" to
% "towards the lowest address of the stack", or just take it out,
% since "decreasing" also suggests that.

Donc, c'est très rapide, peu importe combien de variables locales sont définies.
Ce n'est pas une nécessité de stocker les variables locales sur la pile.
Vous pouvez les stocker où bon vous semble,
mais c'est traditionnellement fait comme cela.

\EN{\input{patterns/02_stack/04_alloca/main_EN}}
\FR{\input{patterns/02_stack/04_alloca/main_FR}}
\RU{\input{patterns/02_stack/04_alloca/main_RU}}
\PTBR{\input{patterns/02_stack/04_alloca/main_PTBR}}
\ITA{\input{patterns/02_stack/04_alloca/main_ITA}}
\DE{\input{patterns/02_stack/04_alloca/main_DE}}

\subsection{(Windows) SEH}
\index{Windows!Structured Exception Handling}

\RU{В стеке хранятся записи \ac{SEH} для функции (если они присутствуют)}%
\EN{\ac{SEH} records are also stored on the stack (if they are present).}.

\ifx\LITE\undefined
\RU{Читайте больше о нем здесь}\EN{Read more about it}: (\myref{sec:SEH}).
\fi

\subsection{\RU{Защита от переполнений буфера}\EN{Buffer overflow protection}\PTBR{Proteção contra estouro de buffer}}

\RU{Здесь больше об этом}\EN{More about it here}\PTBR{Mais sobre aqui}~(\myref{subsec:bufferoverflow}).



\subsubsection{Désallocation automatique de données dans la pile}

Peut-être que la raison pour laquelle les variables locales et les enregistrements SEH sont stockés dans la
pile est qu'ils sont automatiquement libérés quand la fonction se termine en utilisant simplement une
instruction pour corriger la position du pointeur de pile (souvent \ADD).
Les arguments de fonction sont aussi désalloués automatiquement à la fin de la fonction.
À l'inverse, toutes les données allouées sur le \IT{heap} doivent être désallouées de façon explicite.

% sections
\subsection{ Disposition typique de la pile }

Avant l'exécution de la première instruction d'une fonction, la pile ressemble généralement à ceci:

\input{patterns/02_stack/stack_layout}
 % TBT
\section{\RU{Мусор в стеке}\EN{Noise in stack}}

\RU{Часто в этой книге говорится о \q{шуме} или \q{мусоре} в стеке или памяти.}
\EN{Often in this book \q{noise} or \q{garbage} values in the stack or memory are mentioned.}
\RU{Откуда он берется}\EN{Where do they come from}?
\RU{Это то, что осталось там после исполнения предыдущих функций.}
\EN{These are what was left in there after other functions' executions.}
\RU{Короткий пример}\EN{Short example}:

\lstinputlisting{patterns/02_stack/08_noise/st.c}

\RU{Компилируем}\EN{Compiling}\dots

\lstinputlisting[caption=\NonOptimizing MSVC 2010]{patterns/02_stack/08_noise/st.asm}

\RU{Компилятор поворчит немного}\EN{The compiler will grumble a little bit}\dots

\begin{lstlisting}
c:\Polygon\c>cl st.c /Fast.asm /MD
Microsoft (R) 32-bit C/C++ Optimizing Compiler Version 16.00.40219.01 for 80x86
Copyright (C) Microsoft Corporation.  All rights reserved.

st.c
c:\polygon\c\st.c(11) : warning C4700: uninitialized local variable 'c' used
c:\polygon\c\st.c(11) : warning C4700: uninitialized local variable 'b' used
c:\polygon\c\st.c(11) : warning C4700: uninitialized local variable 'a' used
Microsoft (R) Incremental Linker Version 10.00.40219.01
Copyright (C) Microsoft Corporation.  All rights reserved.

/out:st.exe
st.obj
\end{lstlisting}

\RU{Но когда мы запускаем}\EN{But when we run the compiled program}\dots

\begin{lstlisting}
c:\Polygon\c>st
1, 2, 3
\end{lstlisting}

\RU{Ох. Вот это странно. Мы ведь не устанавливали значения никаких переменных в}\EN{Oh, 
what a weird thing! We did not set any variables in} \TT{f2()}. 
\RU{Эти значения --- это \q{привидения}, которые всё ещё в стеке.}
\EN{These are \q{ghosts} values, which are still in the stack.}

\clearpage
\RU{Загрузим пример в}\EN{Let's load the example into} \olly:

\begin{figure}[H]
\centering
\includegraphics[scale=\FigScale]{patterns/02_stack/08_noise/olly1.png}
\caption{\olly: \TT{f1()}}
\label{fig:stack_noise_olly1}
\end{figure}

\RU{Когда}\EN{When} \TT{f1()} \RU{заполняет переменные}\EN{assigns the variables} $a$, $b$ \AndENRU $c$ 
\RU{они сохраняются по адресу}\EN{, their values are stored at the address} \TT{0x1FF860} 
\RU{\etc{}.}\EN{and so on.}

\clearpage
\RU{А когда исполняется}\EN{And when} \TT{f2()}\EN{ executes}:

\begin{figure}[H]
\centering
\includegraphics[scale=\FigScale]{patterns/02_stack/08_noise/olly2.png}
\caption{\olly: \TT{f2()}}
\label{fig:stack_noise_olly2}
\end{figure}

... $a$, $b$ \AndENRU $c$ \RU{в функции}\EN{of} \TT{f2()} \RU{находятся по тем же адресам!}
\EN{are located at the same addresses!}
\RU{Пока никто не перезаписал их, так что они здесь в нетронутом виде.}
\EN{No one has overwritten the values yet, so at that point they are still untouched.}

\RU{Для создания такой странной ситуации несколько функций должны исполняться друг за другом
и \ac{SP} должен быть одинаковым при входе в функции, т.е. у функций должно быть равное количество
аргументов). Тогда локальные переменные будут расположены в том же месте стека.}
\EN{So, for this weird situation to occur, several functions have to be called one after another and
\ac{SP} has to be the same at each function entry (i.e., they have the same number
of arguments). Then the local variables will be located at the same positions in the stack.}

\RU{Подводя итоги, все значения в стеке (да и памяти вообще) это значения оставшиеся от 
исполнения предыдущих функций.}
\EN{Summarizing, all values in the stack (and memory cells in general) 
have values left there from previous function executions.}
\RU{Строго говоря, они не случайны, они скорее непредсказуемы.}
\EN{They are not random in the strict sense, but rather have unpredictable values.}

\RU{А как иначе}\EN{Is there another option}?
\RU{Можно было бы очищать части стека перед исполнением каждой функции,
но это слишком много лишней (и ненужной) работы.}
\EN{It probably would be possible to clear portions of the stack before each function execution,
but that's too much extra (and unnecessary) work.}

\subsection{MSVC 2013}

\EN{The example was compiled by}\RU{Этот пример был скомпилирован в} MSVC 2010.
\EN{But the reader of this book made attempt to compile this example in MSVC 2013, ran it, and got all 3 numbers reversed:}%
\RU{Но один читатель этой книги сделал попытку скомпилировать пример в MSVC 2013, запустил и увидел 3 числа в обратном порядке:}

\begin{lstlisting}
c:\Polygon\c>st
3, 2, 1
\end{lstlisting}

\EN{Why?}\RU{Почему?}

\EN{I also compiled this example in MSVC 2013 and saw this:}%
\RU{Я также попробовал скомпилировать этот пример в MSVC 2013 и увидел это:}

\begin{lstlisting}[caption=MSVC 2013]
_a$ = -12						; size = 4
_b$ = -8						; size = 4
_c$ = -4						; size = 4
_f2	PROC

...

_f2	ENDP

_c$ = -12						; size = 4
_b$ = -8						; size = 4
_a$ = -4						; size = 4
_f1	PROC

...

_f1	ENDP
\end{lstlisting}

\EN{Unlike MSVC 2010, MSVC 2013 allocated a/b/c variables in function \TT{f2()} in reverse order.}%
\RU{В отличии от MSVC 2010, MSVC 2013 разместил переменные a/b/c в функции \TT{f2()} в обратном порядке.}
\EN{And this is completely correct, because \CCpp standards has no rule, in which order local variables must be allocated in the local stack, if at all.}%
\RU{И это полностью корректно, потому что в стандартах \CCpp нет правила, в каком порядке локальные переменные должны быть размещены в локальном стеке,
если вообще.}
\EN{The reason of difference is because MSVC 2010 has one way to do it, and MSVC 2013 has probably something changed inside of compiler guts, so it behaves
slightly different.}%
\RU{Разница есть из-за того что MSVC 2010 делает это одним способом, а в MSVC 2013, вероятно, что-то немного изменили во внутренностях компилятора,
так что он ведет себя слегка иначе.}


\section{\Exercises}

\subsection{\Exercise \#1}
\label{exercise_stack_1}

\RU{Если это скомпилировать в MSVC и запустить, появится три числа. Откуда они берутся? 
Откуда они берутся если скомпилировать в MSVC с оптимизациями (\Ox)?}
\EN{If to compile this piece of code in MSVC and run, a three number will be printed. 
Where they are came from?
Where they are came from if to compile it in MSVC with optimization (\Ox)?}
\RU{Почему в GCC ситуация совсем иная}\EN{Why the situation is completely different in GCC}?

\begin{lstlisting}
#include <stdio.h>

int main()
{
	printf ("%d, %d, %d\n");

	return 0;
};
\end{lstlisting}

\RU{Ответ}\EN{Answer}: \ref{exercise_solutions_stack_1}.

}
\RU{\section{\Stack}
\label{sec:stack}
\myindex{\Stack}

Стек в информатике~--- это одна из наиболее фундаментальных структур данных
\footnote{\href{http://go.yurichev.com/17119}{wikipedia.org/wiki/Call\_stack}}.
\ac{AKA} \ac{LIFO}.

Технически это просто блок памяти в памяти процесса + регистр \ESP в x86 или \RSP в x64, либо \ac{SP} в ARM, который указывает где-то в пределах этого блока.

\myindex{ARM!\Instructions!PUSH}
\myindex{ARM!\Instructions!POP}
\myindex{x86!\Instructions!PUSH}
\myindex{x86!\Instructions!POP}
Часто используемые инструкции для работы со стеком~--- это \PUSH и \POP (в x86 и Thumb-режиме ARM). 
\PUSH уменьшает \ESP/\RSP/\ac{SP} на 4 в 32-битном режиме (или на 8 в 64-битном),
затем записывает по адресу, на который указывает \ESP/\RSP/\ac{SP}, содержимое своего единственного операнда.

\POP это обратная операция~--- сначала достает из \glslink{stack pointer}{указателя стека} значение и помещает его в операнд 
(который очень часто является регистром) и затем увеличивает указатель стека на 4 (или 8).

В самом начале \glslink{stack pointer}{регистр-указатель} указывает на конец стека.
Конец стека находится в начале блока памяти, выделенного под стек. Это странно, но это так.
\PUSH уменьшает \glslink{stack pointer}{регистр-указатель}, а \POP~--- увеличивает.

В процессоре ARM, тем не менее, есть поддержка стеков, растущих как в сторону уменьшения, так и в сторону увеличения.

\myindex{ARM!\Instructions!STMFD}
\myindex{ARM!\Instructions!LDMFD}
\myindex{ARM!\Instructions!STMED}
\myindex{ARM!\Instructions!LDMED}
\myindex{ARM!\Instructions!STMFA}
\myindex{ARM!\Instructions!LDMFA}
\myindex{ARM!\Instructions!STMEA}
\myindex{ARM!\Instructions!LDMEA}

Например, инструкции \ac{STMFD}/\ac{LDMFD}, \ac{STMED}/\ac{LDMED} предназначены для descending-стека (растет назад, начиная с высоких адресов в сторону низких).\\
Инструкции \ac{STMFA}/\ac{LDMFA}, \ac{STMEA}/\ac{LDMEA} предназначены для ascending-стека (растет вперед, начиная с низких адресов в сторону высоких).

% It might be worth mentioning that STMED and STMEA write first,
% and then move the pointer,
% and that LDMED and LDMEA move the pointer first, and then read.
% In other words, ARM not only lets the stack grow in a non-standard direction,
% but also in a non-standard order.
% Maybe this can be in the glossary, which would explain why E stands for "empty".

\subsection{Почему стек растет в обратную сторону?}
\label{stack_grow_backwards}

Интуитивно мы можем подумать, что, как и любая другая структура данных, стек мог бы расти вперед, т.е. в сторону увеличения адресов.

Причина, почему стек растет назад, видимо, историческая.
Когда компьютеры были большие и занимали целую комнату, было очень легко разделить сегмент на две части: для \glslink{heap}{кучи} и для стека.
Заранее было неизвестно, насколько большой может быть \glslink{heap}{куча} или стек, так что это решение было самым простым.

\begin{center}
	\begin{tikzpicture}
	\tikzstyle{every path}=[thick]

	\node [rectangle,draw,minimum width=6cm, minimum height=2cm] (memory) {};
	\node [] [right=0.2cm of memory.west] (heap) {\MLHeap};
	\node [] [left=0.2cm of memory.east] (stack) {\MLStack};

	\node [] (center1) [right=2cm of memory.west] {};
	\node [] (center2) [left=2cm of memory.east] {};

	\draw [->] (heap) -- (center1);
	\draw [->] (stack) -- (center2);

	\node [] [above left=1.1cm and 0.2cm of heap] (t1) {\MLStartOfHeap};
	\node [] [above right=1.1cm and 0.2cm of stack] (t2) {\MLStartOfStack};

	\draw [->] (t1) -- (memory.west);
	\draw [->] (t2) -- (memory.east);

	\end{tikzpicture}
\end{center}


В \RitchieThompsonUNIX можно прочитать:

\begin{framed}
\begin{quotation}
The user-core part of an image is divided into three logical segments. The program text segment begins at location 0 in the virtual address space. During execution, this segment is write-protected and a single copy of it is shared among all processes executing the same program. At the first 8K byte boundary above the program text segment in the virtual address space begins a nonshared, writable data segment, the size of which may be extended by a system call. Starting at the highest address in the virtual address space is a stack segment, which automatically grows downward as the hardware's stack pointer fluctuates.
\end{quotation}
\end{framed}

Это немного напоминает как некоторые студенты
пишут два конспекта в одной тетрадке:
первый конспект начинается обычным образом, второй пишется с конца, перевернув тетрадку.
Конспекты могут встретиться где-то посредине, в случае недостатка свободного места.

% I think if we want to expand on this analogy,
% one might remember that the line number increases as as you go down a page.
% So when you decrease the address when pushing to the stack, visually,
% the stack does grow upwards.
% Of course, the problem is that in most human languages,
% just as with computers,
% we write downwards, so this direction is what makes buffer overflows so messy.

\subsection{Для чего используется стек?}

% subsections
\EN{\input{patterns/02_stack/01_saving_ret_addr_EN}}
\RU{\input{patterns/02_stack/01_saving_ret_addr_RU}}
\DE{\input{patterns/02_stack/01_saving_ret_addr_DE}}
\FR{\input{patterns/02_stack/01_saving_ret_addr_FR}}
\PTBR{\input{patterns/02_stack/01_saving_ret_addr_PTBR}}
\ITA{\input{patterns/02_stack/01_saving_ret_addr_ITA}}

\subsection{\RU{Передача параметров функции}\EN{Passing function arguments}}

\RU{Самый распространенный способ передачи параметров в x86 называется}
\EN{The most popular way to pass parameters in x86 is called} \q{cdecl}:

\begin{lstlisting}
push arg3
push arg2
push arg1
call f
add esp, 12 ; 4*3=12
\end{lstlisting}

\RU{Вызываемая функция получает свои параметры также через указатель стека.}
\EN{\Gls{callee} functions get their arguments via the stack pointer.}

\RU{Следовательно, так расположены значения в стеке перед исполнением самой первой инструкции
функции \ttf{}:}
\EN{Therefore, this is how the argument values are located in the stack before the execution
of the \ttf{} function's very first instruction:}

\begin{center}
\begin{tabular}{ | l | l | }
\hline
ESP & \RU{адрес возврата}\EN{return address} \\
\hline
ESP+4 & \argument \#1, \MarkedInIDAAs{} \TT{arg\_0} \\
\hline
ESP+8 & \argument \#2, \MarkedInIDAAs{} \TT{arg\_4} \\
\hline
ESP+0xC & \argument \#3, \MarkedInIDAAs{} \TT{arg\_8} \\
\hline
\dots & \dots \\
\hline
\end{tabular}
\end{center}

\ifx\LITE\undefined
\RU{См. также в соответствующем разделе о других способах передачи аргументов через стек}
\EN{For more information on other calling conventions see also section}~(\myref{sec:callingconventions}).
\fi
\RU{Важно отметить, что, в общем, никто не заставляет программистов передавать параметры именно через стек,
это не является требованием к исполняемому коду.}
\EN{It is worth noting that nothing obliges programmers to pass arguments through the stack. It is not a requirement.}
\RU{Вы можете делать это совершенно иначе, не используя стек вообще.}
\EN{One could implement any other method without using the stack at all.}

\RU{К примеру, можно выделять в \glslink{heap}{куче} место для аргументов, 
заполнять их и передавать в функцию указатель на это место через \EAX. И это вполне будет работать}%
\EN{For example, it is possible to allocate a space for arguments in the \gls{heap}, fill it and pass it to a function 
via a pointer to this block in the \EAX register. This will work}%
\footnote{\RU{Например, в книге Дональда Кнута \q{Искусство программирования}, в разделе 1.4.1 
посвященном подпрограммам \cite[раздел 1.4.1]{Knuth:1998:ACP:521463}, 
мы можем прочитать о возможности располагать параметры для вызываемой подпрограммы после инструкции \JMP,
передающей управление подпрограмме. Кнут описывает, что это было особенно удобно для компьютеров IBM System/360.}%
\EN{For example, in the \q{The Art of Computer Programming} book by Donald Knuth, 
in section 1.4.1 dedicated to subroutines \cite[section 1.4.1]{Knuth:1998:ACP:521463},
we could read that one way to supply arguments to a subroutine is simply to list them after the \JMP instruction
passing control to subroutine. Knuth explains that this method was particularly convenient on IBM System/360.}}.
\RU{Однако традиционно сложилось, что в x86 и ARM передача аргументов происходит именно через стек.}
% I am unsure about what this comment means.
% My guess is that the arguments are put in the memory position after
% the jump instruction, so you could say:
% "one way to supply arguments to a subroutine is simply to list them in memory
% after the \JMP instruction that passes control to the subroutine."
% Right now, "after" also sounds like it refers to the time after
% the jump happens, which I think is too late.
\EN{However, it is a convenient custom in x86 and ARM to use the stack for this purpose.} \\
\\
\RU{Кстати, вызываемая функция не имеет информации о количестве переданных ей аргументов.}
\EN{By the way, the \gls{callee} function does not have any information about how many arguments were passed.}
\RU{Функции Си с переменным количеством аргументов (как \printf) определяют их количество по 
спецификаторам строки формата (начинающиеся со знака \%).}
\EN{C functions with a variable number of arguments (like \printf) determine their number using format string  specifiers (which begin with the \% symbol).}
\RU{Если написать что-то вроде}\EN{If we write something like} 

\begin{lstlisting}
printf("%d %d %d", 1234);
\end{lstlisting}

\printf \RU{выведет 1234, затем ещё два случайных числа, которые волею случая оказались в стеке рядом.}
\EN{will print 1234, and then two random numbers, which were lying next to it in the stack.}\\
\\
\RU{Вот почему не так уж и важно, как объявлять функцию \main}
\EN{That's why it is not very important how we declare the \main function}: \RU{как}\EN{as} \main, 
\TT{main(int argc, char *argv[])} 
\RU{либо}\EN{or} \TT{main(int argc, char *argv[], char *envp[])}.

\RU{В реальности, \ac{CRT}-код вызывает \main примерно так:}
\EN{In fact, the \ac{CRT}-code is calling \main roughly as:}

\begin{lstlisting}
push envp
push argv
push argc
call main
...
\end{lstlisting}

\RU{Если вы объявляете \main без аргументов, они, тем не менее, присутствуют в стеке, но не используются.}
\EN{If you declare \main as \main without arguments, they are, nevertheless, still present in the stack, but
are not used.}
\RU{Если вы объявите \main как}\EN{If you declare \main as} \TT{main(int argc, char *argv[])}, 
\RU{вы можете использовать два первых аргумента, а третий останется для вашей функции \q{невидимым}.}
\EN{you will be able to use first two arguments, and the third will remain \q{invisible} for your function.}
\RU{Более того, можно даже объявить}\EN{Even more, it is possible to declare} \TT{main(int argc)}, 
\RU{и это будет работать}\EN{and it will work}.


\EN{\subsubsection{Local variable storage}

A function could allocate space in the stack for its local variables just by decreasing 
the \gls{stack pointer} towards the stack bottom.

% I think here, "stack bottom" means the lowest address in the stack space,
% but the reader might also think it means towards the top of the stack space,
% like in a pop, so you might change "towards the stack bottom" to
% "towards the lowest address of the stack", or just take it out,
% since "decreasing" also suggests that.

Hence, it's very fast, no matter how many local variables are defined.
It is also not a requirement to store local variables in the stack.
You could store local variables wherever you like, 
but traditionally this is how it's done.

}
\RU{\subsubsection{Хранение локальных переменных}

Функция может выделить для себя некоторое место в стеке для локальных переменных, просто отодвинув 
\glslink{stack pointer}{указатель стека} глубже к концу стека.

% I think here, "stack bottom" means the lowest address in the stack space,
% but the reader might also think it means towards the top of the stack space,
% like in a pop, so you might change "towards the stack bottom" to
% "towards the lowest address of the stack", or just take it out,
% since "decreasing" also suggests that.

Это очень быстро вне зависимости от количества локальных переменных.
Хранить локальные переменные в стеке не является необходимым требованием. 
Вы можете хранить локальные переменные где угодно. 
Но по традиции всё сложилось так.

}
\PTBR{\subsubsection{Armazenamento de variáveis locais}

Uma função poderia alocar espaço na pilha para suas variáveis locais simplesmente decrementando o ponteiro da pilha.

% I think here, "stack bottom" means the lowest address in the stack space,
% but the reader might also think it means towards the top of the stack space,
% like in a pop, so you might change "towards the stack bottom" to
% "towards the lowest address of the stack", or just take it out,
% since "decreasing" also suggests that.

Consequentemente, é muito rápido, não importando quantas variáveis locais serão definidas.
Também não é um requisito armazenar variáveis locais na pilha.
Você pode armazenar variáveis locais onde você quiser, mas, tradicionalmente, é assim que é feito.

}
\EN{\input{patterns/02_stack/04_alloca/main_EN}}
\FR{\input{patterns/02_stack/04_alloca/main_FR}}
\RU{\input{patterns/02_stack/04_alloca/main_RU}}
\PTBR{\input{patterns/02_stack/04_alloca/main_PTBR}}
\ITA{\input{patterns/02_stack/04_alloca/main_ITA}}
\DE{\input{patterns/02_stack/04_alloca/main_DE}}

\subsection{(Windows) SEH}
\index{Windows!Structured Exception Handling}

\RU{В стеке хранятся записи \ac{SEH} для функции (если они присутствуют)}%
\EN{\ac{SEH} records are also stored on the stack (if they are present).}.

\ifx\LITE\undefined
\RU{Читайте больше о нем здесь}\EN{Read more about it}: (\myref{sec:SEH}).
\fi

\subsection{\RU{Защита от переполнений буфера}\EN{Buffer overflow protection}\PTBR{Proteção contra estouro de buffer}}

\RU{Здесь больше об этом}\EN{More about it here}\PTBR{Mais sobre aqui}~(\myref{subsec:bufferoverflow}).



\subsubsection{Автоматическое освобождение данных в стеке}

Возможно, причина хранения локальных переменных и SEH-записей в стеке в том, что после выхода из функции, всё эти данные освобождаются автоматически,
используя только одну инструкцию корректирования указателя стека (часто это \ADD).
Аргументы функций, можно сказать, тоже освобождаются автоматически в конце функции.
А всё что хранится в куче (\IT{heap}) нужно освобождать явно.

% sections
\EN{\input{patterns/02_stack/07_layout_EN}}
\RU{\subsection{Разметка типичного стека}

Разметка типичного стека в 32-битной среде
перед исполнением самой первой инструкции функции выглядит так:

\input{patterns/02_stack/stack_layout}

% I think this only applies to RISC architectures
% that don't have a POP instruction that only lets you read one value
% (ie. ARM and MIPS).
% In x86, the return address is saved before entering the function,
% and the function does not have the chance to save the frame pointer.
% Also, you should mention that this is how the stack looks like
% right after the function prologue,
% which some readers might think is the first instruction,
% but is needed to save the frame pointer.

}
\PTBR{\subsection{Um modelo típico de pilha}

Um modelo típico de pilha em um ambiente 32-bits no início de uma função,
antes da execução da primeira instrução, se parece com isso:

\input{patterns/02_stack/stack_layout}

% I think this only applies to RISC architectures
% that don't have a POP instruction that only lets you read one value
% (ie. ARM and MIPS).
% In x86, the return address is saved before entering the function,
% and the function does not have the chance to save the frame pointer.
% Also, you should mention that this is how the stack looks like
% right after the function prologue,
% which some readers might think is the first instruction,
% but is needed to save the frame pointer.
}
\section{\RU{Мусор в стеке}\EN{Noise in stack}}

\RU{Часто в этой книге говорится о \q{шуме} или \q{мусоре} в стеке или памяти.}
\EN{Often in this book \q{noise} or \q{garbage} values in the stack or memory are mentioned.}
\RU{Откуда он берется}\EN{Where do they come from}?
\RU{Это то, что осталось там после исполнения предыдущих функций.}
\EN{These are what was left in there after other functions' executions.}
\RU{Короткий пример}\EN{Short example}:

\lstinputlisting{patterns/02_stack/08_noise/st.c}

\RU{Компилируем}\EN{Compiling}\dots

\lstinputlisting[caption=\NonOptimizing MSVC 2010]{patterns/02_stack/08_noise/st.asm}

\RU{Компилятор поворчит немного}\EN{The compiler will grumble a little bit}\dots

\begin{lstlisting}
c:\Polygon\c>cl st.c /Fast.asm /MD
Microsoft (R) 32-bit C/C++ Optimizing Compiler Version 16.00.40219.01 for 80x86
Copyright (C) Microsoft Corporation.  All rights reserved.

st.c
c:\polygon\c\st.c(11) : warning C4700: uninitialized local variable 'c' used
c:\polygon\c\st.c(11) : warning C4700: uninitialized local variable 'b' used
c:\polygon\c\st.c(11) : warning C4700: uninitialized local variable 'a' used
Microsoft (R) Incremental Linker Version 10.00.40219.01
Copyright (C) Microsoft Corporation.  All rights reserved.

/out:st.exe
st.obj
\end{lstlisting}

\RU{Но когда мы запускаем}\EN{But when we run the compiled program}\dots

\begin{lstlisting}
c:\Polygon\c>st
1, 2, 3
\end{lstlisting}

\RU{Ох. Вот это странно. Мы ведь не устанавливали значения никаких переменных в}\EN{Oh, 
what a weird thing! We did not set any variables in} \TT{f2()}. 
\RU{Эти значения --- это \q{привидения}, которые всё ещё в стеке.}
\EN{These are \q{ghosts} values, which are still in the stack.}

\clearpage
\RU{Загрузим пример в}\EN{Let's load the example into} \olly:

\begin{figure}[H]
\centering
\includegraphics[scale=\FigScale]{patterns/02_stack/08_noise/olly1.png}
\caption{\olly: \TT{f1()}}
\label{fig:stack_noise_olly1}
\end{figure}

\RU{Когда}\EN{When} \TT{f1()} \RU{заполняет переменные}\EN{assigns the variables} $a$, $b$ \AndENRU $c$ 
\RU{они сохраняются по адресу}\EN{, their values are stored at the address} \TT{0x1FF860} 
\RU{\etc{}.}\EN{and so on.}

\clearpage
\RU{А когда исполняется}\EN{And when} \TT{f2()}\EN{ executes}:

\begin{figure}[H]
\centering
\includegraphics[scale=\FigScale]{patterns/02_stack/08_noise/olly2.png}
\caption{\olly: \TT{f2()}}
\label{fig:stack_noise_olly2}
\end{figure}

... $a$, $b$ \AndENRU $c$ \RU{в функции}\EN{of} \TT{f2()} \RU{находятся по тем же адресам!}
\EN{are located at the same addresses!}
\RU{Пока никто не перезаписал их, так что они здесь в нетронутом виде.}
\EN{No one has overwritten the values yet, so at that point they are still untouched.}

\RU{Для создания такой странной ситуации несколько функций должны исполняться друг за другом
и \ac{SP} должен быть одинаковым при входе в функции, т.е. у функций должно быть равное количество
аргументов). Тогда локальные переменные будут расположены в том же месте стека.}
\EN{So, for this weird situation to occur, several functions have to be called one after another and
\ac{SP} has to be the same at each function entry (i.e., they have the same number
of arguments). Then the local variables will be located at the same positions in the stack.}

\RU{Подводя итоги, все значения в стеке (да и памяти вообще) это значения оставшиеся от 
исполнения предыдущих функций.}
\EN{Summarizing, all values in the stack (and memory cells in general) 
have values left there from previous function executions.}
\RU{Строго говоря, они не случайны, они скорее непредсказуемы.}
\EN{They are not random in the strict sense, but rather have unpredictable values.}

\RU{А как иначе}\EN{Is there another option}?
\RU{Можно было бы очищать части стека перед исполнением каждой функции,
но это слишком много лишней (и ненужной) работы.}
\EN{It probably would be possible to clear portions of the stack before each function execution,
but that's too much extra (and unnecessary) work.}

\subsection{MSVC 2013}

\EN{The example was compiled by}\RU{Этот пример был скомпилирован в} MSVC 2010.
\EN{But the reader of this book made attempt to compile this example in MSVC 2013, ran it, and got all 3 numbers reversed:}%
\RU{Но один читатель этой книги сделал попытку скомпилировать пример в MSVC 2013, запустил и увидел 3 числа в обратном порядке:}

\begin{lstlisting}
c:\Polygon\c>st
3, 2, 1
\end{lstlisting}

\EN{Why?}\RU{Почему?}

\EN{I also compiled this example in MSVC 2013 and saw this:}%
\RU{Я также попробовал скомпилировать этот пример в MSVC 2013 и увидел это:}

\begin{lstlisting}[caption=MSVC 2013]
_a$ = -12						; size = 4
_b$ = -8						; size = 4
_c$ = -4						; size = 4
_f2	PROC

...

_f2	ENDP

_c$ = -12						; size = 4
_b$ = -8						; size = 4
_a$ = -4						; size = 4
_f1	PROC

...

_f1	ENDP
\end{lstlisting}

\EN{Unlike MSVC 2010, MSVC 2013 allocated a/b/c variables in function \TT{f2()} in reverse order.}%
\RU{В отличии от MSVC 2010, MSVC 2013 разместил переменные a/b/c в функции \TT{f2()} в обратном порядке.}
\EN{And this is completely correct, because \CCpp standards has no rule, in which order local variables must be allocated in the local stack, if at all.}%
\RU{И это полностью корректно, потому что в стандартах \CCpp нет правила, в каком порядке локальные переменные должны быть размещены в локальном стеке,
если вообще.}
\EN{The reason of difference is because MSVC 2010 has one way to do it, and MSVC 2013 has probably something changed inside of compiler guts, so it behaves
slightly different.}%
\RU{Разница есть из-за того что MSVC 2010 делает это одним способом, а в MSVC 2013, вероятно, что-то немного изменили во внутренностях компилятора,
так что он ведет себя слегка иначе.}


\section{\Exercises}

\subsection{\Exercise \#1}
\label{exercise_stack_1}

\RU{Если это скомпилировать в MSVC и запустить, появится три числа. Откуда они берутся? 
Откуда они берутся если скомпилировать в MSVC с оптимизациями (\Ox)?}
\EN{If to compile this piece of code in MSVC and run, a three number will be printed. 
Where they are came from?
Where they are came from if to compile it in MSVC with optimization (\Ox)?}
\RU{Почему в GCC ситуация совсем иная}\EN{Why the situation is completely different in GCC}?

\begin{lstlisting}
#include <stdio.h>

int main()
{
	printf ("%d, %d, %d\n");

	return 0;
};
\end{lstlisting}

\RU{Ответ}\EN{Answer}: \ref{exercise_solutions_stack_1}.


}
\PTBR{\mysection{\Stack}
\label{sec:stack}
\myindex{\Stack}

A pilha é uma das estruturas mais fundamentais na ciência da computação.
\footnote{\href{http://go.yurichev.com/17119}{wikipedia.org/wiki/Call\_stack}}.
\ac{AKA} \ac{LIFO}.

Tecnicamente, é só um bloco de memória junto com os registradores \ESP ou \RSP em x86 e x64, ou o \ac{SP} no ARM, como um ponteiro para aquele bloco.

\myindex{ARM!\Instructions!PUSH}
\myindex{ARM!\Instructions!POP}
\myindex{x86!\Instructions!PUSH}
\myindex{x86!\Instructions!POP}
As instruções mais frequente para o acesso da pilha são \PUSH e \POP (em ambos x86 e x64).
\PUSH subtrai de \ESP/\RSP/\ac{SP} 4 no modo 32-bits (ou 8 no modo 64-bits) e então escreve o conteúdo desse operando único para o endereço de memória apontado por \ESP/\RSP/\ac{SP}.

\POP é a operação reversa: recupera a informação da localização de memória que é apontada por \ac{SP}, 
carrega a mesma no operando da instrução (geralmente um registrador) e então adiciona 4 (ou 8) para o ponteiro da pilha.

Depois da alocação da pilha, o ponteiro aponta para o fundo da pilha.
\PUSH decrementa o ponteiro da pilha e \POP incrementa. O fundo da pilha está na verdade no começo do bloco de memória alocado para ela.
Pode parecer estranho, mas é a maneira como é feita.

ARM: \PTBRph{}

\subsection{Por que a pilha ``cresce'' para trás?}
\label{stack_grow_backwards}

Intuitivamente, nós podemos pensar que a pilha cresce para frente, em direção a endereços mais altos, como qualquer outra estrutura de informação.

O motivo da pilha crescer para trás é provavelmente histórico. Quando os computadores era grandes e ocupavam um cômodo todo, era mais fácil dividir a memória em duas partes, uma para a ‘heap’ e outra para a pilha.
Logicamente, era desconhecido o quão grande a heap e a pilha seriam durante a execução do programa, então essa solução era a mais simples possível.

\begin{center}
	\begin{tikzpicture}
	\tikzstyle{every path}=[thick]

	\node [rectangle,draw,minimum width=6cm, minimum height=2cm] (memory) {};
	\node [] [right=0.2cm of memory.west] (heap) {\MLHeap};
	\node [] [left=0.2cm of memory.east] (stack) {\MLStack};

	\node [] (center1) [right=2cm of memory.west] {};
	\node [] (center2) [left=2cm of memory.east] {};

	\draw [->] (heap) -- (center1);
	\draw [->] (stack) -- (center2);

	\node [] [above left=1.1cm and 0.2cm of heap] (t1) {\MLStartOfHeap};
	\node [] [above right=1.1cm and 0.2cm of stack] (t2) {\MLStartOfStack};

	\draw [->] (t1) -- (memory.west);
	\draw [->] (t2) -- (memory.east);

	\end{tikzpicture}
\end{center}


No \RitchieThompsonUNIX nós podemos ler:

\begin{framed}
\begin{quotation}
A parte relacionada ao usuário é dividida em três segmentos lógicos. O segmento de texto do programa começa na localização 0 no espaço virtual de endereçamento.
Durante a execução, esse segmento é protegido para não ser reescrito e uma única cópia dele é compartilhado entre
todos os processos executando o mesmo programa.
Começando no limite de 8Kbytes acima do segmento de texto do programa no espaço de endereçamento virtual começa um segmento de informação gravável,
não compartilhável e de um tamanho que pode ser extendido por uma chamada do sistema.
Começando no endereço mais alto no espaço de endereçamento virtual está a pilha, que automaticamente cresce para trás conforme o ponteiro da pilha do hardware se altera.
\end{quotation}
\end{framed}

Isso pode ser análogo a como um estudante escreve notas de duas matérias diferentes em um caderno só:
as notas para a primeira matéria são escritas como de costume e as notas para a segunda são escritas do final do caderno,
virando o mesmo. As anotações de uma matéria podem encontrar as da outra no meio, no caso de haver falta de espaço.

\subsection{Para que a pilha é usada?}

% subsections
\EN{\input{patterns/02_stack/01_saving_ret_addr_EN}}
\RU{\input{patterns/02_stack/01_saving_ret_addr_RU}}
\DE{\input{patterns/02_stack/01_saving_ret_addr_DE}}
\FR{\input{patterns/02_stack/01_saving_ret_addr_FR}}
\PTBR{\input{patterns/02_stack/01_saving_ret_addr_PTBR}}
\ITA{\input{patterns/02_stack/01_saving_ret_addr_ITA}}

\subsection{\RU{Передача параметров функции}\EN{Passing function arguments}}

\RU{Самый распространенный способ передачи параметров в x86 называется}
\EN{The most popular way to pass parameters in x86 is called} \q{cdecl}:

\begin{lstlisting}
push arg3
push arg2
push arg1
call f
add esp, 12 ; 4*3=12
\end{lstlisting}

\RU{Вызываемая функция получает свои параметры также через указатель стека.}
\EN{\Gls{callee} functions get their arguments via the stack pointer.}

\RU{Следовательно, так расположены значения в стеке перед исполнением самой первой инструкции
функции \ttf{}:}
\EN{Therefore, this is how the argument values are located in the stack before the execution
of the \ttf{} function's very first instruction:}

\begin{center}
\begin{tabular}{ | l | l | }
\hline
ESP & \RU{адрес возврата}\EN{return address} \\
\hline
ESP+4 & \argument \#1, \MarkedInIDAAs{} \TT{arg\_0} \\
\hline
ESP+8 & \argument \#2, \MarkedInIDAAs{} \TT{arg\_4} \\
\hline
ESP+0xC & \argument \#3, \MarkedInIDAAs{} \TT{arg\_8} \\
\hline
\dots & \dots \\
\hline
\end{tabular}
\end{center}

\ifx\LITE\undefined
\RU{См. также в соответствующем разделе о других способах передачи аргументов через стек}
\EN{For more information on other calling conventions see also section}~(\myref{sec:callingconventions}).
\fi
\RU{Важно отметить, что, в общем, никто не заставляет программистов передавать параметры именно через стек,
это не является требованием к исполняемому коду.}
\EN{It is worth noting that nothing obliges programmers to pass arguments through the stack. It is not a requirement.}
\RU{Вы можете делать это совершенно иначе, не используя стек вообще.}
\EN{One could implement any other method without using the stack at all.}

\RU{К примеру, можно выделять в \glslink{heap}{куче} место для аргументов, 
заполнять их и передавать в функцию указатель на это место через \EAX. И это вполне будет работать}%
\EN{For example, it is possible to allocate a space for arguments in the \gls{heap}, fill it and pass it to a function 
via a pointer to this block in the \EAX register. This will work}%
\footnote{\RU{Например, в книге Дональда Кнута \q{Искусство программирования}, в разделе 1.4.1 
посвященном подпрограммам \cite[раздел 1.4.1]{Knuth:1998:ACP:521463}, 
мы можем прочитать о возможности располагать параметры для вызываемой подпрограммы после инструкции \JMP,
передающей управление подпрограмме. Кнут описывает, что это было особенно удобно для компьютеров IBM System/360.}%
\EN{For example, in the \q{The Art of Computer Programming} book by Donald Knuth, 
in section 1.4.1 dedicated to subroutines \cite[section 1.4.1]{Knuth:1998:ACP:521463},
we could read that one way to supply arguments to a subroutine is simply to list them after the \JMP instruction
passing control to subroutine. Knuth explains that this method was particularly convenient on IBM System/360.}}.
\RU{Однако традиционно сложилось, что в x86 и ARM передача аргументов происходит именно через стек.}
% I am unsure about what this comment means.
% My guess is that the arguments are put in the memory position after
% the jump instruction, so you could say:
% "one way to supply arguments to a subroutine is simply to list them in memory
% after the \JMP instruction that passes control to the subroutine."
% Right now, "after" also sounds like it refers to the time after
% the jump happens, which I think is too late.
\EN{However, it is a convenient custom in x86 and ARM to use the stack for this purpose.} \\
\\
\RU{Кстати, вызываемая функция не имеет информации о количестве переданных ей аргументов.}
\EN{By the way, the \gls{callee} function does not have any information about how many arguments were passed.}
\RU{Функции Си с переменным количеством аргументов (как \printf) определяют их количество по 
спецификаторам строки формата (начинающиеся со знака \%).}
\EN{C functions with a variable number of arguments (like \printf) determine their number using format string  specifiers (which begin with the \% symbol).}
\RU{Если написать что-то вроде}\EN{If we write something like} 

\begin{lstlisting}
printf("%d %d %d", 1234);
\end{lstlisting}

\printf \RU{выведет 1234, затем ещё два случайных числа, которые волею случая оказались в стеке рядом.}
\EN{will print 1234, and then two random numbers, which were lying next to it in the stack.}\\
\\
\RU{Вот почему не так уж и важно, как объявлять функцию \main}
\EN{That's why it is not very important how we declare the \main function}: \RU{как}\EN{as} \main, 
\TT{main(int argc, char *argv[])} 
\RU{либо}\EN{or} \TT{main(int argc, char *argv[], char *envp[])}.

\RU{В реальности, \ac{CRT}-код вызывает \main примерно так:}
\EN{In fact, the \ac{CRT}-code is calling \main roughly as:}

\begin{lstlisting}
push envp
push argv
push argc
call main
...
\end{lstlisting}

\RU{Если вы объявляете \main без аргументов, они, тем не менее, присутствуют в стеке, но не используются.}
\EN{If you declare \main as \main without arguments, they are, nevertheless, still present in the stack, but
are not used.}
\RU{Если вы объявите \main как}\EN{If you declare \main as} \TT{main(int argc, char *argv[])}, 
\RU{вы можете использовать два первых аргумента, а третий останется для вашей функции \q{невидимым}.}
\EN{you will be able to use first two arguments, and the third will remain \q{invisible} for your function.}
\RU{Более того, можно даже объявить}\EN{Even more, it is possible to declare} \TT{main(int argc)}, 
\RU{и это будет работать}\EN{and it will work}.


\EN{\subsubsection{Local variable storage}

A function could allocate space in the stack for its local variables just by decreasing 
the \gls{stack pointer} towards the stack bottom.

% I think here, "stack bottom" means the lowest address in the stack space,
% but the reader might also think it means towards the top of the stack space,
% like in a pop, so you might change "towards the stack bottom" to
% "towards the lowest address of the stack", or just take it out,
% since "decreasing" also suggests that.

Hence, it's very fast, no matter how many local variables are defined.
It is also not a requirement to store local variables in the stack.
You could store local variables wherever you like, 
but traditionally this is how it's done.

}
\RU{\subsubsection{Хранение локальных переменных}

Функция может выделить для себя некоторое место в стеке для локальных переменных, просто отодвинув 
\glslink{stack pointer}{указатель стека} глубже к концу стека.

% I think here, "stack bottom" means the lowest address in the stack space,
% but the reader might also think it means towards the top of the stack space,
% like in a pop, so you might change "towards the stack bottom" to
% "towards the lowest address of the stack", or just take it out,
% since "decreasing" also suggests that.

Это очень быстро вне зависимости от количества локальных переменных.
Хранить локальные переменные в стеке не является необходимым требованием. 
Вы можете хранить локальные переменные где угодно. 
Но по традиции всё сложилось так.

}
\PTBR{\subsubsection{Armazenamento de variáveis locais}

Uma função poderia alocar espaço na pilha para suas variáveis locais simplesmente decrementando o ponteiro da pilha.

% I think here, "stack bottom" means the lowest address in the stack space,
% but the reader might also think it means towards the top of the stack space,
% like in a pop, so you might change "towards the stack bottom" to
% "towards the lowest address of the stack", or just take it out,
% since "decreasing" also suggests that.

Consequentemente, é muito rápido, não importando quantas variáveis locais serão definidas.
Também não é um requisito armazenar variáveis locais na pilha.
Você pode armazenar variáveis locais onde você quiser, mas, tradicionalmente, é assim que é feito.

}
\EN{\input{patterns/02_stack/04_alloca/main_EN}}
\FR{\input{patterns/02_stack/04_alloca/main_FR}}
\RU{\input{patterns/02_stack/04_alloca/main_RU}}
\PTBR{\input{patterns/02_stack/04_alloca/main_PTBR}}
\ITA{\input{patterns/02_stack/04_alloca/main_ITA}}
\DE{\input{patterns/02_stack/04_alloca/main_DE}}

\subsection{(Windows) SEH}
\index{Windows!Structured Exception Handling}

\RU{В стеке хранятся записи \ac{SEH} для функции (если они присутствуют)}%
\EN{\ac{SEH} records are also stored on the stack (if they are present).}.

\ifx\LITE\undefined
\RU{Читайте больше о нем здесь}\EN{Read more about it}: (\myref{sec:SEH}).
\fi

\subsection{\RU{Защита от переполнений буфера}\EN{Buffer overflow protection}\PTBR{Proteção contra estouro de buffer}}

\RU{Здесь больше об этом}\EN{More about it here}\PTBR{Mais sobre aqui}~(\myref{subsec:bufferoverflow}).



\subsubsection{\PTBRph{}}

Talvez, o motivo para armazenar variáveis locais e registros SEH na pilha é que eles são desvinculados automaticamente depois do fim da função,
usando somente uma instrução para corrigir o ponteiro da pilha (geralmente é \ADD). Argumentos de funções, como podemos dizer, são
também desalocados automaticamente com o fim da função.
Como contraste, tudo armazenado na memória heap tem de ser desalocado explicitamente.

% sections
\EN{\input{patterns/02_stack/07_layout_EN}}
\RU{\subsection{Разметка типичного стека}

Разметка типичного стека в 32-битной среде
перед исполнением самой первой инструкции функции выглядит так:

\input{patterns/02_stack/stack_layout}

% I think this only applies to RISC architectures
% that don't have a POP instruction that only lets you read one value
% (ie. ARM and MIPS).
% In x86, the return address is saved before entering the function,
% and the function does not have the chance to save the frame pointer.
% Also, you should mention that this is how the stack looks like
% right after the function prologue,
% which some readers might think is the first instruction,
% but is needed to save the frame pointer.

}
\PTBR{\subsection{Um modelo típico de pilha}

Um modelo típico de pilha em um ambiente 32-bits no início de uma função,
antes da execução da primeira instrução, se parece com isso:

\input{patterns/02_stack/stack_layout}

% I think this only applies to RISC architectures
% that don't have a POP instruction that only lets you read one value
% (ie. ARM and MIPS).
% In x86, the return address is saved before entering the function,
% and the function does not have the chance to save the frame pointer.
% Also, you should mention that this is how the stack looks like
% right after the function prologue,
% which some readers might think is the first instruction,
% but is needed to save the frame pointer.
}
\section{\RU{Мусор в стеке}\EN{Noise in stack}}

\RU{Часто в этой книге говорится о \q{шуме} или \q{мусоре} в стеке или памяти.}
\EN{Often in this book \q{noise} or \q{garbage} values in the stack or memory are mentioned.}
\RU{Откуда он берется}\EN{Where do they come from}?
\RU{Это то, что осталось там после исполнения предыдущих функций.}
\EN{These are what was left in there after other functions' executions.}
\RU{Короткий пример}\EN{Short example}:

\lstinputlisting{patterns/02_stack/08_noise/st.c}

\RU{Компилируем}\EN{Compiling}\dots

\lstinputlisting[caption=\NonOptimizing MSVC 2010]{patterns/02_stack/08_noise/st.asm}

\RU{Компилятор поворчит немного}\EN{The compiler will grumble a little bit}\dots

\begin{lstlisting}
c:\Polygon\c>cl st.c /Fast.asm /MD
Microsoft (R) 32-bit C/C++ Optimizing Compiler Version 16.00.40219.01 for 80x86
Copyright (C) Microsoft Corporation.  All rights reserved.

st.c
c:\polygon\c\st.c(11) : warning C4700: uninitialized local variable 'c' used
c:\polygon\c\st.c(11) : warning C4700: uninitialized local variable 'b' used
c:\polygon\c\st.c(11) : warning C4700: uninitialized local variable 'a' used
Microsoft (R) Incremental Linker Version 10.00.40219.01
Copyright (C) Microsoft Corporation.  All rights reserved.

/out:st.exe
st.obj
\end{lstlisting}

\RU{Но когда мы запускаем}\EN{But when we run the compiled program}\dots

\begin{lstlisting}
c:\Polygon\c>st
1, 2, 3
\end{lstlisting}

\RU{Ох. Вот это странно. Мы ведь не устанавливали значения никаких переменных в}\EN{Oh, 
what a weird thing! We did not set any variables in} \TT{f2()}. 
\RU{Эти значения --- это \q{привидения}, которые всё ещё в стеке.}
\EN{These are \q{ghosts} values, which are still in the stack.}

\clearpage
\RU{Загрузим пример в}\EN{Let's load the example into} \olly:

\begin{figure}[H]
\centering
\includegraphics[scale=\FigScale]{patterns/02_stack/08_noise/olly1.png}
\caption{\olly: \TT{f1()}}
\label{fig:stack_noise_olly1}
\end{figure}

\RU{Когда}\EN{When} \TT{f1()} \RU{заполняет переменные}\EN{assigns the variables} $a$, $b$ \AndENRU $c$ 
\RU{они сохраняются по адресу}\EN{, their values are stored at the address} \TT{0x1FF860} 
\RU{\etc{}.}\EN{and so on.}

\clearpage
\RU{А когда исполняется}\EN{And when} \TT{f2()}\EN{ executes}:

\begin{figure}[H]
\centering
\includegraphics[scale=\FigScale]{patterns/02_stack/08_noise/olly2.png}
\caption{\olly: \TT{f2()}}
\label{fig:stack_noise_olly2}
\end{figure}

... $a$, $b$ \AndENRU $c$ \RU{в функции}\EN{of} \TT{f2()} \RU{находятся по тем же адресам!}
\EN{are located at the same addresses!}
\RU{Пока никто не перезаписал их, так что они здесь в нетронутом виде.}
\EN{No one has overwritten the values yet, so at that point they are still untouched.}

\RU{Для создания такой странной ситуации несколько функций должны исполняться друг за другом
и \ac{SP} должен быть одинаковым при входе в функции, т.е. у функций должно быть равное количество
аргументов). Тогда локальные переменные будут расположены в том же месте стека.}
\EN{So, for this weird situation to occur, several functions have to be called one after another and
\ac{SP} has to be the same at each function entry (i.e., they have the same number
of arguments). Then the local variables will be located at the same positions in the stack.}

\RU{Подводя итоги, все значения в стеке (да и памяти вообще) это значения оставшиеся от 
исполнения предыдущих функций.}
\EN{Summarizing, all values in the stack (and memory cells in general) 
have values left there from previous function executions.}
\RU{Строго говоря, они не случайны, они скорее непредсказуемы.}
\EN{They are not random in the strict sense, but rather have unpredictable values.}

\RU{А как иначе}\EN{Is there another option}?
\RU{Можно было бы очищать части стека перед исполнением каждой функции,
но это слишком много лишней (и ненужной) работы.}
\EN{It probably would be possible to clear portions of the stack before each function execution,
but that's too much extra (and unnecessary) work.}

\subsection{MSVC 2013}

\EN{The example was compiled by}\RU{Этот пример был скомпилирован в} MSVC 2010.
\EN{But the reader of this book made attempt to compile this example in MSVC 2013, ran it, and got all 3 numbers reversed:}%
\RU{Но один читатель этой книги сделал попытку скомпилировать пример в MSVC 2013, запустил и увидел 3 числа в обратном порядке:}

\begin{lstlisting}
c:\Polygon\c>st
3, 2, 1
\end{lstlisting}

\EN{Why?}\RU{Почему?}

\EN{I also compiled this example in MSVC 2013 and saw this:}%
\RU{Я также попробовал скомпилировать этот пример в MSVC 2013 и увидел это:}

\begin{lstlisting}[caption=MSVC 2013]
_a$ = -12						; size = 4
_b$ = -8						; size = 4
_c$ = -4						; size = 4
_f2	PROC

...

_f2	ENDP

_c$ = -12						; size = 4
_b$ = -8						; size = 4
_a$ = -4						; size = 4
_f1	PROC

...

_f1	ENDP
\end{lstlisting}

\EN{Unlike MSVC 2010, MSVC 2013 allocated a/b/c variables in function \TT{f2()} in reverse order.}%
\RU{В отличии от MSVC 2010, MSVC 2013 разместил переменные a/b/c в функции \TT{f2()} в обратном порядке.}
\EN{And this is completely correct, because \CCpp standards has no rule, in which order local variables must be allocated in the local stack, if at all.}%
\RU{И это полностью корректно, потому что в стандартах \CCpp нет правила, в каком порядке локальные переменные должны быть размещены в локальном стеке,
если вообще.}
\EN{The reason of difference is because MSVC 2010 has one way to do it, and MSVC 2013 has probably something changed inside of compiler guts, so it behaves
slightly different.}%
\RU{Разница есть из-за того что MSVC 2010 делает это одним способом, а в MSVC 2013, вероятно, что-то немного изменили во внутренностях компилятора,
так что он ведет себя слегка иначе.}


\section{\Exercises}

\subsection{\Exercise \#1}
\label{exercise_stack_1}

\RU{Если это скомпилировать в MSVC и запустить, появится три числа. Откуда они берутся? 
Откуда они берутся если скомпилировать в MSVC с оптимизациями (\Ox)?}
\EN{If to compile this piece of code in MSVC and run, a three number will be printed. 
Where they are came from?
Where they are came from if to compile it in MSVC with optimization (\Ox)?}
\RU{Почему в GCC ситуация совсем иная}\EN{Why the situation is completely different in GCC}?

\begin{lstlisting}
#include <stdio.h>

int main()
{
	printf ("%d, %d, %d\n");

	return 0;
};
\end{lstlisting}

\RU{Ответ}\EN{Answer}: \ref{exercise_solutions_stack_1}.


}
\ITA{\section{\Stack}
\label{sec:stack}
\myindex{\Stack}

Lo stack e' una delle strutture dati piu' importanti in informatica
\footnote{\href{http://go.yurichev.com/17119}{wikipedia.org/wiki/Call\_stack}}.
\ac{AKA} \ac{LIFO}.

Tecnicamente, e' soltanto un blocco di memoria nella memoria di un processo insieme al registro \ESP o \RSP in x86 o x86, o il registro \ac{SP} in ARM, come puntatore all'interno di quel blocco.

\myindex{ARM!\Instructions!PUSH}
\myindex{ARM!\Instructions!POP}
\myindex{x86!\Instructions!PUSH}
\myindex{x86!\Instructions!POP}
Le istruzioni di accesso allo stack piu' usate sono \PUSH e \POP (sia in x86 che in ARM Thumb-mode).
\PUSH sottrae da \ESP/\RSP/\ac{SP} 4 in modalita' 32-bit (oppur 8 in modalita' 64-bit) e scrive successivamente il contenuto del suo unico operando nell'indirizzo di memoria puntato da \ESP/\RSP/\ac{SP}.

\POP e' l'operazione inversa: recupera il dato dalla memoria a cui punta \ac{SP}, lo carica nell'operando dell'istruzione (di solito un registro)
e successivamente aggiunge 4 (o 8) allo \gls{stack pointer}.

A seguito dell'allocazione dello stack, lo \gls{stack pointer} punta alla base (fondo) dello stack.
\PUSH decrementa lo \gls{stack pointer} e \POP lo incrementa.
La base dello stack e' in realta' all'inizio della memoria allocata per il blocco (porzione) dello stack. Sembra strano, ma e' cosi'.

ARM supporta stack decrescenti e crescenti.

\myindex{ARM!\Instructions!STMFD}
\myindex{ARM!\Instructions!LDMFD}
\myindex{ARM!\Instructions!STMED}
\myindex{ARM!\Instructions!LDMED}
\myindex{ARM!\Instructions!STMFA}
\myindex{ARM!\Instructions!LDMFA}
\myindex{ARM!\Instructions!STMEA}
\myindex{ARM!\Instructions!LDMEA}

Ad esempio le istruzioni \ac{STMFD}/\ac{LDMFD}, \ac{STMED}/\ac{LDMED} sono fatte per operare con uno stack decrescente (che cresce verso il basso, inizia con un indirizzo alto e prosegue verso il basso).
Le istruzioni \ac{STMFA}/\ac{LDMFA}, \ac{STMEA}/\ac{LDMEA} sono fatte per operare con uno stack crescente (che cresce verso l'alto, da un indirizzo basso verso uno piu alto).

% It might be worth mentioning that STMED and STMEA write first,
% and then move the pointer,
% and that LDMED and LDMEA move the pointer first, and then read.
% In other words, ARM not only lets the stack grow in a non-standard direction,
% but also in a non-standard order.
% Maybe this can be in the glossary, which would explain why E stands for "empty".

\subsection{Perche' lo stack cresce al contrario?}
\label{stack_grow_backwards}

Intuitivamente potremmo pensare che lo stack cresca verso l'alto, ovvero verso indirizzi piu' alti, come qualunque altra struttura dati.

La ragione per cui lo stack cresce verso il basso e' probabilmente di natura storica.
Quando i computer erano talmente grandi da occupare un'intera stanza, era facile dividere la memoria in due parti, una per lo 
\gls{heap} e l'altra per lo stack.
Ovviamente non era possibile sapere a priori quanto sarebbero stati grandi lo stack e lo \gls{heap} durante l'esecuzione di un programma,
e questa soluzione era la piu' semplice.

\begin{center}
	\begin{tikzpicture}
	\tikzstyle{every path}=[thick]

	\node [rectangle,draw,minimum width=6cm, minimum height=2cm] (memory) {};
	\node [] [right=0.2cm of memory.west] (heap) {\MLHeap};
	\node [] [left=0.2cm of memory.east] (stack) {\MLStack};

	\node [] (center1) [right=2cm of memory.west] {};
	\node [] (center2) [left=2cm of memory.east] {};

	\draw [->] (heap) -- (center1);
	\draw [->] (stack) -- (center2);

	\node [] [above left=1.1cm and 0.2cm of heap] (t1) {\MLStartOfHeap};
	\node [] [above right=1.1cm and 0.2cm of stack] (t2) {\MLStartOfStack};

	\draw [->] (t1) -- (memory.west);
	\draw [->] (t2) -- (memory.east);

	\end{tikzpicture}
\end{center}


In \RitchieThompsonUNIX possiamo leggere:

\begin{framed}
\begin{quotation}
The user-core part of an image is divided into three logical segments.
The program text segment begins at location 0 in the virtual address space.
During execution, this segment is write-protected and a single copy of it is shared among all processes executing the same program.
At the first 8K byte boundary above the program text segment in the virtual address space begins a nonshared, writable data segment, the size of which may be extended by a system call.
Starting at the highest address in the virtual address space is a stack segment, which automatically grows downward as the hardware's stack pointer fluctuates.

Il nucleo utente di una immagine e' diviso in tre segmenti logici.
Il segmento text del programma inizia in posizione 0 nel virtual address space.
Durante l'esecuzione questo segmento viene protetto da scrittura, ed una sua singola copia viene condivisa tra i processi che eseguono lo stesso programma.
Al primo limite di 8K byte dopra il segmento text del programma, nel virtual address space comincia un segmento dati scrivibile, non condiviso, le cui dimensioni possono essere estese da una chiamata di sistema.
A partire dall'indirizzo piu' alto nel virtual address space c'e' lo stack segment, che automaticammente cresce verso il basso al variare dello stack pointer hardware.
\end{quotation}
\end{framed}

Questo ricorda molto come alcuni studenti utilizzino lo stesso quaderno per prendere appunti di due diverse materie:
gli appunti per la prima materia sono scritti normalmente, e quelli della seconda materia sono scritti a partire dalla fine del quaderno, capovolgendolo.
Le note si potrebbero "incontrare" da qualche parte in mezzo al quaderno, nel caso in cui non ci sia abbastanza spazio libero.

% I think if we want to expand on this analogy,
% one might remember that the line number increases as as you go down a page.
% So when you decrease the address when pushing to the stack, visually,
% the stack does grow upwards.
% Of course, the problem is that in most human languages,
% just as with computers,
% we write downwards, so this direction is what makes buffer overflows so messy.

\subsection{Per cosa viene usato lo stack?}

% subsections
\EN{\input{patterns/02_stack/01_saving_ret_addr_EN}}
\RU{\input{patterns/02_stack/01_saving_ret_addr_RU}}
\DE{\input{patterns/02_stack/01_saving_ret_addr_DE}}
\FR{\input{patterns/02_stack/01_saving_ret_addr_FR}}
\PTBR{\input{patterns/02_stack/01_saving_ret_addr_PTBR}}
\ITA{\input{patterns/02_stack/01_saving_ret_addr_ITA}}

\subsection{\RU{Передача параметров функции}\EN{Passing function arguments}}

\RU{Самый распространенный способ передачи параметров в x86 называется}
\EN{The most popular way to pass parameters in x86 is called} \q{cdecl}:

\begin{lstlisting}
push arg3
push arg2
push arg1
call f
add esp, 12 ; 4*3=12
\end{lstlisting}

\RU{Вызываемая функция получает свои параметры также через указатель стека.}
\EN{\Gls{callee} functions get their arguments via the stack pointer.}

\RU{Следовательно, так расположены значения в стеке перед исполнением самой первой инструкции
функции \ttf{}:}
\EN{Therefore, this is how the argument values are located in the stack before the execution
of the \ttf{} function's very first instruction:}

\begin{center}
\begin{tabular}{ | l | l | }
\hline
ESP & \RU{адрес возврата}\EN{return address} \\
\hline
ESP+4 & \argument \#1, \MarkedInIDAAs{} \TT{arg\_0} \\
\hline
ESP+8 & \argument \#2, \MarkedInIDAAs{} \TT{arg\_4} \\
\hline
ESP+0xC & \argument \#3, \MarkedInIDAAs{} \TT{arg\_8} \\
\hline
\dots & \dots \\
\hline
\end{tabular}
\end{center}

\ifx\LITE\undefined
\RU{См. также в соответствующем разделе о других способах передачи аргументов через стек}
\EN{For more information on other calling conventions see also section}~(\myref{sec:callingconventions}).
\fi
\RU{Важно отметить, что, в общем, никто не заставляет программистов передавать параметры именно через стек,
это не является требованием к исполняемому коду.}
\EN{It is worth noting that nothing obliges programmers to pass arguments through the stack. It is not a requirement.}
\RU{Вы можете делать это совершенно иначе, не используя стек вообще.}
\EN{One could implement any other method without using the stack at all.}

\RU{К примеру, можно выделять в \glslink{heap}{куче} место для аргументов, 
заполнять их и передавать в функцию указатель на это место через \EAX. И это вполне будет работать}%
\EN{For example, it is possible to allocate a space for arguments in the \gls{heap}, fill it and pass it to a function 
via a pointer to this block in the \EAX register. This will work}%
\footnote{\RU{Например, в книге Дональда Кнута \q{Искусство программирования}, в разделе 1.4.1 
посвященном подпрограммам \cite[раздел 1.4.1]{Knuth:1998:ACP:521463}, 
мы можем прочитать о возможности располагать параметры для вызываемой подпрограммы после инструкции \JMP,
передающей управление подпрограмме. Кнут описывает, что это было особенно удобно для компьютеров IBM System/360.}%
\EN{For example, in the \q{The Art of Computer Programming} book by Donald Knuth, 
in section 1.4.1 dedicated to subroutines \cite[section 1.4.1]{Knuth:1998:ACP:521463},
we could read that one way to supply arguments to a subroutine is simply to list them after the \JMP instruction
passing control to subroutine. Knuth explains that this method was particularly convenient on IBM System/360.}}.
\RU{Однако традиционно сложилось, что в x86 и ARM передача аргументов происходит именно через стек.}
% I am unsure about what this comment means.
% My guess is that the arguments are put in the memory position after
% the jump instruction, so you could say:
% "one way to supply arguments to a subroutine is simply to list them in memory
% after the \JMP instruction that passes control to the subroutine."
% Right now, "after" also sounds like it refers to the time after
% the jump happens, which I think is too late.
\EN{However, it is a convenient custom in x86 and ARM to use the stack for this purpose.} \\
\\
\RU{Кстати, вызываемая функция не имеет информации о количестве переданных ей аргументов.}
\EN{By the way, the \gls{callee} function does not have any information about how many arguments were passed.}
\RU{Функции Си с переменным количеством аргументов (как \printf) определяют их количество по 
спецификаторам строки формата (начинающиеся со знака \%).}
\EN{C functions with a variable number of arguments (like \printf) determine their number using format string  specifiers (which begin with the \% symbol).}
\RU{Если написать что-то вроде}\EN{If we write something like} 

\begin{lstlisting}
printf("%d %d %d", 1234);
\end{lstlisting}

\printf \RU{выведет 1234, затем ещё два случайных числа, которые волею случая оказались в стеке рядом.}
\EN{will print 1234, and then two random numbers, which were lying next to it in the stack.}\\
\\
\RU{Вот почему не так уж и важно, как объявлять функцию \main}
\EN{That's why it is not very important how we declare the \main function}: \RU{как}\EN{as} \main, 
\TT{main(int argc, char *argv[])} 
\RU{либо}\EN{or} \TT{main(int argc, char *argv[], char *envp[])}.

\RU{В реальности, \ac{CRT}-код вызывает \main примерно так:}
\EN{In fact, the \ac{CRT}-code is calling \main roughly as:}

\begin{lstlisting}
push envp
push argv
push argc
call main
...
\end{lstlisting}

\RU{Если вы объявляете \main без аргументов, они, тем не менее, присутствуют в стеке, но не используются.}
\EN{If you declare \main as \main without arguments, they are, nevertheless, still present in the stack, but
are not used.}
\RU{Если вы объявите \main как}\EN{If you declare \main as} \TT{main(int argc, char *argv[])}, 
\RU{вы можете использовать два первых аргумента, а третий останется для вашей функции \q{невидимым}.}
\EN{you will be able to use first two arguments, and the third will remain \q{invisible} for your function.}
\RU{Более того, можно даже объявить}\EN{Even more, it is possible to declare} \TT{main(int argc)}, 
\RU{и это будет работать}\EN{and it will work}.


\EN{\subsubsection{Local variable storage}

A function could allocate space in the stack for its local variables just by decreasing 
the \gls{stack pointer} towards the stack bottom.

% I think here, "stack bottom" means the lowest address in the stack space,
% but the reader might also think it means towards the top of the stack space,
% like in a pop, so you might change "towards the stack bottom" to
% "towards the lowest address of the stack", or just take it out,
% since "decreasing" also suggests that.

Hence, it's very fast, no matter how many local variables are defined.
It is also not a requirement to store local variables in the stack.
You could store local variables wherever you like, 
but traditionally this is how it's done.

}
\RU{\subsubsection{Хранение локальных переменных}

Функция может выделить для себя некоторое место в стеке для локальных переменных, просто отодвинув 
\glslink{stack pointer}{указатель стека} глубже к концу стека.

% I think here, "stack bottom" means the lowest address in the stack space,
% but the reader might also think it means towards the top of the stack space,
% like in a pop, so you might change "towards the stack bottom" to
% "towards the lowest address of the stack", or just take it out,
% since "decreasing" also suggests that.

Это очень быстро вне зависимости от количества локальных переменных.
Хранить локальные переменные в стеке не является необходимым требованием. 
Вы можете хранить локальные переменные где угодно. 
Но по традиции всё сложилось так.

}
\PTBR{\subsubsection{Armazenamento de variáveis locais}

Uma função poderia alocar espaço na pilha para suas variáveis locais simplesmente decrementando o ponteiro da pilha.

% I think here, "stack bottom" means the lowest address in the stack space,
% but the reader might also think it means towards the top of the stack space,
% like in a pop, so you might change "towards the stack bottom" to
% "towards the lowest address of the stack", or just take it out,
% since "decreasing" also suggests that.

Consequentemente, é muito rápido, não importando quantas variáveis locais serão definidas.
Também não é um requisito armazenar variáveis locais na pilha.
Você pode armazenar variáveis locais onde você quiser, mas, tradicionalmente, é assim que é feito.

}
\EN{\input{patterns/02_stack/04_alloca/main_EN}}
\FR{\input{patterns/02_stack/04_alloca/main_FR}}
\RU{\input{patterns/02_stack/04_alloca/main_RU}}
\PTBR{\input{patterns/02_stack/04_alloca/main_PTBR}}
\ITA{\input{patterns/02_stack/04_alloca/main_ITA}}
\DE{\input{patterns/02_stack/04_alloca/main_DE}}

\subsection{(Windows) SEH}
\index{Windows!Structured Exception Handling}

\RU{В стеке хранятся записи \ac{SEH} для функции (если они присутствуют)}%
\EN{\ac{SEH} records are also stored on the stack (if they are present).}.

\ifx\LITE\undefined
\RU{Читайте больше о нем здесь}\EN{Read more about it}: (\myref{sec:SEH}).
\fi

\subsection{\RU{Защита от переполнений буфера}\EN{Buffer overflow protection}\PTBR{Proteção contra estouro de buffer}}

\RU{Здесь больше об этом}\EN{More about it here}\PTBR{Mais sobre aqui}~(\myref{subsec:bufferoverflow}).



\subsubsection{Deallocazione automatica dei dati nello stack}

Probabilmente la ragione per cui si memorizzano nello stack le variabili locali e i record SEH deriva dal fatto che questi dati vengono "liberati" automaticamente all'uscita dalla funzione,
usando soltanto un'istruzione per correggere lo stack pointer (spesso e' \ADD).
Si puo' dire che anche gli argomenti delle funzioni sono deallocati automaticamente alla fine della funzione.
Invece, qualunque altra cosa memorizzata nello \IT{heap} deve essere deallocata esplicitamente.

% sections
\EN{\input{patterns/02_stack/07_layout_EN}}
\RU{\subsection{Разметка типичного стека}

Разметка типичного стека в 32-битной среде
перед исполнением самой первой инструкции функции выглядит так:

\input{patterns/02_stack/stack_layout}

% I think this only applies to RISC architectures
% that don't have a POP instruction that only lets you read one value
% (ie. ARM and MIPS).
% In x86, the return address is saved before entering the function,
% and the function does not have the chance to save the frame pointer.
% Also, you should mention that this is how the stack looks like
% right after the function prologue,
% which some readers might think is the first instruction,
% but is needed to save the frame pointer.

}
\PTBR{\subsection{Um modelo típico de pilha}

Um modelo típico de pilha em um ambiente 32-bits no início de uma função,
antes da execução da primeira instrução, se parece com isso:

\input{patterns/02_stack/stack_layout}

% I think this only applies to RISC architectures
% that don't have a POP instruction that only lets you read one value
% (ie. ARM and MIPS).
% In x86, the return address is saved before entering the function,
% and the function does not have the chance to save the frame pointer.
% Also, you should mention that this is how the stack looks like
% right after the function prologue,
% which some readers might think is the first instruction,
% but is needed to save the frame pointer.
}
\section{\RU{Мусор в стеке}\EN{Noise in stack}}

\RU{Часто в этой книге говорится о \q{шуме} или \q{мусоре} в стеке или памяти.}
\EN{Often in this book \q{noise} or \q{garbage} values in the stack or memory are mentioned.}
\RU{Откуда он берется}\EN{Where do they come from}?
\RU{Это то, что осталось там после исполнения предыдущих функций.}
\EN{These are what was left in there after other functions' executions.}
\RU{Короткий пример}\EN{Short example}:

\lstinputlisting{patterns/02_stack/08_noise/st.c}

\RU{Компилируем}\EN{Compiling}\dots

\lstinputlisting[caption=\NonOptimizing MSVC 2010]{patterns/02_stack/08_noise/st.asm}

\RU{Компилятор поворчит немного}\EN{The compiler will grumble a little bit}\dots

\begin{lstlisting}
c:\Polygon\c>cl st.c /Fast.asm /MD
Microsoft (R) 32-bit C/C++ Optimizing Compiler Version 16.00.40219.01 for 80x86
Copyright (C) Microsoft Corporation.  All rights reserved.

st.c
c:\polygon\c\st.c(11) : warning C4700: uninitialized local variable 'c' used
c:\polygon\c\st.c(11) : warning C4700: uninitialized local variable 'b' used
c:\polygon\c\st.c(11) : warning C4700: uninitialized local variable 'a' used
Microsoft (R) Incremental Linker Version 10.00.40219.01
Copyright (C) Microsoft Corporation.  All rights reserved.

/out:st.exe
st.obj
\end{lstlisting}

\RU{Но когда мы запускаем}\EN{But when we run the compiled program}\dots

\begin{lstlisting}
c:\Polygon\c>st
1, 2, 3
\end{lstlisting}

\RU{Ох. Вот это странно. Мы ведь не устанавливали значения никаких переменных в}\EN{Oh, 
what a weird thing! We did not set any variables in} \TT{f2()}. 
\RU{Эти значения --- это \q{привидения}, которые всё ещё в стеке.}
\EN{These are \q{ghosts} values, which are still in the stack.}

\clearpage
\RU{Загрузим пример в}\EN{Let's load the example into} \olly:

\begin{figure}[H]
\centering
\includegraphics[scale=\FigScale]{patterns/02_stack/08_noise/olly1.png}
\caption{\olly: \TT{f1()}}
\label{fig:stack_noise_olly1}
\end{figure}

\RU{Когда}\EN{When} \TT{f1()} \RU{заполняет переменные}\EN{assigns the variables} $a$, $b$ \AndENRU $c$ 
\RU{они сохраняются по адресу}\EN{, their values are stored at the address} \TT{0x1FF860} 
\RU{\etc{}.}\EN{and so on.}

\clearpage
\RU{А когда исполняется}\EN{And when} \TT{f2()}\EN{ executes}:

\begin{figure}[H]
\centering
\includegraphics[scale=\FigScale]{patterns/02_stack/08_noise/olly2.png}
\caption{\olly: \TT{f2()}}
\label{fig:stack_noise_olly2}
\end{figure}

... $a$, $b$ \AndENRU $c$ \RU{в функции}\EN{of} \TT{f2()} \RU{находятся по тем же адресам!}
\EN{are located at the same addresses!}
\RU{Пока никто не перезаписал их, так что они здесь в нетронутом виде.}
\EN{No one has overwritten the values yet, so at that point they are still untouched.}

\RU{Для создания такой странной ситуации несколько функций должны исполняться друг за другом
и \ac{SP} должен быть одинаковым при входе в функции, т.е. у функций должно быть равное количество
аргументов). Тогда локальные переменные будут расположены в том же месте стека.}
\EN{So, for this weird situation to occur, several functions have to be called one after another and
\ac{SP} has to be the same at each function entry (i.e., they have the same number
of arguments). Then the local variables will be located at the same positions in the stack.}

\RU{Подводя итоги, все значения в стеке (да и памяти вообще) это значения оставшиеся от 
исполнения предыдущих функций.}
\EN{Summarizing, all values in the stack (and memory cells in general) 
have values left there from previous function executions.}
\RU{Строго говоря, они не случайны, они скорее непредсказуемы.}
\EN{They are not random in the strict sense, but rather have unpredictable values.}

\RU{А как иначе}\EN{Is there another option}?
\RU{Можно было бы очищать части стека перед исполнением каждой функции,
но это слишком много лишней (и ненужной) работы.}
\EN{It probably would be possible to clear portions of the stack before each function execution,
but that's too much extra (and unnecessary) work.}

\subsection{MSVC 2013}

\EN{The example was compiled by}\RU{Этот пример был скомпилирован в} MSVC 2010.
\EN{But the reader of this book made attempt to compile this example in MSVC 2013, ran it, and got all 3 numbers reversed:}%
\RU{Но один читатель этой книги сделал попытку скомпилировать пример в MSVC 2013, запустил и увидел 3 числа в обратном порядке:}

\begin{lstlisting}
c:\Polygon\c>st
3, 2, 1
\end{lstlisting}

\EN{Why?}\RU{Почему?}

\EN{I also compiled this example in MSVC 2013 and saw this:}%
\RU{Я также попробовал скомпилировать этот пример в MSVC 2013 и увидел это:}

\begin{lstlisting}[caption=MSVC 2013]
_a$ = -12						; size = 4
_b$ = -8						; size = 4
_c$ = -4						; size = 4
_f2	PROC

...

_f2	ENDP

_c$ = -12						; size = 4
_b$ = -8						; size = 4
_a$ = -4						; size = 4
_f1	PROC

...

_f1	ENDP
\end{lstlisting}

\EN{Unlike MSVC 2010, MSVC 2013 allocated a/b/c variables in function \TT{f2()} in reverse order.}%
\RU{В отличии от MSVC 2010, MSVC 2013 разместил переменные a/b/c в функции \TT{f2()} в обратном порядке.}
\EN{And this is completely correct, because \CCpp standards has no rule, in which order local variables must be allocated in the local stack, if at all.}%
\RU{И это полностью корректно, потому что в стандартах \CCpp нет правила, в каком порядке локальные переменные должны быть размещены в локальном стеке,
если вообще.}
\EN{The reason of difference is because MSVC 2010 has one way to do it, and MSVC 2013 has probably something changed inside of compiler guts, so it behaves
slightly different.}%
\RU{Разница есть из-за того что MSVC 2010 делает это одним способом, а в MSVC 2013, вероятно, что-то немного изменили во внутренностях компилятора,
так что он ведет себя слегка иначе.}


\section{\Exercises}

\subsection{\Exercise \#1}
\label{exercise_stack_1}

\RU{Если это скомпилировать в MSVC и запустить, появится три числа. Откуда они берутся? 
Откуда они берутся если скомпилировать в MSVC с оптимизациями (\Ox)?}
\EN{If to compile this piece of code in MSVC and run, a three number will be printed. 
Where they are came from?
Where they are came from if to compile it in MSVC with optimization (\Ox)?}
\RU{Почему в GCC ситуация совсем иная}\EN{Why the situation is completely different in GCC}?

\begin{lstlisting}
#include <stdio.h>

int main()
{
	printf ("%d, %d, %d\n");

	return 0;
};
\end{lstlisting}

\RU{Ответ}\EN{Answer}: \ref{exercise_solutions_stack_1}.

}
\DE{\section{\Stack}
\label{sec:stack}
\myindex{\Stack}

Der Stack ist eine der Fundamentalen Datenstrukturen in der Informatik.
\footnote{\href{http://go.yurichev.com/17119}{wikipedia.org/wiki/Call\_Stack}}.
\ac{AKA} \ac{LIFO}.

Technisch betrachtet ist es ein Stapel Speicher innerhalb des Prozessspeichers der zusammen mit den \ESP (x86), \RSP (x64) oder dem \ac{SP} (ARM) Register als ein Zeiger in diesem Speicherblock fungiert.

\myindex{ARM!\Instructions!PUSH}
\myindex{ARM!\Instructions!POP}
\myindex{x86!\Instructions!PUSH}
\myindex{x86!\Instructions!POP}

Die häufigsten Stack-Zugriffsinstruktionen sind die \PUSH und \POP Instruktionen (in beidem x86 und ARM Thumb-Modus). \PUSH subtrahiert vom \ESP/\RSP/\ac{SP} 4 Byte im 32-Bit Modus (oder 8 im 64-Bit Modus) und schreibt dann den Inhalt des Zeigers an die Adresse auf die von \ESP/\RSP/\ac{SP} gezeigt wird.

\POP ist die umgekehrte Operation: Die Daten des Zeigers für die Speicherregion auf die von \ac{SP}
gezeigt wird werden ausgelesen und die Inhalte in den Instruktionsoperanden geschreiben (oft ist das ein Register). Dann werden 4 (beziehungsweise 8 ) Byte zum \gls{stack pointer} addiert.

Nach der Stackallokation, zeigt der \gls{stack pointer} auf den Boden des Stacks.
\PUSH verringert den \gls{stack pointer} und \POP erhöht ihn.
Der Boden des Stacks ist eigentlich der Anfang der Speicherregion die für den Stack reserviert wurde.
Das wirkt zunächst seltsam, aber so funktioniert es.

ARM unterstützt beides, aufsteigende und absteigende Stacks.

\myindex{ARM!\Instructions!STMFD}
\myindex{ARM!\Instructions!LDMFD}
\myindex{ARM!\Instructions!STMED}
\myindex{ARM!\Instructions!LDMED}
\myindex{ARM!\Instructions!STMFA}
\myindex{ARM!\Instructions!LDMFA}
\myindex{ARM!\Instructions!STMEA}
\myindex{ARM!\Instructions!LDMEA}

Zum Beispiel die \ac{STMFD}/\ac{LDMFD} und \ac{STMED}/\ac{LDMED} Instruktionen sind alle dafür gedacht mit einem absteigendem Stack zu arbeiten ( wächst nach unten, fängt mit hohen Adressen an und entwickelt sich zu niedrigeren Adressen). Die \ac{STMFA}/\ac{LDMFA} und \ac{STMEA}/\ac{LDMEA} Instruktionen sind dazu gedacht mit einem aufsteigendem Stack zu arbeiten (wächst nach oben und fängt mit niedrigeren Adressen an und wächst nach oben).

% It might be worth mentioning that STMED and STMEA write first,
% and then move the pointer, and that LDMED and LDMEA move the pointer first, and then read.
% In other words, ARM not only lets the stack grow in a non-standard direction,
% but also in a non-standard order.
% Maybe this can be in the glossary, which would explain why E stands for "empty".

\subsection{Warum wächst der Stack nach unten?}
\label{stack_grow_backwards}

Intuitiv, würden man annehmen das der Stack nach oben wächst z.B Richtung höherer Adressen, so wie bei jeder anderen Datenstruktur.

Der Grund das der Stack rückwärts wächst ist wohl historisch bedingt. Als Computer so groß waren das sie einen ganzen Raum beansprucht haben war es einfach Speicher in zwei Sektionen zu unterteilen, einen Teil für den \gls{heap} und einen Teil für den Stack. Sicher war zu dieser Zeit nicht bekannt wie groß der \gls{heap} und der Stack wachsen würden, während der Programm Laufzeit, also war die Lösung die einfachste mögliche.

\begin{center}
	\begin{tikzpicture}
	\tikzstyle{every path}=[thick]

	\node [rectangle,draw,minimum width=6cm, minimum height=2cm] (memory) {};
	\node [] [right=0.2cm of memory.west] (heap) {\MLHeap};
	\node [] [left=0.2cm of memory.east] (stack) {\MLStack};

	\node [] (center1) [right=2cm of memory.west] {};
	\node [] (center2) [left=2cm of memory.east] {};

	\draw [->] (heap) -- (center1);
	\draw [->] (stack) -- (center2);

	\node [] [above left=1.1cm and 0.2cm of heap] (t1) {\MLStartOfHeap};
	\node [] [above right=1.1cm and 0.2cm of stack] (t2) {\MLStartOfStack};

	\draw [->] (t1) -- (memory.west);
	\draw [->] (t2) -- (memory.east);

	\end{tikzpicture}
\end{center}


In \RitchieThompsonUNIX können wir folgendes lesen:

\begin{framed}
\begin{quotation}
Der user-core eines Programm Images wird in drei logische Segmente unterteilt. Das Programm-Text Segment beginnt bei 0 im virtuellen Adress Speicher. Während der Ausführung wird das Segment als schreibgeschützt markiert und eine einzelne Kopie des Segments wird unter allen Prozessen geteilt die das Programm ausführen. An der ersten 8K grenze über dem Programm Text Segment im Virtuellen Speicher, fängt der ``nonshared'' Bereich an, der nach Bedarf von Syscalls erweitert werden kann. Beginnend bei der höchsten Adresse im Virtuellen Speicher ist das Stack Segment, das Automatisch nach unten wächst während der Hardware Stackpointer sich ändert.
\end{quotation}
\end{framed}

Das erinnert daran wie manche Schüler Notizen zu  zwei Vorträgen in einem Notebook dokumentieren:
Notizen für den ersten Vortrag werden normal notiert, und Notizen zur zum zweiten Vortrag werden 
ans Ende des Notizbuches geschrieben, indem man das Notizbuch umdreht. Die Notizen treffen sich irgendwann
im Notizbuch aufgrund des fehlenden Freien Platzes.

% I think if we want to expand on this analogy,
% one might remember that the line number increases as as you go down a page.
% So when you decrease the address when pushing to the stack, visually,
% the stack does grow upwards.
% Of course, the problem is that in most human languages,
% just as with computers,
% we write downwards, so this direction is what makes buffer overflows so messy.

\subsection{Für was wird der Stack benutzt?}

% subsections
\EN{\input{patterns/02_stack/01_saving_ret_addr_EN}}
\RU{\input{patterns/02_stack/01_saving_ret_addr_RU}}
\DE{\input{patterns/02_stack/01_saving_ret_addr_DE}}
\FR{\input{patterns/02_stack/01_saving_ret_addr_FR}}
\PTBR{\input{patterns/02_stack/01_saving_ret_addr_PTBR}}
\ITA{\input{patterns/02_stack/01_saving_ret_addr_ITA}}

\subsection{\RU{Передача параметров функции}\EN{Passing function arguments}}

\RU{Самый распространенный способ передачи параметров в x86 называется}
\EN{The most popular way to pass parameters in x86 is called} \q{cdecl}:

\begin{lstlisting}
push arg3
push arg2
push arg1
call f
add esp, 12 ; 4*3=12
\end{lstlisting}

\RU{Вызываемая функция получает свои параметры также через указатель стека.}
\EN{\Gls{callee} functions get their arguments via the stack pointer.}

\RU{Следовательно, так расположены значения в стеке перед исполнением самой первой инструкции
функции \ttf{}:}
\EN{Therefore, this is how the argument values are located in the stack before the execution
of the \ttf{} function's very first instruction:}

\begin{center}
\begin{tabular}{ | l | l | }
\hline
ESP & \RU{адрес возврата}\EN{return address} \\
\hline
ESP+4 & \argument \#1, \MarkedInIDAAs{} \TT{arg\_0} \\
\hline
ESP+8 & \argument \#2, \MarkedInIDAAs{} \TT{arg\_4} \\
\hline
ESP+0xC & \argument \#3, \MarkedInIDAAs{} \TT{arg\_8} \\
\hline
\dots & \dots \\
\hline
\end{tabular}
\end{center}

\ifx\LITE\undefined
\RU{См. также в соответствующем разделе о других способах передачи аргументов через стек}
\EN{For more information on other calling conventions see also section}~(\myref{sec:callingconventions}).
\fi
\RU{Важно отметить, что, в общем, никто не заставляет программистов передавать параметры именно через стек,
это не является требованием к исполняемому коду.}
\EN{It is worth noting that nothing obliges programmers to pass arguments through the stack. It is not a requirement.}
\RU{Вы можете делать это совершенно иначе, не используя стек вообще.}
\EN{One could implement any other method without using the stack at all.}

\RU{К примеру, можно выделять в \glslink{heap}{куче} место для аргументов, 
заполнять их и передавать в функцию указатель на это место через \EAX. И это вполне будет работать}%
\EN{For example, it is possible to allocate a space for arguments in the \gls{heap}, fill it and pass it to a function 
via a pointer to this block in the \EAX register. This will work}%
\footnote{\RU{Например, в книге Дональда Кнута \q{Искусство программирования}, в разделе 1.4.1 
посвященном подпрограммам \cite[раздел 1.4.1]{Knuth:1998:ACP:521463}, 
мы можем прочитать о возможности располагать параметры для вызываемой подпрограммы после инструкции \JMP,
передающей управление подпрограмме. Кнут описывает, что это было особенно удобно для компьютеров IBM System/360.}%
\EN{For example, in the \q{The Art of Computer Programming} book by Donald Knuth, 
in section 1.4.1 dedicated to subroutines \cite[section 1.4.1]{Knuth:1998:ACP:521463},
we could read that one way to supply arguments to a subroutine is simply to list them after the \JMP instruction
passing control to subroutine. Knuth explains that this method was particularly convenient on IBM System/360.}}.
\RU{Однако традиционно сложилось, что в x86 и ARM передача аргументов происходит именно через стек.}
% I am unsure about what this comment means.
% My guess is that the arguments are put in the memory position after
% the jump instruction, so you could say:
% "one way to supply arguments to a subroutine is simply to list them in memory
% after the \JMP instruction that passes control to the subroutine."
% Right now, "after" also sounds like it refers to the time after
% the jump happens, which I think is too late.
\EN{However, it is a convenient custom in x86 and ARM to use the stack for this purpose.} \\
\\
\RU{Кстати, вызываемая функция не имеет информации о количестве переданных ей аргументов.}
\EN{By the way, the \gls{callee} function does not have any information about how many arguments were passed.}
\RU{Функции Си с переменным количеством аргументов (как \printf) определяют их количество по 
спецификаторам строки формата (начинающиеся со знака \%).}
\EN{C functions with a variable number of arguments (like \printf) determine their number using format string  specifiers (which begin with the \% symbol).}
\RU{Если написать что-то вроде}\EN{If we write something like} 

\begin{lstlisting}
printf("%d %d %d", 1234);
\end{lstlisting}

\printf \RU{выведет 1234, затем ещё два случайных числа, которые волею случая оказались в стеке рядом.}
\EN{will print 1234, and then two random numbers, which were lying next to it in the stack.}\\
\\
\RU{Вот почему не так уж и важно, как объявлять функцию \main}
\EN{That's why it is not very important how we declare the \main function}: \RU{как}\EN{as} \main, 
\TT{main(int argc, char *argv[])} 
\RU{либо}\EN{or} \TT{main(int argc, char *argv[], char *envp[])}.

\RU{В реальности, \ac{CRT}-код вызывает \main примерно так:}
\EN{In fact, the \ac{CRT}-code is calling \main roughly as:}

\begin{lstlisting}
push envp
push argv
push argc
call main
...
\end{lstlisting}

\RU{Если вы объявляете \main без аргументов, они, тем не менее, присутствуют в стеке, но не используются.}
\EN{If you declare \main as \main without arguments, they are, nevertheless, still present in the stack, but
are not used.}
\RU{Если вы объявите \main как}\EN{If you declare \main as} \TT{main(int argc, char *argv[])}, 
\RU{вы можете использовать два первых аргумента, а третий останется для вашей функции \q{невидимым}.}
\EN{you will be able to use first two arguments, and the third will remain \q{invisible} for your function.}
\RU{Более того, можно даже объявить}\EN{Even more, it is possible to declare} \TT{main(int argc)}, 
\RU{и это будет работать}\EN{and it will work}.


\EN{\subsubsection{Local variable storage}

A function could allocate space in the stack for its local variables just by decreasing 
the \gls{stack pointer} towards the stack bottom.

% I think here, "stack bottom" means the lowest address in the stack space,
% but the reader might also think it means towards the top of the stack space,
% like in a pop, so you might change "towards the stack bottom" to
% "towards the lowest address of the stack", or just take it out,
% since "decreasing" also suggests that.

Hence, it's very fast, no matter how many local variables are defined.
It is also not a requirement to store local variables in the stack.
You could store local variables wherever you like, 
but traditionally this is how it's done.

}
\RU{\subsubsection{Хранение локальных переменных}

Функция может выделить для себя некоторое место в стеке для локальных переменных, просто отодвинув 
\glslink{stack pointer}{указатель стека} глубже к концу стека.

% I think here, "stack bottom" means the lowest address in the stack space,
% but the reader might also think it means towards the top of the stack space,
% like in a pop, so you might change "towards the stack bottom" to
% "towards the lowest address of the stack", or just take it out,
% since "decreasing" also suggests that.

Это очень быстро вне зависимости от количества локальных переменных.
Хранить локальные переменные в стеке не является необходимым требованием. 
Вы можете хранить локальные переменные где угодно. 
Но по традиции всё сложилось так.

}
\DE{\subsubsection{Local variable storage}

Eine Funktion kann platz für lokale Variablen allokieren in dem sie einfach den \glslink{stack pointer}{Stapel-Zeiger}
verkleinert in richtung der niedrigsten Adresse des Stacks verschiebt. 

% I think here, "stack bottom" means the lowest address in the stack space,
% but the reader might also think it means towards the top of the stack space,
% like in a pop, so you might change "towards the stack bottom" to
% "towards the lowest address of the stack", or just take it out,
% since "decreasing" also suggests that.

Dieser Weg ist ziemlich schnell, egal wie viele Variablen deffiniert werden.
Es ist aber keine Anforderung lokale Variablen auf dem Stack zu speichern.
Man kann lokale Variablen speicher wo immer man will, aber traditionell speichert
man sie auf dem Stack.
}
\PTBR{\subsubsection{Armazenamento de variáveis locais}

Uma função poderia alocar espaço na pilha para suas variáveis locais simplesmente decrementando o ponteiro da pilha.

% I think here, "stack bottom" means the lowest address in the stack space,
% but the reader might also think it means towards the top of the stack space,
% like in a pop, so you might change "towards the stack bottom" to
% "towards the lowest address of the stack", or just take it out,
% since "decreasing" also suggests that.

Consequentemente, é muito rápido, não importando quantas variáveis locais serão definidas.
Também não é um requisito armazenar variáveis locais na pilha.
Você pode armazenar variáveis locais onde você quiser, mas, tradicionalmente, é assim que é feito.

}
\EN{\input{patterns/02_stack/04_alloca/main_EN}}
\FR{\input{patterns/02_stack/04_alloca/main_FR}}
\RU{\input{patterns/02_stack/04_alloca/main_RU}}
\PTBR{\input{patterns/02_stack/04_alloca/main_PTBR}}
\ITA{\input{patterns/02_stack/04_alloca/main_ITA}}
\DE{\input{patterns/02_stack/04_alloca/main_DE}}

\subsection{(Windows) SEH}
\index{Windows!Structured Exception Handling}

\RU{В стеке хранятся записи \ac{SEH} для функции (если они присутствуют)}%
\EN{\ac{SEH} records are also stored on the stack (if they are present).}.

\ifx\LITE\undefined
\RU{Читайте больше о нем здесь}\EN{Read more about it}: (\myref{sec:SEH}).
\fi

\subsection{\RU{Защита от переполнений буфера}\EN{Buffer overflow protection}\PTBR{Proteção contra estouro de buffer}}

\RU{Здесь больше об этом}\EN{More about it here}\PTBR{Mais sobre aqui}~(\myref{subsec:bufferoverflow}).



\subsubsection{Automatisches deallokieren der Daten auf dem Stack}

Vielleicht ist der Grund warum man lokale Variablen und SEH Einträge auf dem Stack speichert, weil sie beim 
verlassen der Funktion automatisch aufgeräumt werden. Man braucht dabei nur eine Instruktion um die Position
des Stackpointers zu korrigieren (oftmals ist es die \ADD Instruktion). Funktions Argumente, könnte man sagen 
werden auch am Ende der Funktion deallokiert. Im Kontrast dazu, alles was auf dem \IT{heap} gespeichert wird muss
explizit deallokiert werden. 

% sections
\EN{\input{patterns/02_stack/07_layout_EN}}
\RU{\subsection{Разметка типичного стека}

Разметка типичного стека в 32-битной среде
перед исполнением самой первой инструкции функции выглядит так:

\input{patterns/02_stack/stack_layout}

% I think this only applies to RISC architectures
% that don't have a POP instruction that only lets you read one value
% (ie. ARM and MIPS).
% In x86, the return address is saved before entering the function,
% and the function does not have the chance to save the frame pointer.
% Also, you should mention that this is how the stack looks like
% right after the function prologue,
% which some readers might think is the first instruction,
% but is needed to save the frame pointer.

}
\DE{\subsection{Ein typisches Stack Layout}

Ein typisches Stacklayout auf einer 32-Bit Umgebung sieht am Anfang 
der ausführung einer Funktion, noch bevor der ausführung der ersten 
Instruktion wie folgt aus:

\input{patterns/02_stack/stack_layout}


% I think this only applies to RISC architectures
% that don't have a POP instruction that only lets you read one value
% (ie. ARM and MIPS).
% In x86, the return address is saved before entering the function,
% and the function does not have the chance to save the frame pointer.
% Also, you should mention that this is how the stack looks like
% right after the function prologue,
% which some readers might think is the first instruction,
% but is needed to save the frame pointer.
}
\PTBR{\subsection{Um modelo típico de pilha}

Um modelo típico de pilha em um ambiente 32-bits no início de uma função,
antes da execução da primeira instrução, se parece com isso:

\input{patterns/02_stack/stack_layout}

% I think this only applies to RISC architectures
% that don't have a POP instruction that only lets you read one value
% (ie. ARM and MIPS).
% In x86, the return address is saved before entering the function,
% and the function does not have the chance to save the frame pointer.
% Also, you should mention that this is how the stack looks like
% right after the function prologue,
% which some readers might think is the first instruction,
% but is needed to save the frame pointer.
}
\section{\RU{Мусор в стеке}\EN{Noise in stack}}

\RU{Часто в этой книге говорится о \q{шуме} или \q{мусоре} в стеке или памяти.}
\EN{Often in this book \q{noise} or \q{garbage} values in the stack or memory are mentioned.}
\RU{Откуда он берется}\EN{Where do they come from}?
\RU{Это то, что осталось там после исполнения предыдущих функций.}
\EN{These are what was left in there after other functions' executions.}
\RU{Короткий пример}\EN{Short example}:

\lstinputlisting{patterns/02_stack/08_noise/st.c}

\RU{Компилируем}\EN{Compiling}\dots

\lstinputlisting[caption=\NonOptimizing MSVC 2010]{patterns/02_stack/08_noise/st.asm}

\RU{Компилятор поворчит немного}\EN{The compiler will grumble a little bit}\dots

\begin{lstlisting}
c:\Polygon\c>cl st.c /Fast.asm /MD
Microsoft (R) 32-bit C/C++ Optimizing Compiler Version 16.00.40219.01 for 80x86
Copyright (C) Microsoft Corporation.  All rights reserved.

st.c
c:\polygon\c\st.c(11) : warning C4700: uninitialized local variable 'c' used
c:\polygon\c\st.c(11) : warning C4700: uninitialized local variable 'b' used
c:\polygon\c\st.c(11) : warning C4700: uninitialized local variable 'a' used
Microsoft (R) Incremental Linker Version 10.00.40219.01
Copyright (C) Microsoft Corporation.  All rights reserved.

/out:st.exe
st.obj
\end{lstlisting}

\RU{Но когда мы запускаем}\EN{But when we run the compiled program}\dots

\begin{lstlisting}
c:\Polygon\c>st
1, 2, 3
\end{lstlisting}

\RU{Ох. Вот это странно. Мы ведь не устанавливали значения никаких переменных в}\EN{Oh, 
what a weird thing! We did not set any variables in} \TT{f2()}. 
\RU{Эти значения --- это \q{привидения}, которые всё ещё в стеке.}
\EN{These are \q{ghosts} values, which are still in the stack.}

\clearpage
\RU{Загрузим пример в}\EN{Let's load the example into} \olly:

\begin{figure}[H]
\centering
\includegraphics[scale=\FigScale]{patterns/02_stack/08_noise/olly1.png}
\caption{\olly: \TT{f1()}}
\label{fig:stack_noise_olly1}
\end{figure}

\RU{Когда}\EN{When} \TT{f1()} \RU{заполняет переменные}\EN{assigns the variables} $a$, $b$ \AndENRU $c$ 
\RU{они сохраняются по адресу}\EN{, their values are stored at the address} \TT{0x1FF860} 
\RU{\etc{}.}\EN{and so on.}

\clearpage
\RU{А когда исполняется}\EN{And when} \TT{f2()}\EN{ executes}:

\begin{figure}[H]
\centering
\includegraphics[scale=\FigScale]{patterns/02_stack/08_noise/olly2.png}
\caption{\olly: \TT{f2()}}
\label{fig:stack_noise_olly2}
\end{figure}

... $a$, $b$ \AndENRU $c$ \RU{в функции}\EN{of} \TT{f2()} \RU{находятся по тем же адресам!}
\EN{are located at the same addresses!}
\RU{Пока никто не перезаписал их, так что они здесь в нетронутом виде.}
\EN{No one has overwritten the values yet, so at that point they are still untouched.}

\RU{Для создания такой странной ситуации несколько функций должны исполняться друг за другом
и \ac{SP} должен быть одинаковым при входе в функции, т.е. у функций должно быть равное количество
аргументов). Тогда локальные переменные будут расположены в том же месте стека.}
\EN{So, for this weird situation to occur, several functions have to be called one after another and
\ac{SP} has to be the same at each function entry (i.e., they have the same number
of arguments). Then the local variables will be located at the same positions in the stack.}

\RU{Подводя итоги, все значения в стеке (да и памяти вообще) это значения оставшиеся от 
исполнения предыдущих функций.}
\EN{Summarizing, all values in the stack (and memory cells in general) 
have values left there from previous function executions.}
\RU{Строго говоря, они не случайны, они скорее непредсказуемы.}
\EN{They are not random in the strict sense, but rather have unpredictable values.}

\RU{А как иначе}\EN{Is there another option}?
\RU{Можно было бы очищать части стека перед исполнением каждой функции,
но это слишком много лишней (и ненужной) работы.}
\EN{It probably would be possible to clear portions of the stack before each function execution,
but that's too much extra (and unnecessary) work.}

\subsection{MSVC 2013}

\EN{The example was compiled by}\RU{Этот пример был скомпилирован в} MSVC 2010.
\EN{But the reader of this book made attempt to compile this example in MSVC 2013, ran it, and got all 3 numbers reversed:}%
\RU{Но один читатель этой книги сделал попытку скомпилировать пример в MSVC 2013, запустил и увидел 3 числа в обратном порядке:}

\begin{lstlisting}
c:\Polygon\c>st
3, 2, 1
\end{lstlisting}

\EN{Why?}\RU{Почему?}

\EN{I also compiled this example in MSVC 2013 and saw this:}%
\RU{Я также попробовал скомпилировать этот пример в MSVC 2013 и увидел это:}

\begin{lstlisting}[caption=MSVC 2013]
_a$ = -12						; size = 4
_b$ = -8						; size = 4
_c$ = -4						; size = 4
_f2	PROC

...

_f2	ENDP

_c$ = -12						; size = 4
_b$ = -8						; size = 4
_a$ = -4						; size = 4
_f1	PROC

...

_f1	ENDP
\end{lstlisting}

\EN{Unlike MSVC 2010, MSVC 2013 allocated a/b/c variables in function \TT{f2()} in reverse order.}%
\RU{В отличии от MSVC 2010, MSVC 2013 разместил переменные a/b/c в функции \TT{f2()} в обратном порядке.}
\EN{And this is completely correct, because \CCpp standards has no rule, in which order local variables must be allocated in the local stack, if at all.}%
\RU{И это полностью корректно, потому что в стандартах \CCpp нет правила, в каком порядке локальные переменные должны быть размещены в локальном стеке,
если вообще.}
\EN{The reason of difference is because MSVC 2010 has one way to do it, and MSVC 2013 has probably something changed inside of compiler guts, so it behaves
slightly different.}%
\RU{Разница есть из-за того что MSVC 2010 делает это одним способом, а в MSVC 2013, вероятно, что-то немного изменили во внутренностях компилятора,
так что он ведет себя слегка иначе.}


\section{\Exercises}

\subsection{\Exercise \#1}
\label{exercise_stack_1}

\RU{Если это скомпилировать в MSVC и запустить, появится три числа. Откуда они берутся? 
Откуда они берутся если скомпилировать в MSVC с оптимизациями (\Ox)?}
\EN{If to compile this piece of code in MSVC and run, a three number will be printed. 
Where they are came from?
Where they are came from if to compile it in MSVC with optimization (\Ox)?}
\RU{Почему в GCC ситуация совсем иная}\EN{Why the situation is completely different in GCC}?

\begin{lstlisting}
#include <stdio.h>

int main()
{
	printf ("%d, %d, %d\n");

	return 0;
};
\end{lstlisting}

\RU{Ответ}\EN{Answer}: \ref{exercise_solutions_stack_1}.

}


\EN{\chapter{\PrintfSeveralArgumentsSectionName}

Now let's extend the \IT{\HelloWorldSectionName}~(\myref{sec:helloworld}) example, replacing \printf in
the \main function body with this:

\lstinputlisting[label=hw_c]{patterns/03_printf/1.c}

% sections
\subsection{x86}

% subsections:
\input{patterns/03_printf/x86/x86}
\input{patterns/03_printf/x86/x64}

\ifdefined\IncludeARM
\subsection{ARM}

\EN{\input{patterns/03_printf/ARM/ARM3_EN}}
\RU{\input{patterns/03_printf/ARM/ARM3_RU}}
\FR{\input{patterns/03_printf/ARM/ARM3_FR}}
\ITA{\input{patterns/03_printf/ARM/ARM3_ITA}}

\EN{\input{patterns/03_printf/ARM/ARM8_EN}}
\RU{\input{patterns/03_printf/ARM/ARM8_RU}}
\FR{\input{patterns/03_printf/ARM/ARM8_FR}}
\ITA{\input{patterns/03_printf/ARM/ARM8_ITA}}

\fi
\ifdefined\IncludeMIPS
\section{MIPS}

\subsection{3 \RU{аргумента}\EN{arguments}}

\subsubsection{\Optimizing GCC 4.4.5}

\RU{Главное отличие от примера ``\HelloWorldSectionName'' в том, что здесь на самом деле
вызывается \printf вместо \puts и еще три аргумента передаются в регистрах}
\EN{One main difference with ``\HelloWorldSectionName'' example is that \printf is actually called here
instead of \puts and 3 more arguments are passed in registers} \$5 \dots \$7 (\OrENRU \$A0 \dots \$A2).

\RU{Вот почему эти регистры имеют префикс A-, это значит, что они используются для передачи аргументов.}
\EN{So that's why these registers are prefixed with A-, meaning they are used for function arguments passing.}

\lstinputlisting[caption=\Optimizing GCC 4.4.5 (\assemblyOutput)]{patterns/03_printf/MIPS/printf3.O3.s.\LANG}

\lstinputlisting[caption=\Optimizing GCC 4.4.5 (IDA)]{patterns/03_printf/MIPS/printf3.O3.IDA.lst.\LANG}

\subsubsection{\NonOptimizing GCC 4.4.5}

\NonOptimizing GCC \RU{более многословен}\EN{is more verbose}:

\lstinputlisting[caption=\NonOptimizing GCC 4.4.5 (\assemblyOutput)]{patterns/03_printf/MIPS/printf3.O0.s.\LANG}

\lstinputlisting[caption=\NonOptimizing GCC 4.4.5 (IDA)]{patterns/03_printf/MIPS/printf3.O0.IDA.lst.\LANG}

\subsection{8 \RU{аргументов}\EN{arguments}}

\RU{Снова воспользуемся примером с 9-ю аргументами из предыдущей секции}\EN{Let's use again the example
with 9 arguments from the previous section}: \ref{example_printf8_x64}.

\lstinputlisting{patterns/03_printf/2.c}

\subsubsection{\Optimizing GCC 4.4.5}

\RU{Только 4 первых аргумента передаются в регистрах \$A0 \dots \$A3, так что остальные передаются 
через стек.}
\EN{Only 4 first arguments are passed in \$A0 \dots \$A3 registers, so others are passed via stack.}
\index{MIPS!O32}
\RU{Это соглашение о вызовах O32 (которое самое популярное в мире MIPS).
Другие соглашения о вызовах (как N32) могут наделять регистры другими ф-циями.}
\EN{That is O32 calling convention (which is most used and popular in MIPS world).
Other calling conventions (like N32) may have differ register purposes.}

\index{MIPS!\Instructions!SW}
\RU{SW означает ``Store Word'' (записать слово, из регистра в память).}
\EN{SW meaning ``Store Word'' (from register to memory).}
\RU{В MIPS нет инструкции для записи значения в память, так что для этого используется пара инструкций (LI/SW).}
\EN{MIPS lacks instruction for storing a value into memory, so instruction pair is to be used instead (LI/SW).}

\lstinputlisting[caption=\Optimizing GCC 4.4.5 (\assemblyOutput)]{patterns/03_printf/MIPS/printf8.O3.s.\LANG}

\lstinputlisting[caption=\Optimizing GCC 4.4.5 (IDA)]{patterns/03_printf/MIPS/printf8.O3.IDA.lst.\LANG}

\subsubsection{\NonOptimizing GCC 4.4.5}

\NonOptimizing GCC \RU{более многословен}\EN{is more verbose}:

\lstinputlisting[caption=\NonOptimizing GCC 4.4.5 (\assemblyOutput)]{patterns/03_printf/MIPS/printf8.O0.s.\LANG}

\lstinputlisting[caption=\NonOptimizing GCC 4.4.5 (IDA)]{patterns/03_printf/MIPS/printf8.O0.IDA.lst.\LANG}

\fi

\section{\Conclusion{}}

Here is a rough skeleton of the function call:

\lstinputlisting[caption=x86]{patterns/03_printf/skel1.lst.\LANG}

\lstinputlisting[caption=x64 (MSVC)]{patterns/03_printf/skel2.lst.\LANG}

\ifdefined\IncludeGCC
\lstinputlisting[caption=x64 (GCC)]{patterns/03_printf/skel3.lst.\LANG}
\fi

\ifdefined\IncludeARM
\lstinputlisting[caption=ARM]{patterns/03_printf/skel4.lst.\LANG}

\lstinputlisting[caption=ARM64]{patterns/03_printf/skel5.lst.\LANG}
\fi

\ifdefined\IncludeMIPS
\index{MIPS!O32}
\lstinputlisting[caption=MIPS (O32 calling convention)]{patterns/03_printf/skel_MIPS.lst.\LANG}
\fi

\section{By the way}

\index{fastcall}
By the way, this difference between the arguments passing in x86, x64, 
fastcall, ARM and MIPS is a good illustration of the fact that the CPU is oblivious to how the arguments are passed to functions. 
It is also possible to create a hypothetical compiler able to pass arguments 
via a special structure without using stack at all.

\ifdefined\IncludeMIPS
\index{MIPS!O32}
MIPS \$A0 \dots \$A3 registers are labelled this way only for convenience (that is in the O32 calling convention).
Programmers may use any other register (well, maybe except \$ZERO) 
to pass data or use any other calling convention.
\fi

The \ac{CPU} is not aware of calling conventions whatsoever.

We may also recall how newcoming assembly language programmers passing arguments into
other functions:
usually via registers, without any explicit order, or even via global variables.
Of course, it works fine.

}
\RU{\chapter{\PrintfSeveralArgumentsSectionName}

Попробуем теперь немного расширить пример \IT{\HelloWorldSectionName}~(\myref{sec:helloworld}),
написав в теле функции \main:

\lstinputlisting[label=hw_c]{patterns/03_printf/1.c}

% sections
\subsection{x86}

% subsections:
\input{patterns/03_printf/x86/x86}
\input{patterns/03_printf/x86/x64}

\ifdefined\IncludeARM
\subsection{ARM}

\EN{\input{patterns/03_printf/ARM/ARM3_EN}}
\RU{\input{patterns/03_printf/ARM/ARM3_RU}}
\FR{\input{patterns/03_printf/ARM/ARM3_FR}}
\ITA{\input{patterns/03_printf/ARM/ARM3_ITA}}

\EN{\input{patterns/03_printf/ARM/ARM8_EN}}
\RU{\input{patterns/03_printf/ARM/ARM8_RU}}
\FR{\input{patterns/03_printf/ARM/ARM8_FR}}
\ITA{\input{patterns/03_printf/ARM/ARM8_ITA}}

\fi
\ifdefined\IncludeMIPS
\section{MIPS}

\subsection{3 \RU{аргумента}\EN{arguments}}

\subsubsection{\Optimizing GCC 4.4.5}

\RU{Главное отличие от примера ``\HelloWorldSectionName'' в том, что здесь на самом деле
вызывается \printf вместо \puts и еще три аргумента передаются в регистрах}
\EN{One main difference with ``\HelloWorldSectionName'' example is that \printf is actually called here
instead of \puts and 3 more arguments are passed in registers} \$5 \dots \$7 (\OrENRU \$A0 \dots \$A2).

\RU{Вот почему эти регистры имеют префикс A-, это значит, что они используются для передачи аргументов.}
\EN{So that's why these registers are prefixed with A-, meaning they are used for function arguments passing.}

\lstinputlisting[caption=\Optimizing GCC 4.4.5 (\assemblyOutput)]{patterns/03_printf/MIPS/printf3.O3.s.\LANG}

\lstinputlisting[caption=\Optimizing GCC 4.4.5 (IDA)]{patterns/03_printf/MIPS/printf3.O3.IDA.lst.\LANG}

\subsubsection{\NonOptimizing GCC 4.4.5}

\NonOptimizing GCC \RU{более многословен}\EN{is more verbose}:

\lstinputlisting[caption=\NonOptimizing GCC 4.4.5 (\assemblyOutput)]{patterns/03_printf/MIPS/printf3.O0.s.\LANG}

\lstinputlisting[caption=\NonOptimizing GCC 4.4.5 (IDA)]{patterns/03_printf/MIPS/printf3.O0.IDA.lst.\LANG}

\subsection{8 \RU{аргументов}\EN{arguments}}

\RU{Снова воспользуемся примером с 9-ю аргументами из предыдущей секции}\EN{Let's use again the example
with 9 arguments from the previous section}: \ref{example_printf8_x64}.

\lstinputlisting{patterns/03_printf/2.c}

\subsubsection{\Optimizing GCC 4.4.5}

\RU{Только 4 первых аргумента передаются в регистрах \$A0 \dots \$A3, так что остальные передаются 
через стек.}
\EN{Only 4 first arguments are passed in \$A0 \dots \$A3 registers, so others are passed via stack.}
\index{MIPS!O32}
\RU{Это соглашение о вызовах O32 (которое самое популярное в мире MIPS).
Другие соглашения о вызовах (как N32) могут наделять регистры другими ф-циями.}
\EN{That is O32 calling convention (which is most used and popular in MIPS world).
Other calling conventions (like N32) may have differ register purposes.}

\index{MIPS!\Instructions!SW}
\RU{SW означает ``Store Word'' (записать слово, из регистра в память).}
\EN{SW meaning ``Store Word'' (from register to memory).}
\RU{В MIPS нет инструкции для записи значения в память, так что для этого используется пара инструкций (LI/SW).}
\EN{MIPS lacks instruction for storing a value into memory, so instruction pair is to be used instead (LI/SW).}

\lstinputlisting[caption=\Optimizing GCC 4.4.5 (\assemblyOutput)]{patterns/03_printf/MIPS/printf8.O3.s.\LANG}

\lstinputlisting[caption=\Optimizing GCC 4.4.5 (IDA)]{patterns/03_printf/MIPS/printf8.O3.IDA.lst.\LANG}

\subsubsection{\NonOptimizing GCC 4.4.5}

\NonOptimizing GCC \RU{более многословен}\EN{is more verbose}:

\lstinputlisting[caption=\NonOptimizing GCC 4.4.5 (\assemblyOutput)]{patterns/03_printf/MIPS/printf8.O0.s.\LANG}

\lstinputlisting[caption=\NonOptimizing GCC 4.4.5 (IDA)]{patterns/03_printf/MIPS/printf8.O0.IDA.lst.\LANG}

\fi

\section{\Conclusion{}}

Вот примерный скелет вызова функции:

\lstinputlisting[caption=x86]{patterns/03_printf/skel1.lst.\LANG}

\lstinputlisting[caption=x64 (MSVC)]{patterns/03_printf/skel2.lst.\LANG}

\ifdefined\IncludeGCC
\lstinputlisting[caption=x64 (GCC)]{patterns/03_printf/skel3.lst.\LANG}
\fi

\ifdefined\IncludeARM
\lstinputlisting[caption=ARM]{patterns/03_printf/skel4.lst.\LANG}

\lstinputlisting[caption=ARM64]{patterns/03_printf/skel5.lst.\LANG}
\fi

\ifdefined\IncludeMIPS
\index{MIPS!O32}
\lstinputlisting[caption=MIPS (соглашение о вызовах O32)]{patterns/03_printf/skel_MIPS.lst.\LANG}
\fi

\section{Кстати}

\index{fastcall}
Кстати, разница между способом передачи параметров принятая в x86, x64, fastcall, ARM и MIPS неплохо иллюстрирует тот важный момент, что процессору, в общем, всё равно, как будут 
передаваться параметры функций. Можно создать гипотетический компилятор, который будет передавать их при 
помощи указателя на структуру с параметрами, не пользуясь стеком вообще.

\ifdefined\IncludeMIPS
\index{MIPS!O32}
Регистры \$A0\dots \$A3 в MIPS так названы только для удобства (это соглашение о вызовах O32).
Программисты могут использовать любые другие регистры (может быть, только кроме \$ZERO) для
передачи данных или любое другое соглашение о вызовах.
\fi

\ac{CPU} не знает о соглашениях о вызовах вообще.

Можно также вспомнить, что начинающие программисты на ассемблере передают параметры 
в другие функции обычно через регистры, без всякого явного порядка, или даже через глобальные переменные.
И всё это нормально работает.

}
\PTBR{\chapter{\PrintfSeveralArgumentsSectionName}

Agora vamos extender o nosso exemplo \IT{\HelloWorldSectionName}~(\myref{sec:helloworld}),
trocando \printf no corpo da função \main() por isso:

\lstinputlisting[label=hw_c]{patterns/03_printf/1.c}

% sections
\subsection{x86}

% subsections:
\input{patterns/03_printf/x86/x86}
\input{patterns/03_printf/x86/x64}

\ifdefined\IncludeARM
\subsection{ARM}

\EN{\input{patterns/03_printf/ARM/ARM3_EN}}
\RU{\input{patterns/03_printf/ARM/ARM3_RU}}
\FR{\input{patterns/03_printf/ARM/ARM3_FR}}
\ITA{\input{patterns/03_printf/ARM/ARM3_ITA}}

\EN{\input{patterns/03_printf/ARM/ARM8_EN}}
\RU{\input{patterns/03_printf/ARM/ARM8_RU}}
\FR{\input{patterns/03_printf/ARM/ARM8_FR}}
\ITA{\input{patterns/03_printf/ARM/ARM8_ITA}}

\fi
\ifdefined\IncludeMIPS
\section{MIPS}

\subsection{3 \RU{аргумента}\EN{arguments}}

\subsubsection{\Optimizing GCC 4.4.5}

\RU{Главное отличие от примера ``\HelloWorldSectionName'' в том, что здесь на самом деле
вызывается \printf вместо \puts и еще три аргумента передаются в регистрах}
\EN{One main difference with ``\HelloWorldSectionName'' example is that \printf is actually called here
instead of \puts and 3 more arguments are passed in registers} \$5 \dots \$7 (\OrENRU \$A0 \dots \$A2).

\RU{Вот почему эти регистры имеют префикс A-, это значит, что они используются для передачи аргументов.}
\EN{So that's why these registers are prefixed with A-, meaning they are used for function arguments passing.}

\lstinputlisting[caption=\Optimizing GCC 4.4.5 (\assemblyOutput)]{patterns/03_printf/MIPS/printf3.O3.s.\LANG}

\lstinputlisting[caption=\Optimizing GCC 4.4.5 (IDA)]{patterns/03_printf/MIPS/printf3.O3.IDA.lst.\LANG}

\subsubsection{\NonOptimizing GCC 4.4.5}

\NonOptimizing GCC \RU{более многословен}\EN{is more verbose}:

\lstinputlisting[caption=\NonOptimizing GCC 4.4.5 (\assemblyOutput)]{patterns/03_printf/MIPS/printf3.O0.s.\LANG}

\lstinputlisting[caption=\NonOptimizing GCC 4.4.5 (IDA)]{patterns/03_printf/MIPS/printf3.O0.IDA.lst.\LANG}

\subsection{8 \RU{аргументов}\EN{arguments}}

\RU{Снова воспользуемся примером с 9-ю аргументами из предыдущей секции}\EN{Let's use again the example
with 9 arguments from the previous section}: \ref{example_printf8_x64}.

\lstinputlisting{patterns/03_printf/2.c}

\subsubsection{\Optimizing GCC 4.4.5}

\RU{Только 4 первых аргумента передаются в регистрах \$A0 \dots \$A3, так что остальные передаются 
через стек.}
\EN{Only 4 first arguments are passed in \$A0 \dots \$A3 registers, so others are passed via stack.}
\index{MIPS!O32}
\RU{Это соглашение о вызовах O32 (которое самое популярное в мире MIPS).
Другие соглашения о вызовах (как N32) могут наделять регистры другими ф-циями.}
\EN{That is O32 calling convention (which is most used and popular in MIPS world).
Other calling conventions (like N32) may have differ register purposes.}

\index{MIPS!\Instructions!SW}
\RU{SW означает ``Store Word'' (записать слово, из регистра в память).}
\EN{SW meaning ``Store Word'' (from register to memory).}
\RU{В MIPS нет инструкции для записи значения в память, так что для этого используется пара инструкций (LI/SW).}
\EN{MIPS lacks instruction for storing a value into memory, so instruction pair is to be used instead (LI/SW).}

\lstinputlisting[caption=\Optimizing GCC 4.4.5 (\assemblyOutput)]{patterns/03_printf/MIPS/printf8.O3.s.\LANG}

\lstinputlisting[caption=\Optimizing GCC 4.4.5 (IDA)]{patterns/03_printf/MIPS/printf8.O3.IDA.lst.\LANG}

\subsubsection{\NonOptimizing GCC 4.4.5}

\NonOptimizing GCC \RU{более многословен}\EN{is more verbose}:

\lstinputlisting[caption=\NonOptimizing GCC 4.4.5 (\assemblyOutput)]{patterns/03_printf/MIPS/printf8.O0.s.\LANG}

\lstinputlisting[caption=\NonOptimizing GCC 4.4.5 (IDA)]{patterns/03_printf/MIPS/printf8.O0.IDA.lst.\LANG}

\fi

\section{\Conclusion{}}

Aqui está uma estrutura bem rústica da chamada da função

\lstinputlisting[caption=x86]{patterns/03_printf/skel1.lst.\LANG}

\lstinputlisting[caption=x64 (MSVC)]{patterns/03_printf/skel2.lst.\LANG}

\ifdefined\IncludeGCC
\lstinputlisting[caption=x64 (GCC)]{patterns/03_printf/skel3.lst.\LANG}
\fi

\ifdefined\IncludeARM
\lstinputlisting[caption=ARM]{patterns/03_printf/skel4.lst.\LANG}

\lstinputlisting[caption=ARM64]{patterns/03_printf/skel5.lst.\LANG}
\fi

\ifdefined\IncludeMIPS
\index{MIPS!O32}
\lstinputlisting[caption=MIPS (\PTBRph{})]{patterns/03_printf/skel_MIPS.lst.\LANG}
\fi

\section{A propósito}

\index{fastcall}
A propósito, a diferença entre os argumentos passados em x86, x64, fastcall, ARM e MIPS  é uma boa demonstração do fato de como a CPU é indiferente sobre como os argumentos são passados para as funções.
Também é possível criar um compilador hipotético capaz de passar argumentos por alguma outra estrutura especial sem usar a pilha de nenhuma maneira.

\ifdefined\IncludeMIPS
\index{MIPS!O32}
\PTBRph{}
\fi

A \ac{CPU} não está ciente de convenções de chamada de funções.

Agora nós podemos também relembrar de dos programadores novatos de assembly passando argumentos para outras funções:
geralmente via registradores, sem nenhuma sequência explícita, ou mesmo por variáveis globais. Logicamente, também funciona.

}
\ITA{\chapter{\PrintfSeveralArgumentsSectionName}

Estendiamo l'esempio \IT{\HelloWorldSectionName}~(\myref{sec:helloworld}) modificando la chiamata a  \printf nella funzione \main:

\lstinputlisting[label=hw_c]{patterns/03_printf/1.c}

% sections
\subsection{x86}

% subsections:
\input{patterns/03_printf/x86/x86}
\input{patterns/03_printf/x86/x64}

\ifdefined\IncludeARM
\subsection{ARM}

\EN{\input{patterns/03_printf/ARM/ARM3_EN}}
\RU{\input{patterns/03_printf/ARM/ARM3_RU}}
\FR{\input{patterns/03_printf/ARM/ARM3_FR}}
\ITA{\input{patterns/03_printf/ARM/ARM3_ITA}}

\EN{\input{patterns/03_printf/ARM/ARM8_EN}}
\RU{\input{patterns/03_printf/ARM/ARM8_RU}}
\FR{\input{patterns/03_printf/ARM/ARM8_FR}}
\ITA{\input{patterns/03_printf/ARM/ARM8_ITA}}

\fi
\ifdefined\IncludeMIPS
\section{MIPS}

\subsection{3 \RU{аргумента}\EN{arguments}}

\subsubsection{\Optimizing GCC 4.4.5}

\RU{Главное отличие от примера ``\HelloWorldSectionName'' в том, что здесь на самом деле
вызывается \printf вместо \puts и еще три аргумента передаются в регистрах}
\EN{One main difference with ``\HelloWorldSectionName'' example is that \printf is actually called here
instead of \puts and 3 more arguments are passed in registers} \$5 \dots \$7 (\OrENRU \$A0 \dots \$A2).

\RU{Вот почему эти регистры имеют префикс A-, это значит, что они используются для передачи аргументов.}
\EN{So that's why these registers are prefixed with A-, meaning they are used for function arguments passing.}

\lstinputlisting[caption=\Optimizing GCC 4.4.5 (\assemblyOutput)]{patterns/03_printf/MIPS/printf3.O3.s.\LANG}

\lstinputlisting[caption=\Optimizing GCC 4.4.5 (IDA)]{patterns/03_printf/MIPS/printf3.O3.IDA.lst.\LANG}

\subsubsection{\NonOptimizing GCC 4.4.5}

\NonOptimizing GCC \RU{более многословен}\EN{is more verbose}:

\lstinputlisting[caption=\NonOptimizing GCC 4.4.5 (\assemblyOutput)]{patterns/03_printf/MIPS/printf3.O0.s.\LANG}

\lstinputlisting[caption=\NonOptimizing GCC 4.4.5 (IDA)]{patterns/03_printf/MIPS/printf3.O0.IDA.lst.\LANG}

\subsection{8 \RU{аргументов}\EN{arguments}}

\RU{Снова воспользуемся примером с 9-ю аргументами из предыдущей секции}\EN{Let's use again the example
with 9 arguments from the previous section}: \ref{example_printf8_x64}.

\lstinputlisting{patterns/03_printf/2.c}

\subsubsection{\Optimizing GCC 4.4.5}

\RU{Только 4 первых аргумента передаются в регистрах \$A0 \dots \$A3, так что остальные передаются 
через стек.}
\EN{Only 4 first arguments are passed in \$A0 \dots \$A3 registers, so others are passed via stack.}
\index{MIPS!O32}
\RU{Это соглашение о вызовах O32 (которое самое популярное в мире MIPS).
Другие соглашения о вызовах (как N32) могут наделять регистры другими ф-циями.}
\EN{That is O32 calling convention (which is most used and popular in MIPS world).
Other calling conventions (like N32) may have differ register purposes.}

\index{MIPS!\Instructions!SW}
\RU{SW означает ``Store Word'' (записать слово, из регистра в память).}
\EN{SW meaning ``Store Word'' (from register to memory).}
\RU{В MIPS нет инструкции для записи значения в память, так что для этого используется пара инструкций (LI/SW).}
\EN{MIPS lacks instruction for storing a value into memory, so instruction pair is to be used instead (LI/SW).}

\lstinputlisting[caption=\Optimizing GCC 4.4.5 (\assemblyOutput)]{patterns/03_printf/MIPS/printf8.O3.s.\LANG}

\lstinputlisting[caption=\Optimizing GCC 4.4.5 (IDA)]{patterns/03_printf/MIPS/printf8.O3.IDA.lst.\LANG}

\subsubsection{\NonOptimizing GCC 4.4.5}

\NonOptimizing GCC \RU{более многословен}\EN{is more verbose}:

\lstinputlisting[caption=\NonOptimizing GCC 4.4.5 (\assemblyOutput)]{patterns/03_printf/MIPS/printf8.O0.s.\LANG}

\lstinputlisting[caption=\NonOptimizing GCC 4.4.5 (IDA)]{patterns/03_printf/MIPS/printf8.O0.IDA.lst.\LANG}

\fi

\section{\Conclusion{}}

Si riporta di seguito una lista di bozze di chiamate alla call:

\lstinputlisting[caption=x86]{patterns/03_printf/skel1.lst.\LANG}

\lstinputlisting[caption=x64 (MSVC)]{patterns/03_printf/skel2.lst.\LANG}

\ifdefined\IncludeGCC
\lstinputlisting[caption=x64 (GCC)]{patterns/03_printf/skel3.lst.\LANG}
\fi

\ifdefined\IncludeARM
\lstinputlisting[caption=ARM]{patterns/03_printf/skel4.lst.\LANG}

\lstinputlisting[caption=ARM64]{patterns/03_printf/skel5.lst.\LANG}
\fi

\ifdefined\IncludeMIPS
\myindex{MIPS!O32}
\lstinputlisting[caption=MIPS (O32 calling convention)]{patterns/03_printf/skel_MIPS.lst.\LANG}
\fi

\section{A proposito...}

\myindex{fastcall}
Le differenze negli approcci utilizzati per il passaggio di argomenti in x86, x64, 
fastcall, ARM and MIPS e' un'ottima dimostrazione del fatto che la CPU e' inconsapevole di come gli argomenti vengono passati alle funzioni. 
Sarebbe anche possibile creare un compilatore ipotetico in grado di passare gli argomenti attraverso una struttura speciale, senza usare lo stack.

\ifdefined\IncludeMIPS
\myindex{MIPS!O32}
I registri MIPS \$A0 \dots \$A3 sono indicati in questo modo soltanto per convenienza (cioe' nella O32 calling convention).
I programmatori possono usare qualunque altro registro (tranne \$ZERO) per passare i dati, o utilizzare qualunque altra calling convention. 
\fi

La \ac{CPU} non e' assolutamente consapevole delle calling conventions.

Possiamo anche ricordare come i programmatori principianti in assembly passano gli argomenti alle altre funzioni: 
di suolito tramite i registri, senza un ordine esplicitop, o attraverso variabili globali.
Questi approcci sono ovviamente validi e funzionanti.
}
\DE{\section{\PrintfSeveralArgumentsSectionName}

An dieser Stelle wird das \IT{\HelloWorldSectionName}~(\myref{sec:helloworld})-Beispiel ein
wenig erweitert, indem \printf in der \main-Funktion durch folgendes ersetzt wird:

\lstinputlisting[label=hw_c,style=customc]{patterns/03_printf/1.c}

% sections
\subsection{x86}

% subsections:
\input{patterns/03_printf/x86/x86}
\input{patterns/03_printf/x86/x64}

\subsection{ARM}

\EN{\input{patterns/03_printf/ARM/ARM3_EN}}
\RU{\input{patterns/03_printf/ARM/ARM3_RU}}
\FR{\input{patterns/03_printf/ARM/ARM3_FR}}
\ITA{\input{patterns/03_printf/ARM/ARM3_ITA}}

\EN{\input{patterns/03_printf/ARM/ARM8_EN}}
\RU{\input{patterns/03_printf/ARM/ARM8_RU}}
\FR{\input{patterns/03_printf/ARM/ARM8_FR}}
\ITA{\input{patterns/03_printf/ARM/ARM8_ITA}}

\section{MIPS}

\subsection{3 \RU{аргумента}\EN{arguments}}

\subsubsection{\Optimizing GCC 4.4.5}

\RU{Главное отличие от примера ``\HelloWorldSectionName'' в том, что здесь на самом деле
вызывается \printf вместо \puts и еще три аргумента передаются в регистрах}
\EN{One main difference with ``\HelloWorldSectionName'' example is that \printf is actually called here
instead of \puts and 3 more arguments are passed in registers} \$5 \dots \$7 (\OrENRU \$A0 \dots \$A2).

\RU{Вот почему эти регистры имеют префикс A-, это значит, что они используются для передачи аргументов.}
\EN{So that's why these registers are prefixed with A-, meaning they are used for function arguments passing.}

\lstinputlisting[caption=\Optimizing GCC 4.4.5 (\assemblyOutput)]{patterns/03_printf/MIPS/printf3.O3.s.\LANG}

\lstinputlisting[caption=\Optimizing GCC 4.4.5 (IDA)]{patterns/03_printf/MIPS/printf3.O3.IDA.lst.\LANG}

\subsubsection{\NonOptimizing GCC 4.4.5}

\NonOptimizing GCC \RU{более многословен}\EN{is more verbose}:

\lstinputlisting[caption=\NonOptimizing GCC 4.4.5 (\assemblyOutput)]{patterns/03_printf/MIPS/printf3.O0.s.\LANG}

\lstinputlisting[caption=\NonOptimizing GCC 4.4.5 (IDA)]{patterns/03_printf/MIPS/printf3.O0.IDA.lst.\LANG}

\subsection{8 \RU{аргументов}\EN{arguments}}

\RU{Снова воспользуемся примером с 9-ю аргументами из предыдущей секции}\EN{Let's use again the example
with 9 arguments from the previous section}: \ref{example_printf8_x64}.

\lstinputlisting{patterns/03_printf/2.c}

\subsubsection{\Optimizing GCC 4.4.5}

\RU{Только 4 первых аргумента передаются в регистрах \$A0 \dots \$A3, так что остальные передаются 
через стек.}
\EN{Only 4 first arguments are passed in \$A0 \dots \$A3 registers, so others are passed via stack.}
\index{MIPS!O32}
\RU{Это соглашение о вызовах O32 (которое самое популярное в мире MIPS).
Другие соглашения о вызовах (как N32) могут наделять регистры другими ф-циями.}
\EN{That is O32 calling convention (which is most used and popular in MIPS world).
Other calling conventions (like N32) may have differ register purposes.}

\index{MIPS!\Instructions!SW}
\RU{SW означает ``Store Word'' (записать слово, из регистра в память).}
\EN{SW meaning ``Store Word'' (from register to memory).}
\RU{В MIPS нет инструкции для записи значения в память, так что для этого используется пара инструкций (LI/SW).}
\EN{MIPS lacks instruction for storing a value into memory, so instruction pair is to be used instead (LI/SW).}

\lstinputlisting[caption=\Optimizing GCC 4.4.5 (\assemblyOutput)]{patterns/03_printf/MIPS/printf8.O3.s.\LANG}

\lstinputlisting[caption=\Optimizing GCC 4.4.5 (IDA)]{patterns/03_printf/MIPS/printf8.O3.IDA.lst.\LANG}

\subsubsection{\NonOptimizing GCC 4.4.5}

\NonOptimizing GCC \RU{более многословен}\EN{is more verbose}:

\lstinputlisting[caption=\NonOptimizing GCC 4.4.5 (\assemblyOutput)]{patterns/03_printf/MIPS/printf8.O0.s.\LANG}

\lstinputlisting[caption=\NonOptimizing GCC 4.4.5 (IDA)]{patterns/03_printf/MIPS/printf8.O0.IDA.lst.\LANG}


\subsection{\Conclusion{}}

Hier ist der grobe Aufbau der Aufruffunktion:

\begin{lstlisting}[caption=x86,style=customasmx86]
...
PUSH Drittes Argument
PUSH Zweites Argument
PUSH Erstes Argument
CALL Funktion
; gegebenenfalls den Stackpointer modifizieren
\end{lstlisting}

\begin{lstlisting}[caption=x64 (MSVC),style=customasmx86]
MOV RCX, Erstes Argument
MOV RDX, Zweites Argument
MOV R8, Drittes Argument
MOV R9, Viertes Argument
...
PUSH fünftes, sechstes Argument, usw. (falls notwendig)
CALL Funktion
; gegebenenfalls den Stackpointer modifizieren
\end{lstlisting}

\begin{lstlisting}[caption=x64 (GCC),style=customasmx86]
MOV RDI, Erstes Argument
MOV RSI, Zweites Argument
MOV RDX, Drittes Argument
MOV RCX, Viertes Argument
MOV R8, Fünftes Argument
MOV R9, Sechstes Argument
...
PUSH Siebtes, Achtes Argument, usw. (falls notwendig)
CALL Funktion
; gegebenenfalls den Stackpointer modifizieren
\end{lstlisting}

\begin{lstlisting}[caption=ARM,style=customasmARM]
MOV R0, Erstes Argument
MOV R1, Zweites Argument
MOV R2, Drittes Argument
MOV R3, Viertes Argument
; Fünftes, Sechstes Argument, usw. auf den Stack (falls notwendig)
BL Funktion
; gegebenenfalls den Stackpointer modifizieren
\end{lstlisting}

\begin{lstlisting}[caption=ARM64,style=customasmARM]
MOV X0, Erstes Argument
MOV X1, Zweites Argument
MOV X2, Drittes Argument
MOV X3, Viertes Argument
MOV X4, Fünftes Argument
MOV X5, Sechstes Argument
MOV X6, Siebtes Argument
MOV X7, Achtes Argument
; Neuntes, Zehntes Argument, usw. auf den Stack (falls notwendig)
BL Funktion
; gegebenenfalls den Stackpointer modifizieren
\end{lstlisting}

\myindex{MIPS!O32}
\begin{lstlisting}[caption=MIPS (O32 calling convention),style=customasmMIPS]
LI $4, Erstes argument ; AKA $A0
LI $5, Zweites argument ; AKA $A1
LI $6, Drittes argument ; AKA $A2
LI $7, Viertes argument ; AKA $A3
; pass Fünftes, Sechstes argument, usw. auf den Stack (falls notwendig)
LW temporäres Register, Adresse der Funktion
JALR temporäres Regist
\end{lstlisting}

\subsection{Übrigens\dots{}}

\myindex{fastcall}
Übrigens ist der Unterschied der Art der Argumenten Übergabe in x86, x64, fastcall, ARM und MIPS eine gute
Darstellung der Tatsache, dass die CPU nicht weiß wie die Argumente an die Funktion übergeben werden.
Es ist auch möglich einen hypothetischen Compiler zu erstellen, der die Möglichkeit hat Argumente mittels
einer speziellen Struktur, ohne den Stack an die Funktionen zu übergebe.

\myindex{MIPS!O32}
MIPS \$A0 \dots \$A3-Register sind aus Bequemlichkeitsgründen auf diese Weise beschriftet (O32 Aufrufkonvention).
Programmierer können auch andere Register (vielleicht außer \$ZERO) nutzen um Daten zu übergeben
oder eine andere Aufrufkonvention zu nutzen.

Die \ac{CPU} hatte jedoch keinerlei Kenntnisse über die Aufrufkonvention.

Man sieht hier auch wie Neulinge der Assemblersprache Argumente an andere Funktionen übergeben:
in der Regel per Register ohne explizite Reihenfolge oder globale Variablen.
Natürlich funktioniert das ebenso gut.
}

\section{scanf()}
\myindex{\CStandardLibrary!scanf()}
\label{label_scanf}

\RU{Теперь попробуем использовать scanf().}%
\EN{Now let's use scanf().}%
\PTBR{Agora vamos usar a função scanf().}%
\FR{Maintenant utilisons la fonction scanf().}

% subsections
\EN{\section{Simple example}

\lstinputlisting{patterns/04_scanf/1_simple/ex1.c}

It's not clever to use \scanf for user interactions nowadays. 
But we can, however, illustrate passing a pointer to a variable of type \Tint.

\subsection{About pointers}
\index{\CLanguageElements!\Pointers}

Pointers are one of the fundamental concepts in computer science.
Often, passing a large array, structure or object as an argument to another function is too expensive, while passing their address is much cheaper.
In addition if the \gls{callee} function needs to modify something in the large array or structure received as a parameter and return back the entire structure then the situation is close to absurd.
So the simplest thing to do is to pass the address of the array or structure to the \gls{callee} function, and let it change what needs to be changed.

A pointer in \CCpp---is simply an address of some memory location.

\index{x86-64}
In x86, the address is represented as a 32-bit number (i.e., it occupies 4 bytes), while in x86-64 it is a 64-bit number (occupying 8 bytes).
By the way, that is the reason behind some people's indignation related to switching to x86-64\EMDASH{}all pointers in the x64-architecture require twice as much space, including cache memory, which is ``expensive'' memory.

% TODO ... а делать разные версии memcpy для разных типов - абсурд
\index{\CStandardLibrary!memcpy()}
It is possible to work with untyped pointers only, given some effort; e.g. the standard C function \TT{memcpy()}, that copies a block from one memory location to another,
takes 2 pointers of type \TT{void*} as arguments, since it is impossible to predict the type of the data you would like to copy. Data types are not important, only the block size matters.

Pointers are also widely used when a function needs to return more than one value
(we are going to get back to this later
\ifx\LITE\undefined
~(\myref{label_pointers})
\fi
).

\IT{scanf()} function---is such a case.

Besides the fact that the function needs to indicate how many values were successfully read, it also needs to return all these values.

In \CCpp the pointer type is only needed for compile-time type checking.

Internally, in the compiled code there is no information about pointer types at all.
% TODO это сильно затрудняет декомпиляцию

\input{patterns/04_scanf/1_simple/x86}
\input{patterns/04_scanf/1_simple/x64}

\ifdefined\IncludeARM
\input{patterns/04_scanf/1_simple/ARM}
\fi
\ifdefined\IncludeMIPS
\input{patterns/04_scanf/1_simple/MIPS/main}
\fi
}
\RU{\subsection{Простой пример}

\lstinputlisting[style=customc]{patterns/04_scanf/1_simple/ex1.c}

Использовать \scanf в наши времена для того, чтобы спросить у пользователя что-то --- не самая хорошая идея.
Но так мы проиллюстрируем передачу указателя на переменную типа \Tint.

\subsubsection{Об указателях}
\myindex{\CLanguageElements!\Pointers}

Это одна из фундаментальных вещей в информатике.
Часто большой массив, структуру или объект передавать в другую функцию путем копирования данных невыгодно, а передать адрес массива, структуры или объекта куда проще.
Например, если вы собираетесь вывести в консоль текстовую строку, достаточно только передать её адрес в ядро \ac{OS}.

К тому же, если вызываемая функция (\gls{callee}) должна изменить что-то в этом большом массиве или структуре, то возвращать её полностью так же абсурдно.
Так что самое простое, что можно сделать, это передать в функцию-\gls{callee} адрес массива или структуры, и пусть \gls{callee} что-то там изменит.

Указатель в \CCpp --- это просто адрес какого-либо места в памяти.

\myindex{x86-64}
В x86 адрес представляется в виде 32-битного числа (т.е. занимает 4 байта), а в x86-64 как 64-битное число (занимает 8 байт).
Кстати, отсюда негодование некоторых людей, связанное с переходом на x86-64 --- на этой архитектуре все указатели занимают в 2 раза больше места, в том числе и в ``дорогой'' кэш-памяти.

% TODO ... а делать разные версии memcpy для разных типов - абсурд
\myindex{\CStandardLibrary!memcpy()}
При некотором упорстве можно работать только с безтиповыми указателями (\TT{void*}), например, стандартная функция Си \TT{memcpy()},
копирующая блок из одного места памяти в другое принимает на вход 2 указателя типа \TT{void*}, потому что нельзя заранее предугадать, какого типа блок вы собираетесь копировать.
Для копирования тип данных не важен, важен только размер блока.

Также указатели широко используются, когда функции нужно вернуть более одного значения
(мы ещё вернемся к этому в будущем
~(\myref{label_pointers})
).

Функция \IT{scanf()}---это как раз такой случай.

Помимо того, что этой функции нужно показать, сколько значений было прочитано успешно, ей ещё и нужно вернуть сами значения.

Тип указателя в \CCpp нужен только для проверки типов на стадии компиляции.

Внутри, в скомпилированном коде, никакой информации о типах указателей нет вообще.
% TODO это сильно затрудняет декомпиляцию

\input{patterns/04_scanf/1_simple/x86}
\input{patterns/04_scanf/1_simple/x64}
\input{patterns/04_scanf/1_simple/ARM}
\input{patterns/04_scanf/1_simple/MIPS/main}
}
\PTBR{% TODO resync with EN version
\subsection{Exemplo simples}

\lstinputlisting[style=customc]{patterns/04_scanf/1_simple/ex1.c}

Não é muito inteligente usar scanf() para interações com o usuário nos dias de hoje.
Mas nós podemos, de qualquer maneira, ilustrar passando um ponteiro para uma variável do tipo \Tint.

\subsubsection{Sobre ponteiros}
\myindex{\CLanguageElements!\Pointers}

Ponteiros são um dos conceitos mais fundamentais na ciência da computação.
Com frequência, passar um array grande, estrutura ou objeto como um argumento para outra função é muito custoso, enquanto passar o endereço de onde ele está é bem mais rápido e gasta menos recursos.
Ainda mais se a função chamada precisa modificar alguma coisa em um array grande ou estrutura recebida como parâmetro e retornar de volta a estrutura inteira se torna perto de absurdo fazer dessa maneira.
Então a coisa mais simples a se fazer é passar o endereço do array ou estrutura para a função chamada e deixar ela fazer as mudanças necessárias.

Um ponteiro em \CCpp é somente um endereço de alguma localização de memória.

\myindex{x86-64}
Em x86, o endereço é representado como um número de 32-bits (ele ocupa 4 bytes), enquanto no x86-64 é um número de 64-bits (ocupando 8 bytes).
A propósito, essa é a razão da indignação de algumas pessoas em relação a trocar para x86-64 todos os ponteiros na arquitetura x64, exigindo o dobro de espaço, incluindo memória cache, que é um lugar ``caro''.

% TODO ... а делать разные версии memcpy для разных типов - абсурд
\myindex{\CStandardLibrary!memcpy()}
É possível ainda se trabalhar com ponteiros sem tipos, como a função padrão em C \TT{memcpy()}, que copia um block de uma localização de memória para outro,
ela recebe como argumento dois ponteiros do tipo void*, uma vez que é impossível de se prever o tipo de informação que você gostaria de copiar.
Tipos não são importantes, só o tamanho do bloco de memória é que importa.

Ponteiros são também largamente usados quando uma função precisa retornar mais de um valor
(nós vamos voltar nisso depois)
~(\myref{label_pointers})
).

\IT{scanf()} é um desses casos.

Além do fato de que a função precisa indicar quantos valores foram lidos com sucesso, ela também precisa retornar todos esses valores.

Em \CCpp os tipos dos ponteiros só são necessários para checagem em tempo de compilação.

Internamente, no código compilado não tem nenhuma informação sobre os tipos de cada ponteiro.
% TODO это сильно затрудняет декомпиляцию

\input{patterns/04_scanf/1_simple/x86}
\input{patterns/04_scanf/1_simple/x64}
\input{patterns/04_scanf/1_simple/ARM}
\input{patterns/04_scanf/1_simple/MIPS/main}
}
\ITA{% TODO resync with EN version
\subsection{Simple example}

\lstinputlisting[style=customc]{patterns/04_scanf/1_simple/ex1.c}

Oggi non e' piu' conveniente usare \scanf per interagire con l'utente. 
Possiamo pero' utilizzarla per illustrare il passaggio di un puntatore ad una variabile di tipo \Tint.

\subsubsection{Puntatori}
\myindex{\CLanguageElements!\Pointers}

I puntatori sono fra i concetti fondamentali in informatica.
Spesso il passaggio di una struttura, array o piu' in generale, un oggetto molto grande, e' troppo costoso in termini di memoria, mentre passare il suo indirizzo e' piu' efficace. 
Inoltre se la funzione chiamata (\gls{callee}) necessita di modificare qualcosa nella struttura ricevuta come parametro e successivamente restituirla per intero, la sitauzione si fa ancora piu' inefficiente.
Percio' la cosa piu' semplice da fare e ' passare l'indirizzo della struttura alla funzione chiamata, e lasciare che operi le modifiche necessarie.

Un puntatore in \CCpp--- e' semplicemente un indirizzo di una locazione di mamoria.

\myindex{x86-64}
In x86, l'indirizzo e' rappresentato con un numero a 32-bit (i.e., occupa 4 byte), mentre in x86-64 e' un numero a 64-bit (8 byte).
Per inciso, questo e' il motivo per cui alcune persone si lamentano nel passaggio a x86-64 --- tutti i puntatori in architettura x64 richiedono il doppio dello spazio, inclusa la memoria cache, che e' memoria ``costosa''.

% TODO ... а делать разные версии memcpy для разных типов - абсурд
\myindex{\CStandardLibrary!memcpy()}
E' possibile lavorare soltanto con puntatori senza tipo, con un po' di sforzo. Ad esempio la funzione C standard \TT{memcpy()}, che copia un blocco di memoria da un indirizzo ad un altro, ha come argomenti 2 puntatori di tipo \TT{void*}, poiche' e' impossibile predire il tipo di dati che si vuole copiare. Il tipo di dato non e' importante, conta solo la dimensione del blocco.

I puntatori sono anche moldo usati quando una funzione deve restituire piu' di un valore
(torneremo su questo argomento piu' avanti
~(\myref{label_pointers})
).

la funzione \IT{scanf()} --- e' uno di questi casi.

Oltre al fatto che la funzione necessita di indicare quanti valori sono stati letti con successo, deve anche restituire tutti questi valori.

In \CCpp il tipo del puntatore e' necessario soltanto per i controlli sui tipi a compile-time.

Internamente, nel codice compilato, non vi e' alcuna informazione sui tipi dei puntatori.
% TODO это сильно затрудняет декомпиляцию

\input{patterns/04_scanf/1_simple/x86}
\input{patterns/04_scanf/1_simple/x64}
\input{patterns/04_scanf/1_simple/ARM}
\input{patterns/04_scanf/1_simple/MIPS/main}
}
\DE{\subsection{Ein einfaches Beispiel}

\lstinputlisting[style=customc]{patterns/04_scanf/1_simple/ex1.c}

Es ist nicht ratsam \scanf heutzutage noch für User Interaktionen zu verwenden. Aber dennoch können wir hier die Übergabe eines Pointers an eine Variable vom Typ \Tint betrachten.

\subsubsection{Pointer}
\myindex{\CLanguageElements!\Pointers}
Pointer sind eines der fundamentalen Konzepte in der Informatik. Oft ist das Übergeben eines großen Array, eines Structs oder Objekts als Funktionsargument zu teuer, während die Übergabe der Adresse wesentlich billiger ist. 
Wenn man zum Beispiel einen Textstring auf der Konsole ausgeben möchte, ist es deutlich einfacher, nur dessen Adresse in den Kernel des \ac{OS} zu übergeben.

Wenn die aufgerufene Funktion außerdem das große Array oder Struct verändern muss und das gesamte Object zurückgeben muss, ist die Situation beinahe absurd. 
Das einfachste ist also die Adresse eines Arrays oder Structs an die aufgerufene Funktion zu übergeben und sie dann die notwendigen Veränderungen durchführen zu lassen.

Ein Pointer ist in \CCpp nichts anderes als die Adresse einer Speicherstelle.


\myindex{x86-64}
In x86 wird die Adresse als 32-Bit-Zahl dargestellt, d.h. sie benötigt 4 Byte, während in x86-64 eine Darstellung durch 64 Bit (d.h. 8 Byte) erfolgt.
Dies ist übrigens der Grund dafür, dass einige Leute den Wechsel zu x86-64 ablehnen--alle Pointer in der x64-Architektur erfordern doppelt soviel Speicherplatz, inklusive Speicher in Cache, der ein sehr teurer Speicher ist.

\myindex{\CStandardLibrary!memcpy()}
Es ist möglich lediglich mit untypisierten Pointern zu arbeiten, wenn man ein wenig zusätzlichen Aufwand betreibt; z.B. in der Standard-C-Funktion \TT{memcpy()}, die einen Datenblock von einer Speicherstelle zu einer anderen kopiert, werden zwei Pointer vom Typ \TT{void*} als Argumente verwendet, da es nicht vorhersagbar ist, welchen Datentyp die Funktion kopieren soll. Datentypen sind hier nicht wichtig, entscheidend ist hier nur die Größe des Speicherblocks.

Pointer werden außerdem häufig verwendet, wenn eine Funktion mehr als einen Wert zurückgeben muss. (Darauf kommen wir später in ~(\myref{label_pointers}) zurück.)

Die Funktion \IT{scanf()} ist solch ein Fall: Neben der Tatsache, dass die Funktion angeben muss wie viele Werte erfolgreich gelesen wurden, muss sie auch alle diese Werte zurückliefern.

In \CCpp wird der Pointertyp nur für Typüberprüfungen zur Compilezeit benötigt.

Intern steckt im kompilierten Code keinerlei Information über die Typen der enthaltenen Pointer.

\input{patterns/04_scanf/1_simple/x86}
\input{patterns/04_scanf/1_simple/x64}
\input{patterns/04_scanf/1_simple/ARM}
\input{patterns/04_scanf/1_simple/MIPS/main}

}
\FR{\subsection{Exemple simple}

\lstinputlisting[style=customc]{patterns/04_scanf/1_simple/ex1.c}

Il n'est pas astucieux d'utiliser \scanf pour les interactions utilisateurs de nos jours.
Mais nous pouvons, toutefois, illustrer le passage d'un pointeur sur une variable
de type \Tint.

\subsubsection{Á propos des pointeurs}
\myindex{\CLanguageElements!\Pointers}

Les pointeurs sont l'un des concepts fondamentaux de l'informatique.
Souvent, passer un gros tableau, structure ou objet comme argument à une autre fonction
est trop couteux, tandis que passer leur adresse l'est très peu.
Par exemple, si vous voulez afficher une chaîne de texte sur la console, il est
plus facile de passer son adresse au noyau de l'\ac{OS}.

En plus, si la fonction \glslink{callee}{appelée} doit modifier quelque chose dans
un gros tableau ou structure reçu comme paramètre et renvoyer le tout, la situation
est proche de l'absurde.
Donc, la chose la plus simple est de passer l'adresse du tableau ou de la structure
à la fonction \glslink{callee}{appelée}, et de la laisser changer ce qui doit l'être.

Un pointeur en \CCpp---est simplement une adresse d'un emplacement mémoire quelconque.

\myindex{x86-64}
En x86, l'adresse est représentée par un nombre de 32-bit (i.e., il occupe 4 octets),
tandis qu'en x86-64 c'est un nombre de 64-bit (occupant 8 octets).
Á propos, c'est la cause de l'indignation de certaines personnes concernant le
changement vers x86-64---tous les pointeurs en architecture x64 ont besoin de deux
fois plus de place, incluant la mémoire cache, qui est de la mémoire ``coûteuse''.

\myindex{\CStandardLibrary!memcpy()}
Il est possible de travailler seulement avec des pointeurs non typés, moyennant
quelques efforts; e.g. la fonction C standard \TT{memcpy()}, qui copie un bloc de
mémoire d'un endroit à un autre, prend 2 pointeurs de type \TT{void*} comme arguments,
puisqu'il est impossible de prévoir le type de données qu'il faudra copier. Les types
de données ne sont pas importants, seule la taille du block compte.

Les pointeurs sont aussi couramment utilisés lorsqu'une fonction doit renvoyer plus
d'une valeur (nous reviendrons là-dessus plus tard~(\myref{label_pointers})).

La fonction \IT{scanf()}---en est une telle.

Hormis le fait que la fonction doit indiquer combien de valeurs ont été lues avec
succès, elle doit aussi renvoyer toutes ces valeurs.

En \CCpp le type du pointeur est seulement nécessaire pour la vérification de type
lors de la compilation.

Il n'y a aucune information du tout sur le type des pointeurs à l'intérieur du code compilé.

\input{patterns/04_scanf/1_simple/x86}
\input{patterns/04_scanf/1_simple/x64}
\input{patterns/04_scanf/1_simple/ARM}
\input{patterns/04_scanf/1_simple/MIPS/main}
}

\EN{\input{patterns/04_scanf/error_EN}}
\DE{\subsection{Häufiger Fehler}
Ein häufiger (Tipp)-Fehler besteht darin, den Wert von \IT{x} anstatt eines Pointers auf \IT{x} zu übergeben.

\lstinputlisting[style=customc]{patterns/04_scanf/error.c}

Was geschieht hier?
\IT{x} ist nicht uninitialisiert und enthält Zufallswerte vom lokalen Stack. Wenn \scanf aufgerufen wird, nimmt es den eingegebenen String vom Benutzer, wandelt ihn in eine Zahl um und versucht ihn nach \IT{x} zu schreiben, wobei \IT{x} wie eine Speicheradresse behandelt wird.
Aber hier liegen Zufallswerte vor, sodass \scanf versucht an eine zufällige Speicherstelle zu schreiben. 
Höchstwahrscheinlich wird der Prozess dadurch abstürzen.

\myindex{0xCCCCCCCC}
\myindex{0x0BADF00D}
Bemerkenswert ist, dass manche \ac{CRT}-Bibliotheken im Debug Build gut erkennbare Muster in den gerade reservierten Speicher schreiben, wie z.B. 0xCCCCCCCC oder 0x0BADF00D usw.
In diesem Fall könnte \IT{x} den Wert 0xCCCCCCCC enthalten und \scanf würde versuchen in die Adresse 0xCCCCCCCC zu schreiben. 
Wenn man nun bemerkt, dass irgendein Code im Prozess in die Adresse 0xCCCCCCCC schreiben möchte, weiß man, dass eine uninitialisierte Variable (oder ein Pointer) verwendet werden.
Dies ist besser als wenn der frisch reservierte Speicher einfach gelöscht würde.


}
\FR{\subsection{Erreur courante}

C'est une erreur très courante (et/ou une typo) de passer la valeur de \IT{x} au
lieu d'un pointeur sur \IT{x}:

\lstinputlisting[style=customc]{patterns/04_scanf/error.c}

Donc que se passe-t-il ici?
\IT{x} n'est pas non-initialisée et contient des données aléatoires de la pile
locale.
Lorsque \scanf est appelée, elle prend la chaîne de l'utilisateur, la convertit
en nombre et essaye de l'écrire dans \IT{x}, la considérant comme une adresse en
mémoire.
Mais il s'agit de bruit aléatoire, donc \scanf va essayer d'écrire à une adresse
aléatoire.
Très probablement, le processus va planter.

\myindex{0xCCCCCCCC}
\myindex{0x0BADF00D}
Assez intéressant, certaines bibliothèques \ac{CRT} compilées en debug, mettent
un signe distinctif lors de l'allocation de la mémoire, comme 0xCCCCCCCC ou
0x0BADF00D etc.
Dans ce cas, \IT{x} peut contenir 0xCCCCCCCC, et \scanf va essayer d'écrire à
l'adresse 0xCCCCCCCC.
Et si vous remarquez que quelque chose dans votre processus essaye d'écrire à
l'adresse 0xCCCCCCCC, vous saurez qu'une variable non initialisée (ou un pointeur)
a été utilisée sans initialisation préalable.
C'est mieux que si la mémoire nouvellement allouée est juste mise à zéro.

}
\PL{\subsection{Popularny błąd}

Bardzo popularnym błędem jest podanie jako argumentu zmiennej \IT{x} zamiast wskaźnika na zmienną \IT{x}:

\lstinputlisting[style=customc]{patterns/04_scanf/error.c}

Więc co tutaj się stanie?
\IT{x} jest niezainicjalizowany i zawiera losowe śmieci z lokalnego stosu.
Kiedy funkcja \scanf jest wywoływana, pobiera ciąg znaków od użytkownika, parsuje go do liczby i próbuje go zapisać w \IT{x}, traktując wartość \IT{x} jako adres w pamięci.
Jednak skoro tam są losowe śmieci ze stosu, \scanf będzie próbować uzyskać dostęp do losowego adresu.
Najczęściej wtedy proces zawiesza działanie.

\myindex{0xCCCCCCCC}
\myindex{0x0BADF00D}
Co ciekawe, niektóre biblioteki \ac{CRT}w wersji do debugowania, umieszczają w pamięci, która jest alokowana, widoczne wzory takie jak  0xCCCCCCCC albo 0x0BADF00D itp.
W tym przypadku, \IT{x} może zawierać 0xCCCCCCCC, więc \scanf będzie próbować dokonać zapisu pod adresem 0xCCCCCCCC.
Jeśli program wyświetli komunikat, że gdzieś w procesie wystąpi próba zapisu pod adresem 0xCCCCCCCC, to znaczy, że
została użyta niezainicjalizowana zmienna (lub wskaźnik).
Jest to lepsze rozwiązanie niż gdyby nowo alokowana pamięć była po prostu wyczyszczona.
}
\JPN{\subsection{一般的な間違い}

\IT{x}へのポインタではなく、\IT{x}の値を渡すのは極めて一般的な間違い(および/またはタイプミス)です。

\lstinputlisting[style=customc]{patterns/04_scanf/error.c}

では、何が起こるでしょうか? 
\IT{x}は初期化されておらず、ローカルスタックからのランダムノイズを含んでいます。 
\scanf が呼び出されると、ユーザーから文字列を受け取り、数値に解析し、\IT{x}に書き込んでメモリ内のアドレスとして扱います。 
しかしランダムなノイズがあるので、 \scanf はランダムなアドレスに書き込もうとします。 
おそらく、プロセスがクラッシュするでしょう。

\myindex{0xCCCCCCCC}
\myindex{0x0BADF00D}
興味深いことに、デバッグビルドのいくつかの \ac{CRT} ライブラリは、
視覚的に特徴的なパターンを 0xCCCCCCCC や 0x0BADF00D のように割り当てられたメモリに入れています。 
この場合、\IT{x}は0xCCCCCCCCを含むことができ、\scanf はアドレス0xCCCCCCCCに書き込みを試みます。 
また、プロセス内の何かがアドレス0xCCCCCCCCに書き込もうとすると、
初期化されていない変数(またはポインタ)が事前初期化なしで使用されることがわかります。 
これは、新しく割り当てられたメモリがちょうどクリアされた場合よりも優れています。
}

\ifdefined\ENGLISH
\newcommand{\GlobalVarsSectionName}{Global variables}
\section{\GlobalVarsSectionName}
\index{\GlobalVarsSectionName}
\label{scanf_global_variable}

What if the \TT{x} variable from the previous example was not local but a global one? 
Then it would have been accessible from any point, not only from the function body. 
Global variables are considered \gls{anti-pattern}, but for the sake of the experiment, we could do this.
\fi

\ifdefined\RUSSIAN
\newcommand{\GlobalVarsSectionName}{Глобальные переменные}
\section{\GlobalVarsSectionName}
\index{\GlobalVarsSectionName}
\label{scanf_global_variable}

А что если переменная \TT{x} из предыдущего примера будет глобальной переменной, а не локальной? 
Тогда к ней смогут обращаться из любого другого места, а не только из тела функции. 
Глобальные переменные считаются \glslink{anti-pattern}{анти-паттерном},
но ради примера мы можем себе это позволить.
\fi

\ifdefined\BRAZILIAN
\newcommand{\GlobalVarsSectionName}{Variáveis globais}
\section{\GlobalVarsSectionName}
\index{\GlobalVarsSectionName}
\label{scanf_global_variable}

E se a variável \TT{x} do último exemplo não fosse local, mas sim global?
Então ela teria que ser acessível de qualquer ponto, não somente pelo corpo da função.
Variáveis globais são consideradas maus hábitos, mas pelo bem do experimento, nós faremos isso.
\fi

\lstinputlisting{patterns/04_scanf/2_global/ex2.c.\LANG}

\subsection{MSVC: x86}

\lstinputlisting{patterns/04_scanf/2_global/ex2_MSVC.asm}

\RU{Ничего особенного, в целом. Теперь \TT{x} объявлена в сегменте \TT{\_DATA}. 
Память для нее в стеке более не выделяется.
Все обращения к ней происходит не через стек, а уже напрямую. 
Неинициализированные глобальные переменные не занимают места в исполняемом файле
(и действительно, зачем в исполняемом файле
нужно выделять место под изначально нулевые переменные?), но тогда, когда к этому месту в памяти
кто-то обратится, \ac{OS} подставит туда блок состоящий из нулей\footnote{Так работает \ac{VM}}.}
\EN{Now \TT{x} variable is defined in the \TT{\_DATA} segment. 
Memory in local stack is not allocated anymore. 
All accesses to it are not via stack but directly to process memory. 
Not initialized global variables takes no place in the executable file
(indeed, why we should allocate a place
in the executable file for initially zeroed variables?), but when someone will access this place
in memory, \ac{OS} will allocate a block of zeroes there\footnote{That is how \ac{VM} behaves}.}

\RU{Попробуем изменить объявление этой переменной:}
\EN{Now let's assign value to variable explicitly:}

\lstinputlisting{patterns/04_scanf/2_global/default_value.c.\LANG}

\RU{Выйдет в итоге:}\EN{We got:}

\begin{lstlisting}
_DATA	SEGMENT
_x	DD	0aH

...
\end{lstlisting}

\RU{Здесь уже по месту этой переменной записано \TT{0xA} с типом DD (dword = 32 бита).}
\EN{Here we see value \TT{0xA} of DWORD type (DD meaning DWORD = 32 bit).}

\RU{Если вы откроете скомпилированный .exe-файл в \IDA, то увидите что \IT{x} 
находится аккурат в начале сегмента \TT{\_DATA}, после этой переменной будут текстовые строки.}
\EN{If you will open compiled .exe in \IDA, you will see the \IT{x} variable placed at the beginning of 
the \TT{\_DATA} segment, and after you'll see text strings.}

\RU{А вот если вы откроете в \IDA, .exe скомпилированный в прошлом примере, 
где значение \IT{x} не определено, то в IDA вы увидите:}
\EN{If you will open compiled .exe in \IDA from previous example where \IT{x} value is not defined, 
you'll see something like this:}

\begin{lstlisting}
.data:0040FA80 _x              dd ?                    ; DATA XREF: _main+10
.data:0040FA80                                         ; _main+22
.data:0040FA84 dword_40FA84    dd ?                    ; DATA XREF: _memset+1E
.data:0040FA84                                         ; unknown_libname_1+28
.data:0040FA88 dword_40FA88    dd ?                    ; DATA XREF: ___sbh_find_block+5
.data:0040FA88                                         ; ___sbh_free_block+2BC
.data:0040FA8C ; LPVOID lpMem
.data:0040FA8C lpMem           dd ?                    ; DATA XREF: ___sbh_find_block+B
.data:0040FA8C                                         ; ___sbh_free_block+2CA
.data:0040FA90 dword_40FA90    dd ?                    ; DATA XREF: _V6_HeapAlloc+13
.data:0040FA90                                         ; __calloc_impl+72
.data:0040FA94 dword_40FA94    dd ?                    ; DATA XREF: ___sbh_free_block+2FE
\end{lstlisting}

\RU{\TT{\_x} обозначен как \TT{?}, наряду с другими переменными не требующими инициализации. 
Это означает, что при загрузке .exe в память, место под все это выделено будет и будет заполнено
нулевыми байтами \cite[6.7.8p10]{C99TC3}. 
Но в самом .exe ничего этого нет. Неинициализированные переменные не занимают места в исполняемых файлах. 
Удобно для больших массивов, например.}
\EN{\TT{\_x} marked as \TT{?} among other variables not required to be initialized. 
This means that after loading .exe to memory, a space for all these variables will be 
allocated and filled by zeroes \cite[6.7.8p10]{C99TC3}. 
But in an .exe file these not initialized variables are not occupy anything. 
E.g. it is suitable for large arrays.}

\ifdefined\IncludeOlly
\input{patterns/04_scanf/2_global/olly.tex}
\fi

\subsection{GCC: x86}

\index{ELF}
\RU{В Linux все также почти. За исключением того, что если значение \TT{x} не определено, 
то эта переменная будет находится в сегменте \TT{\_bss}.
В \ac{ELF} этот сегмент имеет такие атрибуты:}
\EN{It is almost the same in Linux, except segment names and properties: 
not initialized variables are located in the \TT{\_bss} segment. 
In \ac{ELF} file format this segment has such attributes:}

\begin{lstlisting}
; Segment type: Uninitialized
; Segment permissions: Read/Write
\end{lstlisting}

\RU{Ну а если сделать статическое присвоение этой переменной какого-либо
значения, например, $10$, то она будет находится 
в сегменте \TT{\_data},
это сегмент с такими атрибутами:}
\EN{If to statically assign a value to variable, e.g. $10$, it will be placed in the \TT{\_data} segment, 
this is segment with the following attributes:}

\begin{lstlisting}
; Segment type: Pure data
; Segment permissions: Read/Write
\end{lstlisting}

\subsection{MSVC: x64}

\lstinputlisting[caption=MSVC 2012 x64]{patterns/04_scanf/2_global/ex2_MSVC_x64.asm.\LANG}

\RU{Почти такой же код как и в}\EN{Almost the same code as in} x86.
\RU{Обратите внимание что для \TT{scanf()} адрес переменной $x$ передается
при помощи инструкции \LEA, а во второй \printf передается само значение переменной при помощи \MOV}
\EN{Take a notice that $x$ variable address is passed to \TT{scanf()} using \LEA instruction,
while the value of variable is passed to the second \printf using \MOV instruction}.
\TT{``DWORD PTR''}\EMDASH{}\RU{это часть языка ассемблера (не имеющая отношения к машинным кодам) 
показывающая, что тип переменной в памяти\EMDASH{}именно 32-битный, 
и инструкция \MOV должна быть здесь закодирована соответственно}\EN{is a part of assembly language 
(no related to machine codes), showing that the variable data type is 32-bit and the \MOV instruction
should be encoded accordingly}.



\ifdefined\IncludeARM
\subsection{ARM: \OptimizingKeilVI (\ThumbMode)}

\begin{lstlisting}
.text:00000000 ; Segment type: Pure code
.text:00000000                 AREA .text, CODE
...
.text:00000000 main
.text:00000000                 PUSH    {R4,LR}
.text:00000002                 ADR     R0, aEnterX     ; "Enter X:\n"
.text:00000004                 BL      __2printf
.text:00000008                 LDR     R1, =x
.text:0000000A                 ADR     R0, aD          ; "%d"
.text:0000000C                 BL      __0scanf
.text:00000010                 LDR     R0, =x
.text:00000012                 LDR     R1, [R0]
.text:00000014                 ADR     R0, aYouEnteredD___ ; "You entered %d...\n"
.text:00000016                 BL      __2printf
.text:0000001A                 MOVS    R0, #0
.text:0000001C                 POP     {R4,PC}
...
.text:00000020 aEnterX         DCB "Enter X:",0xA,0    ; DATA XREF: main+2
.text:0000002A                 DCB    0
.text:0000002B                 DCB    0
.text:0000002C off_2C          DCD x                   ; DATA XREF: main+8
.text:0000002C                                         ; main+10
.text:00000030 aD              DCB "%d",0              ; DATA XREF: main+A
.text:00000033                 DCB    0
.text:00000034 aYouEnteredD___ DCB "You entered %d...",0xA,0 ; DATA XREF: main+14
.text:00000047                 DCB 0
.text:00000047 ; .text         ends
.text:00000047
...
.data:00000048 ; Segment type: Pure data
.data:00000048                 AREA .data, DATA
.data:00000048                 ; ORG 0x48
.data:00000048                 EXPORT x
.data:00000048 x               DCD 0xA                 ; DATA XREF: main+8
.data:00000048                                         ; main+10
.data:00000048 ; .data         ends
\end{lstlisting}

\RU{Итак, переменная \TT{x} теперь глобальная, и она расположена, почему-то, в другом сегменте, а именно сегменте данных}
\EN{So, the \TT{x} variable is now global and for this reason located in another segment, namely the data segment} (\IT{.data}).
\RU{Можно спросить, почему текстовые строки расположены в сегменте кода (\IT{.text}), а \TT{x} нельзя было разместить тут же?}
\EN{One could ask, why are the text strings located in the code segment (\IT{.text}) and \TT{x} is located right here?}
\RU{Потому что эта переменная, и как следует из определения, она может меняться. И может быть, меняться часто.}
\EN{Because it is a variable and by definition its value could change. Moreover it could possibly change often.}
\RU{Ну а текстовые строки имеют тип констант, они не будут меняться, поэтому они располагаются в сегменте \IT{.text}.}
\EN{While text strings has constant type, they will not be changed, so they are located in the \IT{.text} segment.}
\index{\RAM}
\index{\ROM}
\RU{Сегмент кода иногда может быть расположен в ПЗУ микроконтроллера (не забывайте, 
мы сейчас имеем дело с embedded-микроэлектроникой, где дефицит памяти~--- обычное дело),
а изменяемые переменные~--- в ОЗУ.}
\EN{The code segment might sometimes be located in a \ac{ROM} chip (remember, we now deal
with embedded microelectronics, and memory scarcity is common here), and changeable 
variables~---in \ac{RAM}.}
\RU{Хранить в ОЗУ неизменяемые данные, когда в наличии есть ПЗУ, не экономно.}
\EN{It is not very economical to store constant variables in RAM when you have ROM.}
\RU{К тому же, сегмент данных в ОЗУ с константами нужно инициализировать перед работой,
ведь, после включения ОЗУ, очевидно, она содержит в себе случайную информацию.}
\EN{Furthermore, constant variables in RAM must be initialized, because after powering on, the RAM, obviously, contains random information.}

\index{\RU{Компоновщик}\EN{Linker}}
\RU{Далее мы видим в сегменте кода хранится указатель на переменную \TT{x} (\TT{off\_2C}) и 
все операции с переменной происходят через этот указатель.}
\EN{Moving forward, we see a pointer to the \TT{x} (\TT{off\_2C}) variable in the code segment, and that all
operations with the variable occur via this pointer.}
\RU{Это связано с тем, что переменная \TT{x} может быть расположена где-то довольно далеко от 
данного участка кода, так что её адрес нужно сохранить в непосредственной близости к этому коду.}
\EN{That is because the \TT{x} variable could be located somewhere far from this particular code fragment, so its address
must be saved somewhere in close proximity to the code.}
\index{ARM!\Instructions!LDR}
\RU{Инструкция \TT{LDR} в Thumb-режиме может адресовать только переменные в пределах вплоть 
до 1020 байт от своего местоположения.}
\EN{The \TT{LDR} instruction in Thumb mode can only address variables in a range of 1020 bytes from its location, }
\RU{Эта же инструкция в ARM-режиме~--- переменные в пределах $\pm{}4095$ байт.}
\EN{and in in ARM-mode~---variables in range of $\pm{}4095$ bytes.}
\RU{Таким образом,
адрес глобальной переменной \TT{x} нужно расположить в непосредственной близости, ведь нет никакой гарантии, 
что компоновщик\footnote{linker в англоязычной литературе} сможет разместить саму переменную где-то рядом, 
она может быть даже в другом чипе памяти!}
\EN{And so the address of the \TT{x} variable
must be located somewhere in close proximity, because there is no guarantee that the linker would be able to accommodate the variable somewhere nearby the code, it may well be even in an external memory chip!}

\index{\CLanguageElements!const}
\index{\ROM}
\RU{Ещё одна вещь: если переменную объявить как \IT{const}, то компилятор Keil разместит её в 
сегменте \TT{.constdata}.}
\EN{One more thing: if a variable is declared as \IT{const}, the Keil compiler allocates it in 
the \TT{.constdata} segment.}
\RU{Должно быть, впоследствии компоновщик и этот сегмент сможет разместить в ПЗУ вместе
с сегментом кода.}
\EN{Perhaps, thereafter, the linker could place this segment in ROM too, along with the code segment.}

\subsection{ARM64}

\lstinputlisting[caption=\NonOptimizing GCC 4.9.1 ARM64,numbers=left]{patterns/04_scanf/2_global/ARM64_GCC491_O0.s.\LANG}

\index{ARM!\Instructions!ADRP/ADD pair}
\RU{Теперь $x$ это глобальная переменная, и её адрес вычисляется при помощи пары инструкций ADRP/ADD 
(строки 21 и 25).}
\EN{In this case the $x$ variable is declared as global and its address is calculated using 
the ADRP/ADD instruction pair (lines 21 and 25).}

\fi
\ifdefined\IncludeMIPS
\subsection{MIPS}

\subsubsection{\EN{Uninitialized global variable}\RU{Неинициализированная глобальная переменная}}

\RU{Так что теперь переменная $x$ глобальная.}
\EN{So now $x$ variable is global.}
\RU{Я сделал исполняемый файл вместо объектного и загрузил его в IDA.}
\EN{I made executable file rather than object one and load it into IDA.}
\RU{IDA показывает присутствие переменной $x$ в ELF-секции .sbss (помните о ``Global Pointer''? \ref{MIPS_GP}),
так как переменная не инициализируется в самом начале.}
\EN{IDA shows presence of $x$ variable in .sbss ELF section (remember about ``Global Pointer''? \ref{MIPS_GP}),
since variable is not initialized at the very start.}

\lstinputlisting[caption=\Optimizing GCC 4.4.5 (IDA)]{patterns/04_scanf/2_global/MIPS/O3_IDA.lst.\LANG}

\RU{IDA уменьшает количество информации, так что я также сделал листинг используя objdump и добавил туда
своих комментариев:}
\EN{IDA reduces amount of information, so I also made a listing using objdump and added my comments to it:}

\lstinputlisting[caption=\Optimizing GCC 4.4.5 (objdump),numbers=left]{patterns/04_scanf/2_global/MIPS/O3_objdump.txt.\LANG}

\RU{Теперь мы видим как адрес переменной $x$ берется из буфера 64KiB, используя GP и прибавление
к нему отрицательного смещения (строка 18).}
\EN{Now we see how address of $x$ variable is taken from a 64KiB data buffer using GP and adding
negative offset to it (line 18).}
\RU{И даже более того: адреса трех внешних ф-ций используемых в нашем примере (\puts, \scanf, \printf)
также берутся из буфера 64KiB используя GP (строки 9, 16 и 26).}
\EN{More than that: addresses of three external functions which are used in our example (\puts, \scanf, \printf), 
are also taken from 64KiB data buffer using GP (lines 9, 16 and 26).}
\RU{GP указывает на середину буфера, так что такие смещения могут нам подсказать, что адреса всех трех ф-ций,
а также адрес переменной $x$ расположены где-то в самом начале буфера.}
\EN{GP points to the middle of buffer, so such offset may give us a clue that all three function's addresses,
and also address of $x$ variable, are all stored somewhere at the beginning of data buffer.}
\RU{Действительно, ведь наш пример крохотный}\EN{Indeed, our example is tiny}.

\index{MIPS!\Pseudoinstructions!MOVE}
\index{MIPS!\Pseudoinstructions!NOP}
\RU{Еще нужно отметить что ф-ция заканчивается двумя \ac{NOP}-ами (\TT{MOVE \$AT,\$AT} --- 
это холостая инструкция), чтобы выровнять начало следующей ф-ции по 16-байтной границе.}
\EN{Another thing to mention is that the function is finished by two \ac{NOP}'s (\TT{MOVE \$AT,\$AT} --- 
this is idle instruction), in order to align next function's start on 16-byte boundary.}

\subsubsection{\RU{Инициализированная глобальная переменная}\EN{Initialized global variable}}

\RU{Немного изменим наш пример и сделаем, чтобы у $x$ было значение по умолчанию:}
\EN{Let's alter our example to make $x$ variable some default value:}

\lstinputlisting{patterns/04_scanf/2_global/default_value.c.\LANG}

\RU{Теперь IDA показывает что переменная $x$ располагается в секции .data:}
\EN{Now IDA shows that $x$ variable is residing in .data section:}

\lstinputlisting[caption=\Optimizing GCC 4.4.5 (IDA)]{patterns/04_scanf/2_global/MIPS/O3_IDA_init.lst.\LANG}

\RU{Почему не .sdata? Не знаю, может быть, нужно было указать какую-то опцию в GCC?}
\EN{Why not .sdata? I don't know, I probably should specify some GCC option?}
\RU{Тем не менее, $x$ теперь в .data, а это уже общая память и мы можем посмотреть как происходит
работа с переменными там.}
\EN{Nevertheless, now $x$ is in .data, that's general memory area, and we can now take a look
how to work with variables there.}

\index{MIPS!\Instructions!LUI}
\index{MIPS!\Instructions!ADDIU}
\RU{Адрес переменной должен быть сформирован парой инструкций.}
\EN{Address of variable must be formed using pair of instructions.}
\RU{В нашем случае это LUI (``Load Upper Immediate'' --- загрузить старшие 16 бит) и 
ADDIU (``Add Immediate Unsigned Word'' --- прибавить значение).}
\EN{In our case that's LUI (``Load Upper Immediate'') and ADDIU (``Add Immediate Unsigned Word'').}

\RU{Вот так же листинг сгенерированный objdump-ом для л\'{у}чшего рассмотрения:}
\EN{Here is also objdump listing for close inspection:}

\lstinputlisting[caption=\Optimizing GCC 4.4.5 (objdump)]{patterns/04_scanf/2_global/MIPS/O3_objdump_init.txt.\LANG}

\index{MIPS!\Instructions!LUI}
\index{MIPS!\Instructions!ADDIU}
\index{MIPS!\Instructions!LW}
\RU{Мы видим, что во-первых, адрес формируется используя LUI и ADDIU, но старшая часть адреса
все еще в регистре \$S0, и можно закодировать смещение в инструкции LW (``Load Word''), так что одной
LW достаточно для загрузки значения из переменной и передачи его в \printf.}
\EN{We see that for the first, address is formed using LUI and ADDIU, but high part of address is still in
\$S0 register, and it's possible to encode offset in LW (``Load Word'') instruction, so one single LW is enough 
for loading value from variable and pass it to \printf.}

\RU{Регистры хранящие временные данные имеют префикс T-, но здесь есть также регистры с префиксом S-,
содержимое которых должно быть сохранено в других ф-циях (т.е., ``saved'').}
\EN{Registers holding temporary data are prefixed with T-, but there are also those prefixed with S-, 
contents of which will be preserved in other functions (i.e., ``saved'').}
\RU{Вот почему \$S0 был установлен по адресу 0x4006cc, и затем был использован по адресу 0x4006e8,
после вызова \scanf.}
\EN{That's why \$S0 was set at address 0x4006cc and was used again
at address 0x4006e8, after \scanf call. }
\RU{Ф-ция \scanf не изменяет это значение.}\EN{\scanf function doesn't change its value.}

% TODO non-optimizing example?

\fi

\EN{\section{scanf()}

As was noted before, it is slightly old-fashioned to use \scanf today. 
But if we have to, we need to at least check if \scanf finishes correctly without an error.

\lstinputlisting{patterns/04_scanf/3_checking_retval/ex3.c}

By standard, the \scanf\footnote{scanf, wscanf: \href{http://go.yurichev.com/17255}{MSDN}} function returns the number of fields it has successfully read.

In our case, if everything goes fine and the user enters a number \scanf returns 1, or in case of error (or \ac{EOF}) --- 0.

Let's add some C code to check the \scanf return value and print error message in case of an error.

This works as expected:

\begin{lstlisting}
C:\...>ex3.exe
Enter X:
123
You entered 123...

C:\...>ex3.exe
Enter X:
ouch
What you entered? Huh?
\end{lstlisting}

% subsections
\input{patterns/04_scanf/3_checking_retval/x86}
\input{patterns/04_scanf/3_checking_retval/x64}
\ifdefined\IncludeARM
\input{patterns/04_scanf/3_checking_retval/ARM}
\fi
\ifdefined\IncludeMIPS
\input{patterns/04_scanf/3_checking_retval/MIPS}
\fi

\ifdefined\IncludeExercises
\subsection{\Exercise}

\myindex{x86!\Instructions!Jcc}
\myindex{ARM!\Instructions!Bcc}
As we can see, the JNE/JNZ instruction can be easily replaced by the JE/JZ and vice versa 
(or BNE by BEQ and vice versa).
But then the basic blocks must also be swapped.
Try to do this in some of the examples.
\fi

}
\RU{\subsection{Проверка результата scanf()}

Как уже было упомянуто, использовать \scanf в наше время слегка старомодно. 
Но если уж пришлось, нужно хотя бы проверять, сработал ли \scanf 
правильно или пользователь ввел вместо числа что-то другое, что \scanf не смог трактовать как число.

\lstinputlisting[style=customc]{patterns/04_scanf/3_checking_retval/ex3.c}

По стандарту,\scanf\footnote{scanf, wscanf: \href{http://go.yurichev.com/17255}{MSDN}} возвращает количество успешно полученных значений.

В нашем случае, если всё успешно и пользователь ввел таки некое число, \scanf вернет 1. А если нет, то 0 (или \ac{EOF}).

Добавим код, проверяющий результат \scanf и в случае ошибки он сообщает пользователю что-то другое.

Это работает предсказуемо:

\begin{lstlisting}
C:\...>ex3.exe
Enter X:
123
You entered 123...

C:\...>ex3.exe
Enter X:
ouch
What you entered? Huh?
\end{lstlisting}

% subsections
\input{patterns/04_scanf/3_checking_retval/x86}
\input{patterns/04_scanf/3_checking_retval/x64}
\input{patterns/04_scanf/3_checking_retval/ARM}
\input{patterns/04_scanf/3_checking_retval/MIPS}

\subsubsection{\Exercise}

\myindex{x86!\Instructions!Jcc}
\myindex{ARM!\Instructions!Bcc}
Как мы можем увидеть, инструкцию \INS{JNE}/\INS{JNZ} можно вполне заменить на \INS{JE}/\INS{JZ} или наоборот 
(или \INS{BNE} на \INS{BEQ} и наоборот).
Но при этом ещё нужно переставить базовые блоки местами.
Попробуйте сделать это в каком-нибудь примере.
}
\PTBR{\subsection{Checagem de resultados do scanf()}

Como dito anteriormente, está meio fora de moda usar \scanf atualmente, mas nós temos que fazer.
Precisamos ao menos testar se a função \scanf termina corretamente sem nenhum erro.

\lstinputlisting[style=customc]{patterns/04_scanf/3_checking_retval/ex3.c}

Por padrão, a função \scanf\footnote{scanf, wscanf: \href{http://go.yurichev.com/17255}{MSDN}} retorna o número de campos que ela leu com sucesso.

No nosso caso, se tudo ocorrer bem e o usuário entrar com um número, \scanf retornará 1, ou em caso de erro (ou \ac{EOF}) --- 0.

Vamos adicionar um pouco de código em C para checar o valor de retorno de \scanf e imprimir uma mensagem no caso de um erro.

Isso funciona com o desejado:

\begin{lstlisting}
C:\...>ex3.exe
Enter X:
123
You entered 123...

C:\...>ex3.exe
Enter X:
ouch
What you entered? Huh?
\end{lstlisting}

% subsections
\input{patterns/04_scanf/3_checking_retval/x86}
\input{patterns/04_scanf/3_checking_retval/x64}
\input{patterns/04_scanf/3_checking_retval/ARM}
\input{patterns/04_scanf/3_checking_retval/MIPS}

\subsubsection{\Exercise}

\PTBRph{}

}
\ITA{\subsection{scanf()}

Come gia' detto in precedenza, usare \scanf oggi e' un po' antiquato.
Se proprio dobbiamo, e' necessario almeno controllare se \scanf termina correttamente senza errori.

\lstinputlisting[style=customc]{patterns/04_scanf/3_checking_retval/ex3.c}

Per standard, la funzione \scanf\footnote{scanf, wscanf: \href{http://go.yurichev.com/17255}{MSDN}} restituisce il numero di campi che e' riuscita a leggere con successo.
Nel nostro caso, se tutto va bene e l'utente inserisce un numero, \scanf restituisce 1, oppure 0 (o \ac{EOF}) in caso di errore. 

Aggiungiamo un po' di codice C per controllare che \scanf restituisca un valore e stampi un messaggio in caso di errore.

Funziona come ci si aspetta:

\begin{lstlisting}
C:\...>ex3.exe
Enter X:
123
You entered 123...

C:\...>ex3.exe
Enter X:
ouch
What you entered? Huh?
\end{lstlisting}

% subsections
\input{patterns/04_scanf/3_checking_retval/x86}
\input{patterns/04_scanf/3_checking_retval/x64}
\input{patterns/04_scanf/3_checking_retval/ARM}
\input{patterns/04_scanf/3_checking_retval/MIPS}

\subsubsection{\Exercise}

\myindex{x86!\Instructions!Jcc}
\myindex{ARM!\Instructions!Bcc}
Come possiamo vedere, le istruzioni \INS{JNE}/\INS{JNZ} possono essere scambiate con \INS{JE}/\INS{JZ} e viceversa.
(lo stesso vale per \INS{BNE} e \INS{BEQ}).
Ma se cio' avviene i blocchi base devono anch'essi essere scambiati. Provate a farlo in qualche esempio.

}
\FR{\subsection{scanf()}

Comme il a déjà été écrit, il est plutôt dépassé d'utiliser \scanf aujourd'hui.
Mais si nous devons, il faut vérifier si \scanf se termine correctement sans erreur.

\lstinputlisting[style=customc]{patterns/04_scanf/3_checking_retval/ex3.c}

Par norme, la fonction \scanf\footnote{scanf, wscanf: \href{http://go.yurichev.com/17255}{MSDN}}
renvoie le nombre de champs qui ont été lus avec succès.

Dans notre cas, si tout se passe bien et que l'utilisateur entre un nombre \scanf
renvoie 1, ou en cas d'erreur (ou \ac{EOF}) --- 0.

Ajoutons un peu de code C pour vérifier la valeur de retour de \scanf et afficher
un message d'erreur en cas d'erreur.

Cela fonctionne comme attendu:

\begin{lstlisting}
C:\...>ex3.exe
Enter X:
123
You entered 123...

C:\...>ex3.exe
Enter X:
ouch
What you entered? Huh?
\end{lstlisting}

% subsections
\input{patterns/04_scanf/3_checking_retval/x86}
\input{patterns/04_scanf/3_checking_retval/x64}
\input{patterns/04_scanf/3_checking_retval/ARM}
\input{patterns/04_scanf/3_checking_retval/MIPS}

\subsubsection{\Exercise}

\myindex{x86!\Instructions!Jcc}
\myindex{ARM!\Instructions!Bcc}
Comme nous pouvons voir, les instructions \INS{JNE}/\INS{JNZ} peuvent facilement
être remplacées par \INS{JE}/\INS{JZ} et vice-versa (ou \INS{BNE} par \INS{BEQ} et vice-versa).
Mais les blocs de base doivent aussi être échangés.
Essayez de faire cela pour quelques exemples.

}


\subsection{\Exercise}

\begin{itemize}
	\item \url{http://challenges.re/53}
\end{itemize}


\EN{\mysection{Accessing passed arguments}
\myindex{\Stack}

Now we figured out that the \gls{caller} function is passing arguments to the \gls{callee} via the stack. 
But how does the \gls{callee} access them?

\lstinputlisting[label=src:passing_arguments_ex,caption=simple example,style=customc]{patterns/05_passing_arguments/ex.c}

% sections
\subsubsection{x86}

This compiles to:

\lstinputlisting[caption=MSVC 2012 /GS- /Ob0,label=src:struct_packing_4,numbers=left,style=customasmx86]{patterns/15_structs/4_packing/packing_EN.asm}

We pass the structure as a whole, but in fact, as we can see, the structure
is being copied to a temporary one (a place in stack is allocated in line 10 for it,
and then all 4 fields, one by one, are copied in lines 12 \ldots\ 19), 
and then its pointer (address) is to be passed.

The structure is copied because it's not known whether the \ttf{} 
function going to modify the structure or not.
If it gets changed, then the structure in \main has to remain as it has been.

We could use \CCpp pointers, and the resulting code will be almost the same, but without
the copying.

As we can see, each field's address is aligned on a 4-byte boundary.
That's why each \Tchar occupies 4 bytes here (like \Tint). Why?
Because it is easier for the CPU to access memory at aligned addresses and to cache data from it.

However, it is not very economical.

Let's try to compile it with option (\TT{/Zp1}) 
(\IT{/Zp[n] pack structures on n-byte boundary}).

\lstinputlisting[caption=MSVC 2012 /GS- /Zp1,label=src:struct_packing_1,numbers=left,style=customasmx86]{patterns/15_structs/4_packing/packing_msvc_Zp1_EN.asm}

Now the structure takes only 10 bytes and each \Tchar value takes 1 byte. What does it give to us?
Size economy. And as drawback~---the CPU accessing these fields slower than it could.

\label{short_struct_copying_using_MOV}

The structure is also copied in \main. Not field-by-field, but directly 10 bytes, using three pairs of \MOV.
Why not 4?

The compiler decided that it's better to copy 10 bytes using 3 \MOV pairs than to copy two 32-bit words
and two bytes using 4 \MOV pairs.

By the way, such copy implementation using \MOV instead of calling the \TT{memcpy()} function is widely
used, because it's faster than a call to \TT{memcpy()}---for short blocks, of course:
\myref{copying_short_blocks}.

As it can be easily guessed, if the structure is used in many source and object files,
all these must be compiled with the same convention about structures packing.

\newcommand{\FNURLMSDNZP}{\footnote{\href{http://go.yurichev.com/17067}
{MSDN: Working with Packing Structures}}}
\newcommand{\FNURLGCCPC}{\footnote{\href{http://go.yurichev.com/17068}
{Structure-Packing Pragmas}}}

Aside from MSVC \TT{/Zp} option which sets how to align each structure field, there is also
the \TT{\#pragma pack} compiler option, which can be defined right in the source code.
It is available in both MSVC\FNURLMSDNZP and GCC\FNURLGCCPC{}.

Let's get back to the \TT{SYSTEMTIME} structure that consists of 16-bit fields.
How does our compiler know to pack them on 1-byte alignment boundary?

\TT{WinNT.h} file has this:

\begin{lstlisting}[caption=WinNT.h,style=customc]
#include "pshpack1.h"
\end{lstlisting}

And this:

\begin{lstlisting}[caption=WinNT.h,style=customc]
#include "pshpack4.h"                   // 4 byte packing is the default
\end{lstlisting}

The file PshPack1.h looks like:

\begin{lstlisting}[caption=PshPack1.h,style=customc]
#if ! (defined(lint) || defined(RC_INVOKED))
#if ( _MSC_VER >= 800 && !defined(_M_I86)) || defined(_PUSHPOP_SUPPORTED)
#pragma warning(disable:4103)
#if !(defined( MIDL_PASS )) || defined( __midl )
#pragma pack(push,1)
#else
#pragma pack(1)
#endif
#else
#pragma pack(1)
#endif
#endif /* ! (defined(lint) || defined(RC_INVOKED)) */
\end{lstlisting}

This tell the compiler how to pack the structures defined after \TT{\#pragma pack}.

\input{patterns/15_structs/4_packing/olly_EN.tex}

\subsection{x64}

\myindex{x86-64}
The picture here is similar with the difference that the registers, rather than the stack, are used for arguments passing.

\subsubsection{MSVC}

\lstinputlisting[caption=MSVC 2012 x64]{patterns/04_scanf/1_simple/ex1_MSVC_x64.asm.\LANG}

\ifdefined\IncludeGCC
\subsubsection{GCC}

\lstinputlisting[caption=\Optimizing GCC 4.4.6 x64]{patterns/04_scanf/1_simple/ex1_GCC_x64.s.\LANG}
\fi


\subsection{ARM}

% subsections
\EN{\input{patterns/05_passing_arguments/ARM/no_keil_ARM_EN}}
\RU{\input{patterns/05_passing_arguments/ARM/no_keil_ARM_RU}}
\ITA{\input{patterns/05_passing_arguments/ARM/no_keil_ARM_ITA}}
\EN{\input{patterns/05_passing_arguments/ARM/o_keil_ARM_EN}}
\RU{\input{patterns/05_passing_arguments/ARM/o_keil_ARM_RU}}
\ITA{\input{patterns/05_passing_arguments/ARM/o_keil_ARM_ITA}}
\EN{\input{patterns/05_passing_arguments/ARM/thumb_EN}}
\RU{\input{patterns/05_passing_arguments/ARM/thumb_RU}}
\ITA{\input{patterns/05_passing_arguments/ARM/thumb_ITA}}
\EN{\input{patterns/05_passing_arguments/ARM/ARM64_EN}}
\RU{\input{patterns/05_passing_arguments/ARM/ARM64_RU}}
\ITA{\input{patterns/05_passing_arguments/ARM/ARM64_ITA}}

\subsection{MIPS}

\lstinputlisting[caption=\Optimizing GCC 4.4.5,style=customasmMIPS]{patterns/05_passing_arguments/MIPS_O3_IDA_EN.lst}

The first four function arguments are passed in four registers prefixed by A-.

\myindex{MIPS!\Instructions!MULT}

There are two special registers in MIPS: HI and LO which are filled with the 64-bit result of the multiplication during the execution of the \TT{MULT} instruction.
\myindex{MIPS!\Instructions!MFLO}
\myindex{MIPS!\Instructions!MFHI}

These registers are accessible only by using the \TT{MFLO} and \TT{MFHI} instructions.
\TT{MFLO} here takes the low-part of the multiplication result and stores it into \$V0.
So the high 32-bit part of the multiplication result is dropped (the HI register content is not used).
Indeed: we work with 32-bit \Tint data types here.

\myindex{MIPS!\Instructions!ADDU}

Finally, \TT{ADDU} (\q{Add Unsigned}) adds the value of the third argument to the result.

\myindex{MIPS!\Instructions!ADD}
\myindex{MIPS!\Instructions!ADDU}
\myindex{Ada}
\myindex{Integer overflow}

There are two different addition instructions in MIPS: \TT{ADD} and \TT{ADDU}.
The difference between them is not related to signedness, but to exceptions. \TT{ADD} can raise an exception on overflow, which is sometimes useful\footnote{\url{http://go.yurichev.com/17326}} and supported in Ada \ac{PL}, for instance.
\TT{ADDU} does not raise exceptions on overflow.

Since \CCpp does not support this, in our example we see \TT{ADDU} instead of \TT{ADD}.

The 32-bit result is left in \$V0.

\myindex{MIPS!\Instructions!JAL}
\myindex{MIPS!\Instructions!JALR}

There is a new instruction for us in \main: \TT{JAL} (\q{Jump and Link}). 

The difference between \INS{JAL} and \INS{JALR} is that a relative offset is encoded in the first instruction, 
while \INS{JALR} jumps to the absolute address stored in a register (\q{Jump and Link Register}).

Both \ttf and \main functions are located in the same object file, so the relative address of \ttf 
is known and fixed.


}
\RU{\mysection{Доступ к переданным аргументам}
\myindex{\Stack}

Как мы уже успели заметить, вызывающая функция передает аргументы для вызываемой через стек. 
А как вызываемая функция получает к ним доступ?

\lstinputlisting[label=src:passing_arguments_ex,caption=простой пример,style=customc]{patterns/05_passing_arguments/ex.c}

% sections

\subsection{x86}

\subsubsection{MSVC}

Рассмотрим пример, скомпилированный в (MSVC 2010 Express):

\lstinputlisting[label=src:passing_arguments_ex_MSVC_cdecl,caption=MSVC 2010 Express,style=customasmx86]{patterns/05_passing_arguments/msvc_RU.asm}

\myindex{x86!\Registers!EBP}
Итак, здесь видно: в функции \main заталкиваются три числа в стек и вызывается функция \TT{f(int,int,int)}.
 
Внутри \ttf доступ к аргументам, также как и к локальным переменным, происходит через макросы: 
\TT{\_a\$ = 8}, но разница в том, что эти смещения со знаком \IT{плюс}, 
таким образом если прибавить макрос \TT{\_a\$} к указателю на \EBP, то адресуется \IT{внешняя} 
часть \glslink{stack frame}{фрейма} стека относительно \EBP.

\myindex{x86!\Instructions!IMUL}
\myindex{x86!\Instructions!ADD}
Далее всё более-менее просто: значение $a$ помещается в \EAX. 
Далее \EAX умножается при помощи инструкции \IMUL на то, что лежит в \TT{\_b}, 
и в \EAX остается \glslink{product}{произведение} этих двух значений.

Далее к регистру \EAX прибавляется то, что лежит в \TT{\_c}.

Значение из \EAX никуда не нужно перекладывать, оно уже лежит где надо. 
Возвращаем управление вызываемой функции~--- она возьмет значение из \EAX и отправит его в \printf.

\input{patterns/05_passing_arguments/olly_RU}

\subsubsection{GCC}

Скомпилируем то же в GCC 4.4.1 и посмотрим результат в \IDA:

\lstinputlisting[caption=GCC 4.4.1,style=customasmx86]{patterns/05_passing_arguments/gcc_RU.asm}

Практически то же самое, если не считать мелких отличий описанных ранее.

После вызова обоих функций \glslink{stack pointer}{указатель стека} не возвращается назад, 
потому что предпоследняя инструкция \TT{LEAVE} (\myref{x86_ins:LEAVE}) делает это за один раз, в конце исполнения.


\subsection{x64}

\myindex{x86-64}
В x86-64 всё немного иначе, здесь аргументы функции (4 или 6) передаются через регистры, 
а \gls{callee} из читает их из регистров, а не из стека.

\subsubsection{MSVC}

\Optimizing MSVC:

\lstinputlisting[caption=\Optimizing MSVC 2012 x64,style=customasmx86]{patterns/05_passing_arguments/x64_MSVC_Ox_RU.asm}

Как видно, очень компактная функция \ttf берет аргументы прямо из регистров.

Инструкция \LEA используется здесь для сложения чисел. 
Должно быть компилятор посчитал, что это будет эффективнее использования \TT{ADD}.

\myindex{x86!\Instructions!LEA}
В самой \main{} \LEA{} также используется для подготовки первого и третьего аргумента: должно быть,
компилятор решил, что \LEA{} будет работать здесь быстрее, чем загрузка значения в регистр при помощи \MOV.

Попробуем посмотреть вывод неоптимизирующего MSVC:

\lstinputlisting[caption=MSVC 2012 x64,style=customasmx86]{patterns/05_passing_arguments/x64_MSVC_IDA_RU.asm}

Немного путанее: все 3 аргумента из регистров зачем-то сохраняются в стеке.

\myindex{Shadow space}
\label{shadow_space}
Это называется \q{shadow space} \footnote{\href{http://go.yurichev.com/17256}{MSDN}}: 
каждая функция в Win64 может (хотя и не обязана) сохранять значения 4-х регистров там.

Это делается по крайней мере из-за двух причин: 
1) в большой функции отвести целый регистр (а тем более 4 регистра) для входного аргумента 
слишком расточительно, так что к нему будет обращение через стек;

2) отладчик всегда знает, где найти аргументы функции в момент останова
\footnote{\href{http://go.yurichev.com/17257}{MSDN}}.

Так что, какие-то большие функции могут сохранять входные аргументы в \q{shadows space} 
для использования в будущем, а небольшие функции, как наша, могут этого и не делать.

Место в стеке для \q{shadow space} выделяет именно \gls{caller}.

\subsubsection{GCC}

\Optimizing GCC также делает понятный код:

\lstinputlisting[caption=\Optimizing GCC 4.4.6 x64,style=customasmx86]{patterns/05_passing_arguments/x64_GCC_O3_RU.s}

\NonOptimizing GCC:

\lstinputlisting[caption=GCC 4.4.6 x64,style=customasmx86]{patterns/05_passing_arguments/x64_GCC_RU.s}

\myindex{Shadow space}
В соглашении о вызовах System V *NIX (\SysVABI) нет \q{shadow space}, но \gls{callee} тоже иногда
должен сохранять где-то аргументы, потому что, опять же, регистров может и не хватить на все действия.
Что мы здесь и видим.

\subsubsection{GCC: uint64\_t вместо int}

Наш пример работал с 32-битным \Tint, поэтому использовались 32-битные части регистров с префиксом \TT{E-}.

Его можно немного переделать, чтобы он заработал с 64-битными значениями:

\lstinputlisting[style=customc]{patterns/05_passing_arguments/ex64.c}

\lstinputlisting[caption=\Optimizing GCC 4.4.6 x64,style=customasmx86]{patterns/05_passing_arguments/ex64_GCC_O3_IDA_RU.asm}

Собствено, всё то же самое, только используются регистры \IT{целиком}, с префиксом \TT{R-}.


\subsection{ARM}

% subsections
\EN{\input{patterns/05_passing_arguments/ARM/no_keil_ARM_EN}}
\RU{\input{patterns/05_passing_arguments/ARM/no_keil_ARM_RU}}
\ITA{\input{patterns/05_passing_arguments/ARM/no_keil_ARM_ITA}}
\EN{\input{patterns/05_passing_arguments/ARM/o_keil_ARM_EN}}
\RU{\input{patterns/05_passing_arguments/ARM/o_keil_ARM_RU}}
\ITA{\input{patterns/05_passing_arguments/ARM/o_keil_ARM_ITA}}
\EN{\input{patterns/05_passing_arguments/ARM/thumb_EN}}
\RU{\input{patterns/05_passing_arguments/ARM/thumb_RU}}
\ITA{\input{patterns/05_passing_arguments/ARM/thumb_ITA}}
\EN{\input{patterns/05_passing_arguments/ARM/ARM64_EN}}
\RU{\input{patterns/05_passing_arguments/ARM/ARM64_RU}}
\ITA{\input{patterns/05_passing_arguments/ARM/ARM64_ITA}}

\subsection{MIPS}

\lstinputlisting[caption=\Optimizing GCC 4.4.5,style=customasmMIPS]{patterns/05_passing_arguments/MIPS_O3_IDA_RU.lst}

Первые 4 аргумента функции передаются в четырех регистрах с префиксами A-.

\myindex{MIPS!\Instructions!MULT}
В MIPS есть два специальных регистра: HI и LO, которые выставляются в 64-битный результат умножения
во время исполнения инструкции \TT{MULT}.

\myindex{MIPS!\Instructions!MFLO}
\myindex{MIPS!\Instructions!MFHI}
К регистрам можно обращаться только используя инструкции \TT{MFLO} и \TT{MFHI}.
Здесь \TT{MFLO} берет младшую часть результата умножения и записывает в \$V0.
Так что старшая 32-битная часть результата игнорируется (содержимое регистра HI не используется).
Действительно, мы ведь работаем с 32-битным типом \Tint.


\myindex{MIPS!\Instructions!ADDU}
И наконец, \TT{ADDU} (\q{Add Unsigned}~--- добавить беззнаковое) прибавляет значение третьего аргумента к результату.

\myindex{MIPS!\Instructions!ADD}
\myindex{MIPS!\Instructions!ADDU}
\myindex{Ada}
\myindex{Integer overflow}
В MIPS есть две разных инструкции сложения: \TT{ADD} и \TT{ADDU}.
На самом деле, дело не в знаковых числах, а в исключениях: \TT{ADD} может вызвать исключение
во время переполнения. Это иногда полезно\footnote{\url{http://go.yurichev.com/17326}} и поддерживается,
например, в \ac{PL} Ada.

\TT{ADDU} не вызывает исключения во время переполнения.
А так как \CCpp не поддерживает всё это, мы видим здесь \TT{ADDU} вместо \TT{ADD}.

32-битный результат оставляется в \$V0.

\myindex{MIPS!\Instructions!JAL}
\myindex{MIPS!\Instructions!JALR}
В \main есть новая для нас инструкция: \TT{JAL} (\q{Jump and Link}). 
Разница между \INS{JAL} и \INS{JALR} в том, что относительное смещение кодируется в первой инструкции,
а \INS{JALR} переходит по абсолютному адресу, записанному в регистр (\q{Jump and Link Register}).

Обе функции \ttf и \main расположены в одном объектном файле, так что относительный адрес \ttf известен и фиксирован.



}

\chapter{\RU{Еще о возвращаемых результатах}\EN{More about results returning}}

\index{x86!\Registers!EAX}
\index{ARM!\Registers!R0}
\RU{Результат выполнения функции в x86 обычно возвращается}
\EN{As of x86, function execution result is usually returned}
\footnote{\Seealso: 
MSDN: Return Values (C++): \href{http://go.yurichev.com/17258}{MSDN}}
\RU{через регистр \EAX, 
а если результат имеет тип байт или символ (\Tchar), 
то в самой младшей части \EAX ~--- \AL. Если функция возвращает число с плавающей запятой, 
то будет использован регистр FPU \ST{0}.
В ARM обычно результат возвращается в регистре \Reg{0}.}
\EN{in the \EAX register. 
If it is byte type or character (\Tchar)~---then in the lowest register \EAX part~---\AL. 
If a function returns \Tfloat number, the FPU register 
\ST{0} is to be used instead.
In ARM, result is usually returned in the \Reg{0} register.}

\section{\RU{Попытка использовать результат ф-ции возвращающей \Tvoid}
\EN{Attempt to use result of function returning \Tvoid}}

\RU{Кстати, что будет если возвращаемое значение в ф-ции \main объявлять не как \Tint а как \Tvoid?}
\EN{By the way, what if returning value of the \main function will be declared not as \Tint but as \Tvoid?}

\RU{Т.н. startup-код вызывает \main примерно так:}
\EN{so-called startup-code is calling \main roughly as:}

\begin{lstlisting}
push envp
push argv
push argc
call main
push eax
call exit
\end{lstlisting}

\RU{Т.е., иными словами:}\EN{In other words:}

\begin{lstlisting}
exit(main(argc,argv,envp));
\end{lstlisting}

\RU{Если вы объявите \main как \Tvoid, и ничего не будете возвращать явно (при помощи выражения \IT{return}), 
то в единственный аргумент exit() попадет
то, что лежало в регистре \EAX на момент выхода из \main.}
\EN{If you declare \main as \Tvoid and nothing will be returned explicitly (by \IT{return} statement),
then something random, that was stored in the \EAX register at the moment of the \main finish, will come into
the sole exit() function argument.}
\RU{Там, скорее всего, будет какие-то случайное число, оставшееся от работы вашей ф-ции.}
\EN{Most likely, there will be a random value, left from your function execution.}
\RU{Так что, код завершения программы будет псевдослучайным.}
\EN{So, exit code of program will be pseudorandom.} \\

\RU{Я могу это проиллюстрировать}\EN{I can illustrate this fact}. 
\RU{Заметьте что у ф-ции}\EN{Please notice, the} \main \RU{тип возвращаемого значения именно}\EN{function 
has} \Tvoid\EN{ type}:

\begin{lstlisting}
#include <stdio.h>

void main()
{
	printf ("Hello, world!\n");
};
\end{lstlisting}

\RU{Скомпилируем в}\EN{Let's compile it in} Linux.

\index{puts() \RU{вместо}\EN{instead of} printf()}
GCC 4.8.1 \RU{заменила}\EN{replaced} \printf \RU{на}\EN{to} \puts 
(\RU{мы видели это прежде}\EN{we saw this before}: \ref{puts}), 
\RU{но это нормально, потому что}\EN{but that's OK, since} \puts \RU{возвращает количество
выведенных символов, так же как и}\EN{returns number of characters printed, just like} \printf.
\RU{Обратите внимание на то что}\EN{Please notice that} \EAX \RU{не обнуляется перед выходом их}\EN{is not 
zeroed before} \main\EN{ finish}.
\RU{Это значит}\EN{This means}, \EAX \RU{перед выходом из}\EN{value at the} \main 
\RU{будет содержать то, что}\EN{finish will contain what} \puts \RU{оставит там}\EN{left there}.

\begin{lstlisting}[caption=GCC 4.8.1]
.LC0:
	.string	"Hello, world!"
main:
	push	ebp
	mov	ebp, esp
	and	esp, -16
	sub	esp, 16
	mov	DWORD PTR [esp], OFFSET FLAT:.LC0
	call	puts
	leave
	ret
\end{lstlisting}

\index{bash}
\RU{Напишем небольшой скрипт на bash, показывающий статус возврата (``exit status'' или ``exit code'')}
\EN{Let' s write bash script, showing exit status}:

\begin{lstlisting}[caption=tst.sh]
#!/bin/sh
./hello_world
echo $?
\end{lstlisting}

\RU{И запустим}\EN{And run it}:

\begin{lstlisting}
$ tst.sh 
Hello, world!
14
\end{lstlisting}

$14$ \RU{это как раз количество выведенных символов}\EN{is a number of characters printed}.

\section{\RU{Что если не использовать результат ф-ции?}\EN{What if not to use function result?}}

\RU{\printf возвращает количество успешно выведенных символов, но результат работы этой ф-ции 
редко используется на практике.}
\EN{\printf returns count of characters successfully sent to output, but result of this function 
is rarely used in practice.}
\RU{Можно даже явно вызывать ф-ции, чей смысл именно в возвращаемых значениях, но явно не использовать их:}
\EN{It's possible to call functions which essence in returning values, but not to use them explicitely:}

\begin{lstlisting}
int f()
{
    // skip first 3 random values
    rand();
    rand();
    rand();
    // and use 4th
    return rand();
};
\end{lstlisting}

\EN{Result of rand() function will always be leaved in \EAX, in all four cases.}
\RU{Результат работы rand() будет оставаться в \EAX во всех четырех случаях.}
\EN{But in first 3 cases, a value in \EAX will be just thrown away.}
\RU{Но в первых трех случаях, значение лежащее в \EAX, будет просто выброшено.}

\section{\RU{Возврат структуры}\EN{Returning a structure}}

\index{\CLanguageElements!return}
\RU{Вернемся к тому факту, что возвращаемое значение остается в регистре \EAX}
\EN{Let's back to the fact the returning value is left in the \EAX register}.
\RU{Вот почему старые компиляторы Си не способны создавать функции, возвращающие нечто большее нежели 
помещается 
в один регистр (обычно, тип \Tint), а когда нужно, приходится возвращать через указатели, указываемые 
в аргументах.}
\EN{That is why old C compilers cannot create functions capable of returning something not fitting in one 
register (usually type \Tint) but if one needs it, one should return information via pointers passed 
in function arguments.}
\RU{Так что, как правило, если ф-ция должна вернуть несколько значений, она возвращает только одно, 
а остальные --- через указатели.}
\EN{So, usually, if a function should return several values, it returns only one, and 
all the rest---via pointers.}
\RU{Хотя, позже и стало возможным, вернуть, скажем, целую структуру, но этот метод до сих пор не 
очень популярен. 
Если функция должна вернуть структуру, вызывающая функция должна сама, скрыто и прозрачно для программиста, 
выделить место и передать указатель на него в качестве первого аргумента. Это почти то же самое 
что и сделать это вручную, но компилятор прячет это.}
\EN{Now it is possible, to return, let's say, whole structure, but still it is not very popular. 
If function must return a large structure, \gls{caller} must allocate it and pass pointer to it via first argument, 
transparently for programmer. 
That is almost the same as to pass pointer in first argument manually, but compiler hide this.}

\RU{Небольшой пример:}\EN{Small example:}

\lstinputlisting{patterns/06_return_results/6_1.c}

\dots \RU{получим}\EN{what we got} (MSVC 2010 \Ox):

\lstinputlisting{patterns/06_return_results/6_1.asm}

\RU{Имя внутреннего макроса для передачи указателя на структуру здесь это \TT{\$T3853}.}
\EN{Macro name for internal variable passing pointer to structure is \TT{\$T3853} here.}

\index{\CLanguageElements!C99}
\RU{Этот пример можно даже переписать используя расширения C99}\EN{This example can be rewritten using
C99 language extensions}:

\lstinputlisting{patterns/06_return_results/6_1_C99.c}

\lstinputlisting[caption=GCC 4.8.1]{patterns/06_return_results/6_1_C99.asm}

\RU{Как видно, ф-ция просто заполняет поля в структуре, выделенной вызывающей ф-цией. 
Как если бы передавался просто указатель на структуру.
Так что никаких проблем с эффективностью нет.}
\EN{As we may see, the function is just filling fields in the structure, allocated by
caller function. 
As if a pointer to the structure was passed.
So there are no performance drawbacks.}

\chapter{\RU{Указатели}\EN{Pointers}}
\index{\CLanguageElements!\Pointers}
\label{label_pointers}

\RU{Указатели также часто используются для возврата значений из функции (вспомните случай
со \scanf{}~(\myref{label_scanf})).}
\EN{Pointers are often used to return values from functions (recall \scanf case~(\myref{label_scanf})).}
\RU{Например, когда функции нужно вернуть сразу два значения.}
\EN{For example, when a function needs to return two values.}

\section{\RU{Пример с глобальными переменными}\EN{Global variables example}}

\lstinputlisting{patterns/061_pointers/global.c}

\RU{Это компилируется в}\EN{This compiles to}:

\lstinputlisting[caption=\Optimizing MSVC 2010 (/Ob0)]{patterns/061_pointers/global.asm}

\index{\olly}
\clearpage
\RU{Посмотрим это в}\EN{Let's see this in} \olly:

\begin{figure}[H]
\centering
\includegraphics[scale=\FigScale]{patterns/061_pointers/olly_global1.png}
\caption{\olly: \RU{передаются адреса двух глобальных переменных в}
\EN{global variables addresses are passed to} \ttfone}
\label{fig:pointers_olly_global_1}
\end{figure}

\RU{В начале адреса обоих глобальных переменных передаются в}\EN{First, global
variables' addresses are passed to} \ttfone.
\RU{Можно нажать}\EN{We can click} \q{Follow in dump} 
\RU{на элементе стека и в окне слева 
увидим место в сегменте данных, выделенное для двух переменных.}
\EN{on the stack element, and we can see the place in the data segment allocated 
for the two variables.}
\RU{Эти переменные обнулены, потому что по стандарту неинициализированные данные (\ac{BSS}) 
обнуляются перед началом исполнения: \cite[6.7.8p10]{C99TC3}.}

\clearpage
\EN{These variables are zeroed, because non-initialized data (from \ac{BSS}) is cleared before
the execution begins: \cite[6.7.8p10]{C99TC3}.}
\RU{И они находятся в сегменте данных, о чем можно удостовериться, нажав}
\EN{They reside in the data segment, we can verify this by pressing} Alt-M \RU{и увидев карту
памяти}\EN{and reviewing the memory map}:

\begin{figure}[H]
\centering
\includegraphics[scale=\FigScale]{patterns/061_pointers/olly_global5.png}
\caption{\olly: \RU{карта памяти}\EN{memory map}}
\label{fig:pointers_olly_global_5}
\end{figure}

\clearpage
\RU{Трассируем}\EN{Let's trace} (F7) \RU{до начала исполнения}\EN{to the start of} \ttfone: 

\begin{figure}[H]
\centering
\includegraphics[scale=\FigScale]{patterns/061_pointers/olly_global2.png}
\caption{\olly: \RU{начало работы \ttfone}\EN{\ttfone starts}}
\label{fig:pointers_olly_global_2}
\end{figure}

\RU{В стеке видны значения}\EN{Two values are visible in the stack} 456 (\TT{0x1C8}) \AndENRU 
123 (\TT{0x7B}), \RU{а также адреса двух глобальных переменных}\EN{and also the addresses of the two global variables}.

\clearpage
\RU{Трассируем до конца}\EN{Let's trace until the end of} \ttfone.
\RU{Мы видим в окне слева, как результаты вычисления появились в глобальных переменных}%
\EN{In the left bottom window we see how the results of the calculation appear in the global variables}: 

\begin{figure}[H]
\centering
\includegraphics[scale=\FigScale]{patterns/061_pointers/olly_global3.png}
\caption{\olly: \ttfone \RU{заканчивает работу}\EN{execution completed}}
\label{fig:pointers_olly_global_3}
\end{figure}

\clearpage
\RU{Теперь из глобальных переменных значения загружаются в регистры для передачи в}
\EN{Now the global variables' values are loaded into registers ready for passing to} \printf \EN{(via the stack)}:

\begin{figure}[H]
\centering
\includegraphics[scale=\FigScale]{patterns/061_pointers/olly_global4.png}
\caption{\olly: \RU{адреса глобальных переменных передаются в}
\EN{global variables' addresses are passed into} \printf}
\label{fig:pointers_olly_global_4}
\end{figure}

\section{\RU{Пример с локальными переменными}\EN{Local variables example}}

\RU{Немного переделаем пример}\EN{Let's rework our example slightly}:

\lstinputlisting[caption=\RU{теперь переменные локальные}
\EN{now the \TT{sum} and \TT{product} variables are local}]{patterns/061_pointers/local.c.\LANG}

\RU{Код функции }\ttfone \RU{не изменится}\EN{code will not change}.
\RU{Изменится только \main}\EN{Only the code of \main will do}:

\lstinputlisting[caption=\Optimizing MSVC 2010 (/Ob0)]{patterns/061_pointers/local.asm}

\newcommand{\PtrsAddresses}{\TT{0x2EF854} \AndENRU \TT{0x2EF858}\xspace}

\clearpage
\RU{Снова посмотрим в}\EN{Let's look again with} \olly.
\RU{Адреса локальных переменных в стеке это}\EN{The addresses of the local variables in the stack are} \PtrsAddresses.
\RU{Видно, как они заталкиваются в стек}\EN{We see how these are pushed into the stack}: 

\begin{figure}[H]
\centering
\includegraphics[scale=\FigScale]{patterns/061_pointers/olly_stk1.png}
\caption{\olly: \RU{адреса локальных переменных заталкиваются в стек}\EN{local variables' addresses are
pushed into the stack}}
\label{fig:pointers_olly_stk_1}
\end{figure}

\clearpage
\RU{Начало работы \ttfone}\EN{\ttfone starts}.
\RU{В стеке по адресам}\EN{So far there is only random garbage in the stack at} \PtrsAddresses \RU{пока находится случайный мусор}:

\begin{figure}[H]
\centering
\includegraphics[scale=\FigScale]{patterns/061_pointers/olly_stk2.png}
\caption{\olly: \ttfone \RU{начинает работу}\EN{starting}}
\label{fig:pointers_olly_stk_2}
\end{figure}

\clearpage
\RU{Конец работы \ttfone}\EN{\ttfone completes}:

\begin{figure}[H]
\centering
\includegraphics[scale=\FigScale]{patterns/061_pointers/olly_stk3.png}
\caption{\olly: \ttfone \RU{заканчивает работу}\EN{completes execution}}
\label{fig:pointers_olly_stk_3}
\end{figure}

\RU{В стеке по адресам \PtrsAddresses теперь находятся значения \TT{0xDB18} \AndENRU \TT{0x243}, 
это результаты работы \ttfone.}
\EN{We now find \TT{0xDB18} \AndENRU \TT{0x243} at addresses \PtrsAddresses. These values are
the \ttfone results.}

\section{\Conclusion{}}

\RU{\ttfone может одинаково хорошо возвращать результаты работы в любые места памяти.} 
\EN{\ttfone could return pointers to any place in memory, located anywhere.}
\RU{В этом суть и удобство указателей.}
\EN{This is in essence the usefulness of the pointers.}

\RU{Кстати,}\EN{By the way, \Cpp} \IT{references} \RU{в \Cpp работают точно так же}\EN{work exactly the
same way}. \RU{Читайте больше об этом}\EN{Read more about them}: (\myref{cpp_references}).

\EN{\chapterold{Conditional jumps}
\label{sec:Jcc}
\myindex{\CLanguageElements!if}

% sections
\subsection{\RU{Простой пример}\EN{Simple example}\DEph{}
\FR{Exemple simple}\ITA{Esempio semplice}
}

\lstinputlisting[style=customc]{patterns/07_jcc/simple/ex.c}

% subsections
\input{patterns/07_jcc/simple/x86}
\input{patterns/07_jcc/simple/ARM/main}
\input{patterns/07_jcc/simple/MIPS}

\EN{\input{patterns/07_jcc/abs/main_EN}}
\RU{\input{patterns/07_jcc/abs/main_RU}}
\DE{\input{patterns/07_jcc/abs/main_DE}}
\FR{\input{patterns/07_jcc/abs/main_FR}}

\EN{\input{patterns/07_jcc/cond_operator/main_EN}}
\RU{\input{patterns/07_jcc/cond_operator/main_RU}}


\EN{\input{patterns/07_jcc/minmax/main_EN}}
\RU{\input{patterns/07_jcc/minmax/main_RU}}



\sectionold{\Conclusion{}}

\subsectionold{x86}

Here's the rough skeleton of a conditional jump:

\begin{lstlisting}[caption=x86]
CMP register, register/value
Jcc true ; cc=condition code
false:
... some code to be executed if comparison result is false ...
JMP exit 
true:
... some code to be executed if comparison result is true ...
exit:
\end{lstlisting}

\subsectionold{ARM}

\begin{lstlisting}[caption=ARM]
CMP register, register/value
Bcc true ; cc=condition code
false:
... some code to be executed if comparison result is false ...
JMP exit 
true:
... some code to be executed if comparison result is true ...
exit:
\end{lstlisting}

\subsectionold{MIPS}

\begin{lstlisting}[caption=Check for zero]
BEQZ REG, label
...
\end{lstlisting}

\begin{lstlisting}[caption=Check for less than zero:]
BLTZ REG, label
...
\end{lstlisting}

\begin{lstlisting}[caption=Check for equal values]
BEQ REG1, REG2, label
...
\end{lstlisting}

\begin{lstlisting}[caption=Check for non-equal values]
BNE REG1, REG2, label
...
\end{lstlisting}

\begin{lstlisting}[caption=Check for less than{,} greater than (signed)]
SLT REG1, REG2, REG3
BEQ REG1, label
...
\end{lstlisting}

\begin{lstlisting}[caption=Check for less than{,} greater than (unsigned)]
SLTU REG1, REG2, REG3
BEQ REG1, label
...
\end{lstlisting}

\subsectionold{Branchless}

\myindex{ARM!\Instructions!MOVcc}
\myindex{x86!\Instructions!CMOVcc}
\myindex{ARM!\Instructions!CSEL}
If the body of a condition statement is very short, the conditional move instruction can be used: 
\INS{MOVcc} in ARM (in ARM mode), \INS{CSEL} in ARM64, \INS{CMOVcc} in x86.

\subsubsectionold{ARM}

It's possible to use conditional suffixes in ARM mode for some instructions:

\begin{lstlisting}[caption=ARM (\ARMMode)]
CMP register, register/value
instr1_cc ; some instruction will be executed if condition code is true
instr2_cc ; some other instruction will be executed if other condition code is true
... etc...
\end{lstlisting}

Of course, there is no limit for the number of instructions with conditional code suffixes, 
as long as the CPU flags are not modified by any of them.
% FIXME: list of such instructions or \myref{} to it

\myindex{ARM!\Instructions!IT}

Thumb mode has the \INS{IT} instruction, allowing to add conditional suffixes to the next four instructions.
Read more about it: \myref{ARM_Thumb_IT}.

\begin{lstlisting}[caption=ARM (\ThumbMode)]
CMP register, register/value
ITEEE EQ ; §set these suffixes§: if-then-else-else-else
instr1   ; §instruction will be executed if condition is true§
instr2   ; §instruction will be executed if condition is false§
instr3   ; §instruction will be executed if condition is false§
instr4   ; §instruction will be executed if condition is false§
\end{lstlisting}

\sectionold{\Exercise}

(ARM64) Try rewriting the code in \lstref{cond_ARM64} by removing all 
conditional jump instructions and using the \TT{CSEL} instruction.

}
\RU{\section{Условные переходы}
\label{sec:Jcc}
\myindex{\CLanguageElements!if}

% sections
\subsection{\RU{Простой пример}\EN{Simple example}\DEph{}
\FR{Exemple simple}\ITA{Esempio semplice}
}

\lstinputlisting[style=customc]{patterns/07_jcc/simple/ex.c}

% subsections
\input{patterns/07_jcc/simple/x86}
\input{patterns/07_jcc/simple/ARM/main}
\input{patterns/07_jcc/simple/MIPS}

\EN{\input{patterns/07_jcc/abs/main_EN}}
\RU{\input{patterns/07_jcc/abs/main_RU}}
\DE{\input{patterns/07_jcc/abs/main_DE}}
\FR{\input{patterns/07_jcc/abs/main_FR}}

\EN{\input{patterns/07_jcc/cond_operator/main_EN}}
\RU{\input{patterns/07_jcc/cond_operator/main_RU}}


\EN{\input{patterns/07_jcc/minmax/main_EN}}
\RU{\input{patterns/07_jcc/minmax/main_RU}}



\subsection{\Conclusion{}}

\subsubsection{x86}

Примерный скелет условных переходов:

\begin{lstlisting}[caption=x86,style=customasmx86]
CMP register, register/value
Jcc true ; §cc=код условия§
false:
... код, исполняющийся, если сравнение ложно ...
JMP exit 
true:
... код, исполняющийся, если сравнение истинно ...
exit:
\end{lstlisting}

\subsubsection{ARM}

\begin{lstlisting}[caption=ARM,style=customasmARM]
CMP register, register/value
Bcc true ; §cc=код условия§
false:
... код, исполняющийся, если сравнение ложно ...
JMP exit 
true:
... код, исполняющийся, если сравнение истинно ...
exit:
\end{lstlisting}

\subsubsection{MIPS}

\begin{lstlisting}[caption=Проверка на ноль,style=customasmMIPS]
BEQZ REG, label
...
\end{lstlisting}

\begin{lstlisting}[caption=Меньше ли нуля? (используя псевдоинструкцию),style=customasmMIPS]
BLTZ REG, label
...
\end{lstlisting}

\begin{lstlisting}[caption=Проверка на равенство,style=customasmMIPS]
BEQ REG1, REG2, label
...
\end{lstlisting}

\begin{lstlisting}[caption=Проверка на неравенство,style=customasmMIPS]
BNE REG1, REG2, label
...
\end{lstlisting}

\begin{lstlisting}[caption=Проверка на меньше (знаковое),style=customasmMIPS]
SLT REG1, REG2, REG3
BEQ REG1, label
...
\end{lstlisting}

\begin{lstlisting}[caption=Проверка на меньше (беззнаковое),style=customasmMIPS]
SLTU REG1, REG2, REG3
BEQ REG1, label
...
\end{lstlisting}

\subsubsection{Без инструкций перехода}

\myindex{ARM!\Instructions!MOVcc}
\myindex{x86!\Instructions!CMOVcc}
\myindex{ARM!\Instructions!CSEL}

Если тело условного выражения очень короткое, может быть
использована инструкция условного копирования: \INS{MOVcc} в ARM (в режиме ARM), \INS{CSEL} в ARM64, \INS{CMOVcc} в x86.

\myparagraph{ARM}

В режиме ARM можно использовать условные суффиксы для некоторых инструкций:

\begin{lstlisting}[caption=ARM (\ARMMode),style=customasmARM]
CMP register, register/value
instr1_cc ; инструкция, которая будет исполнена, если условие истинно
instr2_cc ; еще инструкция, которая будет исполнена, если условие истинно
... и т.д....
\end{lstlisting}

Нет никаких ограничений на количество инструкций с условными суффиксами до тех пор,
пока флаги CPU не были модифицированы одной из таких инструкций.

% FIXME: list of such instructions or \myref{} to it

\myindex{ARM!\Instructions!IT}
В режиме Thumb есть инструкция \INS{IT}, позволяющая дополнить следующие 4 инструкции суффиксами, задающими
условие.

Читайте больше об этом: \myref{ARM_Thumb_IT}.

\begin{lstlisting}[caption=ARM (\ThumbMode),style=customasmARM]
CMP register, register/value
ITEEE EQ ; выставить такие суффиксы: if-then-else-else-else
instr1   ; инструкция будет исполнена, если истинно
instr2   ; инструкция будет исполнена, если ложно
instr3   ; инструкция будет исполнена, если ложно
instr4   ; инструкция будет исполнена, если ложно
\end{lstlisting}

\subsection{\Exercise}

(ARM64) Попробуйте переписать код в \lstref{cond_ARM64} 
убрав все инструкции условного перехода, и используйте инструкцию \INS{CSEL}.

}
\ITA{\mysection{Jump condizionali}
\label{sec:Jcc}
\myindex{\CLanguageElements!if}

% sections
\subsection{\RU{Простой пример}\EN{Simple example}\DEph{}
\FR{Exemple simple}\ITA{Esempio semplice}
}

\lstinputlisting[style=customc]{patterns/07_jcc/simple/ex.c}

% subsections
\input{patterns/07_jcc/simple/x86}
\input{patterns/07_jcc/simple/ARM/main}
\input{patterns/07_jcc/simple/MIPS}

\EN{\input{patterns/07_jcc/abs/main_EN}}
\RU{\input{patterns/07_jcc/abs/main_RU}}
\DE{\input{patterns/07_jcc/abs/main_DE}}
\FR{\input{patterns/07_jcc/abs/main_FR}}

\EN{\input{patterns/07_jcc/cond_operator/main_EN}}
\RU{\input{patterns/07_jcc/cond_operator/main_RU}}


\EN{\input{patterns/07_jcc/minmax/main_EN}}
\RU{\input{patterns/07_jcc/minmax/main_RU}}



% Do not translate, this is macro:
\subsection{\Conclusion{}}

\subsubsection{x86}

La forma grezza di un jump condizionale e' la seguente:

\begin{lstlisting}[caption=x86,style=customasmx86]
CMP register, register/value
Jcc true ; cc=condition code
false:
... codice da eseguire se il risultato del confronto e' false ...
JMP exit 
true:
... codice da eseguire se il risultato del confronto e' true ...
exit:
\end{lstlisting}

\subsubsection{ARM}

\begin{lstlisting}[caption=ARM,style=customasmARM]
CMP register, register/value
Bcc true ; cc=condition code
false:
... codice da eseguire se il risultato del confronto e' false ...
JMP exit 
true:
... codice da eseguire se il risultato del confronto e' true ...
exit:
\end{lstlisting}

\subsubsection{MIPS}

\begin{lstlisting}[caption=Check for zero,style=customasmMIPS]
BEQZ REG, label
...
\end{lstlisting}

\begin{lstlisting}[caption=Check for less than zero (using pseudoinstruction),style=customasmMIPS]
BLTZ REG, label
...
\end{lstlisting}

\begin{lstlisting}[caption=Check for equal values,style=customasmMIPS]
BEQ REG1, REG2, label
...
\end{lstlisting}

\begin{lstlisting}[caption=Check for non-equal values,style=customasmMIPS]
BNE REG1, REG2, label
...
\end{lstlisting}

\begin{lstlisting}[caption=Check for less than (signed),style=customasmMIPS]
SLT REG1, REG2, REG3
BEQ REG1, label
...
\end{lstlisting}

\begin{lstlisting}[caption=Check for less than (unsigned),style=customasmMIPS]
SLTU REG1, REG2, REG3
BEQ REG1, label
...
\end{lstlisting}

\subsubsection{Branchless}

\myindex{ARM!\Instructions!MOVcc}
\myindex{x86!\Instructions!CMOVcc}
\myindex{ARM!\Instructions!CSEL}
Se il corpo di uno statement condizionale e' molto piccolo, puo' essere utilizzata l'istruzione "move" condizionale: 
\INS{MOVcc} in ARM (in ARM mode), \INS{CSEL} in ARM64, \INS{CMOVcc} in x86.

\myparagraph{ARM}

In ARM e' possibile usare suffissi condizionali per alcune istruzioni:

\begin{lstlisting}[caption=ARM (\ARMMode),style=customasmARM]
CMP register, register/value
instr1_cc ; istruzione che sara' eseguita se il condition code e' true
instr2_cc ; altra istruzione che sara' eseguita se il condition code e' true
... etc...
\end{lstlisting}

Ovviamente non c'e' limite al numero di istruzioni con il suffisso condizionale, a patto che le flag CPU non siano modificate da nessuna istruzione. 
% FIXME: list of such instructions or \myref{} to it

\myindex{ARM!\Instructions!IT}

La modalita' Thumb ha l'istruzione \INS{IT}, che permette di aggiungere suffissi condizionali alle prossime quattro istruzioni.
Maggiori informazioni qui: \myref{ARM_Thumb_IT}.

\begin{lstlisting}[caption=ARM (\ThumbMode),style=customasmARM]
CMP register, register/value
ITEEE EQ ; set these suffixes: if-then-else-else-else
instr1   ; istruzione da eseguire se la condizione e' true
instr2   ; istruzione da eseguire se la condizione e' false
instr3   ; istruzione da eseguire se la condizione e' false
instr4   ; istruzione da eseguire se la condizione e' false
\end{lstlisting}

% Do not translate, this is macro:
\subsection{\Exercise}

(ARM64) Prova a riscrivere il codice in \lstref{cond_ARM64} rimuovendo tutti i jump condizionali e usando al loro posto l'istruzione \TT{CSEL} instruction.
}

\chapter{\SwitchCaseDefaultSectionName}
\index{\CLanguageElements!switch}

% sections
\section{\RU{Если вариантов мало}\EN{Few number of cases}}

\lstinputlisting{patterns/08_switch/1_few/few.c}

\subsection{x86}

\subsubsection{\NonOptimizing MSVC}

\RU{Это дает в итоге}\EN{Result} (MSVC 2010):

\lstinputlisting[caption=MSVC 2010]{patterns/08_switch/1_few/few_msvc.asm}

\RU{Наша функция со switch()-ем, с небольшим количеством вариантов, 
это практически аналог подобной конструкции:}
\EN{Our function with a few cases in switch(), in fact, is analogous to this construction:}

\lstinputlisting[label=switch_few_ifelse]{patterns/08_switch/1_few/few_analogue.c}

\index{\CLanguageElements!switch}
\index{\CLanguageElements!if}
\RU{Когда вариантов немного, и мы видим подобный код, невозможно сказать с уверенностью, был ли
в оригинальном исходном коде switch(), либо просто набор if()-ов.}
\EN{If we work with switch() with a few cases, it is impossible to be sure, was it
real switch() in source code, or just pack of if() statements.}
\index{\SyntacticSugar}
\RU{То есть, switch() это синтаксический сахар для большого количества вложенных проверок 
при помощи if().}
\EN{This means, switch() is like syntactic sugar for large number of nested checks constructed using if().}

\RU{В самом выходном коде, в принципе, ничего особо нового для нас здесь, 
за исключением того, что компилятор зачем-то 
перекладывает входящую переменную ($a$) во временную в локальном стеке \TT{v64}
\footnote{Локальные переменные в стеке с префиксом \TT{tv} --- 
так MSVC называет внутренние переменные для своих нужд}.}
\EN{Nothing especially new to us in generated code,
with the exception the compiler moving 
input variable 
$a$ to temporary local variable \TT{tv64}
\footnote{Local variables in stack prefixed with \TT{tv} --- 
that's how MSVC names internal variables for its needs}.}

\RU{Если скомпилировать это при помощи GCC 4.4.1, то будет почти то же самое, даже с максимальной оптимизацией 
(ключ \Othree).}
\EN{If to compile the same in GCC 4.4.1, we'll get almost the same, even with maximal optimization 
turned on (\Othree option).}

\subsubsection{\Optimizing MSVC}

\RU{Попробуем, включить оптимизацию кодегенератора}
\EN{Now let's turn on optimization in} MSVC (\Ox): \TT{cl 1.c /Fa1.asm /Ox}

\label{JMP_instead_of_RET}
\lstinputlisting[caption=MSVC]{patterns/08_switch/1_few/few_msvc_Ox.asm}

\RU{Вот здесь уже все немного по-другому, причем не без грязных хаков.}
\EN{Here we can see some dirty hacks.}

\index{x86!\Instructions!JZ}
\index{x86!\Instructions!JE}
\index{x86!\Instructions!SUB}
\RU{Первое: \TT{а} помещается в \EAX и от него отнимается 0. Звучит абсурдно, но нужно это для того, чтобы проверить, 
0 ли в \EAX был до этого? Если да, то выставится флаг \ZF (что означает что результат отнимания $0$ от числа 
стал $0$) и первый условный переход \JE (\IT{Jump if Equal} или его синоним \JZ ~--- \IT{Jump if Zero}) 
сработает на метку \TT{\$LN4@f}, где выводится сообщение \TT{'zero'}.
Если первый переход не сработал, от значения отнимается по единице, 
и если на какой-то стадии образуется в результате $0$, то сработает соответствующий переход.}
\EN{First: the value of the $a$ variable is placed into \EAX and $0$ subtracted from it. Sounds absurd, but it may needs to check if 
$0$ was in the \EAX register before? If yes, flag \ZF will be set (this also means that subtracting from $0$ is $0$) 
and first conditional jump \JE (\IT{Jump if Equal} or synonym \JZ~---\IT{Jump if Zero}) will be triggered 
and control flow passed to the \TT{\$LN4@f} label, where \TT{'zero'} message is being printed. 
If first jump was not triggered, $1$ subtracted from the input value and if at some stage $0$ will be resulted, 
corresponding jump will be triggered.}

\RU{И в конце концов, если ни один из условных переходов не сработал, управление передается \printf
со строковым аргументом \TT{'something unknown'}.}
\EN{And if no jump triggered at all, control flow passed to the \printf with argument \TT{'something unknown'} string.}

\label{jump_to_last_printf}
\index{\Stack}
\RU{Второе: мы видим две, мягко говоря, необычные вещи: указатель на сообщение помещается в переменную $a$, 
и затем \printf вызывается не через \CALL, а через \JMP. Объяснение этому простое. 
Вызывающая функция заталкивает в стек некоторое значение и через \CALL вызывает нашу функцию. 
\CALL в свою очередь заталкивает в стек адрес возврата (\ac{RA}) и делает безусловный переход на адрес нашей функции. 
Наша функция в самом начале (да и в любом её месте, потому что в теле функции нет ни одной инструкции, 
которая меняет что-то в стеке или в \ESP) имеет следующую разметку стека:}
\EN{Second: we see unusual thing for us: string pointer is placed into the $a$ variable, and 
then \printf is called not via \CALL, but via \JMP. This could be explained simply. 
\Gls{caller} pushing to stack a value and calling our function via \CALL. 
\CALL itself pushing returning address (\ac{RA}) to stack and do unconditional jump to our function address. 
Our function at any point of execution (since it do not contain any instruction moving stack 
pointer) has the following stack layout:}

\begin{itemize}
\item\ESP\EMDASH\RU{хранится}\EN{pointing to} \ac{RA}
\item\TT{ESP+4}\EMDASH\RU{хранится значение $a$}\EN{pointing to the $a$ variable} 
\end{itemize}

\RU{С другой стороны, чтобы вызвать \printf нам нужна почти такая же разметка стека, 
только в первом аргументе нужен указатель на строку. Что, собственно, этот код и делает.}
\EN{On the other side, when we need to call \printf here, we need exactly the same stack 
layout, except of first \printf argument pointing to string. 
And that is what our code does.}

\RU{Он заменяет свой первый аргумент на адрес строки, и затем передает управление \printf, как если бы вызвали не 
нашу функцию \ttf, а сразу \printf. 
\printf выводит некую строку на \gls{stdout}, затем исполняет инструкцию \RET, 
которая из стека достает \ac{RA} и управление передается в ту функцию, 
которая вызывала \ttf, минуя при этом конец ф-ции \ttf.}
\EN{It replaces function's first argument to address of the string and 
jumping to the \printf, as if not our function \ttf was called firstly, but immediately \printf.
\printf printing a string to \gls{stdout} and then execute \RET instruction, which POPping 
\ac{RA} from stack and control flow is returned not to \ttf but rather to the \ttf's \gls{callee}, 
bypassing end of \ttf function.}

\index{\CStandardLibrary!longjmp()}
\newcommand{\URLSJ}{\url{http://go.yurichev.com/17121}}
\RU{Все это возможно потому что \printf вызывается в \ttf в самом конце. 
Все это чем-то даже похоже на \TT{longjmp()}\footnote{\URLSJ}.
И все это, разумеется, сделано для экономии времени исполнения.}
\EN{All this is possible since \printf is called right at the end of the \ttf function in any case. 
In some way, it is all similar to the \TT{longjmp()}\footnote{\URLSJ} function.
And of course, it is all done for the sake of speed.}

\ifdefined\IncludeARM
\RU{Похожая ситуация с компилятором для ARM описана в секции}
\EN{Similar case with ARM compiler described in} ``\PrintfSeveralArgumentsSectionName'', 
\RU{здесь}\EN{section, here}~(\ref{ARM_B_to_printf}).
\fi

\ifdefined\IncludeOlly
\clearpage
\myparagraph{\olly}

\RU{Так как этот пример немного запутанный, попробуем оттрассировать его в}\EN{Since this example is tricky, 
let's trace it in} \olly.\\
\\
\olly \RU{может распознавать подобные switch()-конструкции, так что он добавляет полезные комментарии}\EN{can 
detect such switch() constructs, so its add some useful comments}.
\EAX \RU{в начале}\EN{is} $2$\EN{ at start}, \RU{это входное значение ф-ции}\EN{that's function's input value}: 

\begin{figure}[H]
\centering
\includegraphics[scale=\FigScale]{patterns/08_switch/1_few/few_olly1.png}
\caption{\olly: \EAX \RU{содержит первый (и единственный) аргумент ф-ции}
\EN{now contain first (and sole) function argument}}
\label{fig:switch_few_olly1}
\end{figure}

\clearpage
$0$ \RU{отнимается от}\EN{is subtracted from} $2$ \InENRU \EAX. 
\RU{Конечно же}\EN{Of course}, \EAX \RU{все еще содержит}\EN{is still contain} $2$.
\RU{Но флаг}\EN{But} \ZF \RU{теперь}\EN{flag is now} $0$, \RU{что означает что последнее вычисленное значение
не было нулевым}\EN{indicating that resulting value is non-zero}:

\begin{figure}[H]
\centering
\includegraphics[scale=\FigScale]{patterns/08_switch/1_few/few_olly2.png}
\caption{\olly: \SUB \RU{исполнилась}\EN{executed}}
\label{fig:switch_few_olly2}
\end{figure}

\clearpage
\DEC \RU{исполнилась и}\EN{is executed and} \EAX \RU{теперь содержит}\EN{now contain} $1$. 
\RU{Но}\EN{But} $1$ \RU{не ноль, так что флаг}\EN{is non-zero, so the} \ZF \RU{все еще}\EN{flag is still} $0$:

\begin{figure}[H]
\centering
\includegraphics[scale=\FigScale]{patterns/08_switch/1_few/few_olly3.png}
\caption{\olly: \RU{первая}\EN{first} \DEC \RU{исполнилась}\EN{executed}}
\label{fig:switch_few_olly3}
\end{figure}

\clearpage
\RU{Следующая}\EN{Next} \DEC \RU{исполнилась}\EN{is executed}. 
\EAX \RU{наконец}\EN{is finally} $0$ \RU{и флаг}\EN{and} \ZF \RU{выставлен, потому что результат --- ноль}\EN{flag
is set, because the result is zero}:

\begin{figure}[H]
\centering
\includegraphics[scale=\FigScale]{patterns/08_switch/1_few/few_olly4.png}
\caption{\olly: \RU{вторая}\EN{second} \DEC \RU{исполнилась}\EN{executed}}
\label{fig:switch_few_olly4}
\end{figure}

\olly \RU{показывает, что условный переход сейчас сработатет}\EN{shows that this jump will be taken now}.

\clearpage
\RU{Указатель на строку}\EN{A pointer to the string} ``two'' \RU{сейчас будет записан в стек}\EN{will now be 
written into the stack}:

\begin{figure}[H]
\centering
\includegraphics[scale=\FigScale]{patterns/08_switch/1_few/few_olly5.png}
\caption{\olly: \RU{указатель на строку сейчас запишется на место первого аргумента}
\EN{pointer to the string is to be written at the place of first argument}}
\label{fig:switch_few_olly5}
\end{figure}

\RU{Обратите внимание: текущий аргумент ф-ции это $2$ и $2$ прямо сейчас в стеке по адресу}\EN{Please note: 
current argument of the function is $2$ and $2$ is now in the stack at the address} \TT{0x0020FA44}.

\clearpage
\MOV \RU{записывает указатель на строку по адресу}\EN{wrote pointer to the string at the address} 
\TT{0x0020FA44} (\RU{см. окно стека}\EN{see stack window}).
\RU{Переход сработал}\EN{Jump is happen}.
\RU{Это самая первая инструкция ф-ции}\EN{This is the first instruction of} \printf \RU{в}\EN{function in} 
MSVCR100.DLL (\RU{я скомпилировал этот пример с опцией /MD}\EN{I compiled the example with /MD switch}): 

\begin{figure}[H]
\centering
\includegraphics[scale=\FigScale]{patterns/08_switch/1_few/few_olly6.png}
\caption{\olly: \RU{первая инструкция в}\EN{first instruction of} \printf \InENRU MSVCR100.DLL}
\label{fig:switch_few_olly6}
\end{figure}

\RU{Теперь}\EN{Now the} \printf \RU{будет считать строку на}\EN{will treat the string at} \TT{0x0020FA44} 
\RU{как свой единственный аргумент и выведет строку}\EN{as its sole argument and will print the string}.

\clearpage
\RU{Это самая последняя инструкция ф-ции}\EN{This is the very last instruction of} \printf:

\begin{figure}[H]
\centering
\includegraphics[scale=\FigScale]{patterns/08_switch/1_few/few_olly7.png}
\caption{\olly: \RU{последняя инструкция в}\EN{last instruction of} \printf \InENRU MSVCR100.DLL}
\label{fig:switch_few_olly7}
\end{figure}

\RU{Строка }``two'' \RU{была только что выведена в консоли}\EN{string was just printed to the console window}.

\clearpage
\RU{Нажмем}\EN{Let's press} F7 \OrENRU F8 (\stepover) \RU{и вернемся}\EN{and we will return}\dots
\RU{нет, не в ф-цию}\EN{not to} \ttf \RU{но в}\EN{function, but rather to the} \main:

\begin{figure}[H]
\centering
\includegraphics[scale=\FigScale]{patterns/08_switch/1_few/few_olly8.png}
\caption{\olly: \RU{возврат в}\EN{return to} \main}
\label{fig:switch_few_olly8}
\end{figure}

\RU{Да, это прямой переход из внутренностей}\EN{Yes, the jump was direct, from the guts of} \printf 
\RU{в}\EN{to} \main.
\RU{Потому как}\EN{Because} \ac{RA} \RU{в стеке указывает не на какое-то место в ф-ции}\EN{in the stack pointed 
not to some place in} \ttf \RU{а в}\EN{function, but rather to} \main.
\RU{И}\EN{And} \CALL \TT{0x01201000} \RU{это инструкция вызывающая ф-цию}\EN{was the actual instruction which called} 
\ttf\EN{ function}.

\fi

\subsection{ARM: \OptimizingKeilVI (\ARMMode)}
\index{\CLanguageElements!switch}

\lstinputlisting{patterns/08_switch/1_few/few_ARM_ARM_O3.asm}

\RU{Мы снова не сможем сказать, глядя на этот код, был ли в оригинальном исходном коде switch() 
либо же несколько if()-в.}
\EN{Again, by investigating this code, we cannot say, was it switch() in the original source code, 
or pack of if() statements.}

\index{ARM!\Instructions!ADRcc}
\RU{Так или иначе, мы снова видим здесь инструкции с предикатами, например, \ADREQ (\IT{(Equal)}), 
которая будет исполняться только
если $R0=0$, и тогда, в \Reg{0} будет загружен адрес строки \IT{<<zero\textbackslash{}n>>}.}
\EN{Anyway, we see here predicated instructions again (like \ADREQ (\IT{Equal}))
which will be triggered only in $R0=0$ case, and the, address of the \IT{<<zero\textbackslash{}n>>}
string will be loaded into the \Reg{0}.}
\index{ARM!\Instructions!BEQ}
\RU{Следующая инструкция}\EN{The next instruction} \ac{BEQ}
\RU{перенаправит исполнение на}\EN{will redirect control flow to} \TT{loc\_170}, \RU{если}\EN{if} $R0=0$.
\RU{Кстати, наблюдательный читатель может спросить, сработает ли \ac{BEQ} нормально,
ведь \ADREQ перед ним уже заполнила регистр \Reg{0} чем-то другим.}
\EN{By the way, astute reader may ask, will \ac{BEQ} triggered right since \ADREQ before it
is already filled the \Reg{0} register with another value.}
\RU{Сработает, потому что \ac{BEQ} проверяет флаги, установленные инструкцией \CMP, 
а \ADREQ флаги никак не модифицирует.}
\EN{Yes, it will since \ac{BEQ} checking flags set by \CMP instruction, and \ADREQ not modifying flags
at all.}

\RU{Кстати, в ARM имеется также для некоторых инструкций суффикс \IT{-S}, указывающий, 
что эта инструкция будет модифицировать флаги, а при отсутствии суффикса ~--- не будет.}
\EN{By the way, there is \IT{-S} suffix for some instructions in ARM,
indicating the instruction will set the flags according to the result, and without 
it~---the flags will not be touched.}
\index{ARM!\Instructions!ADD}
\index{ARM!\Instructions!ADDS}
\index{ARM!\Instructions!CMP}
\RU{Например, инструкция}\EN{For example} \TT{ADD} \RU{в отличие от}\EN{unlike} \TT{ADDS}
\RU{сложит два числа, но флаги не изменит}
\EN{will add two numbers, but flags will not be touched}.
\RU{Такие инструкции удобно использовать
между \CMP где выставляются флаги и, например, инструкциями перехода, где флаги используются.}
\EN{Such instructions are convenient to use between \CMP where flags are set and, 
e.g. conditional jumps, where flags are used.}

\RU{Далее всё просто и знакомо.}\EN{Other instructions are already familiar to us.} 
\RU{Вызов}\EN{There is only one call to} \printf \RU{один, и в самом конце, 
мы уже рассматривали подобный трюк здесь}\EN{, at the end, and we already examined this trick here}
~(\ref{ARM_B_to_printf}).
\RU{К}\EN{There are three paths to} \printf{}\RU{-у в конце ведут три пути}\EN{at the end}.

\RU{Обратите внимание на то что происходит если $a=2$ и если $a$ не попадает под сравниваемые константы.}
\EN{Also pay attention to what is going on if $a=2$ and if $a$ is not in range of constants it is comparing against.}
\index{ARM!\Instructions!ADRcc}
\index{ARM!\Instructions!CMP}
\RU{Инструкция }\TT{``CMP R0, \#2''} \RU{нужна чтобы узнать $a=2$ или нет}\EN{instruction is needed here
to know, if $a=2$ or not}.
\RU{Если это не так, то при помощи \ADRNE (\IT{Not Equal}) в \Reg{0} будет загружен указатель на 
строку \IT{<<something unknown \textbackslash{}n>>}, ведь $a$ уже было проверено на $0$ и $1$ до этого, 
и здесь $a$ точно не попадает под эти константы.}
\EN{If it is not true, then \ADRNE will load pointer to the string \IT{<<something unknown \textbackslash{}n>>} 
into \Reg{0} since $a$ was already
checked before to be equal to $0$ or $1$,
so we can be assured the $a$ variable is not equal to these numbers
at this point.}
\RU{Ну а если}\EN{And if} $R0=2$, \RU{в \Reg{0} будет загружен указатель на строку}\EN{a pointer to string} 
\IT{<<two\textbackslash{}n>>} 
\RU{при помощи инструкции \ADREQ}\EN{will be loaded by \ADREQ into \Reg{0}}.

\subsection{ARM: \OptimizingKeilVI (\ThumbMode)}

\lstinputlisting{patterns/08_switch/1_few/few_ARM_thumb_O3.asm}

\RU{Как я уже писал, в thumb-режиме нет возможности \IT{присоединять} предикаты к большинству инструкций,
так что thumb-код вышел похожим на код x86, вполне понятный.}
\EN{As I already mentioned, there is no feature of \IT{connecting} predicates to majority of instructions in thumb
mode, so the thumb-code here is somewhat similar to the easily understandable x86 \ac{CISC}-code.}

\subsection{ARM64: \NonOptimizing GCC (Linaro) 4.9}

\lstinputlisting{patterns/08_switch/1_few/ARM64_GCC_O0.lst}

\RU{Входное значение имеет тип \Tint поэтому для него используется регистр \RegW{0},
а не целая часть регистра \RegX{0}.}
\EN{Input value has \Tint type, hence \RegW{0} register is used as input value instead of the whole
\RegX{0} register.}
\RU{Указатели на строки передаются в \puts, как я и показывал в примере}\EN{String pointers are passed to 
\puts just like I showed in} ``\HelloWorldSectionName''\EN{ example}: \ref{pointers_ADRP_and_ADD}.

\subsection{ARM64: \Optimizing GCC (Linaro) 4.9}

\lstinputlisting{patterns/08_switch/1_few/ARM64_GCC_O3.lst}

\RU{Фрагмент кода более оптимизированный}\EN{Better optimized piece of code}.
\RU{Инструкция }\TT{CBZ} (\IT{Compare and Branch on Zero}\RU{ (сравнить и перейти если ноль)}) 
\RU{совершает переход если}\EN{instruction do jump if} \RegW{0} \RU{ноль}\EN{is zero}.
\RU{Здесь также прямой переход на}\EN{There is also direct jump to} \puts \RU{вместо вызова}\EN{instead 
of calling it}: \ref{JMP_instead_of_RET}.


\EN{\subsection{A lot of cases}

If a \TT{switch()} statement contains a lot of cases, it is not very convenient for the compiler to emit too large code
with a lot \JE/\JNE instructions.

\lstinputlisting[label=switch_lot_c,style=customc]{patterns/08_switch/2_lot/lot.c}

\input{patterns/08_switch/2_lot/lot_x86_EN}
\input{patterns/08_switch/2_lot/lot_ARM_EN}
\input{patterns/08_switch/2_lot/lot_MIPS_EN}

\subsubsection{\Conclusion{}}

Rough skeleton of \IT{switch()}:

% TODO: ARM, MIPS skeleton
\lstinputlisting[caption=x86,style=customasmx86]{patterns/08_switch/2_lot/skel1_EN.lst}

The jump to the address in the jump table may also be implemented using this instruction: \\
\TT{JMP jump\_table[REG*4]}.
Or \TT{JMP jump\_table[REG*8]} in x64.

A \IT{jumptable} is just array of pointers, like the one described later: \myref{array_of_pointers_to_strings}.
}
\RU{\subsection{И если много}

Если ветвлений слишком много, то генерировать слишком длинный код с многочисленными \JE/\JNE 
уже не так удобно.

\lstinputlisting[label=switch_lot_c,style=customc]{patterns/08_switch/2_lot/lot.c}

\input{patterns/08_switch/2_lot/lot_x86_RU}
\input{patterns/08_switch/2_lot/lot_ARM_RU}
\input{patterns/08_switch/2_lot/lot_MIPS_RU}

\subsubsection{\Conclusion{}}

Примерный скелет оператора \IT{switch()}:

% TODO: ARM, MIPS skeleton
\lstinputlisting[caption=x86,style=customasmx86]{patterns/08_switch/2_lot/skel1_RU.lst}

Переход по адресу из таблицы переходов может быть также реализован такой инструкцией: \\
\TT{JMP jump\_table[REG*4]}. Или \TT{JMP jump\_table[REG*8]} в x64.

Таблица переходов (\IT{jumptable}) это просто массив указателей, как это будет вскоре описано: \myref{array_of_pointers_to_strings}.
}
\DE{\subsection{Viele Fälle}
Wenn ein \TT{switch()} Ausdruck viele Fälle enthält, ist es für den Compiler nicht günstig sehr großen Code mit vielen
\JE/\JNE Befehlen zu erzeugen.

\lstinputlisting[label=switch_lot_c,style=customc]{patterns/08_switch/2_lot/lot.c}

\input{patterns/08_switch/2_lot/lot_x86_DE}
\input{patterns/08_switch/2_lot/lot_ARM_DE}
\input{patterns/08_switch/2_lot/lot_MIPS_DE}

\subsubsection{\Conclusion{}}

Das grobe Gerüst eines \IT{switch()}:

% TODO: ARM, MIPS skeleton
\lstinputlisting[caption=x86,style=customasmx86]{patterns/08_switch/2_lot/skel1_DE.lst}
Der Sprung zur Adresse in der Jumptable kann auch durch den folgenden Befehl realisiert werden:\\
\TT{JMP jump\_table[REG*4]}
oder \TT{JMP jump\_table[REG*8]} in x64.

Eine \IT{Jumptable} ist nur ein Array von Pointern, genau wie das hier beschriebene:
\myref{array_of_pointers_to_strings}.
}
\FR{\subsection{De nombreux cas}

Si une déclaration \TT{switch()} contient beaucoup de cas, il n'est pas très pratique
pour le compilateur de générer un trop gros code avec de nombreuses instructions
\JE/\JNE.

\lstinputlisting[label=switch_lot_c,style=customc]{patterns/08_switch/2_lot/lot.c}

\input{patterns/08_switch/2_lot/lot_x86_FR}
\input{patterns/08_switch/2_lot/lot_ARM_FR}
\input{patterns/08_switch/2_lot/lot_MIPS_FR}

\subsubsection{\Conclusion{}}

Squelette grossier d'un \IT{switch()}:

% TODO: ARM, MIPS skeleton
\lstinputlisting[caption=x86,style=customasmx86]{patterns/08_switch/2_lot/skel1_FR.lst}

Le saut a une adresse de la table de saut peut aussi être implémenté en utilisant
cette instruction: \\
\TT{JMP jump\_table[REG*4]}.
Ou \TT{JMP jump\_table[REG*8]} en x64.

Une table de saut est juste un tableau de pointeurs, comme celle décrite plus
loin: \myref{array_of_pointers_to_strings}. 
}


\section{\RU{Когда много \IT{case} в одном блоке}
\EN{When there are several \IT{case} in one block}}

\RU{Вот очень часто используемая конструкция: несколько \IT{case} может быть использовано в одном блоке:}
\EN{Here is also a very often used construction: several \IT{case} statements may be used in single block:}

\lstinputlisting{patterns/08_switch/3_several_cases/several_cases.c}

\RU{Слишком расточительно генерировать каждый блок для каждого случая, поэтому обычно
каждый блок генерируется плюс диспетчер.}
\EN{It's too wasteful to generate each block for each possible case,
so what is usually done, is each block generated plus some kind of dispatcher.}

\subsection{MSVC}

\lstinputlisting[caption=\Optimizing MSVC 2010,numbers=left]{patterns/08_switch/3_several_cases/several_cases_MSVC_2010_Ox.asm}

\RU{Здесь видим две таблицы}\EN{We see two tables here}: 
\RU{первая таблица}\EN{the first table} (\TT{\$LN10@f}) \RU{это таблица индексов}\EN{is index table},
\RU{и вторая таблица}\EN{and the second table} (\TT{\$LN11@f}) \RU{это массив указателей на блоки}\EN{is 
an array of pointers to blocks}.

\RU{В начале, входное значение используется как индекс в таблице индексов}\EN{First, input value 
is used as index in index table} (\LineENRU 13). 

\RU{Вот краткое описание значений в таблице}\EN{Here is short legend for values in the table}: 
0 \RU{это первый блок \IT{case}}\EN{is first \IT{case} block} (\RU{для значений}\EN{for values} 1, 2, 7, 10),
1 \RU{это второй}\EN{is second} (\RU{для значений}\EN{for values} 3, 4, 5),
2 \RU{это третий}\EN{is third} (\RU{для значений}\EN{for values} 8, 9, 21),
3 \RU{это четвертый}\EN{is fourth} (\RU{для значений}\EN{for value} 22),
4 \RU{это для default-блока}\EN{is for default block}.

\RU{Мы получаем индекс для второй таблицы указателей на блоки и переходим туда}\EN{We get there index for 
the second table of block pointers and we we jump there} (\LineENRU 14).

\EN{What is also worth to note that there are no case for input value $0$.}
\RU{Что еще нужно отметить, так это то что здесь нет случая для нулевого входного значения.}
\EN{Hence, we see \DEC instruction at line 10, and the table is beginning at $a=1$. 
Because there are no need to allocate table element for $a=0$.}
\RU{Поэтому мы видим инструкцию \DEC на строке 10 и таблица начинается с $a=1$.
Потому что незачем выделять в таблице элемент для $a=0$.}

\RU{Это очень часто используемый шаблон}\EN{This is very often used pattern}.

\RU{В чем же экономия}\EN{So where economy is}?
\RU{Почему нельзя сделать так, как уже обсуждалось}\EN{Why it's not possible to make it as it was 
already discussed} (\ref{switch_lot_GCC}), \RU{используя только одну таблицу, содержащую указатели на 
блоки}\EN{just with one table, consisting of block pointers}?
\RU{Причина в том что элементы в таблице индексов занимают только по 8-битному байту, поэтому всё это более 
компактно}\EN{The reason is because elements in index table has 8-bit byte type, hence it's all more compact}.

\subsection{GCC}

GCC \RU{делает так, как уже обсуждалось}\EN{do the job like it was already discussed} 
(\ref{switch_lot_GCC}), \RU{используя просто таблицу указателей}\EN{using just one table of pointers}.

\section{Fall-through}

\RU{Ещё одно популярное использование оператора}\EN{Another very popular usage of} \TT{switch()} 
\EN{is the fall-through}\RU{это т.н. \q{fallthrough} (\q{провал})}.
\RU{Вот простой пример}\EN{Here is a small example}:

\lstinputlisting[numbers=left]{patterns/08_switch/4_fallthrough/fallthrough.c}

\RU{Если}\EN{If} $type=1$ (R), $read$ \RU{будет выставлен в}\EN{is to be set to} 1, \RU{если}\EN{if} 
$type=2$ (W), $write$ \RU{будет выставлен в}\EN{is to be set to} 2.
\RU{В случае}\EN{In case of} $type=3$ (RW), \RU{обе}\EN{both} $read$ \AndENRU $write$ \RU{будут 
выставлены в}\EN{is to be set to} 1.

\RU{Фрагмент кода на строке 14 будет исполнен в двух случаях: если}\EN{The code at 
line 14 is executed in two cases: if} $type=RW$ \RU{или если}\EN{or if} $type=W$.
\RU{Там нет \q{break} для \q{case RW}, и это нормально}\EN{There is no \q{break} 
for \q{case RW}x and that's OK}.

\subsection{MSVC x86}

\lstinputlisting[caption=MSVC 2012]{patterns/08_switch/4_fallthrough/fallthrough_MSVC.asm}

\RU{Код почти полностью повторяет то, что в исходнике.}
\EN{The code mostly resembles what is in the source.}
\RU{Там нет переходов между метками}\EN{There are no jumps between labels} \TT{\$LN4@f} \AndENRU 
\TT{\$LN3@f}: \RU{так что когда управление (code flow) находится на}\EN{so when code flow is at} 
\TT{\$LN4@f}, $read$ \RU{в начале выставляется в 1, затем}\EN{is first set to 1, then} $write$.
\EN{This is why it's called fall-through: code flow falls through one piece of code
(setting $read$) to another (setting $write$).}
\RU{Наверное, поэтому всё это и называется \q{проваливаться}: управление проваливается через
один фрагмент кода (выставляющий $read$) в другой (выставляющий $write$).}
\RU{Если}\EN{If} $type=W$, \RU{мы оказываемся на}\EN{we land at} \TT{\$LN3@f}, 
\RU{так что код выставляющий $read$ в 1 не исполнится}\EN{so no code setting $read$ to 1 
is executed}.

\ifdefined\IncludeARM
\subsection{ARM64}

\lstinputlisting[caption=GCC (Linaro) 4.9]{patterns/08_switch/4_fallthrough/fallthrough_ARM64.s.\LANG}

\RU{Почти то же самое}\EN{Merely the same thing}.
\RU{Здесь нет переходов между метками}\EN{There are no jumps between labels} \TT{.L4} 
\AndENRU \TT{.L3}.
\fi


\ifdefined\IncludeExercises
\section{\Exercises}

\subsection{\Exercise \#1}
\label{exercise_switch_1}

\RU{Вполне возможно переделать пример на Си в листинге \ref{switch_lot_c} так, чтобы при компиляции
получалось даже еще меньше кода, но работать всё будет точно так же.}
\EN{It's possible to rework the C example in \ref{switch_lot_c} in such way that the compiler
will produce even smaller code, but will work just the same.}
\RU{Попробуйте этого добиться}\EN{Try to achieve it}.

\RU{Подсказка}\EN{Hint}: \ref{exercise_solutions_switch_1}.
\fi

\mysection{\Loops}
\label{sec:loops}

% sections
\section{\RU{Простой пример}\EN{Simple example}}

% subsections
\subsection{x86}

\index{x86!\Instructions!LOOP}
\RU{Для организации циклов в архитектуре x86 есть старая инструкция \LOOP. 
Она проверяет значение регистра \ECX и если оно не 0, делает \glslink{decrement}{декремент} \ECX 
и переход по метке, указанной в операнде. 
Возможно, эта инструкция не слишком удобная, потому что уже почти не бывает современных компиляторов, 
которые использовали бы её. Так что если вы видите где-то \LOOP, то с большой вероятностью это 
вручную написанный код на ассемблере.}
\EN{There is a special \LOOP instruction in x86 instruction set for checking the value in register \ECX and 
if it is not 0, to \gls{decrement} \ECX
and pass control flow to the label in the \LOOP operand. 
Probably this instruction is not very convenient, and there are no any modern compilers which emit it automatically.
So, if you see this instruction somewhere in code, it is most likely that this is a manually written piece 
of assembly code.}\PTBRph{}\ESph{}\PLph{}\ITAph{}\\
\\
\RU{Обычно, циклы на \CCpp создаются при помощи \TT{for()}, \TT{while()}, \TT{do/while()}.}
\EN{In \CCpp loops are usually constructed using \TT{for()}, \TT{while()} or \TT{do/while()} statements.}

\RU{Начнем с}\EN{Let's start with} \TT{for()}.
\index{\CLanguageElements!for}

\RU{Это выражение описывает инициализацию, условие, операцию после каждой итерации
(\glslink{increment}{инкремент}/\glslink{decrement}{декремент})
и тело цикла.}
\EN{This statement defines loop initialization (set loop counter to initial value), 
loop condition (is the counter bigger than a limit?), what is done at each iteration (\gls{increment}/\gls{decrement})
and of course loop body.}

\lstinputlisting{patterns/09_loops/simple/loops_1.c.\LANG}

\RU{Примерно так же, генерируемый код и будет состоять из этих четырех частей.}
\EN{The generated code is consisting of four parts as well.}

\RU{Возьмем пример}\EN{Let's start with a simple example}:

\lstinputlisting[label=loops_src]{patterns/09_loops/simple/loops_2.c}

\RU{Имеем в итоге}\EN{Result} (MSVC 2010):

\lstinputlisting[caption=MSVC 2010]{patterns/09_loops/simple/1_MSVC.asm.\LANG}

\RU{В принципе, ничего необычного.}\EN{As we see, nothing special.}

\ifdefined\IncludeGCC
\RU{GCC 4.4.1 выдает примерно такой же код, с небольшой разницей:}
\EN{GCC 4.4.1 emits almost the same code, with one subtle difference:}

\lstinputlisting[caption=GCC 4.4.1]{patterns/09_loops/simple/1_GCC.asm.\LANG}

\RU{Интересно становится, если скомпилируем этот же код при помощи MSVC 2010 с включенной оптимизацией}
\EN{Now let's see what we get with optimization turned on} (\Ox):
\fi

\lstinputlisting[caption=\Optimizing MSVC]{patterns/09_loops/simple/1_MSVC_Ox.asm}

\RU{Здесь происходит следующее: переменную $i$ компилятор не выделяет в локальном стеке, 
а выделяет целый регистр под нее: \ESI. 
Это возможно для маленьких функций, где мало локальных переменных.}
\EN{What happens here is that space for the $i$ variable is not allocated in the local stack anymore,
but uses an individual register for it, \ESI.
This is possible in such small functions where there aren't many local variables.}

\RU{В принципе, всё то же самое, только теперь одна важная особенность: 
\ttf не должна менять значение \ESI. 
Наш компилятор уверен в этом, а если бы и была необходимость использовать регистр \ESI в функции \ttf, 
то её значение сохранялось бы в стеке. Примерно так же как и в нашем листинге: 
обратите внимание на \TT{PUSH ESI/POP ESI} в начале и конце функции.}
\EN{One very important thing is that the \ttf function must not change the value in \ESI.
Our compiler is sure here. 
And if the compiler decides to use the \ESI register in \ttf too, its value would have to be saved 
at the function's prologue and restored at the function's epilogue,
almost like in our listing: please note \TT{PUSH ESI/POP ESI}
at the function start and end.}

\ifdefined\IncludeGCC
\RU{Попробуем GCC 4.4.1 с максимальной оптимизацией (\Othree):}
\EN{Let's try GCC 4.4.1 with maximal optimization turned on (\Othree option):}

\lstinputlisting[caption=\Optimizing GCC 4.4.1]{patterns/09_loops/simple/1_GCC_O3.asm}

\index{Loop unwinding}
\RU{Однако GCC просто \IT{развернул} цикл\footnote{\gls{loop unwinding} в англоязычной литературе}.}
\EN{Huh, GCC just unwound our loop.}

\RU{Делается это в тех случаях, когда итераций не слишком много (как в нашем примере)
и можно немного сэкономить время, убрав все инструкции, обеспечивающие цикл. 
В качестве обратной стороны медали, размер кода увеличился.}
\EN{\Gls{loop unwinding} has an advantage in the cases when there aren't much iterations and 
we could cut some execution time by removing all loop support instructions. 
On the other side, the resulting code is obviously larger.}

\EN{Big unrolled loops are not recommended in modern times, because bigger functions
may require bigger cache footprint}%
\RU{Использовать большие развернутые циклы в наше время не рекомендуется, потому что большие
функции требуют больше кэш-памяти}%
\footnote{
\EN{A very good article about it}\RU{Очень хорошая статья об этом}: \cite{DrepperMemory}.
\EN{Another recommendations about loop unrolling from Intel are here}
\RU{А также о рекомендациях о развернутых циклах от Intel можно прочитать здесь}: 
\cite[3.4.1.7]{IntelOptimization}.}.\\
\\
\RU{Увеличим максимальное значение $i$ в цикле до 100 и попробуем снова. GCC выдает:}
\EN{OK, let's increase the maximum value of the $i$ variable to 100 and try again. GCC does:}

\lstinputlisting[caption=GCC]{patterns/09_loops/simple/2_GCC.asm.\LANG}

\RU{Это уже похоже на то, что сделал MSVC 2010 в режиме оптимизации (\Ox). 
За исключением того, что под переменную $i$ будет выделен регистр \EBX.}
\EN{It is quite similar to what MSVC 2010 with optimization (\Ox) produce, 
with the exception that the \EBX register is allocated for the $i$ variable.}
\RU{GCC уверен, что этот регистр не будет 
модифицироваться внутри \ttf, а если вдруг это и придётся там сделать, то его значение будет сохранено 
в начале функции, прямо как в \main.}
\EN{GCC is sure this register will not be modified inside of the \ttf function, 
and if it will, it will be saved at the function prologue and restored at epilogue, 
just like here in the \main function.}
\fi

\ifdefined\IncludeOlly
\clearpage
\subsection{x86: \olly}
\index{\olly}

\RU{Скомпилируем наш пример в}\EN{Let's compile our example in} MSVC 2010 \RU{с}\EN{with} \Ox \AndENRU \Obzero 
\RU{и загрузим в}\EN{options and load it into} \olly.

\RU{Оказывается,}\EN{It seems that} \olly \RU{может обнаруживать простые циклы и показывать их в квадратных скобках, 
для удобства}\EN{is able to detect simple loops and show them in square brackets, for convenience}:

\begin{figure}[H]
\centering
\includegraphics[scale=\FigScale]{patterns/09_loops/simple/olly1.png}
\caption{\olly: \RU{начало \main}\EN{\main begin}}
\label{fig:loops_olly_1}
\end{figure}

\RU{Трассируя}\EN{By tracing} (F8~--- \stepover) \RU{мы видим, как}\EN{we see} \ESI \RU{увеличивается на 1.}
\EN{\glslink{increment}{incrementing}.}
\RU{Например, здесь}\EN{Here, for instance,} $ESI=i=6$:

\begin{figure}[H]
\centering
\includegraphics[scale=\FigScale]{patterns/09_loops/simple/olly2.png}
\caption{\olly: \RU{тело цикла только что отработало с}\EN{loop body just executed with} $i=6$}
\label{fig:loops_olly_2}
\end{figure}

9 \RU{это последнее значение цикла}\EN{is the last loop value}.
\RU{Поэтому}\EN{That's why} \JL 
\RU{после \glslink{increment}{инкремента} не срабатывает и функция заканчивается:}
\EN{is not triggering after the \gls{increment}, and the function will finish:}

\begin{figure}[H]
\centering
\includegraphics[scale=\FigScale]{patterns/09_loops/simple/olly3.png}
\caption{\olly: $ESI=10$, \RU{конец цикла}\EN{loop end}}
\label{fig:loops_olly_3}
\end{figure}

\subsection{x86: tracer}
\index{tracer}

\RU{Как видно, трассировать вручную цикл в отладчике\EMDASH{}это не очень удобно.}%
\EN{As we might see, it is not very convenient to trace manulally in the debugger.}
\RU{Поэтому попробуем \tracer.}%
\EN{That's a reason we will try \tracer.}

\RU{Открываем скомпилированный пример в \IDA, находим там адрес инструкции \INS{PUSH ESI}
(передающей единственный аргумент в \ttf,)
а это \TT{0x401026} в нашем случае и запускаем \tracer:}
\EN{We open compiled example in \IDA, find the address of the instruction \INS{PUSH ESI}
(passing the sole argument to \ttf,) which is \TT{0x401026} for this case and we run the \tracer:}

\begin{lstlisting}
tracer.exe -l:loops_2.exe bpx=loops_2.exe!0x00401026
\end{lstlisting}

\RU{Опция }\TT{BPX} 
\RU{просто ставит точку останова по адресу и затем tracer будет выдавать состояние регистров.}
\EN{just sets a breakpoint at the address and tracer will then print the state of the registers.}

\RU{В}\EN{In the} \TT{tracer.log} \RU{после запуска я вижу следующее}\EN{This is what we see}:

\lstinputlisting{patterns/09_loops/simple/tracer.log}

\RU{Видно, как значение}\EN{We see how the value of} \ESI \RU{последовательно изменяется от 2 до 9.}
\EN{register changes from 2 to 9.}

\RU{И даже более того, в \tracer можно собирать значения регистров по всем адресам внутри функции.}
\EN{Even more than that, the \tracer can collect register values for all addresses within the function.}
\RU{Там это называется}\EN{This is called} \IT{trace}\EN{ there}.
\RU{Каждая инструкция трассируется, значения самых интересных регистров запоминаются}\EN{Every instruction
gets traced, all interesting register values are recorded}.
\RU{Затем генерируется .idc-скрипт для \IDA, который добавляет комментарии.}
\EN{Then, an \IDA .idc-script is generated, that adds comments.}
\RU{Итак, в}\EN{So, in the} \IDA \RU{я узнал что адрес}\EN{we've learned that the} \main \RU{это}\EN{function address
is} \TT{0x00401020} \RU{и запускаю}\EN{and we run}:

\begin{lstlisting}
tracer.exe -l:loops_2.exe bpf=loops_2.exe!0x00401020,trace:cc
\end{lstlisting}

\TT{BPF} \RU{означает установить точку останова на функции}\EN{stands for set breakpoint on function}.

\RU{Получаю в итоге скрипты}\EN{As a result, we get the} \TT{loops\_2.exe.idc} \AndENRU 
\TT{loops\_2.exe\_clear.idc}\EN{ scripts}.

\clearpage
\RU{Загружаю}\EN{We load} \TT{loops\_2.exe.idc} \RU{в}\EN{into} \IDA \RU{и увижу следующее}\EN{and see}:

\begin{figure}[H]
\centering
\includegraphics[scale=\FigScale]{patterns/09_loops/simple/IDA_tracer_cc.png}
\caption{\IDA \RU{с загруженным .idc-скриптом}\EN{with .idc-script loaded}}
\label{fig:loops_IDA_tracer}
\end{figure}

\RU{Видно, что}\EN{We see that} \ESI \RU{меняется от 2 до 9 в начале тела цикла, но после 
\glslink{increment}{инкремента} он в пределах [3..0xA]}\EN{can be from 2 to 9 at the start of the loop body,
but from 3 to 0xA (10) after the increment}.
\RU{Видно также, что функция}\EN{We can also see that} \main \RU{заканчивается с 0 в}\EN{is finishing with 0 in} \EAX.

\tracer \RU{также генерирует}\EN{also generates} \TT{loops\_2.exe.txt}, 
\RU{содержащий адреса инструкций, сколько раз была исполнена
каждая и значения регистров}\EN{that contains information about how many times each instruction was executed and
register values}:

\lstinputlisting[caption=loops\_2.exe.txt]{patterns/09_loops/simple/loops_2.exe.txt}
\index{\GrepUsage}
\RU{Так можно использовать grep}\EN{We can use grep here}.

\fi

\ifdefined\IncludeARM
\subsection{ARM}

\subsubsection{\NonOptimizingKeilVI (\ARMMode)}

\lstinputlisting[label=Keil_number_sign]{patterns/09_loops/simple/ARM/Keil_ARM_O0.asm}

\RU{Счетчик итераций $i$ будет храниться в регистре \Reg{4}.}
\EN{Iteration counter $i$ is to be stored in the \Reg{4} register.}

\EN{The}\RU{Инструкция} \TT{\q{MOV R4, \#2}} \RU{просто инициализирует}\EN{instruction just initializes} $i$.

\EN{The}\RU{Инструкции} \TT{\q{MOV R0, R4}} \AndENRU \TT{\q{BL printing\_function}} \RU{составляют тело цикла.}\EN{instructions
compose the body of the loop}, 
\RU{Первая инструкция готовит аргумент для функции, \ttf а вторая вызывает её.}
\EN{the first instruction preparing the argument for \ttf function and the second calling the function.}

\index{ARM!\Instructions!ADD}
\EN{The}\RU{Инструкция} \TT{\q{ADD R4, R4, \#1}} \RU{прибавляет единицу к $i$ при каждой итерации.}
\EN{instruction just adds 1 to the $i$ variable at each iteration.}

\index{ARM!\Instructions!CMP}
\index{ARM!\Instructions!BLT}
\TT{\q{CMP R4, \#0xA}} \RU{сравнивает}\EN{compares} $i$ \RU{с}\EN{with} \TT{0xA} (10). 
\RU{Следующая за ней инструкция \TT{BLT} (\IT{Branch Less Than}) совершит переход, 
если $i$ меньше чем 10.}
\EN{The next instruction \TT{BLT} (\IT{Branch Less Than}) 
jumps if $i$ is less than 10.}

\RU{В противном случае в \Reg{0} запишется 0 (потому что наша функция возвращает 0)
и произойдет выход из функции.}
\EN{Otherwise, 0 is to be written into \Reg{0} (since our function returns 0)
and function execution finishes.}

\subsubsection{\OptimizingKeilVI (\ThumbMode)}

\lstinputlisting{patterns/09_loops/simple/ARM/Keil_thumb_O3.asm}

\RU{Практически всё то же самое.}\EN{Practically the same.}

\subsubsection{\OptimizingXcodeIV (\ThumbTwoMode)}
\label{ARM_unrolled_loops}

\lstinputlisting{patterns/09_loops/simple/ARM/xcode_thumb_O3.asm}

\RU{На самом деле, в моей функции \ttf было такое:}\EN{In fact, this was in my \ttf function:}

\begin{lstlisting}
void printing_function(int i)
{
    printf ("%d\n", i);
};
\end{lstlisting}

\index{Unrolled loop}
\index{Inline code}
\RU{Так что}\EN{So,} LLVM \RU{не только \IT{развернул} цикл}\EN{not just \IT{unrolled} the loop}, 
\RU{но также и представил мою очень простую функцию \ttf как \IT{inline-функцию}}\EN{but also \IT{inlined} my 
very simple function \ttf},
\RU{и вставил её тело вместо цикла 8 раз}\EN{and inserted its body 8 times instead of calling it}. 
\RU{Это возможно, когда функция очень простая (как та что у меня) и когда
она вызывается не очень много раз, как здесь.}
\EN{This is possible when the function is so simple (like mine) and when it is not called too much (like here).}

\subsubsection{ARM64: \Optimizing GCC 4.9.1}

\lstinputlisting[caption=\Optimizing GCC 4.9.1]{patterns/09_loops/simple/ARM/ARM64_GCC491_O3.s.\LANG}

\subsubsection{ARM64: \NonOptimizing GCC 4.9.1}

\lstinputlisting[caption=\NonOptimizing GCC 4.9.1 -fno-inline]{patterns/09_loops/simple/ARM/ARM64_GCC491_O3.s.\LANG}

\fi
\ifdefined\IncludeMIPS
\subsection{MIPS}

\lstinputlisting[caption=\NonOptimizing GCC 4.4.5 (IDA)]{patterns/09_loops/simple/MIPS_O0_IDA.lst.\LANG}

\index{MIPS!\Pseudoinstructions!B}

\ifdefined\RUSSIAN
Новая для нас инструкция это \INS{B}. Вернее, это псевдоинструкция (\INS{BEQ}).
\fi

\ifdefined\ENGLISH
The instruction that's new to us is \TT{B}. It is actually the pseudoinstruction (\INS{BEQ}).
\fi


\fi

\subsection{\RU{Ещё кое-что}\EN{One more thing}}

\RU{По генерируемому коду мы видим следующее}\EN{In the generated code we can see}: 
\RU{после инициализации}\EN{after initializing} $i$%
\RU{, тело цикла не исполняется. Исполняется сразу 
проверка условия $i$, а лишь затем исполняется тело цикла.}%
\EN{, the body of the loop is not to be executed,
as the condition for $i$ is checked first, and only after that loop body can be executed.}
\RU{Это правильно.}\EN{And that is correct.} 
\RU{Потому что если условие в самом начале не выполняется, тело цикла исполнять нельзя.}
\EN{Because, if the loop condition is
not met at the beginning, the body of the loop must not be executed.}
\RU{Так может быть, например, в таком случае:}\EN{This is possible in the following case:}

\lstinputlisting{patterns/09_loops/simple/loops_3.c.\LANG}

\RU{Если}\EN{If} \IT{total\_entries\_to\_process} \RU{равно}\EN{is} 0,
\RU{тело цикла не должно исполниться ни разу}\EN{the body of the loo must not be executed at all}.
\RU{Поэтому проверка условия происходит перед тем как исполнить само тело.}
\EN{This is why the condition checked before
the execution.}

\RU{Впрочем, оптимизирующий компилятор может переставить проверку условия и тело цикла местами, если он уверен,
что описанная здесь ситуация невозможна, как в случае с нашим простейшим примером и компиляторами 
Keil, Xcode (LLVM), MSVC и GCC в режиме оптимизации.}
\EN{However, an optimizing compiler may swap the condition check and loop body,
if it sure that the situation described here is
not possible (like in the case of our very simple example and Keil, Xcode (LLVM), MSVC in optimization mode).}

\section{\RU{Функция копирования блоков памяти}\EN{Memory blocks copying routine}}
\label{loop_memcpy}

\RU{Настоящие функции копирования памяти могут копировать по 4 или 8 байт на каждой итерации, использовать \ac{SIMD},
векторизацию, \etc{}.}
\EN{Real-world memory copy routines may copy 4 or 8 bytes at each iteration, use \ac{SIMD}, 
vectorization, \etc{}.}
\RU{Но ради простоты, этот пример настолько прост, насколько это возможно.}
\EN{But for the sake of simplicity, this example is the simplest possible.}

\lstinputlisting{memcpy.c}

\subsection{\RU{Простейшая реализация}\EN{Straight-forward implementation}}

\lstinputlisting[caption=GCC 4.9 x64 \RU{оптимизация по размеру}\EN{optimized for size} (-Os)]{patterns/09_loops/memcpy/memcpy_GCC49_x64_Os.s.\LANG}

\ifdefined\IncludeARM

\lstinputlisting[caption=GCC 4.9 ARM64 \RU{оптимизация по размеру}\EN{optimized for size} (-Os)]{patterns/09_loops/memcpy/memcpy_GCC49_ARM64_Os.s.\LANG}

\lstinputlisting[caption=\OptimizingKeilVI (\ThumbMode)]{patterns/09_loops/memcpy/memcpy_Keil_Thumb_O3.s.\LANG}

\subsection{ARM \RU{в режиме ARM}\EN{in ARM mode}}

\RU{Keil в режиме ARM пользуется условными суффиксами:}
\EN{Keil in ARM mode takes full advantage of conditional suffixes:}

\lstinputlisting[caption=\OptimizingKeilVI (\ARMMode)]{patterns/09_loops/memcpy/memcpy_Keil_ARM_O3.s.\LANG}

\RU{Вот почему здесь только одна инструкция перехода вместо двух.}
\EN{That's why there is only one branch instruction instead of 2.}

\fi

\ifdefined\IncludeMIPS
\subsection{MIPS}

\lstinputlisting[caption=GCC 4.4.5 \RU{оптимизация по размеру}\EN{optimized for size} (-Os) (IDA)]{patterns/09_loops/memcpy/memcpy_MIPS_Os_IDA.lst.\LANG}

\index{MIPS!\Instructions!LBU}
\index{MIPS!\Instructions!SB}
\RU{Здесь две новых для нас инструкций:}
\EN{Here we have two new instructions:} LBU (\q{Load Byte Unsigned}) \AndENRU SB (\q{Store Byte}).
\RU{Так же как и в ARM, все регистры в MIPS имеют длину в 32 бита. Здесь нет частей регистров равных байту,
как в x86.}
\EN{Just like in ARM, all MIPS registers are 32-bit wide, there are no byte-wide parts like in x86.}
\RU{Так что когда нужно работать с байтами, приходится выделять целый 32-битный регистр для этого.}
\EN{So when dealing with single bytes, we have to allocate whole 32-bit registers for them.}
\RU{LBU загружает байт и сбрасывает все остальные биты (\q{Unsigned}).}
\EN{LBU loads a byte and clears all other bits (\q{Unsigned}).}
\index{MIPS!\Instructions!LB}
\RU{И напротив, инструкция LB (\q{Load Byte}) расширяет байт до 32-битного значения учитывая знак.}
\EN{On the other hand, LB (\q{Load Byte}) instruction sign-extends the loaded byte to a 32-bit value.}
\RU{SB просто записывает байт из младших 8 бит регистра в память.}
\EN{SB just writes a byte from lowest 8 bits of register to memory.}

\fi

\ifx\LITE\undefined
\subsection{\RU{Векторизация}\EN{Vectorization}}

\Optimizing GCC \RU{может из этого примера сделать намного больше}\EN{can do much more on this example}: 
\myref{vec_memcpy}.
\fi

\EN{\subsection{Condition check}

It's important to keep in mind that in \IT{for()} construct, condition is checked not at the end, but at the beginning, before execution of loop body.
But often, it's more convenient for compiler to check it at the end, after body.
Sometimes, additional check can be appended at the beginning.

For example:

\lstinputlisting[style=customc]{patterns/09_loops/cond_check/1.c}

Optimizing GCC 5.4.0 x64:

\lstinputlisting[style=customasmx86]{patterns/09_loops/cond_check/1.s}

We see two checks.

\myindex{Hex-Rays}
Hex-Rays (at least version 2.2.0) decompiles this as:

\lstinputlisting[style=customc]{patterns/09_loops/cond_check/hexrays.c}

In this case, \IT{do/while()} can be replaced by \IT{for()} without any doubt, and the first check can be removed.

}
\RU{\subsection{Проверка условия}

Важно помнить, что в конструкции \IT{for()}, проверка условия происходит не в конце, а в начале, перед исполнением тела цикла.
Но нередко компилятору удобнее проверять условие в конце, после тела.
Иногда может добавляться еще одна проверка в начале.

Например:

\lstinputlisting[style=customc]{patterns/09_loops/cond_check/1.c}

Оптимизирующий GCC 5.4.0 x64:

\lstinputlisting[style=customasmx86]{patterns/09_loops/cond_check/1.s}

Видим две проверки.

\myindex{Hex-Rays}
Hex-Rays (по крайней мере версии 2.2.0) декомпилирует это так:

\lstinputlisting[style=customc]{patterns/09_loops/cond_check/hexrays.c}

В данном случае, \IT{do/while()} можно смело заменять на \IT{for()}, а первую проверку убрать.

}

\EN{% N.B.: \Conclusion{} is a macro name, do not translate
\subsection{\Conclusion{}}

Rough skeleton of loop from 2 to 9 inclusive:

\lstinputlisting[caption=x86,style=customasmx86]{patterns/09_loops/skeleton_x86_2_9_optimized_EN.lst}

The increment operation may be represented as 3 instructions in non-optimized code:

\lstinputlisting[caption=x86,style=customasmx86]{patterns/09_loops/skeleton_x86_2_9_EN.lst}

If the body of the loop is short, a whole register can be dedicated to the counter variable:

\lstinputlisting[caption=x86,style=customasmx86]{patterns/09_loops/skeleton_x86_2_9_reg_EN.lst}

Some parts of the loop may be generated by compiler in different order:

\lstinputlisting[caption=x86,style=customasmx86]{patterns/09_loops/skeleton_x86_2_9_order_EN.lst}

Usually the condition is checked \IT{before} loop body, but the compiler may rearrange it in a way that
the condition is checked \IT{after} loop body.

This is done when the compiler is sure that the condition is always \IT{true} on the first iteration, 
so the body of the loop is to be executed at least once:

\lstinputlisting[caption=x86,style=customasmx86]{patterns/09_loops/skeleton_x86_2_9_reorder_EN.lst}

\myindex{x86!\Instructions!LOOP}

Using the \TT{LOOP} instruction. This is rare, compilers are not using it.
When you see it, it's a sign that this piece of code is hand-written:

\lstinputlisting[caption=x86,style=customasmx86]{patterns/09_loops/skeleton_x86_loop_EN.lst}

ARM. 

The \Reg{4} register is dedicated to counter variable in this example:

\lstinputlisting[caption=ARM,style=customasmARM]{patterns/09_loops/skeleton_ARM_EN.lst}

% TODO MIPS

}
\RU{\subsection{\Conclusion{}}

Примерный скелет цикла от 2 до 9 включительно:

\lstinputlisting[caption=x86,style=customasmx86]{patterns/09_loops/skeleton_x86_2_9_optimized_RU.lst}

Операция инкремента может быть представлена как 3 инструкции в неоптимизированном коде:

\lstinputlisting[caption=x86,style=customasmx86]{patterns/09_loops/skeleton_x86_2_9_RU.lst}

Если тело цикла короткое, под переменную счетчика можно выделить целый регистр:

\lstinputlisting[caption=x86,style=customasmx86]{patterns/09_loops/skeleton_x86_2_9_reg_RU.lst}

Некоторые части цикла могут быть сгенерированы компилятором в другом порядке:

\lstinputlisting[caption=x86,style=customasmx86]{patterns/09_loops/skeleton_x86_2_9_order_RU.lst}

Обычно условие проверяется \IT{перед} телом цикла, но компилятор может перестроить цикл так, 
что условие проверяется \IT{после} тела цикла.

Это происходит тогда, когда компилятор уверен, что условие всегда будет \IT{истинно} на первой итерации,
так что тело цикла исполнится как минимум один раз:

\lstinputlisting[caption=x86,style=customasmx86]{patterns/09_loops/skeleton_x86_2_9_reorder_RU.lst}

\myindex{x86!\Instructions!LOOP}
Используя инструкцию \TT{LOOP}. Это редкость, компиляторы не используют её.
Так что если вы её видите, это верный знак, что этот фрагмент кода написан вручную:

\lstinputlisting[caption=x86,style=customasmx86]{patterns/09_loops/skeleton_x86_loop_RU.lst}

ARM. 
В этом примере регистр \Reg{4} выделен для переменной счетчика:


\lstinputlisting[caption=ARM,style=customasmARM]{patterns/09_loops/skeleton_ARM_RU.lst}

% TODO MIPS
}
\DE{% N.B.: \Conclusion{} is a macro name, do not translate
\subsection{\Conclusion{}}

Gerüst einer Schleife von einschließlich 2 bis einschließlich 9:

\lstinputlisting[caption=x86,style=customasmx86]{patterns/09_loops/skeleton_x86_2_9_optimized_DE.lst}

Das Inkrementieren kann in nicht optimiertem Code durch 3 Instruktionen
dargestellt werden:

\lstinputlisting[caption=x86,style=customasmx86]{patterns/09_loops/skeleton_x86_2_9_DE.lst}

Falls der Körper einer Schleife besonders kurz ist, kann ein Register als Zähler
verwendet werden:

\lstinputlisting[caption=x86,style=customasmx86]{patterns/09_loops/skeleton_x86_2_9_reg_DE.lst}

Einige Teile der Schleife können vom Compiler in unterschiedlichen Reihenfolgen
generiert werden:

\lstinputlisting[caption=x86,style=customasmx86]{patterns/09_loops/skeleton_x86_2_9_order_DE.lst}

Normalerweise wird die Bedingung \IT{vor} dem Körper geprüft, aber der Compiler
kann den Code auch so anordnen, dass die Bedingung \IT{nach} dem Körper geprüft
wird.

Dies geschieht dann, wenn der Compiler sicher sein kann, dass die Bedingung im
ersten Durchlauf stets \IT{wahr} ist, sodass der Körper der Schleife mindestens
einmal tatsächlich ausgeführt wird:

\lstinputlisting[caption=x86,style=customasmx86]{patterns/09_loops/skeleton_x86_2_9_reorder_DE.lst}

\myindex{x86!\Instructions!LOOP}

Verwendung des \IT{LOOP} Befehls. Sehr selten, Compiler verwenden ihn nicht.
Wenn er auftaucht, ist dies ein Zeichen dafür, dass das entsprechende
Codesegment von Hand geschrieben worden ist:

 
\lstinputlisting[caption=x86,style=customasmx86]{patterns/09_loops/skeleton_x86_loop_DE.lst}

ARM. 

Das \Reg{4} Register fungiert in diesem Beispiel als Zähler:

 
\lstinputlisting[caption=ARM,style=customasmARM]{patterns/09_loops/skeleton_ARM_DE.lst}

% TODO MIPS
}
\FR{% N.B.: \Conclusion{} is a macro name, do not translate
\subsection{\Conclusion{}}

Squelette grossier d'une boucle de 2 à 9 inclus:

\lstinputlisting[caption=x86,style=customasmx86]{patterns/09_loops/skeleton_x86_2_9_optimized_FR.lst}

L'opération d'incrémentation peut être représentée par 3 instructions dans du code
non optimisé:

\lstinputlisting[caption=x86,style=customasmx86]{patterns/09_loops/skeleton_x86_2_9_FR.lst}

Si le corps de la boucle est court, un registre entier peut être dédié à la variable
compteur:

\lstinputlisting[caption=x86,style=customasmx86]{patterns/09_loops/skeleton_x86_2_9_reg_FR.lst}

Certaines parties de la boucle peuvent être générées dans un ordre différent par
le compilateur:

\lstinputlisting[caption=x86,style=customasmx86]{patterns/09_loops/skeleton_x86_2_9_order_FR.lst}

En général, la condition est testée \IT{avant} le corps de la boucle, mais le compilateur
peut la réarranger afin que la condition soit testée \IT{aprés} le corps de la boucle.

Cela est fait lorsque le compilateur est certain que la condition est toujours \IT{vraie}
à la première itération, donc que le corps de la boucle doit être exécuté au moins
une fois:

\lstinputlisting[caption=x86,style=customasmx86]{patterns/09_loops/skeleton_x86_2_9_reorder_FR.lst}

\myindex{x86!\Instructions!LOOP}

En utilisant l'instruction \TT{LOOP}. Ceci est rare, les compilateurs ne l'utilisent
pas.
Lorsque vous la voyez, c'est le signe que le morceau de code a été écrit é la main:

\lstinputlisting[caption=x86,style=customasmx86]{patterns/09_loops/skeleton_x86_loop_FR.lst}

ARM. 

Le registre \Reg{4} est dédié à la variable compteur dans cet exemple:

\lstinputlisting[caption=ARM,style=customasmARM]{patterns/09_loops/skeleton_ARM_FR.lst}

% TODO MIPS

}


\section{\Exercises}

\subsection{\Exercise \#1}

\index{x86!\Instructions!LOOP}
\RU{Почему инструкция}\EN{Why} \LOOP \RU{больше не используется современными 
компиляторами}\EN{instruction is not used by modern compilers anymore}?

\subsection{\Exercise \#2}

\RU{Возьмите пример рассмотренный в этой секции}\EN{Take a loop example from this section} 
(\ref{loops_src}), 
\RU{скомпилируйте его в вашей любимой}\EN{compile it in your favorite} \ac{OS}
\RU{и компиляторе, и модифицируйте исполняемый файл так, чтобы цикл был в пределах}\EN{and compiler 
and modify (patch) executable file, so the loop range will be} [6..20].


\chapter{strlen()}
\index{\CStandardLibrary!strlen()}
\index{\CLanguageElements!while}

\IFRU{Еще немного о циклах. Часто, функция \TT{strlen()}\footnote{подсчет длины строки в Си} 
реализуется при помощи \TT{while()}.}
{Now let's talk about loops one more time. Often, \TT{strlen()} 
function\footnote{counting characters in string in C language} is implemented using \TT{while()} 
statement.}
\IFRU{Например, как это сделано в стандартных библиотеках MSVC:}
{Here is how it is done in MSVC standard libraries:}

\lstinputlisting{patterns/10_strlen/ex1.c}

\subsection{x86}

\IFRU{Итак, компилируем:}{Let's compile:}

\lstinputlisting{patterns/10_strlen/10_1_msvc_\IFRU{ru}{en}.asm}

\index{x86!\Instructions!MOVSX}
\index{x86!\Instructions!TEST}
\IFRU{Здесь две новых инструкции: \MOVSX и \TEST.}
{Two new instructions here: \MOVSX and \TEST.}

\label{MOVSX}
\IFRU{О первой: \MOVSX предназначен для того чтобы взять байт из какого-либо места в памяти и положить его, 
в нашем случае, в регистр \EDX. 
Но регистр \EDX ~--- 32-битный. \MOVSX означает \IT{MOV with Sign-Extent}. 
Оставшиеся биты с 8-го по 31-й \MOVSX сделает единицей, если исходный байт в памяти имеет знак \IT{минус}, 
или заполнит нулями, если знак \IT{плюс}.}
{About first: \MOVSX is intended to take byte from a point in memory and store value in a 32-bit register. 
\MOVSX meaning \IT{MOV with Sign-Extent}. 
Rest bits starting at 8th till 31th \MOVSX will set to $1$ if source byte in memory has \IT{minus} 
sign or to 0 if \IT{plus}.}

\IFRU{И вот зачем все это.}{And here is why all this.}

\IFRU{По стандарту \CCpp, тип \Tchar ~--- знаковый. Если у нас есть две переменные, одна \Tchar, а другая \Tint 
(\Tint тоже знаковый), и если в первой переменной лежит $-2$ (что кодируется как \TT{0xFE}) и мы просто 
переложим это в \Tint, 
то там будет \TT{0x000000FE}, а это, с точки зрения \Tint, даже знакового, будет $254$, но никак не $-2$. 
$-2$ в переменной \Tint кодируется как \TT{0xFFFFFFFE}. И для того чтобы значение \TT{0xFE} из переменной типа 
\Tchar переложить 
в знаковый \Tint с сохранением всего, нужно узнать его знак, и затем заполнить остальные биты. 
Это делает \MOVSX.}
{\CCpp standard defines \Tchar type as signed. If we have two values, one is \Tchar 
and another is \Tint, (\Tint is signed too), and if first value contain $-2$ (it is coded as \TT{0xFE}) 
and we just copying this byte into \Tint container, there will be \TT{0x000000FE}, and this, 
from the point of signed \Tint view is $254$, but not $-2$. In signed int, $-2$ is coded as \TT{0xFFFFFFFE}. 
So if we need to transfer \TT{0xFE} value from variable of \Tchar type to \Tint, 
we need to identify its sign and extend it. That is what \MOVSX does.}

\IFRU{См. также об этом раздел}
{See also in section} ``\IT{\SignedNumbersSectionName}''~(\ref{sec:signednumbers}).

\IFRU{Хотя, конкретно здесь, компилятору врядли была особая надобность хранить значение \Tchar в регистре \EDX 
а не его восьмибитной части, скажем, \DL. Но получилось, как получилось: должно быть, 
\gls{register allocator} компилятора сработал именно так.}
{I'm not sure if the compiler needs to store \Tchar variable in the \EDX, it could take 8-bit register part 
(let's say \DL). Apparently, compiler's \gls{register allocator} works like that.}

\index{ARM!\Instructions!TEST}
\IFRU{Позже выполняется \TT{TEST EDX, EDX}. 
Об инструкции \TEST читайте в разделе о битовых полях~(\ref{sec:bitfields}).
Но конкретно здесь, эта инструкция просто проверяет состояние регистра \EDX на $0$.}
{Then we see \TT{TEST EDX, EDX}. 
About \TEST instruction, read more in section about bit fields~(\ref{sec:bitfields}).
But here, this instruction just checking value in the \EDX, if it is equals to $0$.}

\IFRU{Попробуем}{Let's try} GCC 4.4.1:

\lstinputlisting{patterns/10_strlen/10_3_gcc.asm}

\label{movzx}
\index{x86!\Instructions!MOVZX}
\IFRU{Результат очень похож на MSVC, вот только здесь используется \MOVZX а не \MOVSX. 
\MOVZX означает \IT{MOV with Zero-Extent}. Эта инструкция перекладывает какое-либо значение 
в регистр и остальные биты выставляет в $0$.
Фактически, преимущество этой инструкции только в том, что она позволяет 
заменить две инструкции сразу: \TT{xor eax, eax / mov al, [...]}.}
{The result almost the same as MSVC did, but here we see \MOVZX instead of \MOVSX. 
\MOVZX means \IT{MOV with Zero-Extent}. 
This instruction copies 8-bit or 16-bit value into 32-bit register and sets the rest bits to $0$. 
In fact, this instruction is convenient only since it enable us to replace two instructions at once: 
\TT{xor eax, eax / mov al, [...]}.}

\IFRU{С другой стороны, нам очевидно, что здесь можно было бы написать вот так: 
\TT{mov al, byte ptr [eax] / test al, al} ~--- это тоже самое, хотя старшие биты \EAX будут ``замусорены''. 
Но, будем считать, что это погрешность компилятора ~--- 
он не смог сделать код более экономным или более понятным. 
Строго говоря, компилятор вообще не нацелен на то чтобы генерировать понятный (для человека) код.}
{On the other hand, it is obvious to us the compiler could produce the code: 
\TT{mov al, byte ptr [eax] / test al, al}~---it is almost the same, however, 
the highest \EAX register bits will contain random noise. 
But let's think it is compiler's drawback~---it cannot produce more understandable code. 
Strictly speaking, compiler is not obliged to emit understandable (to humans) code at all.}

\index{x86!\Instructions!SETNZ}
\IFRU{Следующая новая инструкция для нас ~--- \SETNZ. В данном случае, если в \AL был не ноль, 
то \TT{test al, al} выставит флаг \ZF в $0$, а \SETNZ, если \TT{ZF==0} 
(\IT{NZ} значит \IT{not zero}) выставит $1$ в \AL. 
Смысл этой процедуры в том, что, если говорить человеческим языком, 
\IT{если AL не ноль, то выполнить переход на} \TT{loc\_80483F0}.
Компилятор выдал немного избыточный код, но не будем забывать, что оптимизация выключена.}
{Next new instruction for us is \SETNZ. Here, if \AL contain not zero, \TT{test al, al} 
will set $0$ to the \ZF flag, but \SETNZ, if \TT{ZF==0} (\IT{NZ} means \IT{not zero}) will set $1$ to the \AL.
Speaking in natural language, \IT{if \AL is not zero, let's jump to loc\_80483F0}. 
Compiler emitted slightly redundant code, but let's not forget the optimization is turned off.}

\IFRU{Теперь скомпилируем все то же самое в MSVC 2010, но с включенной оптимизацией (\Ox)}
{Now let's compile all this in MSVC 2010, with optimization turned on (\Ox)}:

\lstinputlisting{patterns/10_strlen/10_2_\IFRU{ru}{en}.asm}

\IFRU{Здесь все попроще стало. Но следует отметить, что компилятор обычно может так хорошо использовать регистры 
только на не очень больших функциях с не очень большим количеством локальных переменных.}
{Now it is all simpler.
But it is needless to say the compiler could use registers such efficiently 
only in small functions with small number of local variables.}

\index{x86!\Instructions!INC}
\index{x86!\Instructions!DEC}
\INC/\DEC\EMDASH\IFRU{это инструкции \glslink{increment}{инкремента}-\glslink{decrement}{декремента}, попросту говоря: 
увеличить на единицу или уменьшить.}
{are \gls{increment}/\gls{decrement} instruction, in other words: add 1 to variable or subtract.}

\IFRU{Попробуем GCC 4.4.1 с включенной оптимизацией (ключ \Othree:}
{Let's check GCC 4.4.1 with optimization turned on (\Othree key):}

\lstinputlisting{patterns/10_strlen/10_3_gcc_O3.asm}

\IFRU{Здесь GCC не очень отстает от MSVC за исключением наличия \MOVZX.} 
{Here GCC is almost the same as MSVC, except of \MOVZX presence.}

\IFRU {Впрочем, только кроме того, что почему-то используется \MOVZX, который явно можно заменить на}
{However, \MOVZX could be replaced here to} \TT{mov dl, byte ptr [eax]}.

\IFRU{Но, возможно, компилятору GCC просто проще помнить, что у него под переменную типа \Tchar отведен целый 
32-битный регистр и быть уверенным в том, что старшие биты регистра не будут замусорены.}
{Probably, it is simpler for GCC compiler's code generator to \IT{remember} the whole register 
is allocated for \Tchar variable and it can be sure the highest bits will not contain any noise 
at any point.}

\label{strlen_NOT_ADD}
\index{x86!\Instructions!NOT}
\index{x86!\Instructions!XOR}
\IFRU{Далее мы видим новую для нас инструкцию \NOT. Эта инструкция инвертирует все биты в операнде. 
Можно сказать, что здесь это синонимично инструкции \TT{XOR ECX, 0ffffffffh}. 
\NOT и следующая за ней инструкция \ADD вычисляют разницу указателей и отнимают от результата единицу. 
Только происходит это слегка по-другому. Сначала \ECX, где хранится указатель на \IT{str}, 
инвертируется и от него отнимается единица.}
{After, we also see new instruction \NOT. This instruction inverts all bits in operand. 
It can be said, it is synonym to the \TT{XOR ECX, 0ffffffffh} instruction. 
\NOT and following \ADD calculating pointer difference and subtracting 1. 
At the beginning \ECX, where pointer to \IT{str} is stored, inverted and 1 is subtracted from it.}

\IFRU{См. также раздел:}{See also:} ``\SignedNumbersSectionName''~(\ref{sec:signednumbers}).

\IFRU{Иными словами, в конце функции, после цикла, происходит примерно следующее:} 
{In other words, at the end of function, just after loop body, these operations are executed:}

\begin{lstlisting}
ecx=str;
eax=eos;
ecx=(-ecx)-1; 
eax=eax+ecx
return eax
\end{lstlisting}

\dots \IFRU{что эквивалентно}{and this is effectively equivalent to}:

\begin{lstlisting}
ecx=str;
eax=eos;
eax=eax-ecx;
eax=eax-1;
return eax
\end{lstlisting}

\IFRU{Но почему GCC решил, что так будет лучше? Снова не берусь сказать. Но я не сомневаюсь, 
что эти оба варианта работают примерно равноценно в плане эффективности и скорости.}
{Why GCC decided it would be better? I cannot be sure. 
But I'm sure the both variants are effectively equivalent in efficiency sense.}

\section{ARM}

\subsection{\NonOptimizingXcode + \ARMMode}

\lstinputlisting[caption=\NonOptimizingXcode + \ARMMode]{patterns/10_strlen/xcode_ARM_O0_en.asm}

\IFRU{Неоптимизирующий LLVM генерирует слишком много кода, зато на этом примере можно посмотреть, 
как функции работают с локальными переменными в стеке.}
{Non-optimizing LLVM generates too much code, however, here we can see how function works with 
local variables in the stack.}
\IFRU{В нашей функции только локальных переменных две, это два указателя}
{There are only two local variables in our function},
\IT{eos} \AndENRU \IT{str}.

\IFRU{В этом листинге}{In this listing}, \IFRU{сгенерированном при помощи}{generated by} \IDA, 
\IFRU{я переименовал}{I renamed} \IT{var\_8} \AndENRU \IT{var\_4} \IFRU{в}{into} \IT{eos} 
\AndENRU \IT{str} \IFRU{вручную}{manually}.

\IFRU{Итак, первые несколько инструкций просто сохраняют входное значение в переменных}{So, 
first instructions are just saves input value in} \IT{str} \AndENRU \IT{eos}.

\IFRU{Начиная с метки}{Loop body is beginning at} \IT{loc\_2CB8}\IFRU{, начинается тело цикла}{ label}.

\IFRU{Первые три инструкции в теле цикла}{First three instruction in loop body} (\TT{LDR}, \ADD, \TT{STR}) 
\IFRU{загружают значение}{loads} \IT{eos} \IFRU{в}{value into} \Reg{0}, 
\IFRU{затем происходит инкремент значения и оно сохраняется назад в локальной переменной \IT{eos} расположенной 
в стеке.}{then value is \glslink{increment}{incremented} and it is saved back into \IT{eos} local variable located in the stack.}

\index{ARM!\Instructions!LDRSB}
\IFRU{Следующая инструкция}{The next} \TT{``LDRSB R0, [R0]''} (\IT{Load Register Signed Byte}) 
\IFRU{загружает байт из памяти по адресу \Reg{0}, расширяет его до 32-бит считая его знаковым (signed) 
и сохраняет в \Reg{0}}{instruction loading byte from memory at \Reg{0} address and sign-extends it to 32-bit}.
\index{x86!\Instructions!MOVSX}
\IFRU{Это немного похоже на инструкцию}{This is similar to} \MOVSX \IFRU{в}{instruction in} x86.
\IFRU{Компилятор считает этот байт знаковым (signed), потому что тип \Tchar по стандарту Си ~--- знаковый.}
{The compiler treating this byte as signed since \Tchar type in C standard is signed.}
\IFRU{Об это я уже немного писал}{I already wrote about it}~(\ref{MOVSX}) \IFRU{в этой же секции, 
но посвященной x86}{in this section, but related to x86}.

\index{x86!8086}
\index{8080}
\index{ARM}
\IFRU{Следует также заметить, что, в ARM нет возможности использовать 8-битную или 16-битную часть 
регистра, как это возможно в x86.}
{It is should be noted, it is impossible in ARM to use 8-bit part or 16-bit part 
of 32-bit register separately of the whole register,
as it is in x86.}
\IFRU{Вероятно, это связано с тем что за x86 тянется длинный шлейф совместимости со своими предками, 
такими как
16-битный 8086 и даже 8-битный 8080, а ARM разрабатывался с чистого листа как 32-битный RISC-процессор.}
{Apparently, it is because x86 has a huge history of compatibility with its ancestors like 16-bit 8086 
and even 8-bit 8080,
but ARM was developed from scratch as 32-bit RISC-processor.}
\IFRU{Следовательно, чтобы работать с отдельными байтами на ARM, так или иначе, придется использовать 
32-битные регистры.}
{Consequently, in order to process separate bytes in ARM, one have to use 32-bit registers anyway.}

\IFRU{Итак}{So}, \TT{LDRSB} \IFRU{загружает символ из строки в \Reg{0}, по одному}
{loads symbol from string into \Reg{0}, one by one}.
\IFRU{Следующие инструкции}{Next} \CMP \AndENRU \ac{BEQ} \IFRU{проверяют, является ли этот символ $0$.}
{instructions checks, if loaded symbol is $0$.}
\IFRU{Если не $0$, то происходит переход на начало тела цикла.}{If not $0$, control passing to loop body
begin.}
\IFRU{А если $0$, выходим из цикла.}{And if $0$, loop is finishing.}

\IFRU{В конце функции вычисляется разница между}{At the end of function, a difference between} 
\IT{eos} \AndENRU \IT{str}\IFRU{, вычитается еще единица и вычисленное 
значение возвращается через \Reg{0}.}{ is calculated, 1 is also subtracting, and resulting value is returned
via \Reg{0}.}

N.B. \IFRU{В этой функции не сохранялись регистры}{Registers was not saved in this function}.
\index{ARM!\Registers!scratch registers}
\IFRU{Это потому что, по стандарту, регистры \Reg{0}-\Reg{3} называются также ``scratch registers'',
они предназначены для передачи аргументов, 
их значения не нужно восстанавливать при выходе из функции, потому что они больше не нужны в вызывающей функции.
Таким образом, их можно использовать как захочется}
{That's because by ARM calling convention, \Reg{0}-\Reg{3} registers are ``scratch registers'', 
they are intended for arguments passing,
its values may not be restored upon function exit since calling function will not use them anymore.
Consequently, they may be used for anything we want.}
\IFRU{А так как никакие больше регистры не используются, то и сохранять нечего.}
{Other registers are not used here, so that is why we have nothing to save on the stack.}
\IFRU{Поэтому, управление можно вернуть назад вызывающей функции 
простым переходом (\TT{BX}), по адресу в регистре \LR.}
{Thus, control may be returned back to calling function by simple jump (\TT{BX}),
to address in the \LR register.}

%\subsection{\NonOptimizingXcode + режим thumb}
%Практически, точно такой же код.

\subsection{\OptimizingXcode + \ThumbMode}

\lstinputlisting[caption=\OptimizingXcode + \ThumbMode]{patterns/10_strlen/xcode_thumb_O3.asm}

\IFRU{Оптимизирующий LLVM решил, что под переменные \IT{eos} и \IT{str} выделять место в стеке не обязательно}
{As optimizing LLVM concludes, space on the stack for \IT{eos} and \IT{str} may not be allocated},
\IFRU{и эти переменные можно хранить прямо в регистрах.}
{and these variables may always be stored right in registers.}
\IFRU{Перед началом тела цикла}{Before loop body beginning}, \IT{str} \IFRU{будет находиться в}{will always be in} 
\Reg{0}, \IFRU{а}{and} \IT{eos}\EMDASH\InENRU \Reg{1}.

\index{ARM!\Instructions!LDRB.W}
\index{ARM!\IFRU{Режимы адресации}{Adressing modes}}
\RU{Инструкция }\TT{``LDRB.W R2, [R1],\#1''} \IFRU{загружает в \Reg{2} байт из памяти по адресу \Reg{1}, 
расширяя его как знаковый (signed), до 32-битного
значения, но не только это.}
{instruction loads byte from memory at the address \Reg{1} into \Reg{2}, sign-extending it to 32-bit value, but not
only that.}
\TT{\#1} \IFRU{в конце инструкции называется}{at the instruction's end calling} ``Post-indexed addressing'', 
\IFRU{это значит, что после загрузки байта, к \Reg{1} добавится единица.}{this means, $1$ is to be added
to the \Reg{1} after byte load.}
\IFRU{Это очень удобно для работы с массивами.}
{That's convenient when accessing arrays.}

\index{PDP-11}
\index{\CLanguageElements!\PostIncrement}
\index{\CLanguageElements!\PostDecrement}
\index{\CLanguageElements!\PreIncrement}
\index{\CLanguageElements!\PreDecrement}
\IFRU{Такого режима адресации в x86 нет, но он есть в некоторых других процессорах, даже на PDP-11.}
{There is no such addressing mode in x86, but it is present in some other processors, even on PDP-11.}
\IFRU{Существует байка, что режимы пре-инкремента, пост-инкремента, 
пре-декремента и пост-декремента адреса в PDP-11}
{There is a legend the pre-increment, post-increment, pre-decrement and post-decrement modes in PDP-11},
\IFRU{были ``виновны'' в появлении таких конструкций языка Си (который разрабатывался на PDP-11) как}
{were ``guilty'' in appearance such C language (which developed on PDP-11) constructs as}
*ptr++, *++ptr, *ptr-{}-, *-{}-ptr. 
\IFRU{Кстати, это является труднозапоминаемой особенностью в Си.}
{By the way, this is one of hard to memorize C feature.}
\IFRU{Дела обстоят так:}{This is how it is:}

\begin{center}
\begin{tabular}{ | l | l | l | l | }
\hline
\headercolor{} \IFRU{термин в Си}{C term} & 
\headercolor{} \IFRU{термин в ARM}{ARM term} & 
\headercolor{} \IFRU{выражение Си}{C statement} & 
\headercolor{} \IFRU{как это работает}{how it works} \\
\hline
\PostIncrement & 
post-indexed addressing & 
\TT{*ptr++} & 
\IFRU{использовать значение \TT{*ptr}}{use \TT{*ptr} value}, \\
& & & \IFRU{затем инкремент указателя \TT{ptr}}{then \gls{increment} \TT{ptr} pointer} \\
\hline
\PostDecrement & 
post-indexed addressing & 
\TT{*ptr-{}-} & 
\IFRU{использовать значение \TT{*ptr}}{use \TT{*ptr} value}, \\
& & & \IFRU{затем \glslink{decrement}{декремент} указателя \TT{ptr}}{then \gls{decrement} \TT{ptr} pointer} \\
\hline
\PreIncrement & 
pre-indexed addressing & 
\TT{*++ptr} & 
\IFRU{инкремент указателя \TT{ptr}}{\gls{increment} \TT{ptr} pointer}, \\
& & & \IFRU{затем использовать значение \TT{*ptr}}{then use \TT{*ptr} value} \\
\hline
\PreDecrement & 
post-indexed addressing & 
\TT{*-{}-ptr} & 
\IFRU{\glslink{decrement}{декремент} указателя \TT{ptr}}{\gls{decrement} \TT{ptr} pointer}, \\
& & & \IFRU{затем использовать значение \TT{*ptr}}{then use \TT{*ptr} value} \\
\hline
\end{tabular}
\end{center}

\IFRU{Деннис Ритчи (один из создателей ЯП Си) указывал, что, это, вероятно, придумал Кен Томпсон 
(еще один создатель Си),
потому что подобная возможность процессора имелась еще в PDP-7}
{Dennis Ritchie (one of C language creators) mentioned that it is, probably, was invented by Ken Thompson
(another C creator) because this processor feature was present in PDP-7}
\cite{Ritchie:1986}\cite{Ritchie:1993:DCL:155360.155580}.
\IFRU{Таким образом, компиляторы с ЯП Си на тот процессор, где это есть, могут использовать это.}
{Thus, C language compilers may use it, if it is present in target processor.}

\IFRU{Далее в теле цикла можно увидеть \CMP и \ac{BNE}, они продолжают работу цикла до тех пор, 
пока не будет встречен $0$.}
{Then one may spot \CMP and \ac{BNE} in loop body, these instructions continue operation until
$0$ will be met in string.}

\index{ARM!\Instructions!MVNS}
\index{x86!\Instructions!NOT}
\RU{После конца цикла }\TT{MVNS}\footnote{MoVe Not} 
\IFRU{(инвертирование всех бит, аналог \NOT на x86)}
{(inverting all bits, \NOT in x86 analogue)}
\IFRU{и \ADD вычисляют}{instructions and \ADD computes} $eos - str - 1$.
\IFRU{На самом деле, эти две инструкции вычисляют}
{In fact, these two instructions computes}
$R0 = ~str + eos$, 
\IFRU{что эквивалентно тому, что было в исходном коде, а почему это так, я уже описывал чуть раньше, здесь}
{which is effectively equivalent to what was in source code, and why it is so, I already described here}
~(\ref{strlen_NOT_ADD}).

\IFRU{Вероятно, LLVM, как и GCC, посчитал что такой код будет короче, или быстрее.}
{Apparently, LLVM, just like GCC, concludes this code will be shorter, or faster.}

%\subsection{\OptimizingXcode + \ARMMode}
%Практически, точно такой же код.

\subsection{\OptimizingKeil{} + \ARMMode}

\lstinputlisting[caption=\OptimizingKeil + \ARMMode]{patterns/10_strlen/Keil_ARM_O3.asm}

\index{ARM!\Instructions!SUBEQ}
\IFRU{Практически то же самое что мы уже видели, за тем исключением что выражение}
{Almost the same what we saw before, with the exception the}
$str - eos - 1$ 
\IFRU{может быть вычислено не в самом конце функции, а прямо в теле цикла.}
{expression may be computed not at the function's end, but right in loop body.}
\RU{Суффикс }\TT{-EQ}\IFRU{, как мы помним, означает что инструкция будет выполнена только
если операнды в исполненной перед этим инструкции \CMP были равны.}
{suffix, as we may recall, means the instruction will be executed only if operands in executed before
\CMP were equal to each other.}
\IFRU{Таким образом}{Thus}, \IFRU{если в \Reg{0} будет $0$}{if $0$ will be in the \Reg{0} register},
\IFRU{обе инструкции}{both} \TT{SUBEQ} \IFRU{исполнятся и результат останется в \Reg{0}.}
{instructions are to be executed and result is leaving in the \Reg{0} register.}



\section{\DivisionByNineSectionName}
\label{sec:divisionbynine}

\IFRU{Простая функция:}{Very simple function:}

\begin{lstlisting}
int f(int a)
{
	return a/9;
};
\end{lstlisting}

\subsection{x86}

\dots \IFRU{компилируется вполне предсказуемо:}{is compiled in a very predictable way:}

\lstinputlisting[caption=MSVC]{patterns/11_division_by_9/11_1_msvc_\IFRU{ru}{en}.asm}

\index{ARM!\Instructions!IDIV}
\IFRU{\IDIV делит 64-битное число хранящееся в паре регистров \TT{EDX:EAX} на значение в \ECX. 
В результате, \EAX будет содержать частное\FNQUOTIENT, а \EDX ~--- остаток от деления. 
Результат возвращается из функции через \EAX, так что после операции деления, 
это значение не перекладывается больше никуда, 
оно уже там где надо.}
{\IDIV divides 64-bit number stored in the \TT{EDX:EAX} register pair by value in the \ECX register.
As a result, \EAX will contain quotient\FNQUOTIENT, and \EDX~---remainder.
Result is returning from the \TT{f()} function in the \EAX register, 
so, the value is not moved anymore after division 
operation, it is in right place already.}
\IFRU
{Из-за того что \IDIV требует пару регистров \TT{EDX:EAX}, то перед этим инструкция \TT{CDQ} 
расширяет \EAX до 64-битного значения учитывая знак, также как это делает \MOVSX.}
{Since \IDIV requires value in the \TT{EDX:EAX} register pair, \TT{CDQ} instruction (before \IDIV) extending 
value in the \EAX to 64-bit value taking value sign into account, just as \MOVSX does.}
\IFRU{Со включенной оптимизацией (\Ox) получается:}
{If we turn optimization on (\Ox), we got:}

\lstinputlisting[caption=\Optimizing MSVC]{patterns/11_division_by_9/11_1_msvc_Ox.asm}

\newcommand{\URLMSDN}{\href{http://blogs.msdn.com/b/devdev/archive/2005/12/12/502980.aspx}
{MSDN: Integer division by constants}}
\newcommand{\URLN}{http://www.nynaeve.net/?p=115}

\IFRU{Это ~--- деление через умножение. Умножение конечно быстрее работает. 
Поэтому можно используя этот трюк
\footnote{Читайте подробнее о делении через умножение в \cite[10-3]{Warren:2002:HD:515297}
и \URLMSDN, \url{\URLN}} 
создать код эквивалентный тому что мы хотим и работающий быстрее.}
{This is~---division by multiplication. Multiplication operation works much faster. 
And it is possible to use the trick
\footnote{Read more about division by multiplication in \cite[10-3]{Warren:2002:HD:515297}
and: \URLMSDN, \url{\URLN}} 
to produce a code which is effectively equivalent and faster.}
\IFRU
{GCC 4.4.1 даже без включенной оптимизации генерит примерно такой же код как и MSVC с оптимизацией:}
{GCC 4.4.1 even without optimization turned on, generates almost the same code as MSVC with optimization turned on:}

\lstinputlisting[caption=\NonOptimizing GCC 4.4.1]{patterns/11_division_by_9/11_2_gcc.asm}

\subsection{ARM}

\IFRU{В процессоре ARM, как и во многих других ``чистых'' (pure) RISC-процессорах нет инструкции деления.
Нет также возможности умножения на 32-битную константу одной инструкцией.}
{ARM processor, just like in any other ''pure'' RISC-processors, lacks division instruction
It lacks also a single instruction for multiplication by 32-bit constant.}
\IFRU{При помощи одного любопытного трюка (или \IT{хака})\footnote{hack}, можно обойтись только тремя действиями: 
сложением, вычитанием и битовыми сдвигами}
{By taking advantage of the one clever trick (or \IT{hack}), it is possible to do division using only three instructions: addition,
subtraction and bit shifts}~(\ref{sec:bitfields}).

\IFRU{Пример деления 32-битного числа на 10 из}{Here is an example of 32-bit number division by 10 from}
\cite[3.3 Division by a Constant]{ARM:1994}.
\IFRU{На выходе и частное и остаток}{Quotient and remainder on output}.

\begin{lstlisting}
; takes argument in a1
; returns quotient in a1, remainder in a2
; cycles could be saved if only divide or remainder is required
    SUB    a2, a1, #10             ; keep (x-10) for later
    SUB    a1, a1, a1, lsr #2
    ADD    a1, a1, a1, lsr #4
    ADD    a1, a1, a1, lsr #8
    ADD    a1, a1, a1, lsr #16
    MOV    a1, a1, lsr #3
    ADD    a3, a1, a1, asl #2
    SUBS   a2, a2, a3, asl #1      ; calc (x-10) - (x/10)*10
    ADDPL  a1, a1, #1              ; fix-up quotient
    ADDMI  a2, a2, #10             ; fix-up remainder
    MOV    pc, lr
\end{lstlisting}

\subsubsection{\OptimizingXcode + \ARMMode}

\begin{lstlisting}
__text:00002C58 39 1E 08 E3 E3 18 43 E3                 MOV             R1, 0x38E38E39
__text:00002C60 10 F1 50 E7                             SMMUL           R0, R0, R1
__text:00002C64 C0 10 A0 E1                             MOV             R1, R0,ASR#1
__text:00002C68 A0 0F 81 E0                             ADD             R0, R1, R0,LSR#31
__text:00002C6C 1E FF 2F E1                             BX              LR
\end{lstlisting}

\IFRU{Этот код почти тот же, что сгенерирован MSVC и GCC в режиме оптимизации.}
{This code is mostly the same to what was generated by optimizing MSVC and GCC.}
\IFRU{Должно быть, LLVM использует тот же алгоритм для поиска констант.}
{Apparently, LLVM use the same algorithm for constants generating.}

\index{ARM!\Instructions!MOV}
\index{ARM!\Instructions!MOVT}
\IFRU{Наблюдательный читатель может спросить, как \MOV записала в регистр сразу 32-битное число, 
ведь это невозможно в режиме ARM.}
{Observant reader may ask, how \MOV writes 32-bit value in register, while this is not possible in ARM mode.}
\IFRU{Действительно невозможно, но как мы видим, здесь на инструкцию 8 байт вместо стандартных 4-х,
на самом деле, здесь 2 инструкции.}
{It is not possible indeed, but, as we see,
there are 8 bytes per instruction instead of standard 4,
in fact, there are two instructions.}
\IFRU{Первая инструкция загружает в младшие 16 бит регистра значение \TT{0x8E39}, а вторая инструкция, 
на самом деле \TT{MOVT}, загружающая в старшие 16 бит регистра значение \TT{0x383E}.}
{First instruction loading \TT{0x8E39} value into low 16 bit of register and second instruction is in fact
\TT{MOVT}, it loading \TT{0x383E} into high 16-bit of register.}
\IDA \IFRU{распознала эту последовательность и для краткости, сократила всё это до одной ``псевдо-инструкции''.}
{is aware of such sequences, and for the sake of compactness, reduced it to one single ``pseudo-instruction''.}

\index{ARM!\Instructions!SMMUL}
\IFRU{Инструкция }{}\TT{SMMUL} (\IT{Signed Most Significant Word Multiply}) 
\IFRU{умножает числа считая их знаковыми (signed) и оставляет в \Rzero старшие 32 бита результата, 
не сохраняя младшие 32 бита.}
{instruction multiply numbers treating them as signed numbers,
and leaving high 32-bit part of result in the \Rzero register,
dropping low 32-bit part of result.}

\index{ARM!Optional operators!ASR}
\IFRU{Инструкция }{}\TT{``MOV R1, R0,ASR\#1''} \IFRU{это арифметический сдвиг право на один бит.}
{instruction is arithmetic shift right by one bit.}

\index{ARM!\Instructions!ADD}
\index{ARM!Data processing instructions}
\index{ARM!Optional operators!LSR}
\TT{``ADD R0, R1, R0,LSR\#31''} \IFRU{это}{is} $R0=R1 + R0>>31$

\label{shifts_in_ARM_mode}
\IFRU{Дело в том что в режиме ARM нет отдельных инструкций для битовых сдвигов.}
{As a matter of fact, there is no separate shifting instruction in ARM mode.}
\IFRU{Вместо этого, некоторые инструкции}{Instead, an instructions like} 
(\MOV, \ADD, \SUB, \TT{RSB})\footnote{\DataProcessingInstructionsFootNote}
\IFRU{могут быть дополнеты пометкой, сдвигать ли второй операнд и если да, то на сколько и как.}
{may be supplied by option, is the second operand must be shifted, if yes, by what value and how.}
\TT{ASR} \IFRU{означает}{meaning} \IT{Arithmetic Shift Right}, \TT{LSR}\EMDASH\IT{Logican Shift Right}.

\subsubsection{\OptimizingXcode + \ThumbTwoMode}

\begin{lstlisting}
MOV             R1, 0x38E38E39
SMMUL.W         R0, R0, R1
ASRS            R1, R0, #1
ADD.W           R0, R1, R0,LSR#31
BX              LR
\end{lstlisting}

\index{ARM!\Instructions!ASRS}
\IFRU{В режиме thumb отдельные инструкции для битовых сдвигов есть}
{There are separate instructions for shifting in thumb mode}, \IFRU{и здесь применяется одна из них}
{and one of them is used here}\EMDASH\TT{ASRS} (\IFRU{арифметический сдвиг вправо}{arithmetic shift right}).

\subsubsection{\NonOptimizing Xcode (LLVM) \AndENRU Keil}

\NonOptimizing LLVM 
\IFRU{не занимается генерацией подобного кода а вместо этого просто вставляет вызов
библиотечной функции \IT{\_\_\_divsi3}}
{does not generate code we saw before in this section, but inserts a call to library function 
\IT{\_\_\_divsi3} instead}.

\IFRU{А Keil во всех случаях вставляет вызов функции}
{What about Keil: it inserts call to library function} \IT{\_\_aeabi\_idivmod}\IFRU{}{ in all cases}.

\subsection{\IFRU{Определение делителя}{Getting divisor}}

\subsubsection{\IFRU{Вариант}{Variant} \#1}

\IFRU{Часто, код имеет вид}{Often, the code has a form of}:

\lstinputlisting{patterns/11_division_by_9/form_\IFRU{RU}{EN}.asm}

\IFRU{Определим 32-битную магическую константу через}{Let's denote 32-bit magical constant as} $M$, 
\IFRU{коэффициент сдвига через}{shifting coefficient by} $C$ \IFRU{и делитель через}{and divisor by} $D$.

\IFRU{Делитель который нам нужен это}{The divisor we need to get is}:

\[
D=\frac{2^{32} \cdot 2^C}{M}
\]

\IFRU{Например}{For example}:

\lstinputlisting[caption=\Optimizing MSVC 2012]{patterns/11_division_by_9/ex1.asm}

\IFRU{Это}{This is}:

\[
D=\frac{2^{32} \cdot 2^3}{2021161081}
\]

\index{Wolfram Mathematica}
\IFRU{Числа больше чем 32-битные, так что я использовал}
{Numbers are larger than 32-bit ones, so I use} Wolfram Mathematica \IFRU{для удобства}{for convenience}:

\begin{lstlisting}
In[1]:=N[2^32*2^3/2021161081]

Out[1]:=17.
\end{lstlisting}

\IFRU{Так что искомый делитель это}{So the divisor from the code I used for example is} 17.

\subsubsection{\IFRU{Вариант}{Variant} \#2}

\IFRU{Бывает также вариант с пропущенным арифметическим сдвигом, например}{A variant with omitted arithmetic
shift is also exist}:

\begin{lstlisting}
		mov     eax, 55555556h ; 1431655766
		imul    ecx
		mov     eax, edx
		shr     eax, 1Fh
\end{lstlisting}

\IFRU{Метод определения делителя упрощается}{The method of getting divisor is simplified}:

\[
D=\frac{2^{32}}{M}
\]

\IFRU{Для моего примера, это}{As of my example, this is}:

\[
D=\frac{2^{32}}{1431655766}
\]

\index{Wolfram Mathematica}
\IFRU{Снова использую}
{And again I use} Wolfram Mathematica:

\begin{lstlisting}
In[1]:=N[2^32/16^^55555556]

Out[1]:=3.
\end{lstlisting}

\IFRU{Искомый делитель это}{The divisor is} 3.


\chapter{\FPUChapterName}
\label{sec:FPU}

\newcommand{\FNURLSTACK}{\footnote{\href{http://go.yurichev.com/17123}{wikipedia.org/wiki/Stack\_machine}}}
\newcommand{\FNURLFORTH}{\footnote{\href{http://go.yurichev.com/17124}{wikipedia.org/wiki/Forth\_(programming\_language)}}}
\newcommand{\FNURLIEEE}{\footnote{\href{http://go.yurichev.com/17125}{wikipedia.org/wiki/IEEE\_floating\_point}}}
\newcommand{\FNURLSP}{\footnote{\href{http://go.yurichev.com/17126}{wikipedia.org/wiki/Single-precision\_floating-point\_format}}}
\newcommand{\FNURLDP}{\footnote{\href{http://go.yurichev.com/17127}{wikipedia.org/wiki/Double-precision\_floating-point\_format}}}
\newcommand{\FNURLEP}{\footnote{\href{http://go.yurichev.com/17128}{wikipedia.org/wiki/Extended\_precision}}}

\RU{\ac{FPU}\EMDASH блок в процессоре работающий с числами с плавающей запятой.}
\EN{The \ac{FPU} is a device within the main \ac{CPU}, specially designed to deal with floating point numbers.}
\RU{Раньше он назывался \q{сопроцессором} и он стоит немного в стороне от \ac{CPU}.}
\EN{It was called \q{coprocessor} in the past and it stays somewhat aside of the main \ac{CPU}.}

\section{IEEE 754}

\RU{Число с плавающей точкой в формате IEEE 754 состоит из \IT{знака}, \IT{мантиссы}\footnote{\IT{significand} или \IT{fraction} 
в англоязычной литературе} и \IT{экспоненты}.}
\EN{A number in the IEEE 754 format consists of a \IT{sign}, a \IT{significand} (also called \IT{fraction}) and an \IT{exponent}.}

\section{x86}

\RU{Перед изучением \ac{FPU} в x86 полезно ознакомиться с тем как работают стековые машины\FNURLSTACK 
или ознакомиться с основами языка Forth\FNURLFORTH.}
\EN{It is worth looking into stack machines\FNURLSTACK or learning the basics of the Forth language\FNURLFORTH,
before studying the \ac{FPU} in x86.}

\index{Intel!80486}
\index{Intel!FPU}
\RU{Интересен факт, что в свое время (до 80486) сопроцессор был отдельным чипом на материнской плате, 
и вследствие его высокой цены, он не всегда присутствовал. Его можно было докупить и установить отдельно}%
\EN{It is interesting to know that in the past (before the 80486 CPU) the coprocessor was a separate chip 
and it was not always pre-installed on the motherboard. It was possible to buy it separately and install it}%
\footnote{\RU{Например, Джон Кармак использовал в своей игре Doom числа с фиксированной запятой 
(\href{http://go.yurichev.com/17357}{ru.wikipedia.org/wiki/Число\_с\_фиксированной\_запятой}), хранящиеся
в обычных 32-битных \ac{GPR} (16 бит на целую часть и 16 на дробную),
чтобы Doom работал на 32-битных компьютерах без FPU, т.е. 80386 и 80486 SX.}
\EN{For example, John Carmack used fixed-point arithmetic 
(\href{http://go.yurichev.com/17356}{wikipedia.org/wiki/Fixed-point\_arithmetic}) values in his Doom video game, stored in 
32-bit \ac{GPR} registers (16 bit for integral part and another 16 bit for fractional part), so Doom
could work on 32-bit computers without FPU, i.e., 80386 and 80486 SX.}}.
\RU{Начиная с 80486 DX в состав процессора всегда входит FPU.}
\EN{Starting with the 80486 DX CPU, the \ac{FPU} is integrated in the \ac{CPU}.}

\index{x86!\Instructions!FWAIT}
\RU{Этот факт может напоминать такой рудимент как наличие инструкции \TT{FWAIT}, 
которая заставляет
\ac{CPU} ожидать, пока \ac{FPU} закончит работу}\EN{The \TT{FWAIT} instruction reminds us of that fact---it
switches the \ac{CPU} to a waiting state, so it can wait until the \ac{FPU} is done with its work}.
\RU{Другой рудимент это тот факт, что опкоды \ac{FPU}-инструкций начинаются с т.н. \q{escape}-опкодов 
(\TT{D8..DF}) как опкоды, передающиеся в отдельный сопроцессор.}
\EN{Another rudiment is the fact that the \ac{FPU} instruction 
opcodes start with the so called \q{escape}-opcodes (\TT{D8..DF}), i.e., 
opcodes passed to a separate coprocessor.}

\index{IEEE 754}
\label{FPU_is_stack}
\RU{FPU имеет стек из восьми 80-битных регистров:}
\EN{The FPU has a stack capable to holding 8 80-bit registers, and each register can hold a number 
in the IEEE 754\FNURLIEEE format.}
\RU{\ST{0}..\ST{7}. Для краткости, IDA и \olly отображают \ST{0} как \TT{ST},
что в некоторых учебниках и документациях означает \q{Stack Top} (\q{вершина стека}).}
\RU{Каждый регистр может содержать число в формате IEEE 754\FNURLIEEE.}
\EN{They are \ST{0}..\ST{7}. For brevity, IDA and \olly show \ST{0} as \TT{ST}, 
which is represented in some textbooks and manuals as \q{Stack Top}.}

\section{ARM, MIPS, x86/x64 SIMD}

\RU{В ARM и MIPS FPU это не стек, а просто набор регистров.}
\EN{In ARM and MIPS the FPU is not a stack, but a set of registers.}
\RU{Такая же идеология применяется в расширениях SIMD в процессорах x86/x64.}
\EN{The same ideology is used in the SIMD extensions of x86/x64 CPUs.}

\section{\CCpp}

\index{float}
\index{double}
\RU{В стандартных \CCpp имеются два типа для работы с числами с плавающей запятой: 
\Tfloat (\IT{число одинарной точности}\FNURLSP, 32 бита)
\footnote{Формат представления чисел с плавающей точкой одинарной точности затрагивается в разделе 
\IT{\WorkingWithFloatAsWithStructSubSubSectionName}~(\myref{sec:floatasstruct}).}
и \Tdouble (\IT{число двойной точности}\FNURLDP, 64 бита).}
\EN{The standard \CCpp languages offer at least two floating number types, \Tfloat (\IT{single-precision}\FNURLSP, 32 bits)
\footnote{the single precision floating point number format is also addressed in 
the \IT{\WorkingWithFloatAsWithStructSubSubSectionName}~(\myref{sec:floatasstruct}) section}
and \Tdouble (\IT{double-precision}\FNURLDP, 64 bits).}

\index{long double}
\RU{GCC также поддерживает тип \IT{long double} (\IT{extended precision}\FNURLEP, 80 бит), но MSVC~--- нет.}
\EN{GCC also supports the \IT{long double} type (\IT{extended precision}\FNURLEP, 80 bit), which MSVC doesn't.}

\RU{Несмотря на то, что \Tfloat занимает столько же места, сколько и \Tint на 32-битной архитектуре, 
представление чисел, разумеется, совершенно другое.}
\EN{The \Tfloat type requires the same number of bits as the \Tint type in 32-bit environments, 
but the number representation is completely different.}

\section{\RU{Простой пример}\EN{Simple example}}

\RU{Рассмотрим простой пример}\EN{Let's consider this simple example}:

\lstinputlisting{patterns/12_FPU/1_simple/simple.c}

\subsection{x86}

% subsubsections
\input{patterns/12_FPU/1_simple/MSVC}
\input{patterns/12_FPU/1_simple/GCC}

\ifdefined\IncludeARM
\subsection{ARM: \OptimizingXcodeIV (\ARMMode)}

\RU{Пока в ARM не было стандартного набора инструкций для работы с числами с плавающей точкой}%
\EN{Until ARM got standardized floating point support}, \RU{разные производители процессоров
могли добавлять свои расширения для работы с ними}\EN{several processor manufacturers added their own 
instructions extensions}.
\RU{Позже был принят стандарт}\EN{Then, } VFP (\IT{Vector Floating Point})\EN{ was standardized}.

\RU{Важное отличие от x86 в том, что там вы работаете с FPU-стеком, а здесь стека нет, 
вы работаете просто с регистрами.}
\EN{One important difference from x86 is that in ARM, there
is no stack, you work just with registers.}

\lstinputlisting[label=ARM_leaf_example10,caption=\OptimizingXcodeIV (\ARMMode)]{patterns/12_FPU/1_simple/ARM/Xcode_ARM_O3.asm.\LANG}

\index{ARM!D-\registers{}}
\index{ARM!S-\registers{}}
\RU{Итак, здесь мы видим использование новых регистров с префиксом D.}
\EN{So, we see here new some registers used, with D prefix.}
\RU{Это 64-битные регистры. Их 32 и их можно
использовать для чисел с плавающей точкой двойной точности (double) и для 
SIMD (в ARM это называется NEON).}
\EN{These are 64-bit registers, there are 32 of them, and they can be used both for floating-point numbers 
(double) but also for SIMD (it is called NEON here in ARM).}
\RU{Имеются также 32 32-битных S-регистра. Они применяются для работы с числами 
с плавающей точкой одинарной точности (float).}
\EN{There are also 32 32-bit S-registers, intended to be used for single precision 
floating pointer numbers (float).}
\RU{Запомнить легко: D-регистры предназначены для чисел double-точности, 
а S-регистры~--- для чисел single-точности.}
\EN{It is easy to remember: D-registers are for double precision numbers, while
S-registers---for single precision numbers.}
\RU{Больше об этом}\EN{More about it}: \myref{ARM_VFP_registers}.

\RU{Обе константы (3,14 и 4,1)}\EN{Both constants (3.14 and 4.1)} \RU{хранятся в памяти в формате IEEE 754.}
\EN{are stored in memory in IEEE 754 format.}

\index{ARM!\Instructions!VLDR}
\index{ARM!\Instructions!VMOV}
\RU{Инструкции }\TT{VLDR} \AndENRU \TT{VMOV}%
\RU{, как можно догадаться, это аналоги обычных \TT{LDR} и \MOV, но они работают с D-регистрами.}
\EN{, as it can be easily deduced, are analogous to the \TT{LDR} and \MOV instructions,
but they work with D-registers.}
\RU{Важно отметить, что эти инструкции, как и D-регистры, предназначены не только для работы 
с числами с плавающей точкой, но пригодны также и для работы с SIMD (NEON), и позже это также будет видно.}
\EN{It has to be noted that these instructions, just like the D-registers, are intended not only for
floating point numbers, 
but can be also used for SIMD (NEON) operations and this will also be shown soon.}

\RU{Аргументы передаются в функцию обычным путем через R-регистры, однако 
каждое число, имеющее двойную точность, занимает 64 бита, так что для передачи каждого нужны два R-регистра.}
\EN{The arguments are passed to the function in a common way, via the R-registers, however
each number that has double precision has a size of 64 bits, so two R-registers are needed to pass each one.}

\TT{VMOV D17, R0, R1} \RU{в самом начале составляет два 32-битных значения из \Reg{0} и \Reg{1} 
в одно 64-битное и сохраняет в}
\EN{at the start, composes two 32-bit values from \Reg{0} and \Reg{1} into one 64-bit value
and saves it to} \TT{D17}.

\TT{VMOV R0, R1, D16} \RU{в конце это обратная процедура}\EN{is the inverse operation}: 
\RU{то что было в}\EN{what was in} \TT{D16} 
\RU{остается в двух регистрах}\EN{is split in two registers,} \Reg{0} \AndENRU \Reg{1},
\RU{потому что}\EN{because} \RU{число с двойной точностью,}\EN{a double-precision number} 
\RU{занимающее 64 бита}\EN{that needs 64 bits for storage}, \RU{возвращается в паре регистров \Reg{0} и \Reg{1}.}
\EN{is returned in \Reg{0} and \Reg{1}.}

\index{ARM!\Instructions!VDIV}
\index{ARM!\Instructions!VMUL}
\index{ARM!\Instructions!VADD}
\TT{VDIV}, \TT{VMUL} \AndENRU \TT{VADD}, \RU{это инструкции для работы с числами 
с плавающей точкой, вычисляющие, соответственно, \glslink{quotient}{частное}, \glslink{product}{произведение} и сумму.}
\EN{are instruction for processing floating point numbers that compute \gls{quotient}, 
\gls{product} and sum, respectively.}

\RU{Код для Thumb-2 такой же.}\EN{The code for Thumb-2 is same.}

\subsection{ARM: \OptimizingKeilVI (\ThumbMode)}

\lstinputlisting{patterns/12_FPU/1_simple/ARM/Keil_O3_thumb.asm.\LANG}

\RU{Keil компилировал для процессора, в котором может и не быть поддержки FPU или NEON.}
\EN{Keil generated code for a processor without FPU or NEON support.}
\RU{Так что числа с двойной точностью передаются в парах обычных R-регистров,}
\EN{The double-precision floating-point numbers are passed via generic R-registers,}
\RU{а вместо FPU-инструкций вызываются сервисные библиотечные функции}
\EN{and instead of FPU-instructions, service library functions are called (like}
\TT{\_\_aeabi\_dmul}, \TT{\_\_aeabi\_ddiv}, \TT{\_\_aeabi\_dadd}%
\RU{, эмулирующие умножение, деление и сложение чисел с плавающей точкой.}
\EN{) which emulate multiplication, division and addition for floating-point numbers.}
\RU{Конечно, это медленнее чем FPU-сопроцессор, но лучше, чем ничего.}
\EN{Of course, that is slower than FPU-coprocessor, but still better than nothing.}

\RU{Кстати, похожие библиотеки для эмуляции сопроцессорных инструкций были очень распространены в x86 
когда сопроцессор был редким и дорогим и присутствовал далеко не во всех компьютерах.}
\EN{By the way, similar FPU-emulating libraries were very popular in the x86 world when coprocessors were rare
and expensive, and were installed only on expensive computers.}

\index{ARM!soft float}
\index{ARM!armel}
\index{ARM!armhf}
\index{ARM!hard float}
\RU{Эмуляция FPU-сопроцессора в ARM называется \IT{soft float} или \IT{armel} (\IT{emulation}),
а использование FPU-инструкций сопроцессора~--- \IT{hard float} или \IT{armhf}.}
\EN{The FPU-coprocessor emulation is called \IT{soft float} or \IT{armel} (\IT{emulation}) in the ARM world, 
while using the coprocessor's FPU-instructions is called \IT{hard float} or \IT{armhf}.}

\iffalse
% TODO разобраться...
\index{Raspberry Pi}
\RU{Ядро Linux, например, для Raspberry Pi может поставляться в двух вариантах.}
\EN{For example, the Linux kernel for Raspberry Pi is compiled in two variants.}
\RU{В случае \IT{soft float}, аргументы будут передаваться через R-регистры, 
а в случае \IT{hard float}, через D-регистры.}
\EN{In the \IT{soft float} case, arguments are passed via R-registers, and in the \IT{hard float} 
case---via D-registers.}

\RU{И это то, что помешает использовать, например, armhf-библиотеки
из armel-кода или наоборот, поэтому, весь код в дистрибутиве Linux должен быть скомпилирован
в соответствии с выбранным соглашением о вызовах.}
\EN{And that is what stops you from using armhf-libraries from armel-code or vice versa,
and that is
why all the code in Linux distributions must be compiled according to a single convention.}
\fi

\subsection{ARM64: \Optimizing GCC (Linaro) 4.9}

\RU{Очень компактный код}\EN{Very compact code}:

\lstinputlisting[caption=\Optimizing GCC (Linaro) 4.9]{patterns/12_FPU/1_simple/ARM/ARM64_GCC_O3.s.\LANG}

\subsection{ARM64: \NonOptimizing GCC (Linaro) 4.9}

\lstinputlisting[caption=\NonOptimizing GCC (Linaro) 4.9]{patterns/12_FPU/1_simple/ARM/ARM64_GCC_O0.s.\LANG}

\NonOptimizing GCC \RU{более многословный}\EN{is more verbose}.
\RU{Здесь много ненужных перетасовок значений, включая явно избыточный код 
(последние две инструкции \TT{GMOV}).}
\EN{There is a lot of unnecessary value shuffling, including some clearly redundant code 
(the last two \TT{FMOV} instructions).}
\RU{Должно быть}\EN{Probably}, GCC 4.9 \RU{пока ещё не очень хорош для генерации кода под ARM64}\EN{is not 
yet good in generating ARM64 code}.
\RU{Интересно заметить что у ARM64 64-битные регистры и D-регистры так же 64-битные.}
\EN{What is worth noting is that ARM64 has 64-bit registers, and the D-registers are 64-bit ones as well.}
\RU{Так что компилятор может сохранять значения типа \Tdouble в \ac{GPR} вместо локального стека.}
\EN{So the compiler is free to save values of type \Tdouble in \ac{GPR}s instead of the local stack.}
\RU{Это было невозможно на 32-битных CPU}\EN{This isn't possible on 32-bit CPUs}.

\RU{И снова, как упражнение, вы можете попробовать соптимизировать эту функцию вручную, без добавления
новых инструкций вроде \TT{FMADD}.}
\EN{And again, as an exercise, you can try to optimize this function manually, without introducing
new instructions like \TT{FMADD}.}

\fi
\ifdefined\IncludeMIPS
\subsection{MIPS}

\RU{MIPS может поддерживать несколько сопроцессоров (вплоть до 4), нулевой из которых это специальный
управляющий сопроцессор, а первый~--- это FPU.}
\EN{MIPS can support several coprocessors (up to 4), 
the zeroth of which is a special control coprocessor,
and first coprocessor is the FPU.}

\RU{Как и в ARM, сопроцессор в MIPS это не стековая машина. Он имеет 32 32-битных регистра (\$F0-\$F31):}
\EN{As in ARM, the MIPS coprocessor is not a stack machine, it has 32 32-bit registers (\$F0-\$F31):}
\myref{MIPS_FPU_registers}.
\RU{Когда нужно работать с 64-битными значениями типа \Tdouble, используется пара 32-битных F-регистров.}
\EN{When one needs to work with 64-bit \Tdouble values, a pair of 32-bit F-registers is used.}

\lstinputlisting[caption=\Optimizing GCC 4.4.5 (IDA)]{patterns/12_FPU/1_simple/MIPS_O3_IDA.lst.\LANG}

\RU{Новые инструкции}\EN{The new instructions here are}:

\begin{itemize}

\index{MIPS!\Instructions!LWC1}
\item LWC1 \RU{загружает 32-битное слово в регистр первого сопроцессора (отсюда \q{1} в названии инструкции).}
\EN{loads a 32-bit word into a register of the first coprocessor (hence \q{1} in instruction name).}
\index{MIPS!\Pseudoinstructions!L.D}
\RU{Пара инструкций LWC1 может быть объединена в одну псевдоинструкцию L.D.}
\EN{A pair of LWC1 instructions may be combined into a L.D pseudoinstruction.}

\index{MIPS!\Instructions!DIV.D}
\index{MIPS!\Instructions!MUL.D}
\index{MIPS!\Instructions!ADD.D}
\item DIV.D, MUL.D, ADD.D \RU{производят деление, умножение и сложение соответственно}\EN{do division, multiplication, and addition respectively} 
(\q{.D} \RU{в суффиксе означает двойную точность}\EN{in the suffix stands for double precision}, 
\q{.S}\RU{~--- одинарную точность}\EN{ stands for single precision})

\end{itemize}

\index{MIPS!\Instructions!LUI}
\index{\CompilerAnomaly}
\label{MIPS_FPU_LUI}
\RU{Здесь также имеется странная аномалия компилятора: инструкция \INS{LUI} помеченная нами вопросительным знаком.}%
\EN{There is also a weird compiler anomaly: the \INS{LUI} instructions that we've marked with a question mark.}
\RU{Мне трудно понять, зачем загружать часть 64-битной константы типа \Tdouble в регистр \$V0.}%
\EN{It's hard for me to understand why load a part of a 64-bit constant of \Tdouble type into the \$V0 register.}
\RU{От этих инструкций нет толка}\EN{These instruction have no effect}.
% TODO did you try checking out compiler source code?
\RU{Если кто-то об этом что-то знает, пожалуйста, напишите автору емейл}%
\EN{If someone knows more about it, please drop an email to author}\footnote{\EMAIL}.

\fi

\subsection{\RU{Передача чисел с плавающей запятой в аргументах}\EN{Passing floating point numbers via arguments}\DEph{}}
\myindex{\CStandardLibrary!pow()}

\lstinputlisting[style=customc]{patterns/12_FPU/2_passing_floats/pow.c}

\EN{\subsubsection{x86}

Let's see what we get in (MSVC 2010):

\lstinputlisting[caption=MSVC 2010,style=customasmx86]{patterns/12_FPU/2_passing_floats/MSVC_EN.asm}

\myindex{x86!\Instructions!FLD}
\myindex{x86!\Instructions!FSTP}

\FLD and \FSTP move variables between the data segment and the FPU stack. 
\GTT{pow()}\footnote{a standard C function, raises a number to the given power (exponentiation)}
takes both values from the stack and returns its result in the \ST{0} register.
\printf takes 8 bytes from the local stack and interprets them as \Tdouble type variable.

By the way, a pair of \MOV instructions could be used here for moving values from the memory
into the stack, because the values in memory are stored in IEEE 754 format, and pow() also takes them in this
format, so no conversion is necessary.
That's how it's done in the next example, for ARM: \myref{FPU_passing_floats_ARM}.

}
\RU{\subsubsection{x86}

Посмотрим, что у нас вышло (MSVC 2010):

\lstinputlisting[caption=MSVC 2010,style=customasmx86]{patterns/12_FPU/2_passing_floats/MSVC_RU.asm}

\myindex{x86!\Instructions!FLD}
\myindex{x86!\Instructions!FSTP}
\FLD и \FSTP перемещают переменные из сегмента данных в FPU-стек или обратно. 
\GTT{pow()}\footnote{стандартная функция Си, возводящая число в степень} достает оба значения из стека и 
возвращает результат в \ST{0}. 
\printf берет 8 байт из стека и трактует их как переменную типа \Tdouble.

Кстати, с тем же успехом можно было бы перекладывать эти два числа из памяти в стек при помощи пары \MOV:
 
ведь в памяти числа в формате IEEE 754, pow() также принимает их в том же
формате, и никакая конверсия не требуется.

Собственно, так и происходит в следующем примере с ARM: \myref{FPU_passing_floats_ARM}.

}
\DE{\subsubsection{x86}
Schauen wir uns an, was wir in MSVC 2010 erhalten:

\lstinputlisting[caption=MSVC 2010,style=customasmx86]{patterns/12_FPU/2_passing_floats/MSVC_DE.asm}

\myindex{x86!\Instructions!FLD}
\myindex{x86!\Instructions!FSTP}
% TODO bug to be fixed here:
\FLD und \FSTP verschieben Variablen zwischen Datensegment und dem FPU
Stack.\GTT{pow()}\footnote{eine Standard-C-Funktion, die eine Zahl potenziert}
nimmt beide Werte vom Stack der FPU und gibt ihr Ergebnis über das \ST{0} Register zurück. 
Die Funktion \printf nimmt 8 Byte vom lokalen Stack und interpretiert diese als
Variable von Typ \Tdouble.

Übrigens könnte hier auch ein Paar \MOV Befehle verwendet werden, um die Werte
aus dem Speicher zu holen und auf den Stack zu legen, denn die Werte sind im
Speicher im IEEE 754 Format abgelegt und pow() arbeitet mit diesem Format,
sodass keine Umwandlung notwendig ist.
Genau so wird es im folgenden Beispiel für ARM auch
gemacht:\myref{FPU_passing_floats_ARM}
}

\EN{\subsubsection{ARM + \NonOptimizingXcodeIV (\ThumbTwoMode)}
\label{FPU_passing_floats_ARM}

\lstinputlisting[style=customasmARM]{patterns/12_FPU/2_passing_floats/Xcode_thumb_O0.asm}

As it was mentioned before, 64-bit floating pointer numbers are passed in R-registers pairs.

This code is a bit redundant (certainly because optimization is turned off), 
since it is possible to load values into the R-registers directly without touching the D-registers.

So, as we see, the \GTT{\_pow} function receives its first argument in \Reg{0} and \Reg{1}, and its second one in \Reg{2} and \Reg{3}. 
The function leaves its result in \Reg{0} and \Reg{1}.
The result of \GTT{\_pow} is moved into \GTT{D16}, then in the \Reg{1} and \Reg{2} pair, from where \printf takes the resulting number.

\subsubsection{ARM + \NonOptimizingKeilVI (\ARMMode)}

\lstinputlisting[style=customasmARM]{patterns/12_FPU/2_passing_floats/Keil_ARM_O0.asm}

D-registers are not used here, just R-register pairs.

\subsubsection{ARM64 + \Optimizing GCC (Linaro) 4.9}

\lstinputlisting[caption=\Optimizing GCC (Linaro) 4.9,style=customasmARM]{patterns/12_FPU/2_passing_floats/ARM64_EN.s}

The constants are loaded into \RegD{0} and \RegD{1}: \TT{pow()} takes them from there.
The result will be in \RegD{0} after the execution of \TT{pow()}.
It is to be passed to \printf without any modification and moving, 
because \printf takes arguments of \glslink{integral type}{integral types} 
and pointers from X-registers, and floating point arguments from D-registers.

}
\RU{\subsubsection{ARM + \NonOptimizingXcodeIV (\ThumbTwoMode)}
\label{FPU_passing_floats_ARM}

\lstinputlisting[style=customasmARM]{patterns/12_FPU/2_passing_floats/Xcode_thumb_O0.asm}

Как уже было указано, 64-битные числа с плавающей точкой передаются в парах R-регистров.

Этот код слегка избыточен (наверное, потому что не включена оптимизация), ведь можно было бы 
загружать значения напрямую в R-регистры минуя загрузку в D-регистры.

Итак, видно, что функция \GTT{\_pow} получает первый аргумент в \Reg{0} и \Reg{1}, а второй в \Reg{2} и \Reg{3}. 
Функция оставляет результат в \Reg{0} и \Reg{1}.
Результат работы \GTT{\_pow} перекладывается в \GTT{D16}, 
затем в пару \Reg{1} и \Reg{2}, откуда 
\printf берет это число-результат.

\subsubsection{ARM + \NonOptimizingKeilVI (\ARMMode)}

\lstinputlisting[style=customasmARM]{patterns/12_FPU/2_passing_floats/Keil_ARM_O0.asm}

Здесь не используются D-регистры, используются только пары R-регистров.

\subsubsection{ARM64 + \Optimizing GCC (Linaro) 4.9}

\lstinputlisting[caption=\Optimizing GCC (Linaro) 4.9,style=customasmARM]{patterns/12_FPU/2_passing_floats/ARM64_RU.s}

Константы загружаются в \RegD{0} и \RegD{1}: 
функция \TT{pow()} берет их оттуда.
Результат в \RegD{0} после исполнения \TT{pow()}.
Он пропускается в \printf без всякой модификации и перемещений, 
потому что \printf берет аргументы \glslink{integral type}{интегральных типов} и указатели 
из X-регистров, а аргументы типа плавающей точки из D-регистров.

}
\DE{\subsubsection{ARM + \NonOptimizingXcodeIV (\ThumbTwoMode)}
\label{FPU_passing_floats_ARM}

\lstinputlisting[style=customasmARM]{patterns/12_FPU/2_passing_floats/Xcode_thumb_O0.asm}
Wie bereits vorher erwähnt werden Pointer auf 64-Bit-Fließkommazahlen über ein
Paar von R-Registern übergeben.

Dieser Code ist leicht redundant (sicherlich aufgrund der deaktivierten
Optimierung), da es möglich ist Werte direkt in die R-Register zu laden, ohne
die D-Register zu verwenden.

Wie wir also sehen erhält die \GTT{\_pow} Funktion ihr erster Argument in
\Reg{0} und \Reg{1} und das zweite in \Reg{2} und \Reg{3}. Die Funktion
speichert ihr Ergebnis in \Reg{0} und \Reg{1}. 
Das Ergebnis von \GTT{\_pow} wird zunächst nach \GTT{D16} und
anschließend in das Paar \Reg{1} und \Reg{2} verschoben, von wo aus \printf das
Ergebnis übernimmt. 

\subsubsection{ARM + \NonOptimizingKeilVI (\ARMMode)}

\lstinputlisting[style=customasmARM]{patterns/12_FPU/2_passing_floats/Keil_ARM_O0.asm}

Die D-Register werden hier nicht verwendet, sondern nur Paare von R-Registern.

\subsubsection{ARM64 + \Optimizing GCC (Linaro) 4.9}

\lstinputlisting[caption=\Optimizing GCC (Linaro) 4.9,style=customasmARM]{patterns/12_FPU/2_passing_floats/ARM64_DE.s}

Die Konstanten werden nach \RegD{0} und \RegD{1} geladen: \TT{pow()} übernimmt
sie von dort. Das Ergebnis befindet sich nach der Ausführung von \TT{pow()} in
\RegD{0}. 
Es wird ohne weitere Änderung oder Verschiebung an die Funktion \printf
übergeben, da \printf ganzzahlige Werte und Pointer aus X-Registern,
Fließkommaparameter jedoch aus D-Registern übernimmt.

}

\EN{\subsubsection{MIPS}

\lstinputlisting[caption=\Optimizing GCC 4.4.5 (IDA),style=customasmMIPS]{patterns/12_FPU/2_passing_floats/MIPS_O3_IDA_EN.lst}

And again, we see here \INS{LUI} loading a 32-bit part of a \Tdouble number into \$V0.
And again, it's hard to comprehend why.

\myindex{MIPS!\Instructions!MFC1}

The new instruction for us here is \INS{MFC1} (\q{Move From Coprocessor 1}).
The FPU is coprocessor number 1, hence \q{1} in the instruction name.
This instruction transfers values from the coprocessor's registers to the registers of the CPU (\ac{GPR}).
So at the end the result of \TT{pow()} is moved to registers \$A3 and \$A2, 
and \printf takes a 64-bit double value from this register pair.

}
\RU{\subsubsection{MIPS}

\lstinputlisting[caption=\Optimizing GCC 4.4.5 (IDA),style=customasmMIPS]{patterns/12_FPU/2_passing_floats/MIPS_O3_IDA_RU.lst}

И снова мы здесь видим, как \INS{LUI} загружает 32-битную часть числа типа \Tdouble в \$V0.
И снова трудно понять почему.

\myindex{MIPS!\Instructions!MFC1}
Новая для нас инструкция это \INS{MFC1} (\q{Move From Coprocessor 1}) (копировать из первого сопроцессора).
FPU это сопроцессор под номером 1, вот откуда \q{1} в имени инструкции.
Эта инструкция переносит значения из регистров сопроцессора в регистры основного CPU (\ac{GPR}).
Так что результат исполнения \TT{pow()} в итоге копируется в регистры \$A3 и \$A2
и из этой пары регистров \printf берет его как 64-битное значение типа \Tdouble.

}
\DE{\subsubsection{MIPS}

\lstinputlisting[caption=\Optimizing GCC 4.4.5
(IDA)]{patterns/12_FPU/2_passing_floats/MIPS_O3_IDA_DE.lst}
Und wieder sehen wir hier, dass der Befehl \INS{LUI} einen 32-Bit-Teil einer
\Tdouble Zahl nach \$V0 lädt.
Und wiederum ist es schwer nachzuvollziehen warum dies geschieht.

\myindex{MIPS!\Instructions!MFC1}
Der für uns neue Befehl an dieser Stelle ist \INS{MFC1}(\q{Move From Coprocessor
1}). Die Nummer des FPU-Koprozessors ist 1, daher die \q{1} im Namen des
Befehls. 
Dieser Befehl überträgt Werte aus den Registern des Koprozessors in die Register
der CPU (\ac{GPR}).
Auf diese Weise wird das Ergebnis von \TT{pow()} schließlich in die Register
\$A3 und \$A2 verschoben und \printf übernimmt einen 64-Bit-Wert von doppelter
Genauigkeit aus diesem Registerpaar.}


\section{\RU{Пример с сравнением}\EN{Comparison example}}

\RU{Попробуем теперь вот это:}\EN{Let's try this:}

\lstinputlisting{patterns/12_FPU/3_comparison/d_max.c}

\RU{Несмотря на кажущуюся простоту этой функции, понять, как она работает, будет чуть сложнее.}
\EN{Despite the simplicity of the function, it will be harder to understand how it works.}

% subsections
\subsubsection{x86}

% subsubsections
\EN{\input{patterns/12_FPU/3_comparison/x86/MSVC/main_EN}}
\RU{\input{patterns/12_FPU/3_comparison/x86/MSVC/main_RU}}
\DE{\input{patterns/12_FPU/3_comparison/x86/MSVC/main_DE}}
\FR{\input{patterns/12_FPU/3_comparison/x86/MSVC/main_FR}}

\EN{\input{patterns/12_FPU/3_comparison/x86/MSVC_Ox/main_EN}}
\RU{\input{patterns/12_FPU/3_comparison/x86/MSVC_Ox/main_RU}}
\DE{\input{patterns/12_FPU/3_comparison/x86/MSVC_Ox/main_DE}}
\FR{\input{patterns/12_FPU/3_comparison/x86/MSVC_Ox/main_FR}}

\EN{\input{patterns/12_FPU/3_comparison/x86/GCC_EN}}
\RU{\input{patterns/12_FPU/3_comparison/x86/GCC_RU}}
\DE{\input{patterns/12_FPU/3_comparison/x86/GCC_DE}}
\FR{\input{patterns/12_FPU/3_comparison/x86/GCC_FR}}

\EN{\input{patterns/12_FPU/3_comparison/x86/GCC_O3_EN}}
\RU{\input{patterns/12_FPU/3_comparison/x86/GCC_O3_RU}}
\DE{\input{patterns/12_FPU/3_comparison/x86/GCC_O3_DE}}
\FR{\input{patterns/12_FPU/3_comparison/x86/GCC_O3_FR}}

\EN{\input{patterns/12_FPU/3_comparison/x86/GCC481_O3_EN}}
\RU{\input{patterns/12_FPU/3_comparison/x86/GCC481_O3_RU}}
\DE{\input{patterns/12_FPU/3_comparison/x86/GCC481_O3_DE}}
\FR{\input{patterns/12_FPU/3_comparison/x86/GCC481_O3_FR}}


\ifdefined\IncludeARM
\subsection{ARM}

\subsubsection{\OptimizingXcodeIV (\ARMMode)}

\lstinputlisting[caption=\OptimizingXcodeIV (\ARMMode)]{patterns/12_FPU/3_comparison/ARM/Xcode_ARM.lst.\LANG}

\index{ARM!\Registers!APSR}
\index{ARM!\Registers!FPSCR}
\RU{Очень простой случай.}\EN{A very simple case.}
\RU{Входные величины помещаются в}\EN{The input values are placed into the} \TT{D17} \AndENRU \TT{D16} 
\RU{и сравниваются при помощи инструкции}\EN{registers and then compared using the} 
\TT{VCMPE}\EN{ instruction}.
\RU{Как и в сопроцессорах x86, сопроцессор в ARM имеет свой собственный регистр статуса и флагов}%
\EN{Just like in the x86 coprocessor, the ARM coprocessor has its own status and flags register} (\ac{FPSCR}),
\RU{потому что есть необходимость хранить специфичные для его работы флаги.}
\EN{since there is a need to store coprocessor-specific flags.}
% TODO -> расписать регистр по битам
\index{ARM!\Instructions!VMRS}
\RU{И так же, как и в x86}\EN{And just like in x86}, 
\RU{в ARM нет инструкций условного перехода}%
\EN{there are no conditional jump instruction in ARM}, 
\RU{проверяющих биты в регистре статуса сопроцессора}\EN{that can check bits in the status register of the coprocessor}. 
\RU{Поэтому имеется инструкция}\EN{So there is} \TT{VMRS}%
\RU{, копирующая 4 бита}\EN{, which copies 4 bits} (N, Z, C, V) 
\RU{из статуса сопроцессора в биты \IT{общего} статуса (регистр \ac{APSR}).}
\EN{from the coprocessor status word into bits of the \IT{general} status register (\ac{APSR}).}

\index{ARM!\Instructions!VMOVGT}
\TT{VMOVGT} \RU{это аналог}\EN{is the analog of the} \TT{MOVGT}, 
\RU{инструкция для D-регистров, срабатывающая, если при сравнении один операнд был больше чем второй}
\EN{instruction for D-registers, it executes if one operand is greater than the other while comparing} 
(\IT{GT\EMDASH{}Greater Than}). 

\RU{Если она сработает}\EN{If it gets executed}, 
\RU{в \TT{D16} запишется значение $b$}\EN{the value of $b$ is to be written into \TT{D16}}%
\RU{, лежащее в тот момент в}\EN{(that is currently stored in in} \TT{D17}\EN{)}.

\RU{В обратном случае}\EN{Otherwise} 
\RU{в \TT{D16} остается значение $a$.}
\EN{the value of $a$ stays in the \TT{D16} register.}

\index{ARM!\Instructions!VMOV}
\RU{Предпоследняя инструкция \TT{VMOV} готовит то, что было в \TT{D16}, для возврата через 
пару регистров \Reg{0} и \Reg{1}.}
\EN{The penultimate instruction \TT{VMOV} prepares the value in the \TT{D16} register for returning it via the \Reg{0} and \Reg{1}
register pair.}

\subsubsection{\OptimizingXcodeIV (\ThumbTwoMode)}

\begin{lstlisting}[caption=\OptimizingXcodeIV (\ThumbTwoMode)]
VMOV            D16, R2, R3 ; b
VMOV            D17, R0, R1 ; a
VCMPE.F64       D17, D16
VMRS            APSR_nzcv, FPSCR
IT GT 
VMOVGT.F64      D16, D17
VMOV            R0, R1, D16
BX              LR
\end{lstlisting}

\RU{Почти то же самое, что и в предыдущем примере, за парой отличий.}
\EN{Almost the same as in the previous example, however slightly different.}
\RU{Как мы уже знаем, многие инструкции в режиме ARM можно дополнять условием.}
\EN{As we already know, many instructions in ARM mode can be supplemented by condition predicate.}

\RU{Но в режиме Thumb такого нет.}
\EN{But there is no such thing in Thumb mode.} 
\RU{В 16-битных инструкций просто нет места для лишних 4 битов, при помощи
которых можно было бы закодировать условие выполнения.}
\EN{There is no space in the 16-bit instructions for 4 more bits in which conditions can be encoded.}

\index{ARM!\ThumbTwoMode}
\RU{Поэтому в Thumb-2 добавили возможность дополнять Thumb-инструкции условиями.}
\EN{However, Thumb-2 was extended to make it possible to specify predicates to old Thumb instructions.}

\RU{В листинге, сгенерированном при помощи \IDA, мы видим инструкцию \TT{VMOVGT}, 
такую же как и в предыдущем примере.}
\EN{Here, in the \IDA-generated listing, we see the \TT{VMOVGT} instruction, as in previous example.}

\RU{В реальности}\EN{In fact,} 
\RU{там закодирована обычная инструкция \TT{VMOV}}%
\EN{the usual \TT{VMOV} is encoded there}, 
\RU{просто \IDA добавила суффикс \TT{-GT} к ней}%
\EN{but \IDA adds the \TT{-GT} suffix to it}, 
\RU{потому что перед этой инструкцией стоит \TT{IT GT}.}
\EN{since there is a \TT{\q{IT GT}} instruction placed right before it.}

\label{ARM_Thumb_IT}
\index{ARM!\Instructions!IT}
\index{ARM!if-then block}
\EN{The}\RU{Инструкция} \TT{IT} \RU{определяет так называемый}\EN{instruction defines a so-called} \IT{if-then block}. 
\RU{После этой инструкции можно указывать до четырех инструкций, 
к каждой из которых будет добавлен суффикс условия.}
\EN{After the instruction it is possible to place up to 4 instructions, 
each of them has a predicate suffix.}
\RU{В нашем примере}\EN{In our example,} \TT{IT GT} \RU{означает,}\EN{implies}
\RU{что следующая за ней инструкция будет исполнена, если условие}
\EN{that the next instruction is to be executed, if the}
\IT{GT} (\IT{Greater Than}) \RU{справедливо}\EN{condition is true}.

\index{Angry Birds}
\RU{Теперь более сложный пример. Кстати, из}\EN{Here is a more complex code fragment, by the way, from} 
Angry Birds (\RU{для}\EN{for} iOS):

% FIXME russian listing:
\begin{lstlisting}[caption=Angry Birds Classic]
...
ITE NE
VMOVNE          R2, R3, D16
VMOVEQ          R2, R3, D17
BLX             _objc_msgSend ; not prefixed
...
\end{lstlisting}

\TT{ITE} \RU{означает}\EN{stands for} \IT{if-then-else} 
\RU{и кодирует суффиксы для двух следующих за ней инструкций.}
\EN{and it encodes suffixes for the next two instructions.}
\RU{Первая из них исполнится, если условие, закодированное в}
\EN{The first instruction executes if the condition encoded in} \TT{ITE} (\IT{NE, not equal}) 
\RU{будет в тот момент справедливо}\EN{is true at},
\RU{а вторая~--- если это условие не сработает}\EN{and the 
second---if the condition is not true}.
(\RU{Обратное условие от}\EN{The inverse condition of} \TT{NE} \RU{это}\EN{is} \TT{EQ} (\IT{equal})).

\EN{The instruction followed after the second VMOV (or VMOVEQ) is a normal one, not prefixed (BLX).}
\RU{Инструкция следующая за второй VMOV (или VMOEQ) нормальная, без префикса (BLX).}

\index{Angry Birds}
\RU{Ещё чуть сложнее}\EN{One more that's slightly harder}, 
\RU{и снова этот фрагмент из}\EN{which is also from} Angry Birds:

% FIXME russian listing:
\begin{lstlisting}[caption=Angry Birds Classic]
...
ITTTT EQ
MOVEQ           R0, R4
ADDEQ           SP, SP, #0x20
POPEQ.W         {R8,R10}
POPEQ           {R4-R7,PC}
BLX             ___stack_chk_fail ; not prefixed
...
\end{lstlisting}

\RU{Четыре символа \q{T} в инструкции означают, что четыре последующие инструкции будут исполнены если условие соблюдается.}
\EN{Four \q{T} symbols in the instruction mnemonic mean 
that the four subsequent instructions are to be executed if the condition is true.}
\RU{Поэтому \IDA добавила ко всем четырем инструкциям суффикс}
\EN{That's why \IDA adds the} \TT{-EQ}\EN{ suffix
to each one of them}. 

\RU{А если бы здесь было, например,}\EN{And if there was be, for example,}
\TT{ITEEE EQ} (\IT{if-then-else-else-else}), 
\RU{тогда суффиксы для следующих четырех инструкций были бы расставлены так:}
\EN{then the suffixes would have been set as follows:}

\begin{lstlisting}
-EQ
-NE
-NE
-NE
\end{lstlisting}

\index{Angry Birds}
\RU{Ещё фрагмент из}\EN{Another fragment from} Angry Birds:

% FIXME russian listing:
\begin{lstlisting}[caption=Angry Birds Classic]
...
CMP.W           R0, #0xFFFFFFFF
ITTE LE
SUBLE.W         R10, R0, #1
NEGLE           R0, R0
MOVGT           R10, R0
MOVS            R6, #0         ; not prefixed
CBZ             R0, loc_1E7E32 ; not prefixed
...
\end{lstlisting}

\TT{ITTE} (\IT{if-then-then-else}) 
\RU{означает, что первая и вторая инструкции исполнятся, если условие \TT{LE} (\IT{Less or Equal})
справедливо, а третья~--- если справедливо обратное условие (\TT{GT}\EMDASH\IT{Greater Than}).}
\EN{implies that the 1st and 2nd instructions are to be executed if the \TT{LE} (\IT{Less or Equal})
condition is true, and the 3rd---if the inverse condition (\TT{GT}\EMDASH\IT{Greater Than}) 
is true.}

\RU{Компиляторы способны генерировать далеко не все варианты.}
\EN{Compilers usually don't generate all possible combinations.}
\index{Angry Birds}
\RU{Например, в вышеупомянутой игре Angry Birds (версия \IT{classic} для iOS)}
\EN{For example, in the mentioned Angry Birds game (\IT{classic} version for iOS)}
\RU{встречаются только такие варианты инструкции \TT{IT}}\EN{only these variants of the \TT{IT} instruction are used}: 
\TT{IT}, \TT{ITE}, \TT{ITT}, \TT{ITTE}, \TT{ITTT}, \TT{ITTTT}.
\index{\GrepUsage}
\RU{Как это узнать?}\EN{How to learn this?}
\RU{В \IDA можно сгенерировать листинг (что и было сделано), только в опциях был установлен показ 4 байтов для каждого опкода.}
\EN{In \IDA It is possible to produce listing files, so it was created with an option to show 4 bytes for each opcode.}
\RU{Затем, зная что старшая часть 16-битного опкода (\TT{IT} это \TT{0xBF}), сделаем при помощи \TT{grep} это:}
\EN{Then, knowing the high part of the 16-bit opcode (\TT{IT} is \TT{0xBF}), we do the following using \TT{grep}:}

\begin{lstlisting}
cat AngryBirdsClassic.lst | grep " BF" | grep "IT" > results.lst
\end{lstlisting}

\index{ARM!\ThumbTwoMode}
\RU{Кстати, если писать на ассемблере для режима Thumb-2 вручную, и дополнять инструкции суффиксами
условия, то ассемблер автоматически будет добавлять инструкцию \TT{IT} с соответствующими флагами там,
где надо.}
\EN{By the way, if you program in ARM assembly language manually for Thumb-2 mode, 
and you add conditional suffixes,
the assembler will add the \TT{IT} instructions automatically with the required flags where it is necessary.}

\subsubsection{\NonOptimizingXcodeIV (\ARMMode)}

\begin{lstlisting}[caption=\NonOptimizingXcodeIV (\ARMMode)]
b               = -0x20
a               = -0x18
val_to_return   = -0x10
saved_R7        = -4

                STR             R7, [SP,#saved_R7]!
                MOV             R7, SP
                SUB             SP, SP, #0x1C
                BIC             SP, SP, #7
                VMOV            D16, R2, R3
                VMOV            D17, R0, R1
                VSTR            D17, [SP,#0x20+a]
                VSTR            D16, [SP,#0x20+b]
                VLDR            D16, [SP,#0x20+a]
                VLDR            D17, [SP,#0x20+b]
                VCMPE.F64       D16, D17
                VMRS            APSR_nzcv, FPSCR
                BLE             loc_2E08
                VLDR            D16, [SP,#0x20+a]
                VSTR            D16, [SP,#0x20+val_to_return]
                B               loc_2E10

loc_2E08
                VLDR            D16, [SP,#0x20+b]
                VSTR            D16, [SP,#0x20+val_to_return]

loc_2E10
                VLDR            D16, [SP,#0x20+val_to_return]
                VMOV            R0, R1, D16
                MOV             SP, R7
                LDR             R7, [SP+0x20+b],#4
                BX              LR
\end{lstlisting}

\RU{Почти то же самое, что мы уже видели}\EN{Almost the same as we already saw}, 
\RU{но много избыточного кода из-за хранения $a$ и $b$, 
а также выходного значения, в локальном стеке.}
\EN{but there is too much redundant code because the $a$ and $b$ variables are stored in the local stack, as well
as the return value.}

\subsubsection{\OptimizingKeilVI (\ThumbMode)}

\begin{lstlisting}[caption=\OptimizingKeilVI (\ThumbMode)]
                PUSH    {R3-R7,LR}
                MOVS    R4, R2
                MOVS    R5, R3
                MOVS    R6, R0
                MOVS    R7, R1
                BL      __aeabi_cdrcmple
                BCS     loc_1C0
                MOVS    R0, R6
                MOVS    R1, R7
                POP     {R3-R7,PC}

loc_1C0
                MOVS    R0, R4
                MOVS    R1, R5
                POP     {R3-R7,PC}
\end{lstlisting}

\RU{Keil не генерирует FPU-инструкции, потому что не 
рассчитывает на то, что они будет поддерживаться, а простым сравнением побитово здесь не обойтись.}
\EN{Keil doesn't generate FPU-instructions since it cannot rely on them being
supported on the target CPU, and it cannot be done by straightforward bitwise comparing.}
%TODO1: why?
\RU{Для сравнения вызывается библиотечная функция}\EN{So it calls an external library
function to do the comparison:} \TT{\_\_aeabi\_cdrcmple}. 
\index{ARM!\Instructions!BCS}\\
\\
N.B. \RU{Результат
сравнения эта функция оставляет в флагах, чтобы следующая за вызовом инструкция}
\EN{The result of the comparison is to be left in the flags by this function, so the following}
\TT{BCS} (\IT{Carry set\RU{~}---\RU{ }Greater than or equal})
\RU{могла работать без дополнительного кода.}\EN{instruction can work without any additional code.}

\subsection{ARM64}

\subsubsection{\Optimizing GCC (Linaro) 4.9}

\lstinputlisting{patterns/12_FPU/3_comparison/ARM/ARM64_GCC_O3.lst.\LANG}

\RU{В }ARM64 \ac{ISA} \RU{теперь есть FPU-инструкции устанавливающие флаги CPU}\EN{now also have FPU-instructions 
which sets} \ac{APSR} \RU{вместо}\EN{CPU flags instead of} \ac{FPSCR}, \EN{for convenience}\RU{для удобства}.
\ac{FPU} \RU{больше не отдельное устройство (по крайней мере, логически)}\EN{is not separate device here 
anymore (at least, logically)}.
\index{ARM!\Instructions!FCMPE}
\RU{Это}\EN{That is} \TT{FCMPE}, \RU{она сравнивает два значения, переданных в}\EN{it compares two values, 
passed here in} \RegD{0} \AndENRU \RegD{1} 
(\RU{а это первый и второй аргументы ф-ции}\EN{which are first and second function arguments})
\RU{и выставляет флаги в}\EN{and sets} \ac{APSR}\EN{ flags} (N, Z, C, V).

\index{ARM!\Instructions!FCSEL}
\TT{FCSEL} (\IT{Floating Conditional Select}) \RU{копирует значение}\EN{copies value of} \RegD{0} \OrENRU 
\RegD{1} \RU{в}\EN{into} \RegD{0} \RU{в зависимости от условия}\EN{depending on condition} 
(\TT{GT} (\IT{Greater Than}\RU{ (больше чем)}) \RU{здесь}\EN{here}), 
\RU{и снова, она использует флаги в регистре}\EN{and again, it uses flags in} \ac{APSR} \RU{вместо}\EN{register
instead of} \ac{FPSCR}.
\RU{Это куда удобнее, если сравнивать с тем набором инструкций, что был в процессорах раньше.}
\EN{This is much more convenient, if to compare to the instruction set in older CPUs.}

\RU{Если условие верно}\EN{If condition is true} (\TT{GT}) \RU{тогда значение из}\EN{then value of} \RegD{0} 
\RU{копируется в}\EN{is copied into} \RegD{0} (\RU{т.е., ничего не происходит}\EN{i.e., nothing happens}).
\RU{Если условие не верно, то значение}\EN{If condition is not true, value of} \RegD{1} 
\RU{копируется в}\EN{is copied into} \RegD{0}.

\subsubsection{\NonOptimizing GCC (Linaro) 4.9}

\lstinputlisting{patterns/12_FPU/3_comparison/ARM/ARM64_GCC.lst.\LANG}

\RU{Неоптимизирующий GCC более многословен}\EN{Non-optimizing GCC is more verbose}.
\RU{В начале, ф-ция сохраняет значения входных аргументов в локальном стеке}
\EN{First, function saves input argument values in the local stack} (\IT{Register Save Area}).
\RU{Затем код перезагружает значения в регистры}\EN{Then the code reloads these values into}
\RegX{0}/\RegX{1} \RU{и наконец копирует их в}\EN{registers and finally copies them into} 
\RegD{0}/\RegD{1} \RU{для сравнения инструкцией}\EN{for comparison using} \TT{FCMPE}. 
\RU{Много избыточного кода, но так работают неоптимизирующие компиляторы}\EN{A lot of redundant code, 
but that is how non-optimizing compiler may work}.
\TT{FCMPE} \RU{сравнивает значения и устанавливает флаги в}\EN{compare values and set} \ac{APSR}\EN{ flags}.
\RU{В этот момент, компилятор еще не думает о более удобной инструкции}\EN{At this moment, 
compiler is not yet thinking about more convenient} \TT{FCSEL} \RU{так что он работает по старым 
методам}\EN{instruction, so it proceed to old methods}: 
\RU{использует инструкцию}\EN{using} \TT{BLE}\EN{ instruction} (\IT{Branch if Less than or Equal}\RU{ (переход
если меньше или равно)}).
\RU{В одном случае}\EN{In one case} ($a>b$), \RU{значение }$a$ \RU{перезагружается в}\EN{value is reloaded 
into} \RegX{0}.
\RU{В другом случае}\EN{In other case} ($a<=b$), \RU{значение }$b$ \RU{загружается в}\EN{value is placed in} 
\RegX{0}.
\RU{Наконец, значение из}\EN{Finally, value from} \RegX{0} \RU{копируется в}\EN{copied into} \RegD{0}, 
\RU{потому что возвращаемое значение оставляется в этом регистре}\EN{because returning value is leaved in this 
register}.

\myparagraph{\Exercise}

\RU{Для упражнения, вы можете попробовать оптимизировать этот фрагмент кода вручную, удалив избыточные инструкции,
но не добавляя новых (включая \TT{FCSEL})}\EN{As an exercise, you may try to optimize this piece of code 
manually by removing redundant instructions, but do not introduce new ones (including \TT{FCSEL})}.

\subsubsection{\Optimizing GCC (Linaro) 4.9\EMDASH{}float}

\RU{Я еще переписал пример, теперь здесь \Tfloat вместо \Tdouble}\EN{I also rewrote this example, 
now \Tfloat is used instead of \Tdouble}.

\begin{lstlisting}
float f_max (float a, float b)
{
	if (a>b)
		return a;

	return b;
};
\end{lstlisting}

\lstinputlisting{patterns/12_FPU/3_comparison/ARM/ARM64_GCC_O3_float.lst.\LANG}

\RU{Всё то же самое, только используются S-регистры вместо D-.}
\EN{It is a very same code, but S-registers are used instead of D- ones.}
\RU{Так что числа типа \Tfloat передаются в 32-битных S-регистрах (а это младшие части 64-битных D-регистров).}
\EN{So numbers of \Tfloat type is passed in 32-bit S-registers (which are in fact lower parts of 64-bit D-registers).}


\fi
\ifdefined\IncludeMIPS
\subsection{MIPS}

\index{MIPS!\Registers!FCCR}
\EN{Most popular MIPS FPU coprocessor has only one condition bit which can be set in FPU 
and checked in CPU.}
\RU{В сопроцессоре наиболее популярных MIPS есть только один бит результата который устанавливается в FPU и 
проверяется в CPU.}
\EN{Earlier MIPS-es has only one condition bit (called FCC0), later has 8 (called FCC7-FCC0).}
\RU{Ранние MIPS имели только один бит (с названием FCC0), у поздних их 8 (с названием FCC7-FCC0).}
\RU{Эти биты находятся в регистре с названием FCCR.}
\EN{These bits are located in register named FCCR.}

\lstinputlisting[caption=\Optimizing GCC 4.4.5 (IDA)]{patterns/12_FPU/3_comparison/MIPS_O3_IDA.lst.\LANG}

\index{MIPS!\Instructions!C.LT.D}
``C.LT.D'' \EN{is comparing two values}\RU{сравнивает два значения}. 
``LT'' \EN{is condition}\RU{это условие} ``Less Than''\RU{ (меньше чем)}.
``D'' \EN{mean values of type}\RU{означает переменные типа} \Tdouble.
\EN{Depending on comparison result, FCC0 condition bit is set or cleared.}
\RU{В зависимости от результата сравнения, бит FCC0 устанавливается или очищается.}

\index{MIPS!\Instructions!BC1T}
\index{MIPS!\Instructions!BC1F}
``BC1T'' \EN{checks for FCC0 bit and takes jump if bit is set}\RU{проверяет бит FCC0 и делает переход, если бит выставлен}.
``T'' \EN{mean jump taken if bit is set}\RU{означает что переход произойдет если бит выставлен} (``True'').
\EN{There are also instruction}\RU{Имеется также инструцкия} ``BC1F'' \EN{which takes jump if bit is cleared}\RU{которая сработает, если бит сброшен} (``False'').

\RU{В зависимости от перехода, один из аргументов ф-ции помещается в регистр \$F0.}
\EN{Depending on jump, one of function arguments is placed into \$F0 register.}

\fi


\section{\RU{Стек, калькуляторы и обратная польская запись}\EN{Stack, calculators and reverse Polish notation}}

\index{\RU{Обратная польская запись}\EN{Reverse Polish notation}}
\RU{Теперь понятно, почему некоторые старые калькуляторы использовали обратную польскую запись%
\footnote{\href{http://go.yurichev.com/17355}{ru.wikipedia.org/wiki/Обратная\_польская\_запись}}.}
\EN{Now we undestand why some old calculators used reverse Polish notation
\footnote{\href{http://go.yurichev.com/17354}{wikipedia.org/wiki/Reverse\_Polish\_notation}}.}
\RU{Например для сложения 12 и 34 нужно было набрать 12, потом 34, потом нажать знак \q{плюс}.}
\EN{For example, for addition of 12 and 34 one has to enter 12, then 34, then press \q{plus} sign.}
\RU{Это потому что старые калькуляторы просто реализовали стековую машину и это было куда проще, 
чем обрабатывать сложные выражения со скобками.}
\EN{It's because old calculators were just stack machine implementations, and this was much simpler
than to handle complex parenthesized expressions.}
\section{x64}

\RU{О том, как происходит работа с числами с плавающей запятой в x86-64, читайте здесь: \myref{floating_SIMD}.}
\EN{On how floating point numbers are processed in x86-64, read more here: \myref{floating_SIMD}.}

% sections
\ifdefined\IncludeExercises
\section{\Exercises}

\subsection{\Exercise \#1}

\RU{Избавтесь от инструкции}\EN{Eliminate} FXCH \RU{в примере}\EN{instruciton in example} 
\ref{gcc481_o3} \RU{и протестируйте его}\EN{and test it}.

\subsection{\Exercise \#2}
\label{exercise_FPU_2}

\WhatThisCodeDoes\

\begin{lstlisting}[caption=\Optimizing MSVC 2010]
__real@4014000000000000 DQ 04014000000000000r	; 5

_a1$ = 8	; size = 8
_a2$ = 16	; size = 8
_a3$ = 24	; size = 8
_a4$ = 32	; size = 8
_a5$ = 40	; size = 8
_f	PROC
	fld	QWORD PTR _a1$[esp-4]
	fadd	QWORD PTR _a2$[esp-4]
	fadd	QWORD PTR _a3$[esp-4]
	fadd	QWORD PTR _a4$[esp-4]
	fadd	QWORD PTR _a5$[esp-4]
	fdiv	QWORD PTR __real@4014000000000000
	ret	0
_f	ENDP
\end{lstlisting}

\begin{lstlisting}[caption=\NonOptimizingKeilVI (\ThumbMode{} / \RU{скомпилировано для}\EN{compiled for} Cortex-R4F CPU)]
f PROC
        VADD.F64 d0,d0,d1
        VMOV.F64 d1,#5.00000000
        VADD.F64 d0,d0,d2
        VADD.F64 d0,d0,d3
        VADD.F64 d2,d0,d4
        VDIV.F64 d0,d2,d1
        BX       lr
        ENDP
\end{lstlisting}

\Answer\: \ref{exercise_solutions_FPU_2}.

\fi

\chapter{\Arrays}
\label{arrays}

\RU{Массив, это просто набор переменных в памяти, 
обязательно лежащих рядом, и обязательно одного типа
\footnote{\ac{AKA} ``гомогенный контейнер''}.}
\EN{Array is just a set of variables in memory, 
always lying next to each other, always has same type
\footnote{\ac{AKA} ``homogeneous container''}.}

% sections
\subsection{\RU{Простой пример}\EN{Simple example}}

\label{arrays_simple}
\lstinputlisting[style=customc]{patterns/13_arrays/1_simple/simple.c}

\EN{\subsubsection{x86}

\myparagraph{MSVC}

Let's compile:

\lstinputlisting[caption=MSVC 2008,style=customasmx86]{patterns/13_arrays/1_simple/simple_msvc.asm}

\myindex{x86!\Instructions!SHL}

Nothing very special, just two loops: the first is a filling loop and second is a printing loop.
The \TT{shl ecx, 1} instruction is used for value multiplication by 2 in \ECX, more about below~\myref{SHR}.

80 bytes are allocated on the stack for the array, 20 elements of 4 bytes.

\clearpage
Let's try this example in \olly.
\myindex{\olly}

We see how the array gets filled: 

each element is 32-bit word of \Tint type and its value is the index multiplied by 2:

\begin{figure}[H]
\centering
\myincludegraphics{patterns/13_arrays/1_simple/olly.png}
\caption{\olly: after array filling}
\label{fig:array_simple_olly}
\end{figure}

Since this array is located in the stack, we can see all its 20 elements there.

\myparagraph{GCC}

Here is what GCC 4.4.1 does:

\lstinputlisting[caption=GCC 4.4.1,style=customasmx86]{patterns/13_arrays/1_simple/simple_gcc.asm}

By the way, variable $a$ is of type  \IT{int*} 
(the pointer to \Tint{})---you can pass a pointer to an array to another function,
but it's more correct to say that a pointer to the first element of the array is passed
(the addresses of rest of the elements are calculated in an obvious way).

If you index this pointer as \IT{a[idx]}, \IT{idx} is just to be added to the pointer 
and the element placed there (to which calculated pointer is pointing) is to be returned.

An interesting example: a string of characters like 
\IT{\q{string}} is an array of characters and it has a type of \IT{const char[]}.

An index can also be applied to this pointer.

And that is why it is possible to write things like \TT{\q{string}[i]}---this is a correct \CCpp expression!

}\RU{\subsubsection{x86}

\myparagraph{MSVC}

Компилируем:

\lstinputlisting[caption=MSVC 2008,style=customasmx86]{patterns/13_arrays/1_simple/simple_msvc.asm}

\myindex{x86!\Instructions!SHL}
Ничего особенного, просто два цикла. Один изменяет массив, второй печатает его содержимое. 
Команда \INS{shl ecx, 1} используется для умножения \ECX на 2, об этом ниже~(\myref{SHR}).

Под массив выделено в стеке 80 байт, это 20 элементов по 4 байта.

\clearpage
Попробуем этот пример в \olly.
\myindex{\olly}

Видно, как заполнился массив: каждый элемент это 32-битное слово типа \Tint, с шагом 2:

\begin{figure}[H]
\centering
\myincludegraphics{patterns/13_arrays/1_simple/olly.png}
\caption{\olly: после заполнения массива}
\label{fig:array_simple_olly}
\end{figure}

А так как этот массив находится в стеке, то мы видим все его 20 элементов внутри стека.

\myparagraph{GCC}

Рассмотрим результат работы GCC 4.4.1:

\lstinputlisting[caption=GCC 4.4.1,style=customasmx86]{patterns/13_arrays/1_simple/simple_gcc.asm}

Переменная $a$ в нашем примере имеет тип \IT{int*} (указатель на \Tint{}).
Вы можете попробовать передать в другую функцию указатель на массив,
но точнее было бы сказать, что передается указатель на первый элемент массива
(а адреса остальных элементов массива можно вычислить очевидным образом).

Если индексировать этот указатель как \IT{a[idx]}, \IT{idx} просто прибавляется к указателю 
и возвращается элемент, расположенный там, куда ссылается вычисленный указатель.

Вот любопытный пример. Строка символов вроде \IT{\q{string}} это массив из символов. 
Она имеет тип \IT{const char[]}.
К этому указателю также можно применять индекс.

Поэтому можно написать даже так:  \TT{\q{string}[i]}~--- это совершенно легальное выражение в \CCpp!

}
\EN{\subsubsection{ARM}

\myparagraph{\NonOptimizingKeilVI (\ARMMode)}

\lstinputlisting[style=customasmARM]{patterns/13_arrays/1_simple/simple_Keil_ARM_O0_EN.asm}

\Tint type requires 32 bits for storage (or 4 bytes),

so to store 20 \Tint variables 80 (\TT{0x50}) bytes are needed.
So that is why the \INS{SUB SP, SP, \#0x50} 

instruction in the function's prologue allocates exactly this amount of space in the stack.

In both the first and second loops, the loop iterator \var{i} is placed in the \Reg{4} register.

\myindex{ARM!Optional operators!LSL}

The number that is to be written into the array is calculated as $i*2$, which is effectively equivalent 
to shifting it left by one bit,\\
so \INS{MOV R0, R4,LSL\#1} instruction does this.

\myindex{ARM!\Instructions!STR}
\INS{STR R0, [SP,R4,LSL\#2]} writes the contents of \Reg{0} into the array.

Here is how a pointer to array element is calculated: \ac{SP} points to the start of the array, \Reg{4} is $i$.

So shifting $i$ left by 2 bits is effectively equivalent to multiplication by 4
(since each array element has a size of 4 bytes) and then it's added to the address of the start of the array.

\myindex{ARM!\Instructions!LDR}

The second loop has an inverse \INS{LDR R2, [SP,R4,LSL\#2]}
instruction. It loads the value we need from the array, and the pointer to it is calculated likewise.

\myparagraph{\OptimizingKeilVI (\ThumbMode)}

\lstinputlisting[style=customasmARM]{patterns/13_arrays/1_simple/simple_Keil_thumb_O3_EN.asm}

Thumb code is very similar.
\myindex{ARM!\Instructions!LSLS}

Thumb mode has special instructions for bit shifting (like \TT{LSLS}),
which calculates the value to be written into the array and the address of each element in the array as well.

The compiler allocates slightly more space in the local stack, however, the last 4 bytes are not used.

\myparagraph{\NonOptimizing GCC 4.9.1 (ARM64)}

\lstinputlisting[caption=\NonOptimizing GCC 4.9.1 (ARM64),style=customasmARM]{patterns/13_arrays/1_simple/ARM64_GCC491_O0_EN.s}

}\RU{\subsubsection{ARM}

\myparagraph{\NonOptimizingKeilVI (\ARMMode)}

\lstinputlisting[style=customasmARM]{patterns/13_arrays/1_simple/simple_Keil_ARM_O0_RU.asm}

Тип \Tint требует 32 бита для хранения (или 4 байта),

так что для хранения 20 переменных типа \Tint, нужно 80 (\TT{0x50}) байт.

Поэтому инструкция \INS{SUB SP, SP, \#0x50} 
в прологе функции выделяет в локальном стеке под массив именно столько места.

И в первом и во втором цикле итератор цикла \var{i} будет постоянно находится в регистре \Reg{4}.

\myindex{ARM!Optional operators!LSL}
Число, которое нужно записать в массив, вычисляется так: $i*2$, и это эквивалентно 
сдвигу на 1 бит влево,\\
так что инструкция \INS{MOV R0, R4,LSL\#1} делает это.

\myindex{ARM!\Instructions!STR}
\INS{STR R0, [SP,R4,LSL\#2]} записывает содержимое \Reg{0} в массив.
Указатель на элемент массива вычисляется так: \ac{SP} указывает на начало массива, \Reg{4} это $i$.

Так что сдвигаем $i$ на 2 бита влево, что эквивалентно умножению на 4 
(ведь каждый элемент массива занимает 4 байта) и прибавляем это к адресу начала массива.

\myindex{ARM!\Instructions!LDR}
Во втором цикле используется обратная инструкция\\
\INS{LDR R2, [SP,R4,LSL\#2]}.
Она загружает из массива нужное значение и указатель на него вычисляется точно так же.

\myparagraph{\OptimizingKeilVI (\ThumbMode)}

\lstinputlisting[style=customasmARM]{patterns/13_arrays/1_simple/simple_Keil_thumb_O3_RU.asm}

Код для Thumb очень похожий.
\myindex{ARM!\Instructions!LSLS}
В Thumb имеются отдельные инструкции для битовых сдвигов (как \TT{LSLS}), 
вычисляющие и число для записи в массив и адрес каждого элемента массива.

Компилятор почему-то выделил в локальном стеке немного больше места, 
однако последние 4 байта не используются.

\myparagraph{\NonOptimizing GCC 4.9.1 (ARM64)}

\lstinputlisting[caption=\NonOptimizing GCC 4.9.1 (ARM64),style=customasmARM]{patterns/13_arrays/1_simple/ARM64_GCC491_O0_RU.s}

}
\EN{\subsubsection{MIPS}
% FIXME better start at non-optimizing version?

The function uses a lot of S- registers which must be preserved, so that's why its 
values are saved in the function prologue and restored in the epilogue.

\lstinputlisting[caption=\Optimizing GCC 4.4.5 (IDA),style=customasmMIPS]{patterns/13_arrays/1_simple/MIPS_O3_IDA_EN.lst}

Something interesting: there are two loops and the first one doesn't need $i$, it needs only 
$i*2$ (increased by 2 at each iteration) and also the address in memory (increased by 4 at each iteration).

So here we see two variables, one (in \$V0) increasing by 2 each time, and another (in \$V1) --- by 4.

The second loop is where \printf is called and it reports the value of $i$ to the user, 
so there is a variable
which is increased by 1 each time (in \$S0) and also a memory address (in \$S1) increased by 4 each time.

That reminds us of loop optimizations we considered earlier: \myref{loop_iterators}.

Their goal is to get rid of of multiplications.

}\RU{\subsubsection{MIPS}
% FIXME better start at non-optimizing version?
Функция использует много S-регистров, которые должны быть сохранены. Вот почему их значения сохраняются
в прологе функции и восстанавливаются в эпилоге.

\lstinputlisting[caption=\Optimizing GCC 4.4.5 (IDA),style=customasmMIPS]{patterns/13_arrays/1_simple/MIPS_O3_IDA_RU.lst}

Интересная вещь: здесь два цикла и в первом не нужна переменная $i$, а нужна только переменная
$i*2$ (скачущая через 2 на каждой итерации) и ещё адрес в памяти (скачущий через 4 на каждой итерации).

Так что мы видим здесь две переменных: одна (в \$V0) увеличивается на 2 каждый раз, и вторая (в \$V1) --- на 4.

Второй цикл содержит вызов \printf. Он должен показывать значение $i$ пользователю,
поэтому здесь есть переменная, увеличивающаяся на 1 каждый раз (в \$S0), а также адрес в памяти (в \$S1) 
увеличивающийся на 4 каждый раз.

Это напоминает нам оптимизацию циклов, которую мы рассматривали ранее: \myref{loop_iterators}.
Цель оптимизации в том, чтобы избавиться от операций умножения.

}


\subsection{\RU{Переполнение буфера}\EN{Buffer overflow}}
\label{subsec:bufferoverflow}
\myindex{\BufferOverflow}

\EN{\subsubsection{Reading outside array bounds}

So, array indexing is just \IT{array\lbrack{}index\rbrack}.
If you study the generated code closely, you'll probably note the missing index bounds checking,
which could check \IT{if it is less than 20}.
What if the index is 20 or greater?
That's the one \CCpp feature it is often blamed for.

Here is a code that successfully compiles and works:

\lstinputlisting[style=customc]{patterns/13_arrays/2_BO/r.c}

Compilation results (MSVC 2008):

\lstinputlisting[caption=\NonOptimizing MSVC 2008,style=customasmx86]{patterns/13_arrays/2_BO/r_msvc.asm}

The code produced this result:

\lstinputlisting[caption=\olly: console output]{patterns/13_arrays/2_BO/console.txt}

It is just \IT{something} that has been lying in the stack near to the array, 80 bytes away from its first element.

\clearpage
\myindex{\olly}
Let's try to find out where did this value come from, using \olly.

Let's load and find the value located right after the last array element:

\begin{figure}[H]
\centering
\myincludegraphics{patterns/13_arrays/2_BO/olly_r1.png}
\caption{\olly: reading of the 20th element and execution of \printf}
\label{fig:array_BO_olly_r1}
\end{figure}

What is this? 
Judging by the stack layout,
this is the saved value of the EBP register.
\clearpage
Let's trace further and see how it gets restored:

\begin{figure}[H]
\centering
\myincludegraphics{patterns/13_arrays/2_BO/olly_r2.png}
\caption{\olly: restoring value of EBP}
\label{fig:array_BO_olly_r2}
\end{figure}

Indeed, how it could be different?
The compiler may generate some additional code to check the index value to be always
in the array's bounds (like in higher-level programming languages\footnote{Java, Python, etc.})
but this makes the code slower.

}
\RU{\subsubsection{Чтение за пределами массива}

Итак, индексация массива --- это просто \IT{массив\lbrack{}индекс\rbrack}.  % TODO1 как-то плохо отображаются []
Если вы присмотритесь к коду, в цикле печати значений массива через \printf вы 
не увидите проверок индекса, \IT{меньше ли он двадцати?} 
А что будет если он будет 20 или больше? 
Эта одна из особенностей \CCpp, за которую их, собственно, и ругают.

Вот код, который и компилируется и работает:

\lstinputlisting[style=customc]{patterns/13_arrays/2_BO/r.c}

Вот результат компиляции в (MSVC 2008):

\lstinputlisting[caption=\NonOptimizing MSVC 2008,style=customasmx86]{patterns/13_arrays/2_BO/r_msvc.asm}

Данный код при запуске выдал вот такой результат:

\lstinputlisting[caption=\olly: вывод в консоль]{patterns/13_arrays/2_BO/console.txt}

Это просто \IT{что-то}, что волею случая лежало в стеке рядом с массивом, 
через 80 байт от его первого элемента.

\clearpage
\myindex{\olly}
Попробуем узнать в \olly, что это за значение.
Загружаем и находим это значение, находящееся точно после последнего элемента массива:

\begin{figure}[H]
\centering
\myincludegraphics{patterns/13_arrays/2_BO/olly_r1.png}
\caption{\olly: чтение 20-го элемента и вызов \printf}
\label{fig:array_BO_olly_r1}
\end{figure}

Что это за значение? 
Судя по разметке стека, это сохраненное значение регистра EBP.
\clearpage
Трассируем далее, и видим, как оно восстанавливается:

\begin{figure}[H]
\centering
\myincludegraphics{patterns/13_arrays/2_BO/olly_r2.png}
\caption{\olly: восстановление EBP}
\label{fig:array_BO_olly_r2}
\end{figure}

Действительно, а как могло бы быть иначе? Компилятор мог бы встроить какой-то код, 
каждый раз проверяющий индекс на соответствие пределам массива, как в языках программирования 
более высокого уровня\footnote{Java, Python, итд.}, что делало бы запускаемый код медленнее.

}
\DE{\subsubsection{Lesezugriff außerhalb von Arraygrenzen}
Der indizierte Zugriff auf ein Array wird durch \IT{array\lbrack{}index\rbrack} realisiert.
Wenn man sich den erzeugten Code genau ansieht, bemerkt man, dass eine Prüfung der Indexgrenzen fehlt, welche die
Bedingung \IT{kleiner als 20} validiert.
Was also passiert, wenn der Index 20 oder größer ist? 
Hier haben wir es mit einem unschönen Feature von \CCpp zu tun

Hier ein Beipsielcode der erfolgreich kompiliert wurde und funktioniert:

\lstinputlisting[style=customc]{patterns/13_arrays/2_BO/r.c}

Ergebnis des Kompiliervorgangs (MSVC 2008):

\lstinputlisting[caption=\NonOptimizing MSVC 2008,style=customasmx86]{patterns/13_arrays/2_BO/r_msvc.asm}

Der Code produziert dieses Ergebnis:

\lstinputlisting[caption=\olly: console output]{patterns/13_arrays/2_BO/console.txt}
Es handelt sich um \IT{irgendetwas}, das auf dem Stack in der Nähe des Arrays gelegen hat, 80 Byte von dessen erstem
Element entfernt.

\clearpage
\myindex{\olly}
Versuchen wir mit \olly herauszufinden, woher dieser Wert kommt.

Laden und finden wir also den Wert, der sich direkt hinter dem letzten Arrayelement befindet:

\begin{figure}[H]
\centering
\myincludegraphics{patterns/13_arrays/2_BO/olly_r1.png}
\caption{\olly: das 20. Element lesen und \printf ausführen}
\label{fig:array_BO_olly_r1}
\end{figure}

Worum handelt es sich? 
Dem Stacklayout nach zu urteilen ist dies der gespeicherte Wert des EBP Registers.
\clearpage
Verfolgen wir das ganze weiter und schauen uns an, wie dieser wiederhergestellt wird:

\begin{figure}[H]
\centering
\myincludegraphics{patterns/13_arrays/2_BO/olly_r2.png}
\caption{\olly: Wert von EBP wiederherstellen}
\label{fig:array_BO_olly_r2}
\end{figure}
Wie könnte es anders gelöst werden?
Der Compiler könnte zusätzlichen Code erzeugen, der sicherstellt, dass der Index sich stets innerhalb der Arraygrenzen
befindet (wie in höheren Programmiersprachen\footnote{Java, Python, etc.}), aber das würde den Code langsamer machen.
}

\EN{\subsubsection{Writing beyond array bounds}

OK, we read some values from the stack \IT{illegally}, but what if we could write something to it?

Here is what we have got:

\lstinputlisting[style=customc]{patterns/13_arrays/2_BO/w.c}

\myparagraph{MSVC}

And what we get:

\lstinputlisting[caption=\NonOptimizing MSVC 2008,style=customasmx86]{patterns/13_arrays/2_BO/w_EN.asm}

The compiled program crashes after running. No wonder. Let's see where exactly does it is crash.

\clearpage
\myindex{\olly}

Let's load it into \olly, and trace until all 30 elements are written:

\begin{figure}[H]
\centering
\myincludegraphics{patterns/13_arrays/2_BO/olly_w1.png}
\caption{\olly: after restoring the value of EBP}
\label{fig:array_BO_olly_w1}
\end{figure}

\clearpage
Trace until the function end:

\begin{figure}[H]
\centering
\myincludegraphics{patterns/13_arrays/2_BO/olly_w2.png}
\caption{\olly: 
\TT{EIP} has been restored, but \olly can't disassemble at 0x15}
\label{fig:array_BO_olly_w2}
\end{figure}

Now please keep your eyes on the registers.

\EIP is 0x15 now. It is not a legal address for code---at least for win32 code!
We got there somehow against our will.
It is also interesting that the \EBP register contain 0x14,
\ECX and \EDX contain 0x1D.

Let's study stack layout a bit more.

After the control flow has been passed to \TT{\main}, the value in the \EBP register was saved on the stack.
Then, 84 bytes were allocated for the array and the $i$ variable.
That's \TT{(20+1)*sizeof(int)}.
\ESP now points to the \TT{\_i} variable in the local stack and after the execution of 
the next \TT{PUSH something}, \IT{something} is appearing next to \TT{\_i}.

That's the stack layout while the control is in \main:

\begin{center}
\begin{tabular}{ | l | l | }
\hline
  \TT{ESP}    & 4 bytes allocated for $i$ variable \\
\hline
  \TT{ESP+4}  & 80 bytes allocated for \TT{a[20]} array \\
\hline
  \TT{ESP+84} & saved \EBP value \\
\hline
  \TT{ESP+88} & return address \\
\hline
\end{tabular}
\end{center}

\TT{a[19]=something} statement writes the last \Tint in the bounds of the array (in bounds so far!).

\TT{a[20]=something} statement writes \IT{something} to the place where the value of \EBP is saved.

Please take a look at the register state at the moment of the crash. In our case,
20 has been written in the 20th element. 
At the function end, the function epilogue restores the original \EBP value.
(20 in decimal is \TT{0x14} in hexadecimal).
Then \RET gets executed, which is effectively equivalent to \TT{POP EIP} instruction.

The \RET instruction takes the return address from the stack (that is the address in \ac{CRT}),
which has called \main),
and 21 is stored there (\TT{0x15} in hexadecimal).
The CPU traps at address \TT{0x15},
but there is no executable code there, so exception gets raised.

\myindex{\BufferOverflow}

Welcome! It is called a \IT{buffer overflow}\footnote{\href{http://go.yurichev.com/17132}{wikipedia}}.

Replace the \Tint array with a string (\Tchar array), create a long string deliberately
and pass it to the program, to the function, which doesn't check the length of the string and copies it in a short buffer,
and you'll able to point the program to an address to which it must jump.
It's not that simple in reality, but that is how it emerged.
Classic article about it: \AlephOne.

\myparagraph{GCC}

Let's try the same code in GCC 4.4.1. We get:

\lstinputlisting[style=customasmx86]{patterns/13_arrays/2_BO/w_gcc.asm}

Running this in Linux will produce: \TT{Segmentation fault}.

\myindex{GDB}

If we run this in the GDB debugger, we get this:

\begin{lstlisting}
(gdb) r
Starting program: /home/dennis/RE/1 

Program received signal SIGSEGV, Segmentation fault.
0x00000016 in ?? ()
(gdb) info registers
eax            0x0	0
ecx            0xd2f96388	-755407992
edx            0x1d	29
ebx            0x26eff4	2551796
esp            0xbffff4b0	0xbffff4b0
ebp            0x15	0x15
esi            0x0	0
edi            0x0	0
eip            0x16	0x16
eflags         0x10202	[ IF RF ]
cs             0x73	115
ss             0x7b	123
ds             0x7b	123
es             0x7b	123
fs             0x0	0
gs             0x33	51
(gdb) 
\end{lstlisting}

The register values are slightly different than in win32 example, 
since the stack layout is slightly different too.

}
\RU{\subsubsection{Запись за пределы массива}

Итак, мы прочитали какое-то число из стека явно \IT{нелегально}, а что если мы запишем?

Вот что мы пишем:

\lstinputlisting[style=customc]{patterns/13_arrays/2_BO/w.c}

\myparagraph{MSVC}

И вот что имеем на ассемблере:

\lstinputlisting[caption=\NonOptimizing MSVC 2008,style=customasmx86]{patterns/13_arrays/2_BO/w_RU.asm}

Запускаете скомпилированную программу, и она падает. Немудрено. Но давайте теперь узнаем, где именно.

\clearpage
\myindex{\olly}

Загружаем в \olly, трассируем пока запишутся все 30 элементов:

\begin{figure}[H]
\centering
\myincludegraphics{patterns/13_arrays/2_BO/olly_w1.png}
\caption{\olly: после восстановления EBP}
\label{fig:array_BO_olly_w1}
\end{figure}

\clearpage
Доходим до конца функции:

\begin{figure}[H]
\centering
\myincludegraphics{patterns/13_arrays/2_BO/olly_w2.png}
\caption{\olly: EIP восстановлен, но \olly не может дизассемблировать по адресу 0x15}
\label{fig:array_BO_olly_w2}
\end{figure}

Итак, следите внимательно за регистрами.

\EIP теперь 0x15. Это явно нелегальный адрес для кода~--- по крайней мере, win32-кода! 
Мы там как-то очутились, причем, сами того не хотели. Интересен также тот факт, что в \EBP хранится 0x14, 
а в \ECX и \EDX хранится 0x1D.

Ещё немного изучим разметку стека.

После того как управление передалось в \main, в стек было сохранено значение \EBP. 
Затем для массива и переменной $i$ было выделено 84 байта. Это \TT{(20+1)*sizeof(int)}. 
\ESP сейчас указывает на переменную \TT{\_i} в локальном стеке и при исполнении следующего \INS{PUSH что-либо}, 
\IT{что-либо} появится рядом с \TT{\_i}.

Вот так выглядит разметка стека пока управление находится внутри \main:

\begin{center}
\begin{tabular}{ | l | l | }
\hline
  \TT{ESP}    & 4 байта выделенных для переменной $i$ \\
\hline
  \TT{ESP+4}  & 80 байт выделенных для массива \TT{a[20]} \\
\hline
  \TT{ESP+84} & сохраненное значение \EBP \\
\hline
  \TT{ESP+88} & адрес возврата \\
\hline
\end{tabular}
\end{center}

Выражение \TT{a[19]=что\_нибудь} записывает последний \Tint в пределах массива (пока что в пределах!).

Выражение \TT{a[20]=что\_нибудь} записывает \IT{что\_нибудь} на место где сохранено значение \EBP.

Обратите внимание на состояние регистров на момент падения процесса. В нашем случае 
в 20-й элемент записалось значение 20. 
И вот всё дело в том, что заканчиваясь, эпилог функции восстанавливал значение \EBP 
(20 в десятичной системе это как раз \TT{0x14} в шестнадцатеричной). 
Далее выполнилась инструкция \RET, которая на самом деле эквивалентна \TT{POP EIP}.

Инструкция \RET вытащила из стека адрес возврата (это адрес где-то внутри \ac{CRT}), которая вызвала \main),
а там было записано 21 в десятичной системе, то есть 0x15 в шестнадцатеричной. 
И вот процессор оказался по адресу 0x15, но исполняемого кода там нет, так что случилось исключение.

\myindex{\BufferOverflow}
Добро пожаловать! Это называется \IT{buffer overflow}\footnote{\href{http://go.yurichev.com/17132}{wikipedia}}.

Замените массив \Tint на строку (массив \Tchar), нарочно создайте слишком длинную строку, 
передайте её в ту программу, 
в ту функцию, которая не проверяя длину строки скопирует её в слишком короткий буфер, 
и вы сможете указать программе, по какому именно адресу перейти. 
Не всё так просто в реальности, конечно, но началось всё с этого.
Классическая статья об этом: \AlephOne.

\myparagraph{GCC}

Попробуем то же самое в GCC 4.4.1. У нас выходит такое:

\lstinputlisting[style=customasmx86]{patterns/13_arrays/2_BO/w_gcc.asm}

Запуск этого в Linux выдаст: \TT{Segmentation fault}.

\myindex{GDB}
Если запустить полученное в отладчике GDB, получим:

\begin{lstlisting}
(gdb) r
Starting program: /home/dennis/RE/1 

Program received signal SIGSEGV, Segmentation fault.
0x00000016 in ?? ()
(gdb) info registers
eax            0x0	0
ecx            0xd2f96388	-755407992
edx            0x1d	29
ebx            0x26eff4	2551796
esp            0xbffff4b0	0xbffff4b0
ebp            0x15	0x15
esi            0x0	0
edi            0x0	0
eip            0x16	0x16
eflags         0x10202	[ IF RF ]
cs             0x73	115
ss             0x7b	123
ds             0x7b	123
es             0x7b	123
fs             0x0	0
gs             0x33	51
(gdb) 
\end{lstlisting}

Значения регистров немного другие, чем в примере win32, потому что разметка стека чуть другая.

}
\DE{\subsubsection{Schreibzugriff außerhalb von Arraygrenzen}
Nehmen wir an, wir hätte ein paar Werte illegalerweise vom Stack gelesen, wie könnten wir etwas hineinschreiben?

Hier ist, was wir haben:

\lstinputlisting[style=customc]{patterns/13_arrays/2_BO/w.c}

\myparagraph{MSVC}

Wir erhalten das Folgende:

\lstinputlisting[caption=\NonOptimizing MSVC 2008,style=customasmx86]{patterns/13_arrays/2_BO/w_DE.asm}
Das kompilierte Programm stürzt nach der Ausführung ab. Das verwundert nicht. Schauen wir, was genau den Absturz
verursacht.

\clearpage
\myindex{\olly}
Laden wir das Programm in \olly und verfolgen den Ablauf, bis alle 30 Elemente geschrieben worden sind:

\begin{figure}[H]
\centering
\myincludegraphics{patterns/13_arrays/2_BO/olly_w1.png}
\caption{\olly: nach Wiederherstellung des Wertes von EBP}
\label{fig:array_BO_olly_w1}
\end{figure}

\clearpage
Nachverfolgen bis zum Ende der Funktion:

\begin{figure}[H]
\centering
\myincludegraphics{patterns/13_arrays/2_BO/olly_w2.png}
\caption{\olly: 
\TT{EIP} wurde wiederhergestellt, aber \olly kann an 0x15 nicht disassemblieren}
\label{fig:array_BO_olly_w2}
\end{figure}
Richten wir unser Augenmerk auf die Register.

\EIP ist jetzt gerade 0x15. Das ist keine gültige Adreses für Code---zumindest nicht für win32 Code!
Interessant ist auch, dass das \EBP Register 0x14 enthält und \ECX sowie \EDX jeweils 0x1D

Schauen wir uns das Stacklayout etwas genauer an.

Nachdem der Control Flow an \TT{\main} übergeben worde ist, wurde der Wert in \EBP auf dem Stack abgelegt.
Danach wurden 84 Byte für das Array und die Variable $i$ reserviert.
Das entspricht \TT{(20+1)*sizeof(int)}.
\ESP zeigt jetzt auf die Variable \TT{\_i} im lokalen Stack und nach der Ausführung von \TT{PUSH something} scheint sich
\TT{something} neben \TT{\_i} zu befinden.

Hier ist das Stacklayout während der Control Flow in der \main ist:

\begin{center}
\begin{tabular}{ | l | l | }
\hline
  \TT{ESP}    & 4 Byte reserviert für Variable $i$ \\
\hline
  \TT{ESP+4}  & 80 Byte reserviert für Array \TT{a[20]} \\
\hline
  \TT{ESP+84} & sichere Wert von \EBP \\
\hline
  \TT{ESP+88} & Rücksprungadresse \\
\hline
\end{tabular}
\end{center}
Der Befehl \TT{a[19]=something} schreibt den letzten \Tint innerhalb der Grenzen des Arrays (bis hierhin ist alles in
Ordnung!).
Der Befehl \TT{a[20]=something} schreibt \IT{something} an die Stelle, an der der \EBP gespeichert ist.

Sehen wir uns den Zustand der Register im Moment des Absturzes an. In unserem Fall wurde 20 in das zwanzigste Element
geschrieben. Am Ende der Funktion stellt der Funktionsepilog den originalen Wert von \EBP wieder her.
(20 dezimal entspricht \TT{0x14} hexadezimal).
Danach wird \RET ausgeführt, was äquivalent zum Befehl \TT{POP EIP} ist.

Der Befehl \RET nimmt die Rücksprungadresse vom Stack (das ist die Adresse in \ac{CRT}, die \main aufgerufen hat) und
speichert hier den Wert 21 (\TT{0x15} hexadezimal).
Die CPU springt an die Adresse \TT{0x15}, aber hier befindet sich kein ausführbarer Code, sodass eine Exception geworfen
wird.

\myindex{\BufferOverflow}
Dies nennt man einen \IT{Buffer Overflow}\footnote{\href{http://go.yurichev.com/17132}{wikipedia}}.

Ersetzt man das \Tint Array durch einen String (\Tchar Array) und erzeugt absichtlich einen langen String und übergibt
ihn im Programm an eine Funktion, die die Länge des Strings nicht prüft und ihn in einen kurzen Buffer kopiert, kann man
das Programm zwingen an eine bestimmte Adresse zu springen.
In der Realität ist dieses Verhalten nicht so einfach zu erzeugen, funktioniert aber von Prinzip her genau wie hier.
Ein klassischer Artikel dazu:\AlephOne.

\myparagraph{GCC}

Kompilieren wir denselben Code mit GCC 4.4.1, erhalten wir:

\lstinputlisting[style=customasmx86]{patterns/13_arrays/2_BO/w_gcc.asm}

Lässt man das Programm unter Linux laufen, lautet das Ergebnis: \TT{Segmentation fault}.

\myindex{GDB}
Wenn wir es mit dem GDB Debugger laufen lassen, erhalten wir das Folgende:


\begin{lstlisting}
(gdb) r
Starting program: /home/dennis/RE/1 

Program received signal SIGSEGV, Segmentation fault.
0x00000016 in ?? ()
(gdb) info registers
eax            0x0	0
ecx            0xd2f96388	-755407992
edx            0x1d	29
ebx            0x26eff4	2551796
esp            0xbffff4b0	0xbffff4b0
ebp            0x15	0x15
esi            0x0	0
edi            0x0	0
eip            0x16	0x16
eflags         0x10202	[ IF RF ]
cs             0x73	115
ss             0x7b	123
ds             0x7b	123
es             0x7b	123
fs             0x0	0
gs             0x33	51
(gdb) 
\end{lstlisting}
Die Registerwerte unterscheiden sich geringfügig vom win32 Beispiel, da auch das Stacklayout ein wenig anders ist.
}

\section{\RU{Защита от переполнения буфера}\EN{Buffer overflow protection methods}}
\label{subsec:BO_protection}

\RU{В наше время пытаются бороться с переполнением буфера невзирая на халатность программистов на \CCpp. 
В MSVC есть опции вроде}%
\EN{There are several methods to protect against this scourge, regardless of the \CCpp programmers' negligence.
MSVC has options like}\footnote{
\RU{описания защит, которые компилятор может вставлять в код}%
\EN{compiler-side buffer overflow protection methods}:
\href{http://go.yurichev.com/17133}{wikipedia.org/wiki/Buffer\_overflow\_protection}}:

\begin{lstlisting}
 /RTCs Stack Frame runtime checking
 /GZ Enable stack checks (/RTCs)
\end{lstlisting}

\index{x86!\Instructions!RET}
\index{Function prologue}
\index{Security cookie}
\RU{Одним из методов является вставка в прологе функции некоего случайного значения в область локальных переменных 
и проверка этого значения в эпилоге функции перед выходом. 
Если проверка не прошла, то не выполнять инструкцию \RET, а остановиться (или зависнуть). 
Процесс зависнет, но это лучше, чем удаленная атака на ваш компьютер.}
\EN{One of the methods is to write a random value between the local variables in stack at function prologue 
and to check it in function epilogue before the function exits.
If value is not the same, do not execute the last instruction \RET, but stop (or hang).
The process will halt, but that is much better than a remote attack to your host.}
    
\newcommand{\CANARYURL}{\RU{\href{http://go.yurichev.com/17135}{miningwiki.ru/wiki/Канарейка\_в\_шахте}}%
\EN{\href{http://go.yurichev.com/17134}{wikipedia.org/wiki/Domestic\_canary\#Miner.27s\_canary}}}

\index{Canary}
\RU{Это случайное значение иногда называют \q{канарейкой}%
\footnote{\q{canary} в англоязычной литературе}, 
по аналогии с шахтной канарейкой\footnote{\CANARYURL}.
Раньше использовали шахтеры, чтобы определять, есть ли в шахте опасный газ.
}
\EN{This random value is called a \q{canary} sometimes, it is related to the miners' canary\footnote{\CANARYURL},
they were used by miners in the past days in order to detect poisonous gases quickly.}
\RU{Канарейки очень к нему чувствительны и либо проявляли сильное беспокойство, либо гибли от газа.}
\EN{Canaries are very sensitive to mine gases, they become very agitated in case of danger, or even die.}

\RU{Если скомпилировать наш простейший пример работы с массивом}
\EN{If we compile our very simple array example}~(\myref{arrays_simple}) \InENRU \ac{MSVC}
\RU{с опцией RTC1 или RTCs}\EN{with RTC1 and RTCs option}, \RU{в конце нашей функции будет вызов 
функции}\EN{you can see a call to}
\TT{@\_RTC\_CheckStackVars@8}\RU{, проверяющей корректность \q{канарейки}.}
\EN{ a function at the end of the function that checks if the \q{canary} is correct.}

\RU{Посмотрим, как дела обстоят в GCC}\EN{Let's see how GCC handles this}. 
\RU{Возьмем пример из секции про}\EN{Let's take an} \TT{alloca()}~(\myref{alloca})\EN{ example}:

\lstinputlisting{patterns/02_stack/04_alloca/2_1.c}

\RU{По умолчанию, без дополнительных ключей, GCC 4.7.3 вставит в код проверку \q{канарейки}:}
\EN{By default, without any additional options, GCC 4.7.3 inserts a \q{canary} check into the code:}

\lstinputlisting[caption=GCC 4.7.3]{patterns/13_arrays/3_BO_protection/gcc_canary.asm.\LANG}

\index{x86!\Registers!GS}
\RU{Случайное значение находится в}\EN{The random value is located in} \TT{gs:20}. 
\RU{Оно записывается в стек, затем, в конце функции, значение в стеке
сравнивается с корректной \q{канарейкой} в}\EN{It gets written on the stack and then at the end of the function
the value in the stack is compared with the correct \q{canary} in} \TT{gs:20}. 
\RU{Если значения не равны, будет вызвана функция}\EN{If the values are not equal, the} 
\TT{\_\_stack\_chk\_fail} \RU{и в консоли мы увидим что-то вроде такого}
\EN{function is called and we can see in the console something like that} (Ubuntu 13.04 x86):

\begin{lstlisting}
*** buffer overflow detected ***: ./2_1 terminated
======= Backtrace: =========
/lib/i386-linux-gnu/libc.so.6(__fortify_fail+0x63)[0xb7699bc3]
/lib/i386-linux-gnu/libc.so.6(+0x10593a)[0xb769893a]
/lib/i386-linux-gnu/libc.so.6(+0x105008)[0xb7698008]
/lib/i386-linux-gnu/libc.so.6(_IO_default_xsputn+0x8c)[0xb7606e5c]
/lib/i386-linux-gnu/libc.so.6(_IO_vfprintf+0x165)[0xb75d7a45]
/lib/i386-linux-gnu/libc.so.6(__vsprintf_chk+0xc9)[0xb76980d9]
/lib/i386-linux-gnu/libc.so.6(__sprintf_chk+0x2f)[0xb7697fef]
./2_1[0x8048404]
/lib/i386-linux-gnu/libc.so.6(__libc_start_main+0xf5)[0xb75ac935]
======= Memory map: ========
08048000-08049000 r-xp 00000000 08:01 2097586    /home/dennis/2_1
08049000-0804a000 r--p 00000000 08:01 2097586    /home/dennis/2_1
0804a000-0804b000 rw-p 00001000 08:01 2097586    /home/dennis/2_1
094d1000-094f2000 rw-p 00000000 00:00 0          [heap]
b7560000-b757b000 r-xp 00000000 08:01 1048602    /lib/i386-linux-gnu/libgcc_s.so.1
b757b000-b757c000 r--p 0001a000 08:01 1048602    /lib/i386-linux-gnu/libgcc_s.so.1
b757c000-b757d000 rw-p 0001b000 08:01 1048602    /lib/i386-linux-gnu/libgcc_s.so.1
b7592000-b7593000 rw-p 00000000 00:00 0
b7593000-b7740000 r-xp 00000000 08:01 1050781    /lib/i386-linux-gnu/libc-2.17.so
b7740000-b7742000 r--p 001ad000 08:01 1050781    /lib/i386-linux-gnu/libc-2.17.so
b7742000-b7743000 rw-p 001af000 08:01 1050781    /lib/i386-linux-gnu/libc-2.17.so
b7743000-b7746000 rw-p 00000000 00:00 0
b775a000-b775d000 rw-p 00000000 00:00 0
b775d000-b775e000 r-xp 00000000 00:00 0          [vdso]
b775e000-b777e000 r-xp 00000000 08:01 1050794    /lib/i386-linux-gnu/ld-2.17.so
b777e000-b777f000 r--p 0001f000 08:01 1050794    /lib/i386-linux-gnu/ld-2.17.so
b777f000-b7780000 rw-p 00020000 08:01 1050794    /lib/i386-linux-gnu/ld-2.17.so
bff35000-bff56000 rw-p 00000000 00:00 0          [stack]
Aborted (core dumped)
\end{lstlisting}

\index{MS-DOS}
gs \RU{это так называемый сегментный регистр. Эти регистры широко использовались во времена MS-DOS 
и DOS-экстендеров.}\EN{is the so-called segment register. These registers were used widely in MS-DOS and DOS-extenders
times.}
\RU{Сейчас их функция немного изменилась.}\EN{Today, its function is different.}
\index{TLS}
\index{Windows!TIB}
\RU{Если говорить кратко, в Linux \TT{gs} всегда указывает на \ac{TLS}~(\myref{TLS})~--- там находится различная 
информация, специфичная для выполняющегося потока.}
\EN{To say it briefly, the \TT{gs} register in Linux always points to the
\ac{TLS}~(\myref{TLS})---some information specific to thread is stored there.}
\RU{Кстати, в win32 эту же роль играет сегментный регистр \TT{fs},
он всегда указывает на}\EN{By the way, in win32
the \TT{fs} register plays the same role, pointing to}
\ac{TIB} \footnote{\href{http://go.yurichev.com/17104}{wikipedia.org/wiki/Win32\_Thread\_Information\_Block}}. 

\RU{Больше информации можно почерпнуть из исходных кодов Linux (по крайней мере, в версии 3.11): 
в файле}\EN{More information can be found in the Linux kernel source code (at least in 3.11 version), in}
\IT{arch/x86/include/asm/stackprotector.h}\RU{ в комментариях описывается эта переменная.}
\EN{ this variable is described in the comments.}

\ifdefined\IncludeARM
\subsection{\OptimizingXcodeIV (\ThumbTwoMode)}

\RU{Возвращаясь к нашему простому примеру}
\EN{Let's get back to our simple array example} (\myref{arrays_simple}),
\RU{можно посмотреть, как LLVM добавит проверку \q{канарейки}:}
\EN{again, now we can see how LLVM checks the correctness of the \q{canary}:}

% TODO shorten the listing a bit? is full display of unrolled loop necessary?
\lstinputlisting{patterns/13_arrays/3_BO_protection/simple_Xcode_thumb_O3.asm.\LANG}

\index{Unrolled loop}
\RU{Во-первых, LLVM \q{развернул} цикл и все значения записываются в массив по одному, 
уже вычисленные, 
потому что LLVM посчитал что так будет быстрее.}
\EN{First of all, as we see, LLVM \q{unrolled} the loop and all values were written into an array one-by-one,
pre-calculated, as LLVM concluded it can work faster.}
\RU{Кстати, инструкции режима ARM позволяют сделать это ещё быстрее и это может быть вашим 
домашним заданием.}\EN{By the way, instructions in ARM mode may help to do this even faster, 
and finding this could be your homework.}

\RU{В конце функции мы видим сравнение \q{канареек}~--- той что лежит в локальном стеке и корректной, 
на которую ссылается регистр \Reg{8}.}
\EN{At the function end we see the comparison of the \q{canaries}---the one in the local stack and the correct one,
to which \Reg{8} points.}
\index{ARM!\Instructions!IT}
\RU{Если они равны, срабатывает блок из четырех инструкций при помощи \INS{ITTTT EQ}.
Это запись 0 в \Reg{0}, эпилог функции и выход из нее.}
\EN{If they are equal to each other, a 4-instruction block is triggered by \INS{ITTTT EQ},
which contains writing 0 in \Reg{0}, the function epilogue and exit.}
\RU{Если \q{канарейки} не равны, блок не срабатывает и происходит
переход на функцию}\EN{If the \q{canaries} are not equal, the block being skipped,
and the jump to} \TT{\_\_\_stack\_chk\_fail}\RU{, которая, вероятно, остановит работу программы.}
\EN{ function will occur, which, perhaps, will halt execution.}
% TODO1 illustrate this!

\fi

\section{\RU{Еще немного о массивах}\EN{One more word about arrays}}

\RU{Теперь понятно, почему нельзя написать в исходном коде на \CCpp что-то вроде
\footnote{Впрочем, по стандарту C99 это возможно\cite[6.7.5/2]{C99TC3}: 
GCC может это сделать выделяя место под массив динамически в стеке (как alloca()~(\ref{alloca}))}}
\EN{Now we understand, why it is impossible to write something like that in \CCpp code
\footnote{However, it is possible in C99 standard\cite[6.7.5/2]{C99TC3}: 
GCC is actually do this by allocating array dynamically on the stack (like alloca()~(\ref{alloca}))}}:

\begin{lstlisting}
void f(int size)
{
    int a[size];
...
};
\end{lstlisting}

\RU{Все просто потому, чтобы выделять место под массив в локальном стеке, 
компилятору нужно знать его размер, чего он, на стадии компиляции, 
разумеется, знать не может.}
\EN{That's just because compiler must know exact array size to allocate space for 
it in local stack layout on compiling stage.}

\index{\CLanguageElements!C99!variable length arrays}
\index{\CStandardLibrary!alloca()}
\RU{Если вам нужен массив произвольной длины, то выделите столько, сколько нужно, через \TT{malloc()}, 
затем обращайтесь к выделенному блоку байт как к массиву того типа, который вам нужен.
Либо используйте возможность стандарта C99\cite[6.7.5/2]{C99TC3}, 
но внутри это очень похоже на alloca()~(\ref{alloca})}
\EN{If you need array of arbitrary size, allocate it by \TT{malloc()}, then access allocated memory block
as array of variables of type you need.
Or use C99 standard feature\cite[6.7.5/2]{C99TC3}, 
but it looks like alloca()~(\ref{alloca}) internally.}

\section{\RU{Многомерные массивы}\EN{Multidimensional arrays}}

\RU{Внутри, многомерный массив выглядит так же, как и линейный.}
\EN{Internally, a multidimensional array is essentially the same thing as a linear array.}

\RU{Ведь память компьютера линейная, это одномерный массив.
Но для удобства, этот одномерный массив легко представить как многомерный.}
\EN{Since the computer memory is linear, it is an one-dimensional array.
For convenience, this multi-dimensional array can be easily represented as one-dimensional.}

\RU{К примеру, вот как элементы массива $a[3][4]$ расположены в одномерном массиве из 12-и ячеек:}
\EN{For example, thit is how the elements of the $a[3][4]$ array are placed in one-dimensional array of 12 cells:}

\begin{table}[H]
\centering
\begin{tabular}{ | l | }
\hline
[0][0] \\
\hline
[0][1] \\
\hline
[0][2] \\
\hline
[0][3] \\
\hline
[1][0] \\
\hline
[1][1] \\
\hline
[1][2] \\
\hline
[1][3] \\
\hline
[2][0] \\
\hline
[2][1] \\
\hline
[2][2] \\
\hline
[2][3] \\
\hline
\end{tabular}
\caption{\RU{Двухмерный массив представляется в памяти как одномерный}
\EN{Two-dimensional array represented in memory as one-dimensional}}
\end{table}

\RU{Вот по каким адресам в памяти располагается каждая ячейка двухмерного массива 3*4:}
\EN{Here is how each cell of 3*4 array are placed in memory:}

\begin{table}[H]
\centering
\begin{tabular}{ | l | l | l | l | }
\hline                        
0 & 1 & 2 & 3 \\
\hline  
4 & 5 & 6 & 7 \\
\hline  
8 & 9 & 10 & 11 \\
\hline  
\end{tabular}
\caption{\RU{Адреса в памяти каждой ячейки двухмерного массива}
\EN{Memory addresses of each cell of two-dimensional array}}
\end{table}

\index{row-major order}
\RU{То есть, чтобы вычислить адрес нужного элемента, в начале умножаем первый индекс на 4 (ширину матрицы), 
затем прибавляем второй индекс.}
\EN{So, in order to calculate the address of the element we need, we first multiply the first index by
4 (matrix width) and then add the second index.}
\RU{Это называется}\EN{That's called} \IT{row-major order}, 
\RU{и такой способ представления массивов и матриц используется по крайней мере в}
\EN{and this method of array and matrix representation is used in at least} \CCpp \AndENRU Python. 
\EN{The term}\RU{Термин} \IT{row-major order} \RU{означает по-русски
примерно следующее: ``в начале записываем элементы первой строки, затем второй \dots и элементы последней 
строки в самом конце''.}
\EN{in plain English language means: ``first, write the elements of the first row, then the second row \dots 
and finally the elements of the last row''.}

\index{column-major order}
\index{FORTRAN}
\RU{Другой способ представления называется}\EN{Another method for representation is called} 
\IT{column-major order} 
(\RU{индексы массива используются в обратном порядке}\EN{the array indices are used in reverse order}) 
\RU{и это используется по крайней мере в}\EN{and it is used at least in} FORTRAN, MATLAB \AndENRU R. 
\RU{Термин }\IT{column-major order} \RU{означает по-русски
следующее: ``в начале записываем элементы первого столбца, затем второго \dots и элементы последнего столбца
в самом конце''.}
\EN{term in plain English language means: ``first, write the elements of the first column, then the second column \dots
and finally the elements of the last column''.}

\RU{Какой из способов лучше}\EN{Which method is better}?
\RU{Вообще, в терминах производительности и кэш-памяти, лучший метод организации данных это тот,
при котором к данным обращаются последовательно.}
\EN{In general, in terms of performance and cache memory, 
the best scheme for data organization is the one,
in which the elements are accessed sequentially.}
\RU{Так что если ваша функция обращается к данным построчно, то \IT{row-major order} лучше,
и наоборот.}
\EN{So if your function accesses data per row, \IT{row-major order} is better, and vice versa.}

% subsections
\subsection{\RU{Пример с двумерным массивов}\EN{Two-dimensional array example}}

\EN{We are going to work with an array of type \Tchar, which implies that each element requires only one 
byte in memory.}
\RU{Мы будем работать с массивом типа \Tchar. Это значит, что каждый элемент требует
только одного байта в памяти.}

\subsubsection{\RU{Пример с заполнением строки}\EN{Row filling example}}
\index{\olly}

\RU{Заполняем вторую строку значениями}\EN{Let's fill the second row with these values} 0..3:

\lstinputlisting[caption=\RU{Пример с заполнением строки}\EN{Row filling example}]{patterns/13_arrays/5_multidimensional/two1.c.\LANG}

\RU{Все три строки обведены красным}\EN{All three rows are marked with red}. 
\RU{Видно, что во второй теперь имеются байты}\EN{We see that second row now has values} 0, 1, 2 \AndENRU 3:

\begin{figure}[H]
\centering
\includegraphics[scale=\NormalScale]{patterns/13_arrays/5_multidimensional/olly_2D_1.png}
\caption{\olly: \RU{массив заполнен}\EN{array is filled}}
\end{figure}

\subsubsection{\RU{Пример с заполнением столбца}\EN{Column filling example}}
\index{\olly}

\RU{Заполняем третий столбец значениями}\EN{Let's fill the third column with values:} 0..2:

\lstinputlisting[caption=\RU{Пример с заполнением столбца}\EN{Column filling example}]{patterns/13_arrays/5_multidimensional/two2.c.\LANG}

\RU{Здесь также обведены красным три строки}\EN{The three rows are also marked in red here}. 
\RU{Видно, что в каждой строке, на третьей позиции, теперь записаны}
\EN{We see that in each row, at third position these values are written:} 0, 1 \AndENRU 2.

\begin{figure}[H]
\centering
\includegraphics[scale=\NormalScale]{patterns/13_arrays/5_multidimensional/olly_2D_2.png}
\caption{\olly: \RU{массив заполнен}\EN{array is filled}}
\end{figure}

\subsection{\RU{Работа с двухмерным массивом как с одномерным}
\EN{Access two-dimensional array as one-dimensional}}

\RU{Я могу легко показать как работать с двухмерным массивом как с одномерным, по крайней мере используя
два метода:}
\EN{I can easily show you how to access a two-dimensional array as one-dimensional array in at least two other ways:}

\lstinputlisting{patterns/13_arrays/5_multidimensional/2D_as_1D.c.\LANG}

\RU{Компилируете и запускаете: мы увидим корректные значения.}
\EN{Compile and run it: it will show correct values.}

\RU{Что сделал MSVC 2013 это очаровательно, все три процедуры одинаковые!}
\EN{What MSVC 2013 did is fascinating, all three routines are just the same!}

\lstinputlisting[caption=\Optimizing MSVC 2013 x64]{patterns/13_arrays/5_multidimensional/2D_as_1D_MSVC_2013_Ox_x64.asm.\LANG}

\ifdefined\IncludeGCC
\RU{GCC также сгенерировал эквивалентные процедуры, но немного отличающиеся:}
\EN{GCC also generates equivalent routines, but slightly different:}

\lstinputlisting[caption=\Optimizing GCC 4.9 x64]{patterns/13_arrays/5_multidimensional/2D_as_1D_GCC49_x64_O3.s.\LANG}
\fi

\subsection{\RU{Пример с трехмерным массивом}\EN{Three-dimensional array example}}

\RU{То же самое и для многомерных массивов.}\EN{It's thing in multidimensional arrays.}
\RU{На этот раз будем работать с массивом типа \Tint: каждый элемент требует 4 байта в памяти.}
\EN{Now we are going to work with an array of type \Tint: each element requires 4 bytes in memory.}

\RU{Попробуем}\EN{Let's see}:

\lstinputlisting[caption=\RU{простой пример}\EN{simple example}]{patterns/13_arrays/5_multidimensional/multi.c}

\subsubsection{x86}

\RU{В итоге}\EN{We get} (MSVC 2010):

\lstinputlisting[caption=MSVC 2010]{patterns/13_arrays/5_multidimensional/multi_msvc.asm.\LANG}

\RU{В принципе, ничего удивительного. В \TT{insert()} для вычисления адреса нужного элемента массива 
три входных аргумента перемножаются по формуле $address=600 \cdot 4 \cdot x + 30 \cdot 4 \cdot y + 4z$, 
чтобы представить массив трехмерным.
Не забывайте также, что тип \Tint 32-битный (4 байта), поэтому все коэффициенты нужно умножить на 4.}
\EN{Nothing special. For index calculation, three input arguments are used 
in the formula $address=600 \cdot 4 \cdot x + 30 \cdot 4 \cdot y + 4z$, to represent the array as multidimensional.
Do not forget that the \Tint type is 32-bit (4 bytes),
so all coefficients must be multiplied by 4.}

\lstinputlisting[caption=GCC 4.4.1]{patterns/13_arrays/5_multidimensional/multi_gcc.asm.\LANG}

\RU{Компилятор GCC решил всё сделать немного иначе}\EN{The GCC compiler does it differently}.
\RU{Для вычисления одной из операций ($30y$), GCC создал код, где нет самой операции умножения.}
\EN{For one of the operations in the calculation ($30y$), GCC produces code without multiplication instructions.}
\RU{Происходит это так}\EN{This is how it done}: 
$(y+y) \ll 4 - (y+y) = (2y) \ll 4 - 2y = 2 \cdot 16 \cdot y - 2y = 32y - 2y = 30y$. 
\RU{Таким образом, для вычисления $30y$ используется только операция сложения, 
операция битового сдвига и операция вычитания.}\EN{Thus, for the $30y$ calculation, only one addition operation,
one bitwise shift operation and one subtraction operation are used.}
\RU{Это работает быстрее}\EN{This works faster}.

\ifdefined\IncludeARM
\subsubsection{ARM + \NonOptimizingXcodeIV (\ThumbMode)}

\lstinputlisting[caption=\NonOptimizingXcodeIV (\ThumbMode)]{patterns/13_arrays/5_multidimensional/multi_Xcode_thumb_O0.asm.\LANG}

\NonOptimizing LLVM \RU{сохраняет все переменные в локальном стеке, хотя это и избыточно.}
\EN{saves all variables in local stack, which is redundant.}
\RU{Адрес элемента массива вычисляется по уже рассмотренной формуле.}
\EN{The address of the array element is calculated by the formula we already saw.}

\subsubsection{ARM + \OptimizingXcodeIV (\ThumbMode)}

\lstinputlisting[caption=\OptimizingXcodeIV (\ThumbMode)]{patterns/13_arrays/5_multidimensional/multi_Xcode_thumb_O3.asm.\LANG}

\RU{Тут используются уже описанные трюки для замены умножения на операции сдвига, сложения и вычитания.}
\EN{The tricks for replacing multiplication by shift, addition and subtraction which we already saw
are also present here.}

\index{ARM!\Instructions!RSB}
\index{ARM!\Instructions!SUB}
\RU{Также мы видим новую для себя инструкцию}\EN{Here we also see a new instruction for us:} 
\RSB (\IT{Reverse Subtract}).
\RU{Она работает так же, как и \SUB, только меняет операнды местами.}
\EN{It works just as \SUB, but it swaps its operands with each other before execution.}
\RU{Зачем?}\EN{Why?}
\index{ARM!Optional operators!LSL}
\SUB \AndENRU \RSB\RU{ это те инструкции, ко второму операнду которых 
можно применить коэффициент сдвига, как мы видим и здесь}%
\EN{ are instructions, to the second operand of which shift coefficient may be applied}: 
(\INS{LSL\#4}). 
\RU{Но этот коэффициент можно применить только ко второму операнду.}
\EN{But this coefficient can be applied only to second operand.}
\RU{Для коммутативных операций, таких как сложение или умножение, 
операнды можно менять местами и это не влияет на результат.}
\EN{That's fine for commutative operations like addition or multiplication 
(operands may be swapped there without changing the result).}
\RU{Но вычитание~--- операция некоммутативная, так что для этих случаев существует инструкция \RSB.}
\EN{But subtraction is a non-commutative operation, so \RSB exist for these cases.}

\index{ARM!\Instructions!LDR.W}
\EN{The}\RU{Инструкция} \INS{LDR.W R9, [R9]} \RU{работает как}\EN{instruction works like} \LEA~(\myref{sec:LEA})
\RU{в x86, и здесь она ничего не делает, она избыточна.}
\EN{in x86, but it does nothing here, it is redundant.}
\RU{Вероятно, компилятор не оптимизировал её.}\EN{Apparently, the compiler did not optimize it out.}
\fi

\ifdefined\IncludeMIPS
\subsubsection{MIPS}

\index{MIPS!Global Pointer}
\EN{My example is tiny, so the GCC compiler decided to put the $a$ array into the 64KiB area 
addressable by the Global Pointer.}
\RU{Мой пример такой крошечный, что компилятор GCC решил разместить массив $a$ в 64KiB-области,
адресуемой при помощи Global Pointer.}

\lstinputlisting[caption=\Optimizing GCC 4.4.5 (IDA)]{patterns/13_arrays/5_multidimensional/multi_MIPS_O3_IDA.lst.\LANG}

\fi


\ifx\LITE\undefined
\subsection{\RU{Еще примеры}\EN{More examples}}

\RU{Компьютерный экран представляет собой двухмерный массив, но видеобуфер это линейный
одномерный массив}\EN{The computer screen is represented as a 2D array, but the video-buffer is 
a linear 1D array}. 
\RU{Мы рассматриваем это здесь}\EN{We talk about it here}: \myref{Mandelbrot_demo}.
\fi

\section{\Exercises}

\begin{itemize}
	\item \url{http://challenges.re/62}
	\item \url{http://challenges.re/63}
	\item \url{http://challenges.re/64}
	\item \url{http://challenges.re/65}
	\item \url{http://challenges.re/66}
\end{itemize}



\EN{\mysection{\BitfieldsChapter}
\label{sec:bitfields}

A lot of functions define their input arguments as flags in bit fields.
\myindex{\CLanguageElements!C99!bool}

Of course, they could be substituted by a set of \Tbool-typed variables, but it is not frugally.

% sections
\subsection{\RU{Проверка какого-либо бита}\EN{Specific bit checking}}

\EN{\input{patterns/14_bitfields/1_check/x86_EN}}
\RU{\input{patterns/14_bitfields/1_check/x86_RU}}
\EN{\input{patterns/14_bitfields/1_check/ARM_EN}}
\RU{\input{patterns/14_bitfields/1_check/ARM_RU}}


\subsection{\RU{Установка и сброс отдельного бита}\EN{Setting and clearing specific bits}}

\RU{Например}\EN{For example}:

\lstinputlisting[style=customc]{patterns/14_bitfields/2_set_reset/set_reset.c}

\EN{\input{patterns/14_bitfields/2_set_reset/x86_EN}}
\RU{\input{patterns/14_bitfields/2_set_reset/x86_RU}}
\EN{\input{patterns/14_bitfields/2_set_reset/ARM_EN}}
\RU{\input{patterns/14_bitfields/2_set_reset/ARM_RU}}
\EN{\input{patterns/14_bitfields/2_set_reset/MIPS_EN}}
\RU{\input{patterns/14_bitfields/2_set_reset/MIPS_RU}}


\section{\ShiftsSectionName}

\RU{Битовые сдвиги в \CCpp реализованы при помощи операторов $\ll$ и $\gg$.}
\EN{Bit shifts in \CCpp are implemented via $\ll$ and $\gg$ operators.}

\RU{В x86 есть инструкции}\EN{x86 \ac{ISA} has} SHL (SHift Left) \AndENRU SHR (SHift Right) 
\RU{для этого}\EN{instructions for this}.

\subsection{\RU{Деление и умножение при помощи сдвигов}\EN{Division and multiplication using shifts}}
\label{subsec:mult_div_shifts}

\RU{Инструкции сдвига также активно применяются при делении или умножении 
на числа-степени двойки: $2^{n}$ (т.е., $1$, $2$, $4$, $8$, и т.д.).}
\EN{Shift instructions are often used in division and multiplications by power of two numbers:
$2^{n}$ (e.g., $1$, $2$, $4$, $8$, etc).}

\subsubsection{\RU{Умножение}\EN{Multiplication}}

\begin{lstlisting}
unsigned int f(unsigned int a)
{
	return a*4;
};
\end{lstlisting}

\begin{lstlisting}[caption=\NonOptimizing MSVC 2010]
_a$ = 8		; size = 4
_f	PROC
	push	ebp
	mov	ebp, esp
	mov	eax, DWORD PTR _a$[ebp]
	shl	eax, 2
	pop	ebp
	ret	0
_f	ENDP
\end{lstlisting}

\RU{Умножить на $4$ это просто сдвинуть число на 2 бита влево, 
вставив 2 нулевых бита справа (как два самых младших бита). 
Это как умножить $3$ на $100$ ~--- нужно просто дописать два нуля справа.}
\EN{Multiplication by $4$ is just shifting the number to the left by 2 bits,
while inserting 2 zero bits at right (as the last two bits).
It is just like to multiply $3$ by $100$~---we need just to add two zeroes at the right.}

\RU{Вот как работает инструкция сдвига влево}\EN{That's how shift left instruction works}:

\index{x86!\Instructions!SHL}
\input{shift_left}

\RU{Добавленные биты справа --- всегда нули}\EN{Added bits at right---always zeroes}.

\RU{Умножение на 4 в}\EN{Multiplication by 4 in} ARM:

\begin{lstlisting}[caption=\NonOptimizingKeilVI + \ARMMode]
f PROC
        LSL      r0,r0,#2
        BX       lr
        ENDP
\end{lstlisting}

\subsubsection{\RU{Деление}\EN{Division}}

\RU{Например}\EN{For example}:

\begin{lstlisting}
unsigned int f(unsigned int a)
{
	return a/4;
};
\end{lstlisting}

\RU{Имеем в итоге}\EN{We got} (MSVC 2010):

\begin{lstlisting}[caption=MSVC 2010]
_a$ = 8							; size = 4
_f	PROC
	mov	eax, DWORD PTR _a$[esp-4]
	shr	eax, 2
	ret	0
_f	ENDP
\end{lstlisting}

\label{SHR}
\index{x86!\Instructions!SHR}
\RU{Инструкция \SHR (\IT{SHift Right}) в данном примере сдвигает число на 2 бита вправо. 
При этом, освободившиеся два бита слева (т.е., самые 
старшие разряды), выставляются в нули. А самые младшие 2 бита выкидываются. 
Фактически, эти два выкинутых бита ~--- остаток от деления.}
\EN{\SHR (\IT{SHift Right}) instruction in this example is shifting a number by 2 bits right.
Two freed bits at left (e.g., two most significant bits) are set to zero.
Two least significant bits are dropped.
In fact, these two dropped bits~---division operation remainder.}

\index{x86!\Instructions!SHR}
\RU{Инструкция \SHR работает так же, как и \SHL, только в другую сторону.}
\EN{\SHR instruction works just like as \SHL but in other direction.}

\input{shift_right}

\label{division_by_shifting}
\RU{Для того, чтобы это проще понять, представьте себе десятичную систему счисления и число $23$. 
$23$ можно разделить на $10$ просто откинув последний разряд ($3$ ~--- это остаток от деления). 
После этой операции останется $2$ как \glslink{quotient}{частное}.}
\EN{It can be easily understood if to imagine decimal numeral system and number $23$.
$23$ can be easily divided by $10$ just by dropping last digit ($3$~---is division remainder). 
$2$ is leaving after operation as a \gls{quotient}.}

\RU{Деление на 4 в}\EN{Division by 4 in} ARM:

\begin{lstlisting}[caption=\NonOptimizingKeilVI + \ARMMode]
f PROC
        LSR      r0,r0,#2
        BX       lr
        ENDP
\end{lstlisting}

\EN{\input{patterns/14_bitfields/35_set_reset_FPU/main_EN}}
\RU{\input{patterns/14_bitfields/35_set_reset_FPU/main_RU}}

\EN{\input{patterns/14_bitfields/4_popcnt/main_EN}}
\RU{\input{patterns/14_bitfields/4_popcnt/main_RU}}


% TODO: add ROL/ROR
\subsection{\Conclusion{}}

\myindex{x86!\Instructions!SHR}
\myindex{x86!\Instructions!SHL}
\myindex{x86!\Instructions!SAR}

Analogous to the \CCpp shifting operators \TT{$\ll$} and \TT{$\gg$},
the shift instructions in x86 are \SHR/\SHL (for unsigned values) and \SAR/\SHL (for signed values).

\myindex{ARM!\Instructions!LSR}
\myindex{ARM!\Instructions!LSL}
\myindex{ARM!\Instructions!ASR}

The shift instructions in ARM are \LSR/\LSL (for unsigned values) and \ASR/\LSL (for signed values).

It's also possible to add shift suffix to some instructions 
(which are called \q{data processing instructions}).
% FIXME: which instructions?

\subsubsection{Check for specific bit (known at compile stage)}

Test if the 0b1000000 bit (0x40) is present in the register's value:

\lstinputlisting[caption=\CCpp,style=customc]{patterns/14_bitfields/c_snippet0.c}

\lstinputlisting[caption=x86,style=customasmx86]{patterns/14_bitfields/TEST_JNZ_EN.lst}

\lstinputlisting[caption=x86,style=customasmx86]{patterns/14_bitfields/TEST_JZ_EN.lst}

\lstinputlisting[caption=ARM (\ARMMode),style=customasmARM]{patterns/14_bitfields/TST_BNE_EN.lst}

\myindex{x86!\Instructions!AND}
\myindex{x86!\Instructions!TEST}

Sometimes, \AND is used instead of \TEST, but the flags that are set are the same.

\subsubsection{Check for specific bit (specified at runtime)}

This is usually done by this \CCpp code snippet (shift value by $n$ bits right, then cut off lowest bit):

\lstinputlisting[caption=\CCpp,style=customc]{patterns/14_bitfields/c_snippet1.c}

This is usually implemented in x86 code as:

\begin{lstlisting}[caption=x86,style=customasmx86]
; REG=input_value
; CL=n
SHR REG, CL
AND REG, 1
\end{lstlisting}

Or (shift 1 bit $n$ times left, isolate this bit in input value and check if it's not zero):

\lstinputlisting[caption=\CCpp,style=customc]{patterns/14_bitfields/c_snippet2.c}

This is usually implemented in x86 code as:

\begin{lstlisting}[caption=x86,style=customasmx86]
; CL=n
MOV REG, 1
SHL REG, CL
AND input_value, REG
\end{lstlisting}

\subsubsection{Set specific bit (known at compile stage)}

\begin{lstlisting}[caption=\CCpp]
value=value|0x40;
\end{lstlisting}

\begin{lstlisting}[caption=x86,style=customasmx86]
OR REG, 40h
\end{lstlisting}

\begin{lstlisting}[caption=ARM (\ARMMode) and ARM64,style=customasmARM]
ORR R0, R0, #0x40
\end{lstlisting}

\subsubsection{Set specific bit (specified at runtime)}

\lstinputlisting[caption=\CCpp,style=customc]{patterns/14_bitfields/c_snippet3.c}

This is usually implemented in x86 code as:

\begin{lstlisting}[caption=x86,style=customasmx86]
; CL=n
MOV REG, 1
SHL REG, CL
OR input_value, REG
\end{lstlisting}

\subsubsection{Clear specific bit (known at compile stage)}

Just apply \AND operation with the inverted value:

\begin{lstlisting}[caption=\CCpp,style=customc]
value=value&(~0x40);
\end{lstlisting}

\begin{lstlisting}[caption=x86,style=customasmx86]
AND REG, 0FFFFFFBFh
\end{lstlisting}

\begin{lstlisting}[caption=x64,style=customasmx86]
AND REG, 0FFFFFFFFFFFFFFBFh
\end{lstlisting}

This is actually leaving all bits set except one.

\myindex{ARM!\Instructions!BIC}

ARM in ARM mode has \BIC instruction, which works like the \NOT+\AND instruction pair:

\begin{lstlisting}[caption=ARM (\ARMMode),style=customasmARM]
BIC R0, R0, #0x40
\end{lstlisting}

\subsubsection{
Clear specific bit (specified at runtime)}

\lstinputlisting[caption=\CCpp,style=customc]{patterns/14_bitfields/c_snippet4.c}

\begin{lstlisting}[caption=x86,style=customasmx86]
; CL=n
MOV REG, 1
SHL REG, CL
NOT REG
AND input_value, REG
\end{lstlisting}

\section{\Exercises}

\subsection{\Exercise \#1}
\label{exercise_bitfields_1}

\WhatThisCodeDoes\

\begin{lstlisting}[caption=\Optimizing MSVC 2010]
_a$ = 8
_f	PROC
	mov	ecx, DWORD PTR _a$[esp-4]
	mov	eax, ecx
	mov	edx, ecx
	shl	edx, 16		; 00000010H
	and	eax, 65280	; 0000ff00H
	or	eax, edx
	mov	edx, ecx
	and	edx, 16711680	; 00ff0000H
	shr	ecx, 16		; 00000010H
	or	edx, ecx
	shl	eax, 8
	shr	edx, 8
	or	eax, edx
	ret	0
_f	ENDP
\end{lstlisting}

\begin{lstlisting}[caption=\OptimizingKeilVI (\ARMMode)]
f PROC
        MOV      r1,#0xff0000
        AND      r1,r1,r0,LSL #8
        MOV      r2,#0xff00
        ORR      r1,r1,r0,LSR #24
        AND      r2,r2,r0,LSR #8
        ORR      r1,r1,r2
        ORR      r0,r1,r0,LSL #24
        BX       lr
        ENDP
\end{lstlisting}

\begin{lstlisting}[caption=\OptimizingKeilVI (\ThumbMode)]
f PROC
        MOVS     r3,#0xff
        LSLS     r2,r0,#8
        LSLS     r3,r3,#16
        ANDS     r2,r2,r3
        LSRS     r1,r0,#24
        ORRS     r1,r1,r2
        LSRS     r2,r0,#8
        ASRS     r3,r3,#8
        ANDS     r2,r2,r3
        ORRS     r1,r1,r2
        LSLS     r0,r0,#24
        ORRS     r0,r0,r1
        BX       lr
        ENDP
\end{lstlisting}

\begin{lstlisting}[caption=\Optimizing GCC 4.9 (ARM64)]
f:
	rev	w0, w0
	ret
\end{lstlisting}

\lstinputlisting[caption=\Optimizing GCC 4.4.5 (MIPS) (IDA)]{patterns/14_bitfields/1_MIPS_O3_IDA.lst}

\Answer{}: \myref{exercise_solutions_bitfields_1}.

\subsection{\Exercise \#2}
\label{exercise_bitfields_2}

\WhatThisCodeDoes\

\begin{lstlisting}[caption=\Optimizing MSVC 2010]
_a$ = 8							; size = 4
_f	PROC
	push	esi
	mov	esi, DWORD PTR _a$[esp]
	xor	ecx, ecx
	push	edi
	lea	edx, DWORD PTR [ecx+1]
	xor	eax, eax
	npad	3 ; align next label
$LL3@f:
	mov	edi, esi
	shr	edi, cl
	add	ecx, 4
	and	edi, 15
	imul	edi, edx
	lea	edx, DWORD PTR [edx+edx*4]
	add	eax, edi
	add	edx, edx
	cmp	ecx, 28
	jle	SHORT $LL3@f
	pop	edi
	pop	esi
	ret	0
_f	ENDP
\end{lstlisting}

\begin{lstlisting}[caption=\OptimizingKeilVI (\ARMMode)]
f PROC
        MOV      r3,r0
        MOV      r1,#0
        MOV      r2,#1
        MOV      r0,r1
|L0.16|
        LSR      r12,r3,r1
        AND      r12,r12,#0xf
        MLA      r0,r12,r2,r0
        ADD      r1,r1,#4
        ADD      r2,r2,r2,LSL #2
        CMP      r1,#0x1c
        LSL      r2,r2,#1
        BLE      |L0.16|
        BX       lr
        ENDP
\end{lstlisting}

\begin{lstlisting}[caption=\OptimizingKeilVI (\ThumbMode)]
f PROC
        PUSH     {r4,lr}
        MOVS     r3,r0
        MOVS     r1,#0
        MOVS     r2,#1
        MOVS     r0,r1
|L0.10|
        MOVS     r4,r3
        LSRS     r4,r4,r1
        LSLS     r4,r4,#28
        LSRS     r4,r4,#28
        MULS     r4,r2,r4
        ADDS     r0,r4,r0
        MOVS     r4,#0xa
        MULS     r2,r4,r2
        ADDS     r1,r1,#4
        CMP      r1,#0x1c
        BLE      |L0.10|
        POP      {r4,pc}
        ENDP
\end{lstlisting}

\begin{lstlisting}[caption=\NonOptimizing GCC 4.9 (ARM64)]
f:
	sub	sp, sp, #32
	str	w0, [sp,12]
	str	wzr, [sp,28]
	mov	w0, 1
	str	w0, [sp,24]
	str	wzr, [sp,20]
	b	.L2
.L3:
	ldr	w0, [sp,28]
	ldr	w1, [sp,12]
	lsr	w0, w1, w0
	and	w1, w0, 15
	ldr	w0, [sp,24]
	mul	w0, w1, w0
	ldr	w1, [sp,20]
	add	w0, w1, w0
	str	w0, [sp,20]
	ldr	w0, [sp,28]
	add	w0, w0, 4
	str	w0, [sp,28]
	ldr	w1, [sp,24]
	mov	w0, w1
	lsl	w0, w0, 2
	add	w0, w0, w1
	lsl	w0, w0, 1
	str	w0, [sp,24]
.L2:
	ldr	w0, [sp,28]
	cmp	w0, 28
	ble	.L3
	ldr	w0, [sp,20]
	add	sp, sp, 32
	ret
\end{lstlisting}

\lstinputlisting[caption=\Optimizing GCC 4.4.5 (MIPS) (IDA)]{patterns/14_bitfields/2_MIPS_O3_IDA.lst}

\Answer{}: \myref{exercise_solutions_bitfields_2}.

\subsection{\Exercise \#3}
\label{exercise_bitfields_3}

\EN{Using the \ac{MSDN} documentation, find out which flags were used in the \TT{MessageBox()} win32 function call.}
\RU{Используя документацию \ac{MSDN}, найдите, какие флаги использовались в вызове win32-функции 
\TT{MessageBox()}.}

\begin{lstlisting}[caption=\Optimizing MSVC 2010]
_main	PROC
	push	278595		; 00044043H
	push	OFFSET $SG79792 ; 'caption'
	push	OFFSET $SG79793 ; 'hello, world!'
	push	0
	call	DWORD PTR __imp__MessageBoxA@16
	xor	eax, eax
	ret	0
_main	ENDP
\end{lstlisting}

\Answer{}: \myref{exercise_solutions_bitfields_3}.

\subsection{\Exercise \#4}
\label{exercise_bitfields_4}

\WhatThisCodeDoes\

\begin{lstlisting}[caption=\Optimizing MSVC 2010]
_m$ = 8		; size = 4
_n$ = 12	; size = 4
_f	PROC
	mov	ecx, DWORD PTR _n$[esp-4]
	xor	eax, eax
	xor	edx, edx
	test	ecx, ecx
	je	SHORT $LN2@f
	push	esi
	mov	esi, DWORD PTR _m$[esp]
$LL3@f:
	test	cl, 1
	je	SHORT $LN1@f
	add	eax, esi
	adc	edx, 0
$LN1@f:
	add	esi, esi
	shr	ecx, 1
	jne	SHORT $LL3@f
	pop	esi
$LN2@f:
	ret	0
_f	ENDP
\end{lstlisting}

\begin{lstlisting}[caption=\OptimizingKeilVI (\ARMMode)]
f PROC
        PUSH     {r4,lr}
        MOV      r3,r0
        MOV      r0,#0
        MOV      r2,r0
        MOV      r12,r0
        B        |L0.48|
|L0.24|
        TST      r1,#1
        BEQ      |L0.40|
        ADDS     r0,r0,r3
        ADC      r2,r2,r12
|L0.40|
        LSL      r3,r3,#1
        LSR      r1,r1,#1
|L0.48|
        CMP      r1,#0
        MOVEQ    r1,r2
        BNE      |L0.24|
        POP      {r4,pc}
        ENDP
\end{lstlisting}

\begin{lstlisting}[caption=\OptimizingKeilVI (\ThumbMode)]
f PROC
        PUSH     {r4,r5,lr}
        MOVS     r3,r0
        MOVS     r0,#0
        MOVS     r2,r0
        MOVS     r4,r0
        B        |L0.24|
|L0.12|
        LSLS     r5,r1,#31
        BEQ      |L0.20|
        ADDS     r0,r0,r3
        ADCS     r2,r2,r4
|L0.20|
        LSLS     r3,r3,#1
        LSRS     r1,r1,#1
|L0.24|
        CMP      r1,#0
        BNE      |L0.12|
        MOVS     r1,r2
        POP      {r4,r5,pc}
        ENDP
\end{lstlisting}

\begin{lstlisting}[caption=\Optimizing GCC 4.9 (ARM64)]
f:
	mov	w2, w0
	mov	x0, 0
	cbz	w1, .L2
.L3:
	and	w3, w1, 1
	lsr	w1, w1, 1
	cmp	w3, wzr
	add	x3, x0, x2, uxtw
	lsl	w2, w2, 1
	csel	x0, x3, x0, ne
	cbnz	w1, .L3
.L2:
	ret
\end{lstlisting}

\lstinputlisting[caption=\Optimizing GCC 4.4.5 (MIPS) (IDA)]{patterns/14_bitfields/4_MIPS_O3_IDA.lst}

\Answer{}: \myref{exercise_solutions_bitfields_4}.

}
\RU{\mysection{\BitfieldsChapter}
\label{sec:bitfields}

Немало функций задают различные флаги в аргументах при помощи битовых полей\footnote{bit fields в англоязычной литературе}.

\myindex{\CLanguageElements!C99!bool}
Наверное, вместо этого можно было бы использовать набор переменных типа \Tbool, но это было бы 
не очень экономно.

% sections
\subsection{\RU{Проверка какого-либо бита}\EN{Specific bit checking}}

\EN{\input{patterns/14_bitfields/1_check/x86_EN}}
\RU{\input{patterns/14_bitfields/1_check/x86_RU}}
\EN{\input{patterns/14_bitfields/1_check/ARM_EN}}
\RU{\input{patterns/14_bitfields/1_check/ARM_RU}}


\subsection{\RU{Установка и сброс отдельного бита}\EN{Setting and clearing specific bits}}

\RU{Например}\EN{For example}:

\lstinputlisting[style=customc]{patterns/14_bitfields/2_set_reset/set_reset.c}

\EN{\input{patterns/14_bitfields/2_set_reset/x86_EN}}
\RU{\input{patterns/14_bitfields/2_set_reset/x86_RU}}
\EN{\input{patterns/14_bitfields/2_set_reset/ARM_EN}}
\RU{\input{patterns/14_bitfields/2_set_reset/ARM_RU}}
\EN{\input{patterns/14_bitfields/2_set_reset/MIPS_EN}}
\RU{\input{patterns/14_bitfields/2_set_reset/MIPS_RU}}


\section{\ShiftsSectionName}

\RU{Битовые сдвиги в \CCpp реализованы при помощи операторов $\ll$ и $\gg$.}
\EN{Bit shifts in \CCpp are implemented via $\ll$ and $\gg$ operators.}

\RU{В x86 есть инструкции}\EN{x86 \ac{ISA} has} SHL (SHift Left) \AndENRU SHR (SHift Right) 
\RU{для этого}\EN{instructions for this}.

\subsection{\RU{Деление и умножение при помощи сдвигов}\EN{Division and multiplication using shifts}}
\label{subsec:mult_div_shifts}

\RU{Инструкции сдвига также активно применяются при делении или умножении 
на числа-степени двойки: $2^{n}$ (т.е., $1$, $2$, $4$, $8$, и т.д.).}
\EN{Shift instructions are often used in division and multiplications by power of two numbers:
$2^{n}$ (e.g., $1$, $2$, $4$, $8$, etc).}

\subsubsection{\RU{Умножение}\EN{Multiplication}}

\begin{lstlisting}
unsigned int f(unsigned int a)
{
	return a*4;
};
\end{lstlisting}

\begin{lstlisting}[caption=\NonOptimizing MSVC 2010]
_a$ = 8		; size = 4
_f	PROC
	push	ebp
	mov	ebp, esp
	mov	eax, DWORD PTR _a$[ebp]
	shl	eax, 2
	pop	ebp
	ret	0
_f	ENDP
\end{lstlisting}

\RU{Умножить на $4$ это просто сдвинуть число на 2 бита влево, 
вставив 2 нулевых бита справа (как два самых младших бита). 
Это как умножить $3$ на $100$ ~--- нужно просто дописать два нуля справа.}
\EN{Multiplication by $4$ is just shifting the number to the left by 2 bits,
while inserting 2 zero bits at right (as the last two bits).
It is just like to multiply $3$ by $100$~---we need just to add two zeroes at the right.}

\RU{Вот как работает инструкция сдвига влево}\EN{That's how shift left instruction works}:

\index{x86!\Instructions!SHL}
\input{shift_left}

\RU{Добавленные биты справа --- всегда нули}\EN{Added bits at right---always zeroes}.

\RU{Умножение на 4 в}\EN{Multiplication by 4 in} ARM:

\begin{lstlisting}[caption=\NonOptimizingKeilVI + \ARMMode]
f PROC
        LSL      r0,r0,#2
        BX       lr
        ENDP
\end{lstlisting}

\subsubsection{\RU{Деление}\EN{Division}}

\RU{Например}\EN{For example}:

\begin{lstlisting}
unsigned int f(unsigned int a)
{
	return a/4;
};
\end{lstlisting}

\RU{Имеем в итоге}\EN{We got} (MSVC 2010):

\begin{lstlisting}[caption=MSVC 2010]
_a$ = 8							; size = 4
_f	PROC
	mov	eax, DWORD PTR _a$[esp-4]
	shr	eax, 2
	ret	0
_f	ENDP
\end{lstlisting}

\label{SHR}
\index{x86!\Instructions!SHR}
\RU{Инструкция \SHR (\IT{SHift Right}) в данном примере сдвигает число на 2 бита вправо. 
При этом, освободившиеся два бита слева (т.е., самые 
старшие разряды), выставляются в нули. А самые младшие 2 бита выкидываются. 
Фактически, эти два выкинутых бита ~--- остаток от деления.}
\EN{\SHR (\IT{SHift Right}) instruction in this example is shifting a number by 2 bits right.
Two freed bits at left (e.g., two most significant bits) are set to zero.
Two least significant bits are dropped.
In fact, these two dropped bits~---division operation remainder.}

\index{x86!\Instructions!SHR}
\RU{Инструкция \SHR работает так же, как и \SHL, только в другую сторону.}
\EN{\SHR instruction works just like as \SHL but in other direction.}

\input{shift_right}

\label{division_by_shifting}
\RU{Для того, чтобы это проще понять, представьте себе десятичную систему счисления и число $23$. 
$23$ можно разделить на $10$ просто откинув последний разряд ($3$ ~--- это остаток от деления). 
После этой операции останется $2$ как \glslink{quotient}{частное}.}
\EN{It can be easily understood if to imagine decimal numeral system and number $23$.
$23$ can be easily divided by $10$ just by dropping last digit ($3$~---is division remainder). 
$2$ is leaving after operation as a \gls{quotient}.}

\RU{Деление на 4 в}\EN{Division by 4 in} ARM:

\begin{lstlisting}[caption=\NonOptimizingKeilVI + \ARMMode]
f PROC
        LSR      r0,r0,#2
        BX       lr
        ENDP
\end{lstlisting}

\EN{\input{patterns/14_bitfields/35_set_reset_FPU/main_EN}}
\RU{\input{patterns/14_bitfields/35_set_reset_FPU/main_RU}}

\EN{\input{patterns/14_bitfields/4_popcnt/main_EN}}
\RU{\input{patterns/14_bitfields/4_popcnt/main_RU}}


% TODO: add ROL/ROR
\subsection{\Conclusion{}}

\myindex{x86!\Instructions!SHR}
\myindex{x86!\Instructions!SHL}
\myindex{x86!\Instructions!SAR}
Инструкции сдвига, аналогичные операторам \CCpp \TT{$\ll$} и \TT{$\gg$}, в x86 это \SHR/\SHL (для беззнаковых значений), \SAR/\SHL (для знаковых значений).

\myindex{ARM!\Instructions!LSR}
\myindex{ARM!\Instructions!LSL}
\myindex{ARM!\Instructions!ASR}
Инструкции сдвига в ARM это \LSR/\LSL (для беззнаковых значений), \ASR/\LSL (для знаковых значений).

Можно также добавлять суффикс сдвига для некоторых инструкций 
(которые называются \q{data processing instructions}).

% FIXME: which instructions?

\subsubsection{Проверка определенного бита (известного на стадии компиляции)}

Проверить, присутствует ли бит 0b1000000 (0x40) в значении в регистре:

\lstinputlisting[caption=\CCpp,style=customc]{patterns/14_bitfields/c_snippet0.c}

\lstinputlisting[caption=x86,style=customasmx86]{patterns/14_bitfields/TEST_JNZ_RU.lst}

\lstinputlisting[caption=x86,style=customasmx86]{patterns/14_bitfields/TEST_JZ_RU.lst}

\lstinputlisting[caption=ARM (\ARMMode),style=customasmARM]{patterns/14_bitfields/TST_BNE_RU.lst}

\myindex{x86!\Instructions!AND}
\myindex{x86!\Instructions!TEST}
Иногда \AND используется вместо \TEST, но флаги выставляются точно также.

\subsubsection{Проверка определенного бита (заданного во время исполнения)}

Это обычно происходит при помощи вот такого фрагмента на \CCpp (сдвинуть значение на $n$ бит вправо,
затем отрезать самый младший бит):

\lstinputlisting[caption=\CCpp,style=customc]{patterns/14_bitfields/c_snippet1.c}

Это обычно реализуется в x86-коде так:

\begin{lstlisting}[caption=x86,style=customasmx86]
; REG=input_value
; CL=n
SHR REG, CL
AND REG, 1
\end{lstlisting}

Или (сдвинуть 1 $n$ раз влево, изолировать этот же бит во входном значении и проверить, не ноль ли он):

\lstinputlisting[caption=\CCpp,style=customc]{patterns/14_bitfields/c_snippet2.c}

Это обычно так реализуется в x86-коде:

\begin{lstlisting}[caption=x86,style=customasmx86]
; CL=n
MOV REG, 1
SHL REG, CL
AND input_value, REG
\end{lstlisting}

\subsubsection{Установка определенного бита (известного во время компиляции)}

\begin{lstlisting}[caption=\CCpp,style=customc]
value=value|0x40;
\end{lstlisting}

\begin{lstlisting}[caption=x86,style=customasmx86]
OR REG, 40h
\end{lstlisting}

\begin{lstlisting}[caption=ARM (\ARMMode) и ARM64,style=customasmARM]
ORR R0, R0, #0x40
\end{lstlisting}

\subsubsection{Установка определенного бита (заданного во время исполнения)}

\lstinputlisting[caption=\CCpp,style=customc]{patterns/14_bitfields/c_snippet3.c}

Это обычно так реализуется в x86-коде:

\begin{lstlisting}[caption=x86,style=customasmx86]
; CL=n
MOV REG, 1
SHL REG, CL
OR input_value, REG
\end{lstlisting}

\subsubsection{Сброс определенного бита (известного во время компиляции)}

Просто исполните операцию логического \q{И} (\AND) с инвертированным значением:

\begin{lstlisting}[caption=\CCpp,style=customc]
value=value&(~0x40);
\end{lstlisting}

\begin{lstlisting}[caption=x86,style=customasmx86]
AND REG, 0FFFFFFBFh
\end{lstlisting}

\begin{lstlisting}[caption=x64,style=customasmx86]
AND REG, 0FFFFFFFFFFFFFFBFh
\end{lstlisting}

Это на самом деле сохранение всех бит кроме одного.

\myindex{ARM!\Instructions!BIC}
В ARM в режиме ARM есть инструкция \BIC, работающая как две инструкции \NOT+\AND:

\begin{lstlisting}[caption=ARM (\ARMMode),style=customasmARM]
BIC R0, R0, #0x40
\end{lstlisting}

\subsubsection{Сброс определенного бита (заданного во время исполнения)}

\lstinputlisting[caption=\CCpp,style=customc]{patterns/14_bitfields/c_snippet4.c}

\begin{lstlisting}[caption=x86,style=customasmx86]
; CL=n
MOV REG, 1
SHL REG, CL
NOT REG
AND input_value, REG
\end{lstlisting}

\section{\Exercises}

\subsection{\Exercise \#1}
\label{exercise_bitfields_1}

\WhatThisCodeDoes\

\begin{lstlisting}[caption=\Optimizing MSVC 2010]
_a$ = 8
_f	PROC
	mov	ecx, DWORD PTR _a$[esp-4]
	mov	eax, ecx
	mov	edx, ecx
	shl	edx, 16		; 00000010H
	and	eax, 65280	; 0000ff00H
	or	eax, edx
	mov	edx, ecx
	and	edx, 16711680	; 00ff0000H
	shr	ecx, 16		; 00000010H
	or	edx, ecx
	shl	eax, 8
	shr	edx, 8
	or	eax, edx
	ret	0
_f	ENDP
\end{lstlisting}

\begin{lstlisting}[caption=\OptimizingKeilVI (\ARMMode)]
f PROC
        MOV      r1,#0xff0000
        AND      r1,r1,r0,LSL #8
        MOV      r2,#0xff00
        ORR      r1,r1,r0,LSR #24
        AND      r2,r2,r0,LSR #8
        ORR      r1,r1,r2
        ORR      r0,r1,r0,LSL #24
        BX       lr
        ENDP
\end{lstlisting}

\begin{lstlisting}[caption=\OptimizingKeilVI (\ThumbMode)]
f PROC
        MOVS     r3,#0xff
        LSLS     r2,r0,#8
        LSLS     r3,r3,#16
        ANDS     r2,r2,r3
        LSRS     r1,r0,#24
        ORRS     r1,r1,r2
        LSRS     r2,r0,#8
        ASRS     r3,r3,#8
        ANDS     r2,r2,r3
        ORRS     r1,r1,r2
        LSLS     r0,r0,#24
        ORRS     r0,r0,r1
        BX       lr
        ENDP
\end{lstlisting}

\begin{lstlisting}[caption=\Optimizing GCC 4.9 (ARM64)]
f:
	rev	w0, w0
	ret
\end{lstlisting}

\lstinputlisting[caption=\Optimizing GCC 4.4.5 (MIPS) (IDA)]{patterns/14_bitfields/1_MIPS_O3_IDA.lst}

\Answer{}: \myref{exercise_solutions_bitfields_1}.

\subsection{\Exercise \#2}
\label{exercise_bitfields_2}

\WhatThisCodeDoes\

\begin{lstlisting}[caption=\Optimizing MSVC 2010]
_a$ = 8							; size = 4
_f	PROC
	push	esi
	mov	esi, DWORD PTR _a$[esp]
	xor	ecx, ecx
	push	edi
	lea	edx, DWORD PTR [ecx+1]
	xor	eax, eax
	npad	3 ; align next label
$LL3@f:
	mov	edi, esi
	shr	edi, cl
	add	ecx, 4
	and	edi, 15
	imul	edi, edx
	lea	edx, DWORD PTR [edx+edx*4]
	add	eax, edi
	add	edx, edx
	cmp	ecx, 28
	jle	SHORT $LL3@f
	pop	edi
	pop	esi
	ret	0
_f	ENDP
\end{lstlisting}

\begin{lstlisting}[caption=\OptimizingKeilVI (\ARMMode)]
f PROC
        MOV      r3,r0
        MOV      r1,#0
        MOV      r2,#1
        MOV      r0,r1
|L0.16|
        LSR      r12,r3,r1
        AND      r12,r12,#0xf
        MLA      r0,r12,r2,r0
        ADD      r1,r1,#4
        ADD      r2,r2,r2,LSL #2
        CMP      r1,#0x1c
        LSL      r2,r2,#1
        BLE      |L0.16|
        BX       lr
        ENDP
\end{lstlisting}

\begin{lstlisting}[caption=\OptimizingKeilVI (\ThumbMode)]
f PROC
        PUSH     {r4,lr}
        MOVS     r3,r0
        MOVS     r1,#0
        MOVS     r2,#1
        MOVS     r0,r1
|L0.10|
        MOVS     r4,r3
        LSRS     r4,r4,r1
        LSLS     r4,r4,#28
        LSRS     r4,r4,#28
        MULS     r4,r2,r4
        ADDS     r0,r4,r0
        MOVS     r4,#0xa
        MULS     r2,r4,r2
        ADDS     r1,r1,#4
        CMP      r1,#0x1c
        BLE      |L0.10|
        POP      {r4,pc}
        ENDP
\end{lstlisting}

\begin{lstlisting}[caption=\NonOptimizing GCC 4.9 (ARM64)]
f:
	sub	sp, sp, #32
	str	w0, [sp,12]
	str	wzr, [sp,28]
	mov	w0, 1
	str	w0, [sp,24]
	str	wzr, [sp,20]
	b	.L2
.L3:
	ldr	w0, [sp,28]
	ldr	w1, [sp,12]
	lsr	w0, w1, w0
	and	w1, w0, 15
	ldr	w0, [sp,24]
	mul	w0, w1, w0
	ldr	w1, [sp,20]
	add	w0, w1, w0
	str	w0, [sp,20]
	ldr	w0, [sp,28]
	add	w0, w0, 4
	str	w0, [sp,28]
	ldr	w1, [sp,24]
	mov	w0, w1
	lsl	w0, w0, 2
	add	w0, w0, w1
	lsl	w0, w0, 1
	str	w0, [sp,24]
.L2:
	ldr	w0, [sp,28]
	cmp	w0, 28
	ble	.L3
	ldr	w0, [sp,20]
	add	sp, sp, 32
	ret
\end{lstlisting}

\lstinputlisting[caption=\Optimizing GCC 4.4.5 (MIPS) (IDA)]{patterns/14_bitfields/2_MIPS_O3_IDA.lst}

\Answer{}: \myref{exercise_solutions_bitfields_2}.

\subsection{\Exercise \#3}
\label{exercise_bitfields_3}

\EN{Using the \ac{MSDN} documentation, find out which flags were used in the \TT{MessageBox()} win32 function call.}
\RU{Используя документацию \ac{MSDN}, найдите, какие флаги использовались в вызове win32-функции 
\TT{MessageBox()}.}

\begin{lstlisting}[caption=\Optimizing MSVC 2010]
_main	PROC
	push	278595		; 00044043H
	push	OFFSET $SG79792 ; 'caption'
	push	OFFSET $SG79793 ; 'hello, world!'
	push	0
	call	DWORD PTR __imp__MessageBoxA@16
	xor	eax, eax
	ret	0
_main	ENDP
\end{lstlisting}

\Answer{}: \myref{exercise_solutions_bitfields_3}.

\subsection{\Exercise \#4}
\label{exercise_bitfields_4}

\WhatThisCodeDoes\

\begin{lstlisting}[caption=\Optimizing MSVC 2010]
_m$ = 8		; size = 4
_n$ = 12	; size = 4
_f	PROC
	mov	ecx, DWORD PTR _n$[esp-4]
	xor	eax, eax
	xor	edx, edx
	test	ecx, ecx
	je	SHORT $LN2@f
	push	esi
	mov	esi, DWORD PTR _m$[esp]
$LL3@f:
	test	cl, 1
	je	SHORT $LN1@f
	add	eax, esi
	adc	edx, 0
$LN1@f:
	add	esi, esi
	shr	ecx, 1
	jne	SHORT $LL3@f
	pop	esi
$LN2@f:
	ret	0
_f	ENDP
\end{lstlisting}

\begin{lstlisting}[caption=\OptimizingKeilVI (\ARMMode)]
f PROC
        PUSH     {r4,lr}
        MOV      r3,r0
        MOV      r0,#0
        MOV      r2,r0
        MOV      r12,r0
        B        |L0.48|
|L0.24|
        TST      r1,#1
        BEQ      |L0.40|
        ADDS     r0,r0,r3
        ADC      r2,r2,r12
|L0.40|
        LSL      r3,r3,#1
        LSR      r1,r1,#1
|L0.48|
        CMP      r1,#0
        MOVEQ    r1,r2
        BNE      |L0.24|
        POP      {r4,pc}
        ENDP
\end{lstlisting}

\begin{lstlisting}[caption=\OptimizingKeilVI (\ThumbMode)]
f PROC
        PUSH     {r4,r5,lr}
        MOVS     r3,r0
        MOVS     r0,#0
        MOVS     r2,r0
        MOVS     r4,r0
        B        |L0.24|
|L0.12|
        LSLS     r5,r1,#31
        BEQ      |L0.20|
        ADDS     r0,r0,r3
        ADCS     r2,r2,r4
|L0.20|
        LSLS     r3,r3,#1
        LSRS     r1,r1,#1
|L0.24|
        CMP      r1,#0
        BNE      |L0.12|
        MOVS     r1,r2
        POP      {r4,r5,pc}
        ENDP
\end{lstlisting}

\begin{lstlisting}[caption=\Optimizing GCC 4.9 (ARM64)]
f:
	mov	w2, w0
	mov	x0, 0
	cbz	w1, .L2
.L3:
	and	w3, w1, 1
	lsr	w1, w1, 1
	cmp	w3, wzr
	add	x3, x0, x2, uxtw
	lsl	w2, w2, 1
	csel	x0, x3, x0, ne
	cbnz	w1, .L3
.L2:
	ret
\end{lstlisting}

\lstinputlisting[caption=\Optimizing GCC 4.4.5 (MIPS) (IDA)]{patterns/14_bitfields/4_MIPS_O3_IDA.lst}

\Answer{}: \myref{exercise_solutions_bitfields_4}.

}
\DE{\mysection{\BitfieldsChapter}
\label{sec:bitfields}
Eine Menge Funktionen definiert ihre Eingabeargumente als Flags in Bitfields.

\myindex{\CLanguageElements!C99!bool}
Natürlich können diese auch durch Variablen von Typ \Tbool ersetzt werden; das
wäre jedoch umständlicher als nötig.

% sections
\subsection{\RU{Проверка какого-либо бита}\EN{Specific bit checking}}

\EN{\input{patterns/14_bitfields/1_check/x86_EN}}
\RU{\input{patterns/14_bitfields/1_check/x86_RU}}
\EN{\input{patterns/14_bitfields/1_check/ARM_EN}}
\RU{\input{patterns/14_bitfields/1_check/ARM_RU}}


\subsection{\RU{Установка и сброс отдельного бита}\EN{Setting and clearing specific bits}}

\RU{Например}\EN{For example}:

\lstinputlisting[style=customc]{patterns/14_bitfields/2_set_reset/set_reset.c}

\EN{\input{patterns/14_bitfields/2_set_reset/x86_EN}}
\RU{\input{patterns/14_bitfields/2_set_reset/x86_RU}}
\EN{\input{patterns/14_bitfields/2_set_reset/ARM_EN}}
\RU{\input{patterns/14_bitfields/2_set_reset/ARM_RU}}
\EN{\input{patterns/14_bitfields/2_set_reset/MIPS_EN}}
\RU{\input{patterns/14_bitfields/2_set_reset/MIPS_RU}}


\section{\ShiftsSectionName}

\RU{Битовые сдвиги в \CCpp реализованы при помощи операторов $\ll$ и $\gg$.}
\EN{Bit shifts in \CCpp are implemented via $\ll$ and $\gg$ operators.}

\RU{В x86 есть инструкции}\EN{x86 \ac{ISA} has} SHL (SHift Left) \AndENRU SHR (SHift Right) 
\RU{для этого}\EN{instructions for this}.

\subsection{\RU{Деление и умножение при помощи сдвигов}\EN{Division and multiplication using shifts}}
\label{subsec:mult_div_shifts}

\RU{Инструкции сдвига также активно применяются при делении или умножении 
на числа-степени двойки: $2^{n}$ (т.е., $1$, $2$, $4$, $8$, и т.д.).}
\EN{Shift instructions are often used in division and multiplications by power of two numbers:
$2^{n}$ (e.g., $1$, $2$, $4$, $8$, etc).}

\subsubsection{\RU{Умножение}\EN{Multiplication}}

\begin{lstlisting}
unsigned int f(unsigned int a)
{
	return a*4;
};
\end{lstlisting}

\begin{lstlisting}[caption=\NonOptimizing MSVC 2010]
_a$ = 8		; size = 4
_f	PROC
	push	ebp
	mov	ebp, esp
	mov	eax, DWORD PTR _a$[ebp]
	shl	eax, 2
	pop	ebp
	ret	0
_f	ENDP
\end{lstlisting}

\RU{Умножить на $4$ это просто сдвинуть число на 2 бита влево, 
вставив 2 нулевых бита справа (как два самых младших бита). 
Это как умножить $3$ на $100$ ~--- нужно просто дописать два нуля справа.}
\EN{Multiplication by $4$ is just shifting the number to the left by 2 bits,
while inserting 2 zero bits at right (as the last two bits).
It is just like to multiply $3$ by $100$~---we need just to add two zeroes at the right.}

\RU{Вот как работает инструкция сдвига влево}\EN{That's how shift left instruction works}:

\index{x86!\Instructions!SHL}
\input{shift_left}

\RU{Добавленные биты справа --- всегда нули}\EN{Added bits at right---always zeroes}.

\RU{Умножение на 4 в}\EN{Multiplication by 4 in} ARM:

\begin{lstlisting}[caption=\NonOptimizingKeilVI + \ARMMode]
f PROC
        LSL      r0,r0,#2
        BX       lr
        ENDP
\end{lstlisting}

\subsubsection{\RU{Деление}\EN{Division}}

\RU{Например}\EN{For example}:

\begin{lstlisting}
unsigned int f(unsigned int a)
{
	return a/4;
};
\end{lstlisting}

\RU{Имеем в итоге}\EN{We got} (MSVC 2010):

\begin{lstlisting}[caption=MSVC 2010]
_a$ = 8							; size = 4
_f	PROC
	mov	eax, DWORD PTR _a$[esp-4]
	shr	eax, 2
	ret	0
_f	ENDP
\end{lstlisting}

\label{SHR}
\index{x86!\Instructions!SHR}
\RU{Инструкция \SHR (\IT{SHift Right}) в данном примере сдвигает число на 2 бита вправо. 
При этом, освободившиеся два бита слева (т.е., самые 
старшие разряды), выставляются в нули. А самые младшие 2 бита выкидываются. 
Фактически, эти два выкинутых бита ~--- остаток от деления.}
\EN{\SHR (\IT{SHift Right}) instruction in this example is shifting a number by 2 bits right.
Two freed bits at left (e.g., two most significant bits) are set to zero.
Two least significant bits are dropped.
In fact, these two dropped bits~---division operation remainder.}

\index{x86!\Instructions!SHR}
\RU{Инструкция \SHR работает так же, как и \SHL, только в другую сторону.}
\EN{\SHR instruction works just like as \SHL but in other direction.}

\input{shift_right}

\label{division_by_shifting}
\RU{Для того, чтобы это проще понять, представьте себе десятичную систему счисления и число $23$. 
$23$ можно разделить на $10$ просто откинув последний разряд ($3$ ~--- это остаток от деления). 
После этой операции останется $2$ как \glslink{quotient}{частное}.}
\EN{It can be easily understood if to imagine decimal numeral system and number $23$.
$23$ can be easily divided by $10$ just by dropping last digit ($3$~---is division remainder). 
$2$ is leaving after operation as a \gls{quotient}.}

\RU{Деление на 4 в}\EN{Division by 4 in} ARM:

\begin{lstlisting}[caption=\NonOptimizingKeilVI + \ARMMode]
f PROC
        LSR      r0,r0,#2
        BX       lr
        ENDP
\end{lstlisting}

\EN{\input{patterns/14_bitfields/35_set_reset_FPU/main_EN}}
\RU{\input{patterns/14_bitfields/35_set_reset_FPU/main_RU}}

\EN{\input{patterns/14_bitfields/4_popcnt/main_EN}}
\RU{\input{patterns/14_bitfields/4_popcnt/main_RU}}


% TODO: add ROL/ROR
\subsection{\Conclusion{}}

\myindex{x86!\Instructions!SHR}
\myindex{x86!\Instructions!SHL}
\myindex{x86!\Instructions!SAR}
Analog zu den Schiebebefehlen \TT{$\ll$} und \TT{$\gg$} in \CCpp gibt es in
x86 die Befehle \SHR/\SHL (für vorzeichenlose Werte) und \SAR/\SHL (für
vorzeichenbehaftete Werte).

\myindex{ARM!\Instructions!LSR}
\myindex{ARM!\Instructions!LSL}
\myindex{ARM!\Instructions!ASR}
Die Schiebebefehle in ARM sind \LSR/\LSL (für vorzeichenlose Werte) und
\ASR/\LSL (für vorzeichenbehaftete Werte).

Es sind bei manchen Befehlen auch mögliche Suffixe für die Verschiebung
anzuhängen (diese heiße \q{data processing instructions}).
% FIXME: which instructions?

\subsubsection{Prüfung auf spezifisches Bit (zur Compilezeit bekannt)}
Prüfung, ob das Bit 0b10000000 (0x40) sich im Registerwert befindet:

\lstinputlisting[caption=\CCpp,style=customc]{patterns/14_bitfields/c_snippet0.c}

\lstinputlisting[caption=x86,style=customasmx86]{patterns/14_bitfields/TEST_JNZ_DE.lst}

\lstinputlisting[caption=x86,style=customasmx86]{patterns/14_bitfields/TEST_JZ_DE.lst}

\lstinputlisting[caption=ARM
(\ARMMode),style=customasmARM]{patterns/14_bitfields/TST_BNE_DE.lst}

\myindex{x86!\Instructions!AND}
\myindex{x86!\Instructions!TEST}
Manchmal wird \AND anstelle von \TEST verwendet, aber die gesetzten Flags sind
die gleichen.

\subsubsection{Prüfung auf spezifisches Bit (zur Laufzeit angegeben)}
Dies wird normalerweise durch den folgenden \CCpp Code gelöst (verschiebe Wert
um $n$ Bits nach rechts und schneide dann niederwertigstes Bit ab):

\lstinputlisting[caption=\CCpp,style=customc]{patterns/14_bitfields/c_snippet1.c}
In x86 Code wird dies gewöhnlich wie folgt implementiert:

\begin{lstlisting}[caption=x86,style=customasmx86]
; REG=input_value
; CL=n
SHR REG, CL
AND REG, 1
\end{lstlisting}
Eine andere Möglichkeit: (verschiebe 1 Bit $n$-mal nach links, isoliere dieses
Bit im Eingabewert und prüfe, ob es nicht 0 ist):

\lstinputlisting[caption=\CCpp,style=customc]{patterns/14_bitfields/c_snippet2.c}

In x86 Code wird dies gewöhnlich wie folgt implementiert:

\begin{lstlisting}[caption=x86,style=customasmx86]
; CL=n
MOV REG, 1
SHL REG, CL
AND input_value, REG
\end{lstlisting}

\subsubsection{Setzen eines spezifischen Bits (zur Compilerzeit bekannt)}

\begin{lstlisting}[caption=\CCpp]
value=value|0x40;
\end{lstlisting}

\begin{lstlisting}[caption=x86,style=customasmx86]
OR REG, 40h
\end{lstlisting}

\begin{lstlisting}[caption=ARM (\ARMMode) and ARM64,style=customasmARM]
ORR R0, R0, #0x40
\end{lstlisting}

\subsubsection{Setzen eines spezifischen Bits (zur Laufzeit angegeben)}

\lstinputlisting[caption=\CCpp,style=customc]{patterns/14_bitfields/c_snippet3.c}

In x86 Code wird dies gewöhnlich wie folgt implementiert:

\begin{lstlisting}[caption=x86,style=customasmx86]
; CL=n
MOV REG, 1
SHL REG, CL
OR input_value, REG
\end{lstlisting}

\subsubsection{Löschen eines spezifischen Bits (zur Compilezeit bekannt)}
Man verwendet einfach den \AND Befehl mit dem invertierten Wert:

\begin{lstlisting}[caption=\CCpp,style=customc]
value=value&(~0x40);
\end{lstlisting}

\begin{lstlisting}[caption=x86,style=customasmx86]
AND REG, 0FFFFFFBFh
\end{lstlisting}

\begin{lstlisting}[caption=x64,style=customasmx86]
AND REG, 0FFFFFFFFFFFFFFBFh
\end{lstlisting}

Dies sorgt dafür, dass alle Bits bis auf eines gesetzt werden.

\myindex{ARM!\Instructions!BIC}

ARM im ARM mode verfügt über den Befehl \BIC, der wie ein \NOT+\AND Befehlspaar
arbeitet:

\begin{lstlisting}[caption=ARM (\ARMMode),style=customasmARM]
BIC R0, R0, #0x40
\end{lstlisting}

\subsubsection{Löschen eines spezifischen Bits (zur Laufzeit angegeben)}

\lstinputlisting[caption=\CCpp,style=customc]{patterns/14_bitfields/c_snippet4.c}

\begin{lstlisting}[caption=x86,style=customasmx86]
; CL=n
MOV REG, 1
SHL REG, CL
NOT REG
AND input_value, REG
\end{lstlisting}

\section{\Exercises}

\subsection{\Exercise \#1}
\label{exercise_bitfields_1}

\WhatThisCodeDoes\

\begin{lstlisting}[caption=\Optimizing MSVC 2010]
_a$ = 8
_f	PROC
	mov	ecx, DWORD PTR _a$[esp-4]
	mov	eax, ecx
	mov	edx, ecx
	shl	edx, 16		; 00000010H
	and	eax, 65280	; 0000ff00H
	or	eax, edx
	mov	edx, ecx
	and	edx, 16711680	; 00ff0000H
	shr	ecx, 16		; 00000010H
	or	edx, ecx
	shl	eax, 8
	shr	edx, 8
	or	eax, edx
	ret	0
_f	ENDP
\end{lstlisting}

\begin{lstlisting}[caption=\OptimizingKeilVI (\ARMMode)]
f PROC
        MOV      r1,#0xff0000
        AND      r1,r1,r0,LSL #8
        MOV      r2,#0xff00
        ORR      r1,r1,r0,LSR #24
        AND      r2,r2,r0,LSR #8
        ORR      r1,r1,r2
        ORR      r0,r1,r0,LSL #24
        BX       lr
        ENDP
\end{lstlisting}

\begin{lstlisting}[caption=\OptimizingKeilVI (\ThumbMode)]
f PROC
        MOVS     r3,#0xff
        LSLS     r2,r0,#8
        LSLS     r3,r3,#16
        ANDS     r2,r2,r3
        LSRS     r1,r0,#24
        ORRS     r1,r1,r2
        LSRS     r2,r0,#8
        ASRS     r3,r3,#8
        ANDS     r2,r2,r3
        ORRS     r1,r1,r2
        LSLS     r0,r0,#24
        ORRS     r0,r0,r1
        BX       lr
        ENDP
\end{lstlisting}

\begin{lstlisting}[caption=\Optimizing GCC 4.9 (ARM64)]
f:
	rev	w0, w0
	ret
\end{lstlisting}

\lstinputlisting[caption=\Optimizing GCC 4.4.5 (MIPS) (IDA)]{patterns/14_bitfields/1_MIPS_O3_IDA.lst}

\Answer{}: \myref{exercise_solutions_bitfields_1}.

\subsection{\Exercise \#2}
\label{exercise_bitfields_2}

\WhatThisCodeDoes\

\begin{lstlisting}[caption=\Optimizing MSVC 2010]
_a$ = 8							; size = 4
_f	PROC
	push	esi
	mov	esi, DWORD PTR _a$[esp]
	xor	ecx, ecx
	push	edi
	lea	edx, DWORD PTR [ecx+1]
	xor	eax, eax
	npad	3 ; align next label
$LL3@f:
	mov	edi, esi
	shr	edi, cl
	add	ecx, 4
	and	edi, 15
	imul	edi, edx
	lea	edx, DWORD PTR [edx+edx*4]
	add	eax, edi
	add	edx, edx
	cmp	ecx, 28
	jle	SHORT $LL3@f
	pop	edi
	pop	esi
	ret	0
_f	ENDP
\end{lstlisting}

\begin{lstlisting}[caption=\OptimizingKeilVI (\ARMMode)]
f PROC
        MOV      r3,r0
        MOV      r1,#0
        MOV      r2,#1
        MOV      r0,r1
|L0.16|
        LSR      r12,r3,r1
        AND      r12,r12,#0xf
        MLA      r0,r12,r2,r0
        ADD      r1,r1,#4
        ADD      r2,r2,r2,LSL #2
        CMP      r1,#0x1c
        LSL      r2,r2,#1
        BLE      |L0.16|
        BX       lr
        ENDP
\end{lstlisting}

\begin{lstlisting}[caption=\OptimizingKeilVI (\ThumbMode)]
f PROC
        PUSH     {r4,lr}
        MOVS     r3,r0
        MOVS     r1,#0
        MOVS     r2,#1
        MOVS     r0,r1
|L0.10|
        MOVS     r4,r3
        LSRS     r4,r4,r1
        LSLS     r4,r4,#28
        LSRS     r4,r4,#28
        MULS     r4,r2,r4
        ADDS     r0,r4,r0
        MOVS     r4,#0xa
        MULS     r2,r4,r2
        ADDS     r1,r1,#4
        CMP      r1,#0x1c
        BLE      |L0.10|
        POP      {r4,pc}
        ENDP
\end{lstlisting}

\begin{lstlisting}[caption=\NonOptimizing GCC 4.9 (ARM64)]
f:
	sub	sp, sp, #32
	str	w0, [sp,12]
	str	wzr, [sp,28]
	mov	w0, 1
	str	w0, [sp,24]
	str	wzr, [sp,20]
	b	.L2
.L3:
	ldr	w0, [sp,28]
	ldr	w1, [sp,12]
	lsr	w0, w1, w0
	and	w1, w0, 15
	ldr	w0, [sp,24]
	mul	w0, w1, w0
	ldr	w1, [sp,20]
	add	w0, w1, w0
	str	w0, [sp,20]
	ldr	w0, [sp,28]
	add	w0, w0, 4
	str	w0, [sp,28]
	ldr	w1, [sp,24]
	mov	w0, w1
	lsl	w0, w0, 2
	add	w0, w0, w1
	lsl	w0, w0, 1
	str	w0, [sp,24]
.L2:
	ldr	w0, [sp,28]
	cmp	w0, 28
	ble	.L3
	ldr	w0, [sp,20]
	add	sp, sp, 32
	ret
\end{lstlisting}

\lstinputlisting[caption=\Optimizing GCC 4.4.5 (MIPS) (IDA)]{patterns/14_bitfields/2_MIPS_O3_IDA.lst}

\Answer{}: \myref{exercise_solutions_bitfields_2}.

\subsection{\Exercise \#3}
\label{exercise_bitfields_3}

\EN{Using the \ac{MSDN} documentation, find out which flags were used in the \TT{MessageBox()} win32 function call.}
\RU{Используя документацию \ac{MSDN}, найдите, какие флаги использовались в вызове win32-функции 
\TT{MessageBox()}.}

\begin{lstlisting}[caption=\Optimizing MSVC 2010]
_main	PROC
	push	278595		; 00044043H
	push	OFFSET $SG79792 ; 'caption'
	push	OFFSET $SG79793 ; 'hello, world!'
	push	0
	call	DWORD PTR __imp__MessageBoxA@16
	xor	eax, eax
	ret	0
_main	ENDP
\end{lstlisting}

\Answer{}: \myref{exercise_solutions_bitfields_3}.

\subsection{\Exercise \#4}
\label{exercise_bitfields_4}

\WhatThisCodeDoes\

\begin{lstlisting}[caption=\Optimizing MSVC 2010]
_m$ = 8		; size = 4
_n$ = 12	; size = 4
_f	PROC
	mov	ecx, DWORD PTR _n$[esp-4]
	xor	eax, eax
	xor	edx, edx
	test	ecx, ecx
	je	SHORT $LN2@f
	push	esi
	mov	esi, DWORD PTR _m$[esp]
$LL3@f:
	test	cl, 1
	je	SHORT $LN1@f
	add	eax, esi
	adc	edx, 0
$LN1@f:
	add	esi, esi
	shr	ecx, 1
	jne	SHORT $LL3@f
	pop	esi
$LN2@f:
	ret	0
_f	ENDP
\end{lstlisting}

\begin{lstlisting}[caption=\OptimizingKeilVI (\ARMMode)]
f PROC
        PUSH     {r4,lr}
        MOV      r3,r0
        MOV      r0,#0
        MOV      r2,r0
        MOV      r12,r0
        B        |L0.48|
|L0.24|
        TST      r1,#1
        BEQ      |L0.40|
        ADDS     r0,r0,r3
        ADC      r2,r2,r12
|L0.40|
        LSL      r3,r3,#1
        LSR      r1,r1,#1
|L0.48|
        CMP      r1,#0
        MOVEQ    r1,r2
        BNE      |L0.24|
        POP      {r4,pc}
        ENDP
\end{lstlisting}

\begin{lstlisting}[caption=\OptimizingKeilVI (\ThumbMode)]
f PROC
        PUSH     {r4,r5,lr}
        MOVS     r3,r0
        MOVS     r0,#0
        MOVS     r2,r0
        MOVS     r4,r0
        B        |L0.24|
|L0.12|
        LSLS     r5,r1,#31
        BEQ      |L0.20|
        ADDS     r0,r0,r3
        ADCS     r2,r2,r4
|L0.20|
        LSLS     r3,r3,#1
        LSRS     r1,r1,#1
|L0.24|
        CMP      r1,#0
        BNE      |L0.12|
        MOVS     r1,r2
        POP      {r4,r5,pc}
        ENDP
\end{lstlisting}

\begin{lstlisting}[caption=\Optimizing GCC 4.9 (ARM64)]
f:
	mov	w2, w0
	mov	x0, 0
	cbz	w1, .L2
.L3:
	and	w3, w1, 1
	lsr	w1, w1, 1
	cmp	w3, wzr
	add	x3, x0, x2, uxtw
	lsl	w2, w2, 1
	csel	x0, x3, x0, ne
	cbnz	w1, .L3
.L2:
	ret
\end{lstlisting}

\lstinputlisting[caption=\Optimizing GCC 4.4.5 (MIPS) (IDA)]{patterns/14_bitfields/4_MIPS_O3_IDA.lst}

\Answer{}: \myref{exercise_solutions_bitfields_4}.

}
\FR{\mysection{\BitfieldsChapter}
\label{sec:bitfields}

Beaucoup de fonctions définissent leurs arguments comme des flags dans un champ
de bits.
\myindex{\CLanguageElements!C99!bool}

Bien sûr, ils pourraient être substitués par un ensemble de variables de type \Tbool,
mais ce n'est pas frugal.

% sections
\subsection{\RU{Проверка какого-либо бита}\EN{Specific bit checking}}

\EN{\input{patterns/14_bitfields/1_check/x86_EN}}
\RU{\input{patterns/14_bitfields/1_check/x86_RU}}
\EN{\input{patterns/14_bitfields/1_check/ARM_EN}}
\RU{\input{patterns/14_bitfields/1_check/ARM_RU}}


\subsection{\RU{Установка и сброс отдельного бита}\EN{Setting and clearing specific bits}}

\RU{Например}\EN{For example}:

\lstinputlisting[style=customc]{patterns/14_bitfields/2_set_reset/set_reset.c}

\EN{\input{patterns/14_bitfields/2_set_reset/x86_EN}}
\RU{\input{patterns/14_bitfields/2_set_reset/x86_RU}}
\EN{\input{patterns/14_bitfields/2_set_reset/ARM_EN}}
\RU{\input{patterns/14_bitfields/2_set_reset/ARM_RU}}
\EN{\input{patterns/14_bitfields/2_set_reset/MIPS_EN}}
\RU{\input{patterns/14_bitfields/2_set_reset/MIPS_RU}}


\section{\ShiftsSectionName}

\RU{Битовые сдвиги в \CCpp реализованы при помощи операторов $\ll$ и $\gg$.}
\EN{Bit shifts in \CCpp are implemented via $\ll$ and $\gg$ operators.}

\RU{В x86 есть инструкции}\EN{x86 \ac{ISA} has} SHL (SHift Left) \AndENRU SHR (SHift Right) 
\RU{для этого}\EN{instructions for this}.

\subsection{\RU{Деление и умножение при помощи сдвигов}\EN{Division and multiplication using shifts}}
\label{subsec:mult_div_shifts}

\RU{Инструкции сдвига также активно применяются при делении или умножении 
на числа-степени двойки: $2^{n}$ (т.е., $1$, $2$, $4$, $8$, и т.д.).}
\EN{Shift instructions are often used in division and multiplications by power of two numbers:
$2^{n}$ (e.g., $1$, $2$, $4$, $8$, etc).}

\subsubsection{\RU{Умножение}\EN{Multiplication}}

\begin{lstlisting}
unsigned int f(unsigned int a)
{
	return a*4;
};
\end{lstlisting}

\begin{lstlisting}[caption=\NonOptimizing MSVC 2010]
_a$ = 8		; size = 4
_f	PROC
	push	ebp
	mov	ebp, esp
	mov	eax, DWORD PTR _a$[ebp]
	shl	eax, 2
	pop	ebp
	ret	0
_f	ENDP
\end{lstlisting}

\RU{Умножить на $4$ это просто сдвинуть число на 2 бита влево, 
вставив 2 нулевых бита справа (как два самых младших бита). 
Это как умножить $3$ на $100$ ~--- нужно просто дописать два нуля справа.}
\EN{Multiplication by $4$ is just shifting the number to the left by 2 bits,
while inserting 2 zero bits at right (as the last two bits).
It is just like to multiply $3$ by $100$~---we need just to add two zeroes at the right.}

\RU{Вот как работает инструкция сдвига влево}\EN{That's how shift left instruction works}:

\index{x86!\Instructions!SHL}
\input{shift_left}

\RU{Добавленные биты справа --- всегда нули}\EN{Added bits at right---always zeroes}.

\RU{Умножение на 4 в}\EN{Multiplication by 4 in} ARM:

\begin{lstlisting}[caption=\NonOptimizingKeilVI + \ARMMode]
f PROC
        LSL      r0,r0,#2
        BX       lr
        ENDP
\end{lstlisting}

\subsubsection{\RU{Деление}\EN{Division}}

\RU{Например}\EN{For example}:

\begin{lstlisting}
unsigned int f(unsigned int a)
{
	return a/4;
};
\end{lstlisting}

\RU{Имеем в итоге}\EN{We got} (MSVC 2010):

\begin{lstlisting}[caption=MSVC 2010]
_a$ = 8							; size = 4
_f	PROC
	mov	eax, DWORD PTR _a$[esp-4]
	shr	eax, 2
	ret	0
_f	ENDP
\end{lstlisting}

\label{SHR}
\index{x86!\Instructions!SHR}
\RU{Инструкция \SHR (\IT{SHift Right}) в данном примере сдвигает число на 2 бита вправо. 
При этом, освободившиеся два бита слева (т.е., самые 
старшие разряды), выставляются в нули. А самые младшие 2 бита выкидываются. 
Фактически, эти два выкинутых бита ~--- остаток от деления.}
\EN{\SHR (\IT{SHift Right}) instruction in this example is shifting a number by 2 bits right.
Two freed bits at left (e.g., two most significant bits) are set to zero.
Two least significant bits are dropped.
In fact, these two dropped bits~---division operation remainder.}

\index{x86!\Instructions!SHR}
\RU{Инструкция \SHR работает так же, как и \SHL, только в другую сторону.}
\EN{\SHR instruction works just like as \SHL but in other direction.}

\input{shift_right}

\label{division_by_shifting}
\RU{Для того, чтобы это проще понять, представьте себе десятичную систему счисления и число $23$. 
$23$ можно разделить на $10$ просто откинув последний разряд ($3$ ~--- это остаток от деления). 
После этой операции останется $2$ как \glslink{quotient}{частное}.}
\EN{It can be easily understood if to imagine decimal numeral system and number $23$.
$23$ can be easily divided by $10$ just by dropping last digit ($3$~---is division remainder). 
$2$ is leaving after operation as a \gls{quotient}.}

\RU{Деление на 4 в}\EN{Division by 4 in} ARM:

\begin{lstlisting}[caption=\NonOptimizingKeilVI + \ARMMode]
f PROC
        LSR      r0,r0,#2
        BX       lr
        ENDP
\end{lstlisting}

\EN{\input{patterns/14_bitfields/35_set_reset_FPU/main_EN}}
\RU{\input{patterns/14_bitfields/35_set_reset_FPU/main_RU}}

\EN{\input{patterns/14_bitfields/4_popcnt/main_EN}}
\RU{\input{patterns/14_bitfields/4_popcnt/main_RU}}


% TODO: add ROL/ROR
\subsection{\Conclusion{}}

\myindex{x86!\Instructions!SHR}
\myindex{x86!\Instructions!SHL}
\myindex{x86!\Instructions!SAR}

Semblables aux opérateurs de décalage de \CCpp \TT{$\ll$} et \TT{$\gg$}, les instructions
de décalage en x86 sont \SHR/\SHL (pour les valeurs non-signées) et \SAR/\SHL (pour
les valeurs signées).

\myindex{ARM!\Instructions!LSR}
\myindex{ARM!\Instructions!LSL}
\myindex{ARM!\Instructions!ASR}

Les instructions de décalages en ARM sont \LSR/\LSL (pour les valeurs non-signées)
et \ASR/\LSL (pour les valeurs signées).

Il est aussi possible d'ajouter un suffixe de décalage à certaines instructions (qui
sont appelées \q{data processing instructions/instructions de traitement de données}).
% FIXME: which instructions?

\subsubsection{Tester un bit spécifique (connu à l'étape de compilation)}

Tester si le bit 0b1000000 (0x40) est présent dans la valeur du registre:

\lstinputlisting[caption=\CCpp,style=customc]{patterns/14_bitfields/c_snippet0.c}

\lstinputlisting[caption=x86,style=customasmx86]{patterns/14_bitfields/TEST_JNZ_FR.lst}

\lstinputlisting[caption=x86,style=customasmx86]{patterns/14_bitfields/TEST_JZ_FR.lst}

\lstinputlisting[caption=ARM (\ARMMode),style=customasmARM]{patterns/14_bitfields/TST_BNE_FR.lst}

\myindex{x86!\Instructions!AND}
\myindex{x86!\Instructions!TEST}

Parfois, \AND est utilisé au lieu de \TEST, mais les flags qui sont mis sont les
même.

\subsubsection{Tester un bit spécifique (spécifié lors de l'exécution)}

Ceci est effectué en général par ce bout de code \CCpp (décaler la valeur de $n$
bits vers la droite, puis couper le plus petit bit):

\lstinputlisting[caption=\CCpp,style=customc]{patterns/14_bitfields/c_snippet1.c}

Ceci est en général implémenté en code x86 avec:

\begin{lstlisting}[caption=x86,style=customasmx86]
; REG=input_value
; CL=n
SHR REG, CL
AND REG, 1
\end{lstlisting}

Ou (décaler 1 bit $n$ fois à gauche, isoler ce bit dans la valeur entrée et tester
si ce n'est pas zéro):

\lstinputlisting[caption=\CCpp,style=customc]{patterns/14_bitfields/c_snippet2.c}

Ceci est en général implémenté en code x86 avec:

\begin{lstlisting}[caption=x86,style=customasmx86]
; CL=n
MOV REG, 1
SHL REG, CL
AND input_value, REG
\end{lstlisting}

\subsubsection{Mettre à 1 un bit spécifique (connu à l'étape de compilation)}

\begin{lstlisting}[caption=\CCpp]
value=value|0x40;
\end{lstlisting}

\begin{lstlisting}[caption=x86,style=customasmx86]
OR REG, 40h
\end{lstlisting}

\begin{lstlisting}[caption=ARM (\ARMMode) and ARM64,style=customasmARM]
ORR R0, R0, #0x40
\end{lstlisting}

\subsubsection{Mettre à 1 un bit spécifique (spécifié lors de l'exécution)}

\lstinputlisting[caption=\CCpp,style=customc]{patterns/14_bitfields/c_snippet3.c}

Ceci est en général implémenté en code x86 avec:

\begin{lstlisting}[caption=x86,style=customasmx86]
; CL=n
MOV REG, 1
SHL REG, CL
OR input_value, REG
\end{lstlisting}

\subsubsection{Mettre à 0 un bit spécifique (connu à l'étape de compilation)}

Il suffit d'effectuer l'opération \AND sur la valeur inversée:

\begin{lstlisting}[caption=\CCpp,style=customc]
value=value&(~0x40);
\end{lstlisting}

\begin{lstlisting}[caption=x86,style=customasmx86]
AND REG, 0FFFFFFBFh
\end{lstlisting}

\begin{lstlisting}[caption=x64,style=customasmx86]
AND REG, 0FFFFFFFFFFFFFFBFh
\end{lstlisting}

Ceci laisse tous les bits qui sont à 1 inchangés excepté un.

\myindex{ARM!\Instructions!BIC}

ARM en mode ARM a l'instruction \BIC, qui fonctionne comme la paire d'instructions:
\NOT+\AND:

\begin{lstlisting}[caption=ARM (\ARMMode),style=customasmARM]
BIC R0, R0, #0x40
\end{lstlisting}

\subsubsection{
Mettre à 0 un bit spécifique (spécifié lors de l'exécution)}

\lstinputlisting[caption=\CCpp,style=customc]{patterns/14_bitfields/c_snippet4.c}

\begin{lstlisting}[caption=x86,style=customasmx86]
; CL=n
MOV REG, 1
SHL REG, CL
NOT REG
AND input_value, REG
\end{lstlisting}

\section{\Exercises}

\subsection{\Exercise \#1}
\label{exercise_bitfields_1}

\WhatThisCodeDoes\

\begin{lstlisting}[caption=\Optimizing MSVC 2010]
_a$ = 8
_f	PROC
	mov	ecx, DWORD PTR _a$[esp-4]
	mov	eax, ecx
	mov	edx, ecx
	shl	edx, 16		; 00000010H
	and	eax, 65280	; 0000ff00H
	or	eax, edx
	mov	edx, ecx
	and	edx, 16711680	; 00ff0000H
	shr	ecx, 16		; 00000010H
	or	edx, ecx
	shl	eax, 8
	shr	edx, 8
	or	eax, edx
	ret	0
_f	ENDP
\end{lstlisting}

\begin{lstlisting}[caption=\OptimizingKeilVI (\ARMMode)]
f PROC
        MOV      r1,#0xff0000
        AND      r1,r1,r0,LSL #8
        MOV      r2,#0xff00
        ORR      r1,r1,r0,LSR #24
        AND      r2,r2,r0,LSR #8
        ORR      r1,r1,r2
        ORR      r0,r1,r0,LSL #24
        BX       lr
        ENDP
\end{lstlisting}

\begin{lstlisting}[caption=\OptimizingKeilVI (\ThumbMode)]
f PROC
        MOVS     r3,#0xff
        LSLS     r2,r0,#8
        LSLS     r3,r3,#16
        ANDS     r2,r2,r3
        LSRS     r1,r0,#24
        ORRS     r1,r1,r2
        LSRS     r2,r0,#8
        ASRS     r3,r3,#8
        ANDS     r2,r2,r3
        ORRS     r1,r1,r2
        LSLS     r0,r0,#24
        ORRS     r0,r0,r1
        BX       lr
        ENDP
\end{lstlisting}

\begin{lstlisting}[caption=\Optimizing GCC 4.9 (ARM64)]
f:
	rev	w0, w0
	ret
\end{lstlisting}

\lstinputlisting[caption=\Optimizing GCC 4.4.5 (MIPS) (IDA)]{patterns/14_bitfields/1_MIPS_O3_IDA.lst}

\Answer{}: \myref{exercise_solutions_bitfields_1}.

\subsection{\Exercise \#2}
\label{exercise_bitfields_2}

\WhatThisCodeDoes\

\begin{lstlisting}[caption=\Optimizing MSVC 2010]
_a$ = 8							; size = 4
_f	PROC
	push	esi
	mov	esi, DWORD PTR _a$[esp]
	xor	ecx, ecx
	push	edi
	lea	edx, DWORD PTR [ecx+1]
	xor	eax, eax
	npad	3 ; align next label
$LL3@f:
	mov	edi, esi
	shr	edi, cl
	add	ecx, 4
	and	edi, 15
	imul	edi, edx
	lea	edx, DWORD PTR [edx+edx*4]
	add	eax, edi
	add	edx, edx
	cmp	ecx, 28
	jle	SHORT $LL3@f
	pop	edi
	pop	esi
	ret	0
_f	ENDP
\end{lstlisting}

\begin{lstlisting}[caption=\OptimizingKeilVI (\ARMMode)]
f PROC
        MOV      r3,r0
        MOV      r1,#0
        MOV      r2,#1
        MOV      r0,r1
|L0.16|
        LSR      r12,r3,r1
        AND      r12,r12,#0xf
        MLA      r0,r12,r2,r0
        ADD      r1,r1,#4
        ADD      r2,r2,r2,LSL #2
        CMP      r1,#0x1c
        LSL      r2,r2,#1
        BLE      |L0.16|
        BX       lr
        ENDP
\end{lstlisting}

\begin{lstlisting}[caption=\OptimizingKeilVI (\ThumbMode)]
f PROC
        PUSH     {r4,lr}
        MOVS     r3,r0
        MOVS     r1,#0
        MOVS     r2,#1
        MOVS     r0,r1
|L0.10|
        MOVS     r4,r3
        LSRS     r4,r4,r1
        LSLS     r4,r4,#28
        LSRS     r4,r4,#28
        MULS     r4,r2,r4
        ADDS     r0,r4,r0
        MOVS     r4,#0xa
        MULS     r2,r4,r2
        ADDS     r1,r1,#4
        CMP      r1,#0x1c
        BLE      |L0.10|
        POP      {r4,pc}
        ENDP
\end{lstlisting}

\begin{lstlisting}[caption=\NonOptimizing GCC 4.9 (ARM64)]
f:
	sub	sp, sp, #32
	str	w0, [sp,12]
	str	wzr, [sp,28]
	mov	w0, 1
	str	w0, [sp,24]
	str	wzr, [sp,20]
	b	.L2
.L3:
	ldr	w0, [sp,28]
	ldr	w1, [sp,12]
	lsr	w0, w1, w0
	and	w1, w0, 15
	ldr	w0, [sp,24]
	mul	w0, w1, w0
	ldr	w1, [sp,20]
	add	w0, w1, w0
	str	w0, [sp,20]
	ldr	w0, [sp,28]
	add	w0, w0, 4
	str	w0, [sp,28]
	ldr	w1, [sp,24]
	mov	w0, w1
	lsl	w0, w0, 2
	add	w0, w0, w1
	lsl	w0, w0, 1
	str	w0, [sp,24]
.L2:
	ldr	w0, [sp,28]
	cmp	w0, 28
	ble	.L3
	ldr	w0, [sp,20]
	add	sp, sp, 32
	ret
\end{lstlisting}

\lstinputlisting[caption=\Optimizing GCC 4.4.5 (MIPS) (IDA)]{patterns/14_bitfields/2_MIPS_O3_IDA.lst}

\Answer{}: \myref{exercise_solutions_bitfields_2}.

\subsection{\Exercise \#3}
\label{exercise_bitfields_3}

\EN{Using the \ac{MSDN} documentation, find out which flags were used in the \TT{MessageBox()} win32 function call.}
\RU{Используя документацию \ac{MSDN}, найдите, какие флаги использовались в вызове win32-функции 
\TT{MessageBox()}.}

\begin{lstlisting}[caption=\Optimizing MSVC 2010]
_main	PROC
	push	278595		; 00044043H
	push	OFFSET $SG79792 ; 'caption'
	push	OFFSET $SG79793 ; 'hello, world!'
	push	0
	call	DWORD PTR __imp__MessageBoxA@16
	xor	eax, eax
	ret	0
_main	ENDP
\end{lstlisting}

\Answer{}: \myref{exercise_solutions_bitfields_3}.

\subsection{\Exercise \#4}
\label{exercise_bitfields_4}

\WhatThisCodeDoes\

\begin{lstlisting}[caption=\Optimizing MSVC 2010]
_m$ = 8		; size = 4
_n$ = 12	; size = 4
_f	PROC
	mov	ecx, DWORD PTR _n$[esp-4]
	xor	eax, eax
	xor	edx, edx
	test	ecx, ecx
	je	SHORT $LN2@f
	push	esi
	mov	esi, DWORD PTR _m$[esp]
$LL3@f:
	test	cl, 1
	je	SHORT $LN1@f
	add	eax, esi
	adc	edx, 0
$LN1@f:
	add	esi, esi
	shr	ecx, 1
	jne	SHORT $LL3@f
	pop	esi
$LN2@f:
	ret	0
_f	ENDP
\end{lstlisting}

\begin{lstlisting}[caption=\OptimizingKeilVI (\ARMMode)]
f PROC
        PUSH     {r4,lr}
        MOV      r3,r0
        MOV      r0,#0
        MOV      r2,r0
        MOV      r12,r0
        B        |L0.48|
|L0.24|
        TST      r1,#1
        BEQ      |L0.40|
        ADDS     r0,r0,r3
        ADC      r2,r2,r12
|L0.40|
        LSL      r3,r3,#1
        LSR      r1,r1,#1
|L0.48|
        CMP      r1,#0
        MOVEQ    r1,r2
        BNE      |L0.24|
        POP      {r4,pc}
        ENDP
\end{lstlisting}

\begin{lstlisting}[caption=\OptimizingKeilVI (\ThumbMode)]
f PROC
        PUSH     {r4,r5,lr}
        MOVS     r3,r0
        MOVS     r0,#0
        MOVS     r2,r0
        MOVS     r4,r0
        B        |L0.24|
|L0.12|
        LSLS     r5,r1,#31
        BEQ      |L0.20|
        ADDS     r0,r0,r3
        ADCS     r2,r2,r4
|L0.20|
        LSLS     r3,r3,#1
        LSRS     r1,r1,#1
|L0.24|
        CMP      r1,#0
        BNE      |L0.12|
        MOVS     r1,r2
        POP      {r4,r5,pc}
        ENDP
\end{lstlisting}

\begin{lstlisting}[caption=\Optimizing GCC 4.9 (ARM64)]
f:
	mov	w2, w0
	mov	x0, 0
	cbz	w1, .L2
.L3:
	and	w3, w1, 1
	lsr	w1, w1, 1
	cmp	w3, wzr
	add	x3, x0, x2, uxtw
	lsl	w2, w2, 1
	csel	x0, x3, x0, ne
	cbnz	w1, .L3
.L2:
	ret
\end{lstlisting}

\lstinputlisting[caption=\Optimizing GCC 4.4.5 (MIPS) (IDA)]{patterns/14_bitfields/4_MIPS_O3_IDA.lst}

\Answer{}: \myref{exercise_solutions_bitfields_4}.

}


\section{\IFRU{Структуры}{Structures}}

\IFRU{В принципе, структура в \CCpp это, с некоторыми допущениями, просто всегда лежащий рядом, 
и в той же последовательности, набор переменных, не обязательно одного типа
\footnote{\ac{AKA} ``гетерогенный контейнер''}.}
{It can be defined the \CCpp structure, with some assumptions, just a set of variables, always stored
in memory together, not necessary of the same type
\footnote{\ac{AKA} ``heterogeneous container''}.}

\section{\RU{Пример SYSTEMTIME}\EN{SYSTEMTIME example}}

\newcommand{\FNSYSTEMTIME}{\footnote{\href{http://msdn.microsoft.com/en-us/library/ms724950(VS.85).aspx}{MSDN: SYSTEMTIME structure}}}

\RU{Возьмем, к примеру, структуру SYSTEMTIME\FNSYSTEMTIME{} из win32 описывающую время.}
\EN{Let's take SYSTEMTIME\FNSYSTEMTIME{} win32 structure describing time.}

\RU{Она объявлена так:}\EN{That's how it is defined:}

\begin{lstlisting}[caption=WinBase.h]
typedef struct _SYSTEMTIME {
  WORD wYear;
  WORD wMonth;
  WORD wDayOfWeek;
  WORD wDay;
  WORD wHour;
  WORD wMinute;
  WORD wSecond;
  WORD wMilliseconds;
} SYSTEMTIME, *PSYSTEMTIME;
\end{lstlisting}

\RU{Пишем на Си функцию для получения текущего системного времени:}
\EN{Let's write a C function to get current time:}

\lstinputlisting{patterns/15_structs/systemtime.c}

\RU{Что в итоге}\EN{We got} (MSVC 2010):

\lstinputlisting[caption=MSVC 2010]{patterns/15_structs/systemtime.asm}

\RU{Под структуру в стеке выделено 16 байт ~--- именно столько будет \TT{sizeof(WORD)*8}
(в структуре 8 переменных с типом WORD).}
\EN{16 bytes are allocated for this structure in local stack~---that is exactly \TT{sizeof(WORD)*8}
(there are 8 WORD variables in the structure).}

\newcommand{\FNMSDNGST}{\footnote{\href{http://msdn.microsoft.com/en-us/library/ms724390(VS.85).aspx}{MSDN: GetSystemTime function}}}

\RU{Обратите внимание на тот факт, что структура начинается с поля \TT{wYear}. 
Можно сказать, что в качестве аргумента для \TT{GetSystemTime()}\FNMSDNGST передается указатель на структуру 
SYSTEMTIME, но можно также сказать, что передается указатель на поле \TT{wYear}, 
что одно и тоже! 
\TT{GetSystemTime()} пишет текущий год в тот WORD на который указывает переданный указатель, 
затем сдвигается на 2 байта вправо, пишет текущий месяц, и т.д., и т.д.}
\EN{Pay attention to the fact the structure beginning with \TT{wYear} field.
It can be said, an pointer to SYSTEMTIME structure is passed to the \TT{GetSystemTime()}\FNSYSTEMTIME,
but it is also can be said, pointer to the \TT{wYear} field is passed, and that is the same!
\TT{GetSystemTime()} writes current year to the WORD pointer pointing to, then shifts 2 bytes
ahead, then writes current month, etc, etc.}

\RU{Тот факт, что поля структуры это просто переменные расположенные рядом, 
я могу проиллюстрировать следующим образом.}
\EN{The fact the structure fields are just variables located side-by-side, 
I can demonstrate by the following technique.}
\RU{Глядя на описание структуры}\EN{Keeping in ming} \TT{SYSTEMTIME}\RU{, я могу переписать этот простой пример так:}
\EN{ structure description, I can rewrite this simple example like this:}

\lstinputlisting{patterns/15_structs/systemtime2.c}

\RU{Компилятор немного поворчит:}\EN{Compiler will grumble for a little:}

\begin{lstlisting}
systemtime2.c(7) : warning C4133: 'function' : incompatible types - from 'WORD [8]' to 'LPSYSTEMTIME'
\end{lstlisting}

\RU{Тем не менее, выдаст такой код}\EN{But nevertheless, it will produce this code}:

\lstinputlisting[caption=MSVC 2010]{patterns/15_structs/systemtime2.asm}

\RU{И это работает так же}\EN{And it works just as the same}!

\RU{Любопытно что результат на ассемблере неотличим от предыдущего}
\EN{It is very interesting fact the
result in assembly form cannot be distinguished from the result of previous compilation}.
\RU{Таким образом, глядя на этот код, 
никогда нельзя сказать с уверенностью, была ли там объявлена структура, либо просто набор переменных.}
\EN{So by looking at this code, one cannot say for sure, was there structure declared, or just pack of variables.} 

\RU{Тем не менее, никто в здравом уме делать так не будет}\EN{Nevertheless, no one will do it in sane state of mind}.
\RU{Потому что это неудобно}\EN{Since it is not convenient}. 
\RU{К тому же, иногда, поля в структуре могут меняться разработчиками, 
переставляться местами, и т.д}\EN{Also structure fields may be changed by developers, swapped, etc}.


\subsection{\IFRU{Выделяем место для структуры через malloc()}{Let's allocate place for structure using malloc()}}

\IFRU{Однако, бывает и так, что проще хранить структуры не в стеке а в куче\footnote{heap}:}
{However, sometimes it's simpler to place structures not in local stack, but in heap:}

\lstinputlisting{15_structs/systemtime_malloc.c}

\IFRU{Скомпилируем на этот раз с оптимизацией (\Ox) чтобы было проще увидеть то, что нам нужно.}
{Let's compile it now with optimization (\Ox) so to easily see what we need.}

\lstinputlisting[caption=\Optimizing MSVC]{15_structs/systemtime_malloc.asm}

\index{\CLanguageElements!malloc()}
\IFRU{Итак, \TT{sizeof(SYSTEMTIME) = 16}, именно столько байт выделяется при помощи \TT{malloc()}. 
Она возвращает указатель на только что выделенный блок памяти в \EAX, который копируется в \ESI. 
Win32 функция \TT{GetSystemTime()} обязуется сохранить состояние \ESI, 
поэтому здесь оно нигде не сохраняется и продолжает использоваться после вызова \TT{GetSystemTime()}.}
{So, \TT{sizeof(SYSTEMTIME) = 16}, that's exact number of bytes to be allocated by \TT{malloc()}.
It return the pointer to freshly allocated memory block in \EAX, which is then moved into \ESI.
\TT{GetSystemTime()} win32 function undertake to save \ESI value, 
and that's why it is not saved here and continue to be used after \TT{GetSystemTime()} call.}

\index{x86!\Instructions!MOVZX}
\IFRU{
Новая инструкция ~--- \MOVZX (\IT{Move with Zero eXtent}). 
Она нужна почти там же где и \MOVSX, 
только всегда очищает остальные биты в $0$. Дело в том что \printf требует 32-битный тип \Tint, 
а в структуре лежит WORD ~--- это 16-битный беззнаковый тип. Поэтому копируя значение из WORD в \Tint, 
нужно очистить биты от 16 до 31, иначе там будет просто случайный мусор, оставшийся от предыдущих действий 
с регистрами.}
{New instruction ~--- \MOVZX (\IT{Move with Zero eXtent}).
It may be used almost in those cases as \MOVSX, but, it clearing other bits to $0$.
That's because \printf require 32-bit \Tint, but we got WORD in structure ~--- that's 16-bit unsigned type.
That's why by copying value from WORD into \Tint{}, bits from 16 to 31 should be cleared, 
because there will be random noise otherwise, leaved from previous operations on registers.}

\IFRU{В этом примере я тоже могу представить структуру как массив WORD-ов}{In this example, I can represent
structure as array of WORD-s}:

\lstinputlisting{15_structs/systemtime_malloc2.c}

\IFRU{Получим такое}{We got}:

\lstinputlisting[caption=\Optimizing MSVC]{15_structs/systemtime_malloc2.asm}

\IFRU{И снова мы получаем идетичный код, неотличимый от предыдущего}{Again, we got a code that cannot be distinguished
from previous}.
\IFRU{Но и снова я должен отметить, что в реальности так лучше не делать}{And again I should note, one shouldn't do
this in practice}.


\subsection{struct tm}

\subsubsection{Linux}

\IFRU{В Линуксе, для примера, возьем структуру \TT{tm} из \TT{time.h}:}
{As of Linux, let's take \TT{tm} structure from \TT{time.h} for example:}

\lstinputlisting{15_structs/GCC_tm.c}

\IFRU{Компилируем при помощи}{Let's compile it in} GCC 4.4.1:

\IFRU{\lstinputlisting[caption=GCC 4.4.1]{15_structs/GCC_tm_ru.asm}}{\lstinputlisting{15_structs/GCC_tm_en.asm}}

\IFRU{К сожалению, по какой-то причине, \IDA не сформировала названия локальных переменных в стеке. 
Но так как мы уже опытные реверсеры :-) то можем обойтись и без этого в таком простом примере.}
{Somehow, \IDA didn't created local variables names in local stack.
But since we already experienced reverse engineers :-) we may do it without this information in 
this simple example.}

\IFRU{Обратите внимание на \TT{lea edx, [eax+76Ch]} ~--- эта инструкция прибавляет $0x76C$ к \EAX, 
но не модифицирует флаги. См. также соответствующий раздел об инструкции \LEA{}~\ref{sec:LEA}.}
{Please also pay attention to \TT{lea edx, [eax+76Ch]} ~--- this instruction just adding $0x76C$ to \EAX,
but not modify any flags. See also relevant section about \LEA{}~\ref{sec:LEA}.}

Чтобы проиллюстрировать то что структура это просто набор переменных лежащих в одном месте, переделаем немного
пример, заглянув предварительно в файл time.h:

\begin{lstlisting}[caption=time.h]
struct tm
{
  int	tm_sec;
  int	tm_min;
  int	tm_hour;
  int	tm_mday;
  int	tm_mon;
  int	tm_year;
  int	tm_wday;
  int	tm_yday;
  int	tm_isdst;
};
\end{lstlisting}

\lstinputlisting{15_structs/GCC_tm2.c}

Обратите внимание на то что в \TT{localtime\_r} передается указатель именно на \TT{tm\_sec}, 
т.е., на первый элемент ``структуры''.

В итоге, и этот компилятор поворчит:

\begin{lstlisting}[caption=GCC 4.7.3]
GCC_tm2.c: In function 'main':
GCC_tm2.c:11:5: warning: passing argument 2 of 'localtime_r' from incompatible pointer type [enabled by default]
In file included from GCC_tm2.c:2:0:
/usr/include/time.h:59:12: note: expected 'struct tm *' but argument is of type 'int *'
\end{lstlisting}

Тем не менее, сгенерирует такоу:

\lstinputlisting[caption=GCC 4.7.3]{15_structs/GCC_tm2.asm}

Этот код почти идентичен уже рассмотренному, и нельзя сказать, была ли структура
в оригинальном исходном коде либо набор переменных.

И это работает. Однако, в реальности так лучше не делать. Обычно, компилятор располагает переменные в локальном
стеке в том же порядке, в котором они объявляются в функции. Тем не менее, никакой гарантии нет.

Кстати, какой-нибудь другой компилятор может предупредить, что переменные \TT{tm\_year}, \TT{tm\_mon}, \TT{tm\_mday},
\TT{tm\_hour}, \TT{tm\_min}, но не \TT{tm\_sec}, используются без инициализации. 
Действительно, ведь компилятор не знает
что они будут заполнены при вызове функции \TT{localtime\_r()}.

Я выбрал именно этот пример для иллюстрации, потому что члены структуры имеют тип \Tint, а члены структуры
\TT{SYSTEMTIME} ~--- 16-битные \TT{WORD}, и если их объявлять так же, то они будут выровнены по 32-битной границе 
и ничего не выйдет (потому что \TT{GetSystemTime()} заполнит их неверно). Читайте об этом в следующей секции
``\StructurePackingSectionName''.

\index{\SyntacticSugar}
Так что, структура это просто набор переменных лежащих в одном месте, рядом. Я мог бы сказать что структура
это такой синтаксический сахар, заставляющий компилятор удерживать их в одном месте. Впрочем, я не специалист
по языкам программирования, так что, скорее всего, ошибаюсь с этим термином.
Кстати, когда-то, в очень ранних версиях Си (перед 1972) структур не 
было вовсе\cite{Ritchie:1993:DCL:155360.155580}.

\subsubsection{ARM + \OptimizingKeil + \ThumbMode}

Этот же пример:

\lstinputlisting[caption=\OptimizingKeil + \ThumbMode]{15_structs/tm_ARM_keil_thumb.asm}

\subsubsection{ARM + \OptimizingXcode + \ThumbTwoMode}

\IDA ``узнала'' структуру tm (потому что \IDA ``знает'' типы аргументов библиотечных функций, 
таких как \TT{localtime\_r()}), поэтому показала здесь обращения к элементам структуры.

\lstinputlisting[caption=\OptimizingXcode + \ThumbTwoMode]{15_structs/tm_ARM_xcode_thumb.asm}


\section{\StructurePackingSectionName}

\RU{Достаточно немаловажный момент, это упаковка полей в структурах\footnote{См. также: \URLWPDA}.}
\EN{One important thing is fields packing in structures\footnote{See also: \URLWPDA}.}

\RU{Возьмем простой пример:}\EN{Let's take a simple example:}

\lstinputlisting{patterns/15_structs/packing.c}

\RU{Как видно, мы имеем два поля \Tchar (занимающий один байт) и еще два ~--- \Tint (по 4 байта).}
\EN{As we see, we have two \Tchar fields (each is exactly one byte) and two more~---\Tint (each - 4 bytes).}

\subsection{x86}

\RU{Компилируется это все в:}\EN{That's all compiling into:}

\lstinputlisting{patterns/15_structs/packing.asm}

\RU{Мы видим здесь что адрес каждого поля в структуре выравнивается по 4-байтной границе. 
Так что каждый \Tchar здесь занимает те же 4 байта что и \Tint. Зачем? 
Затем что процессору удобнее обращаться по таким адресам и кэшировать данные из памяти.}
\EN{As we can see, each field's address is aligned on a 4-bytes border.
That's why each \Tchar occupies 4 bytes here (like \Tint). Why?
Thus it is easier for CPU to access memory at aligned addresses and to cache data from it.}

\RU{Но это не экономично по размеру данных.}\EN{However, it is not very economical in size sense.}

\RU{Попробуем скомпилировать тот же исходник с опцией}\EN{Let's try to compile it with option} (\TT{/Zp1}) 
(\IT{/Zp[n] pack structures on n-byte boundary}).

\lstinputlisting[caption=MSVC /Zp1]{patterns/15_structs/packing_msvc_Zp1.asm}

\RU{Теперь структура занимает 10 байт и все \Tchar занимают по байту. Что это дает? 
Экономию места. Недостаток ~--- процессор будет обращаться к этим полям не так эффективно 
по скорости, как мог бы.}
\EN{Now the structure takes only 10 bytes and each \Tchar value takes 1 byte. What it give to us?
Size economy. And as drawback~---CPU will access these fields without maximal performance it can.}

\RU{Как нетрудно догадаться, если структура используется много в каких исходниках и объектных файлах, 
все они должны быть откомпилированы с одним и тем же соглашением об упаковке структур.}
\EN{As it can be easily guessed, if the structure is used in many source and object files,
all these must be compiled with the same convention about structures packing.}

\newcommand{\FNURLMSDNZP}{\footnote{\href{http://msdn.microsoft.com/en-us/library/ms253935.aspx}
{MSDN: Working with Packing Structures}}}
\newcommand{\FNURLGCCPC}{\footnote{\href{http://gcc.gnu.org/onlinedocs/gcc/Structure_002dPacking-Pragmas.html}
{Structure-Packing Pragmas}}}

\RU{Помимо ключа MSVC \TT{/Zp}, указывающего, по какой границе упаковывать поля структур, есть также 
опция компилятора \TT{\#pragma pack}, её можно указывать прямо в исходнике. 
Это справедливо и для MSVC\FNURLMSDNZP и GCC\FNURLGCCPC{}.}
\EN{Aside from MSVC \TT{/Zp} option which set how to align each structure field, here is also
\TT{\#pragma pack} compiler option, it can be defined right in source code.
It is available in both MSVC\FNURLMSDNZP and GCC\FNURLGCCPC{}.}

\RU{Давайте теперь вернемся к \TT{SYSTEMTIME}, которая состоит из 16-битных полей. 
Откуда наш компилятор знает что их надо паковать по однобайтной границе?}
\EN{Let's back to the \TT{SYSTEMTIME} structure consisting in 16-bit fields.
How our compiler know to pack them on 1-byte alignment boundary?}

\RU{В файле \TT{WinNT.h} попадается такое:}\EN{\TT{WinNT.h} file has this:}

\begin{lstlisting}[caption=WinNT.h]
#include "pshpack1.h"
\end{lstlisting}

\RU{И такое:}\EN{And this:}

\begin{lstlisting}[caption=WinNT.h]
#include "pshpack4.h"                   // 4 byte packing is the default
\end{lstlisting}

\RU{Сам файл PshPack1.h выглядит так:}\EN{The file PshPack1.h looks like:}

\begin{lstlisting}[caption=PshPack1.h]
#if ! (defined(lint) || defined(RC_INVOKED))
#if ( _MSC_VER >= 800 && !defined(_M_I86)) || defined(_PUSHPOP_SUPPORTED)
#pragma warning(disable:4103)
#if !(defined( MIDL_PASS )) || defined( __midl )
#pragma pack(push,1)
#else
#pragma pack(1)
#endif
#else
#pragma pack(1)
#endif
#endif /* ! (defined(lint) || defined(RC_INVOKED)) */
\end{lstlisting}

\RU{Собственно, так и задается компилятору, как паковать объявленные после \TT{\#pragma pack} структуры.}
\EN{That's how compiler will pack structures defined after \TT{\#pragma pack}.}

\subsection{ARM + \OptimizingKeil + \ThumbMode}

\lstinputlisting[caption=\OptimizingKeil + \ThumbMode]{patterns/15_structs/packing_Keil_thumb.asm}

\RU{Как мы помним, здесь передается не указатель на структуру, а сама структура, а так как в ARM первые 4 аргумента
функции передаются через регистры, то поля структуры передаются через}
\EN{As we may recall, here a structure passed instead of pointer to structure,
and since first 4 function arguments in ARM are passed via registers,
so then structure fields are passed via} \TT{R0-R3}.

\index{ARM!\Instructions!LDRB}
\index{x86!\Instructions!MOVSX}
\RU{Инструкция }\TT{LDRB} \RU{загружает один байт из памяти и расширяет до 32-бит учитывая знак.}
\EN{loads one byte from memory and extending it to 32-bit, taking into account its sign.}
\RU{Это то же что и инструкция}\EN{This is akin to} \MOVSX \RU{в}\EN{instruction in} x86.
\RU{Она здесь применяется для загрузки полей}\EN{Here it is used for loading fields} $a$ \AndENRU $c$ 
\RU{из структуры}\EN{from structure}.

\index{Function epilogue}
\RU{Еще что бросается в глаза, так это то что вместо эпилога функции, переход на эпилог другой функции!}
\EN{One more thing we spot easily, instead of function epilogue, here is jump to another function's epilogue!}
\RU{Действительно, то была совсем другая, не относящаяся к этой, функция, однако, она имела точно такой же
эпилог}\EN{Indeed, that was quite different function, not related in any way to our function, however, it has exactly
the same epilogue} 
(\RU{видимо, тоже хранила в стеке 5 локальных переменных}\EN{probably because, it hold 5 local variables too} 
($5*4=0x14$)).
\RU{К тому же, она находится рядом (обратите внимание на адреса).}
\EN{Also it is located nearly (take a look on addresses).}
\RU{Действительно, нет никакой разницы, какой эпилог исполнять, если он работает так же, как нам нужно.}
\EN{Indeed, there is no difference, which epilogue to execute,
if it works just as we need.}
\RU{Keil решил использовать часть другой ф-ции, вероятно, из-за экономии.}
\EN{Apparently, Keil decides to reuse a part of another function by a reason of economy.}
\RU{Эпилог занимает 4 байта, а переход ~--- только 2.}
\EN{Epilogue takes 4 bytes while jump~---only 2.}

\subsection{ARM + \OptimizingXcode + \ThumbTwoMode}

\lstinputlisting[caption=\OptimizingXcode + \ThumbTwoMode]{patterns/15_structs/packing_Xcode_thumb.asm}

\index{ARM!\Instructions!SXTB}
\index{x86!\Instructions!MOVSX}
\TT{SXTB} (\IT{Signed Extend Byte}) \RU{это также аналог}\EN{is analogous to} \MOVSX \InENRU 
x86\RU{, только работает не с памятью, а с регистром.}\EN{ as well, but works not with memory, but with register.}
\RU{Всё остальное ~--- так же.}\EN{All the rest~---just the same.}


\subsection{\IFRU{Вложенные структуры}{Nested structures}}

\IFRU{Теперь, как насчет ситуаций, когда одна структура определяет внутри себя еще одну структуру?}
{Now what about situations when one structure defines another structure inside?}

\lstinputlisting{patterns/15_structs/nested.c}

\dots \IFRU{в этом случае, оба поля \TT{inner\_struct} просто будут располагаться между полями a,b и d,e в 
\TT{outer\_struct}.}
{in this case, both \TT{inner\_struct} fields will be placed between a,b and d,e fields of
\TT{outer\_struct}.}

\IFRU{Компилируем}{Let's compile} (MSVC 2010):

\lstinputlisting[caption=MSVC 2010]{patterns/15_structs/nested_msvc.asm}

\IFRU{Очень любопытный момент в том, что глядя на этот код на ассемблере, мы даже не видим, 
что была использована какая-то еще другая структура внутри этой!
Так что, пожалуй, можно сказать, что все вложенные структуры в итоге разворачиваются в одну, \IT{линейную} 
или \IT{одномерную} структуру.}
{One curious point here is that by looking onto this assembly code, we do not even see that
another structure was used inside of it!
Thus, we would say, nested structures are finally unfolds into \IT{linear} or \IT{one-dimensional} structure.}

\IFRU{Конечно, если заменить объявление \TT{struct inner\_struct c;} на \TT{struct inner\_struct *c;} 
(объявляя таким образом указатель), ситауция будет совсем иная.}
{Of course, if to replace \TT{struct inner\_struct c;} declaration to \TT{struct inner\_struct *c;} 
(thus making a pointer here) situation will be quite different.}


\subsection{\IFRU{Работа с битовыми полями в структуре}{Bit fields in structure}}

\subsubsection{\IFRU{Пример CPUID}{CPUID example}}

\IFRU{Язык \CCpp позволяет указывать, сколько именно бит отвести для каждого поля структуры. 
Это удобно если нужно экономить место в памяти. К примеру, для переменной типа \Tbool достаточно одного бита.
Но, это не очень удобно, если нужна скорость.}
{\CCpp language allow to define exact number of bits for each structure fields.
It is very useful if one needs to save memory space. 
For example, one bit is enough for variable of \Tbool type.
But of course, it is not rational if speed is important.}

\newcommand{\FNCPUID}{\footnote{\url{http://en.wikipedia.org/wiki/CPUID}}}

\index{x86!\Instructions!CPUID}
\label{cpuid}
\IFRU{Рассмотрим пример с инструкцией \CPUID\FNCPUID. 
Эта инструкция возвращает информацию о том, какой процессор имеется в наличии и какие возможности он имеет.}
{Let's consider \CPUID\FNCPUID instruction example.
This instruction returning information about current CPU and its features.}

\IFRU{Если перед исполнением инструкции в \EAX будет 1, 
то \CPUID вернет упакованную в \EAX такую информацию о процессоре:}
{If the \EAX is set to 1 before instruction execution, 
\CPUID will return this information packed into the \EAX register:}

\begin{center}
\begin{tabular}{ | l | l | }
\hline
3:0 & Stepping \\
7:4 & Model \\
11:8 & Family \\
13:12 & Processor Type \\
19:16 & Extended Model \\
27:20 & Extended Family \\
\hline
\end{tabular}
\end{center}

\newcommand{\FNGCCAS}{\footnote{\href{http://www.ibiblio.org/gferg/ldp/GCC-Inline-Assembly-HOWTO.html}
{\IFRU{Подробнее о встроенном ассемблере GCC}{More about internal GCC assembler}}}}

\IFRU{MSVC 2010 имеет макрос для \CPUID, а GCC 4.4.1 ~--- нет. 
Поэтому для GCC сделаем эту функцию сами, используя его встроенный ассемблер\FNGCCAS.}
{MSVC 2010 has \CPUID macro, but GCC 4.4.1~---has not.
So let's make this function by yourself for GCC with the help of its built-in assembler\FNGCCAS.}

\lstinputlisting{patterns/15_structs/CPUID.c}

\IFRU{После того как \CPUID заполнит \EAX/\EBX/\ECX/\EDX, у нас они отразятся в массиве \TT{b[]}. 
Затем, мы имеем указатель на структуру \TT{CPUID\_1\_EAX}, и мы указываем его на значение 
\EAX из массива \TT{b[]}.}
{After \CPUID will fill \EAX/\EBX/\ECX/\EDX, these registers will be reflected in the \TT{b[]} array.
Then, we have a pointer to the \TT{CPUID\_1\_EAX} structure and we point it to the value in the \EAX from \TT{b[]} array.}

\IFRU{Иными словами, мы трактуем 32-битный \Tint как структуру.}
{In other words, we treat 32-bit \Tint value as a structure.}

\IFRU{Затем мы читаем из структуры.}{Then we read from the stucture.}

\IFRU{Компилируем в MSVC 2008 с опцией \Ox}{Let's compile it in MSVC 2008 with \Ox option}:

\lstinputlisting[caption=\Optimizing MSVC 2008]{patterns/15_structs/CPUID_msvc_Ox.asm}

\index{x86!\Instructions!SHR}
\IFRU{Инструкция \TT{SHR} сдвигает значение из \EAX на то количество бит, 
которое нужно \IT{пропустить}, то есть, мы игнорируем некоторые биты \IT{справа}.}
{\TT{SHR} instruction shifting value in the \EAX register by number of bits must be
\IT{skipped}, e.g., we ignore a bits \IT{at right}.}

\index{x86!\Instructions!AND}
\IFRU{А инструкция \ANDIns очищает биты \IT{слева} которые нам не нужны, или же, говоря иначе, 
она оставляет по маске только те биты в \EAX, которые нам сейчас нужны.}
{\ANDIns instruction clears bits not needed \IT{at left}, or, in other words, 
leaves only those bits in the \EAX register we need now.}

\IFRU{Попробуем GCC 4.4.1 с опцией \Othree.}{Let's try GCC 4.4.1 with \Othree option.}

\lstinputlisting[caption=\Optimizing GCC 4.4.1]{patterns/15_structs/CPUID_gcc_O3.asm}

\IFRU{Практически, то же самое. Единственное что стоит отметить это то, что GCC решил зачем-то объединить 
вычисление \TT{extended\_model\_id} и \TT{extended\_family\_id} в один блок, 
вместо того чтобы вычислять их перед соответствующим вызовом \printf.}
{Almost the same.
The only thing worth noting is the GCC somehow united calculation of
\TT{extended\_model\_id} and \TT{extended\_family\_id} into one block,
instead of calculating them separately, before corresponding each \printf call.}

\subsubsection{\WorkingWithFloatAsWithStructSubSubSectionName}
\label{sec:floatasstruct}

\IFRU{Как уже раннее указывалось в секции о FPU~(\ref{sec:FPU}), и \Tfloat и \Tdouble содержат в себе знак, 
мантиссу и экспоненту. 
Однако, можем ли мы работать с этими полями напрямую? Попробуем с \Tfloat.}
{As it was already noted in section about FPU~(\ref{sec:FPU}), both \Tfloat and \Tdouble types consisted of sign,
significand (or fraction) and exponent.
But will we able to work with these fields directly? Let's try with \Tfloat.}

\bigskip
% a hack used here! http://tex.stackexchange.com/questions/73524/bytefield-package
\begin{center}
\begin{bytefield}{32}
	\bitheader[endianness=big]{0,22,23,30,31} \\
	\bitbox{1}{S} & 
	\bitbox{8}{\IFRU{экспонента}{exponent}} & 
	\bitbox{23}{\IFRU{мантисса}{mantissa or fraction}}
\end{bytefield}
\end{center}

\begin{center}
( S\EMDASH{}\IFRU{знак}{sign} )
\end{center}

\lstinputlisting{patterns/15_structs/float_en.c}

\IFRU{Структура \TT{float\_as\_struct} занимает в памяти столько же места сколько и \Tfloat, 
то есть 4 байта или 32 бита.}
{\TT{float\_as\_struct} structure occupies as much space is memory as \Tfloat, e.g., 4 bytes or 32 bits.}

\IFRU{Далее мы выставляем во входящем значении отрицательный знак, 
а также прибавляя двойку к экспоненте, мы тем 
самым умножаем всё значение на \TT{$2^2$}, то есть на 4.}
{Now we setting negative sign in input value and also by adding 2 to exponent we thereby multiplicating
the whole number by \TT{$2^2$}, e.g., by 4.}

\IFRU{Компилируем в MSVC 2008 без оптимизации:}{Let's compile in MSVC 2008 without optimization:}

\lstinputlisting[caption=\NonOptimizing MSVC 2008]{patterns/15_structs/float_msvc_\LANG.asm}

\IFRU{Слегка избыточно. В версии скомпилированной с флагом \Ox нет вызовов \TT{memcpy()}, 
там работа происходит сразу с переменной f. Но по неоптимизированной версии будет проще понять.}
{Redundant for a bit.
If it is compiled with \Ox flag there is no \TT{memcpy()} call,
\TT{f} variable is used directly.
But it is easier to understand it all considering unoptimized version.}

\IFRU{А что сделает GCC 4.4.1 с опцией \Othree?}{What GCC 4.4.1 with \Othree will do?}

\lstinputlisting[caption=\Optimizing GCC 4.4.1]{patterns/15_structs/float_gcc_O3_\LANG.asm}

\IFRU{Да, функция \TT{f()} в целом понятна. Однако, что интересно, еще при компиляции, 
не взирая на мешанину с полями структуры, GCC умудрился вычислить результат функции \TT{f(1.234)} и 
сразу подставить его в аргумент для \printf{}!}
{The \TT{f()} function is almost understandable. However, what is interesting, GCC was able to calculate
\TT{f(1.234)} result during compilation stage despite all this hodge-podge with structure fields
and prepared this argument to the \printf{} as precalculated!}





\chapter{\RU{Объединения (union)}\EN{Unions}}

\EN{\CCpp \IT{union} is mostly used for interpreting a variable (or memory block) of one data type as a variable of another data type.}
\RU{\IT{union} в \CCpp используется в основном для интерпертации переменной (или блока памяти) одного типа как переменной другого типа.}

% sections
\section{\RU{Пример генератора случайных чисел}\EN{Pseudo-random number generator example}}
\label{FPU_PRNG}

\RU{Если нам нужны случайные значения с плавающей запятой в интервале от 0 до 1, самое простое это взять
\ac{PRNG} вроде Mersenne twister.
Он выдает случайные 32-битные числа в виде DWORD.
Затем мы можем преобразовать это число в \Tfloat и затем разделить на \TT{RAND\_MAX} (\TT{0xFFFFFFFF} в данном случае)\EMDASH{}
полученное число будет в интервале от 0 до 1.}
\EN{If we need float random numbers between 0 and 1, the simplest thing is to use a \ac{PRNG} like
the Mersenne twister. 
It produces random 32-bit values in DWORD form. 
Then we can transform this value to \Tfloat and then
divide it by \TT{RAND\_MAX} (\TT{0xFFFFFFFF} in our case)\EMDASH{}
we getting a value in the 0..1 interval.}

\RU{Но как известно, операция деления\EMDASH{}это медленная операция. 
Да и вообще хочется избежать лишних операций с FPU.
Сможем ли мы избежать деления?}
\EN{But as we know, division is slow.
Also, we would like to issue as few FPU operations as possible.
Can we get rid of the division?}

\index{IEEE 754}
\RU{Вспомним состав числа с плавающей запятой: это бит знака, биты мантиссы и биты экспоненты. 
Для получения случайного числа, нам нужно просто заполнить случайными битами все биты мантиссы!}
\EN{Let's recall what a floating point number consists of: sign bit, significand bits and exponent bits.
We just need to store random bits in all significand bits to get a random float number!}

\RU{Экспонента не может быть нулевой (иначе число с плавающей точкой будет денормализованным), 
так что в эти биты мы запишем \TT{01111111}\EMDASH{}
это будет означать что экспонента равна единице. Далее заполняем мантиссу случайными битами, 
знак оставляем в виде 0 (что значит наше число положительное), и вуаля. 
Генерируемые числа будут в интервале от 1 до 2, так что нам еще нужно будет отнять единицу.}
\EN{The exponent cannot be zero (the floating number is denormalized in this case), so we are storing \TT{01111111} 
to exponent\EMDASH{}this means that the exponent is 1. 
Then we filling the significand with random bits, set the sign bit to
0 (which means a positive number) and voilà.
The generated numbers is to be between 1 and 2, so we must also subtract 1.}

\newcommand{\URLXOR}{\url{http://go.yurichev.com/17308}}

\RU{В моем примере\footnote{идея взята здесь: \URLXOR} 
применяется очень простой линейный конгруэнтный генератор случайных чисел, выдающий 32-битные числа.
Генератор инициализируется текущим временем в стиле UNIX.}
\EN{A very simple linear congruential random numbers generator is used in my 
example\footnote{the idea was taken from: \URLXOR}, it produces 32-bit numbers. 
The \ac{PRNG} is initialized with the current time in UNIX timestamp format.}

\RU{Далее, тип \Tfloat представляется в виде \IT{union}\EMDASH{}это конструкция \CCpp позволяющая 
интерпретировать часть памяти по-разному. В нашем случае, мы можем создать переменную типа \TT{union} 
и затем обращаться к ней как к \Tfloat или как к \IT{uint32\_t}. Можно сказать, что это хак, причем грязный.}
\EN{Here we represent the \Tfloat type as an \IT{union}\EMDASH{}it is the \CCpp construction that enables us
to interpret a piece of memory as different types.
In our case, we are able to create a variable
of type \TT{union} and then access to it as it is \Tfloat or as it is \IT{uint32\_t}. 
It can be said, it is just a hack. A dirty one.}

% WTF?
\RU{Код целочисленного \ac{PRNG} точно такой же, как мы уже рассматривали ранее:}
\EN{The integer \ac{PRNG} code is the same as we already considered:} \myref{LCG_simple}.
\RU{Так что и в скомпилированном виде этот код будет опущен.}
\EN{So this code in compiled form is omitted.}

\lstinputlisting{patterns/17_unions/FPU_PRNG/FPU_PRNG.cpp.\LANG}

\subsection{x86}

\lstinputlisting[caption=\Optimizing MSVC 2010]{patterns/17_unions/FPU_PRNG/MSVC2010_Ox_Ob0.asm.\LANG}

\EN{Function names are so strange here because this example was compiled as C++ and this is name mangling in C++,
we will talk about it later:}%
\RU{Имена функций такие странные, потому что этот пример был скомпилирован как Си++, и это манглинг имен в Си++, 
мы будем рассматривать это позже:} \myref{namemangling}.

\RU{Если скомпилировать это в MSVC 2012, компилятор будет использовать SIMD-инструкции для FPU, читайте об этом
здесь:}
\EN{If we compile this in MSVC 2012, it uses the SIMD instructions for the FPU, read more about it here:}
\myref{FPU_PRNG_SIMD}.

\subsection{MIPS}

\lstinputlisting[caption=\Optimizing GCC 4.4.5]{patterns/17_unions/FPU_PRNG/MIPS_O3_IDA.lst.\LANG}

\EN{There is also an useless LUI instruction added for some weird reason.}
\RU{Здесь снова зачем-то добавлена инструкция LUI, которая ничего не делает.}
\EN{We considered this artifact earlier:}
\RU{Мы уже рассматривали этот артефакт ранее:} \myref{MIPS_FPU_LUI}.

\subsection{ARM (\ARMMode)}

\lstinputlisting[caption=\Optimizing GCC 4.6.3 (IDA)]{patterns/17_unions/FPU_PRNG/raspberry_GCC_O3_IDA.lst.\LANG}

\index{objdump}
\index{binutils}
\index{IDA}
\RU{Мы также сделаем дамп в objdump и увидим что FPU-инструкции имеют немного другие имена чем в \IDA.}%
\EN{We'll also make a dump in objdump and we'll see that the FPU instructions have different names than in \IDA.}
\EN{Apparently, IDA and binutils developers used different manuals?}
\RU{Наверное, разработчики IDA и binutils пользовались разной документацией?}
\EN{Perhaps, it would be good to know both instruction name variants.}
\RU{Должно быть, будет полезно знать оба варианта названий инструкций.}

\lstinputlisting[caption=\Optimizing GCC 4.6.3 (objdump)]{patterns/17_unions/FPU_PRNG/raspberry_GCC_O3_objdump.lst}

\EN{The instructions at 0x5c in float\_rand() and at 0x38 in main() are random noise.}
\RU{Инструкции по адресам 0x5c в float\_rand() и 0x38 в main() это случайный мусор.}

\ifdefined\RUSSIAN
\else
\section{Calculating machine epsilon}

\subsection{x86}

Machine epsilon is a smallest possible granule \ac{FPU} can work with\RU{\footnote{В русскоязычной
литературе встречается также термин ``машинный ноль''.}}.
The more bits allocated for floating point number, the smaller machine epsilon.
It is $2^{-23} = 1.19e-07$ for \Tfloat and $2^{-52} = 2.22e-16$ for double.

It's interesting, how easy it's possible to calculate machine epsilon:

\lstinputlisting{patterns/17_unions/epsilon/float.c}

What we do here is just treating fraction part of IEE 754 number as integer and adding 1 to it.
Resulting number will be $starting\_value+machine\_epsilon$, so we just need to subtract
starting value (using floating point arithmetics) to measure, what number one bit reflects
in the single precision (\Tfloat).

union serves here as a way to access IEEE 754 number as a regular integer.
Adding 1 to it is in fact adds 1 to \IT{fraction} part of number, however, needless to say,
overflow is possible, which will add yet another bit to exponent part.

\lstinputlisting[caption=\Optimizing MSVC 2010]{patterns/17_unions/epsilon/float_MSVC_2010_Ox.asm}

Second FST instruction is redundant: there are no need to store input value to the same
place (compiler decided to allocate $v$ variable at the same point of local stack as input 
argument).

Then it is incremented with INC, as it is usual integer variable.
Then it is loaded into FPU as it is 32-bit IEEE 754 number, FSUBR do the job and resulting
value is in the ST0.

Two last FSTP/FLD instruction pair is redundant, but compiler didn't optimized them.

\ifdefined\IncludeARM
\subsection{ARM64}

Let's extend our example to 64-bit:

\lstinputlisting[label=machine_epsilon_double_c]{patterns/17_unions/epsilon/double.c}

ARM64 has no instruction which can add a number to FPU D-register, 
so input value (came in D0) is first copied into GPR,
incremented, copied to FPU register D1, then subtraction occurred.

\lstinputlisting[caption=\Optimizing GCC 4.9 ARM64]{patterns/17_unions/epsilon/double_GCC49_ARM64_O3.s}

See also this example compiled for x64 with SIMD instructions: \ref{machine_epsilon_x64_and_SIMD}.
\fi

\subsection{Conclusion}

It's hard to say, whether someone will need this trickery in real-world code, 
but as I write many times in this book, this example is serving well 
for explaining IEEE 754 format and union feature of \CCpp.
\fi


\section{\RU{Быстрое вычисление квадратного корня}\EN{Fast square root calculation}}

\RU{Вот где еще можно на практике применить трактовку типа \Tfloat как целочисленного, это быстрое вычисление квадратного корня.}%
\EN{Another well-known algorithm where \Tfloat is interpreted as integer is fast calculation of square root.}

\begin{lstlisting}[caption=\EN{The source code is taken from Wikipedia}\RU{Исходный код взят из Wikipedia}: \url{http://go.yurichev.com/17364}]
/* Assumes that float is in the IEEE 754 single precision floating point format
 * and that int is 32 bits. */
float sqrt_approx(float z)
{
    int val_int = *(int*)&z; /* Same bits, but as an int */
    /*
     * To justify the following code, prove that
     *
     * ((((val_int / 2^m) - b) / 2) + b) * 2^m = ((val_int - 2^m) / 2) + ((b + 1) / 2) * 2^m)
     *
     * where
     *
     * b = exponent bias
     * m = number of mantissa bits
     *
     * .
     */
 
    val_int -= 1 << 23; /* Subtract 2^m. */
    val_int >>= 1; /* Divide by 2. */
    val_int += 1 << 29; /* Add ((b + 1) / 2) * 2^m. */
 
    return *(float*)&val_int; /* Interpret again as float */
}
\end{lstlisting}

\RU{В качестве упражнения, вы можете попробовать скомпилировать эту функцию и разобраться, как она работает.}
\EN{As an exercise, you can try to compile this function and to understand, how it works.}\ESph{}\PTBRph{}\PLph{}\ITAph{}\\
\\
\RU{Имеется также известный алгоритм быстрого вычисления}\EN{There is also well-known algorithm of fast calculation of} $\frac{1}{\sqrt{x}}$.
\index{Quake III Arena}
\RU{Алгоритм стал известным, вероятно потому, что был применен в Quake III Arena.}%
\EN{Algorithm became popular, supposedly, because it was used in Quake III Arena.}

\RU{Описание алгоритма есть в}\EN{Algorithm description is present in} Wikipedia:
\EN{\url{http://go.yurichev.com/17360}}\RU{\url{http://go.yurichev.com/17361}}.


\newcommand{\comp}{\TT{comp()}\xspace}
\chapter{\RU{Указатели на функции}\EN{Pointers to functions}}
\label{sec:pointerstofunctions}

\index{\CLanguageElements!\Pointers}
\RU{Указатель на функцию, в целом, как и любой другой указатель, просто адрес указывающий на начало функции 
в сегменте кода.}
\EN{Pointer to function, as any other pointer, is just an address of function beginning in its code segment.}

\index{Callbacks}
\RU{Это применяется часто в т.н. callback-ах}\EN{It is often used in callbacks}
\footnote{\url{http://en.wikipedia.org/wiki/Callback_(computer_science)}}.

\RU{Известные примеры:}\EN{Well-known examples are:}

\begin{itemize}
\item
\qsort\footnote{\url{http://en.wikipedia.org/wiki/Qsort_(C_standard_library)}},
{\TT{atexit()}}\footnote{\url{http://www.opengroup.org/onlinepubs/009695399/functions/atexit.html}} \RU{из стандартной библиотеки Си}\EN{from the standard C library}; 

\item
\RU{сигналы в *NIX ОС}\EN{*NIX OS signals}\footnote{\url{http://en.wikipedia.org/wiki/Signal.h}};

\item
\RU{запуск тредов}\EN{thread starting}: \TT{CreateThread()} (win32), \TT{pthread\_create()} (POSIX);

\item
\RU{множество функций win32, например}\EN{a lot of win32 functions, e.g.} \TT{EnumChildWindows()}\footnote{\url{http://msdn.microsoft.com/en-us/library/ms633494(VS.85).aspx}}.

\item
\EN{a lot of places in Linux kernel, for example, filesystem driver functions are called via
callbacks}\RU{множество мест в ядре Linux, например, ф-ции драйверов файловой системы вызываются
через callback-и}: 
\url{http://lxr.free-electrons.com/source/include/linux/fs.h?v=3.14\#L1525}

\item
\EN{GCC plugin functions are also called via callbacks}\RU{ф-ции плагинов GCC также вызываются
через callback-и}: 
\url{https://gcc.gnu.org/onlinedocs/gccint/Plugin-API.html\#Plugin-API}

\ifdefined\RUSSIAN
\else
\item
One example of function pointers is a table in ``dwm'' Linux window manager, 
consisting of shortcuts. 
Each shortcut has corresponding function to call if the specific key has been pressed:\\
\url{https://github.com/cdown/dwm/blob/master/config.def.h#L117}\\
As one may see, such table is much more easier to handle then large switch() statement.
\fi
\end{itemize}

\index{\CStandardLibrary!qsort()}
\RU{Итак, функция \qsort это реализация алгоритма ``быстрой сортировки''. 
Функция может сортировать что угодно, 
любые типы данных, но при условии, что вы имеете функцию сравнения этих двух элементов данных и 
\qsort может вызывать её.}
\EN{So, \qsort function is a \CCpp standard library quicksort implementation. 
The functions is able to sort anything, any types of data, 
as long as you have a function for these two elements comparison, 
and \qsort is able to call it.}

\RU{Эта функция сравнения может определяться так:}\EN{The comparison function can be defined as:}

\begin{lstlisting}
int (*compare)(const void *, const void *)
\end{lstlisting}

\RU{Воспользуемся немного модифицированным примером, который я нашел вот}
\EN{Let's use slightly modified example I found} \href{http://cplus.about.com/od/learningc/ss/pointers2_8.htm}
{\RU{здесь}\EN{here}}:

\lstinputlisting[numbers=left,label=qsort_c_src]{patterns/18_pointers_to_functions/17_1.c}

\section{MSVC}

\RU{Компилируем в MSVC 2010 (я убрал некоторые части для краткости) с опцией \Ox}
\EN{Let's compile it in MSVC 2010 (I omitted some parts for the sake of brevity) with \Ox option}:

\lstinputlisting[caption=\Optimizing MSVC 2010: /GS- /MD]{patterns/18_pointers_to_functions/17_2_msvc_Ox.asm}

\RU{Ничего особо удивительного здесь мы не видим. В качестве четвертого аргумента, 
в \qsort просто передается адрес метки \TT{\_comp}, где собственно и располагается функция \comp,
или, можно сказать, самая первая инструкция этой ф-ции.}
\EN{Nothing surprising so far.
As a fourth argument, an address of label \TT{\_comp} is passed, that is just a place
where function \comp located, or, in other words, address of the very first instruction of 
this function.}

\RU{Как \qsort вызывает её?}\EN{How \qsort calling it?}

\index{Windows!MSVCR80.DLL}
\RU{Посмотрим в MSVCR80.DLL (эта DLL куда в MSVC вынесены функции из стандартных библиотек Си):}
\EN{Let's take a look into this function located in MSVCR80.DLL (a MSVC DLL module with C standard library functions):}

\lstinputlisting[caption=MSVCR80.DLL]{patterns/18_pointers_to_functions/17_3_MSVCR.lst}

\TT{comp}\EMDASH{}\RU{это четвертый аргумент функции. 
Здесь просто передается управление по адресу указанному в \TT{comp}. 
Перед этим подготавливается два аргумента для функции \comp. 
Далее, проверяется результат её выполнения.}
\EN{is fourth function argument.
Here the control is just passed to the address in the \TT{comp} argument.
Before it, two arguments prepared for \comp. Its result is checked after its execution.}

\RU{Вот почему использование указателей на функции ~--- это опасно. 
Во-первых, если вызвать \qsort с неправильным указателем на функцию, 
то \qsort, дойдя до этого вызова, может передать управление неизвестно куда, 
процесс упадет, и эту ошибку можно будет найти не сразу.}
\EN{That's why it is dangerous to use pointers to functions.
First of all, if you call \qsort with incorrect pointer to function, \qsort may pass control
to incorrect point, a process may crash and this bug will be hard to find.}

\RU{Во-вторых, типизация callback-функции должна строго соблюдаться, 
вызов не той функции с не теми аргументами не того типа, 
может привести к плачевным результатам, 
хотя падение процесса это и не проблема, проблема ~--- это найти ошибку, ведь компилятор 
на стадии компиляции может вас и не предупредить о потенциальных неприятностях.}
\EN{Second reason is the callback function types must comply strictly, calling wrong function
with wrong arguments of wrong types may lead to serious problems, however, process crashing is not a 
big problem~---big problem is to determine a reason of crashing~---because compiler may be 
silent about potential trouble while compiling.}

\ifdefined\IncludeOlly
\clearpage
\subsection{MSVC + \olly}
\index{\olly}

\RU{Загрузим наш пример в \olly и установим брякпойнт на ф-ции \comp}
\EN{Let's load our example into \olly and set breakpoint on \comp function}.

\RU{Как значения сравниваются, мы можем увидеть во время самого первого вызова \comp}
\EN{How values are compared we can see at the very first \comp call}:

\begin{figure}[H]
\centering
\includegraphics[scale=\FigScale]{patterns/18_pointers_to_functions/olly1.png}
\caption{\olly: \RU{первый вызов}\EN{first call of} \comp}
\label{fig:qsort_olly1}
\end{figure}

\RU{Для удобства, }\olly \RU{показывает сравниваемые значения в окне под окном кода}
\EN{shows compared values in the window under code window, for convenience}.
\RU{Мы можем так же увидеть что}\EN{We can also see that the} \ac{SP} \RU{указывает на}\EN{pointing to} 
\ac{RA} \RU{где находится место в ф-ции}\EN{where the place in} 
\qsort \EN{function is }(\RU{на самом деле, находится в}\EN{actually located in} \TT{MSVCR100.DLL}).

\clearpage
\RU{Трассируя}\EN{By tracing} (F8) \RU{до инструкции}\EN{until} \TT{RETN} 
\RU{и нажав F8 еще один раз, мы возвращаемся в ф-цию}\EN{instruction, and pressing F8 one more time, 
we returning into} \qsort\EN{ function}:

\begin{figure}[H]
\centering
\includegraphics[scale=\FigScale]{patterns/18_pointers_to_functions/olly2.png}
\caption{\olly: \RU{код в}\EN{the code in} \qsort \RU{сразу после вызова}\EN{right after} \comp\EN{ call}}
\label{fig:qsort_olly2}
\end{figure}

\RU{Это был вызов ф-ции сравнения}\EN{That was a call to comparison function}.

\clearpage
\RU{Вот также скриншот момента второго вызова ф-ции}\EN{Here is also screenshot of the moment of the 
second call of} \comp\EMDASH{}\RU{теперь сравниваемые значения другие}
\EN{now values to be compared are different}:

\begin{figure}[H]
\centering
\includegraphics[scale=\FigScale]{patterns/18_pointers_to_functions/olly3.png}
\caption{\olly: \RU{второй вызов}\EN{second call of} \comp}
\label{fig:qsort_olly3}
\end{figure}

\subsection{MSVC + tracer}
\index{tracer}

\RU{Посмотрим, какие пары сравниваются}\EN{Let's also see, which pairs are compared}.
\RU{Эти 10 чисел будут сортироваться}\EN{These 10 numbers are being sorted}: 
1892, 45, 200, -98, 4087, 5, -12345, 1087, 88, -100000.

\RU{Я нашел адрес первой инструкции}\EN{I found the address of the first} \CMP 
\RU{в}\EN{instruction in} \comp, \RU{и это}\EN{it is} \TT{0x0040100C} 
\RU{и я ставлю брякпойнт на нем}\EN{and I'm setting breakpoint on it}:

\begin{lstlisting}
tracer.exe -l:17_1.exe bpx=17_1.exe!0x0040100C
\end{lstlisting}

\RU{Получаю информацию о регистрах на брякпойнте}
\EN{I'm getting information about registers at breakpoint}:

\begin{lstlisting}
PID=4336|New process 17_1.exe
(0) 17_1.exe!0x40100c
EAX=0x00000764 EBX=0x0051f7c8 ECX=0x00000005 EDX=0x00000000
ESI=0x0051f7d8 EDI=0x0051f7b4 EBP=0x0051f794 ESP=0x0051f67c
EIP=0x0028100c
FLAGS=IF
(0) 17_1.exe!0x40100c
EAX=0x00000005 EBX=0x0051f7c8 ECX=0xfffe7960 EDX=0x00000000
ESI=0x0051f7d8 EDI=0x0051f7b4 EBP=0x0051f794 ESP=0x0051f67c
EIP=0x0028100c
FLAGS=PF ZF IF
(0) 17_1.exe!0x40100c
EAX=0x00000764 EBX=0x0051f7c8 ECX=0x00000005 EDX=0x00000000
ESI=0x0051f7d8 EDI=0x0051f7b4 EBP=0x0051f794 ESP=0x0051f67c
EIP=0x0028100c
FLAGS=CF PF ZF IF
...
\end{lstlisting}

\RU{Я отфильтровал}\EN{I filtered out} \TT{EAX} \AndENRU \TT{ECX} \RU{и получил}\EN{and got}:

\begin{lstlisting}
EAX=0x00000764 ECX=0x00000005
EAX=0x00000005 ECX=0xfffe7960
EAX=0x00000764 ECX=0x00000005
EAX=0x0000002d ECX=0x00000005
EAX=0x00000058 ECX=0x00000005
EAX=0x0000043f ECX=0x00000005
EAX=0xffffcfc7 ECX=0x00000005
EAX=0x000000c8 ECX=0x00000005
EAX=0xffffff9e ECX=0x00000005
EAX=0x00000ff7 ECX=0x00000005
EAX=0x00000ff7 ECX=0x00000005
EAX=0xffffff9e ECX=0x00000005
EAX=0xffffff9e ECX=0x00000005
EAX=0xffffcfc7 ECX=0xfffe7960
EAX=0x00000005 ECX=0xffffcfc7
EAX=0xffffff9e ECX=0x00000005
EAX=0xffffcfc7 ECX=0xfffe7960
EAX=0xffffff9e ECX=0xffffcfc7
EAX=0xffffcfc7 ECX=0xfffe7960
EAX=0x000000c8 ECX=0x00000ff7
EAX=0x0000002d ECX=0x00000ff7
EAX=0x0000043f ECX=0x00000ff7
EAX=0x00000058 ECX=0x00000ff7
EAX=0x00000764 ECX=0x00000ff7
EAX=0x000000c8 ECX=0x00000764
EAX=0x0000002d ECX=0x00000764
EAX=0x0000043f ECX=0x00000764
EAX=0x00000058 ECX=0x00000764
EAX=0x000000c8 ECX=0x00000058
EAX=0x0000002d ECX=0x000000c8
EAX=0x0000043f ECX=0x000000c8
EAX=0x000000c8 ECX=0x00000058
EAX=0x0000002d ECX=0x000000c8
EAX=0x0000002d ECX=0x00000058
\end{lstlisting}

\RU{Это}\EN{That's} 34 \RU{пары}\EN{pairs}.
\RU{Следовательно, алгоритму быстрой сортировки нужно 34 операции сравнения для сортировки этих
10-и чисел}\EN{Therefore, quick sort algorithm needs 34 comparison operations for sorting these 10 numbers}.

\clearpage
\subsection{MSVC + tracer (code coverage)}
\index{tracer}

\RU{Но можно также и воспользоваться возможностью tracer накапливать все возможные состояния регистров
и показать их в \IDA}\EN{We can also use tracer's feature to collect all possible register's values
and show them in \IDA}.

\RU{Трассируем все инструкции в ф-ции \comp}\EN{Let's trace all instructions in \comp function}:

\begin{lstlisting}
tracer.exe -l:17_1.exe bpf=17_1.exe!0x00401000,trace:cc
\end{lstlisting}

\RU{Получем .idc-скрипт для загрузки в \IDA и загружаем его}
\EN{We getting .idc-script for loading into \IDA and load it}:

\begin{figure}[H]
\centering
\includegraphics[scale=\FigScale]{patterns/18_pointers_to_functions/tracer_cc.png}
\caption{tracer \AndENRU IDA. N.B.: 
\RU{некоторые значения обрезаны справа}\EN{some values are cutted at right}}
\label{fig:qsort_tracer_cc}
\end{figure}

\RU{Имя этой ф-ции (PtFuncCompare) дала \IDA}\EN{\IDA gave the function name (PtFuncCompare)}
\EMDASH{}\RU{видимо, потому что видит что указатель на эту ф-цию передается в \qsort}\EN{it seems,
because \IDA sees that pointer to this function is passed into \qsort}.

\RU{Мы видим что указатели $a$ и $b$ указывают на разные места внутри массива, 
но шаг между указателями --- 4, что логично, ведь в массиве хранятся 32-битные значения}
\EN{We see that $a$ and $b$ pointers are pointing to various places in array, but step between
points is 4---indeed, 32-bit values are stored in the array}.

\RU{Видно что инструкции по адресам}\EN{We see that the instructions at} \TT{0x401010} \AndENRU 
\TT{0x401012} \RU{никогда не исполнялись}\EN{was never executed} 
(\RU{они и остались белыми}\EN{so they leaved as white}): 
\RU{действительно, ф-ция}\EN{indeed,} \comp \RU{никогда не возвращала 0,
потому что в массиве нет одинаковых элементов}\EN{was never returned 0, because there no equal elements in array}.

\fi

\section{GCC}

\RU{Не слишком большая разница}\EN{Not a big difference}:

\begin{lstlisting}[caption=GCC]
                lea     eax, [esp+40h+var_28]
                mov     [esp+40h+var_40], eax
                mov     [esp+40h+var_28], 764h
                mov     [esp+40h+var_24], 2Dh
                mov     [esp+40h+var_20], 0C8h
                mov     [esp+40h+var_1C], 0FFFFFF9Eh
                mov     [esp+40h+var_18], 0FF7h
                mov     [esp+40h+var_14], 5
                mov     [esp+40h+var_10], 0FFFFCFC7h
                mov     [esp+40h+var_C], 43Fh
                mov     [esp+40h+var_8], 58h
                mov     [esp+40h+var_4], 0FFFE7960h
                mov     [esp+40h+var_34], offset comp
                mov     [esp+40h+var_38], 4
                mov     [esp+40h+var_3C], 0Ah
                call    _qsort
\end{lstlisting}

\RU{Функция \comp}\EN{\comp function}:

\begin{lstlisting}
                public comp
comp            proc near

arg_0           = dword ptr  8
arg_4           = dword ptr  0Ch

                push    ebp
                mov     ebp, esp
                mov     eax, [ebp+arg_4]
                mov     ecx, [ebp+arg_0]
                mov     edx, [eax]
                xor     eax, eax
                cmp     [ecx], edx
                jnz     short loc_8048458
                pop     ebp
                retn
loc_8048458:
                setnl   al
                movzx   eax, al
                lea     eax, [eax+eax-1]
                pop     ebp
                retn
comp            endp
\end{lstlisting}

\index{Linux!libc.so.6}
\RU{Реализация \qsort находится в \TT{libc.so.6}, и представляет собой просто wrapper
\footnote{понятие близкое к \gls{thunk function}} для \TT{qsort\_r()}.}
\EN{\qsort implementation is located in the \TT{libc.so.6} and it is in fact just a wrapper
\footnote{a concept like \gls{thunk function}} for \TT{qsort\_r()}.}

\RU{Она, в свою очередь, вызывает \TT{quicksort()}, где есть вызовы определенной нами функции через 
переданный указатель:}
\EN{It will call then \TT{quicksort()}, where our defined function will be called via passed pointer:}

\begin{lstlisting}[caption=
(\RU{файл libc.so.6{,} версия glibc}\EN{file libc.so.6{,} glibc version}\EMDASH{}2.10.1)]

.text:0002DDF6                 mov     edx, [ebp+arg_10]
.text:0002DDF9                 mov     [esp+4], esi
.text:0002DDFD                 mov     [esp], edi
.text:0002DE00                 mov     [esp+8], edx
.text:0002DE04                 call    [ebp+arg_C]
...
\end{lstlisting}

\ifdefined\IncludeGDB
\subsection{GCC + GDB (\RU{с исходными кодами}\EN{with source code})}
\index{GDB}

\RU{Очевидно, у нас есть исходный код нашего примера на Си (\ref{qsort_c_src}), 
так что мы можем установить брякпойнт ($b$) на
номере строки}\EN{Obviously, we have a C-source code of our example (\ref{qsort_c_src}), 
so we can set breakpoint ($b$) on line number}
(\RU{11-й --- это номер строки где происходит первое сравнение}\EN{11th---the line where 
first comparison is occurred}).
\RU{Нам также нужно скомпилировать наш пример с ключом \TT{-g}, чтобы в исполняемом файле была
полная отладочная информация}\EN{We also need to compile example with debugging information 
included (\TT{-g}), so the table
with addresses and corresponding line numbers is present}.
\RU{Мы можем так же выводить значения используя имена переменных}
\EN{We can also print values by variable name} (\TT{p}):
\RU{отладочная информация также содержит информацию о том, в каком регистре и/или элементе локального
стека находится какая переменная}\EN{debugging information also has information about which register and/or 
local stack element contain which variable}.

\index{Glibc}
\RU{Мы можем также увидеть стек}\EN{We can also see stack} (\TT{bt}) 
\RU{и обнаружить что в Glibc используется какая-то вспомогательная ф-ция с именем}
\EN{and find out that there are some intermediate function} 
\TT{msort\_with\_tmp()}\EN{ used in Glibc}.

\lstinputlisting[caption=GDB\RU{-сессия}\EN{ session}]{patterns/18_pointers_to_functions/GDB_source.txt}

\subsection{GCC + GDB (\RU{без исходных кодов}\EN{no source code})}
\index{GDB}

\RU{Но часто никаких исходных кодов нет вообще, так что мы можем дизассемблировать ф-цию \comp}
\EN{But often there are no source code at all, so we can disassemble \comp function} (\TT{disas}), 
\RU{найти самую первую инструкцию \CMP и установить брякпойнт}\EN{find the very first
\CMP instruction and set breakpoint} ($b$) \RU{по этому адресу}\EN{at that address}.
\RU{На каждом брякпойнте мы будем видеть содержимое регистров}
\EN{At each breakpoint, we will dump all register contents} (\TT{info registers}).
\RU{Информация из стека так же доступна}\EN{Stack information is also available} (\TT{bt}), 
\RU{но частичная: здесь нет номеров строк для ф-ции \comp}
\EN{but partial: there are no line number information for \comp function}.

\lstinputlisting[caption=GDB\RU{-сессия}\EN{ session}]{patterns/18_pointers_to_functions/GDB_no_source.txt}
\fi

\ifdefined\ENGLISH
\section{64-bit values in 32-bit environment}
\label{sec:64bit_in_32_env}

In a 32-bit environment, \ac{GPR}'s are 32-bit, so 64-bit values are stored and passed as 32-bit value pairs
\footnote{By the way, 32-bit values are passed as pairs in 16-bit environment in the same way: \myref{win16_32bit_values}}.
\fi

\ifdefined\RUSSIAN
\section{64-битные значения в 32-битной среде}
\label{sec:64bit_in_32_env}

В среде, где \ac{GPR}-ы 32-битные, 64-битные значения хранятся и передаются как пары 32-битных значений
\footnote{Кстати, в 16-битной среде, 32-битные значения передаются 16-битными парами точно так же: \myref{win16_32bit_values}}.
\fi

\ifdefined\GERMAN
\section{64-Bit-Werte in 32-Bit-Umgebungen}
\label{sec:64bit_in_32_env}

In einer 32-Bit-Umgebung sind \ac{GPR} 32 Bit groß. Also werden 64-Bit-Werte in
32-Bit-Wertepaaren gespeichert und übergeben\footnote{Übrigens, 32-Bit-Werte werden
als Paare in 16--Bit-Umgebungen auf der gleiche Art übergeben: \myref{win16_32bit_values}}.
\fi

\EN{\subsection{Returning of 64-bit value}

\lstinputlisting[style=customc]{patterns/185_64bit_in_32_env/ret/0.c}

\subsubsection{x86}

In a 32-bit environment, 64-bit values are returned from functions in the \EDX{}:\EAX{} register pair.

\lstinputlisting[caption=\Optimizing MSVC 2010,style=customasmx86]{patterns/185_64bit_in_32_env/ret/0_MSVC_2010_Ox.asm}

\subsubsection{ARM}

A 64-bit value is returned in the \Reg{0}-\Reg{1} register pair (\Reg{1} is for the high part and \Reg{0} for the low part):

\lstinputlisting[caption=\OptimizingKeilVI (\ARMMode),style=customasmARM]{patterns/185_64bit_in_32_env/ret/Keil_ARM_O3.s}

\subsubsection{MIPS}

A 64-bit value is returned in the \TT{V0}-\TT{V1} (\$2-\$3) register pair (\TT{V0} (\$2) is for the high part and \TT{V1} (\$3) for the low part):

\lstinputlisting[caption=\Optimizing GCC 4.4.5 (assembly listing),style=customasmMIPS]{patterns/185_64bit_in_32_env/ret/0_MIPS.s}

\lstinputlisting[caption=\Optimizing GCC 4.4.5 (IDA),style=customasmMIPS]{patterns/185_64bit_in_32_env/ret/0_MIPS_IDA.lst}
}
\RU{\subsection{Возврат 64-битного значения}

\lstinputlisting[style=customc]{patterns/185_64bit_in_32_env/ret/0.c}

\subsubsection{x86}

64-битные значения в 32-битной среде возвращаются из функций в паре регистров \EDX{}:\EAX{}.

\lstinputlisting[caption=\Optimizing MSVC 2010,style=customasmx86]{patterns/185_64bit_in_32_env/ret/0_MSVC_2010_Ox.asm}

\subsubsection{ARM}

64-битное значение возвращается в паре регистров \Reg{0}-\Reg{1} --- (\Reg{1} это старшая часть и \Reg{0} --- младшая часть):

\lstinputlisting[caption=\OptimizingKeilVI (\ARMMode),style=customasmARM]{patterns/185_64bit_in_32_env/ret/Keil_ARM_O3.s}

\subsubsection{MIPS}

64-битное значение возвращается в паре регистров \TT{V0}-\TT{V1} (\$2-\$3) --- (\TT{V0} (\$2) это старшая часть и \TT{V1} (\$3) --- младшая часть):

\lstinputlisting[caption=\Optimizing GCC 4.4.5 (assembly listing),style=customasmMIPS]{patterns/185_64bit_in_32_env/ret/0_MIPS.s}

\lstinputlisting[caption=\Optimizing GCC 4.4.5 (IDA),style=customasmMIPS]{patterns/185_64bit_in_32_env/ret/0_MIPS_IDA.lst}

}
\DE{\subsection{Rückgabe von 64-Bit-Werten}

\lstinputlisting[style=customc]{patterns/185_64bit_in_32_env/ret/0.c}

\subsubsection{x86}
In einer 32-Bit-Umgebung werden 64-Bit-Werte von Funktionen über das Registerpaar \EDX:\EAX zurückgegeben:

\lstinputlisting[caption=\Optimizing MSVC 2010,style=customasmx86]{patterns/185_64bit_in_32_env/ret/0_MSVC_2010_Ox.asm}

\subsubsection{ARM}
Ein 64-Bit-Wert wird über das \Reg{0}-\Reg{1} Registerpaar zurückgegeben (\Reg{1} enthält dabei den höheren und \Reg{0}
den niederen Teil):

\lstinputlisting[caption=\OptimizingKeilVI (\ARMMode),style=customasmARM]{patterns/185_64bit_in_32_env/ret/Keil_ARM_O3.s}

\subsubsection{MIPS}

Ein 64-Bit-Wert wird über das \TT{V0}-\TT{V1} (\$2-\$3) Registerpaar zurückgegeben (\TT{V0} (\$2) enthält dabei den
höheren und \TT{V1} (\$3) den niederen Teil):

\lstinputlisting[caption=\Optimizing GCC 4.4.5 (assembly listing),style=customasmMIPS]{patterns/185_64bit_in_32_env/ret/0_MIPS.s}

\lstinputlisting[caption=\Optimizing GCC 4.4.5 (IDA),style=customasmMIPS]{patterns/185_64bit_in_32_env/ret/0_MIPS_IDA.lst}
}

\EN{\subsection{Arguments passing, addition, subtraction}

\lstinputlisting[style=customc]{patterns/185_64bit_in_32_env/passing_add_sub/1.c}

\subsubsection{x86}

\lstinputlisting[caption=\Optimizing MSVC 2012 /Ob1,style=customasmx86]{patterns/185_64bit_in_32_env/passing_add_sub/1_MSVC.asm}

We can see in the \GTT{f\_add\_test()} function that each 64-bit value is passed using two 32-bit values,
high part first, then low part. 

Addition and subtraction occur in pairs as well.

\myindex{x86!\Instructions!ADC}
In addition, the low 32-bit part are added first.
If carry has been occurred while adding, the \TT{CF} flag is set.

The following \INS{ADC} instruction adds the high parts of the values, and also adds 1 if $CF=1$.

\myindex{x86!\Instructions!SBB}
Subtraction also occurs in pairs.
The first \SUB may also turn on the CF flag, which is to be checked in the subsequent \INS{SBB} instruction:
if the carry flag is on, then 1 is also to be subtracted from the result.

It is easy to see how the \GTT{f\_add()} function result is then passed to \printf{}.

\lstinputlisting[caption=GCC 4.8.1 -O1 -fno-inline,style=customasmx86]{patterns/185_64bit_in_32_env/passing_add_sub/1_GCC.asm}

GCC code is the same.

\subsubsection{ARM}

\lstinputlisting[caption=\OptimizingKeilVI (\ARMMode),style=customasmARM]{patterns/185_64bit_in_32_env/passing_add_sub/Keil_ARM_O3.s}

\myindex{ARM!\Instructions!ADDS}
\myindex{ARM!\Instructions!SUBS}
\myindex{ARM!\Instructions!ADC}
\myindex{ARM!\Instructions!SBC}

The first 64-bit value is passed in \Reg{0} and \Reg{1} register pair, the second in \Reg{2} and \Reg{3} register pair.
ARM has the \INS{ADC} instruction as well (which counts carry flag) and \INS{SBC} (\q{subtract with carry}).
Important thing: when the low parts are added/subtracted, \INS{ADDS} and \INS{SUBS} instructions with -S suffix are used.
The -S suffix stands for \q{set flags}, and flags (esp. carry flag) is what consequent \INS{ADC}/\INS{SBC} instructions definitely need.
Otherwise, instructions without the -S suffix would do the job (\ADD and \SUB).

\subsubsection{MIPS}

\lstinputlisting[caption=\Optimizing GCC 4.4.5 (IDA),style=customasmMIPS]{patterns/185_64bit_in_32_env/passing_add_sub/MIPS_O3_IDA_EN.lst}

MIPS has no flags register, so there is no such information present after the execution of arithmetic operations.
So there are no instructions like x86's \INS{ADC} and \INS{SBB}.
To know if the carry flag would be set, a comparison (using \INS{SLTU} instruction) also occurs, 
which sets the destination register to 1 or 0.
This 1 or 0 is then added or subtracted to/from the final result.

}
\RU{\subsection{Передача аргументов, сложение, вычитание}

\lstinputlisting[style=customc]{patterns/185_64bit_in_32_env/passing_add_sub/1.c}

\subsubsection{x86}

\lstinputlisting[caption=\Optimizing MSVC 2012 /Ob1,style=customasmx86]{patterns/185_64bit_in_32_env/passing_add_sub/1_MSVC.asm}

В \GTT{f\_add\_test()} видно, как каждое 64-битное число передается двумя 32-битными значениями,
сначала старшая часть, затем младшая.

Сложение и вычитание происходит также парами. 

\myindex{x86!\Instructions!ADC}
При сложении, в начале складываются младшие 32 бита.
Если при сложении был перенос, выставляется флаг CF.
Следующая инструкция \INS{ADC} складывает старшие части чисел, но также прибавляет единицу если $CF=1$.

\myindex{x86!\Instructions!SBB}
Вычитание также происходит парами.
Первый \SUB может также включить флаг переноса CF, который затем будет проверяться в \INS{SBB}:
если флаг переноса включен, то от результата отнимется единица.

Легко увидеть, как результат работы \GTT{f\_add()} затем передается в \printf{}.

\lstinputlisting[caption=GCC 4.8.1 -O1 -fno-inline,style=customasmx86]{patterns/185_64bit_in_32_env/passing_add_sub/1_GCC.asm}

Код GCC почти такой же.

\subsubsection{ARM}

\lstinputlisting[caption=\OptimizingKeilVI (\ARMMode),style=customasmARM]{patterns/185_64bit_in_32_env/passing_add_sub/Keil_ARM_O3.s}

\myindex{ARM!\Instructions!ADDS}
\myindex{ARM!\Instructions!SUBS}
\myindex{ARM!\Instructions!ADC}
\myindex{ARM!\Instructions!SBC}
Первое 64-битное значение передается в паре регистров \Reg{0} и \Reg{1}, второе --- в паре \Reg{2} и \Reg{3}.
В ARM также есть инструкция \INS{ADC} (учитывающая флаг переноса) и \INS{SBC} (\q{subtract with carry} --- вычесть с переносом).
Важная вещь: когда младшие части слагаются/вычитаются, используются инструкции \INS{ADDS} и \INS{SUBS} с суффиксом -S.
Суффикс -S означает \q{set flags} (установить флаги), а флаги (особенно флаг переноса) это то что однозначно нужно последующим инструкциями \INS{ADC}/\INS{SBC}.
А иначе инструкции без суффикса -S здесь вполне бы подошли (\ADD и \SUB).

\subsubsection{MIPS}

\lstinputlisting[caption=\Optimizing GCC 4.4.5 (IDA),style=customasmMIPS]{patterns/185_64bit_in_32_env/passing_add_sub/MIPS_O3_IDA_RU.lst}

В MIPS нет регистра флагов, так что эта информация не присутствует после исполнения арифметических операций.

Так что здесь нет инструкций как \INS{ADC} или \INS{SBB} в x86.
Чтобы получить информацию о том, был бы выставлен флаг переноса, происходит сравнение (используя инструкцию
\INS{SLTU}), которая выставляет целевой регистр в 1 или 0.

Эта 1 или 0 затем прибавляется к итоговому результату, или вычитается.

}
\DE{\subsection{Übergabe von Argumenten bei Addition und Subtraktion}

\lstinputlisting[style=customc]{patterns/185_64bit_in_32_env/passing_add_sub/1.c}

\subsubsection{x86}

\lstinputlisting[caption=\Optimizing MSVC 2012 /Ob1,style=customasmx86]{patterns/185_64bit_in_32_env/passing_add_sub/1_MSVC.asm}
Wir sehen, dass in der Funktion \GTT{f\_add\_test()} jeder 64-Bit-Wert über zwei 32-Bit-Werten übergeben wird: zuerst
der höhere Teil, dann der niedere Teil.

Addition und Subtraktion werden auch mit Paaren ausgeführt.

\myindex{x86!\Instructions!ADC}
Bei einer Addition werden die niederen 32-Bit-Teile zuerst addiert.
Tritt hierbei ein Übertrag auf, wird das \CF Flag gesetzt.

Der folgende \INS{ADC} Befehl addiert die höheren Teile der Operanden und addiert 1, falls $CF=1$.

\myindex{x86!\Instructions!SBB}
Subtraktion wird auch mit den Wertepaaren durchgeführt.
Das erste \SUB setzt ggf. das \CF Flag, das von dem folgenden \INS{SBB} Befehl geprüft wird:
wenn das Carryflag gesetzt ist, wird am Ende 1 vom Ergebnis abgezogen.

Man erkennt im Code leicht, wie das Ergebnis der Funktion \GTT{f\_add()} an \printf übergeben wird.

\lstinputlisting[caption=GCC 4.8.1 -O1 -fno-inline,style=customasmx86]{patterns/185_64bit_in_32_env/passing_add_sub/1_GCC.asm}

Der Code von GCC ist identisch.

\subsubsection{ARM}

\lstinputlisting[caption=\OptimizingKeilVI (\ARMMode),style=customasmARM]{patterns/185_64bit_in_32_env/passing_add_sub/Keil_ARM_O3.s}

\myindex{ARM!\Instructions!ADDS}
\myindex{ARM!\Instructions!SUBS}
\myindex{ARM!\Instructions!ADC}
\myindex{ARM!\Instructions!SBC}
Der erste 64-Bit-Wert wird über das Registerpaar \Reg{0} und \Reg{1} übergeben, der zweite über \Reg{2} und \Reg{3}.
ARM verfügt ebenfalls über die Befehle \INS{ADC} und \INS{SBC} (die das Carryflag beachten).
Man beachte: wenn die niederen Teile addiert bzw. subtrahiert werden, werden \INS{ADDS} und \INS{SUBS} Befehle mit dem
Suffy -S verwendet. Dieser Suffix steht für \q{set flags} (dt. setze Flags) und wird von den folgenden
\INS{ADC}/\INS{SBC} Befehlen unbedingt benötigt. Würden die Flags nicht weiter beachtet werden, könnten hier \ADD und
\SUB verwendet werden.

\subsubsection{MIPS}

\lstinputlisting[caption=\Optimizing GCC 4.4.5
(IDA),style=customasmMIPS]{patterns/185_64bit_in_32_env/passing_add_sub/MIPS_O3_IDA_DE.lst}
MIPS besitzt kein Register für die Flags, sodass keine derartige Information nach der Ausführung von arithmetischen
Operationen verfügbar ist.
Es gibt also keine Befehle wie \INS{ADC} oder \INS{SBB} in x86.
Um zu prüfen, ob das Carryflag gesetzt werden muss, wird ein \INS{SLTU} Befehl verwendet, der das Zielregister auf 1
oder 0 setzt. Diese 1 oder 0 wird dann zum Ergebnis addiert bzw. davon subtrahiert.

}

\section{\RU{Умножение, деление}\EN{Multiplication, division}}

\lstinputlisting{patterns/185_64bit_in_32_env/multdiv/2.c}

\subsection{x86}

\lstinputlisting[caption=\Optimizing MSVC 2013 /Ob1]{patterns/185_64bit_in_32_env/multdiv/2_MSVC.asm.\LANG}

\RU{Умножение и деление --- это более сложная операция, так что обычно, компилятор встраивает вызовы библиотечных функций,
делающих это}\EN{Multiplication and division are more complex operations, so usually the compiler embeds calls to
a library functions doing that}.

\ifx\LITE\undefined
\RU{Значение этих библиотечных функций, здесь}\EN{These functions are described here}: \myref{sec:MSVC_library_func}.
\fi

\ifdefined\IncludeGCC
\lstinputlisting[caption=\Optimizing GCC 4.8.1 -fno-inline]{patterns/185_64bit_in_32_env/multdiv/2_GCC.asm.\LANG}

\RU{GCC делает почти то же самое, тем не менее,
встраивает код умножения прямо в функцию, посчитав что так будет эффективнее}\EN{GCC does the expected, but the multiplication
code is inlined right in the function, thinking it could be more efficient}.
\RU{У GCC другие имена библиотечных функций}\EN{GCC has different library function names}: \myref{sec:GCC_library_func}.
\fi

\ifdefined\IncludeARM
\subsection{ARM}

\RU{Keil для режима Thumb вставляет вызовы библиотечных функций:}
\EN{Keil for Thumb mode inserts library subroutine calls:}

\lstinputlisting[caption=\OptimizingKeilVI (\ThumbMode)]{patterns/185_64bit_in_32_env/multdiv/Keil_thumb_O3.s}

\RU{Keil для режима ARM, тем не менее, может сгенерировать код для умножения 64-битных чисел:}
\EN{Keil for ARM mode, on the other hand, is able to produce 64-bit multiplication code:}

\lstinputlisting[caption=\OptimizingKeilVI (\ARMMode)]{patterns/185_64bit_in_32_env/multdiv/Keil_ARM_O3.s}
% TODO add explanation
\fi

\ifdefined\IncludeMIPS
\subsection{MIPS}

\Optimizing GCC \ForENRU MIPS 
\EN{can generate 64-bit multiplication code, but has to call a library routine for 64-bit division:}
\RU{может генерировать код для 64-битного умножения, но для 64-битного деления приходится вызывать библиотечную функцию:}

\lstinputlisting[caption=\Optimizing GCC 4.4.5 (IDA)]{patterns/185_64bit_in_32_env/multdiv/MIPS_O3_IDA.lst}

\RU{Тут также много \ac{NOP}-ов, это возможно заполнение delay slot-ов после инструкции умножения (она ведь работает
медленнее прочих инструкций).}
\EN{There are a lot of \ac{NOP}s, probably delay slots filled after the multiplication instruction (it's slower
than other instructions, after all).}

% TODO add explanation
\fi

\EN{\subsection{Shifting right}

\lstinputlisting{patterns/185_64bit_in_32_env/shifting/3.c}

\subsubsection{x86}

\lstinputlisting[caption=\Optimizing MSVC 2012 /Ob1]{patterns/185_64bit_in_32_env/shifting/3_MSVC.asm}

\lstinputlisting[caption=\Optimizing GCC 4.8.1 -fno-inline]{patterns/185_64bit_in_32_env/shifting/3_GCC.asm}

\myindex{x86!\Instructions!SHRD}

Shifting also occurs in two passes: first the lower part is shifted, then the higher part.
But the lower part is shifted with the help of the \INS{SHRD} instruction, it shifts the value of \EDX{} by 7 bits, but pulls new bits
from \EAX{}, i.e., from the higher part.
The higher part is shifted using the more popular \SHR{} instruction: indeed, the freed bits in the higher part
must be filled with zeroes.

\subsubsection{ARM}

ARM doesn't have such instruction as \INS{SHRD} in x86, so the Keil compiler ought to do this using simple shifts and \INS{OR} operations:

\lstinputlisting[caption=\OptimizingKeilVI (\ARMMode)]{patterns/185_64bit_in_32_env/shifting/Keil_ARM_O3.s}

\lstinputlisting[caption=\OptimizingKeilVI (\ThumbMode)]{patterns/185_64bit_in_32_env/shifting/Keil_thumb_O3.s}
% TODO add explanation

\subsubsection{MIPS}

GCC for MIPS follows the same algorithm as Keil does for Thumb mode:

\lstinputlisting[caption=\Optimizing GCC 4.4.5 (IDA)]{patterns/185_64bit_in_32_env/shifting/MIPS_O3_IDA.lst}

% TODO add explanation

}
\RU{\sectionold{Сдвиг вправо}

\lstinputlisting{patterns/185_64bit_in_32_env/shifting/3.c}

\subsectionold{x86}

\lstinputlisting[caption=\Optimizing MSVC 2012 /Ob1]{patterns/185_64bit_in_32_env/shifting/3_MSVC.asm}

\lstinputlisting[caption=\Optimizing GCC 4.8.1 -fno-inline]{patterns/185_64bit_in_32_env/shifting/3_GCC.asm}

\myindex{x86!\Instructions!SHRD}
Сдвиг происходит также в две операции: в начале сдвигается младшая часть, затем старшая.
Но младшая часть сдвигается при помощи инструкции \INS{SHRD}, она сдвигает значение в \EDX{} на 7 бит, но подтягивает новые биты из \EAX{}, т.е. из старшей части.
Старшая часть сдвигается более известной инструкцией \SHR{}: действительно, ведь освободившиеся биты в старшей части нужно
просто заполнить нулями.

\subsectionold{ARM}

В ARM нет такой инструкции как \INS{SHRD} в x86, так что компилятору Keil приходится всё это делать,
используя простые сдвиги и операции \q{ИЛИ}:

\lstinputlisting[caption=\OptimizingKeilVI (\ARMMode)]{patterns/185_64bit_in_32_env/shifting/Keil_ARM_O3.s}

\lstinputlisting[caption=\OptimizingKeilVI (\ThumbMode)]{patterns/185_64bit_in_32_env/shifting/Keil_thumb_O3.s}
% TODO add explanation

\subsectionold{MIPS}

GCC для MIPS реализует тот же алгоритм, что сделал Keil для режима Thumb:

\lstinputlisting[caption=\Optimizing GCC 4.4.5 (IDA)]{patterns/185_64bit_in_32_env/shifting/MIPS_O3_IDA.lst}

% TODO add explanation

}
\DE{\subsection{Verschiebung nach rechts}

\lstinputlisting[style=customc]{patterns/185_64bit_in_32_env/shifting/3.c}

\subsubsection{x86}

\lstinputlisting[caption=\Optimizing MSVC 2012 /Ob1,style=customasmx86]{patterns/185_64bit_in_32_env/shifting/3_MSVC.asm}

\lstinputlisting[caption=\Optimizing GCC 4.8.1 -fno-inline,style=customasmx86]{patterns/185_64bit_in_32_env/shifting/3_GCC.asm}

\myindex{x86!\Instructions!SHRD}
Das Verschieben geschieht ebenfalls zweigeteilt: zunächst wird der niedere Teil verschoben, danach der höhere.
Der niedere Teil wird mithilfe des Befehls \INS{SHRD} verschoben; er verschiebt den Wert in \EAX um 7 Bits, holt aber
die nachrutschenden Bits aus \EDX, d.h. aus dem höheren Teil.
Mit anderen Worten: der 64-Bit-Wert aus \TT{EDX:EAX} wird als ganzes um 7 Bits verschoben und die niederen 32 Bits des
Ergebnisses werden in \EAX abgelegt. Der höhere Teil wird mit dem häufig verwendeten \SHR Befehl verschoben, da die frei
werdenden Bits im höheren Teil mit Nullen aufgefüllt werden müssen.

\subsubsection{ARM}
ARM verfügt im Gegensatz zu x86 nicht über einen \INS{SHRD} Befehl, sodass der Keil Compiler die Aufgabe mit einer
Kombination aus einfachen Schiebebefehlen und \OR-Operationen durchführen muss:

\lstinputlisting[caption=\OptimizingKeilVI (\ARMMode),style=customasmARM]{patterns/185_64bit_in_32_env/shifting/Keil_ARM_O3.s}

\lstinputlisting[caption=\OptimizingKeilVI (\ThumbMode),style=customasmARM]{patterns/185_64bit_in_32_env/shifting/Keil_thumb_O3.s}
% TODO add explanation

\subsubsection{MIPS}
GCC für MIPS folgt dem gleichen Algorithmus wie Keil für Thumb mode:

\lstinputlisting[caption=\Optimizing GCC 4.4.5 (IDA),style=customasmMIPS]{patterns/185_64bit_in_32_env/shifting/MIPS_O3_IDA.lst}

% TODO add explanation

}
\section{\RU{Конвертирование 32-битного значения в 64-битное}\EN{Converting 32-bit value into 64-bit one}}
\label{subsec:sign_extending_32_to_64}

\lstinputlisting{patterns/185_64bit_in_32_env/conversion/4.c}

\subsection{x86}

\lstinputlisting[caption=\Optimizing MSVC 2012]{patterns/185_64bit_in_32_env/conversion/MSVC2012_Ox.asm}

\RU{Здесь появляется необходимость расширить 32-битное знаковое значение в 64-битное знаковое.}
\EN{Here we also run into necessity to extend 32-bit signed value into 64-bit signed.}
\RU{Конвертировать беззнаковые значения очень просто: нужно просто выставить в 0 все биты в старшей части}
\EN{Unsigned values are converted straightforwardly: all bits in higher part must be set to 0}.
\RU{Но для знаковых типов это не подходит: знак числа должен быть скопирован в старшую часть числа-результата}
\EN{But it is not appropriate for signed data types: sign should be copied into higher part of resulting number}.
\index{x86!\Instructions!CDQ}
\RU{Здесь это делает инструкция \TT{CDQ}, она берет входное значение в \EAX{}, расширяет его до 64-битного,
и оставляет его в паре регистров \EDX{}:\EAX{}}
\EN{\TT{CDQ} instruction doing that here, it takes input value in \EAX{}, extending it to 64-bit and leaving it
in the \EDX{}:\EAX{} registers pair}.
\RU{Иными словами, инструкция \TT{CDQ} узнает знак числа в \EAX{} (просто берет самый старший бит в \EAX{}) и в зависимости от этого,
выставляет все 32 бита в \EDX{} в 0 или в 1}\EN{In other words, \TT{CDQ} instruction gets number sign in \EAX{} (by getting just
most significant bit in \EAX{}), and depending of it, setting all 32-bits in \EDX{} to 0 or 1}.
\RU{Её работа в каком-то смысле напоминает работу инструкции \MOVSX{}}\EN{Its operation is somewhat
similar to the \MOVSX{} instruction}.

\ifdefined\IncludeARM
\subsection{ARM}

\lstinputlisting[caption=\OptimizingKeilVI (\ARMMode)]{patterns/185_64bit_in_32_env/conversion/Keil_ARM_O3.s}

\RU{Keil для ARM работает иначе: он просто сдвигает (арифметически) входное значение на 31 бит вправо.}
\EN{Keil for ARM is different: it just arithmetically shifts input value by 31 bit right.}
\RU{Как мы знаем, бит знака это \ac{MSB}, и арифметический сдвиг копирует бит знака в ``появляющихся'' битах.}
\EN{As we know, sign bit is \ac{MSB}, and arithmetical shift copies sign bit into ``emerged'' bits.}
\RU{Так что после инструкции ``ASR r1,r0,\#31'', R1 будет содержать 0xFFFFFFFF если входное значение
было отрицательным, или 0 в противном случае.}
\EN{So after ``ASR r1,r0,\#31'' instruction, R1 will contain 0xFFFFFFFF if input value was negative
and 0 otherwise.}
\RU{R1 содержит старшую часть возвращаемого 64-битного значения.}
\EN{R1 contain high part of resulting 64-bit value.}

\RU{Другими словами, этот код просто копирует \ac{MSB} (бит знака) из входного значения в R0 во все
биты старшей 32-битной части итогового 64-битного значения.}
\EN{In other words, this code just copies \ac{MSB} (sign bit) from input value in R0 into all bits
of high 32-bit part of resulting 64-bit value.}

\fi



\section{SIMD}

\label{SIMD_x86}
\ac{SIMD} \IFRU{это акроним:}{is just acronym:} \IT{Single Instruction, Multiple Data}.

\IFRU{Как можно судить по названию, это обработка множества данных исполняя только одну инструкцию.}
{As it is said, it is multiple data processing using only one instruction.}

\IFRU{Как и \ac{FPU}, эта подсистема процессора выглядит также отдельным процессором внутри x86.}
{Just as \ac{FPU}, that \ac{CPU} subsystem looks like separate processor inside x86.}

\index{x86!MMX}
\IFRU{SIMD в x86 начался с MMX. Появилось 8 64-битных регистров MM0-MM7.}
{SIMD began as MMX in x86. 8 new 64-bit registers appeared: MM0-MM7.}

\IFRU{Каждый MMX-регистр может содержать 2 32-битных значения, 4 16-битных или же 8 байт. 
Например, складывая значения двух MMX-регистров, можно складывать одновременно 8 8-битных значений.}
{Each MMX register may hold 2 32-bit values, 4 16-bit values or 8 bytes.
For example, it is possible to add 8 8-bit values (bytes) simultaneously by adding two values in MMX-registers.}

\IFRU{Простой пример, это некий графический редактор, который хранит открытое изображение как двумерный массив. 
Когда пользователь меняет яркость изображения, редактору нужно, например, прибавить некий коэффициент 
ко всем пикселям, или отнять. 
Для простоты можно представить, что изображение у нас бело-серо-черное и каждый пиксель занимает один байт, 
то с помощью MMX можно менять яркость сразу у восьми пикселей.}
{One simple example is graphics editor, representing image as a two dimensional array.
When user change image brightness, the editor must add a coefficient to each pixel value, or to subtract.
For the sake of brevity, our image may be grayscale and each pixel defined by one 8-bit byte, then it is possible
to change brightness of 8 pixels simultaneously.}

\IFRU{Когда MMX только появилось, эти регистры на самом деле располагались в FPU-регистрах. 
Можно было использовать 
либо FPU либо MMX в одно и то же время. Можно подумать, что Intel решило немного сэкономить на транзисторах, 
но на самом деле причина такого симбиоза проще ~--- более старая \ac{OS} не знающая о дополнительных 
регистрах процессора не будет сохранять их во время переключения задач, а вот регистры FPU сохранять будет. 
Таким образом, процессор с MMX + старая \ac{OS} + задача использующая возможности MMX = все 
это может работать вместе.}
{When MMX appeared, these registers was actually located in FPU registers. 
It was possible to use either FPU or MMX at the same time. One might think, Intel saved on transistors,
but in fact, the reason of such symbiosis is simpler~---older \ac{OS} may not aware 
of additional CPU registers would not save them at the context switching, but will save FPU registers.
Thus, MMX-enabled CPU + old \ac{OS} + process utilizing MMX features = that all will work together.}

\index{x86!SSE}
\index{x86!SSE2}
SSE\EMDASH\IFRU{это расширение регистров до 128 бит, теперь уже отдельно от FPU.}{is extension of SIMD registers up to 128 bits, now separately from FPU.}

\index{x86!AVX}
AVX\EMDASH\IFRU{расширение регистров до 256 бит.}{another extension to 256 bits.}

\IFRU{Немного о практическом применении.}{Now about practical usage.}

\IFRU{Конечно же, копирование блоков в памяти (\TT{memcpy}), сравнение (\TT{memcmp}), и подобное.}
{Of course, memory copying (\TT{memcpy}), memory comparing (\TT{memcmp}) and so on.}

\index{DES}
\IFRU{Еще пример: имеется алгоритм шифрования DES, который берет 64-битный блок, 56-битный ключ, 
шифрует блок с ключом и образуется 64-битный результат.
Алгоритм DES можно легко представить в виде очень большой электронной цифровой схемы, 
с проводами, элементами И, ИЛИ, НЕ.}
{One more example: we got DES encryption algorithm, it takes 64-bit block, 56-bit key, encrypt block and produce 64-bit result.
DES algorithm may be considered as a very large electronic circuit, with wires and AND/OR/NOT gates.}

\label{bitslicedes}
\newcommand{\URLBS}{\url{http://www.darkside.com.au/bitslice/}}

\IFRU{Идея bitslice DES\footnote{\URLBS} ~--- это обработка сразу группы блоков и ключей одновременно. 
Скажем, на x86 переменная типа \IT{unsigned int} вмещает в себе 32 бита, так что там можно хранить 
промежуточные результаты сразу для 32-х блоков-ключей, используя 64+56 переменных типа \IT{unsigned int}.}
{Bitslice DES\footnote{\URLBS}~---is an idea of processing group of blocks and keys simultaneously.
Let's say, variable of type \IT{unsigned int} on x86 may hold up to 32 bits, so, it is possible to store there
intermediate results for 32 blocks-keys pairs simultaneously, using 64+56 variables of \IT{unsigned int} type.}

\index{Oracle RDBMS}
\IFRU{Я написал утилиту для перебора паролей/хешей Oracle RDBMS (которые основаны на алгоритме DES), 
переделав алгоритм bitslice DES для SSE2 и AVX ~--- и теперь возможно шифровать одновременно 
128 или 256 блоков-ключей:}
{I wrote an utility to brute-force Oracle RDBMS passwords/hashes (ones based on DES),
slightly modified bitslice DES algorithm for SSE2 and AVX~---now it is possible to encrypt 128 
or 256 block-keys pairs simultaneously.}

\url{http://conus.info/utils/ops_SIMD/}
 
\subsection{\IFRU{Векторизация}{Vectorization}}

\newcommand{\URLVEC}{\href{http://en.wikipedia.org/wiki/Vectorization_(computer_science)}{Wikipedia: vectorization}}

\IFRU{Векторизация\footnote{\URLVEC} это когда у вас есть цикл, который берет на вход несколько массивов и выдает, 
например, один массив данных. 
Тело цикла берет некоторые элементы из входных массивов, что-то делает с ними и помещает в выходной. 
Важно, что операция применяемая ко всем элементам одна и та же. 
Векторизация ~--- это обрабатывать несколько элементов одновременно.}
{Vectorization\footnote{\URLVEC}, for example, is when you have a loop taking couple of arrays at input and produces one array.
Loop body takes values from input arrays, do something and put result into output array.
It is important that there is only one single operation applied to each element.
Vectorization~---is to process several elements simultaneously.}

\IFRU{Векторизация ~--- это не самая новая технология: автор сих строк видел её по крайней мере на 
линейке суперкомпьютеров Cray Y-MP от 1988, когда работал на его версии-``лайт'' Cray Y-MP EL
\footnote{Удаленно. Он находится в музее суперкомпьютеров: \url{http://www.cray-cyber.org}}}
{Vectorization is not very fresh technology: author of this textbook saw it at least on Cray Y-MP 
supercomputer line from 1988 when played with its ``lite'' version Cray Y-MP EL
\footnote{Remotely. It is installed in the museum of supercomputers: \url{http://www.cray-cyber.org}}}.

\IFRU{Например:}{For example:}

\begin{lstlisting}
for (i = 0; i < 1024; i++)
{
    C[i] = A[i]*B[i];
}
\end{lstlisting}

\IFRU{Этот фрагмент кода берет элементы из A и B, перемножает и сохраняет результат в C.}
{This fragment of code takes elements from A and B, multiplies them and save result into C.}

\index{x86!\Instructions!PLMULLD}
\index{x86!\Instructions!PLMULHW}
\newcommand{\PMULLD}{\IT{PMULLD} (\IT{\IFRU{Перемножить запакованные знаковые DWORD и сохранить младшую часть результата}
{Multiply Packed Signed Dword Integers and Store Low Result}})}
\newcommand{\PMULHW}{\TT{PMULHW} (\IT{\IFRU{Перемножить запакованные знаковые DWORD и сохранить старшую часть результата}
{Multiply Packed Signed Integers and Store High Result}})}

\IFRU{Если представить, что каждый элемент массива ~--- это 32-битный \Tint, то их можно загружать сразу 
по 4 из А в 128-битный XMM-регистр, 
из B в другой XMM-регистр и выполнив инструкцию \PMULLD{} и \PMULHW{}, можно получить 4 64-битных 
\glslink{product}{произведения} сразу.}
{If each array element we have is 32-bit \Tint, then it is possible to load 4 elements from A into 128-bit 
XMM-register, from B to another XMM-registers, and by executing \PMULLD{} and \PMULHW{}, 
it is possible to get 4 64-bit \glspl{product} at once.}

\IFRU{Таким образом, тело цикла исполняется $1024/4$ раза вместо 1024, что в 4 раза меньше, и, конечно, быстрее.}
{Thus, loop body count is $1024/4$ instead of $1024$, that is 4 times less and, of course, faster.}

\newcommand{\URLINTELVEC}{\href{http://www.intel.com/intelpress/sum_vmmx.htm}{Excerpt: Effective Automatic Vectorization}}

\index{Intel C++}
\IFRU{Некоторые компиляторы умеют делать автоматическую векторизацию в простых случаях, 
например Intel C++\footnote{Еще о том, как Intel C++ умеет автоматически векторизовать циклы: \URLINTELVEC}.}
{Some compilers can do vectorization automatically in a simple cases, 
e.g., Intel C++\footnote{More about Intel C++ automatic vectorization: \URLINTELVEC}.}

\IFRU{Я написал очень простую функцию:}{I wrote tiny function:}

\begin{lstlisting}
int f (int sz, int *ar1, int *ar2, int *ar3)
{
	for (int i=0; i<sz; i++)
		ar3[i]=ar1[i]+ar2[i];

	return 0;
};
\end{lstlisting}

\subsubsection{Intel C++}

\IFRU{Компилирую при помощи}{Let's compile it with} Intel C++ 11.1.051 win32:

\begin{verbatim}
icl intel.cpp /QaxSSE2 /Faintel.asm /Ox
\end{verbatim}

\IFRU{Имеем такое (в \IDA):}{We got (in \IDA):}

\lstinputlisting{patterns/19_SIMD/18_1_en.asm}

\IFRU{Инструкции, имеющие отношение к SSE2 это:}{SSE2-related instructions are:}
\index{x86!\Instructions!MOVDQA}
\index{x86!\Instructions!MOVDQU}
\index{x86!\Instructions!PADDD}
\begin{itemize}
\item
\MOVDQU (\IT{Move Unaligned Double Quadword})\EMDASH\IFRU{она просто загружает 16 байт из памяти в XMM-регистр}
{it just load 16 bytes from memory into a XMM-register}.

\item
\PADDD (\IT{Add Packed Integers})\EMDASH\IFRU{складывает сразу 4 пары 32-битных чисел и оставляет в первом операнде результат. 
Кстати, если произойдет переполнение, то исключения не произойдет и никакие флаги не установятся, 
запишутся просто младшие 32 бита результата. 
Если один из операндов \PADDD ~--- адрес значения в памяти, 
то требуется чтобы адрес был выровнен по 16-байтной границе. Если он не выровнен, произойдет исключение
\footnote{О выравнивании данных см. также: \URLWPDA}.}
{adding 4 pairs of 32-bit numbers and leaving result in first operand.
By the way, no exception raised in case of overflow and no flags will be set, just low 32-bit of result will
be stored.
If one of \PADDD operands is address of value in memory,
then address must be aligned on a 16-byte boundary. If it is not aligned, exception will be occurred
\footnote{More about data aligning: \URLWPDA}.}

\item
\MOVDQA (\IT{Move Aligned Double Quadword})\EMDASH\IFRU{тоже что и \MOVDQU, только подразумевает 
что адрес в памяти выровнен по 16-байтной границе. 
Если он не выровнен, произойдет исключение. 
\MOVDQA работает быстрее чем \MOVDQU, но требует вышеозначенного.}
{the same as \MOVDQU, but requires address of value in memory to be aligned on a 16-bit border.
If it is not aligned, exception will be raised.
\MOVDQA works faster than \MOVDQU, but requires aforesaid.}

\end{itemize}

\IFRU{Итак, эти SSE2-инструкции исполнятся только в том случае если еще осталось просуммировать 
4 пары переменных типа \Tint плюс если указатель \TT{ar3} выровнен по 16-байтной границе.}
{So, these SSE2-instructions will be executed only in case if there are more 4 pairs to work on
plus pointer \TT{ar3} is aligned on a 16-byte boundary.}

\IFRU{Более того, если еще и \TT{ar2} выровнен по 16-байтной границе, то будет выполняться этот фрагмент кода:}
{More than that, if \TT{ar2} is aligned on a 16-byte boundary as well, this fragment of code will be executed:}

\begin{lstlisting}
movdqu  xmm0, xmmword ptr [ebx+edi*4] ; ar1+i*4
paddd   xmm0, xmmword ptr [esi+edi*4] ; ar2+i*4
movdqa  xmmword ptr [eax+edi*4], xmm0 ; ar3+i*4
\end{lstlisting}

\IFRU{А иначе, значение из \TT{ar2} загрузится в \XMMZERO используя инструкцию \MOVDQU, 
которая не требует выровненного указателя, зато может работать чуть медленнее:}
{Otherwise, value from \TT{ar2} will be loaded into \XMMZERO using \MOVDQU,
it does not require aligned pointer, but may work slower:}

\begin{lstlisting}
movdqu  xmm1, xmmword ptr [ebx+edi*4] ; ar1+i*4
movdqu  xmm0, xmmword ptr [esi+edi*4] ; ar2+i*4 is not 16-byte aligned, so load it to xmm0
paddd   xmm1, xmm0
movdqa  xmmword ptr [eax+edi*4], xmm1 ; ar3+i*4
\end{lstlisting}

\IFRU{А во всех остальных случаях, будет исполняться код, который был бы, как если бы не была 
включена поддержка SSE2.}
{In all other cases, non-SSE2 code will be executed.}

\subsubsection{GCC}

\newcommand{\URLGCCVEC}{\url{http://gcc.gnu.org/projects/tree-ssa/vectorization.html}}

\IFRU{Но и GCC умеет кое-что векторизировать\footnote{Подробнее о векторизации в GCC: \URLGCCVEC}, 
если компилировать с опциями \Othree и включить поддержку SSE2: \TT{-msse2}.}
{GCC may also vectorize in a simple cases\footnote{More about GCC vectorization support: \URLGCCVEC},
if to use \Othree option and to turn on SSE2 support: \TT{-msse2}.}

\IFRU{Вот что вышло}{What we got} (GCC 4.4.1):

\lstinputlisting{patterns/19_SIMD/18_2_gcc_O3.asm}

\IFRU{Почти то же самое, хотя и не так дотошно как Intel C++.}
{Almost the same, however, not as meticulously as Intel C++ doing it.}

\subsection{\IFRU{Реализация \strlen при помощи SIMD}{SIMD \strlen implementation}}

\newcommand{\URLMSDNSSE}{\href{http://msdn.microsoft.com/en-us/library/y0dh78ez(VS.80).aspx}{MSDN: MMX, SSE, and SSE2 Intrinsics}}

\IFRU{Прежде всего, следует заметить, что SIMD-инструкции можно вставлять в \CCpp код при помощи специальных 
макросов\footnote{\URLMSDNSSE}. В MSVC, часть находится в файле \TT{intrin.h}.}
{It should be noted the \ac{SIMD}-instructions may be inserted into \CCpp code via 
special macros\footnote{\URLMSDNSSE}.
As of MSVC, some of them are located in the \TT{intrin.h} file.}

\index{\CStandardLibrary!strlen()}
\IFRU{Имеется возможность реализовать функцию \strlen\footnote{strlen() ~--- стандартная функция Си 
для подсчета длины строки} при помощи SIMD-инструкций, работающий в 2-2.5 раза быстрее обычной реализации. 
Эта функция будет загружать в XMM-регистр сразу 16 байт и проверять каждый на ноль.}
{It is possible to implement \strlen function\footnote{strlen()~---standard C library function for calculating
string length} using SIMD-instructions, working 2-2.5 times faster than common implementation.
This function will load 16 characters into a XMM-register and check each against zero.}

\lstinputlisting{patterns/19_SIMD/18_3.c}

\newcommand{\URLSTRLEN}{http://www.strchr.com/sse2\_optimised\_strlen}

\IFRU{(пример базируется на исходнике \href{\URLSTRLEN}{отсюда}).}
{(the example is based on source code from \href{\URLSTRLEN}{there}).}

\IFRU{Компилируем в MSVC 2010 с опцией \Ox:}{Let's compile in MSVC 2010 with \Ox option:}

\lstinputlisting{patterns/19_SIMD/18_4_msvc_Ox.asm}

\IFRU{Итак, прежде всего, мы проверяем указатель \TT{str}, выровнен ли он по 16-байтной границе. 
Если нет, то мы вызовем обычную реализацию \strlen.}
{First of all, we check \TT{str} pointer, if it is aligned on a 16-byte boundary.
If not, let's call generic \strlen implementation.}

\IFRU{Далее мы загружаем по 16 байт в регистр \XMMONE при помощи команды \MOVDQA.}
{Then, load next 16 bytes into the \XMMONE register using \MOVDQA instruction.}

\IFRU{Наблюдательный читатель может спросить, почему в этом месте мы не можем использовать \MOVDQU, 
которая может загружать откуда угодно не взирая на факт, выровнен ли указатель?}
{Observant reader might ask, why \MOVDQU cannot be used here since it can load data from the memory
regardless the fact if the pointer aligned or not.}

\IFRU{Да, можно было бы сделать вот как: если указатель выровнен, загружаем используя \MOVDQA, 
иначе используем работающую чуть медленнее \MOVDQU.}
{Yes, it might be done in this way: if pointer is aligned, load data using \MOVDQA,
if not~---use slower \MOVDQU.}

\IFRU{Однако здесь кроется не сразу заметная проблема, которая проявляется вот в чем:}
{But here we are may stick into hard to notice caveat:}

\index{Page (memory)}
\newcommand{\URLPAGE}{\url{http://en.wikipedia.org/wiki/Page_(computer_memory)}}

\IFRU{В \ac{OS} линии \gls{Windows NT}, и не только, память выделяется страницами по 4 KiB (4096 байт). 
Каждый win32-процесс якобы имеет в наличии 4 GiB, но на самом деле, 
только некоторые части этого адресного пространства присоединены к реальной физической памяти. 
Если процесс обратится к блоку памяти, которого не существует, сработает исключение. 
Так работает виртуальная память\footnote{\URLPAGE}.}
{In \gls{Windows NT} line of \ac{OS} but not limited to it, memory allocated by pages of 4 KiB (4096 bytes).
Each win32-process has ostensibly 4 GiB, but in fact, only some parts
of address space are connected to real physical memory.
If the process accessing to the absent memory block, exception will be raised.
That's how virtual memory works\footnote{\URLPAGE}.}

\IFRU{Так вот, функция, читающая сразу по 16 байт, имеет возможность нечаянно вылезти за границу 
выделенного блока памяти. 
Предположим, \ac{OS} выделила программе 8192 (0x2000) байт по адресу 0x008c0000. 
Таким образом, блок занимает байты с адреса 0x008c0000 по 0x008c1fff включительно.}
{So, a function loading 16 bytes at once, may step over a border of allocated memory block.
Let's consider, \ac{OS} allocated 8192 (0x2000) bytes at the address 0x008c0000.
Thus, the block is the bytes starting from address 0x008c0000 to 0x008c1fff inclusive.}

\IFRU{За этим блоком, то есть начиная с адреса 0x008c2000 нет вообще ничего, т.е., \ac{OS} не выделяла там память. 
Обращение к памяти начиная с этого адреса вызовет исключение.}
{After the block, that is, starting from address 0x008c2000 there is nothing at all, e.g., \ac{OS} not allocated
any memory there.
Attempt to access a memory starting from the address will raise exception.}

\IFRU{И предположим, что программа хранит некую строку из, скажем, пяти символов почти в самом конце блока, 
что не является преступлением:}
{And let's consider, the program holding a string containing 5 characters almost at the end of block,
and that is not a crime.}

\begin{center}
  \begin{tabular}{ | l | l | }
    \hline
        0x008c1ff8 & 'h' \\
        0x008c1ff9 & 'e' \\
        0x008c1ffa & 'l' \\
        0x008c1ffb & 'l' \\
        0x008c1ffc & 'o' \\
        0x008c1ffd & '\textbackslash{}x00' \\
        0x008c1ffe & \IFRU{здесь случайный мусор}{random noise} \\
        0x008c1fff & \IFRU{здесь случайный мусор}{random noise} \\
    \hline
  \end{tabular}
\end{center}

\IFRU{В обычных условиях, программа вызывает \strlen передав ей указатель на строку \TT{'hello'} 
лежащую по адресу 0x008c1ff8. 
\strlen будет читать по одному байту до 0x008c1ffd, где ноль, и здесь она закончит работу.}
{So, in common conditions the program calling \strlen passing it a pointer to string \TT{'hello'} 
lying in memory at address 0x008c1ff8.
\strlen will read one byte at a time until 0x008c1ffd, where zero-byte, and so here it will stop working.}

\IFRU{Теперь, если мы напишем свою реализацию \strlen читающую сразу по 16 байт, с любого адреса, 
будь он выровнен по 16-байтной границе или нет, 
\MOVDQU попытается загрузить 16 байт с адреса 0x008c1ff8 по 0x008c2008, и произойдет исключение. 
Это ситуация которой, конечно, хочется избежать.}
{Now if we implement our own \strlen reading 16 byte at once, starting at any address, will it be aligned or not,
\MOVDQU may attempt to load 16 bytes at once at address 0x008c1ff8 up to 0x008c2008, 
and then exception will be raised.
That's the situation to be avoided, of course.}

\IFRU{Поэтому мы будем работать только с адресами, выровненными по 16 байт, что в сочетании со знанием 
что размер страницы \ac{OS} также как правило выровнен по 16 байт, 
даст некоторую гарантию что наша функция не будет пытаться читать из мест в невыделенной памяти.}
{So then we'll work only with the addresses aligned on a 16 byte boundary, what in combination with a knowledge
of \ac{OS} page size is usually aligned on a 16-byte boundary too, give us some warranty our function will not
read from unallocated memory.}

\IFRU{Вернемся к нашей функции}{Let's back to our function}.

\index{x86!\Instructions!PXOR}
\verb|_mm_setzero_si128()|\EMDASH\IFRU{это макрос, генерирующий \TT{pxor xmm0, xmm0} ~--- инструкция просто обнуляет регистр \XMMZERO.}
{is a macro generating \TT{pxor xmm0, xmm0}~---instruction just clears the \XMMZERO register}

\verb|_mm_load_si128()|\EMDASH\IFRU{это макрос для \MOVDQA, он просто загружает 16 байт по адресу из указателя в \XMMONE.}
{is a macro for \MOVDQA, it just loading 16 bytes from the address in the \XMMONE register.}

\index{x86!\Instructions!PCMPEQB}
\verb|_mm_cmpeq_epi8()|\EMDASH\IFRU{это макрос для \PCMPEQB, это инструкция которая 
побайтово сравнивает значения из двух XMM регистров.} 
{is a macro for \PCMPEQB, is an instruction comparing two XMM-registers bytewise.}

\IFRU{И если какой-то из байт равен другому, то в результирующем значении будет выставлено на месте этого 
байта \TT{0xff}, либо 0, если байты не были равны.}
{And if some byte was equals to other, there will be \TT{0xff} at this point in the result or 0 if otherwise.}

\IFRU{Например.}{For example.}

\begin{verbatim}
XMM1: 11223344556677880000000000000000
XMM0: 11ab3444007877881111111111111111
\end{verbatim}

\IFRU{После исполнения \TT{pcmpeqb xmm1, xmm0}, регистр \XMMONE будет содержать:}
{After \TT{pcmpeqb xmm1, xmm0} execution, the \XMMONE register shall contain:}

\begin{verbatim}
XMM1: ff0000ff0000ffff0000000000000000
\end{verbatim}

\IFRU{Эта инструкция в нашем случае, сравнивает каждый 16-байтный блок с блоком состоящим из 16-и нулевых байт, 
выставленным в \XMMZERO при помощи \TT{pxor xmm0, xmm0}.}
{In our case, this instruction comparing each 16-byte block with the block of 16 zero-bytes,
was set in the \XMMZERO register by \TT{pxor xmm0, xmm0}.}

\index{x86!\Instructions!PMOVMSKB}
\IFRU{Следующий макрос \TT{\_mm\_movemask\_epi8()} ~--- это инструкция \TT{PMOVMSKB}.}
{The next macro is \TT{\_mm\_movemask\_epi8()}~---that is \TT{PMOVMSKB} instruction.}

\IFRU{Она очень удобна как раз для использования в паре с \PCMPEQB.}
{It is very useful if to use it with \PCMPEQB.}

\TT{pmovmskb eax, xmm1}

\IFRU{Эта инструкция выставит самый первый бит \EAX в единицу, если старший бит первого байта в 
регистре \XMMONE является единицей. 
Иными словами, если первый байт в регистре \XMMONE является \TT{0xff}, то первый бит в \EAX будет также единицей, 
иначе нулем.}
{This instruction will set first \EAX bit into 1 if most significant bit of the first byte in the \XMMONE is $1$.
In other words, if first byte of the \XMMONE register is \TT{0xff}, first \EAX bit will be set to 1 too.}

\IFRU{Если второй байт в регистре \XMMONE является \TT{0xff}, то второй бит в \EAX также будет единицей. 
Иными словами, инструкция отвечает на вопрос, \IT{какие из байт в \XMMONE являются \TT{0xff}?}
В результате приготовит 16 бит и запишет в \EAX. Остальные биты в \EAX обнулятся.}
{If second byte in the \XMMONE register is \TT{0xff}, then second \EAX bit will be set to 1 too.
In other words, the instruction is answer to the question \IT{which bytes in the \XMMONE are \TT{0xff?}}
And will prepare 16 bits in the \EAX register. Other bits in the \EAX register are to be cleared.}

\IFRU{Кстати, не забывайте также вот о какой особенности нашего алгоритма:}
{By the way, do not forget about this feature of our algorithm:}

\IFRU{На вход может прийти 16 байт вроде}{There might be 16 bytes on input like} \TT{hello\textbackslash{}x00garbage\textbackslash{}x00ab}

\IFRU{Это строка \TT{'hello'}, после нее терминирующий ноль, затем немного мусора в памяти.}
{It is a \TT{'hello'} string, terminating zero, and also a random noise in memory.}

\newcommand{\MSBFOOTNOTE}{\footnote{most significant bit}}
\newcommand{\LSBFOOTNOTE}{\footnote{least significant bit}}

\IFRU{Если мы загрузим эти 16 байт в \XMMONE и сравним с нулевым \XMMZERO, то в итоге получим такое 
(я использую здесь порядок с MSB\MSBFOOTNOTE до LSB\LSBFOOTNOTE):}
{If we load these 16 bytes into \XMMONE and compare them with zeroed \XMMZERO, we will get something like
(I use here order from MSB\MSBFOOTNOTE to LSB\LSBFOOTNOTE):}

\begin{verbatim}
XMM1: 0000ff00000000000000ff0000000000
\end{verbatim}

\IFRU{Это означает что инструкция сравнения обнаружила два нулевых байта, что и не удивительно.}
{This means, the instruction found two zero bytes, and that is not surprising.}

\IFRU{\TT{PMOVMSKB} в нашем случае подготовит \EAX вот так (в двоичном представлении):} 
{\TT{PMOVMSKB} in our case will prepare \EAX like (in binary representation):} \IT{0010000000100000b}.

\IFRU{Совершенно очевидно, что далее наша функция должна учитывать только первый встретившийся
нулевой бит и игнорировать все остальное.}
{Obviously, our function must consider only first zero bit and ignore the rest ones.}

\index{x86!\Instructions!BSF}
\label{instruction_BSF}
\IFRU{Следующая инструкция}{The next instruction}\EMDASH\TT{BSF} (\IT{Bit Scan Forward}). 
\IFRU{Это инструкция находит самый младший бит во втором операнде и записывает его позицию в первый операнд.}
{This instruction find first bit set to 1 and stores its position into first operand.}

\begin{verbatim}
EAX=0010000000100000b
\end{verbatim}

\IFRU{После исполнения этой инструкции \TT{bsf eax, eax}, в \EAX будет 5, что означает, 
что единица найдена в пятой позиции (считая с нуля).}
{After \TT{bsf eax, eax} instruction execution, \EAX will contain 5, this means, 
1 found at 5th bit position (starting from zero).}

\IFRU{Для использования этой инструкции, в MSVC также имеется макрос}
{MSVC has a macro for this instruction:} \TT{\_BitScanForward}.

\IFRU{А дальше все просто. Если нулевой байт найден, его позиция прибавляется к тому что 
мы уже насчитали и возвращается результат.}
{Now it is simple. If zero byte found, its position added to what we already counted and now we have 
ready to return result.}

\IFRU{Почти всё.}{Almost all.}

\IFRU{Кстати, следует также отметить, что компилятор MSVC сгенерировал два тела цикла сразу, для оптимизации.}
{By the way, it is also should be noted, MSVC compiler emitted two loop bodies side by side, for optimization.}

\IFRU{Кстати, в SSE 4.2 (который появился в Intel Core i7) все эти манипуляции со строками могут быть еще проще:}
{By the way, SSE 4.2 (appeared in Intel Core i7) offers more instructions where these string manipulations might be
even easier:} \url{http://www.strchr.com/strcmp\_and\_strlen\_using\_sse\_4.2}


\chapter{\IFRU{64 бита}{64 bits}}

\section{x86-64}
\index{x86-64}
\label{x86-64}

\IFRU{Это расширение x86-архитуктуры до 64 бит.}{It is a 64-bit extension to x86-architecture.}

\IFRU{С точки зрения начинающего reverse engineer-а, наиболее важные отличия от 32-битного x86 это:}
{From the reverse engineer's perspective, most important differences are:}

\index{\CLanguageElements!\Pointers}
\begin{itemize}

\item
\IFRU{Почти все регистры (кроме FPU и SIMD) расширены до 64-бит и получили префикс r-. 
И еще 8 регистров добавлено. 
В итоге имеются эти \ac{GPR}-ы:}
{Almost all registers (except FPU and SIMD) are extended to 64 bits and got r- prefix.
8 additional registers added.
Now \ac{GPR}'s are:} \RAX, \RBX, \RCX, \RDX, 
\RBP, \RSP, \RSI, \RDI, \Reg{8}, \Reg{9}, \Reg{10}, 
\Reg{11}, \Reg{12}, \Reg{13}, \Reg{14}, \Reg{15}. 

\IFRU{К ним также можно обращаться так же, как и прежде. Например, для доступа к младшим 32 битам \TT{RAX} 
можно использовать \EAX.}
{It is still possible to access to \IT{older} register parts as usual. 
For example, it is possible to access lower 32-bit part of the \TT{RAX} register using \EAX.}

\IFRU{У новых регистров \TT{r8-r15} также имеются их \IT{младшие части}: \TT{r8d-r15d} 
(младшие 32-битные части), 
\TT{r8w-r15w} (младшие 16-битные части), \TT{r8b-r15b} (младшие 8-битные части).}
{New \TT{r8-r15} registers also has its \IT{lower parts}: \TT{r8d-r15d} (lower 32-bit parts),
\TT{r8w-r15w} (lower 16-bit parts), \TT{r8b-r15b} (lower 8-bit parts).}

\IFRU{Удвоено количество SIMD-регистров: с 8 до 16:}
{SIMD-registers number are doubled: from 8 to 16:} \TT{XMM0-XMM15}.

\item
\IFRU{В win64 передача всех параметров немного иная, это немного похоже на fastcall~(\ref{fastcall}).
Первые 4 аргумента записываются в регистры \RCX, \RDX, \Reg{8}, \Reg{9}, а остальные ~--- в стек. 
Вызывающая функция также должна подготовить место из 32 байт чтобы вызываемая функция могла сохранить 
там первые 4 аргумента и использовать эти регистры по своему усмотрению. 
Короткие функции могут использовать аргументы прямо из регистров, но б\'{о}льшие функции могут сохранять 
их значения на будущее.}
{In Win64, function calling convention is slightly different, somewhat resembling fastcall~(\ref{fastcall}).
First 4 arguments stored in the \RCX, \RDX, \Reg{8}, \Reg{9} registers, others~---in the stack.
\Gls{caller} function must also allocate 32 bytes so the \gls{callee} may save there 4 first arguments and use these 
registers for own needs.
Short functions may use arguments just from registers, but larger may save their values on the stack.}

\RU{Соглашение }System V AMD64 ABI (Linux, *BSD, \MacOSX)\cite{SysVABI} \IFRU{также напоминает}{also somewhat resembling}
fastcall, \IFRU{использует 6 регистров}{it uses 6 registers} 
\RDI, \RSI, \RDX, \RCX, \Reg{8}, \Reg{9} \IFRU{для первых шести аргументов}{for the first 6 arguments}.
\IFRU{Остальные передаются через стек}{All the rest are passed in the stack}.

\IFRU{См. также в соответствующем разделе о способах передачи аргументов через стек}
{See also section about calling conventions}~(\ref{sec:callingconventions}).

\item
\IFRU{Сишный \Tint остается 32-битным для совместимости.}
{C \Tint type is still 32-bit for compatibility.}

\item
\IFRU{Все указатели теперь 64-битные}{All pointers are 64-bit now}.

% to be proofreaded (begin)
\IFRU{На это иногда сетуют: ведь теперь для хранения всех указателей нужно в 2 раза больше места 
в памяти, в т.ч. и в кэш-памяти, не смотря на то что x64-процессоры адресуют только 48 бит
внешней \ac{RAM}}
{This provokes irritation sometimes: now one need twice as much memory for storing pointers,
including, cache memory, despite the fact x64 \ac{CPU}s addresses only 48 bits of external 
\ac{RAM}}.
% to be proofreaded (end)

\end{itemize}

\index{Register allocation}
\IFRU{Из-за того, что регистров общего пользования теперь вдвое больше, у компиляторов теперь больше 
свободного места для маневра называемого \glslink{register allocator}{register allocation}.
Для нас это означает, что в итоговом коде будет меньше локальных переменных.}
{Since now registers number are doubled, compilers has more space now for maneuvering calling 
\glslink{register allocator}{register allocation}.
What it meanings for us, emitted code will contain less local variables.}

\index{DES}
\IFRU{Для примера, функция вычисляющая первый S-бокс алгоритма шифрования DES, 
она обрабатывает сразу 32/64/128/256 значений, в зависимости от типа \TT{DES\_type} (uint32, uint64, SSE2 или AVX), 
методом bitslice DES (больше об этом методе читайте здесь~(\ref{bitslicedes})):}
{For example, function calculating first S-box of DES encryption algorithm, it processing
32/64/128/256 values at once (depending on \TT{DES\_type} type (uint32, uint64, SSE2 or AVX)) 
using bitslice DES method
(read more about this technique here ~(\ref{bitslicedes})):}

\lstinputlisting{patterns/20_x64/19_1.c}

\IFRU{Здесь много локальных переменных. Конечно, далеко не все они будут в локальном стеке. 
Компилируем обычным MSVC 2008 с опцией \Ox:}
{There is a lot of local variables. Of course, not all those will be in local stack.
Let's compile it with MSVC 2008 with \Ox option:}

\lstinputlisting[caption=\Optimizing MSVC 2008]{patterns/20_x64/19_2_msvc_Ox.asm}

\IFRU{5 переменных компилятору пришлось разместить в локальном стеке.}
{5 variables was allocated in local stack by compiler.}

\IFRU{Теперь попробуем то же самое только в 64-битной версии MSVC 2008:}
{Now let's try the same thing in 64-bit version of MSVC 2008:}

\lstinputlisting[caption=\Optimizing MSVC 2008]{patterns/20_x64/19_3_msvc_x64.asm}

\IFRU{Компилятор ничего не выделил в локальном стеке, а \TT{x36} это синоним для \TT{a5}.}
{Nothing allocated in local stack by compiler, \TT{x36} is synonym for \TT{a5}.}

\IFRU{Кстати, видно, что функция сохраняет регистры \RCX, \RDX в отведенных для 
этого вызываемой функцией местах, 
а \Reg{8} и \Reg{9} не сохраняет, а начинает использовать их сразу.}
{By the way, we can see here, the function saved \RCX and \RDX registers in allocated by \gls{caller} space,
but \Reg{8} and \Reg{9} are not saved but used from the beginning.}

\IFRU{Кстати, существуют процессоры с еще большим количеством \ac{GPR}, например, 
Itanium ~--- 128 регистров.}
{By the way, there are CPUs with much more \ac{GPR}'s, e.g. Itanium (128 registers).}

\section{ARM}

\IFRU{64-битные инструкции в ARM появились в}{In ARM, 64-bit instructions are appeared in} ARMv8.

\section{\IFRU{Числа с плавающей запятой}{Float point numbers}}

\IFRU{О том как происходит работа с числами с плавающей запятой в x86-64, читайте здесь: 
\ref{floating_SIMD}.}
{Read more here\ref{floating_SIMD} about how float point numbers are processed in x86-64.}

% FIXME1 divide this file into separate ones...
\chapter{\RU{Работа с числами с плавающей запятой используя SIMD}\EN{Working with floating point numbers using SIMD}}

\label{floating_SIMD}
\index{IEEE 754}
\index{SIMD}
\index{SSE}
\index{SSE2}
\RU{Разумеется, FPU остался в x86-совместимых процессорах в то время, когда ввели расширения \ac{SIMD}}
\EN{Of course, the \ac{FPU} has remained in x86-compatible processors when the \ac{SIMD} extensions were added}.

\EN{The }\ac{SIMD}\RU{-расширения}\EN{ extensions} (SSE2) \RU{позволяют удобнее работать с числами с плавающей 
запятой}\EN{offer an easier way to work with floating-point numbers}.

\RU{Формат чисел остается тот же}\EN{The number format remains the same} (IEEE 754).

\index{x86-64}
\RU{Так что современные компиляторы (включая те, что компилируют под x86-64) 
обычно используют \ac{SIMD}-инструкции вместо FPU-инструкций.}\EN{So, modern compilers (including those generating
for x86-64) usually use \ac{SIMD} instructions instead of FPU ones.}

\RU{Это, можно сказать, хорошая новость, потому что работать с ними легче}
\EN{It can be said that it's good news, because it's easier to work with them}.

\RU{Примеры будем использовать из секции о FPU}
\EN{We are going to reuse the examples from the FPU section here}: \myref{sec:FPU}.

\section{\RU{Простой пример}\EN{Simple example}}

\lstinputlisting{patterns/12_FPU/1_simple/simple.c}

\subsection{x64}

\lstinputlisting[caption=\Optimizing MSVC 2012 x64]{patterns/205_floating_SIMD/simple_MSVC_2012_x64_Ox.asm}

\RU{Собственно, входные значения с плавающей запятой передаются через регистры \XMM{0}-\XMM{3}, 
а остальные --- через стек}\EN{The input floating point values are passed in the \XMM{0}-\XMM{3} registers,
all the rest---via the stack}
\footnote{\href{http://go.yurichev.com/17263}{MSDN: Parameter Passing}}.

$a$ \RU{передается через}\EN{is passed in} \XMM{0}, $b$\EMDASH{}\RU{через}\EN{via} \XMM{1}.
\RU{Но XMM-регистры (как мы уже знаем из секции о \ac{SIMD}: \myref{SIMD_x86}) 128-битные, 
а значения типа \Tdouble --- 64-битные,
так что используется только младшая половина регистра}
\EN{The XMM-registers are 128-bit (as we know from the section about \ac{SIMD}: \myref{SIMD_x86}), 
but the \Tdouble values are 64 bit, so only lower register half is used}.

\index{x86!\Instructions!DIVSD}
\TT{DIVSD} \RU{это SSE-инструкция, означает}\EN{is an SSE-instruction that stands for} 
\q{Divide Scalar Double-Precision Floating-Point Values}, 
\RU{и просто делит значение типа \Tdouble на другое, лежащие в младших половинах операндов}\EN{it just divides
one value of type \Tdouble by another, stored in the lower halves of operands}.

\RU{Константы закодированы компилятором в формате IEEE 754}\EN{The constants are encoded by compiler in IEEE 754 format}.

\index{x86!\Instructions!MULSD}
\index{x86!\Instructions!ADDSD}
\TT{MULSD} \AndENRU \TT{ADDSD} \RU{работают так же, только производят умножение и сложение}
\EN{work just as the same, but do multiplication and addition}.

\RU{Результат работы функции типа \Tdouble функция оставляет в регистре \XMM{0}}
\EN{The result of the function's execution in type \Tdouble is left in the in \XMM{0} register}.\\
\\
\RU{Как работает неоптимизирующий MSVC}\EN{That is how non-optimizing MSVC works}:

\lstinputlisting[caption=MSVC 2012 x64]{patterns/205_floating_SIMD/simple_MSVC_2012_x64.asm}

\index{Shadow space}
\RU{Чуть более избыточно}\EN{Slightly redundant}. 
\RU{Входные аргументы сохраняются в}\EN{The input arguments are saved in the} \q{shadow space} (\myref{shadow_space}), 
\RU{причем, только младшие половины регистров, т.е. только 64-битные значения типа \Tdouble}
\EN{but only their lower register halves, i.e., only 64-bit values of type \Tdouble}.
\ifdefined\IncludeGCC
\RU{Результат работы компилятора GCC точно такой же}\EN{GCC produces the same code}.
\fi

\subsection{x86}

\RU{Скомпилируем этот пример также и под x86. MSVC 2012 даже генерируя под x86, использует SSE2-инструкции:}
\EN{Let's also compile this example for x86. Despite the fact it's generating for x86, MSVC 2012 uses SSE2 instructions:}

\lstinputlisting[caption=\NonOptimizing MSVC 2012 x86]{patterns/205_floating_SIMD/simple_MSVC_2012_x86.asm}

\lstinputlisting[caption=\Optimizing MSVC 2012 x86]{patterns/205_floating_SIMD/simple_MSVC_2012_x86_Ox.asm}

\RU{Код почти такой же, правда есть пара отличий связанных с соглашениями о вызовах:}
\EN{It's almost the same code, however, there are some differences related to calling conventions:}
1) \RU{аргументы передаются не в XMM-регистрах, а через стек, как и прежде, в примерах с FPU (\myref{sec:FPU});}
\EN{the arguments are passed not in XMM registers, but in the stack, like in the FPU examples (\myref{sec:FPU});}
2) \RU{результат работы функции возвращается через \ST{0} --- для этого он через стек
(через локальную переменную \TT{tv}) копируется из XMM-регистра в \ST{0}.}
\EN{the result of the function is returned in \ST{0} --- in order to do so, it's copied
(through local variable \TT{tv}) from one of the XMM registers to \ST{0}.}

\ifdefined\IncludeOlly
\clearpage
\RU{Попробуем соптимизированный пример в}\EN{Let's try the optimized example in} \olly:

\begin{figure}[H]
\centering
\includegraphics[scale=\FigScale]{patterns/205_floating_SIMD/simple_olly1.png}
\caption{\olly: \TT{MOVSD} \RU{загрузила значение}\EN{loads the value of} $a$ \RU{в}\EN{into} \XMM{1}}
\label{fig:FPU_SIMD_simple_olly1}
\end{figure}

\clearpage
\begin{figure}[H]
\centering
\includegraphics[scale=\FigScale]{patterns/205_floating_SIMD/simple_olly2.png}
\caption{\olly: \TT{DIVSD} \RU{вычислила}\EN{calculated} \gls{quotient} 
\RU{и оставила его в}\EN{and stored it in} \XMM{1}}
\label{fig:FPU_SIMD_simple_olly2}
\end{figure}

\clearpage
\begin{figure}[H]
\centering
\includegraphics[scale=\FigScale]{patterns/205_floating_SIMD/simple_olly3.png}
\caption{\olly: \TT{MULSD} \RU{вычислила}\EN{calculated} \gls{product} \RU{и оставила его в}\EN{and stored it
in} \XMM{0}}
\label{fig:FPU_SIMD_simple_olly3}
\end{figure}

\clearpage
\begin{figure}[H]
\centering
\includegraphics[scale=\FigScale]{patterns/205_floating_SIMD/simple_olly4.png}
\caption{\olly: \TT{ADDSD} \RU{прибавила значение в}\EN{adds value in} \XMM{0} \RU{к}\EN{to} \XMM{1}}
\label{fig:FPU_SIMD_simple_olly4}
\end{figure}

\clearpage
\begin{figure}[H]
\centering
\includegraphics[scale=\FigScale]{patterns/205_floating_SIMD/simple_olly5.png}
\caption{\olly: \FLD \RU{оставляет результат функции в}\EN{left function result in} \ST{0}}
\label{fig:FPU_SIMD_simple_olly5}
\end{figure}

\RU{Видно, что \olly показывает XMM-регистры как пары чисел в формате \Tdouble,
но используется только \IT{младшая} часть.}
\EN{We see that \olly shows the XMM registers as pairs of \Tdouble numbers,
but only the \IT{lower} part is used.}
\RU{Должно быть, \olly показывает их именно так, потому что сейчас исполняются SSE2-инструкции
с суффиксом \TT{-SD}.}
\EN{Apparently, \olly shows them in that format because the SSE2 instructions (suffixed with \TT{-SD}) 
are executed right now.}
\RU{Но конечно же, можно переключить отображение значений в регистрах и посмотреть содержимое
как 4 \Tfloat{}-числа или просто как 16 байт.}
\EN{But of course, it's possible to switch the register format and to see their contents as
4 \Tfloat{}-numbers or just as 16 bytes.}
\fi

\clearpage
\section{\RU{Передача чисел с плавающей запятой в аргументах}\EN{Passing floating point number via arguments}}

\lstinputlisting{patterns/12_FPU/2_passing_floats/pow.c}

\RU{Они передаются в младших половинах регистров}\EN{They are passed in the lower halves
of the} \XMM{0}-\XMM{3}\EN{ registers}.

\lstinputlisting[caption=\Optimizing MSVC 2012 x64]{patterns/205_floating_SIMD/pow_MSVC_2012_x64_Ox.asm}

\index{x86!\Instructions!MOVSD}
\index{x86!\Instructions!MOVSDX}
\RU{Инструкции}\EN{There is no} \TT{MOVSDX} \RU{нет в документации от}\EN{instruction in} 
Intel \cite{Intel} \AndENRU AMD \cite{AMD}\EN{ manuals}, 
\RU{там она называется просто}\EN{there it is called just} \TT{MOVSD}.
\RU{Таким образом, в процессорах x86 две инструкции с одинаковым именем}\EN{So there are two instructions
sharing the same name in x86} (\RU{о второй}\EN{about the other see}: \myref{REP_MOVSx}).
\RU{Возможно, в Microsoft решили избежать
путаницы и переименовали инструкцию в}\EN{Apparently, Microsoft developers wanted to get rid of the mess,
so they renamed it to} \TT{MOVSDX}.
\RU{Она просто загружает значение в младшую половину XMM-регистра}\EN{It just loads a value into
the lower half of a XMM register}.

\RU{Функция }\TT{pow()} \RU{берет аргументы из}\EN{takes arguments from} \XMM{0} \AndENRU \XMM{1}, 
\RU{и возвращает результат в}\EN{and returns result in} \XMM{0}.
\RU{Далее он перекладывается в}\EN{It is then moved to} \RDX \ForENRU \printf. 
\RU{Почему}\EN{Why}? 
\RU{Может быть, это потому что}\EN{Maybe because} 
\printf\EMDASH{}\RU{функция с переменным количеством аргументов}\EN{is a variable arguments function}?

\lstinputlisting[caption=\Optimizing GCC 4.4.6 x64]{patterns/205_floating_SIMD/pow_GCC446_x64_O3.s.\LANG}

GCC \RU{работает понятнее}\EN{generates clearer output}. 
\RU{Значение для}\EN{The value for} \printf \RU{передается в}\EN{is passed in} \XMM{0}. 
\RU{Кстати, вот тот случай, когда в}\EN{By the way, here is a case when 1 is written into} \EAX
\ForENRU \printf \RU{записывается 1 --- это значит, что будет передан один аргумент в векторных регистрах, 
так того требует стандарт}\EN{---this implies that one argument will be passed in vector registers,
just as the standard requires} \cite{SysVABI}.

\section{\RU{Пример с сравнением}\EN{Comparison example}}

\lstinputlisting{patterns/12_FPU/3_comparison/d_max.c}

\subsection{x64}

\lstinputlisting[caption=\Optimizing MSVC 2012 x64]{patterns/205_floating_SIMD/d_max_MSVC_2012_x64_Ox.asm}

\Optimizing MSVC \RU{генерирует очень понятный код}\EN{generates a code very easy to understand}.

\index{x86!\Instructions!COMISD}
\RU{Инструкция }\TT{COMISD} \RU{это}\EN{is} \q{Compare Scalar Ordered Double-Precision Floating-Point 
Values and Set EFLAGS}. \RU{Собственно, это она и делает}\EN{Essentially, that is what it does}.\\
\\
\NonOptimizing MSVC \RU{генерирует более избыточно, но тоже всё понятно}\EN{generates more redundant code,
but it is still not hard to understand}:

\lstinputlisting[caption=MSVC 2012 x64]{patterns/205_floating_SIMD/d_max_MSVC_2012_x64.asm}

\index{x86!\Instructions!MAXSD}
\RU{А вот}\EN{However,} GCC 4.4.6 \RU{дошел в оптимизации дальше и применил инструкцию}
\EN{did more optimizations and used the} \TT{MAXSD} (\q{Return Maximum Scalar 
Double-Precision Floating-Point Value})\RU{, которая просто выбирает максимальное значение}\EN{ instruction,
which just choose the maximum value}!

\lstinputlisting[caption=\Optimizing GCC 4.4.6 x64]{patterns/205_floating_SIMD/d_max_GCC446_x64_O3.s}

\clearpage
\subsection{x86}

\RU{Скомпилируем этот пример в MSVC 2012 с включенной оптимизацией:}
\EN{Let's compile this example in MSVC 2012 with optimization turned on:}

\lstinputlisting[caption=\Optimizing MSVC 2012 x86]{patterns/205_floating_SIMD/d_max_MSVC_2012_x86_Ox.asm}

\RU{Всё то же самое, только значения}\EN{Almost the same, but the values of} $a$ \AndENRU $b$ 
\RU{берутся из стека, а результат функции оставляется в}\EN{are taken from the stack and the function result 
is left in} \ST{0}.

\ifdefined\IncludeOlly
\RU{Если загрузить этот пример в}\EN{If we load this example in} \olly, 
\RU{увидим, как инструкция}\EN{we can see how the} \TT{COMISD} \RU{сравнивает значения и устанавливает/сбрасывает
флаги}\EN{instruction compares values and sets/clears the} \CF \AndENRU \PF\EN{ flags}:

\begin{figure}[H]
\centering
\includegraphics[scale=\FigScale]{patterns/205_floating_SIMD/d_max_olly.png}
\caption{\olly: \TT{COMISD} \RU{изменила флаги}\EN{changed} \CF \AndENRU \PF\EN{ flags}}
\label{fig:FPU_SIMD_d_max_olly}
\end{figure}
\fi

\section{\RU{Вычисление машинного эпсилона}\EN{Calculating machine epsilon}: x64 \AndENRU SIMD}
\label{machine_epsilon_x64_and_SIMD}

\RU{Вернемся к примеру \q{вычисление машинного эпсилона} для \Tdouble \lstref{machine_epsilon_double_c}.}
\EN{Let's revisit the \q{calculating machine epsilon} example for \Tdouble \lstref{machine_epsilon_double_c}.}

\RU{Теперь скомпилируем его для x64}\EN{Now we compile it for x64}:

\lstinputlisting[caption=\Optimizing MSVC 2012 x64]{patterns/205_floating_SIMD/epsilon_double_MSVC_2012_x64_Ox.asm}

\RU{Нет способа прибавить 1 к значению в 128-битном XMM-регистре, так что его нужно в начале поместить в память.}
\EN{There is no way to add 1 to a value in 128-bit XMM register, so it must be placed into memory.}

\RU{Впрочем, есть инструкция ADDSD (\IT{Add Scalar Double-Precision Floating-Point Values}),
которая может прибавить значение к младшей 64-битной части XMM-регистра игнорируя старшую половину,
но наверное MSVC 2012 пока недостаточно хорош для этого}
\EN{There is, however, the ADDSD instruction (\IT{Add Scalar Double-Precision Floating-Point Values}) 
which can add a value to the lowest 64-bit half of a XMM register while ignoring the higher one, 
but MSVC 2012 probably is not that good yet}
\footnote{\RU{В качестве упражнения, вы можете попробовать переработать этот код, чтобы избавиться 
от использования локального стека}\EN{As an exercise, you may try to rework this code to 
eliminate the usage of the local stack}.}.

\RU{Так или иначе, значение затем перезагружается в XMM-регистр и происходит вычитание.}
\EN{Nevertheless, the value is then reloaded to a XMM register and subtraction occurs.}
SUBSD \RU{это}\EN{is} \q{Subtract Scalar Double-Precision Floating-Point Values}, 
\RU{т.е. операция производится над младшей 64-битной частью 128-битного XMM-регистра}
\EN{i.e., it operates on the lower 64-bit part of 128-bit XMM register}.
\RU{Результат возвращается в регистре XMM0}\EN{The result is returned in the XMM0 register}.

\section{\RU{И снова пример генератора случайных чисел}\EN{Pseudo-random number generator example revisited}}
\label{FPU_PRNG_SIMD}

\RU{Вернемся к примеру ``пример генератора случайных чисел'' \lstref{FPU_PRNG}.}
\EN{Let's revisit ``pseudo-random number generator example'' example \lstref{FPU_PRNG}.}

\RU{Если скомпилировать это в MSVC 2012, компилятор будет использовать SIMD-инструкции для FPU.}
\EN{If we compile this in MSVC 2012, it will use the SIMD instructions for the FPU.}

\lstinputlisting[caption=\Optimizing MSVC 2012]{patterns/205_floating_SIMD/FPU_PRNG/MSVC2012_Ox_Ob0.asm.\LANG}

\RU{У всех инструкций суффикс -SS, это означает}\EN{All instructions have the -SS suffix, this means } ``Scalar Single''.
``Scalar'' \RU{означает что только одно значение хранится в регистре}\EN{means that only one value is stored in the register}.
``Single'' \RU{означает что это тип \Tfloat}\EN{means \Tfloat data type}.


\section{\RU{Итог}\EN{Summary}}

\RU{Во всех приведенных примерах, в XMM-регистрах используется только младшая половина регистра, там
хранится значение в формате IEEE 754}\EN{Only the lower half of XMM registers is used in all examples here, 
to store number in IEEE 754 format}.

\RU{Собственно, все инструкции с суффиксом}\EN{Essentially, all instructions prefixed by} 
\TT{-SD} (\q{Scalar Double-Precision})\EMDASH{}\RU{это инструкции для работы с числами с плавающей 
запятой в формате IEEE 754, 
хранящиеся в младшей 64-битной половине XMM-регистра}\EN{are instructions working with floating point numbers
in IEEE 754 format, stored in the lower 64-bit half of a XMM register}.

\RU{Всё удобнее чем это было в FPU, видимо, сказывается тот факт, что расширения 
SIMD развивались не так хаотично как FPU в прошлом.}
\EN{And it is easier than in the FPU, probably because the SIMD extensions 
were evolved in a less chaotic way than the FPU ones in the past.}
\RU{Стековая модель регистров не используется}\EN{The stack register model is not used}.

\index{x86!\Instructions!ADDSS}
\index{x86!\Instructions!MOVSS}
\index{x86!\Instructions!COMISS}
% TODO1: do this!
\RU{Если вы попробуете заменить в этих примерах}\EN{If you would try to replace} \Tdouble \RU{на}\EN{with} \Tfloat
\RU{, то инструкции будут использоваться те же,
только с суффиксом}
% FIXME1 ... but their -SS versions
\EN{in these examples, the same instructions will be used, but prefixed with} \TT{-SS} 
(\q{Scalar Single-Precision}), \RU{например}\EN{for example}, \TT{MOVSS}, \TT{COMISS}, \TT{ADDSS}, \etc{}.

\q{Scalar} \RU{означает что SIMD-регистр будет хранить только одно значение, вместо нескольких.}
\EN{implies that the SIMD register containing only one value instead of several.}
\RU{Инструкции, работающие с несколькими значениями в регистре одновременно, имеют \q{Packed} в названии}
\EN{Instructions working with several values in a register simultaneously have \q{Packed} in their name}.

\RU{Нужно также обратить внимание, что SSE2-инструкции работают с 64-битными числами (\Tdouble) в формате IEEE 754,
в то время как внутреннее представление в FPU --- 80-битные числа.}
\EN{Needless to say, the SSE2 instructions work with 64-bit IEEE 754 numbers (\Tdouble),
while the internal representation of the floating-point numbers in FPU is 80-bit numbers.}
\RU{Поэтому ошибок округления (\IT{round-off error}) в FPU может быть меньше чем в SSE2,
как следствие, можно сказать, работа с FPU может давать более точные результаты вычислений.}
\EN{Hence, the FPU may produce less round-off errors and as a consequence, FPU may give more precise
calculation results.}

\chapter{C99 restrict}
\index{\CLanguageElements!C99!restrict}
\index{FORTRAN}

\RU{А вот причина, из-за которой программы на FORTRAN, в некоторых случаях, работают быстрее чем на Си.}
\EN{Here is a reason why FORTRAN programs, in some cases, works faster than \CCpp ones.}

\begin{lstlisting}
void f1 (int* x, int* y, int* sum, int* product, int* sum_product, int* update_me, size_t s)
{
	for (int i=0; i<s; i++)
	{
		sum[i]=x[i]+y[i];
		product[i]=x[i]*y[i];
		update_me[i]=i*123; // some dummy value
		sum_product[i]=sum[i]+product[i];	
	};
};
\end{lstlisting}

\RU{Это очень простой пример, в котором есть одна особенность}
\EN{That's very simple example with one specific
thing in it}: 
\RU{указатель на массив}\EN{pointer to} \TT{update\_me} \RU{может быть указателем на массив}\EN{array could be
a pointer to}
\TT{sum}\EN{ array}, \TT{product}\EN{ array}, \RU{или даже}\EN{or even} 
\TT{sum\_product}\EN{ array}\EMDASH\RU{ведь нет ничего криминального в том 
чтобы аргументам функции быть такими, верно?}\EN{since it is not a crime in it, right?}

\RU{Компилятор знает об этом, поэтому генерирует код, где в теле цикла будет 4 основных стадии:}
\EN{Compiler is fully aware about it, so it generates a code with four stages in loop body:}
\begin{itemize}
\item \RU{вычислить следующий}\EN{calculate next} \TT{sum[i]}
\item \RU{вычислить следующий}\EN{calculate next} \TT{product[i]}
\item \RU{вычислить следующий}\EN{calculate next} \TT{update\_me[i]}
\item \RU{вычислить следующий}\EN{calculate next} \TT{sum\_product[i]}\EMDASH\RU{на этой стадии придется снова загружать из памяти подсчитанные}
\EN{on this stage, we need to load from memory already calculated} \TT{sum[i]} \AndENRU \TT{product[i]}
\end{itemize}

\RU{Возможно ли соптимизировать последнюю стадию?}\EN{Is it possible to optimize the last stage?}
\RU{Ведь подсчитанные}\EN{Since already calculated} \TT{sum[i]} \AndENRU \TT{product[i]} 
\RU{не обязательно снова загружать из памяти, ведь мы их только что подсчитали.}
\EN{are not necessary to load from memory again, because we already calculated them.}
\RU{Можно, но компилятор не уверен, что на третьей стадии ничего не затерлось!}
\EN{Yes, but compiler is not sure that nothing was overwritten on 3rd stage!}
\RU{Это называется}\EN{This is called}
``pointer aliasing'', \RU{ситуация, когда компилятор не может быть уверен что память на которую указывает 
какой-то указатель, не изменилась.}
\EN{a situation, when compiler cannot be sure that a memory to which a pointer is pointing, was not changed.}

\IT{restrict} \RU{в стандарте Си C99}\EN{in C99 standard}\cite[6.7.3/1]{C99TC3} 
\RU{это обещание, даваемое компилятору программистом, что аргументы функции, отмеченные этим ключевым словом,
всегда будут указывать на разные места в памяти и пересекаться не будут.}
\EN{is a promise, given by programmer to compiler the function arguments marked by this keyword will always
be pointing to different memory locations and never be crossed.}

\RU{Если быть более точным, и описывать это формально, \IT{restrict} показывает, что только данный указатель будет
использоваться для доступа к этому объекту, с которым мы работаем через этот указатель, больше никакой указатель для
этого использоваться не будет.}
\EN{If to be more precise and describe this formally, \IT{restrict} shows that only this pointer is to be used
to access an object, with which we are working via this pointer, and no other pointer will be used for it.}
\RU{Можно даже сказать, что к всякому объекту, доступ будет осуществляться только через
один единственный указатель, если он отмечен как}
\EN{It can be even said the object will be accessed
only via one single pointer, if it is marked as} \IT{restrict}.

\RU{Добавим это ключевое слово к каждому аргументу-указателю}\EN{Let's add this keyword to each argument-pointer}:

\begin{lstlisting}
void f2 (int* restrict x, int* restrict y, int* restrict sum, int* restrict product, int* restrict sum_product, 
	int* restrict update_me, size_t s)
{
	for (int i=0; i<s; i++)
	{
		sum[i]=x[i]+y[i];
		product[i]=x[i]*y[i];
		update_me[i]=i*123; // some dummy value
		sum_product[i]=sum[i]+product[i];	
	};
};
\end{lstlisting}

\RU{Посмотрим результаты}\EN{Let's see results}:

\lstinputlisting[caption=GCC x64: f1()]{patterns/21_C99_restrict/f1_\LANG.asm}

\lstinputlisting[caption=GCC x64: f2()]{patterns/21_C99_restrict/f2_\LANG.asm}

\RU{Разница между скомпилированной функцией \TT{f1()} и \TT{f2()} такая}
\EN{The difference between compiled \TT{f1()} and \TT{f2()} function is as follows}:
\InENRU \TT{f1()}, \TT{sum[i]} \AndENRU \TT{product[i]} \RU{загружаются снова посреди тела цикла}
\EN{are reloaded in the middle of loop},
\RU{а в}\EN{and in} \TT{f2()} \RU{этого нет, используются уже подсчитанные значения}
\EN{there are no such thing,
already calculated values are used}, 
\RU{ведь мы ``пообещали'' компилятору}\EN{since we ``promised'' to compiler}, 
\RU{что никто и ничто не изменит значения в}
\EN{that no one and nothing will change values in} \TT{sum[i]} 
\AndENRU \TT{product[i]} \RU{во время исполнения тела цикла}\EN{while execution of loop body}, 
\RU{поэтому он ``уверен'', что значения из памяти можно не загружать снова}
\EN{so it is ``sure'' the value from memory may not be loaded again}.
\RU{Очевидно, второй вариант будет работать быстрее.}\EN{Obviously, second example will work faster.}

\RU{Но что будет если указатели в аргументах функций все же будут пересекаться?}
\EN{But what if pointers in function arguments will be crossed somehow?}
\RU{Это останется на совести программиста, а результаты вычислений будут неверными.}
\EN{This will be on programmer's conscience, but results will be incorrect.}

\RU{Вернемся к}\EN{Let's back to} FORTRAN. 
\RU{Компиляторы с этого ЯП, по умолчанию, все указатели считают таковыми}
\EN{Compilers from this programming language treats all pointers as such}, 
\RU{поэтому, когда в Си не было возможности указать}
\EN{so when it was not possible to set} \IT{restrict}, 
FORTRAN \RU{в этих случаях мог генерировать более быстрый код}\EN{in these cases may generate faster code}.

\RU{Насколько это практично}\EN{How practical is it}? 
\RU{Там, где функция работает с несколькими большими блоками в памяти.}
\EN{In the cases when function works with several big blocks in memory.}
\RU{Такого очень много в линейной алгебре, например.}
\EN{E.g. there are a lot of such in linear algebra.}
\RU{Очень много линейной алгебры используется на суперкомпьютерах/\ac{HPC},
возможно, поэтому, традиционно, там часто используется FORTRAN, до сих пор}
\EN{A lot of linear algebra used on supercomputers/\ac{HPC}, probably, that is why, traditionally, FORTRAN is still
used there}\cite{Loh:2010:IHP:1810226.1820518}.

\RU{Ну а когда итераций цикла не очень много, конечно, тогда прирост скорости не будет ощутимым.}
\EN{But when a number of iterations is not very big,
certainly, speed boost will not be significant.}


\chapter{\RU{Inline-функции}\EN{Inline functions}}
\index{Inline code}
\label{inline_code}

\RU{Inline-код это когда компилятор, вместо того чтобы генерировать инструкцию вызова небольшой функции,
просто вставляет её тело прямо в это место.}
\EN{Inlined code is when compiler, instead of placing call instruction to small or tiny function,
just placing its body right in-place.}

\lstinputlisting[caption=\RU{Простой пример}\EN{Simple example}]{patterns/22_inline_function/1.c}

\RU{... это компилируется вполне предсказуемо, хотя, если включить оптимизации GCC (\Othree), мы увидим:}
\EN{... is compiled in very predictable way, however, if to turn on GCC optimization (\Othree), we'll see:}

\lstinputlisting[caption=GCC 4.8.1 \Othree]{patterns/22_inline_function/1.s}

(\RU{Здесь деление заменено умножением}\EN{Here division is done by multiplication}(\ref{sec:divisionbynine}).)

\RU{Да, наша маленькая ф-ция \TT{celsius\_to\_fahrenheit()} была помещена прямо перед вызовом \printf.}
\EN{Yes, our small function \TT{celsius\_to\_fahrenheit()} was just placed before \printf call.}
\RU{Почему? Это может быть быстрее чем исполнять код самой ф-ции плюс затраты на вызов и возврат.}
\EN{Why? It may be faster than executing this function's code plus calling/returning overhead.}

\RU{В прошлом, такие ф-ции нужно было маркировать ключевым словом ``inline'' в определении ф-ции, хотя,
в наше время, такие ф-ции выбираются компилятором автоматически.}
\EN{In past, such function must be marked with ``inline'' keyword in function's declaration, however,
in modern times, these functions are chosen automatically by compiler.}

% sections
\section{\RU{Ф-ции работы со строками и памятью}\EN{Strings and memory functions}}

\RU{Другая очень частая оптимизация это вставка кода строковых ф-ций таких как}
\EN{Another very common automatic optimization tactic is inlining of string functions like}
\IT{strcpy()}, \IT{strcmp()}, \IT{strlen()}, \IT{memcmp()}, \IT{memcpy()}, \RU{и т.д}\EN{etc}.

\RU{Иногда это быстрее, чем вызывать отдельную ф-цию.}\EN{Sometimes it's faster then to call separate function.}

\RU{Это очень часто встречающшиеся шаблонные вставки, которые желательно распозновать ``на глаз''.}
\EN{These are very frequent patterns, which are highly advisable to learn to detect automatically.}

% subsections
\subsection{strcmp()}
\index{\CStandardLibrary!strcmp()}

\lstinputlisting[caption=\RU{пример с strcmp()}\EN{strcmp() example}]{patterns/22_inline_function/str_mem/strcmp.c}

\lstinputlisting[caption=\Optimizing GCC 4.8.1]{patterns/22_inline_function/str_mem/strcmp_GCC_O3.s}

\lstinputlisting[caption=\Optimizing MSVC 2010]{patterns/22_inline_function/str_mem/strcmp_MSVC_2010_Ox.asm}

\subsection{strlen()}
\index{\CStandardLibrary!strlen()}

\lstinputlisting[caption=\RU{пример с strlen()}\EN{strlen() example}]{patterns/22_inline_function/str_mem/strlen.c}

\lstinputlisting[caption=\Optimizing MSVC 2010]{patterns/22_inline_function/str_mem/strlen_MSVC_2010_Ox.asm}

\subsection{strcpy()}
\index{\CStandardLibrary!strcpy()}

\lstinputlisting[caption=\RU{пример с strcpy()}\EN{strcpy() example}]{patterns/22_inline_function/str_mem/strcpy.c}

\lstinputlisting[caption=\Optimizing MSVC 2010]{patterns/22_inline_function/str_mem/strcpy_MSVC_2010_Ox.asm}

\subsection{memcpy()}

\subsubsection{\RU{Короткие блоки}\EN{Short blocks}}
\label{copying_short_blocks}

\RU{Если нужно скопировать немного байт, то, нередко, 
\TT{memcpy()} заменяется на несколько инструкций \MOV.}
\EN{Short block copy routine is often implemented as pack of \MOV instructions.}

\lstinputlisting[caption=\RU{пример с memcpy()}\EN{memcpy() example}]{patterns/22_inline_function/str_mem/memcpy_7.c}

\lstinputlisting[caption=MSVC 2010 /Ox]{patterns/22_inline_function/str_mem/memcpy_7_MSVC_2010_Ox.asm}

\lstinputlisting[caption=GCC 4.8.1 \Othree]{patterns/22_inline_function/str_mem/memcpy_7_GCC_O3.s}

\RU{Обынчо это происходит так: в начале копируются 4-байтные блоки, затем 16-битное слово (если нужно), 
затем последний байт (если нужно).}
\EN{That's usually done as follows: 4-byte blocks are copied first, then 16-bit word (if needed), 
then the last byte (if needed).}

\RU{Точно так же при помощи \MOV копируются структуры}\EN{Structures are also copied using
\MOV}: \ref{short_struct_copying_using_MOV}.

\subsubsection{\RU{Длинные блоки}\EN{Long blocks}}

\RU{Здесь компиляторы ведут себя по-разному.}\EN{Compilers behave differently here.}

\lstinputlisting[caption=\RU{пример с memcpy()}\EN{memcpy() example}]{patterns/22_inline_function/str_mem/memcpy.c}

\RU{При копировании 128 байт, MSVC может обойтись одной инструкцией \TT{MOVSD} (ведь 128 кратно 4):}
\EN{While copying 128 bytes, MSVC can do this with single \TT{MOVSD} instruction (because 128 
divides evenly by 4):}

\lstinputlisting[caption=MSVC 2010 /Ox]{patterns/22_inline_function/str_mem/memcpy_128_MSVC_2010_Ox.asm}

\RU{При копировании 123-х байт, в начале копируется 30 32-битных слов при помощи \TT{MOVSD} 
(это 120 байт), 
затем копируется 2 байта при помощи \TT{MOVSW}, 
затем еще один байт при помощи \TT{MOVSB}.}
\EN{When 123 bytes are copying, 30 32-byte words are copied first using instruction \TT{MOVSD}
(that's 120 bytes),
then 2 bytes are copied using \TT{MOVSW}, 
then one more byte using \TT{MOVSB}.}

\lstinputlisting[caption=MSVC 2010 /Ox]{patterns/22_inline_function/str_mem/memcpy_123_MSVC_2010_Ox.asm}

\RU{GCC во всех случаях вставляет большую универсальную ф-цию, работающую для всех размеров блоков:}
\EN{GCC uses one big universal functions, working for any block size:}

\lstinputlisting[caption=GCC 4.8.1 \Othree]{patterns/22_inline_function/str_mem/memcpy_GCC.s}

\RU{Универсальные ф-ции копирования блоков обычно работают по следующей схеме: 
вычислить, сколько 32-битных слов
можно скопировать, затем сделать это при помощи \TT{MOVSD}, затем скопировать остатки.}
\EN{Universal memory copy functions are usually works as follows:
calculate, how many 32-bit words can be copied, then copy then by \TT{MOVSD}, then copy
remaining bytes.}

\RU{Более сложные ф-ции копирования используют \ac{SIMD} и учитывают выравнивание.}
\EN{More complex copy functions uses \ac{SIMD} instructions and take aligning into consideration.}

\subsection{memcmp()}
\index{\CStandardLibrary!memcmp()}

\lstinputlisting[caption=\RU{пример с memcmp()}\EN{memcmp() example}]{\CURPATH/str_mem/memcmp.c}

\RU{Для блоков разной длины, MSVC 2010 вставляет одну и ту же универсальную функцию:}
\EN{For any block size, MSVC 2010 inserts the same universal function:}

\lstinputlisting[caption=\Optimizing MSVC 2010]{\CURPATH/str_mem/memcmp_MSVC_2010_Ox.asm}


\subsection{\RU{Скрипт для IDA}\EN{IDA script}}

\RU{Я написал небольшой скрипт для \IDA для поиска и сворачивания таких очень часто 
попадающихся inline-функций:}
\EN{I wrote small \IDA script for searching and folding such very frequently seen pieces of 
inline code:} \\
\url{\YurichevIDAIDCScripts}.



\section{\IFRU{Неверно дизассемблированный код}{Incorrectly disassembled code}}

\IFRU{Практикующие reverse engineer-ы часто сталкиваются с неверно дизассемблированным кодом}
{Practicing reverse engineers often dealing with incorrectly disassembled code}.

\subsection{\IFRU{Дизассемблирование началось в неверном месте}{Disassembling started incorrectly} (x86)}

\IFRU{В отличие от ARM и MIPS (где у каждой инструкции длина или 2 или 4 байта), x86-инструкции имеют переменную длину,
так что, любой дизассемблер, начиная работу с середины x86-инструкции, может выдать неверные результаты.}
{Unlike ARM and MIPS (where any instruction has length of 2 or 4 bytes), x86 instructions has variable size,
so, any disassembler, starting at the middle of x86 instruction, may produce incorrect results.}

\IFRU{Как пример}{As an example}:

\lstinputlisting{patterns/23_incorrect_disassembly/x86_wrong_start.asm}

\IFRU{В начале мы видим неверно дизассемблированные инструкции, но потом, так или иначе, дизассемблер находит верный след}
{There are incorrectly disassembled instructions at the beginning, but eventually, disassembler finds right 
track}.

\subsection{\IFRU{Как выглядят случайные данные в дизассемблированном виде}{How random noise looks disassembled}?}

\IFRU{Общее, что можно сразу заметить, это}{Common properties which can be easily spotted are}:

\begin{itemize}
\item \IFRU{Необычно большой разброс инструкций}{Unusually big instruction dispersion}.
\IFRU{Самые частые x86-инструкции это}{Most frequent x86 instructions are} \PUSH{}, \MOV{}, \CALL{}, \IFRU{но здесь мы видим
инструкции из любых групп: \ac{FPU}-инструкции, инструкции \TT{IN}/\TT{OUT}, редкие и системные инструкции, всё друг с другом смешано 
в одном месте}{but here we will see
instructions from any instruction group: \ac{FPU} instructions, \TT{IN}/\TT{OUT} instructions, rare and system instructions,
everything messed up in one single place}.

\item \IFRU{Большие и случайные значения, смещения,}{Big and random values, offsets and} immediates.

\item \IFRU{Переходы с неверными смещениями часто имеют адрес перехода в середину другой инструкции}
{Jumps having incorrect offsets often jumping into the middle of another instructions}.
\end{itemize}

\lstinputlisting[caption=\randomNoise{} (x86)]{patterns/23_incorrect_disassembly/x86.asm}

\lstinputlisting[caption=\randomNoise{} (x86-64)]{patterns/23_incorrect_disassembly/x64.asm}

\index{ARM}
\lstinputlisting[caption=\randomNoise{} (ARM \IFRU{в режиме ARM}{in ARM mode})]{patterns/23_incorrect_disassembly/ARM.asm}

\lstinputlisting[caption=\randomNoise{} (ARM \IFRU{в режиме Thumb}{in Thumb mode})]{patterns/23_incorrect_disassembly/ARM_thumb.asm}

\index{MIPS}
\lstinputlisting[caption=\randomNoise (MIPS little endian)]{patterns/23_incorrect_disassembly/MIPS.asm}

\IFRU{Также важно помнить, что хитрым образом написанный код для распаковки и дешифровки (включая самомодифицирующийся),
также может выглядеть как случайный шум, тем не менее, он исполняется корректно}{It is also important to keep in mind that 
cleverly constructed unpacking and decrypting code 
(including self-modifying) may looks like noise as well, nevertheless, it executes correctly}.

\subsection{\IFRU{Информационная энтропия среднестатистического кода}{Information entropy of average code}}

\index{\IFRU{Информационная энтропия}{Information entropy}}
\IFRU{Результаты работы утилиты \IT{ent}}{\IT{ent} utility results}\footnote{\url{http://www.fourmilab.ch/random/}}.

(\IFRU{Энтропия идеально сжатого (или зашифрованного) файла\EMDASH{}8 бит на байт; файла с нулями любой длины\EMDASH{}0 бит на байт.}
{Entropy of ideally compressed (or encrypted) file is 8 bits per byte; of zero file of arbitrary size if 0 bits per byte.})

\IFRU{Здесь видно что код для CPU с 4-байтными инструкциями (ARM в режиме ARM и MIPS) наименее экономичны в этом смысле.}
{Here we can see that a code for CPU with 4-byte instructions (ARM in ARM mode and MIPS) is least effective in this sense.}

\subsubsection{x86}

\IFRU{Секция \TT{.text} файла \TT{ntoskrnl.exe} из}
{\TT{.text} section of \TT{ntoskrnl.exe} file from} Windows 2003:

\begin{lstlisting}
Entropy = 6.662739 bits per byte.

Optimum compression would reduce the size
of this 593920 byte file by 16 percent.
...
\end{lstlisting}

\IFRU{Секция \TT{.text} файла}{\TT{.text} section of} \TT{ntoskrnl.exe} \IFRU{из}{from} Windows 7 x64:

\begin{lstlisting}
Entropy = 6.549586 bits per byte.

Optimum compression would reduce the size
of this 1685504 byte file by 18 percent.
...
\end{lstlisting}

\subsubsection{ARM (Thumb)}
\index{ARM}

AngryBirds Classic:

\begin{lstlisting}
Entropy = 7.058766 bits per byte.

Optimum compression would reduce the size
of this 3336888 byte file by 11 percent.
...
\end{lstlisting}

\subsubsection{ARM (\IFRU{режим ARM}{ARM mode})}

Linux Kernel 3.8.0:

\begin{lstlisting}
Entropy = 6.036160 bits per byte.

Optimum compression would reduce the size
of this 6946037 byte file by 24 percent.
...
\end{lstlisting}

\subsubsection{MIPS (little endian)}
\index{MIPS}

\IFRU{Секция \TT{.text} файла}{\TT{.text} section of} \TT{user32.dll} \IFRU{из}{from} Windows NT 4:

\begin{lstlisting}
Entropy = 6.098227 bits per byte.

Optimum compression would reduce the size
of this 433152 byte file by 23 percent.
....
\end{lstlisting}


\chapter{\IFRU{Обфускация}{Obfuscation}}

\IFRU{Обфускация это попытка спрятать код (или его значение) от reverse engineer-а}
{Obfuscation is an attempt to hide the code (or its meaning) from reverse engineer}.

\section{\IFRU{Текстовые строки}{Text strings}}

\IFRU{Как я указывал в}{As I revealed in} (\ref{sec:digging_strings}) \IFRU{текстовые строки могут быть крайне
полезны}{text strings may be utterly helpful}.
\IFRU{Знающие об этом программисты могут попытаться их спрятать так, чтобы их не было видно в \IDA{} или любом
шестнадцатеричном редакторе}
{Programmers who aware of this, may try to hide them resulting unableness to find the string in \IDA{} or any hex editor}.

\IFRU{Вот простейший метод}{Here is the simpliest method}.

\IFRU{Вот как строка может быть сконструирована}{That is how the string may be constructed}:

\begin{lstlisting}
mov     byte ptr [ebx], 'h'
mov     byte ptr [ebx+1], 'e'
mov     byte ptr [ebx+2], 'l'
mov     byte ptr [ebx+3], 'l'
mov     byte ptr [ebx+4], 'o'
mov     byte ptr [ebx+5], ' '
mov     byte ptr [ebx+6], 'w'
mov     byte ptr [ebx+7], 'o'
mov     byte ptr [ebx+8], 'r'
mov     byte ptr [ebx+9], 'l'
mov     byte ptr [ebx+10], 'd'
\end{lstlisting}

\IFRU{Строка также может сравниваться с другой}{The string is also can be compared with another like}:

\begin{lstlisting}
mov	ebx, offset username
cmp	byte ptr [ebx], 'j'
jnz	fail
cmp	byte ptr [ebx+1], 'o'
jnz	fail
cmp	byte ptr [ebx+2], 'h'
jnz	fail
cmp	byte ptr [ebx+3], 'n'
jnz	fail
jz	it_is_john
\end{lstlisting}

\IFRU{В обоих случаях, эти строки нельзя так просто нати в шестнадцатеричном редакторе}
{In both cases, it is impossible to find these strings straightforwardly in hex editor}.

\index{shellcode}
\IFRU{Кстати, точно также со строками можно работать в тех случаях, когда строку нельзя разместить в сегменте данных,
например, в \ac{PIC}, или в шелл-коде.}{By the way, it is a chance to work with the strings when it is impossible
to allocate it in data segment, for example, in \ac{PIC} or in shellcode.}

\IFRU{Еще один виденный мною метод с использованием ф-ции}{Another method I once saw is to use} \TT{sprintf()} 
\IFRU{для конструирования}{for constructing}:

\begin{lstlisting}
sprintf(buf, "%s%c%s%c%s", "hel",'l',"o w",'o',"rld");
\end{lstlisting}

\IFRU{Код выглядит ужасно, но как простейшая мера для анти-реверсинга, это может помочь}
{The code looks weird, but as a simpliest anti-reversing measure it may be helpul}.

\IFRU{Текстовые строки могут также присутствовать в зашифрованном виде, в таком случае,
их использование будет предварять вызов ф-ции для дешифровки}
{Text strings may also be present in encrypted form, then all string usage will precede string decrypting routine}.

\section{\IFRU{Исполняемый код}{Executable code}}

\subsection{\IFRU{Вставка мусора}{Inserting garbage}}

\IFRU{Обфускация исполняемого кода это вставка случайного мусора (между настоящим кодом), который исполняется, но не делает
ничего полезного}{Executable code obfuscation mean inserting random garbage code between real one,
which executes but not doing anything useful}.

\IFRU{Просто пример}{Simple example is}:

\begin{lstlisting}
add	eax, ebx
mul	ecx
\end{lstlisting}

\lstinputlisting[caption=obfuscated code]{patterns/obfuscation/1.asm.\LANG}

\IFRU{Здесь код-мусор использует регистры, которые не используются в настоящем коде}
{Here garbage code uses registers which are not used in the real code} (\TT{ESI} \AndENRU \TT{EDX}).
\IFRU{Впрочем, промежуточные результаты полученные при исполнении настоящего кода вполне могут использоваться
кодом-мусором для б\'{о}льшей путанницы}{However, intermediate results produced by the real code 
may be used by garbage instructions for extra mess}\EMDASH{}\IFRU{почему нет}{why not}?

\subsection{\IFRU{Замена инструкций на раздутые эквиваленты}{Replacing instructions to bloated equivalents}}

\begin{itemize}
\item \TT{MOV op1, op2} \IFRU{может быть заменена на пару}{can be replaced by} \TT{PUSH op2 / POP op1}\EN{ pair}.
\item \TT{JMP label} \IFRU{может быть заменена на пару}{can be replaced by} \TT{PUSH label / RET}\EN{ pair}. 
\IDA{} \IFRU{не покажет ссылок на эту метку}{will not show references to the label}.
\item \TT{CALL label} \IFRU{может быть заменена на тройку}{can be replaced by}
\TT{PUSH label\_after\_CALL\_instruction / PUSH label / RET}\EN{ triplet}.
\item \TT{PUSH op} \IFRU{также можно заменить на пару}{may also be replaced by} 
\TT{SUB ESP, 4 (\OrENRU 8) / MOV [ESP], op}\EN{ pair}.
\end{itemize}

\subsection{\IFRU{Всегда исполняющийся/никогда не исполняющийся код}{Always executed/never executed code}}

\IFRU{Если разработчик уверен что в}{If the developer is sure that} ESI \IFRU{всегда будет 0 в этом месте}
{at the point is always 0}:

\lstinputlisting{patterns/obfuscation/2.asm.\LANG}

Reverse engineer\IFRU{-у понадобится какое-то время чтобы с этим разобраться}{ need some time to get into it}.

\index{opaque predicate}
\IFRU{Это также называется}{This is also called} \IT{opaque predicate}.

\IFRU{Еще один пример}{Another example} (\IFRU{и снова разработчик уверен что}
{and again, developer is sure that} ESI\EMDASH{}\IFRU{всегда ноль}{is always zero}):

\lstinputlisting{patterns/obfuscation/3.asm.\LANG}

\subsection{\IFRU{Сделать побольше путанницы}{Making a lot of mess}}

\begin{lstlisting}
instruction 1
instruction 2
instruction 3
\end{lstlisting}

\IFRU{Можно заменить на}{Can be replaced to}:

\begin{lstlisting}
begin:		jmp	ins1_label

ins2_label:	instruction 2
		jmp	ins3_label

ins3_label:	instruction 3
		jmp	exit:

ins1_label:	instruction 1
		jmp	ins2_label
exit:
\end{lstlisting}

\subsection{\IFRU{Использование косвенных указателей}{Using indirect pointers}}

\begin{lstlisting}
dummy_data1	db	100h dup (0)
message1	db	'hello world',0

dummy_data2	db	200h dup (0)
message2	db	'another message',0

func		proc
		...
		mov	eax, offset dummy_data1 ; PE or ELF reloc here
		add	eax, 100h
		push	eax
		call	dump_string
		...
		mov	eax, offset dummy_data2 ; PE or ELF reloc here
		add	eax, 200h
		push	eax
		call	dump_string
		...
func		endp
\end{lstlisting}

\IDA{} \IFRU{покажет ссылки на}{will show references only to} \TT{dummy\_data1} \AndENRU \TT{dummy\_data2}, 
\IFRU{но не на сами текстовые строки}{but not to the text strings}.

\IFRU{К глобальным переменным и даже ф-циям можно обращаться так же}
{Global variables and even functions may be accessed like that}.

\section{\IFRU{Виртуальная машина / псевдо-код}{Virtual machine / pseudo-code}}

\IFRU{Программист может также создать свой собственный}
{Programmer may construct his/her own} \ac{PL} \OrENRU \ac{ISA} \IFRU{и интерпретатор для него}{and interpreter for it}.
(\IFRU{Как версии Visual Basic перед 5.0}{Like pre-5.0 Visual Basic}, .NET, Java machine).
Reverse engineer\IFRU{-у придется потратить какое-то время для понимания деталей всех инструкций в}
{will have to spend some time to understand meaning and details of all} \ac{ISA}\EN{ instructions}.
\IFRU{Ему также возможно придется писать что-то вроде дизассемблера/декомпилятора}
{Probably, he/she will also need to write a disassembler/decompiler of some sort}.

\section{\IFRU{Еще кое-что}{Other thing to mention}}

\IFRU{Моя попытка (хотя и слабая) пропатчить компилятор Tiny C чтобы он выдавал обфусцированный код}
{My own (yet weak) attempt to patch Tiny C compiler to produce obfuscated code}: \url{http://blog.yurichev.com/node/58}.

\IFRU{Использование инструкции}{Using} \MOV \IFRU{для сложных вещей}
{instruction for really complicated things}: \cite{MOV_is_TM}.


\chapter{Windows 16-bit}
\index{Windows!Windows 3.x}

\RU{16-битные программы под Windows в наше время редки, хотя я иногда вожусь с ними, в смысле ретрокомпьютинга,
либо защищенные донглами (\ref{dongles})}
\EN{16-bit Windows program are rare nowadays, but in the sense of retrocomputing,
or dongle hacking (\ref{dongles}), I sometimes digging into these}.

\RU{16-битные версии Windows были вплоть до}\EN{16-bit Windows versions were up to} 3.11.
96/98/ME \RU{также поддерживает 16-битный код, как и все 32-битные OS линейки}
\EN{also support 16-bit code, as well as 32-bit versions of} \gls{Windows NT}\EN{ line}.
\RU{64-битные версии}\EN{64-bit versions of} \gls{Windows NT} \RU{не поддерживают 16-битный код вообще}
\EN{line are not support 16-bit executable code at all}.

\RU{Код напоминает тот что под MS-DOS}\EN{The code is resembling MS-DOS one}.

\RU{Исполняемые файлы имеют не MZ-тип, и не PE-тип, а NE-тип (так называемый ``new executable'')}
\EN{Executable files has not MZ-type, nor PE-type, they are NE-type (so-called ``new executable'')}.

\RU{Все рассмотренные здесь примеры скомпилированы компилятором}
\EN{All examples considered here were compiled by} OpenWatcom 1.9 \RU{используя эти опции}\EN{compiler, using these switches}:\\
\TT{wcl.exe -i=C:/WATCOM/h/win/ -s -os -bt=windows -bcl=windows example.c}

\subsection{\Example \#1}

\begin{lstlisting}[style=customc]
#include <windows.h>

int PASCAL WinMain( HINSTANCE hInstance,
                    HINSTANCE hPrevInstance,
                    LPSTR lpCmdLine,
                    int nCmdShow )
{
	MessageBeep(MB_ICONEXCLAMATION);
	return 0;
};
\end{lstlisting}

\begin{lstlisting}[style=customasmx86]
WinMain         proc near
                push    bp
                mov     bp, sp
                mov     ax, 30h ; '0'   ; MB_ICONEXCLAMATION constant
                push    ax
                call    MESSAGEBEEP
                xor     ax, ax          ; return 0
                pop     bp
                retn    0Ah
WinMain         endp
\end{lstlisting}

\RU{Пока всё просто}\EN{Seems to be easy, so far}.

\subsection{\Example{} \#2}
\label{win16_messagebox}

\begin{lstlisting}
#include <windows.h>

int PASCAL WinMain( HINSTANCE hInstance,
                    HINSTANCE hPrevInstance,
                    LPSTR lpCmdLine,
                    int nCmdShow )
{
	MessageBox (NULL, "hello, world", "caption", MB_YESNOCANCEL);
	return 0;
};
\end{lstlisting}

\begin{lstlisting}
WinMain         proc near
                push    bp
                mov     bp, sp
                xor     ax, ax          ; NULL
                push    ax
                push    ds
                mov     ax, offset aHelloWorld ; 0x18. "hello, world"
                push    ax
                push    ds
                mov     ax, offset aCaption ; 0x10. "caption"
                push    ax
                mov     ax, 3           ; MB_YESNOCANCEL
                push    ax
                call    MESSAGEBOX
                xor     ax, ax          ; return 0
                pop     bp
                retn    0Ah
WinMain         endp

dseg02:0010 aCaption        db 'caption',0
dseg02:0018 aHelloWorld     db 'hello, world',0
\end{lstlisting}

\IFRU{Пара важных моментов: соглашение о передаче аргументов здесь \TT{PASCAL}: оно указывает что самый
последний аргумент должен передаваться первым}
{Couple important things here: \TT{PASCAL} calling convention dictates passing the last argument first} 
(\TT{MB\_YESNOCANCEL}), \IFRU{а самый первый аргмент\EMDASH{}последним}{and the first argument\EMDASH{}last} (NULL).
\IFRU{Это соглашение также указывает вызываемой ф-ции восстановить}
{This convention also tells \gls{callee} to restore} \gls{stack pointer}:
\IFRU{поэтому инструкция}{hence} \TT{RETN} \IFRU{имеет аргумент}{instruction has} \TT{0Ah} 
\IFRU{означая что указатель нужно сдвинуть вперед на $10$ байт во время возврата из ф-ции}
{argument, meaning pointer should be shifted above by $10$ bytes upon function exit}.

\IFRU{Указатели передаются парами: сначала сегмент данных, потом указатель внутри сегмента}
{Pointers are passed by pairs: a segment of data is first passed, then the pointer inside of segment}.
\IFRU{В этом примере только один сегмент, так что \TT{DS} всегда указывает на сегмент данных в исполняемом
файле}{Here is only one segment in this example, so \TT{DS} is always pointing to data segment of executable}.


\section{\Example{} \#3}

\lstinputlisting{patterns/win16/ex3.c}

\lstinputlisting{patterns/win16/ex3.lst}

\RU{Немного расширенная версия примера из предыдущей секции}
\EN{Somewhat extended example from the previous section}.

\section{\Example{} \#4}

\label{win16_32bit_values}

\lstinputlisting{patterns/win16/ex4.c}

\lstinputlisting{patterns/win16/ex4.lst}

\IFRU{32-битные значения (тип данных \TT{long} означает 32-бита, а \Tint здесь 16-битный) 
в 16-битном коде (и в MS-DOS и в Win16) передаются парами)}
{32-bit values (\TT{long} data type mean 32-bit, while \Tint is fixed on 16-bit data type)
in 16-bit code (both MS-DOS and Win16) are passed by pairs}.
\IFRU{Это так же как и 64-битные значения передаются в 32-битной среде}
{It is just like 64-bit values are used in 32-bit environment} (\ref{sec:64bit_in_32_env}).

\TT{sub\_B2 here} \IFRU{здесь это библиотечная ф-ция написанная разработчиками компилятора, делающая}
{is a library function written by compiler developers, doing} ``long multiplication'', \IFRU{т.е., перемножает
два 32-битных значения}{i.e., multiplies two 32-bit values}.
\IFRU{Другие ф-ции компиляторов делающие то же самое перечислены здесь}
{Other compiler functions doing the same are listed here}: \ref{sec:MSVC_library_func}, \ref{sec:GCC_library_func}.

\index{x86!\Instructions!ADD}
\index{x86!\Instructions!ADC}
\RU{Пара инструкций }\TT{ADD}/\TT{ADC} \IFRU{используется для сложения этих составных значений}
{instruction pair is used for addition of compound values}: 
\TT{ADD} \IFRU{может установить или сбросить флаг}{may set/clear} \TT{CF}\EN{ carry flag}, \TT{ADC} \IFRU{будет
использовать его}{will use it}.
\index{x86!\Instructions!ADD}
\index{x86!\Instructions!ADC}
\RU{Пара инструкций }\TT{SUB}/\TT{SBB} \IFRU{используется для вычитания}{instruction pair is used for subtraction}: 
\TT{SUB} \IFRU{может установить или сбросить флаг}{may set/clear} \TT{CF}\EN{ flag}, \TT{SBB} \IFRU{будет использовать
его}{will use it}.

\IFRU{32-битные значения возвращаются из ф-ций в паре регистров \TT{DX:AX}}
{32-bit values are returned from functions in \TT{DX:AX} register pair}.

\IFRU{Константы так же передаются как пары в}{Constant also passed by pairs in} \TT{WinMain()}\EN{ here}.

\index{x86!\Instructions!CWD}
\IFRU{Константа 123 типа \Tint в начале конвертируется (учитывая знак) в 32-битное значение 
используя инструкция \TT{CWD}}
{\Tint{}-typed 123 constant is first converted respecting its sign into 32-bit value using \TT{CWD} instruction}.


\section{\Example{} \#5}
\label{win16_near_far_pointers}

\lstinputlisting{patterns/win16/ex5.c}

\lstinputlisting{patterns/win16/ex5.lst}

\index{8086!\IFRU{Модель памяти}{Memory model}}
\IFRU{Здесь мы можем увидеть разницу между указателями}
{Here we see a difference between so-called} ``near'' \IFRU{и указателями}{pointers and} ``far'' 
\IFRU{еще один ужасный артефакт сегментированной памяти 16-битного 8086}
{pointers: another weird artefact of segmented memory of 16-bit 8086}.

\IFRU{Читайте больше об этом}{Read more about it}: \ref{8086_memory_model}.

\RU{Указатели }``near'' \IFRU{(``близкие'') это те которые указывают в пределах текущего сегмента}
{pointers are those which points within current data segment}.
\IFRU{Поэтому}{Hence}, \RU{ф-ция }\TT{string\_compare()} \IFRU{берет на вход только 2 16-битных
значения и работает с данными расположеными в сегменте, на который указывает \TT{DS}}{function takes only
two 16-bit pointers, and accesses data as it is located in the segment \TT{DS} pointing to} 
(\RU{инструкция }\TT{mov al, [bx]} \IFRU{на самом деле работает как}{instruction actually works like} 
\TT{mov al, ds:[bx]}\EMDASH{}\TT{DS} \IFRU{используется здесь неявно}{is implicitly used here}).

\RU{Указатели }``far'' \IFRU{(далекие) могут указывать на данные в другом сегменте памяти}
{pointers are those which may point to data in another segment memory}.
\IFRU{Поэтому}{Hence} \TT{string\_compare\_far()} \IFRU{берет на вход 16-битную пару как указатель, загружает старшую
часть в сегментный регистр \TT{ES} и обращается к данным через него}
{takes 16-bit pair as a pointer, loads high part of it to \TT{ES} segment register and accessing
data through it} (\TT{mov al, es:[bx]}).
\RU{Указатели }``far'' \IFRU{также используются в моем win16-примере касательно}
{pointers are also used in my} \TT{MessageBox()}\EN{ win16 example}: \ref{win16_messagebox}. 
\IFRU{Действительно, ядро Windows должно знать, из какого сегмента данных читать текстовые строки, так что ему нужна
полная информация}{Indeed, Windows kernel is not aware which data segment to use when accessing text strings,
so it need more complete information}.

\IFRU{Причина этой разница в том что компактная программа вполне может обойтись одним сегментом данных размером 64 килобайта,
так что старшую часть указателя передавать не нужна (ведь она одинаковая везде)}
{The reason for this distinction is that compact program may use just one 64kb data segment, so it doesn't need
to pass high part of the address, which is always the same}.
\IFRU{Б\`{о}льшие программы могут использовать несколько сегментов данных размером 64 килобайта,
так что нужно указывать каждый раз, в каком сегменте расположены данные}
{Bigger program may use several 64kb data segments, so it needs to specify each time, in which segment data is located}.

\IFRU{То же касается и сегментов кода}{The same story for code segments}.
\IFRU{Компактная программа может расположиться в пределах одного 64kb-сегмента, тогда
ф-ции в ней будут вызываться инструкцией}{Compact program may have all executable code within one 64kb-segment, 
then all functions will be called in it using} 
\TT{CALL NEAR}\IFRU{, а возвращаться управление используя}{ instruction, and code flow will be returned using} \TT{RETN}.
\IFRU{Но если сегментов кода несколько, тогда и адрес вызываемой ф-ции будет задаваться парой, 
вызываться она будет используя}
{But if there are several code segments, then the address of the function will be specified by pair,
it will be called using}
\TT{CALL FAR}\IFRU{, а возвращаться управление используя}{ instruction, and the code flow will be returned using} \TT{RETF}.

\IFRU{Это то что задается в компиляторе указывая}{This is what to be set in compiler by specifying} ``memory model''.

\IFRU{Компиляторы под MS-DOS и Win16 имели разные библиотеки под разные модели памяти: они отличались типами указателей для
кода и данных}{Compilers targeting MS-DOS and Win16 has specific libraries for each memory model: they were differ
by pointer types for code and data}.


\subsection{\Example{} \#6}

\lstinputlisting{patterns/win16/ex6.c}

\lstinputlisting{patterns/win16/ex6.lst}

\index{\CStandardLibrary!time()}
\index{\CStandardLibrary!localtime()}
\IFRU{Время в формате UNIX это 32-битное значение, так что оно возвращается в паре регистров \TT{DX:AX} и сохраняется
в двух локальны 16-битных переменных}
{UNIX time is 32-bit value, so it is returned in \TT{DX:AX} register pair and stored into two local 16-bit variables}.
\IFRU{Потом указатель на эту пару передается в ф-цию}{Then a pointer to the pair is passed to}
\TT{localtime()}\EN{ function}.
\IFRU{Ф-ция}{The} \TT{localtime()} \IFRU{имеет структуру}{function has} \TT{struct tm} \IFRU{расположенную у себя
где-то внутри, так что только указатель на нее возвращается}
{allocated somewhere in guts of the C library, so only pointer to it is returned}. 
\IFRU{Кстати, это также означает что функцию нельзя вызывать еще раз, пока её результаты не были использованы}
{By the way, this is also means that the function cannot be called again until its results are used}.

\IFRU{Для ф-ций}{For the} \TT{time()} \AndENRU \TT{localtime()} \IFRU{используется
Watcom-соглашение о вызовах: первые четыре аргумента передаются через регистры}
{functions, a Watcom calling convention is used here:
first four arguments are passed in} \TT{AX}, \TT{DX}, \TT{BX} \AndENRU \TT{CX}, \IFRU{а остальные аргументы через стек}
{registers, all the rest arguments are via stack}.
\IFRU{Ф-ции, использующие это соглашение, маркируется символом подчеркивания в конце имени}
{Functions used this convention are also marked by underscore at the end of name}.

\RU{Для вызова ф-ции }\TT{sprintf()} \IFRU{используется обычное соглашение \IT{cdecl} (\ref{cdecl}) вместо 
\TT{PASCAL} или Watcom, так что аргументы передаются привычным образом}
{does not use \TT{PASCAL} calling convention, nor Watcom one,
so the arguments are passed in usual \IT{cdecl} way (\ref{cdecl})}.

\subsubsection{\IFRU{Глобальные переменные}{Global variables}}

\IFRU{Это тот же пример, только переменные теперь глобальные}
{This is the same example, but now these variables are global}:

\lstinputlisting{patterns/win16/ex6_global.c}

\lstinputlisting{patterns/win16/ex6_global.lst}

\TT{t} \IFRU{не будет использоваться, но компилятор создал код, записывающий в эту переменную}
{will not be used, but compiler emitted the code which stores the value}.
\IFRU{Потому что он не уверен, может быть это значение где-то еще будет прочитано}
{Because it is not sure, maybe that value will be eventually used somewhere}.



