\section{\Stack}
\label{sec:stack}
\index{\Stack}

\IFRU{Стек в компьютерных науках ~--- это одна из наиболее фундаментальных вещей}
{Stack~---is one of the most fundamental things in computer science}
\footnote{\url{http://en.wikipedia.org/wiki/Call_stack}}.

\IFRU{Технически, это просто блок памяти в памяти процесса + регистр \ESP или \RSP в x86 или x64, либо \SP в ARM, который указывает где-то в пределах этого блока.}
{Technically it is just a memory block in process memory + the an \ESP or the \RSP register in x86 or x64 or the \SP register in ARM as a pointer within the block.}

\index{ARM!\Instructions!PUSH}
\index{ARM!\Instructions!POP}
\index{x86!\Instructions!PUSH}
\index{x86!\Instructions!POP}
\IFRU{Часто используемые инструкции для работы со стеком это \PUSH и \POP (в x86 и thumb-режиме ARM). 
\PUSH уменьшает \ESP/\RSP/\SP на $4$ в 32-битном режиме (или на $8$ в 64-битном),
затем записывает по адресу на который указывает \ESP/\RSP/\SP содержимое своего единственного операнда.}
{The most frequently used stack access instructions are \PUSH and \POP (in both x86 and ARM thumb-mode). 
\PUSH subtracts $4$ in 32-bit mode (or $8$ in 64-bit mode) from \ESP/\RSP/\SP and then writes the contents of its sole operand to the memory address pointed to by \ESP/\RSP/\SP.} 

\IFRU{\POP это обратная операция ~--- сначала достает из \glslink{stack pointer}{указателя стека} значение и помещает его в операнд 
(который очень часто является регистром) и затем увеличивает указатель стека на $4$ (или $8$).}
{\POP is the reverse operation: get the data from memory pointed to by \SP, 
put it in the operand (often a register) and then add $4$ (or $8$) to the \gls{stack pointer}.}

\IFRU{В самом начале, \glslink{stack pointer}{регистр-указатель} указывает на конец стека.}
{After stack allocation the \gls{stack pointer} points to the end of stack.}
\IFRU{\PUSH уменьшает \glslink{stack pointer}{регистр-указатель}, а \POP ~--- увеличивает.}
{\PUSH increases the \gls{stack pointer} and \POP decreases it.}
\IFRU{Конец стека находится в начале блока памяти, выделенного под стек. Это странно, но это так.}
{The end of the stack is actually at the beginning of the memory allocated for the stack block. 
It seems strange, but it is so.}

\IFRU{В процессоре ARM, тем не менее, есть поддержка стеков, растущих как в сторону уменьшения, так и в
сторону увеличения}
{Nevertheless ARM has not only instructions supporting ascending stacks but also descending stacks}. \\
\\
\index{ARM!\Instructions!STMFD}
\index{ARM!\Instructions!LDMFD}
\index{ARM!\Instructions!STMED}
\index{ARM!\Instructions!LDMED}
\index{ARM!\Instructions!STMFA}
\index{ARM!\Instructions!LDMFA}
\index{ARM!\Instructions!STMEA}
\index{ARM!\Instructions!LDMEA}
\IFRU{Например, инструкции}{For example the} 
STMFD\footnote{\STMFDdesc}/LDMFD\footnote{\LDMFDDESC}, 
STMED\footnote{\STMEDdesc}/LDMED\footnote{\LDMEDdesc} 
\IFRU{предназначены для descending-стека, т.е., уменьшающегося}{instructions are intended to deal with 
a descending stack}.
\IFRU{Инструкции}{The}
STMFA\footnote{\STMFAdesc}/LMDFA\footnote{\LDMFAdesc}, 
STMEA\footnote{\STMEAdesc}/LDMEA\footnote{\LDMEAdesc} 
\IFRU{предназначены для ascending-стека, т.е., увеличивающегося}{instructions are intended to deal with 
an ascending stack}.

\subsection{\IFRU{Почему стек растет в обратную сторону?}{Why stack grows backward?}}

\IFRU{Интуитивно, мы можем подумать что, как и любая другая структура данных, стек мог бы расти вперед, 
т.е., в сторону увеличения адресов}{Intuitively, we might think that, like any other data structure, 
stack may grow upward, i.e., towards higher addresses}.

\IFRU{Причина, почему стек растет назад, вероятно, историческая}
{The reason stack grows backward is probably historical}.
\IFRU{Когда компьютеры были большие и занимали целую комнату, было очень легко разделить сегмент на две части,
для \glslink{heap}{кучи} и стека}{When computers were big and occupied whole room, 
it was easy to divide segment by two parts, for the \gls{heap} and the stack}.
\IFRU{Конечно, ведь зараннее было неизвестно, насколько большой может быть \glslink{heap}{куча} или стек, 
так что это решение было самым простым}{Of course, 
it was unknown how big \gls{heap} and stack could be during program execution, 
so this solution was simplest possible}.

\begin{center}
	\begin{tikzpicture}
	\tikzstyle{every path}=[thick]

	\node [rectangle,draw,minimum width=6cm, minimum height=2cm] (memory) {};
	\node [] [right=0.2cm of memory.west] (heap) {Heap};
	\node [] [left=0.2cm of memory.east] (stack) {Stack};

	\node [] (center1) [right=2cm of memory.west] {};
	\node [] (center2) [left=2cm of memory.east] {};

	\draw [->] (heap) -- (center1);
	\draw [->] (stack) -- (center2);

	\node [] [above left=1.1cm and 0.2cm of heap] (t1) {\IFRU{Начало кучи}{Start of heap}};
	\node [] [above right=1.1cm and 0.2cm of stack] (t2) {\IFRU{Вершина стека}{Start of stack}};

	\draw [->] (t1) -- (memory.west);
	\draw [->] (t2) -- (memory.east);

	\end{tikzpicture}
\end{center}

\subsection{\IFRU{Для чего используется стек?}{What is the stack used for?}}

% subsubsections here
\EN{\subsubsection{Save the function's return address}

\myparagraph{x86}

\myindex{x86!\Instructions!CALL}
When calling another function with a \CALL instruction, the address of the point exactly after the \CALL instruction is saved 
to the stack and then an unconditional jump to the address in the \CALL operand is executed.

\myindex{x86!\Instructions!PUSH}
\myindex{x86!\Instructions!JMP}
The \CALL instruction is equivalent to a\\
\INS{PUSH address\_after\_call / JMP operand} instruction pair.

\myindex{x86!\Instructions!RET}
\myindex{x86!\Instructions!POP}
\RET fetches a value from the stack and jumps to it~---that is equivalent to a \TT{POP tmp / JMP tmp} instruction pair.

\myindex{\Stack!\MLStackOverflow}
\myindex{\Recursion}
Overflowing the stack is straightforward. Just run eternal recursion:

\begin{lstlisting}[style=customc]
void f()
{
	f();
};
\end{lstlisting}

MSVC 2008 reports the problem:

\begin{lstlisting}
c:\tmp6>cl ss.cpp /Fass.asm
Microsoft (R) 32-bit C/C++ Optimizing Compiler Version 15.00.21022.08 for 80x86
Copyright (C) Microsoft Corporation.  All rights reserved.

ss.cpp
c:\tmp6\ss.cpp(4) : warning C4717: 'f' : recursive on all control paths, function will cause runtime stack overflow
\end{lstlisting}

\dots but generates the right code anyway:

\begin{lstlisting}[style=customasmx86]
?f@@YAXXZ PROC			; f
; File c:\tmp6\ss.cpp
; Line 2
	push	ebp
	mov	ebp, esp
; Line 3
	call	?f@@YAXXZ	; f
; Line 4
	pop	ebp
	ret	0
?f@@YAXXZ ENDP			; f
\end{lstlisting}

\dots Also if we turn on the compiler optimization (\TT{\Ox} option) the optimized code will not overflow the stack 
and will work \IT{correctly}\footnote{irony here} instead:

\begin{lstlisting}[style=customasmx86]
?f@@YAXXZ PROC			; f
; File c:\tmp6\ss.cpp
; Line 2
$LL3@f:
; Line 3
	jmp	SHORT $LL3@f
?f@@YAXXZ ENDP			; f
\end{lstlisting}

GCC 4.4.1 generates similar code in both cases without, however,  issuing any warning about the problem.

\myparagraph{ARM}

\myindex{ARM!\Registers!Link Register}
ARM programs also use the stack for saving return addresses, but differently.
As mentioned in \q{\HelloWorldSectionName}~(\myref{sec:hw_ARM}),
the \ac{RA} is saved to the \ac{LR} (\gls{link register}).
If one needs, however, to call another function and use the \ac{LR} register
one more time, its value has to be saved.
\myindex{Function prologue}
Usually it is saved in the function prologue.

\myindex{ARM!\Instructions!PUSH}
\myindex{ARM!\Instructions!POP}
Often, we see instructions like \INS{PUSH {R4-R7,LR}} along with this instruction in epilogue
\INS{POP {R4-R7,PC}}---thus register values to be used in the function are saved in the stack, including \ac{LR}.

\myindex{ARM!Leaf function}
Nevertheless, if a function never calls any other function, in \ac{RISC} terminology it is called a
\IT{\gls{leaf function}}\footnote{\href{http://go.yurichev.com/17064}{infocenter.arm.com/help/index.jsp?topic=/com.arm.doc.faqs/ka13785.html}}. 
As a consequence, leaf functions do not save the \ac{LR} register (because they don't modify it).
If such function is small and uses a small number of registers, it may not use the stack at all.
Thus, it is possible to call leaf functions without using the stack,
which can be faster than on older x86 machines because external RAM is not used for the stack
\footnote{Some time ago, on PDP-11 and VAX, the CALL instruction (calling other functions) was expensive; up to 50\%
of execution time might be spent on it, so it was considered that having a big number of small functions is an \gls{anti-pattern} \InSqBrackets{\TAOUP Chapter 4, Part II}.}.
This can be also useful for situations when memory for the stack is not yet allocated or not available.

Some examples of leaf functions:
\myref{ARM_leaf_example1}, \myref{ARM_leaf_example2}, 
\myref{ARM_leaf_example3}, \myref{ARM_leaf_example4}, \myref{ARM_leaf_example5},
\myref{ARM_leaf_example6}, \myref{ARM_leaf_example7}, \myref{ARM_leaf_example10}.

}
\RU{\subsubsection{Сохранение адреса возврата управления}

\myparagraph{x86}

\myindex{x86!\Instructions!CALL}
При вызове другой функции через \CALL сначала в стек записывается адрес, указывающий на место после 
инструкции \CALL, затем делается безусловный переход (почти как \TT{JMP}) на адрес, указанный в операнде.

\myindex{x86!\Instructions!PUSH}
\myindex{x86!\Instructions!JMP}
\CALL~--- это аналог пары инструкций \INS{PUSH address\_after\_call / JMP}.

\myindex{x86!\Instructions!RET}
\myindex{x86!\Instructions!POP}
\RET вытаскивает из стека значение и передает управление по этому адресу~--- 
это аналог пары инструкций \TT{POP tmp / JMP tmp}.

\myindex{\Stack!\MLStackOverflow}
\myindex{\Recursion}
Крайне легко устроить переполнение стека, запустив бесконечную рекурсию:

\begin{lstlisting}[style=customc]
void f()
{
	f();
};
\end{lstlisting}

MSVC 2008 предупреждает о проблеме:

\begin{lstlisting}
c:\tmp6>cl ss.cpp /Fass.asm
Microsoft (R) 32-bit C/C++ Optimizing Compiler Version 15.00.21022.08 for 80x86
Copyright (C) Microsoft Corporation.  All rights reserved.

ss.cpp
c:\tmp6\ss.cpp(4) : warning C4717: 'f' : recursive on all control paths, function will cause runtime stack overflow
\end{lstlisting}

\dots но, тем не менее, создает нужный код:

\begin{lstlisting}[style=customasmx86]
?f@@YAXXZ PROC			; f
; File c:\tmp6\ss.cpp
; Line 2
	push	ebp
	mov	ebp, esp
; Line 3
	call	?f@@YAXXZ	; f
; Line 4
	pop	ebp
	ret	0
?f@@YAXXZ ENDP			; f
\end{lstlisting}

\dots причем, если включить оптимизацию (\TT{\Ox}), то будет даже интереснее, без переполнения стека, 
но работать будет \IT{корректно}\footnote{здесь ирония}:

\begin{lstlisting}[style=customasmx86]
?f@@YAXXZ PROC			; f
; File c:\tmp6\ss.cpp
; Line 2
$LL3@f:
; Line 3
	jmp	SHORT $LL3@f
?f@@YAXXZ ENDP			; f
\end{lstlisting}

GCC 4.4.1 генерирует точно такой же код в обоих случаях, хотя и не предупреждает о проблеме.

\myparagraph{ARM}

\myindex{ARM!\Registers!Link Register}
Программы для ARM также используют стек для сохранения \ac{RA}, куда нужно вернуться, но несколько иначе.
Как уже упоминалось в секции \q{\HelloWorldSectionName}~(\myref{sec:hw_ARM}),
\ac{RA} записывается в регистр \ac{LR} (\gls{link register}).
Но если есть необходимость вызывать какую-то другую функцию и использовать регистр \ac{LR} ещё
раз, его значение желательно сохранить.
\myindex{Function prologue}
\myindex{ARM!\Instructions!PUSH}
\myindex{ARM!\Instructions!POP}

Обычно это происходит в прологе функции, часто мы видим там инструкцию вроде \INS{PUSH \{R4-R7,LR\}}, а в эпилоге
\INS{POP \{R4-R7,PC\}}~--- так сохраняются регистры, которые будут использоваться в текущей функции, в том числе \ac{LR}.

\myindex{ARM!Leaf function}
Тем не менее, если некая функция не вызывает никаких более функций, в терминологии \ac{RISC} она называется
\IT{\gls{leaf function}}\footnote{\href{http://go.yurichev.com/17064}{infocenter.arm.com/help/index.jsp?topic=/com.arm.doc.faqs/ka13785.html}}. 
Как следствие, \q{leaf}-функция не сохраняет регистр \ac{LR} (потому что не изменяет его).
А если эта функция небольшая, использует мало регистров, она может не использовать стек вообще.
Таким образом, в ARM возможен вызов небольших leaf-функций не используя стек.
Это может быть быстрее чем в старых x86, ведь внешняя память для стека не используется
\footnote{Когда-то, очень давно, на PDP-11 и VAX на инструкцию CALL (вызов других функций) могло тратиться
вплоть до 50\% времени (возможно из-за работы с памятью),
поэтому считалось, что много небольших функций это \glslink{anti-pattern}{анти-паттерн}
\InSqBrackets{\TAOUP Chapter 4, Part II}.}.
Либо это может быть полезным для тех ситуаций, когда память для стека ещё не выделена, либо недоступна,

Некоторые примеры таких функций:
\myref{ARM_leaf_example1}, \myref{ARM_leaf_example2}, 
\myref{ARM_leaf_example3}, \myref{ARM_leaf_example4}, \myref{ARM_leaf_example5},
\myref{ARM_leaf_example6}, \myref{ARM_leaf_example7}, \myref{ARM_leaf_example10}.

}
\DE{\subsection{Rückgabe Adresse der Funktion speichern}

\myparagraph{x86}

\myindex{x86!\Instructions!CALL}
Wenn man eine Funktion mit der \CALL Instruktion aufruft, wird die Adresse direkt nach der
\CALL Instruktion auf dem Stack gespeichert und der unbedingte jump wird ausgeführt.

\myindex{x86!\Instructions!PUSH}
\myindex{x86!\Instructions!JMP}
Die \CALL Instruktion ist äquivalent zu dem \INS{PUSH address\_after\_call / JMP operand} Instruktions paar.

\myindex{x86!\Instructions!RET}
\myindex{x86!\Instructions!POP}
\RET ruft die Rückkehr Adresse vom Stack ab und springt zu dieser~---was äquivalent zu einem \TT{POP tmp / JMP tmp} Instruktions
paar ist.

\myindex{\Stack!\MLStackOverflow}
\myindex{\Recursion}

Den Stack zum überlaufen zu bringen ist recht einfach, einfach eine 
endlos rekursive Funktion Aufrufen:


\begin{lstlisting}[style=customc]
void f()
{
	f();
};
\end{lstlisting}


MSVC 2008 hat eine Erkennung für das Problem:


\begin{lstlisting}
c:\tmp6>cl ss.cpp /Fass.asm
Microsoft (R) 32-bit C/C++ Optimizing Compiler Version 15.00.21022.08 for 80x86
Copyright (C) Microsoft Corporation.  All rights reserved.

ss.cpp
c:\tmp6\ss.cpp(4) : warning C4717: 'f' : recursive on all control paths, function will cause runtime stack overflow
\end{lstlisting}

\dots aber der Compiler erzeugt den Code trotzdem:

\begin{lstlisting}[style=customasmx86]
?f@@YAXXZ PROC			; f
; File c:\tmp6\ss.cpp
; Line 2
	push	ebp
	mov	ebp, esp
; Line 3
	call	?f@@YAXXZ	; f
; Line 4
	pop	ebp
	ret	0
?f@@YAXXZ ENDP			; f
\end{lstlisting}

\dots Auch wenn wir die Compiler Optimierungen einschalten (\TT{/0x} Option) wird der optimierte Code nicht
den Stack zum überlaufen bringen. Stattdessen wird der Code \IT{korrekt}\footnote{Ironie hier} ausgeführt: 

\begin{lstlisting}[style=customasmx86]
?f@@YAXXZ PROC			; f
; File c:\tmp6\ss.cpp
; Line 2
$LL3@f:
; Line 3
	jmp	SHORT $LL3@f
?f@@YAXXZ ENDP			; f
\end{lstlisting}


GCC 4.4.1 generiert vergleichbaren Code in beiden Fällen, jedoch ohne über das Overflow Problem zu warnen.


\myparagraph{ARM}

\myindex{ARM!\Registers!Link Register}

ARM Programme benutzen den Stack um Rücksprung Adressen zu speichern, aber anders.
Wie bereits erwähnt in \q{\HelloWorldSectionName}~(\myref{sec:hw_ARM}),
wird der \ac{RA} Wert im \ac{LR} (\gls{link register}) gespeichert.
Wenn nun eine andere Funktion aufgerufen werden muss und auf das \ac{LR} Register 
zu greift, muss der aktuelle Wert im Register irgendwo gespeichert werden.

\myindex{Funktion Prologe}
Normal wird der Wert im Funktion Prolog gespeichert.

\myindex{ARM!\Instructions!PUSH}
\myindex{ARM!\Instructions!POP}

Oft sieht man Instruktionen wie z.B \INS{PUSH {R4-R7,LR}} zusammen mit dieser Instruktion im 
Epilog \INS{POP {R4-R7,PC}}---Somit werden Werte die in den Funktionen benötigt werden auf dem 
Stack gespeichert, inklusive \ac{LR}.

\myindex{ARM!Leaf Funktion}
Wenn eine Funktion nie eine andere Funktion aufruft, nennt man das in der \ac{RISC} Terminologie eine
\IT{\glslink{leaf function}{leaf Funktion}}\footnote{\href{http://go.yurichev.com/17064}{infocenter.arm.com/help/index.jsp?topic=/com.arm.doc.faqs/ka13785.html}}.  % <-- attention could be a compilier bug
Als Konsequenz ergibt sich, das leaf Funktionen nicht das \ac{LR} Register speichern (da sie es nicht modifizieren).
Wenn solche Funktionen klein sind und nur eine geringe Anzahl an Registern benutzt, ist es möglich das der Stack
gar nicht benutzt wird. Es ist also möglich leaf Funktionen zu benutzen ohne den Stack zurück zu greifen, die Ausführung
ist hier schneller als auf älteren x86 Maschinen weil kein externer RAM für den Stack benutzt wird 
\footnote{Bis vor einer weile war es sehr teuer auf PDP-11 und VAX Maschinen die CALL Instruktion zu benutzen; bis zu 50\%
der Rechenzeit wurde allein für diese Instruktion verschwendet, man hat dabei festgestellt das eine große Anzahl an kleinen
Funktionen zu haben ein \gls{anti-pattern} \InSqBrackets{\TAOUP Chapter 4, Part II}.} ist.
Diese Eigenschaft kann nützlich sein wenn der Speicher für den Stack noch nicht alloziert oder verfügbar ist.

Ein paar Beispiele für leaf Funktionen:

\myref{ARM_leaf_example1}, \myref{ARM_leaf_example2}, 
\myref{ARM_leaf_example3}, \myref{ARM_leaf_example4}, \myref{ARM_leaf_example5},
\myref{ARM_leaf_example6}, \myref{ARM_leaf_example7}, \myref{ARM_leaf_example10}.

}
\FR{\subsubsection{Sauvegarder l'adresse de retour de la fonction}

\myparagraph{x86}

\myindex{x86!\Instructions!CALL}
Lorsque l'on appelle une fonction avec une instruction \CALL, l'adresse du point
exactement après cette dernière est sauvegardée sur la pile et un saut inconditionnel
à l'adresse de l'opérande \CALL est exécuté.

\myindex{x86!\Instructions!PUSH}
\myindex{x86!\Instructions!JMP}
L'instruction \CALL est équivalente à la\\
paire d'instructions \INS{PUSH address\_after\_call / JMP operand}.

\myindex{x86!\Instructions!RET}
\myindex{x86!\Instructions!POP}
\RET va chercher une valeur sur la pile et y saute~---ce qui est équivalent à
la paire d'instructions \TT{POP tmp / JMP tmp}.

\myindex{\Stack!\MLStackOverflow}
\myindex{\Recursion}
Déborder de la pile est très facile. Il suffit de lancer une récursion éternelle:

\begin{lstlisting}[style=customc]
void f()
{
	f();
};
\end{lstlisting}

MSVC 2008 signale le problème:

\begin{lstlisting}
c:\tmp6>cl ss.cpp /Fass.asm
Microsoft (R) 32-bit C/C++ Optimizing Compiler Version 15.00.21022.08 for 80x86
Copyright (C) Microsoft Corporation.  All rights reserved.

ss.cpp
c:\tmp6\ss.cpp(4) : warning C4717: 'f' : recursive on all control paths, function will cause runtime stack overflow
\end{lstlisting}

\dots mais génère tout de même le code correspondant:

\begin{lstlisting}[style=customasmx86]
?f@@YAXXZ PROC			; f
; File c:\tmp6\ss.cpp
; Line 2
	push	ebp
	mov	ebp, esp
; Line 3
	call	?f@@YAXXZ	; f
; Line 4
	pop	ebp
	ret	0
?f@@YAXXZ ENDP			; f
\end{lstlisting}

\dots Si nous utilisons l'option d'optimisation du compilateur (option \TT{\Ox})
le code optimisé ne va pas déborder de la pile et au lieu de cela va fonctionner
\IT{correctemment}\footnote{ironique ici}:

\begin{lstlisting}[style=customasmx86]
?f@@YAXXZ PROC			; f
; File c:\tmp6\ss.cpp
; Line 2
$LL3@f:
; Line 3
	jmp	SHORT $LL3@f
?f@@YAXXZ ENDP			; f
\end{lstlisting}

GCC 4.4.1 génère un code similaire dans les deux cas, sans, toutefois émettre
d'avertissement à propos de ce problème.

\myparagraph{ARM}

\myindex{ARM!\Registers!Link Register}
Les programmes ARM utilisent également la pile pour sauver les adresses de retour,
mais différemment.
Comme mentionné dans \q{\HelloWorldSectionName}~(\myref{sec:hw_ARM}),
\ac{RA} est sauvegardé dans \ac{LR} (\gls{link register}).
Si l'on a toutefois besoin d'appeler une autre fonction et d'utiliser le registre
\ac{LR} une fois de plus, sa valeur doit être sauvegardée.
\myindex{Function prologue}
Usuellement, cela se fait dans le prologue de la fonction.

\myindex{ARM!\Instructions!PUSH}
\myindex{ARM!\Instructions!POP}
Souvent, nous voyons des instructions comme \INS{PUSH {R4-R7,LR}} en même temps
que cette instruction dans le prologue \INS{POP {R4-R7,PC}}---ces registres qui
sont utilisés dans la fonction sont sauvegardés sur la pile, \ac{LR} inclus.

\myindex{ARM!Fonction leaf} % FIXME traduire avec feuille ?
Néanmoins, si une fonction n'appelle jamais d'autre fonction, dans la terminologie
\ac{RISC} elle est appelée \IT{\glslink{leaf function}{fonction leaf}}\footnote{\href{http://go.yurichev.com/17064}{infocenter.arm.com/help/index.jsp?topic=/com.arm.doc.faqs/ka13785.html}}.
Ceci a comme conséquence que les fonctions leaf ne sauvegardent pas le registre
\ac{LR} (car elles ne le modifient pas).
Si une telle fonction est petite et utilise un petit nombre de registres, elle
peut ne pas utiliser du tout la pile.
Ainsi, il est possible d'appeler des fonctions leaf sans utiliser la pile.
Ce qui peut être plus rapide sur des vieilles machines x86 car la mémoire externe
n'est pas utilisée pour la pile
\footnote{Il y a quelques temps, sur PDP-11 et VAX, l'instruction CALL (appel d'autres fonctions) était coûteux; jusqu'à 50\%
du temps d'exécution pouvait être passé à ça, il était donc considèré qu'avoir un grand nombre de petite fonction était un \gls{anti-pattern} \InSqBrackets{\TAOUP Chapter 4, Part II}.}.
Cela peut être utile pour des situations où la mémoire pour la pile n'est pas
encore allouée ou disponible.

Quelques exemples de fonctions leaf:
\myref{ARM_leaf_example1}, \myref{ARM_leaf_example2},
\myref{ARM_leaf_example3}, \myref{ARM_leaf_example4}, \myref{ARM_leaf_example5},
\myref{ARM_leaf_example6}, \myref{ARM_leaf_example7}, \myref{ARM_leaf_example10}.

}
\PTBR{\subsubsection{Salvar o endereço de retorno de uma função}

\myparagraph{x86}

\myindex{x86!\Instructions!CALL}
Quando você chama outra função utilizando a instrução CALL, o endereço do ponto exato onde a 
instrução \CALL se encontra é salvo na pilha e então um jump incondicional para o endereço no operando de \CALL é executado.

\myindex{x86!\Instructions!PUSH}
\myindex{x86!\Instructions!JMP}
A instrução \CALL é equivalente a usar o par de instruções \TT{PUSH endereço\_depois\_chamada / JMP}.

\myindex{x86!\Instructions!RET}
\myindex{x86!\Instructions!POP}
\RET pega um valor da pilha e usa um jump para ele --- isso é equivalente a usar \INS{POP tmp / JMP tmp}.

\myindex{\Stack!\MLStackOverflow}
\myindex{\Recursion}
Estourar uma stack é fácil. Só execute alguma recursão externa:

\begin{lstlisting}[style=customc]
void f()
{
	f();
};
\end{lstlisting}

O compilador MSVC 2008 informa o problema:

\begin{lstlisting}
c:\tmp6>cl ss.cpp /Fass.asm
Microsoft (R) 32-bit C/C++ Optimizing Compiler Version 15.00.21022.08 for 80x86
Copyright (C) Microsoft Corporation.  All rights reserved.

ss.cpp
c:\tmp6\ss.cpp(4) : warning C4717: 'f' : recursive on all control paths, function will cause runtime stack overflow
\end{lstlisting}

\dots mas gera o código de qualquer maneira:

\begin{lstlisting}[style=customasmx86]
?f@@YAXXZ PROC			; f
; File c:\tmp6\ss.cpp
; Line 2
	push	ebp
	mov	ebp, esp
; Line 3
	call	?f@@YAXXZ	; f
; Line 4
	pop	ebp
	ret	0
?f@@YAXXZ ENDP			; f
\end{lstlisting}

\dots também, se ativarmos a otimização do compilador (opção \TT{/Ox}) 
o código otimizado não vai estourar a pilha e funcionará \IT{corretamente} \footnote{ironia aqui}:

\begin{lstlisting}[style=customasmx86]
?f@@YAXXZ PROC			; f
; File c:\tmp6\ss.cpp
; Line 2
$LL3@f:
; Line 3
	jmp	SHORT $LL3@f
?f@@YAXXZ ENDP			; f
\end{lstlisting}

\PTBRph{}

}
\ITA{\subsubsection{Salvare l'indirizzo di ritorno della funzione}

\myparagraph{x86}

\myindex{x86!\Instructions!CALL}
Quando si chiama una funzione con l'istruzione \CALL, l'indirizzo del punto esattamente dopo la \CALL viene salvato nello stack, e successivamente
viene eseguito un jump non condizionale all'indirizzo dell'operando di \CALL.

\myindex{x86!\Instructions!PUSH}
\myindex{x86!\Instructions!JMP}
L'istruzione \CALL e' equivalente alla coppia di istruzioni \INS{PUSH indirizzo\_dopo\_call / JMP operando}.

\myindex{x86!\Instructions!RET}
\myindex{x86!\Instructions!POP}
\RET preleva un valore dallo stack e effettua un jump ad esso~--- cio' equivale alla coppia di istruzioni \TT{POP tmp / JMP tmp}.

\myindex{\Stack!\MLStackOverflow}
\myindex{\Recursion}

Riempire lo stack fino allo straripamento e' semplicissimo. Basta ricorrere alla ricorsione eterna:

\begin{lstlisting}[style=customc]
void f()
{
	f();
};
\end{lstlisting}

MSVC 2008 riporta il problema:

\begin{lstlisting}
c:\tmp6>cl ss.cpp /Fass.asm
Microsoft (R) 32-bit C/C++ Optimizing Compiler Version 15.00.21022.08 for 80x86
Copyright (C) Microsoft Corporation.  All rights reserved.

ss.cpp
c:\tmp6\ss.cpp(4) : warning C4717: 'f' : recursive on all control paths, function will cause runtime stack overflow
\end{lstlisting}

\dots ma genera in ogni caso il codice correttamente:

\begin{lstlisting}[style=customasmx86]
?f@@YAXXZ PROC			; f
; File c:\tmp6\ss.cpp
; Line 2
	push	ebp
	mov	ebp, esp
; Line 3
	call	?f@@YAXXZ	; f
; Line 4
	pop	ebp
	ret	0
?f@@YAXXZ ENDP			; f
\end{lstlisting}

\dots Se attiviamo le ottimizzazioni del compilatore (\TT{\Ox} option) il codice ottimizzato non causera' overflow dello stack 
e funzionera' invece \IT{correttamente}\footnote{sarcasmo, si fa per dire}:

\begin{lstlisting}[style=customasmx86]
?f@@YAXXZ PROC			; f
; File c:\tmp6\ss.cpp
; Line 2
$LL3@f:
; Line 3
	jmp	SHORT $LL3@f
?f@@YAXXZ ENDP			; f
\end{lstlisting}

GCC 4.4.1 genera codice simile in antrambi i casi, senza avvertire del problema.

\myparagraph{ARM}

\myindex{ARM!\Registers!Link Register}
Anche i programmi ARM usano lo stack per salvare gli indirizzi di ritorno, ma lo fanno in maniera diversa.
Come detto in \q{\HelloWorldSectionName}~(\myref{sec:hw_ARM}),
As mentioned in 
il \ac{RA} viene salvato nel \ac{LR} (\gls{link register}).
Se si presenta comunque la necessita' di chiamare un'altra funzione ed usare il registro \ac{LR} ancora una volta, 
il suo valore deve essere salvato.
\myindex{Function prologue}
Solitamente questo valore e' slvato nel preambolo della funzione.

\myindex{ARM!\Instructions!PUSH}
\myindex{ARM!\Instructions!POP}
Spesso vediamo istruzioni come \INS{PUSH {R4-R7,LR}} insieme ad isrtuzioni nell'epilogo come 
\INS{POP {R4-R7,PC}}---percio' i valori dei registri che saranno usati nella funzione vengono salvati nello stack, incluso \ac{LR}.

\myindex{ARM!Leaf function}
Ciononostante, se una funzione non chiama al suo interno nessun'altra funzione, in terminologia \ac{RISC} e' detta 
\IT{\gls{leaf function}}, o funzione foglia.\footnote{\href{http://go.yurichev.com/17064}{infocenter.arm.com/help/index.jsp?topic=/com.arm.doc.faqs/ka13785.html}}. 
Di conseguenza, le leaf functions non salvano il registro \ac{LR} register (perche' difatti non lo modificano).
Se una simile funzione e' molto breve e usa un piccolo numero di registri, potrebbe non usare del tutto lo stack. 
E' quindi possible chiamare le leaf functions senza usare lo stack, cosa che puo' essere piu' veloce che sulle macchine x86 perche' ;a RA< esterna non viene usata per lo stack
\footnote{Tempo fa, su PDP-11 and VAX, l'istruzione CALL instruction (chiamare altre funzioni) era costosa; poteva richiedere fino al 50\%
del tempo di esecuzione, ed era quindi consuetudine pensare che avere un grande numero di piccole funzioni fosee un \gls{anti-pattern} \InSqBrackets{\TAOUP Chapter 4, Part II}.}.
Lo stesso principio puo' tornare utile quando la memoria per lo stack non e' stata ancora allocata o non e' disponibile.

Alcuni esempi di funzioni foglia:
\myref{ARM_leaf_example1}, \myref{ARM_leaf_example2}, 
\myref{ARM_leaf_example3}, \myref{ARM_leaf_example4}, \myref{ARM_leaf_example5},
\myref{ARM_leaf_example6}, \myref{ARM_leaf_example7}, \myref{ARM_leaf_example10}.
}

\subsection{\RU{Передача параметров функции}\EN{Passing function arguments}}

\RU{Самый распространенный способ передачи параметров в x86 называется}
\EN{The most popular way to pass parameters in x86 is called} \q{cdecl}:

\begin{lstlisting}
push arg3
push arg2
push arg1
call f
add esp, 12 ; 4*3=12
\end{lstlisting}

\RU{Вызываемая функция получает свои параметры также через указатель стека.}
\EN{\Gls{callee} functions get their arguments via the stack pointer.}

\RU{Следовательно, так расположены значения в стеке перед исполнением самой первой инструкции
функции \ttf{}:}
\EN{Therefore, this is how the argument values are located in the stack before the execution
of the \ttf{} function's very first instruction:}

\begin{center}
\begin{tabular}{ | l | l | }
\hline
ESP & \RU{адрес возврата}\EN{return address} \\
\hline
ESP+4 & \argument \#1, \MarkedInIDAAs{} \TT{arg\_0} \\
\hline
ESP+8 & \argument \#2, \MarkedInIDAAs{} \TT{arg\_4} \\
\hline
ESP+0xC & \argument \#3, \MarkedInIDAAs{} \TT{arg\_8} \\
\hline
\dots & \dots \\
\hline
\end{tabular}
\end{center}

\ifx\LITE\undefined
\RU{См. также в соответствующем разделе о других способах передачи аргументов через стек}
\EN{For more information on other calling conventions see also section}~(\myref{sec:callingconventions}).
\fi
\RU{Важно отметить, что, в общем, никто не заставляет программистов передавать параметры именно через стек,
это не является требованием к исполняемому коду.}
\EN{It is worth noting that nothing obliges programmers to pass arguments through the stack. It is not a requirement.}
\RU{Вы можете делать это совершенно иначе, не используя стек вообще.}
\EN{One could implement any other method without using the stack at all.}

\RU{К примеру, можно выделять в \glslink{heap}{куче} место для аргументов, 
заполнять их и передавать в функцию указатель на это место через \EAX. И это вполне будет работать}%
\EN{For example, it is possible to allocate a space for arguments in the \gls{heap}, fill it and pass it to a function 
via a pointer to this block in the \EAX register. This will work}%
\footnote{\RU{Например, в книге Дональда Кнута \q{Искусство программирования}, в разделе 1.4.1 
посвященном подпрограммам \cite[раздел 1.4.1]{Knuth:1998:ACP:521463}, 
мы можем прочитать о возможности располагать параметры для вызываемой подпрограммы после инструкции \JMP,
передающей управление подпрограмме. Кнут описывает, что это было особенно удобно для компьютеров IBM System/360.}%
\EN{For example, in the \q{The Art of Computer Programming} book by Donald Knuth, 
in section 1.4.1 dedicated to subroutines \cite[section 1.4.1]{Knuth:1998:ACP:521463},
we could read that one way to supply arguments to a subroutine is simply to list them after the \JMP instruction
passing control to subroutine. Knuth explains that this method was particularly convenient on IBM System/360.}}.
\RU{Однако традиционно сложилось, что в x86 и ARM передача аргументов происходит именно через стек.}
% I am unsure about what this comment means.
% My guess is that the arguments are put in the memory position after
% the jump instruction, so you could say:
% "one way to supply arguments to a subroutine is simply to list them in memory
% after the \JMP instruction that passes control to the subroutine."
% Right now, "after" also sounds like it refers to the time after
% the jump happens, which I think is too late.
\EN{However, it is a convenient custom in x86 and ARM to use the stack for this purpose.} \\
\\
\RU{Кстати, вызываемая функция не имеет информации о количестве переданных ей аргументов.}
\EN{By the way, the \gls{callee} function does not have any information about how many arguments were passed.}
\RU{Функции Си с переменным количеством аргументов (как \printf) определяют их количество по 
спецификаторам строки формата (начинающиеся со знака \%).}
\EN{C functions with a variable number of arguments (like \printf) determine their number using format string  specifiers (which begin with the \% symbol).}
\RU{Если написать что-то вроде}\EN{If we write something like} 

\begin{lstlisting}
printf("%d %d %d", 1234);
\end{lstlisting}

\printf \RU{выведет 1234, затем ещё два случайных числа, которые волею случая оказались в стеке рядом.}
\EN{will print 1234, and then two random numbers, which were lying next to it in the stack.}\\
\\
\RU{Вот почему не так уж и важно, как объявлять функцию \main}
\EN{That's why it is not very important how we declare the \main function}: \RU{как}\EN{as} \main, 
\TT{main(int argc, char *argv[])} 
\RU{либо}\EN{or} \TT{main(int argc, char *argv[], char *envp[])}.

\RU{В реальности, \ac{CRT}-код вызывает \main примерно так:}
\EN{In fact, the \ac{CRT}-code is calling \main roughly as:}

\begin{lstlisting}
push envp
push argv
push argc
call main
...
\end{lstlisting}

\RU{Если вы объявляете \main без аргументов, они, тем не менее, присутствуют в стеке, но не используются.}
\EN{If you declare \main as \main without arguments, they are, nevertheless, still present in the stack, but
are not used.}
\RU{Если вы объявите \main как}\EN{If you declare \main as} \TT{main(int argc, char *argv[])}, 
\RU{вы можете использовать два первых аргумента, а третий останется для вашей функции \q{невидимым}.}
\EN{you will be able to use first two arguments, and the third will remain \q{invisible} for your function.}
\RU{Более того, можно даже объявить}\EN{Even more, it is possible to declare} \TT{main(int argc)}, 
\RU{и это будет работать}\EN{and it will work}.


\subsection{\RU{Хранение локальных переменных}\EN{Local variable storage}}

\RU{Функция может выделить для себя некоторое место в стеке для локальных переменных, просто отодвинув 
\glslink{stack pointer}{указатель стека} глубже к концу стека.}
\EN{A function could allocate space in the stack for its local variables just by decreasing 
the \gls{stack pointer} towards the stack bottom.}
% I think here, "stack bottom" means the lowest address in the stack space,
% but the reader might also think it means towards the top of the stack space,
% like in a pop, so you might change "towards the stack bottom" to
% "towards the lowest address of the stack", or just take it out,
% since "decreasing" also suggests that.
\RU{Это очень быстро вне зависимости от количества локальных переменных.}
\EN{Hence, it's very fast, no matter how many local variables are defined.}

\RU{Хранить локальные переменные в стеке не является необходимым требованием. 
Вы можете хранить локальные переменные где угодно. 
Но по традиции всё сложилось так.}
\EN{It is also not a requirement to store local variables in the stack.
You could store local variables wherever you like, 
but traditionally this is how it's done.}

\subsubsection{x86: \IFRU{Функция alloca()}{alloca() function}}
\label{alloca}
\index{\CStandardLibrary!alloca()}
\IFRU{Интересен случай с функцией \TT{alloca()}}
{It is worth noting \TT{alloca()} function.}\footnote{
\IFRU
{В MSVC, реализацию функции можно посмотреть в файлах}
{As of MSVC, function implementation can be found in} 
  \TT{alloca16.asm} 
  \AndENRU 
  \TT{chkstk.asm} 
  \InENRU 
  \TT{C:\textbackslash{}Program Files (x86)\textbackslash{}Microsoft Visual Studio 10.0\textbackslash{}VC\textbackslash{}crt\textbackslash{}src\textbackslash{}intel}}. 

\IFRU{Эта функция работает как \TT{malloc()}, но выделяет память прямо в стеке.} 
{This function works like \TT{malloc()} but allocates memory just on the stack.}

\IFRU{Память освобождать через \TT{free()} не нужно, так как эпилог функции~(\ref{sec:prologepilog})
вернет \ESP назад в изначальное состояние и выделенная память просто аyнулируется.}
{Allocated memory chunk is not needed to be freed via \TT{free()} function call since 
function epilogue~(\ref{sec:prologepilog}) shall return value of the \ESP back to initial state and 
allocated memory will be just annulled.} 

\IFRU{Интересна реализация функции \TT{alloca()}.}
{It is worth noting how \TT{alloca()} implemented.}

\IFRU{Эта функция, если упрощенно, просто сдвигает \ESP вглубь стека 
на столько байт сколько вам нужно и возвращает \ESP в качестве указателя на выделенный блок.}
{This function, if to simplify, just shifting \ESP deeply to stack bottom so much bytes you 
need and set \ESP as a pointer to the \IT{allocated} block.}
\IFRU{Попробуем:}{Let's try:}

\lstinputlisting{patterns/02_stack/2_1.c}

\IFRU{(Функция \TT{\_snprintf()} работает так же как и \printf, только вместо выдачи результата в 
stdout (т.е., на терминал или в консоль),
записывает его в буфер \TT{buf}. \puts выдает содержимое буфера \TT{buf} в stdout. Конечно, можно было бы
заменить оба этих вызова на один \printf, но мне нужно проиллюстрировать использование небольшого буфера.)}
{(\TT{\_snprintf()} function works just like \printf, but instead dumping result into stdout (e.g., to terminal or 
console), write it to the \TT{buf} buffer. \puts copies \TT{buf} contents to stdout. Of course, these two
function calls might be replaced by one \printf call, but I would like to illustrate small buffer usage.)}

\paragraph{MSVC}

\IFRU{Компилируем}{Let's compile} (MSVC 2010):

\lstinputlisting[caption=MSVC 2010]{patterns/02_stack/2_2_msvc.asm}

\index{Compiler intrinsic}
\IFRU {Единственный параметр в \TT{alloca()} передается через \EAX, а не как обычно через стек}
{The sole \TT{alloca()} argument passed via \EAX (instead of pushing into stack)}
\footnote{
\IFRU{Это потому что alloca() это не сколько функция, сколько т.е. \IT{compiler intrinsic} (\ref{compiler_intrinsic})}
{It is because alloca() is rather compiler intrinsic (\ref{compiler_intrinsic}) than usual function}.

\IFRU{Одна из причин, почему здесь нужна именно ф-ция а не несколько инструкций прямо в коде, в том что в реализации 
ф-ции alloca() от \ac{MSVC}
есть так же код читающий из только что выделенной памяти, чтобы \ac{OS} подключила физическую память к этому региону \ac{VM}.}
{One of the reason the is in the separated function instead of couple instructions just in the code,
because \ac{MSVC} implementation
of the alloca() function also has a code which reads from the memory just allocated, in order to let \ac{OS} to map
physical memory to this \ac{VM} region.}
}.
\IFRU{После вызова \TT{alloca()}, \ESP теперь указывает на блок в 600 байт который 
мы можем использовать под \TT{buf}.}
{After \TT{alloca()} call, \ESP is now pointing to the block of 600 bytes and we can 
use it as memory for \TT{buf} array.}

\paragraph{GCC + \IntelSyntax}

\IFRU{А GCC 4.4.1 обходится без вызова других функций:}
{GCC 4.4.1 can do the same without calling external functions:}

\lstinputlisting[caption=GCC 4.7.3]{patterns/02_stack/2_1_gcc_intel_O3_\IFRU{ru}{en}.asm}

\paragraph{GCC + \ATTSyntax}

\IFRU{Посмотрим на тот же код, только в синтаксисе AT\&T}{Let's see the same code, but in AT\&T syntax}:

\lstinputlisting[caption=GCC 4.7.3]{patterns/02_stack/2_1_gcc_ATT_O3.s}

\index{\ATTSyntax}
\IFRU{Всё то же самое что и в прошлом листинге.}{The same code as in previos listing.}

N.B. \IFRU{Например,}{E.g.} \TT{movl \$3, 20(\%esp)} 
\IFRU{это аналог}{is analogous to} \TT{mov DWORD PTR [esp+20], 3} \IFRU{в Intel-синтаксисе}
{in Intel-syntax}\IFRU{ ~--- при адресации памяти в виде}{~---when addressing memory in form} \IT{\IFRU{регистр+смещение}{register+offset}}, 
\IFRU{это записывается в AT\&T синтаксисе как}{it is written in AT\&T syntax as} 
\TT{\IFRU{смещение}{offset}(\%\IFRU{регистр}{register})}.


\subsection{(Windows) SEH}
\index{Windows!Structured Exception Handling}

\RU{В стеке хранятся записи \ac{SEH} для функции (если они присутствуют)}%
\EN{\ac{SEH} records are also stored on the stack (if they are present).}.

\ifx\LITE\undefined
\RU{Читайте больше о нем здесь}\EN{Read more about it}: (\myref{sec:SEH}).
\fi

\subsection{\RU{Защита от переполнений буфера}\EN{Buffer overflow protection}\PTBR{Proteção contra estouro de buffer}}

\RU{Здесь больше об этом}\EN{More about it here}\PTBR{Mais sobre aqui}~(\myref{subsec:bufferoverflow}).


