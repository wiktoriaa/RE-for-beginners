\chapter{\Stack}
\label{sec:stack}
\index{\Stack}

\RU{Стек в информатике~--- это одна из наиболее фундаментальных структур данных}%
\EN{The stack is one of the most fundamental data structures in computer science}%
\footnote{\href{http://go.yurichev.com/17119}{wikipedia.org/wiki/Call\_stack}}.

\RU{Технически это просто блок памяти в памяти процесса + регистр \ESP в x86 или \RSP в x64, либо \ac{SP} в ARM, который указывает где-то в пределах этого блока.}
\EN{Technically, it is just a block of memory in process memory along with the \ESP or \RSP register in x86 or x64, or the \ac{SP} register in ARM, as a pointer within that block.}

\index{ARM!\Instructions!PUSH}
\index{ARM!\Instructions!POP}
\index{x86!\Instructions!PUSH}
\index{x86!\Instructions!POP}
\RU{Часто используемые инструкции для работы со стеком~--- это \PUSH и \POP (в x86 и Thumb-режиме ARM). 
\PUSH уменьшает \ESP/\RSP/\ac{SP} на 4 в 32-битном режиме (или на 8 в 64-битном),
затем записывает по адресу, на который указывает \ESP/\RSP/\ac{SP}, содержимое своего единственного операнда.}
\EN{The most frequently used stack access instructions are \PUSH and \POP (in both x86 and ARM Thumb-mode). 
\PUSH subtracts from \ESP/\RSP/\ac{SP} 4 in 32-bit mode (or 8 in 64-bit mode) and then writes the contents of its sole operand to the memory address pointed by \ESP/\RSP/\ac{SP}.} 

\RU{\POP это обратная операция~--- сначала достает из \glslink{stack pointer}{указателя стека} значение и помещает его в операнд 
(который очень часто является регистром) и затем увеличивает указатель стека на 4 (или 8).}
\EN{\POP is the reverse operation: retrieve the data from the memory location that \ac{SP} points to, 
load it into the instruction operand (often a register) and then add 4 (or 8) to the \gls{stack pointer}.}

\RU{В самом начале \glslink{stack pointer}{регистр-указатель} указывает на конец стека.}
\EN{After stack allocation, the \gls{stack pointer} points at the bottom of the stack.}
\RU{\PUSH уменьшает \glslink{stack pointer}{регистр-указатель}, а \POP~--- увеличивает.}
\EN{\PUSH decreases the \gls{stack pointer} and \POP increases it.}
\RU{Конец стека находится в начале блока памяти, выделенного под стек. Это странно, но это так.}
\EN{The bottom of the stack is actually at the beginning of the memory allocated for the stack block. 
It seems strange, but that's the way it is.}

\ifdefined\IncludeARM
\RU{В процессоре ARM, тем не менее, есть поддержка стеков, растущих как в сторону уменьшения, так и в
сторону увеличения.}
\EN{ARM supports both descending and ascending stacks.}
\index{ARM!\Instructions!STMFD}
\index{ARM!\Instructions!LDMFD}
\index{ARM!\Instructions!STMED}
\index{ARM!\Instructions!LDMED}
\index{ARM!\Instructions!STMFA}
\index{ARM!\Instructions!LDMFA}
\index{ARM!\Instructions!STMEA}
\index{ARM!\Instructions!LDMEA}

\RU{Например, инструкции}\EN{For example the} 
\ac{STMFD}/\ac{LDMFD}, \ac{STMED}/\ac{LDMED} 
\RU{предназначены для descending-стека 
(растет назад, начиная с высоких адресов в сторону низких).}
\EN{instructions are intended to deal with a descending stack 
(grows downwards, starting with a high address and progressing to a lower one).}
\RU{Инструкции}\EN{The}
\ac{STMFA}/\ac{LDMFA}, \ac{STMEA}/\ac{LDMEA} 
\RU{предназначены для ascending-стека 
(растет вперед, начиная с низких адресов в сторону высоких).}
\EN{instructions are intended to deal with an ascending stack 
(grows upwards, starting from a low address and progressing to a higher one).}
\fi

% It might be worth mentioning that STMED and STMEA write first,
% and then move the pointer,
% and that LDMED and LDMEA move the pointer first, and then read.
% In other words, ARM not only lets the stack grow in a non-standard direction,
% but also in a non-standard order.
% Maybe this can be in the glossary, which would explain why E stands for "empty".

\section{\RU{Почему стек растет в обратную сторону?}\EN{Why does the stack grow backwards?}}

\RU{Интуитивно мы можем подумать, что, как и любая другая структура данных, стек мог бы расти вперед, 
т.е. в сторону увеличения адресов}\EN{Intuitively, we might think that the stack grows upwards, i.e. towards
higher addresses, like any other data structure}.

\RU{Причина, почему стек растет назад, вероятно, историческая}%
\EN{The reason that the stack grows backward is probably historical}.
\RU{Когда компьютеры были большие и занимали целую комнату, было очень легко разделить сегмент на две части:
для \glslink{heap}{кучи} и для стека}\EN{When the computers were big and occupied a whole room, 
it was easy to divide memory into two parts, one for the \gls{heap} and one for the stack}.
\RU{Заранее было неизвестно, насколько большой может быть \glslink{heap}{куча} или стек, 
так что это решение было самым простым}\EN{Of course, 
it was unknown how big the \gls{heap} and the stack would be during program execution, 
so this solution was the simplest possible}.

\begin{center}
	\begin{tikzpicture}
	\tikzstyle{every path}=[thick]

	\node [rectangle,draw,minimum width=6cm, minimum height=2cm] (memory) {};
	\node [] [right=0.2cm of memory.west] (heap) {Heap};
	\node [] [left=0.2cm of memory.east] (stack) {Stack};

	\node [] (center1) [right=2cm of memory.west] {};
	\node [] (center2) [left=2cm of memory.east] {};

	\draw [->] (heap) -- (center1);
	\draw [->] (stack) -- (center2);

	\node [] [above left=1.1cm and 0.2cm of heap] (t1) {\RU{Начало кучи}\EN{Start of heap}};
	\node [] [above right=1.1cm and 0.2cm of stack] (t2) {\RU{Вершина стека}\EN{Start of stack}};

	\draw [->] (t1) -- (memory.west);
	\draw [->] (t2) -- (memory.east);

	\end{tikzpicture}
\end{center}

\RU{В}\EN{In} \cite{Ritchie74} \RU{можно прочитать}\EN{we can read}:

\begin{framed}
\begin{quotation}
The user-core part of an image is divided into three logical segments. The program text segment begins at location 0 in the virtual address space. During execution, this segment is write-protected and a single copy of it is shared among all processes executing the same program. At the first 8K byte boundary above the program text segment in the virtual address space begins a nonshared, writable data segment, the size of which may be extended by a system call. Starting at the highest address in the virtual address space is a stack segment, which automatically grows downward as the hardware's stack pointer fluctuates.
\end{quotation}
\end{framed}

\RU{Это немного напоминает как некоторые студенты
пишут два конспекта в одной тетрадке:
первый конспект начинается обычным образом, второй пишется с конца, перевернув тетрадку.
Конспекты могут встретиться где-то посредине, в случае недостатка свободного места.}
\EN{This reminds us how some students write two lecture notes using only one notebook:
notes for the first lecture are written as usual, 
and notes for the second one are written from the end of notebook, by flipping it.
Notes may meet each other somewhere in between, in case of lack of free space.}
% I think if we want to expand on this analogy,
% one might remember that the line number increases as as you go down a page.
% So when you decrease the address when pushing to the stack, visually,
% the stack does grow upwards.
% Of course, the problem is that in most human languages,
% just as with computers,
% we write downwards, so this direction is what makes buffer overflows so messy.

\section{\RU{Для чего используется стек?}\EN{What is the stack used for?}}

% subsections
\EN{\subsubsection{Save the function's return address}

\myparagraph{x86}

\myindex{x86!\Instructions!CALL}
When calling another function with a \CALL instruction, the address of the point exactly after the \CALL instruction is saved 
to the stack and then an unconditional jump to the address in the \CALL operand is executed.

\myindex{x86!\Instructions!PUSH}
\myindex{x86!\Instructions!JMP}
The \CALL instruction is equivalent to a\\
\INS{PUSH address\_after\_call / JMP operand} instruction pair.

\myindex{x86!\Instructions!RET}
\myindex{x86!\Instructions!POP}
\RET fetches a value from the stack and jumps to it~---that is equivalent to a \TT{POP tmp / JMP tmp} instruction pair.

\myindex{\Stack!\MLStackOverflow}
\myindex{\Recursion}
Overflowing the stack is straightforward. Just run eternal recursion:

\begin{lstlisting}[style=customc]
void f()
{
	f();
};
\end{lstlisting}

MSVC 2008 reports the problem:

\begin{lstlisting}
c:\tmp6>cl ss.cpp /Fass.asm
Microsoft (R) 32-bit C/C++ Optimizing Compiler Version 15.00.21022.08 for 80x86
Copyright (C) Microsoft Corporation.  All rights reserved.

ss.cpp
c:\tmp6\ss.cpp(4) : warning C4717: 'f' : recursive on all control paths, function will cause runtime stack overflow
\end{lstlisting}

\dots but generates the right code anyway:

\begin{lstlisting}[style=customasmx86]
?f@@YAXXZ PROC			; f
; File c:\tmp6\ss.cpp
; Line 2
	push	ebp
	mov	ebp, esp
; Line 3
	call	?f@@YAXXZ	; f
; Line 4
	pop	ebp
	ret	0
?f@@YAXXZ ENDP			; f
\end{lstlisting}

\dots Also if we turn on the compiler optimization (\TT{\Ox} option) the optimized code will not overflow the stack 
and will work \IT{correctly}\footnote{irony here} instead:

\begin{lstlisting}[style=customasmx86]
?f@@YAXXZ PROC			; f
; File c:\tmp6\ss.cpp
; Line 2
$LL3@f:
; Line 3
	jmp	SHORT $LL3@f
?f@@YAXXZ ENDP			; f
\end{lstlisting}

GCC 4.4.1 generates similar code in both cases without, however,  issuing any warning about the problem.

\myparagraph{ARM}

\myindex{ARM!\Registers!Link Register}
ARM programs also use the stack for saving return addresses, but differently.
As mentioned in \q{\HelloWorldSectionName}~(\myref{sec:hw_ARM}),
the \ac{RA} is saved to the \ac{LR} (\gls{link register}).
If one needs, however, to call another function and use the \ac{LR} register
one more time, its value has to be saved.
\myindex{Function prologue}
Usually it is saved in the function prologue.

\myindex{ARM!\Instructions!PUSH}
\myindex{ARM!\Instructions!POP}
Often, we see instructions like \INS{PUSH {R4-R7,LR}} along with this instruction in epilogue
\INS{POP {R4-R7,PC}}---thus register values to be used in the function are saved in the stack, including \ac{LR}.

\myindex{ARM!Leaf function}
Nevertheless, if a function never calls any other function, in \ac{RISC} terminology it is called a
\IT{\gls{leaf function}}\footnote{\href{http://go.yurichev.com/17064}{infocenter.arm.com/help/index.jsp?topic=/com.arm.doc.faqs/ka13785.html}}. 
As a consequence, leaf functions do not save the \ac{LR} register (because they don't modify it).
If such function is small and uses a small number of registers, it may not use the stack at all.
Thus, it is possible to call leaf functions without using the stack,
which can be faster than on older x86 machines because external RAM is not used for the stack
\footnote{Some time ago, on PDP-11 and VAX, the CALL instruction (calling other functions) was expensive; up to 50\%
of execution time might be spent on it, so it was considered that having a big number of small functions is an \gls{anti-pattern} \InSqBrackets{\TAOUP Chapter 4, Part II}.}.
This can be also useful for situations when memory for the stack is not yet allocated or not available.

Some examples of leaf functions:
\myref{ARM_leaf_example1}, \myref{ARM_leaf_example2}, 
\myref{ARM_leaf_example3}, \myref{ARM_leaf_example4}, \myref{ARM_leaf_example5},
\myref{ARM_leaf_example6}, \myref{ARM_leaf_example7}, \myref{ARM_leaf_example10}.

}
\RU{\subsubsection{Сохранение адреса возврата управления}

\myparagraph{x86}

\myindex{x86!\Instructions!CALL}
При вызове другой функции через \CALL сначала в стек записывается адрес, указывающий на место после 
инструкции \CALL, затем делается безусловный переход (почти как \TT{JMP}) на адрес, указанный в операнде.

\myindex{x86!\Instructions!PUSH}
\myindex{x86!\Instructions!JMP}
\CALL~--- это аналог пары инструкций \INS{PUSH address\_after\_call / JMP}.

\myindex{x86!\Instructions!RET}
\myindex{x86!\Instructions!POP}
\RET вытаскивает из стека значение и передает управление по этому адресу~--- 
это аналог пары инструкций \TT{POP tmp / JMP tmp}.

\myindex{\Stack!\MLStackOverflow}
\myindex{\Recursion}
Крайне легко устроить переполнение стека, запустив бесконечную рекурсию:

\begin{lstlisting}[style=customc]
void f()
{
	f();
};
\end{lstlisting}

MSVC 2008 предупреждает о проблеме:

\begin{lstlisting}
c:\tmp6>cl ss.cpp /Fass.asm
Microsoft (R) 32-bit C/C++ Optimizing Compiler Version 15.00.21022.08 for 80x86
Copyright (C) Microsoft Corporation.  All rights reserved.

ss.cpp
c:\tmp6\ss.cpp(4) : warning C4717: 'f' : recursive on all control paths, function will cause runtime stack overflow
\end{lstlisting}

\dots но, тем не менее, создает нужный код:

\begin{lstlisting}[style=customasmx86]
?f@@YAXXZ PROC			; f
; File c:\tmp6\ss.cpp
; Line 2
	push	ebp
	mov	ebp, esp
; Line 3
	call	?f@@YAXXZ	; f
; Line 4
	pop	ebp
	ret	0
?f@@YAXXZ ENDP			; f
\end{lstlisting}

\dots причем, если включить оптимизацию (\TT{\Ox}), то будет даже интереснее, без переполнения стека, 
но работать будет \IT{корректно}\footnote{здесь ирония}:

\begin{lstlisting}[style=customasmx86]
?f@@YAXXZ PROC			; f
; File c:\tmp6\ss.cpp
; Line 2
$LL3@f:
; Line 3
	jmp	SHORT $LL3@f
?f@@YAXXZ ENDP			; f
\end{lstlisting}

GCC 4.4.1 генерирует точно такой же код в обоих случаях, хотя и не предупреждает о проблеме.

\myparagraph{ARM}

\myindex{ARM!\Registers!Link Register}
Программы для ARM также используют стек для сохранения \ac{RA}, куда нужно вернуться, но несколько иначе.
Как уже упоминалось в секции \q{\HelloWorldSectionName}~(\myref{sec:hw_ARM}),
\ac{RA} записывается в регистр \ac{LR} (\gls{link register}).
Но если есть необходимость вызывать какую-то другую функцию и использовать регистр \ac{LR} ещё
раз, его значение желательно сохранить.
\myindex{Function prologue}
\myindex{ARM!\Instructions!PUSH}
\myindex{ARM!\Instructions!POP}

Обычно это происходит в прологе функции, часто мы видим там инструкцию вроде \INS{PUSH \{R4-R7,LR\}}, а в эпилоге
\INS{POP \{R4-R7,PC\}}~--- так сохраняются регистры, которые будут использоваться в текущей функции, в том числе \ac{LR}.

\myindex{ARM!Leaf function}
Тем не менее, если некая функция не вызывает никаких более функций, в терминологии \ac{RISC} она называется
\IT{\gls{leaf function}}\footnote{\href{http://go.yurichev.com/17064}{infocenter.arm.com/help/index.jsp?topic=/com.arm.doc.faqs/ka13785.html}}. 
Как следствие, \q{leaf}-функция не сохраняет регистр \ac{LR} (потому что не изменяет его).
А если эта функция небольшая, использует мало регистров, она может не использовать стек вообще.
Таким образом, в ARM возможен вызов небольших leaf-функций не используя стек.
Это может быть быстрее чем в старых x86, ведь внешняя память для стека не используется
\footnote{Когда-то, очень давно, на PDP-11 и VAX на инструкцию CALL (вызов других функций) могло тратиться
вплоть до 50\% времени (возможно из-за работы с памятью),
поэтому считалось, что много небольших функций это \glslink{anti-pattern}{анти-паттерн}
\InSqBrackets{\TAOUP Chapter 4, Part II}.}.
Либо это может быть полезным для тех ситуаций, когда память для стека ещё не выделена, либо недоступна,

Некоторые примеры таких функций:
\myref{ARM_leaf_example1}, \myref{ARM_leaf_example2}, 
\myref{ARM_leaf_example3}, \myref{ARM_leaf_example4}, \myref{ARM_leaf_example5},
\myref{ARM_leaf_example6}, \myref{ARM_leaf_example7}, \myref{ARM_leaf_example10}.

}
\DE{\subsection{Rückgabe Adresse der Funktion speichern}

\myparagraph{x86}

\myindex{x86!\Instructions!CALL}
Wenn man eine Funktion mit der \CALL Instruktion aufruft, wird die Adresse direkt nach der
\CALL Instruktion auf dem Stack gespeichert und der unbedingte jump wird ausgeführt.

\myindex{x86!\Instructions!PUSH}
\myindex{x86!\Instructions!JMP}
Die \CALL Instruktion ist äquivalent zu dem \INS{PUSH address\_after\_call / JMP operand} Instruktions paar.

\myindex{x86!\Instructions!RET}
\myindex{x86!\Instructions!POP}
\RET ruft die Rückkehr Adresse vom Stack ab und springt zu dieser~---was äquivalent zu einem \TT{POP tmp / JMP tmp} Instruktions
paar ist.

\myindex{\Stack!\MLStackOverflow}
\myindex{\Recursion}

Den Stack zum überlaufen zu bringen ist recht einfach, einfach eine 
endlos rekursive Funktion Aufrufen:


\begin{lstlisting}[style=customc]
void f()
{
	f();
};
\end{lstlisting}


MSVC 2008 hat eine Erkennung für das Problem:


\begin{lstlisting}
c:\tmp6>cl ss.cpp /Fass.asm
Microsoft (R) 32-bit C/C++ Optimizing Compiler Version 15.00.21022.08 for 80x86
Copyright (C) Microsoft Corporation.  All rights reserved.

ss.cpp
c:\tmp6\ss.cpp(4) : warning C4717: 'f' : recursive on all control paths, function will cause runtime stack overflow
\end{lstlisting}

\dots aber der Compiler erzeugt den Code trotzdem:

\begin{lstlisting}[style=customasmx86]
?f@@YAXXZ PROC			; f
; File c:\tmp6\ss.cpp
; Line 2
	push	ebp
	mov	ebp, esp
; Line 3
	call	?f@@YAXXZ	; f
; Line 4
	pop	ebp
	ret	0
?f@@YAXXZ ENDP			; f
\end{lstlisting}

\dots Auch wenn wir die Compiler Optimierungen einschalten (\TT{/0x} Option) wird der optimierte Code nicht
den Stack zum überlaufen bringen. Stattdessen wird der Code \IT{korrekt}\footnote{Ironie hier} ausgeführt: 

\begin{lstlisting}[style=customasmx86]
?f@@YAXXZ PROC			; f
; File c:\tmp6\ss.cpp
; Line 2
$LL3@f:
; Line 3
	jmp	SHORT $LL3@f
?f@@YAXXZ ENDP			; f
\end{lstlisting}


GCC 4.4.1 generiert vergleichbaren Code in beiden Fällen, jedoch ohne über das Overflow Problem zu warnen.


\myparagraph{ARM}

\myindex{ARM!\Registers!Link Register}

ARM Programme benutzen den Stack um Rücksprung Adressen zu speichern, aber anders.
Wie bereits erwähnt in \q{\HelloWorldSectionName}~(\myref{sec:hw_ARM}),
wird der \ac{RA} Wert im \ac{LR} (\gls{link register}) gespeichert.
Wenn nun eine andere Funktion aufgerufen werden muss und auf das \ac{LR} Register 
zu greift, muss der aktuelle Wert im Register irgendwo gespeichert werden.

\myindex{Funktion Prologe}
Normal wird der Wert im Funktion Prolog gespeichert.

\myindex{ARM!\Instructions!PUSH}
\myindex{ARM!\Instructions!POP}

Oft sieht man Instruktionen wie z.B \INS{PUSH {R4-R7,LR}} zusammen mit dieser Instruktion im 
Epilog \INS{POP {R4-R7,PC}}---Somit werden Werte die in den Funktionen benötigt werden auf dem 
Stack gespeichert, inklusive \ac{LR}.

\myindex{ARM!Leaf Funktion}
Wenn eine Funktion nie eine andere Funktion aufruft, nennt man das in der \ac{RISC} Terminologie eine
\IT{\glslink{leaf function}{leaf Funktion}}\footnote{\href{http://go.yurichev.com/17064}{infocenter.arm.com/help/index.jsp?topic=/com.arm.doc.faqs/ka13785.html}}.  % <-- attention could be a compilier bug
Als Konsequenz ergibt sich, das leaf Funktionen nicht das \ac{LR} Register speichern (da sie es nicht modifizieren).
Wenn solche Funktionen klein sind und nur eine geringe Anzahl an Registern benutzt, ist es möglich das der Stack
gar nicht benutzt wird. Es ist also möglich leaf Funktionen zu benutzen ohne den Stack zurück zu greifen, die Ausführung
ist hier schneller als auf älteren x86 Maschinen weil kein externer RAM für den Stack benutzt wird 
\footnote{Bis vor einer weile war es sehr teuer auf PDP-11 und VAX Maschinen die CALL Instruktion zu benutzen; bis zu 50\%
der Rechenzeit wurde allein für diese Instruktion verschwendet, man hat dabei festgestellt das eine große Anzahl an kleinen
Funktionen zu haben ein \gls{anti-pattern} \InSqBrackets{\TAOUP Chapter 4, Part II}.} ist.
Diese Eigenschaft kann nützlich sein wenn der Speicher für den Stack noch nicht alloziert oder verfügbar ist.

Ein paar Beispiele für leaf Funktionen:

\myref{ARM_leaf_example1}, \myref{ARM_leaf_example2}, 
\myref{ARM_leaf_example3}, \myref{ARM_leaf_example4}, \myref{ARM_leaf_example5},
\myref{ARM_leaf_example6}, \myref{ARM_leaf_example7}, \myref{ARM_leaf_example10}.

}
\FR{\subsubsection{Sauvegarder l'adresse de retour de la fonction}

\myparagraph{x86}

\myindex{x86!\Instructions!CALL}
Lorsque l'on appelle une fonction avec une instruction \CALL, l'adresse du point
exactement après cette dernière est sauvegardée sur la pile et un saut inconditionnel
à l'adresse de l'opérande \CALL est exécuté.

\myindex{x86!\Instructions!PUSH}
\myindex{x86!\Instructions!JMP}
L'instruction \CALL est équivalente à la\\
paire d'instructions \INS{PUSH address\_after\_call / JMP operand}.

\myindex{x86!\Instructions!RET}
\myindex{x86!\Instructions!POP}
\RET va chercher une valeur sur la pile et y saute~---ce qui est équivalent à
la paire d'instructions \TT{POP tmp / JMP tmp}.

\myindex{\Stack!\MLStackOverflow}
\myindex{\Recursion}
Déborder de la pile est très facile. Il suffit de lancer une récursion éternelle:

\begin{lstlisting}[style=customc]
void f()
{
	f();
};
\end{lstlisting}

MSVC 2008 signale le problème:

\begin{lstlisting}
c:\tmp6>cl ss.cpp /Fass.asm
Microsoft (R) 32-bit C/C++ Optimizing Compiler Version 15.00.21022.08 for 80x86
Copyright (C) Microsoft Corporation.  All rights reserved.

ss.cpp
c:\tmp6\ss.cpp(4) : warning C4717: 'f' : recursive on all control paths, function will cause runtime stack overflow
\end{lstlisting}

\dots mais génère tout de même le code correspondant:

\begin{lstlisting}[style=customasmx86]
?f@@YAXXZ PROC			; f
; File c:\tmp6\ss.cpp
; Line 2
	push	ebp
	mov	ebp, esp
; Line 3
	call	?f@@YAXXZ	; f
; Line 4
	pop	ebp
	ret	0
?f@@YAXXZ ENDP			; f
\end{lstlisting}

\dots Si nous utilisons l'option d'optimisation du compilateur (option \TT{\Ox})
le code optimisé ne va pas déborder de la pile et au lieu de cela va fonctionner
\IT{correctemment}\footnote{ironique ici}:

\begin{lstlisting}[style=customasmx86]
?f@@YAXXZ PROC			; f
; File c:\tmp6\ss.cpp
; Line 2
$LL3@f:
; Line 3
	jmp	SHORT $LL3@f
?f@@YAXXZ ENDP			; f
\end{lstlisting}

GCC 4.4.1 génère un code similaire dans les deux cas, sans, toutefois émettre
d'avertissement à propos de ce problème.

\myparagraph{ARM}

\myindex{ARM!\Registers!Link Register}
Les programmes ARM utilisent également la pile pour sauver les adresses de retour,
mais différemment.
Comme mentionné dans \q{\HelloWorldSectionName}~(\myref{sec:hw_ARM}),
\ac{RA} est sauvegardé dans \ac{LR} (\gls{link register}).
Si l'on a toutefois besoin d'appeler une autre fonction et d'utiliser le registre
\ac{LR} une fois de plus, sa valeur doit être sauvegardée.
\myindex{Function prologue}
Usuellement, cela se fait dans le prologue de la fonction.

\myindex{ARM!\Instructions!PUSH}
\myindex{ARM!\Instructions!POP}
Souvent, nous voyons des instructions comme \INS{PUSH {R4-R7,LR}} en même temps
que cette instruction dans le prologue \INS{POP {R4-R7,PC}}---ces registres qui
sont utilisés dans la fonction sont sauvegardés sur la pile, \ac{LR} inclus.

\myindex{ARM!Fonction leaf} % FIXME traduire avec feuille ?
Néanmoins, si une fonction n'appelle jamais d'autre fonction, dans la terminologie
\ac{RISC} elle est appelée \IT{\glslink{leaf function}{fonction leaf}}\footnote{\href{http://go.yurichev.com/17064}{infocenter.arm.com/help/index.jsp?topic=/com.arm.doc.faqs/ka13785.html}}.
Ceci a comme conséquence que les fonctions leaf ne sauvegardent pas le registre
\ac{LR} (car elles ne le modifient pas).
Si une telle fonction est petite et utilise un petit nombre de registres, elle
peut ne pas utiliser du tout la pile.
Ainsi, il est possible d'appeler des fonctions leaf sans utiliser la pile.
Ce qui peut être plus rapide sur des vieilles machines x86 car la mémoire externe
n'est pas utilisée pour la pile
\footnote{Il y a quelques temps, sur PDP-11 et VAX, l'instruction CALL (appel d'autres fonctions) était coûteux; jusqu'à 50\%
du temps d'exécution pouvait être passé à ça, il était donc considèré qu'avoir un grand nombre de petite fonction était un \gls{anti-pattern} \InSqBrackets{\TAOUP Chapter 4, Part II}.}.
Cela peut être utile pour des situations où la mémoire pour la pile n'est pas
encore allouée ou disponible.

Quelques exemples de fonctions leaf:
\myref{ARM_leaf_example1}, \myref{ARM_leaf_example2},
\myref{ARM_leaf_example3}, \myref{ARM_leaf_example4}, \myref{ARM_leaf_example5},
\myref{ARM_leaf_example6}, \myref{ARM_leaf_example7}, \myref{ARM_leaf_example10}.

}
\PTBR{\subsubsection{Salvar o endereço de retorno de uma função}

\myparagraph{x86}

\myindex{x86!\Instructions!CALL}
Quando você chama outra função utilizando a instrução CALL, o endereço do ponto exato onde a 
instrução \CALL se encontra é salvo na pilha e então um jump incondicional para o endereço no operando de \CALL é executado.

\myindex{x86!\Instructions!PUSH}
\myindex{x86!\Instructions!JMP}
A instrução \CALL é equivalente a usar o par de instruções \TT{PUSH endereço\_depois\_chamada / JMP}.

\myindex{x86!\Instructions!RET}
\myindex{x86!\Instructions!POP}
\RET pega um valor da pilha e usa um jump para ele --- isso é equivalente a usar \INS{POP tmp / JMP tmp}.

\myindex{\Stack!\MLStackOverflow}
\myindex{\Recursion}
Estourar uma stack é fácil. Só execute alguma recursão externa:

\begin{lstlisting}[style=customc]
void f()
{
	f();
};
\end{lstlisting}

O compilador MSVC 2008 informa o problema:

\begin{lstlisting}
c:\tmp6>cl ss.cpp /Fass.asm
Microsoft (R) 32-bit C/C++ Optimizing Compiler Version 15.00.21022.08 for 80x86
Copyright (C) Microsoft Corporation.  All rights reserved.

ss.cpp
c:\tmp6\ss.cpp(4) : warning C4717: 'f' : recursive on all control paths, function will cause runtime stack overflow
\end{lstlisting}

\dots mas gera o código de qualquer maneira:

\begin{lstlisting}[style=customasmx86]
?f@@YAXXZ PROC			; f
; File c:\tmp6\ss.cpp
; Line 2
	push	ebp
	mov	ebp, esp
; Line 3
	call	?f@@YAXXZ	; f
; Line 4
	pop	ebp
	ret	0
?f@@YAXXZ ENDP			; f
\end{lstlisting}

\dots também, se ativarmos a otimização do compilador (opção \TT{/Ox}) 
o código otimizado não vai estourar a pilha e funcionará \IT{corretamente} \footnote{ironia aqui}:

\begin{lstlisting}[style=customasmx86]
?f@@YAXXZ PROC			; f
; File c:\tmp6\ss.cpp
; Line 2
$LL3@f:
; Line 3
	jmp	SHORT $LL3@f
?f@@YAXXZ ENDP			; f
\end{lstlisting}

\PTBRph{}

}
\ITA{\subsubsection{Salvare l'indirizzo di ritorno della funzione}

\myparagraph{x86}

\myindex{x86!\Instructions!CALL}
Quando si chiama una funzione con l'istruzione \CALL, l'indirizzo del punto esattamente dopo la \CALL viene salvato nello stack, e successivamente
viene eseguito un jump non condizionale all'indirizzo dell'operando di \CALL.

\myindex{x86!\Instructions!PUSH}
\myindex{x86!\Instructions!JMP}
L'istruzione \CALL e' equivalente alla coppia di istruzioni \INS{PUSH indirizzo\_dopo\_call / JMP operando}.

\myindex{x86!\Instructions!RET}
\myindex{x86!\Instructions!POP}
\RET preleva un valore dallo stack e effettua un jump ad esso~--- cio' equivale alla coppia di istruzioni \TT{POP tmp / JMP tmp}.

\myindex{\Stack!\MLStackOverflow}
\myindex{\Recursion}

Riempire lo stack fino allo straripamento e' semplicissimo. Basta ricorrere alla ricorsione eterna:

\begin{lstlisting}[style=customc]
void f()
{
	f();
};
\end{lstlisting}

MSVC 2008 riporta il problema:

\begin{lstlisting}
c:\tmp6>cl ss.cpp /Fass.asm
Microsoft (R) 32-bit C/C++ Optimizing Compiler Version 15.00.21022.08 for 80x86
Copyright (C) Microsoft Corporation.  All rights reserved.

ss.cpp
c:\tmp6\ss.cpp(4) : warning C4717: 'f' : recursive on all control paths, function will cause runtime stack overflow
\end{lstlisting}

\dots ma genera in ogni caso il codice correttamente:

\begin{lstlisting}[style=customasmx86]
?f@@YAXXZ PROC			; f
; File c:\tmp6\ss.cpp
; Line 2
	push	ebp
	mov	ebp, esp
; Line 3
	call	?f@@YAXXZ	; f
; Line 4
	pop	ebp
	ret	0
?f@@YAXXZ ENDP			; f
\end{lstlisting}

\dots Se attiviamo le ottimizzazioni del compilatore (\TT{\Ox} option) il codice ottimizzato non causera' overflow dello stack 
e funzionera' invece \IT{correttamente}\footnote{sarcasmo, si fa per dire}:

\begin{lstlisting}[style=customasmx86]
?f@@YAXXZ PROC			; f
; File c:\tmp6\ss.cpp
; Line 2
$LL3@f:
; Line 3
	jmp	SHORT $LL3@f
?f@@YAXXZ ENDP			; f
\end{lstlisting}

GCC 4.4.1 genera codice simile in antrambi i casi, senza avvertire del problema.

\myparagraph{ARM}

\myindex{ARM!\Registers!Link Register}
Anche i programmi ARM usano lo stack per salvare gli indirizzi di ritorno, ma lo fanno in maniera diversa.
Come detto in \q{\HelloWorldSectionName}~(\myref{sec:hw_ARM}),
As mentioned in 
il \ac{RA} viene salvato nel \ac{LR} (\gls{link register}).
Se si presenta comunque la necessita' di chiamare un'altra funzione ed usare il registro \ac{LR} ancora una volta, 
il suo valore deve essere salvato.
\myindex{Function prologue}
Solitamente questo valore e' slvato nel preambolo della funzione.

\myindex{ARM!\Instructions!PUSH}
\myindex{ARM!\Instructions!POP}
Spesso vediamo istruzioni come \INS{PUSH {R4-R7,LR}} insieme ad isrtuzioni nell'epilogo come 
\INS{POP {R4-R7,PC}}---percio' i valori dei registri che saranno usati nella funzione vengono salvati nello stack, incluso \ac{LR}.

\myindex{ARM!Leaf function}
Ciononostante, se una funzione non chiama al suo interno nessun'altra funzione, in terminologia \ac{RISC} e' detta 
\IT{\gls{leaf function}}, o funzione foglia.\footnote{\href{http://go.yurichev.com/17064}{infocenter.arm.com/help/index.jsp?topic=/com.arm.doc.faqs/ka13785.html}}. 
Di conseguenza, le leaf functions non salvano il registro \ac{LR} register (perche' difatti non lo modificano).
Se una simile funzione e' molto breve e usa un piccolo numero di registri, potrebbe non usare del tutto lo stack. 
E' quindi possible chiamare le leaf functions senza usare lo stack, cosa che puo' essere piu' veloce che sulle macchine x86 perche' ;a RA< esterna non viene usata per lo stack
\footnote{Tempo fa, su PDP-11 and VAX, l'istruzione CALL instruction (chiamare altre funzioni) era costosa; poteva richiedere fino al 50\%
del tempo di esecuzione, ed era quindi consuetudine pensare che avere un grande numero di piccole funzioni fosee un \gls{anti-pattern} \InSqBrackets{\TAOUP Chapter 4, Part II}.}.
Lo stesso principio puo' tornare utile quando la memoria per lo stack non e' stata ancora allocata o non e' disponibile.

Alcuni esempi di funzioni foglia:
\myref{ARM_leaf_example1}, \myref{ARM_leaf_example2}, 
\myref{ARM_leaf_example3}, \myref{ARM_leaf_example4}, \myref{ARM_leaf_example5},
\myref{ARM_leaf_example6}, \myref{ARM_leaf_example7}, \myref{ARM_leaf_example10}.
}

\subsection{\RU{Передача параметров функции}\EN{Passing function arguments}}

\RU{Самый распространенный способ передачи параметров в x86 называется}
\EN{The most popular way to pass parameters in x86 is called} \q{cdecl}:

\begin{lstlisting}
push arg3
push arg2
push arg1
call f
add esp, 12 ; 4*3=12
\end{lstlisting}

\RU{Вызываемая функция получает свои параметры также через указатель стека.}
\EN{\Gls{callee} functions get their arguments via the stack pointer.}

\RU{Следовательно, так расположены значения в стеке перед исполнением самой первой инструкции
функции \ttf{}:}
\EN{Therefore, this is how the argument values are located in the stack before the execution
of the \ttf{} function's very first instruction:}

\begin{center}
\begin{tabular}{ | l | l | }
\hline
ESP & \RU{адрес возврата}\EN{return address} \\
\hline
ESP+4 & \argument \#1, \MarkedInIDAAs{} \TT{arg\_0} \\
\hline
ESP+8 & \argument \#2, \MarkedInIDAAs{} \TT{arg\_4} \\
\hline
ESP+0xC & \argument \#3, \MarkedInIDAAs{} \TT{arg\_8} \\
\hline
\dots & \dots \\
\hline
\end{tabular}
\end{center}

\ifx\LITE\undefined
\RU{См. также в соответствующем разделе о других способах передачи аргументов через стек}
\EN{For more information on other calling conventions see also section}~(\myref{sec:callingconventions}).
\fi
\RU{Важно отметить, что, в общем, никто не заставляет программистов передавать параметры именно через стек,
это не является требованием к исполняемому коду.}
\EN{It is worth noting that nothing obliges programmers to pass arguments through the stack. It is not a requirement.}
\RU{Вы можете делать это совершенно иначе, не используя стек вообще.}
\EN{One could implement any other method without using the stack at all.}

\RU{К примеру, можно выделять в \glslink{heap}{куче} место для аргументов, 
заполнять их и передавать в функцию указатель на это место через \EAX. И это вполне будет работать}%
\EN{For example, it is possible to allocate a space for arguments in the \gls{heap}, fill it and pass it to a function 
via a pointer to this block in the \EAX register. This will work}%
\footnote{\RU{Например, в книге Дональда Кнута \q{Искусство программирования}, в разделе 1.4.1 
посвященном подпрограммам \cite[раздел 1.4.1]{Knuth:1998:ACP:521463}, 
мы можем прочитать о возможности располагать параметры для вызываемой подпрограммы после инструкции \JMP,
передающей управление подпрограмме. Кнут описывает, что это было особенно удобно для компьютеров IBM System/360.}%
\EN{For example, in the \q{The Art of Computer Programming} book by Donald Knuth, 
in section 1.4.1 dedicated to subroutines \cite[section 1.4.1]{Knuth:1998:ACP:521463},
we could read that one way to supply arguments to a subroutine is simply to list them after the \JMP instruction
passing control to subroutine. Knuth explains that this method was particularly convenient on IBM System/360.}}.
\RU{Однако традиционно сложилось, что в x86 и ARM передача аргументов происходит именно через стек.}
% I am unsure about what this comment means.
% My guess is that the arguments are put in the memory position after
% the jump instruction, so you could say:
% "one way to supply arguments to a subroutine is simply to list them in memory
% after the \JMP instruction that passes control to the subroutine."
% Right now, "after" also sounds like it refers to the time after
% the jump happens, which I think is too late.
\EN{However, it is a convenient custom in x86 and ARM to use the stack for this purpose.} \\
\\
\RU{Кстати, вызываемая функция не имеет информации о количестве переданных ей аргументов.}
\EN{By the way, the \gls{callee} function does not have any information about how many arguments were passed.}
\RU{Функции Си с переменным количеством аргументов (как \printf) определяют их количество по 
спецификаторам строки формата (начинающиеся со знака \%).}
\EN{C functions with a variable number of arguments (like \printf) determine their number using format string  specifiers (which begin with the \% symbol).}
\RU{Если написать что-то вроде}\EN{If we write something like} 

\begin{lstlisting}
printf("%d %d %d", 1234);
\end{lstlisting}

\printf \RU{выведет 1234, затем ещё два случайных числа, которые волею случая оказались в стеке рядом.}
\EN{will print 1234, and then two random numbers, which were lying next to it in the stack.}\\
\\
\RU{Вот почему не так уж и важно, как объявлять функцию \main}
\EN{That's why it is not very important how we declare the \main function}: \RU{как}\EN{as} \main, 
\TT{main(int argc, char *argv[])} 
\RU{либо}\EN{or} \TT{main(int argc, char *argv[], char *envp[])}.

\RU{В реальности, \ac{CRT}-код вызывает \main примерно так:}
\EN{In fact, the \ac{CRT}-code is calling \main roughly as:}

\begin{lstlisting}
push envp
push argv
push argc
call main
...
\end{lstlisting}

\RU{Если вы объявляете \main без аргументов, они, тем не менее, присутствуют в стеке, но не используются.}
\EN{If you declare \main as \main without arguments, they are, nevertheless, still present in the stack, but
are not used.}
\RU{Если вы объявите \main как}\EN{If you declare \main as} \TT{main(int argc, char *argv[])}, 
\RU{вы можете использовать два первых аргумента, а третий останется для вашей функции \q{невидимым}.}
\EN{you will be able to use first two arguments, and the third will remain \q{invisible} for your function.}
\RU{Более того, можно даже объявить}\EN{Even more, it is possible to declare} \TT{main(int argc)}, 
\RU{и это будет работать}\EN{and it will work}.


\subsection{\RU{Хранение локальных переменных}\EN{Local variable storage}}

\RU{Функция может выделить для себя некоторое место в стеке для локальных переменных, просто отодвинув 
\glslink{stack pointer}{указатель стека} глубже к концу стека.}
\EN{A function could allocate space in the stack for its local variables just by decreasing 
the \gls{stack pointer} towards the stack bottom.}
% I think here, "stack bottom" means the lowest address in the stack space,
% but the reader might also think it means towards the top of the stack space,
% like in a pop, so you might change "towards the stack bottom" to
% "towards the lowest address of the stack", or just take it out,
% since "decreasing" also suggests that.
\RU{Это очень быстро вне зависимости от количества локальных переменных.}
\EN{Hence, it's very fast, no matter how many local variables are defined.}

\RU{Хранить локальные переменные в стеке не является необходимым требованием. 
Вы можете хранить локальные переменные где угодно. 
Но по традиции всё сложилось так.}
\EN{It is also not a requirement to store local variables in the stack.
You could store local variables wherever you like, 
but traditionally this is how it's done.}

\EN{\subsubsection{x86: alloca() function}
\label{alloca}
\myindex{\CStandardLibrary!alloca()}

\newcommand{\AllocaSrcPath}{C:\textbackslash{}Program Files (x86)\textbackslash{}Microsoft Visual Studio 10.0\textbackslash{}VC\textbackslash{}crt\textbackslash{}src\textbackslash{}intel}

It is worth noting the \TT{alloca()} function
\footnote{In MSVC, the function implementation can be found in \TT{alloca16.asm} and \TT{chkstk.asm} in \\
\TT{\AllocaSrcPath{}}}.
This function works like \TT{malloc()}, but allocates memory directly on the stack.
% page break added to prevent "\vref on page boundary" error. it may be dropped in future.
The allocated memory chunk does not have to be freed via a \TT{free()} function call, \\
since the function epilogue (\myref{sec:prologepilog}) returns \ESP back to its initial state and 
the allocated memory is just \IT{dropped}.
It is worth noting how \TT{alloca()} is implemented.
In simple terms, this function just shifts \ESP downwards toward the stack bottom by the number of bytes you need and sets \ESP as a pointer to the \IT{allocated} block.

Let's try:

\lstinputlisting[style=customc]{patterns/02_stack/04_alloca/2_1.c}

\TT{\_snprintf()} function works just like \printf, but instead of dumping the result into \gls{stdout} (e.g., to terminal or 
console), it writes it to the \TT{buf} buffer. Function \puts copies the contents of \TT{buf} to \gls{stdout}. Of course, these two
function calls might be replaced by one \printf call, but we have to illustrate small buffer usage.

\myparagraph{MSVC}

Let's compile (MSVC 2010):

\lstinputlisting[caption=MSVC 2010,style=customasmx86]{patterns/02_stack/04_alloca/2_2_msvc.asm}

\myindex{Compiler intrinsic}
The sole \TT{alloca()} argument is passed via \EAX (instead of pushing it into the stack)
\footnote{It is because alloca() is rather a compiler intrinsic (\myref{sec:compiler_intrinsic}) than a normal function.
One of the reasons we need a separate function instead of just a couple of instructions in the code,
is because the \ac{MSVC} alloca() implementation also has code which reads from the memory just allocated, in order to let the \ac{OS} map
physical memory to this \ac{VM} region.
After the \TT{alloca()} call, \ESP points to the block of 600 bytes and we can use it as memory for the \TT{buf} array.}.

\myparagraph{GCC + \IntelSyntax}

GCC 4.4.1 does the same without calling external functions:

\lstinputlisting[caption=GCC 4.7.3,style=customasmx86]{patterns/02_stack/04_alloca/2_1_gcc_intel_O3_EN.asm}

\myparagraph{GCC + \ATTSyntax}

Let's see the same code, but in AT\&T syntax:

\lstinputlisting[caption=GCC 4.7.3,style=customasmx86]{patterns/02_stack/04_alloca/2_1_gcc_ATT_O3.s}

\myindex{\ATTSyntax}
The code is the same as in the previous listing.

By the way, \INS{movl \$3, 20(\%esp)} corresponds to
\INS{mov DWORD PTR [esp+20], 3} in Intel-syntax.
In the AT\&T syntax, the register+offset format of addressing memory looks like
\TT{offset(\%{register})}.

}
\FR{\subsubsection{x86: alloca() function}
\label{alloca}
\myindex{\CStandardLibrary!alloca()}

\newcommand{\AllocaSrcPath}{C:\textbackslash{}Program Files (x86)\textbackslash{}Microsoft Visual Studio 10.0\textbackslash{}VC\textbackslash{}crt\textbackslash{}src\textbackslash{}intel}

Intéressons-nous à la fonction \TT{alloca()}
\footnote{Avec MSVC, l'implémentation de cette fonction peut être trouvée dans les fichiers \TT{alloca16.asm} et \TT{chkstk.asm} dans \\
\TT{\AllocaSrcPath{}}}

Cette fonction fonctionne comme \TT{malloc()}, mais alloue de la mémoire directement sur la pile.
% page break added to prevent "\vref on page boundary" error. it may be dropped in future.
Le bout de mémoire ne doit pas être libéré via un appel à la fonction \TT{free()}, \\
puisque l'épilogue de fonction (\myref{sec:prologepilog}) retourne \ESP à son état initial précédant ce qui va automatiquement désallouer ce bout de mémoire.

Intéressons-nous à l'implémentation d'\TT{alloca()}.
Cette fonction décale simplement \ESP du nombre d'octets demandé vers le bas et vers le fond de la pile et définit \ESP en tant que pointeur vers la mémoire \IT{allouée}.

Essayons :

\lstinputlisting[style=customc]{patterns/02_stack/04_alloca/2_1.c}

La fonction \TT{\_snprintf()} fonctionne comme \printf, mais au lieu d'afficher le résultat sur \gls{stdout} (ex., dans un terminal ou une console), il l'écrit dans le buffer \TT{buf}. La fonction \puts copie le contenu de \TT{buf} dans \gls{stdout}. Évidemment, ces deux appels de fonctions peuvent être remplacés par un seul appel à la fonction \printf, mais nous devons illustrer l'utilisation de petit buffer.

\myparagraph{MSVC}

Compilons (MSVC 2010) :

\lstinputlisting[caption=MSVC 2010,style=customasmx86]{patterns/02_stack/04_alloca/2_2_msvc.asm}

\myindex{Compiler intrinsic}
Le seul argument d'\TT{alloca()} est passé via \EAX (au lieu de le mettre sur la pile )
\footnote{C'est parce que alloca() est plutôt une fonctionnalité intrinsèque du compilateur (\myref{sec:compiler_intrinsic}) qu'une fonction normale. Une des raisons pour lequelle nous avons besoin d'une fonction séparée au lieu de quelques instructions dans le code, est parce que l'implementation d'alloca() par \ac{MSVC} à également du code qui lit depuis la mémoire récemment allouée pour laisser l'\ac{OS} mapper la memoire physique vers la \ac{mémoire virtuelle}. Aprés l'appel à la fonction \TT{alloca()}, \ESP pointe sur un bloc de 600 octets que nous pouvons utiliser pour le tableau \TT{buf}.}

\myparagraph{GCC + \IntelSyntax}

GCC 4.4.1 fait la même chose sans effectuer d'appel à des fonctions externes :

\lstinputlisting[caption=GCC 4.7.3,style=customasmx86]{patterns/02_stack/04_alloca/2_1_gcc_intel_O3_FR.asm}

\myparagraph{GCC + \ATTSyntax}

Voyons le même code mais avec la syntaxe AT\&T :

\lstinputlisting[caption=GCC 4.7.3,style=customasmx86]{patterns/02_stack/04_alloca/2_1_gcc_ATT_O3.s}

\myindex{\ATTSyntax}
Le code est le même que le précédent.

Au fait, \INS{movl \$3, 20(\%esp)} correspond à
\INS{mov DWORD PTR [esp+20], 3} avec la syntaxe intel.
Dans la syntaxe AT\&T, le format registre+offset pour l'adressage mémoire
ressemble à \TT{offset(\%{register})}.
}
\RU{\subsubsection{x86: Функция alloca()}
\label{alloca}
\myindex{\CStandardLibrary!alloca()}

\newcommand{\AllocaSrcPath}{C:\textbackslash{}Program Files (x86)\textbackslash{}Microsoft Visual Studio 10.0\textbackslash{}VC\textbackslash{}crt\textbackslash{}src\textbackslash{}intel}

Интересен случай с функцией \TT{alloca()}
\footnote{В MSVC, реализацию функции можно посмотреть в файлах \TT{alloca16.asm} и \TT{chkstk.asm} в \\
\TT{\AllocaSrcPath{}}}. 
Эта функция работает как \TT{malloc()}, но выделяет память прямо в стеке.
Память освобождать через \TT{free()} не нужно, так как эпилог функции~(\myref{sec:prologepilog})
вернет \ESP в изначальное состояние и выделенная память просто \IT{выкидывается}.
Интересна реализация функции \TT{alloca()}.
Эта функция, если упрощенно, просто сдвигает \ESP вглубь стека на столько байт, сколько вам нужно и возвращает \ESP в качестве указателя на выделенный блок.

Попробуем:

\lstinputlisting[style=customc]{patterns/02_stack/04_alloca/2_1.c}

Функция \TT{\_snprintf()} работает так же, как и \printf, только вместо выдачи результата в \gls{stdout} (т.е. на терминал или в консоль),
записывает его в буфер \TT{buf}. Функция \puts выдает содержимое буфера \TT{buf} в \gls{stdout}. Конечно, можно было бы
заменить оба этих вызова на один \printf, но здесь нужно проиллюстрировать использование небольшого буфера.

\myparagraph{MSVC}

Компилируем (MSVC 2010):

\lstinputlisting[caption=MSVC 2010,style=customasmx86]{patterns/02_stack/04_alloca/2_2_msvc.asm}

\myindex{Compiler intrinsic}
Единственный параметр в \TT{alloca()} передается через \EAX, а не как обычно через стек
\footnote{Это потому, что alloca()~--- это не сколько функция, сколько т.н. \IT{compiler intrinsic} (\myref{sec:compiler_intrinsic})
Одна из причин, почему здесь нужна именно функция, а не несколько инструкций прямо в коде в том, что в реализации 
функции alloca() от \ac{MSVC}
есть также код, читающий из только что выделенной памяти, чтобы \ac{OS} подключила физическую память к этому региону \ac{VM}.
После вызова \TT{alloca()} \ESP указывает на блок в 600 байт, который мы можем использовать под \TT{buf}.}.

\myparagraph{GCC + \IntelSyntax}

А GCC 4.4.1 обходится без вызова других функций:

\lstinputlisting[caption=GCC 4.7.3,style=customasmx86]{patterns/02_stack/04_alloca/2_1_gcc_intel_O3_RU.asm}

\myparagraph{GCC + \ATTSyntax}

Посмотрим на тот же код, только в синтаксисе AT\&T:

\lstinputlisting[caption=GCC 4.7.3,style=customasmx86]{patterns/02_stack/04_alloca/2_1_gcc_ATT_O3.s}

\myindex{\ATTSyntax}
Всё то же самое, что и в прошлом листинге.

Кстати, \INS{movl \$3, 20(\%esp)}~--- это аналог \INS{mov DWORD PTR [esp+20], 3} в синтаксисе Intel.
Адресация памяти в виде \IT{регистр+смещение} записывается в синтаксисе AT\&T как \TT{смещение(\%{регистр})}.

}
\PTBR{\subsubsection{x86: a função alloca()}
\label{alloca}
\myindex{\CStandardLibrary!alloca()}

\newcommand{\AllocaSrcPath}{C:\textbackslash{}Program Files (x86)\textbackslash{}Microsoft Visual Studio 10.0\textbackslash{}VC\textbackslash{}crt\textbackslash{}src\textbackslash{}intel}

A função \TT{alloca()}
\footnote{No MSVC, a implementação da função pode ser encontrada nos arquivos \TT{alloca16.asm} e \TT{chkstk.asm} em \\
\TT{\AllocaSrcPath{}}}
funciona da mesma maneira que \TT{malloc()}, mas aloca memória diretamente na pilha.
O bloco de memória alocado não precisa ser limpo através da chamada da função free(),
desde que o rodapé da função (\myref{sec:prologepilog}) retorna \ESP de volta para seu estado inicial e a memória alocada é simplesmente desassociada.
Sobre como a função \TT{alloca()} é implementada, em termos simples, essa função só desloca \ESP para baixo 
(em direção ao fundo da pilha) pelo número de bytes que você precisa e define o ESP como um ponteiro para o bloco alocado.

\RU{Попробуем:}\EN{Let's try:}\PTBR{Vamos tentar:}

\lstinputlisting[style=customc]{patterns/02_stack/04_alloca/2_1.c}

A função \TT{\_snprintf()} funciona exatamente como \printf, mas ao invés de jogar o resultado em stdout
(terminal ou console, por exemplo), ela escreve no buffer buf.
A função \puts copia o conteúdo para um buf do stdout.
Lógico, essas duas chamadas de funções podem ser substituídas por um \printf, mas nós temos que ilustrar o uso pequeno do buffer.

\myparagraph{MSVC}

Vamos compilar (MSVC 2010):

\lstinputlisting[caption=MSVC 2010,style=customasmx86]{patterns/02_stack/04_alloca/2_2_msvc.asm}

\myindex{Compiler intrinsic}
O único argumento da função alloca() é passado via EAX (ao invés de ser empurrado na pilha)
\footnote{Isso é devido ao fato de que alloca() é mais nativa do compilador do que uma função normal (\myref{sec:compiler_intrinsic}).
Um dos motivos que se faz necessário o separamento da função ao invés de um pouco de linhas de código no código,
é porque a implementação da alloca() no MSVC também tem código que é lido da memória que acabou de ser alocada,
para deixar o sistema operacional mapear a memória física para essa região da memória virtual.}.

Depois da chamada de \TT{alloca()}, \ESP aponta para o bloco de 600 bytes que nós podemos usar como memória para o array.

\myparagraph{GCC + \IntelSyntax}

\PTBRph{}

}
\ITA{\subsubsection{x86: la funzione alloca() }
\label{alloca}
\myindex{\CStandardLibrary!alloca()}

\newcommand{\AllocaSrcPath}{C:\textbackslash{}Program Files (x86)\textbackslash{}Microsoft Visual Studio 10.0\textbackslash{}VC\textbackslash{}crt\textbackslash{}src\textbackslash{}intel}

Vale la pena esaminare la funzione \TT{alloca()}
\footnote{In MSVC, l'implementazione della funzione si trova in \TT{alloca16.asm} e \TT{chkstk.asm} in \\
\TT{\AllocaSrcPath{}}}.
Questa funzione opera come \TT{malloc()}, ma alloca memoria direttamente nello stack.
% page break added to prevent "\vref on page boundary" error. it may be dropped in future.
Il pezzo di memoria allocato non necessita di essere liberato tramite una chiamata alla funzione \TT{free()} function call, \\
poiche' l'epilogo della funzione (\myref{sec:prologepilog}) ripristina \ESP al suo valore iniziale e la memoria allocata viene semplicemente \IT{abbandonata}.
Vale anche la pena notare come e' implementata la funzione \TT{alloca()}.
In termini semplici, questa funzione shifta \ESP in basso, verso la base dello stack, per il numero di byte necessari e setta \ESP  
per puntare al blocco \IT{allocato}.

Proviamo:

\lstinputlisting[style=customc]{patterns/02_stack/04_alloca/2_1.c}

La funzione \TT{\_snprintf()} opera come \printf, ma invece di inviare il risultato a \gls{stdout} (es. al terminale o console),
lo scrive nel buffer \TT{buf}. La funzione \puts copia il contenuto di \TT{buf} in \gls{stdout}.
Ovviamente questo due chiamate potrebbero essere rimpiazzate da una sola chiamata a \printf, ma in questo caso era necessario per illustrare
l'uso di un piccolo buffer.

\myparagraph{MSVC}

Compiliamo (MSVC 2010):

\lstinputlisting[caption=MSVC 2010,style=customasmx86]{patterns/02_stack/04_alloca/2_2_msvc.asm}

\myindex{Compiler intrinsic}
L'unico argomento di \TT{alloca()} e' passato tramite il registro \EAX (anziche' metterlo nello stack)
\footnote{Questo perche' alloca() e' una "compiler intrinsic" (\myref{sec:compiler_intrinsic}) piuttosto che una funzione normale.
Una delle ragioni per cui abbiamo bisogno di una funzione separata, invece di un paio di istruzioni nel codice, e' che
l'implementazione di alloca() di \ac{MSVC} ha anche del codice che legge dalla memoria appena llocata, per far si che l'\ac{OS} effettui il mapping
della memoria fisica in questa regione della \ac{VM}.
Dopo la chiamata a \TT{alloca()} , \ESP punta al blocco di 600 byte, ed e' possibile utilizzarlo come memoria per l'array \TT{buf}.}.

\myparagraph{GCC + \IntelSyntax}

GCC 4.4.1 fa lo stesso senza chiamare funzioni esterne:

\lstinputlisting[caption=GCC 4.7.3,style=customasmx86]{patterns/02_stack/04_alloca/2_1_gcc_intel_O3_EN.asm}

\myparagraph{GCC + \ATTSyntax}

Esaminiamo lo stesso codice, ma in sintassi AT\&T:

\lstinputlisting[caption=GCC 4.7.3,style=customasmx86]{patterns/02_stack/04_alloca/2_1_gcc_ATT_O3.s}

\myindex{\ATTSyntax}
The code e' uguale a quello del listato precedente.

A proposito, \INS{movl \$3, 20(\%esp)} corrisponde a \INS{mov DWORD PTR [esp+20], 3} in sintassi Intel.
In sintassi AT\&T, il formato registro+offset per indirizzare memoria appare come \TT{offset(\%{register})}.
}
\DE{\subsubsection{x86: alloca() Funktion}
\label{alloca}
\myindex{\CStandardLibrary!alloca()}

\newcommand{\AllocaSrcPath}{C:\textbackslash{}Program Files (x86)\textbackslash{}Microsoft Visual Studio 10.0\textbackslash{}VC\textbackslash{}crt\textbackslash{}src\textbackslash{}intel}

Es macht Sinn einen Blick auf die \TT{alloca()} Funktion zu werfen
\footnote{In MSVC, kann die Funktions Implementierung in \TT{alloca16.asm} und \TT{chkstk.asm} in \\
\TT{\AllocaSrcPath{}}} gefunden werden.
Diese Funktion arbeitet wie \TT{malloc()}, nur das sie Speicher direkt auf dem Stack bereit stellt.

Der allozierte Speicher Chunk muss nicht wieder mit \TT{free()} freigegeben werden, weil
der Funktions Epilog (\myref{sec:prologepilog}) das \ESP Register wieder in seinen ursprünglichen 
Zustand versetzt und der allozierte Speicher wird einfach \IT{verworfen}. 
Es macht Sinn sich anzuschauen wie \TT{alloca()} implementiert ist.
Mit einfachen Begriffen erklärt, diese Funktion verschiebt \ESP in Richtung des Stack ende mit der 
Anzahl der Bytes die alloziert werden müssen und setzt \ESP als einen Zeiger auf den \IT{allozierten} block.

Beispiel:

\lstinputlisting{patterns/02_stack/04_alloca/2_1.c}


Die \TT{\_snprintf()} Funktion arbeitetet genau wie \printf, nur statt die Ergebnisse nach \gls{stdout} aus zu geben ( bsp. auf dem Terminal oder Konsole), schreibt sie in den \TT{buf} buffer. Die Funktion \puts kopiert den Inhalt aus \TT{buf} nach \gls{stdout}. Sicher könnte man die beiden Funktions Aufrufe könnten durch einen \printf Aufruf ersetzt werden, aber wir sollten einen genaueren Blick auf die Benutzung kleiner Buffer anschauen.

\myparagraph{MSVC}

Compilierung mit MSVC 2010: 

\lstinputlisting[caption=MSVC 2010]{patterns/02_stack/04_alloca/2_2_msvc.asm}

\myindex{Compiler intrinsisch}
Das einzige \\TT{alloca()} Argument wird über \EAX übergeben (anstatt es erst auf den Stack zu pushen)
\footnote{Das liegt daran, das alloca() Verhalten Compiler intrinsisch bestimmt (\myref{sec:compiler_intrinsic}) im Gegensatz zu einer normalen Funktion. Einer der Gründe dafür das man braucht eine separate Funktion braucht, statt ein paar Code Instruktionen im Code,  ist weil die \ac{MSCV} alloca() Implementierung ebenfalls Code hat welcher aus dem gerade allozierten Speicher gelesen wird. Damit in Folge das \ac{Betriebssystem} physikalischen Speicher in dieser \ac{VM} Region zu allozieren. Nach dem \TT{alloca()} Aufruf, zeigt \ESP auf den Block von 600 Bytes der nun als Speicher für das \TT{buf} Array dienen kann.}.

\myparagraph{GCC + \IntelSyntax}

GCC 4.4.1 macht das selbe, aber ohne externe Funktions aufrufe.

\lstinputlisting[caption=GCC 4.7.3]{patterns/02_stack/04_alloca/2_1_gcc_intel_O3_EN.asm}

\myparagraph{GCC + \ATTSyntax}

Nun der gleiche Code, aber in AT\&T Syntax:

\lstinputlisting[caption=GCC 4.7.3]{patterns/02_stack/04_alloca/2_1_gcc_ATT_O3.s}

\myindex{\ATTSyntax}
Der Code ist der gleiche wie im vorherigen listig.

Übrigens, \INS{movl \$3, 20(\%esp)} in AT\&T Syntax wird zu \
\INS{mov DWORD PTR [esp+20], 3} in Intel-syntax.
In der AT\&T Syntax, sehen Register+Offset Formatierungen einer Adresse so aus:
\TT{offset(\%{register})}.
}

\subsection{(Windows) SEH}
\index{Windows!Structured Exception Handling}

\RU{В стеке хранятся записи \ac{SEH} для функции (если они присутствуют)}%
\EN{\ac{SEH} records are also stored on the stack (if they are present).}.

\ifx\LITE\undefined
\RU{Читайте больше о нем здесь}\EN{Read more about it}: (\myref{sec:SEH}).
\fi

\subsection{\RU{Защита от переполнений буфера}\EN{Buffer overflow protection}\PTBR{Proteção contra estouro de buffer}}

\RU{Здесь больше об этом}\EN{More about it here}\PTBR{Mais sobre aqui}~(\myref{subsec:bufferoverflow}).



\subsection{\EN{Automatic deallocation of data in stack}\RU{Автоматическое освобождение данных в стеке}}

\RU{Возможно, причина хранения локальных переменных и SEH-записей в стеке в том, что после выхода из функции, всё эти данные освобождаются автоматически,
используя только одну инструкцию корректирования указателя стека (часто это ADD).}
\EN{Perhaps, the reason for storing local variables and SEH records in the stack is that they are freed automatically upon function exit,
using just one instruction to correct the stack pointer (it is often ADD).}
\RU{Аргументы функций, можно сказать, тоже освобождаются автоматически в конце функции.}
\EN{Function arguments, as we could say, are also deallocated automatically at the end of function.}
\RU{А всё что хранится в куче (\IT{heap}) нужно освобождать явно.}
\EN{In contrast, everything stored in the \IT{heap} must be deallocated explicitly.}

% sections
\section{\RU{Разметка типичного стека}\EN{Typical stack layout}}

\RU{Разметка типичного стека в 32-битной среде, на момент начала ф-ции, 
перед исполнением самой первой инструкции, выглядит так}
\EN{A very typical stack layout in a 32-bit environment at the start of a function, 
before first instruction executed:}

\begin{center}
\begin{tabular}{ | l | l | }
\hline
\dots & \dots \\
\hline
ESP-0xC & \RU{локальная переменная}\EN{local variable} \#2, \MarkedInIDAAs{} \TT{var\_8} \\
\hline
ESP-8 & \RU{локальная переменная}\EN{local variable} \#1, \MarkedInIDAAs{} \TT{var\_4} \\
\hline
ESP-4 & \RU{сохраненное значение}\EN{saved value of} \EBP \\
\hline
ESP & \RU{адрес возврата}\EN{return address} \\
\hline
ESP+4 & \argument \#1, \MarkedInIDAAs{} \TT{arg\_0} \\
\hline
ESP+8 & \argument \#2, \MarkedInIDAAs{} \TT{arg\_4} \\
\hline
ESP+0xC & \argument \#3, \MarkedInIDAAs{} \TT{arg\_8} \\
\hline
\dots & \dots \\
\hline
\end{tabular}
\end{center}

\ifx\LITE\undefined
\section{\RU{Мусор в стеке}\EN{Noise in stack}}

\RU{Часто в этой книге говорится о \q{шуме} или \q{мусоре} в стеке или памяти.}
\EN{Often in this book \q{noise} or \q{garbage} values in the stack or memory are mentioned.}
\RU{Откуда он берется}\EN{Where do they come from}?
\RU{Это то, что осталось там после исполнения предыдущих функций.}
\EN{These are what was left in there after other functions' executions.}
\RU{Короткий пример}\EN{Short example}:

\lstinputlisting{patterns/02_stack/08_noise/st.c}

\RU{Компилируем}\EN{Compiling}\dots

\lstinputlisting[caption=\NonOptimizing MSVC 2010]{patterns/02_stack/08_noise/st.asm}

\RU{Компилятор поворчит немного}\EN{The compiler will grumble a little bit}\dots

\begin{lstlisting}
c:\Polygon\c>cl st.c /Fast.asm /MD
Microsoft (R) 32-bit C/C++ Optimizing Compiler Version 16.00.40219.01 for 80x86
Copyright (C) Microsoft Corporation.  All rights reserved.

st.c
c:\polygon\c\st.c(11) : warning C4700: uninitialized local variable 'c' used
c:\polygon\c\st.c(11) : warning C4700: uninitialized local variable 'b' used
c:\polygon\c\st.c(11) : warning C4700: uninitialized local variable 'a' used
Microsoft (R) Incremental Linker Version 10.00.40219.01
Copyright (C) Microsoft Corporation.  All rights reserved.

/out:st.exe
st.obj
\end{lstlisting}

\RU{Но когда мы запускаем}\EN{But when we run the compiled program}\dots

\begin{lstlisting}
c:\Polygon\c>st
1, 2, 3
\end{lstlisting}

\RU{Ох. Вот это странно. Мы ведь не устанавливали значения никаких переменных в}\EN{Oh, 
what a weird thing! We did not set any variables in} \TT{f2()}. 
\RU{Эти значения --- это \q{привидения}, которые всё ещё в стеке.}
\EN{These are \q{ghosts} values, which are still in the stack.}

\clearpage
\RU{Загрузим пример в}\EN{Let's load the example into} \olly:

\begin{figure}[H]
\centering
\includegraphics[scale=\FigScale]{patterns/02_stack/08_noise/olly1.png}
\caption{\olly: \TT{f1()}}
\label{fig:stack_noise_olly1}
\end{figure}

\RU{Когда}\EN{When} \TT{f1()} \RU{заполняет переменные}\EN{assigns the variables} $a$, $b$ \AndENRU $c$ 
\RU{они сохраняются по адресу}\EN{, their values are stored at the address} \TT{0x1FF860} 
\RU{\etc{}.}\EN{and so on.}

\clearpage
\RU{А когда исполняется}\EN{And when} \TT{f2()}\EN{ executes}:

\begin{figure}[H]
\centering
\includegraphics[scale=\FigScale]{patterns/02_stack/08_noise/olly2.png}
\caption{\olly: \TT{f2()}}
\label{fig:stack_noise_olly2}
\end{figure}

... $a$, $b$ \AndENRU $c$ \RU{в функции}\EN{of} \TT{f2()} \RU{находятся по тем же адресам!}
\EN{are located at the same addresses!}
\RU{Пока никто не перезаписал их, так что они здесь в нетронутом виде.}
\EN{No one has overwritten the values yet, so at that point they are still untouched.}

\RU{Для создания такой странной ситуации несколько функций должны исполняться друг за другом
и \ac{SP} должен быть одинаковым при входе в функции, т.е. у функций должно быть равное количество
аргументов). Тогда локальные переменные будут расположены в том же месте стека.}
\EN{So, for this weird situation to occur, several functions have to be called one after another and
\ac{SP} has to be the same at each function entry (i.e., they have the same number
of arguments). Then the local variables will be located at the same positions in the stack.}

\RU{Подводя итоги, все значения в стеке (да и памяти вообще) это значения оставшиеся от 
исполнения предыдущих функций.}
\EN{Summarizing, all values in the stack (and memory cells in general) 
have values left there from previous function executions.}
\RU{Строго говоря, они не случайны, они скорее непредсказуемы.}
\EN{They are not random in the strict sense, but rather have unpredictable values.}

\RU{А как иначе}\EN{Is there another option}?
\RU{Можно было бы очищать части стека перед исполнением каждой функции,
но это слишком много лишней (и ненужной) работы.}
\EN{It probably would be possible to clear portions of the stack before each function execution,
but that's too much extra (and unnecessary) work.}

\subsection{MSVC 2013}

\EN{The example was compiled by}\RU{Этот пример был скомпилирован в} MSVC 2010.
\EN{But the reader of this book made attempt to compile this example in MSVC 2013, ran it, and got all 3 numbers reversed:}%
\RU{Но один читатель этой книги сделал попытку скомпилировать пример в MSVC 2013, запустил и увидел 3 числа в обратном порядке:}

\begin{lstlisting}
c:\Polygon\c>st
3, 2, 1
\end{lstlisting}

\EN{Why?}\RU{Почему?}

\EN{I also compiled this example in MSVC 2013 and saw this:}%
\RU{Я также попробовал скомпилировать этот пример в MSVC 2013 и увидел это:}

\begin{lstlisting}[caption=MSVC 2013]
_a$ = -12						; size = 4
_b$ = -8						; size = 4
_c$ = -4						; size = 4
_f2	PROC

...

_f2	ENDP

_c$ = -12						; size = 4
_b$ = -8						; size = 4
_a$ = -4						; size = 4
_f1	PROC

...

_f1	ENDP
\end{lstlisting}

\EN{Unlike MSVC 2010, MSVC 2013 allocated a/b/c variables in function \TT{f2()} in reverse order.}%
\RU{В отличии от MSVC 2010, MSVC 2013 разместил переменные a/b/c в функции \TT{f2()} в обратном порядке.}
\EN{And this is completely correct, because \CCpp standards has no rule, in which order local variables must be allocated in the local stack, if at all.}%
\RU{И это полностью корректно, потому что в стандартах \CCpp нет правила, в каком порядке локальные переменные должны быть размещены в локальном стеке,
если вообще.}
\EN{The reason of difference is because MSVC 2010 has one way to do it, and MSVC 2013 has probably something changed inside of compiler guts, so it behaves
slightly different.}%
\RU{Разница есть из-за того что MSVC 2010 делает это одним способом, а в MSVC 2013, вероятно, что-то немного изменили во внутренностях компилятора,
так что он ведет себя слегка иначе.}


\fi
\ifdefined\IncludeExercises
\section{\Exercises}

\subsection{\Exercise \#1}
\label{exercise_stack_1}

\RU{Если это скомпилировать в MSVC и запустить, появится три числа. Откуда они берутся? 
Откуда они берутся если скомпилировать в MSVC с оптимизациями (\Ox)?}
\EN{If to compile this piece of code in MSVC and run, a three number will be printed. 
Where they are came from?
Where they are came from if to compile it in MSVC with optimization (\Ox)?}
\RU{Почему в GCC ситуация совсем иная}\EN{Why the situation is completely different in GCC}?

\begin{lstlisting}
#include <stdio.h>

int main()
{
	printf ("%d, %d, %d\n");

	return 0;
};
\end{lstlisting}

\RU{Ответ}\EN{Answer}: \ref{exercise_solutions_stack_1}.

\fi
