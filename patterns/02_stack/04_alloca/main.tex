\subsection{x86: \RU{Функция alloca()}\EN{alloca() function}\PTBR{a função alloca()}}
\label{alloca}
\index{\CStandardLibrary!alloca()}

\newcommand{\AllocaSrcPath}{C:\textbackslash{}Program Files (x86)\textbackslash{}Microsoft Visual Studio 10.0\textbackslash{}VC\textbackslash{}crt\textbackslash{}src\textbackslash{}intel}

\RU{Интересен случай с функцией \TT{alloca()}
\footnote{В MSVC, реализацию функции можно посмотреть в файлах \TT{alloca16.asm} и \TT{chkstk.asm} в \TT{\AllocaSrcPath{}}}. 
Эта функция работает как \TT{malloc()}, но выделяет память прямо в стеке.}%
\EN{It is worth noting the \TT{alloca()} function
\footnote{In MSVC, the function implementation can be found in \TT{alloca16.asm} and \TT{chkstk.asm} in \TT{\AllocaSrcPath{}}}.
This function works like \TT{malloc()}, but allocates memory directly on the stack.}%
\PTBR{A função \TT{alloca()}
\footnote{No MSVC, a implementação da função pode ser encontrada nos arquivos \TT{alloca16.asm} e \TT{chkstk.asm} em \TT{\AllocaSrcPath{}}}
funciona da mesma maneira que \TT{malloc()}, mas aloca memória diretamente na pilha.}%

\RU{Память освобождать через \TT{free()} не нужно, так как эпилог функции~(\myref{sec:prologepilog})
вернет \ESP в изначальное состояние и выделенная память просто \IT{выкидывается}.}%
\EN{The allocated memory chunk does not need to be freed via a \TT{free()} function call, since the 
function epilogue~(\myref{sec:prologepilog}) returns \ESP back to its initial state and 
the allocated memory is just \IT{dropped}.}%
\PTBR{O bloco de memória alocado não precisa ser limpo através da chamada da função free(),
desde que o rodapé da função (\myref{sec:prologepilog}) retorna \ESP de volta para seu estado inicial e a memória alocada é simplesmente desassociada.}

\RU{Интересна реализация функции \TT{alloca()}.
Эта функция, если упрощенно, просто сдвигает \ESP вглубь стека на столько байт, сколько вам нужно и возвращает \ESP в качестве указателя на выделенный блок.}%
\EN{It is worth noting how \TT{alloca()} is implemented.
In simple terms, this function just shifts \ESP downwards toward the stack bottom by the number of bytes you need and sets \ESP as a pointer to the \IT{allocated} block.}%
\PTBR{Sobre como a função \TT{alloca()} é implementada, em termos simples, essa função só desloca \ESP para baixo 
(em direção ao fundo da pilha) pelo número de bytes que você precisa e define o ESP como um ponteiro para o bloco alocado.}%

\RU{Попробуем:}\EN{Let's try:}\PTBR{Vamos tentar:}

\lstinputlisting{patterns/02_stack/04_alloca/2_1.c}

\RU{Функция \TT{\_snprintf()} работает так же, как и \printf, только вместо выдачи результата в \gls{stdout} (т.е. на терминал или в консоль),
записывает его в буфер \TT{buf}. Функция \puts выдает содержимое буфера \TT{buf} в \gls{stdout}. Конечно, можно было бы
заменить оба этих вызова на один \printf, но здесь нужно проиллюстрировать использование небольшого буфера.}%
\EN{\TT{\_snprintf()} function works just like \printf, but instead of dumping the result into \gls{stdout} (e.g., to terminal or 
console), it writes it to the \TT{buf} buffer. Function \puts copies the contents of \TT{buf} to \gls{stdout}. Of course, these two
function calls might be replaced by one \printf call, but we have to illustrate small buffer usage.}%
\PTBR{A função \TT{\_snprintf()} funciona exatamente como \printf, mas ao invés de jogar o resultado em stdout
(terminal ou console, por exemplo), ela escreve no buffer buf.
A função \puts copia o conteúdo para um buf do stdout.
Lógico, essas duas chamadas de funções podem ser substituídas por um \printf, mas nós temos que ilustrar o uso pequeno do buffer.}%

\subsubsection{MSVC}

\RU{Компилируем}\EN{Let's compile}\PTBR{Vamos compilar} (MSVC 2010):

\lstinputlisting[caption=MSVC 2010]{patterns/02_stack/04_alloca/2_2_msvc.asm}

\index{Compiler intrinsic}
\RU {Единственный параметр в \TT{alloca()} передается через \EAX, а не как обычно через стек}%
\EN{The sole \TT{alloca()} argument is passed via \EAX (instead of pushing it into the stack)}%
\PTBR{O único argumento da função alloca() é passado via EAX (ao invés de ser empurrado na pilha)}%
\footnote{
\RU{Это потому, что alloca()~--- это не сколько функция, сколько т.н. \IT{compiler intrinsic}%
(\myref{sec:compiler_intrinsic})
}%
\PTBR{
Isso é devido ao fato de que alloca() é mais nativa do compilador do que uma função normal
(\myref{sec:compiler_intrinsic}) 
}%
\EN{It is because alloca() is rather a compiler intrinsic 
(\myref{sec:compiler_intrinsic}) 
than a normal function}.

\RU{Одна из причин, почему здесь нужна именно функция, а не несколько инструкций прямо в коде в том, что в реализации 
функции alloca() от \ac{MSVC}
есть также код, читающий из только что выделенной памяти, чтобы \ac{OS} подключила физическую память к этому региону \ac{VM}.}%
\EN{One of the reasons we need a separate function instead of just a couple of instructions in the code,
is because the \ac{MSVC} alloca() implementation also has code which reads from the memory just allocated, in order to let the \ac{OS} map
physical memory to this \ac{VM} region.}%
\PTBR{Um dos motivos que se faz necessário o separamento da função ao invés de um pouco de linhas de código no código,
é porque a implementação da alloca() no MSVC também tem código que é lido da memória que acabou de ser alocada,
para deixar o sistema operacional mapear a memória física para essa região da memória virtual.}
}.
\RU{После вызова \TT{alloca()} \ESP указывает на блок в 600 байт, который мы можем использовать под \TT{buf}.}%
\EN{After the \TT{alloca()} call, \ESP points to the block of 600 bytes and we can use it as memory for the \TT{buf} array.}%
\PTBR{Depois da chamada de \TT{alloca()}, \ESP aponta para o bloco de 600 bytes que nós podemos usar como memória para o array.}%

\ifdefined\IncludeGCC
\subsubsection{GCC + \IntelSyntax}

\RU{А GCC 4.4.1 обходится без вызова других функций:}
\EN{GCC 4.4.1 does the same without calling external functions:}

\lstinputlisting[caption=GCC 4.7.3]{patterns/02_stack/04_alloca/2_1_gcc_intel_O3.asm.\LANG}

\subsubsection{GCC + \ATTSyntax}

\RU{Посмотрим на тот же код, только в синтаксисе AT\&T}\EN{Let's see the same code, but in AT\&T syntax}:

\lstinputlisting[caption=GCC 4.7.3]{patterns/02_stack/04_alloca/2_1_gcc_ATT_O3.s}

\index{\ATTSyntax}
\RU{Всё то же самое, что и в прошлом листинге.}\EN{The code is the same as in the previous listing.}

\RU{Кстати}\EN{By the way}, \TT{movl \$3, 20(\%esp)}%
\RU{~--- это аналог}\EN{corresponds to} \TT{mov DWORD PTR [esp+20], 3}
\RU{в синтаксисе Intel.}\EN{ in Intel-syntax.}
\RU{Адресация памяти в виде \IT{регистр+смещение} записывается в синтаксисе AT\&T как \TT{смещение(\%{регистр})}.}
\EN{In the AT\&T syntax, the register+offset format of addressing memory looks like \TT{offset(\%{register})}.}
\fi
