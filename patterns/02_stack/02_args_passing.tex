\subsection{\RU{Передача параметров функции}\EN{Passing function arguments}\PTBR{Passando argumento de funções}}

\RU{Самый распространенный способ передачи параметров в x86 называется \q{cdecl}:}%
\EN{The most popular way to pass parameters in x86 is called \q{cdecl}:}%
\PTBR{A maneira mais comum de se passar parâmetros no x86 é chamada \q{cdecl}:}%

\begin{lstlisting}
push arg3
push arg2
push arg1
call f
add esp, 12 ; 4*3=12
\end{lstlisting}

\RU{Вызываемая функция получает свои параметры также через указатель стека.}%
\EN{\Gls{callee} functions get their arguments via the stack pointer.}%
\PTBR{Uma função chamada recebe seus argumentos pelo ponteiro da pilha.}%

\RU{Следовательно, так расположены значения в стеке перед исполнением самой первой инструкции функции \ttf{}:}%
\EN{Therefore, this is how the argument values are located in the stack before the execution of the \ttf{} function's very first instruction:}%
\PTBR{Portanto, é assim que os valores dos argumentos são alocados na pilha antes da execução das primeiras intruções da função \ttf{}:}%

\begin{center}
\begin{tabular}{ | l | l | }
\hline
ESP & \RU{адрес возврата}\EN{return address}\PTBR{endereço de retorno} \\
\hline
ESP+4 & \argument \#1, \MarkedInIDAAs{} \TT{arg\_0} \\
\hline
ESP+8 & \argument \#2, \MarkedInIDAAs{} \TT{arg\_4} \\
\hline
ESP+0xC & \argument \#3, \MarkedInIDAAs{} \TT{arg\_8} \\
\hline
\dots & \dots \\
\hline
\end{tabular}
\end{center}

\ifx\LITE\undefined
\RU{См. также в соответствующем разделе о других способах передачи аргументов через стек}
\EN{For more information on other calling conventions see also section}~(\myref{sec:callingconventions}).
\fi % LITE
\RU{Важно отметить, что, в общем, никто не заставляет программистов передавать параметры именно через стек, это не является требованием к исполняемому коду.
Вы можете делать это совершенно иначе, не используя стек вообще.}%
\EN{It is worth noting that nothing obliges programmers to pass arguments through the stack. It is not a requirement.
One could implement any other method without using the stack at all.}%
\PTBR{Vale ressaltar que nada obriga o programador a passar os argumentos pela pilha. Não é um requerimento.
Você pode implementar qualquer outro método usando a pilha da maneira que desejar.}%

\RU{К примеру, можно выделять в \glslink{heap}{куче} место для аргументов, заполнять их и передавать в функцию указатель на это место через \EAX. И это вполне будет работать}%
\EN{For example, it is possible to allocate a space for arguments in the \gls{heap}, fill it and pass it to a function via a pointer to this block in the \EAX register. This will work}%
\PTBR{Por exemplo, é possível alocar um espaço para argumentos na \gls{heap}, preencher e passar para a função via um ponteiro para esse bloco no registrador \EAX.}%
\footnote{\RU{Например, в книге Дональда Кнута \q{Искусство программирования}, в разделе 1.4.1 
посвященном подпрограммам \cite[раздел 1.4.1]{Knuth:1998:ACP:521463}, 
мы можем прочитать о возможности располагать параметры для вызываемой подпрограммы после инструкции \JMP,
передающей управление подпрограмме. Кнут описывает, что это было особенно удобно для компьютеров IBM System/360.}%
\EN{For example, in the \q{The Art of Computer Programming} book by Donald Knuth, 
in section 1.4.1 dedicated to subroutines \cite[section 1.4.1]{Knuth:1998:ACP:521463},
we could read that one way to supply arguments to a subroutine is simply to list them after the \JMP instruction
passing control to subroutine. Knuth explains that this method was particularly convenient on IBM System/360.}
\PTBR{Por exemplo, no livro ``The Art of Computer Programming'' por Donald Knuth, na seção 1.4.1 dedicada a subrotinas \cite[section 1.4.1]{Knuth:1998:ACP:521463}.
Nós podemos ler que uma das maneiras de mandar argumentos para uma subrotina é simplesmente listá-los depois de uma instrução \JMP passando o controle para a subrotina.
Knuth explica que esse método era particularmente conveniente no IBM System/360.}}.
% I am unsure about what this comment means.
% My guess is that the arguments are put in the memory position after
% the jump instruction, so you could say:
% "one way to supply arguments to a subroutine is simply to list them in memory
% after the \JMP instruction that passes control to the subroutine."
% Right now, "after" also sounds like it refers to the time after
% the jump happens, which I think is too late.
\RU{Однако традиционно сложилось, что в x86 и ARM передача аргументов происходит именно через стек.}%
\EN{However, it is a convenient custom in x86 and ARM to use the stack for this purpose.}%
\PTBR{Isso vai funcionar, entretando, é de senso comum no x86 e ARM a usar a pilha para esse fim.}%

\par
\RU{Кстати, вызываемая функция не имеет информации о количестве переданных ей аргументов.
Функции Си с переменным количеством аргументов (как \printf) определяют их количество по спецификаторам строки формата (начинающиеся со знака \%).}%
\EN{By the way, the \gls{callee} function does not have any information about how many arguments were passed.
C functions with a variable number of arguments (like \printf) determine their number using format string  specifiers (which begin with the \% symbol).}%
\PTBR{A propósito, a função chamada não tem nenhuma informação sobre quantos argumentos foram passados.
Funções em C com um número variável de argumentos (como \printf) determina seu número usando formatações específicas de string (que começam com o símbolo \%).}%

\RU{Если написать что-то вроде:}\EN{If we write something like:}\PTBR{Se nós escrevermos algo como:}

\begin{lstlisting}
printf("%d %d %d", 1234);
\end{lstlisting}

\RU{\printf выведет 1234, затем ещё два случайных числа, которые волею случая оказались в стеке рядом.}%
\EN{\printf will print 1234, and then two random numbers, which were lying next to it in the stack.}%
\PTBR{\printf vai mostrar 1234, e então dois números aleatórios, que estariam próximos a stack.}%

\par
\RU{Вот почему не так уж и важно, как объявлять функцию \main{}: как \main{}, \TT{main(int argc, char *argv[])} либо \TT{main(int argc, char *argv[], char *envp[])}.}%
\EN{That's why it is not very important how we declare the \main function: as \main, \TT{main(int argc, char *argv[])} or \TT{main(int argc, char *argv[], char *envp[])}.}%
\PTBR{É por isso que não é muito importante como declaramos a função \main{}: como \main{}, \TT{main(int argc, char *argv[])} ou \TT{main(int argc, char *argv[], char *envp[])}.}%

\RU{В реальности, \ac{CRT}-код вызывает \main примерно так:}%
\EN{In fact, the \ac{CRT}-code is calling \main roughly as:}%
\PTBR{Na verdade, o código da C Runtime Library está chamando grosseiramente \main{} dessa maneira:}%
	
\begin{lstlisting}
push envp
push argv
push argc
call main
...
\end{lstlisting}

\RU{Если вы объявляете \main без аргументов, они, тем не менее, присутствуют в стеке, но не используются.
Если вы объявите \main как \TT{main(int argc, char *argv[])}, 
вы можете использовать два первых аргумента, а третий останется для вашей функции \q{невидимым}.
Более того, можно даже объявить \TT{main(int argc)}, и это будет работать.}%
\EN{If you declare \main as \main without arguments, they are, nevertheless, still present in the stack, but are not used.
If you declare \main as  \TT{main(int argc, char *argv[])},
you will be able to use first two arguments, and the third will remain \q{invisible} for your function.
Even more, it is possible to declare \TT{main(int argc)}, and it will work.}%
\PTBR{Se você declarar \main como \main sem argumentos, mesmo assim eles ainda estarão presentes na pilha, mas não são usados.
Se você declarar \main como \TT{main(int argc, char *argv[])}, você será capaz de utilizar os primeiros dois argumentos e o terceiro vai continuar \q{invisível} para a sua função.
Da mesma maneira, é possível declarar a \main como \TT{main(int argc)} e ainda assim vai funcionar.}

