\ifdefined\RUSSIAN
\else

\chapter{Branchless abs() function}
\label{chap:branchless_abs}

Let's revisit example I gave earlier \ref{sec:abs} and ask ourselves, is it possible
to make a branchless version of the function in x86 code?

\lstinputlisting{abs.c}

And the answer is yes.

\section{\Optimizing GCC 4.9.1 x64}

We could see it if to compile it using optimizing GCC 4.9:

\lstinputlisting[caption=\Optimizing GCC 4.9 x64]{patterns/27_abs_branchless/abs_GCC491_x64_O3.asm}

That's how it works:

Algorithmically shift input value left by 31. 
Algorithmical shift meaning sign extension, so if \ac{MSB} is 1, all 32 bits will be filled with 1, or with 0
if otherwise.
\index{x86!\Instructions!SAR}
In other words, \TT{SAR reg, 31} instruction makes \TT{0xFFFFFFFF} if the sign was negative or 0 if positive.
After \TT{SAR} executed, we have this value in EDX.
\index{x86!\Instructions!XOR}
Then, if value is \TT{0xFFFFFFFF} (i.e., sign is negative), input value is inverted 
(because \TT{XOR reg, 0xFFFFFFFF} is effectively inverse operation).
Then, again, if value is \TT{0xFFFFFFFF} (i.e., sign is negative), 1 is added to the final result (because
subtracting -1 from some value resulting in incrementing it).
Inversion and incrementing is exactly how two's complement value is negated: \ref{sec:signednumbers}.

We may observe that the last two instruction does something if sign of input value is negative.
Otherwise (if sign is positive) they do not do anything at all, leaving input value untouched.

The algorithm is described in \cite[2-4]{Warren:2002:HD:515297}.
I'm not sure, how GCC does it, deduced by itself or find suitable pattern among known ones?

\section{\Optimizing GCC 4.9 ARM64}

GCC 4.9 for ARM64 makes mostly the same, decide to use full 64-bit registers.
There are less number of instructions, because input value can be shifted using instruction suffixed (``asr'')
instead of separate instruction.

\lstinputlisting[caption=\Optimizing GCC 4.9 ARM64]{patterns/27_abs_branchless/abs_GCC49_ARM64_O3.s}

\fi