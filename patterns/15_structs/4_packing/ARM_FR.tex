\subsubsection{ARM}

\myparagraph{\OptimizingKeilVI (\ThumbMode)}

\lstinputlisting[caption=\OptimizingKeilVI (\ThumbMode),style=customasmARM]{patterns/15_structs/4_packing/packing_Keil_thumb.asm}

Rappelons-nous que c'est une structure qui est pass�e ici et non pas un pointeur vers une structure. Comme 
les 4 premiers arguments d'une fonction sont pass�s dans les registres sur les processeurs ARM, les champs 
de la structure sont pass�s dans les registres \TT{R0-R3}.

\myindex{ARM!\Instructions!LDRB}
\myindex{x86!\Instructions!MOVSX}
\TT{LDRB} charge un octet pr�sent en m�moire et l'�tend sur 32bits en prenant en compte son signe. Cette 
op�ration est similaire � celle effectu�e par \MOVSX dans les architectures x86. Elle est utilis�e ici pour 
charger les champs $a$ et $c$ de la structure.

\myindex{Epilogue de fonction}

Un autre d�tail que nous remarquons ais�ment est que la fonction ne s'ach�ve pas sur un �pilogue qui lui est 
propre. A la place, il y a un saut vers l'�pilogue d'une autre fonction! Qui plus est celui d'une fonction 
tr�s diff�rente sans aucun lien avec la n�tre. Cependant elle poss�de exactement le m�me �pilogue, 
probablement parce qu'elle accepte utilise elle aussi 5 variables locales ($5*4=0x14$).

De plus elle est situ�e � une adresse proche.

En r�alit�, peut importe l'�pilogue qui est utilis� du moment que le fonctionnement est celui attendu.

Il semble donc que le compilateur Keil d�cide de r�utiliser � des fins d'�conomie un fragment d'une autre 
fonction. Notre �pilogue aurait n�cessit� 4 octets. L'instruction de saut n'en utilise que 2.

\myparagraph{ARM + \OptimizingXcodeIV (\ThumbTwoMode)}

\lstinputlisting[caption=\OptimizingXcodeIV (\ThumbTwoMode),style=customasmARM]{patterns/15_structs/4_packing/packing_Xcode_thumb.asm}

\myindex{ARM!\Instructions!SXTB}
\myindex{x86!\Instructions!MOVSX}
\TT{SXTB} (\IT{Signed Extend Byte}) est similaire � \MOVSX pour les architectures x86.
Pour le reste---c'est identique.
