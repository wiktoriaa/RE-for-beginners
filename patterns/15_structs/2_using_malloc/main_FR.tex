\subsection{Allouons de l'espace pour une structure avec malloc()}
\label{struct_malloc_example}

Il est parfois plus simple de placer une structure sur le \gls{heap} que sur la pile:

\lstinputlisting[style=customc]{patterns/15_structs/2_using_malloc/systemtime_malloc.c}

Compilons cet exemple en utilisant l'option (\Ox) qui facilitera nos observations.

\lstinputlisting[caption=\Optimizing MSVC,style=customasmx86]{patterns/15_structs/2_using_malloc/systemtime_malloc.asm}

\myindex{\CStandardLibrary!malloc()}

Puisque \TT{sizeof(SYSTEMTIME) = 16} c'est aussi le nombre d'octets qui doit être alloué par 
\TT{malloc()}. Celui-ci renvoie dans le registre \EAX un pointeur vers un bloc mémoire fraîchement 
alloué. Puis le pointeur est copié dans le registre \ESI.
La fonction win32 \TT{GetSystemTime()} prend soin que la valeur de \ESI soit la même à l'issue de la 
fonction que lors de son appel. C'est pourquoi nous pouvons continuer à l'utiliser après sans avoir 
eu besoin de le sauvegarder.

\myindex{x86!\Instructions!MOVZX}

Tiens, une nouvelle instruction ~---\MOVZX (\IT{Move with Zero eXtend}).
La plupart du temps, elle peut être utilisée comme \MOVSX. La différence est qu'elle positionne 
systématiquement les bits supplémentaires à 0.
Elle est utilisée ici car \printf attend une valeur sur 32 bits et que nous ne disposons que d'un 
WORD dans la stucture~---c'est à dire une valeur non signée sur 16 bits. Il nous faut donc forcer 
à zéro les bits 16 à 31 lorsque le WORD est copié dans un \Tint{}, sinon nous risquons de 
récupérer des bits résiduels de la précédente opération sur le registre.

Dans cet exemple, il reste possible de représenter la structure sous forme d'un tableau de 8 WORDs:

\lstinputlisting[style=customc]{patterns/15_structs/2_using_malloc/systemtime_malloc2.c}

Nous avons alors:

\lstinputlisting[caption=\Optimizing MSVC,style=customasmx86]{patterns/15_structs/2_using_malloc/systemtime_malloc2.asm}

Encore une fois nous obtenons un code qu'il n'est pas possible de discerner du précédent.

Et encore une fois, vous n'avez pas intérêt à faire cela, sauf si vous savez exactement ce que vous 
faites.

