\subsubsection{\WorkingWithFloatAsWithStructSubSubSectionName}
\label{sec:floatasstruct}
Wie wir bereits im Abschnitt über die FPU~(\myref{sec:FPU}) besprochen haben, bestehen \Tfloat und \Tdouble jeweils aus
einem Vorzeichen, einem Signifikanden (oder Bruch) und einem Exponenten.
Wir wollen am Beispiel des Typs \Tfloat untersuchen, ob wir direkt mit diesen Feldern arbeiten können.

\bigskip
% a hack used here! http://tex.stackexchange.com/questions/73524/bytefield-package
\begin{center}
\begin{bytefield}{32}
	\bitheader[endianness=big]{0,22,23,30,31} \\
	\bitbox{1}{S} & 
	\bitbox{8}{%
		\RU{экспонента}%
		\EN{exponent}%
		\ES{exponente}%
		\PTBRph{}%
		\DEph{}\PLph{}%
		\ITAph{}%
		\FR{exposant}
	} & 
	\bitbox{23}{%
		\RU{мантисса}%
		\EN{mantissa or fraction}%
		\ES{mantisa o fracci\'on}%
		\PTBRph{}%
		\DEph{}\PLph{}%
		\ITAph{}%
		\FR{mantisse ou fraction}
	}
\end{bytefield}
\end{center}

\begin{center}
( S\EMDASH{}%
	\RU{знак}%
	\EN{sign}%
	\ES{signo}%
	\PTBRph{}%
	\DEph{}\PLph{}%
	\ITAph{}%
	\FR{signe}
)
\end{center}


\lstinputlisting[style=customc]{patterns/15_structs/6_bitfields/float/float_DE.c}

Das struct \TT{float\_as\_struct} belegt den gleichen Speicherplatz wie ein \Tfloat, d.h., 4 Byte oder 32 Bit.

Wir setzen das negative Vorzeichen im Eingabewert und multiplizieren die gesamte Zahl mit \TT{$2^2$}, d.h. 4, indem wir
zum Exponenten 2 hinzuaddieren.

Kompilieren wir das Beispiel mit MSVC 2008 ohne Optimierung:

\lstinputlisting[caption=\NonOptimizing MSVC
2008,style=customasmx86]{patterns/15_structs/6_bitfields/float/float_msvc_DE.asm}
Ein wenig redundant.
Hätten wir mit dem Flag \Ox kompiliert, wäre dort ein Aufruf von TT{memcpy()}, sondern die Variable \TT{f} wäre direkt
verwendet worden.
Einfacher ist es aber auf jeden Fall, sich die nicht optimierte Version anzuschauen.

Was würde GCC 4.4.1 mit \Othree hier tun?

\lstinputlisting[caption=\Optimizing GCC
4.4.1,style=customasmx86]{patterns/15_structs/6_bitfields/float/float_gcc_O3_DE.asm}
Die Funktion \ttf ist fast verständlich. Viel interessanter ist jedoch, dass GCC trotz des Sammelsuriums an Feldern in
der Lage war, das Ergebnis von \TT{f(1.234)} während des Kompilierens zu berechnen und der Funktion \printf 
während der Compilezeit als vorberechneten Wert bereitzustellen.
