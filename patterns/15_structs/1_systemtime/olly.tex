\clearpage
\subsection{\olly}
\index{\olly}

\RU{Компилируем этот пример в}\EN{Let's compile this example in} MSVC 2010 \RU{с ключами}\EN{with} 
\TT{/GS- /MD} \RU{и запускаем в}\EN{keys and run it in} \olly.
\RU{Открываем окна данных и стека по адресу, который передается в качестве первого аргумента в функцию}
\EN{Let's open windows for data and stack at the address which is passed as the first argument of the}
\TT{GetSystemTime()}\EN{ function}, 
\RU{ждем пока эта функция исполнится, и видим следующее}\EN{ and let's wait until it's executed. We see this}:

\begin{figure}[H]
\centering
\includegraphics[scale=\FigScale]{patterns/15_structs/1_systemtime/olly_systemtime1.png}
\caption{\olly: \TT{GetSystemTime()} \RU{только что исполнилась}\EN{just executed}}
\label{fig:struct_olly_1}
\end{figure}

\RU{Точное системное время на моем компьютере, в которое исполнилась функция, это}
\EN{The system time of the function execution on my computer is} 9 \RU{декабря}\EN{december} 2014, 22:29:52:

\begin{figure}[H]
\centering
\includegraphics[scale=\NormalScale]{patterns/15_structs/1_systemtime/olly_systemtime2.png}
\caption{\olly: \RU{Вывод \printf}\EN{\printf output}}
\label{fig:struct_olly_2}
\end{figure}

\RU{Таким образом, в окне данных мы видим следующие 16 байт}\EN{So we see these 16 bytes in the
data window}: 
\begin{lstlisting}
DE 07 0C 00 02 00 09 00 16 00 1D 00 34 00 D4 03
\end{lstlisting}

\RU{Каждые два байта отражают одно поле структуры}\EN{Each two bytes represent one field of the structure}. 
\RU{А так как порядок байт (\gls{endianness}) \IT{little endian},
то в начале следует младший байт, затем старший}\EN{Since the \gls{endianness} is \IT{little endian}, 
we see the low byte first and then the high one}.
\RU{Следовательно, вот какие 16-битные числа сейчас записаны в памяти}
\EN{Hence, these are the values currently stored in memory}:

\begin{center}
\begin{tabular}{ | l | l | l | }
\hline
\headercolor{} \RU{Шестнадцатеричное число}\EN{Hexadecimal number} & 
\headercolor{} \RU{десятичное число}\EN{decimal number} & 
\headercolor{} \RU{имя поля}\EN{field name} \\
\hline
0x07DE & 2014	& wYear \\
\hline
0x000C & 12	& wMonth \\
\hline
0x0002 & 2	& wDayOfWeek \\
\hline
0x0009 & 9	& wDay \\
\hline
0x0016 & 22	& wHour \\
\hline
0x001D & 29	& wMinute \\
\hline
0x0034 & 52	& wSecond \\
\hline	
0x03D4 & 980	& wMilliseconds \\
\hline
\end{tabular}
\end{center}

\RU{В окне стека, видны те же значения, только они сгруппированы как 32-битные значения}
\EN{The same values are seen in the stack window, but they are grouped as 32-bit values}.

\RU{Затем}\EN{And then} \printf \RU{просто берет нужные значения и выводит их на консоль}
\EN{just takes the values it needs and outputs them to the console}.

\RU{Некоторые поля}\EN{Some values aren't output by} \printf \RU{не выводит} (\TT{wDayOfWeek} \AndENRU 
\TT{wMilliseconds}), \RU{но они находятся в памяти и доступны для использования.}
\EN{but they are in memory right now, available for use.}
