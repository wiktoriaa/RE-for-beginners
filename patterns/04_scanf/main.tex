\section{scanf()}
\index{\CStandardLibrary!scanf()}
\label{label_scanf}

\IFRU{Теперь попробуем использовать scanf().}{Now let's use scanf().}

\lstinputlisting{patterns/04_scanf/ex1.c}

\IFRU
{Да, согласен, использовать \scanf в наши времена для того, чтобы спросить у пользователя что-то, 
не самая хорошая идея.
Но я хотел проиллюстрировать передачу указателя на \Tint.}
{OK, I agree, it is not clever to use \scanf today. But I wanted to illustrate passing pointer to \Tint.}

\subsection{\IFRU{Об указателях}{About pointers}}
\index{\CLanguageElements!\Pointers}

\IFRU{Это одна из фундаментальных вещей в компьютерных науках.}{It is one of the most fundamental things in computer
science.}
\IFRU{Часто большой массив, структуру или объект передавать в другую функцию никак не выгодно, 
а передать её адрес куда проще.}
{Often, large array, structure or object, it is too costly to pass to other function, 
while passing its address is much easier.}
\IFRU{К тому же, если вызываемая функция должна изменить что-то в этом большом массиве или структуре,
то возвращать её полностью ~--- это так же абсурдно.}
{More than that: if calling function must modify something in the large array or structure,
to return it as a whole is absurdly as well.}
\IFRU{Так что самое простое, что можно сделать, это передать в функцию адрес массива или структуры,
и пусть она что-то там изменит.}
{So the simplest thing to do is to pass an address of array or structure to function,
and let it change what must be changed.}

\IFRU{Указатель в}{In} \CCpp \IFRU{ ~--- это просто адрес какого-либо места в памяти.}
{it is just an address of some point in memory.}

\index{x86-64}
\IFRU{В x86 адрес представляется в виде 32-битного числа (т.е., занимает 4 байта), а в x86--64 как 64-битное число 
(занимает 8 байт).}
{In x86, address is represented as 32-bit number (i.e., occupying 4 bytes), while in x86--64 it is 64-bit number
(occupying 8 bytes).}
\IFRU{Кстати, отсюда негодование некоторых людей, связанное с переходом на x86-64 ~--- на этой архитектуре все указатели
будут занимать места в 2 раза больше.}
{By the way, that is the reason of some people's indignation related to switching to x86-64~---all pointers
on x64-architecture will require twice as more space.}

\index{\CStandardLibrary!memcpy()}
\IFRU{При некотором упорстве можно работать только с бестиповыми указателями (\TT{void*})}{With some effort,
it is possible to work only with untyped pointers}; \IFRU{например,}{e.g.} 
\IFRU{стандартная функция Си}{standard C function} \TT{memcpy()},
\IFRU{копирующая блок из одного места памяти в другое}{copying a block from one place in memory to another}, 
\IFRU{принимает на вход 2 указателя типа}{takes 2 pointers of} \TT{void*}\EN{ type on input}, 
\IFRU{потому что нельзя
заранее предугадать, какого типа блок вы собираетесь копировать. Да в общем это и не важно, важно только знать размер
блока.}
{since it is impossible to predict block type you would like to copy. And it is not even important to know, 
only block size is important.}

\IFRU{Также указатели широко используются, когда функции нужно вернуть более одного значения}
{Also pointers are widely used when function needs to return more than one value}
(\IFRU{мы еще вернемся к этому в будущем}{we will back to this in future}~(\ref{label_pointers})).
\IT{scanf()} \IFRU{ ~--- это как раз такой случай}{is just that case}. 
\IFRU{Помимо того, что этой функции нужно показать, сколько значений
было прочитано успешно, ей еще и нужно вернуть сами значения.}
{In addition to the function's need to show how many values were read successfully, 
it also should return all these values.}

\IFRU{Тип указателя в}{In} \CCpp \IFRU{нужен для проверки типов на стадии компиляции.}
{pointer type is needed only for type checking on compiling stage.}
\IFRU{Внутри, в скомпилированном коде, никакой информации о типах указателей нет.}
{Internally, in compiled code, there is no information about pointers types.}

\subsection{x86: \IFRU{3 аргумента}{3 arguments}}

\subsubsection{MSVC}

\IFRU{Компилируем при помощи MSVC 2010 Express, и в итоге получим:}
{Let's compile it by MSVC 2010 Express and we got:}

\begin{lstlisting}
$SG3830	DB	'a=%d; b=%d; c=%d', 00H

...

	push	3
	push	2
	push	1
	push	OFFSET $SG3830
	call	_printf
	add	esp, 16					; 00000010H
\end{lstlisting}

\IFRU{Все почти то же, за исключением того, что теперь видно, что аргументы для \printf заталкиваются в стек в обратном порядке: самый первый аргумент заталкивается последним.}
{Almost the same, but now we can see the \printf arguments are pushing into stack in reverse order: and the first argument is pushing in as the last one.}

\IFRU{Кстати, вспомним что переменные типа \Tint в 32-битной системе, как известно, имеет ширину 32 бита, это 4 байта}
{By the way, variables of \Tint type in 32-bit environment has 32-bit width that is 4 bytes}.

\IFRU{Итак, у нас всего 4 аргумента. $4*4 = 16$ ~--- именно 16 байт занимают в стеке указатель на строку плюс еще 3 числа типа \Tint.}
{So, we got here 4 arguments. $4*4 = 16$~---they occupy exactly 16 bytes in the stack: 32-bit pointer to string and 3 number of \Tint type.}

\index{x86!\Instructions!ADD}
\index{x86!\Registers!ESP}
\index{cdecl}
\IFRU{Когда при помощи инструкции \TT{``ADD ESP, X''} корректируется \glslink{stack pointer}{указатель стека} \ESP 
после вызова какой-либо функции, зачастую можно сделать вывод о том, сколько аргументов 
у вызываемой функции было, разделив X на 4.}
{When \gls{stack pointer} (the \ESP register) is corrected by \TT{``ADD ESP, X''}
instruction after a function 
call, often, the number of function arguments could be deduced here: just divide X by 4.}

\IFRU{Конечно, это относится только к cdecl-методу передачи аргументов через стек.}
{Of course, this is related only to \IT{cdecl} calling convention.}

\IFRU{См. также в соответствующем разделе о способах передачи аргументов через стек}
{See also section about calling conventions}~(\ref{sec:callingconventions}).

\IFRU{Иногда бывает так, что подряд идут несколько вызовов разных функций, 
но стек корректируется только один раз, после последнего вызова:}
{It is also possible for compiler to merge several \TT{``ADD ESP, X''} instructions into one, after last call:}

\begin{lstlisting}
push a1
push a2
call ...
...
push a1
call ...
...
push a1
push a2
push a3
call ...
add esp, 24
\end{lstlisting}

\subsubsection{MSVC \AndENRU \olly}
\index{\olly}

\IFRU{Попробуем этот же пример в}{Now let's try to load this example in} \olly.
\IFRU{Это один из наиболее популярных win32-отладчиков user-режима}{It is one of the most 
popular user-land win32 debugger}.
\IFRU{Мы можем компилировать наш пример в}{We can try to compile our example in} MSVC 2012 
\IFRU{с опцией}{with} \TT{/MD} \IFRU{что означает, линковать с библиотекой}{option, meaning, to link 
against} \TT{MSVCR*.DLL},
\IFRU{чтобы импортируемые ф-ции были хорошо видны в отладчике}{so we will able to see imported 
functions clearly in debugger}.

\IFRU{Затем загружаем исполняемый файл в}{Then load executable in} \olly.
\IFRU{Самый первый брякпойнт в}{The very first breakpoint is in} \TT{ntdll.dll}, \IFRU{нажмите}{press} 
F9 (\IFRU{запустить}{run}).
\IFRU{Второй брякпойнт в}{The second breakpoint is in} \ac{CRT}-\IFRU{коде}{code}.
\IFRU{Теперь мы должны найти ф-цию}{Now we should find} \main\EN{ function}.

\IFRU{Найдите этот код скроллируя окно кода до самого верха (MSVC располагает ф-цию \main в самом начале
секции кода)}{Find this code by scrolling the code to the very bottom (MSVC allocates \main function at
the very beginning of the code section)}: 
\figname \ref{fig:printf3_olly_1}.

\IFRU{Кликните на инструкции}{Click on} \TT{PUSH EBP}\IFRU{, нажмите}{ instruction, press} F2 
(\IFRU{установка брякпойнта}{set breakpoint}) \IFRU{и нажмите}{and press} F9 (\IFRU{запустить}{run}).
\IFRU{Нам нужно произвести все эти манипуляции, чтобы пропустить \ac{CRT}-код, потому что нам он пока
не интересен}{We need to do these manupulations in order to skip \ac{CRT}-code, because, we don't really 
interesting in it yet}.

\IFRU{Нажмите}{Press} F8 (\stepover) 6 \IFRU{раз, т.е., пропустить
6 инструкций}{times, i.e., skip 6 instructions}: \figname \ref{fig:printf3_olly_2}.

\IFRU{Теперь}{Now the} \PC \IFRU{указывает на инструкцию}{points to the}
\TT{CALL printf}\EN{ instruction}.
\olly, \IFRU{как и другие отладчики, подсвечивает регистры со значениями, которые изменились}
{like other debuggers, highlights value of registers which were changed}.
\IFRU{Так что, каждый раз, когда мы нажимаем}{So each time you press F8}, \EIP 
\IFRU{изменяется и его значение подсвечивается красным}{is changing and its value looking red}.
\ESP \IFRU{также меняется, потому что значения заталкиваются в стек}{is changing as well, 
because values are pushed into the stack}.

\IFRU{Где находятся эти значения в стеке}{Where are the values in the stack}?
\IFRU{Посмотрите на правое/нижнее окно в отладчике}{Take a look into right/bottom window of debugger}:

\begin{figure}[H]
\centering
\includegraphics[scale=0.66]{patterns/03_printf/olly3_stack.png}
\caption{\olly: \IFRU{стек, после того как значения там сохранены}{stack after values pushed}
(\IFRU{я сделал здесь округлую красную пометку в графическом редакторе}{I made round red mark 
here in graphics editor})}
\end{figure}

\IFRU{Так что здесь видно 3 столбца: адрес в стеке, значение в стеке и еще дополнительный комментарий
от \olly}{So we can see there 3 columns: address in the stack, 
value in the stack and some additional \olly comments}. 
\olly \IFRU{понимает}{understands} \printf\IFRU{-строки}{-like strings}, 
\IFRU{так что он показывает здесь и строку и 3 значения \IT{привязанных} к ней}{so it reports the 
string here and 3 values \IT{attached} to it}.

\IFRU{Нажмите}{Press} F8 (\stepover).

\IFRU{В коносил мы видим вывод}{In the console we'll see the output}:

\begin{figure}[H]
\centering
\includegraphics[scale=0.66]{patterns/03_printf/olly3_console.png}
\caption{\RU{Ф-ция }\printf \IFRU{исполнилась}{function executed}}
\end{figure}

\IFRU{Посмотрим, как изменились регистры и состояние стека}{Let's see how registers and stack state 
are changed}: \figname \ref{fig:printf3_olly_3}.

\RU{Регистр }\EAX \IFRU{теперь содержит}{register now contains} \TT{0xD} (13).
That's correct, \printf returns number of characters printed.
\RU{Значение }\EIP \IFRU{изменилось: действительно, теперь здесь адрес инструкции после}
{value is changed: indeed, now there is address of the instruction after} \TT{CALL printf}.
\RU{Значения регистров }\ECX \AndENRU \EDX \IFRU{также изменились}{values are changed as well}.
\IFRU{Очевидно, внутренности ф-ции \printf используют их для каких-то своих нужд}{Apparently, 
\printf function's hidden machinery used them for its own needs}.

\IFRU{Очень важный момент в том что значение \ESP не изменилось. И состояние стека также!}
{A very important thing is that \ESP value is not changed. And stack state too!}
\IFRU{Мы ясно видим здесь и строку формата и соответствующие ей 3 значения, они все еще здесь}
{We clearly see that format string and corresponding 3 values are still there}.
\IFRU{Действительно, по соглашению вызовов \IT{cdecl}, вызывающая ф-ция не очищает аргументы из стека}
{Indeed, that's \IT{cdecl} calling convention, calling function doesn't clear arguments in stack}.
\IFRU{Это должна делать вызывающая ф-ция}{It's caller's duty to do so}.

\IFRU{Нажмите}{Press} F8 \IFRU{снова, чтобы исполнилась инструкция}{again to execute} 
\TT{ADD ESP, 10}\EN{ instruction}: \figname \ref{fig:printf3_olly_4}.

\ESP \IFRU{изменился, но значения все еще в стеке}{is changed, but values are still in the stack}!
\IFRU{Конечно, никому не нужно заполнять эти значения нулями или что-то в этом роде}{Yes, 
of course, no one needs to fill these values by zero or something like that}.
\IFRU{Потому что всё что выше указателя стека}{Because, everything above stack pointer} (\SP) 
\IFRU{это}{is} \IT{\IFRU{шум}{noise}} \OrENRU \IT{\IFRU{мусор}{garbage}}, \IFRU{это всё не имеет
особой ценности}{it has no value at all}.
\IFRU{Было бы очень затратно по времени очищать ненужные элементы стека, к тому же, никому это и не 
нужно}{It would be time consuming to clear unused stack entries, besides, no one really needs to}.

\begin{figure}[H]
\centering
\includegraphics[scale=0.66]{patterns/03_printf/olly3_1.png}
\caption{\olly: \IFRU{самое начало ф-ции}{the very start of the} \main\EN{ function}}
\label{fig:printf3_olly_1}
\end{figure}

\begin{figure}[H]
\centering
\includegraphics[scale=0.66]{patterns/03_printf/olly3_2.png}
\caption{\olly: \IFRU{перед исполнением}{before} \printf\EN{ execution}}
\label{fig:printf3_olly_2}
\end{figure}

\begin{figure}[H]
\centering
\includegraphics[scale=0.66]{patterns/03_printf/olly3_3.png}
\caption{\olly: \IFRU{после исполнения}{after} \printf\EN{ execution}}
\label{fig:printf3_olly_3}
\end{figure}

\begin{figure}[H]
\centering
\includegraphics[scale=0.66]{patterns/03_printf/olly3_4.png}
\caption{\olly: \IFRU{после исполнения инструкции}{after} \TT{ADD ESP, 10}\EN{ instruction execution}}
\label{fig:printf3_olly_4}
\end{figure}

\subsubsection{GCC}

\IFRU{Скомпилируем то же самое в Linux при помощи GCC 4.4.1 и посмотрим в \IDA что вышло:}
{Now let's compile the same in Linux by GCC 4.4.1 and take a look in \IDA what we got:}

\begin{lstlisting}
main            proc near

var_10          = dword ptr -10h
var_C           = dword ptr -0Ch
var_8           = dword ptr -8
var_4           = dword ptr -4

                push    ebp
                mov     ebp, esp
                and     esp, 0FFFFFFF0h
                sub     esp, 10h
                mov     eax, offset aADBDCD ; "a=%d; b=%d; c=%d"
                mov     [esp+10h+var_4], 3
                mov     [esp+10h+var_8], 2
                mov     [esp+10h+var_C], 1
                mov     [esp+10h+var_10], eax
                call    _printf
                mov     eax, 0
                leave
                retn
main            endp
\end{lstlisting}

\IFRU{Можно сказать, что этот короткий код, созданный GCC, отличается от кода MSVC только способом помещения 
значений в стек.
Здесь GCC снова работает со стеком напрямую без \PUSH/\POP.}
{It can be said, the difference between code by MSVC and GCC is only in method of placing arguments on the stack.
Here GCC working directly with stack without \PUSH/\POP.}

\subsection{x64}

\index{x86-64}
\RU{Всё то же самое, только используются регистры вместо стека для передачи аргументов функций}%
\EN{The picture here is similar with the difference that the registers, rather than the stack, are used for arguments passing}.

\subsubsection{MSVC}

\lstinputlisting[caption=MSVC 2012 x64]{patterns/04_scanf/1_simple/ex1_MSVC_x64.asm.\LANG}

\ifdefined\IncludeGCC
\subsubsection{GCC}

\lstinputlisting[caption=\Optimizing GCC 4.4.6 x64]{patterns/04_scanf/1_simple/ex1_GCC_x64.s.\LANG}
\fi

\subsection{ARM}

\subsubsection{\OptimizingKeil + \ThumbMode}

\begin{lstlisting}
.text:00000042             scanf_main
.text:00000042
.text:00000042             var_8           = -8
.text:00000042
.text:00000042 08 B5                       PUSH    {R3,LR}
.text:00000044 A9 A0                       ADR     R0, aEnterX     ; "Enter X:\n"
.text:00000046 06 F0 D3 F8                 BL      __2printf
.text:0000004A 69 46                       MOV     R1, SP
.text:0000004C AA A0                       ADR     R0, aD          ; "%d"
.text:0000004E 06 F0 CD F8                 BL      __0scanf
.text:00000052 00 99                       LDR     R1, [SP,#8+var_8]
.text:00000054 A9 A0                       ADR     R0, aYouEnteredD___ ; "You entered %d...\n"
.text:00000056 06 F0 CB F8                 BL      __2printf
.text:0000005A 00 20                       MOVS    R0, #0
.text:0000005C 08 BD                       POP     {R3,PC}
\end{lstlisting}

\index{\CLanguageElements!\Pointers}
\IFRU{Чтобы \scanf мог вернуть значение, ему нужно передать указатель на переменную типа \Tint.}
{A pointer
to a \Tint-typed variable must be passed to a \scanf so it can return value via it.}
\Tint \IFRU{~--- 32-битное значение, для его хранения нужно только 4 байта, и оно помещается в 
32-битный регистр.}
{is 32-bit value, so we need 4 bytes for storing it somewhere in memory, and it fits exactly 
in 32-bit register.}
\index{IDA!var\_?}
\IFRU{Место для локальной переменной \TT{x} выделяется в стеке, \IDA наименовала её \IT{var\_8}, 
впрочем, место для нее выделять не обязательно, т.к., \glslink{stack pointer}{указатель стека} \SP уже указывает на место, 
свободное для использования сразу же.}{A place for the local variable \TT{x} is allocated in the stack and \IDA
named it \IT{var\_8}, however, it is not necessary to allocate it since \SP \gls{stack pointer}
is already pointing to the space may be used instantly.}
\IFRU{Так что значение указателя \SP копируется в регистр \Reg{1}, и вместе с format-строкой, 
передается в \scanf.}
{So, \SP \gls{stack pointer} value is copied to the \Reg{1} register and, together with format-string, passed
into \scanf.}
\index{ARM!\Instructions!LDR}
\IFRU{Позже, при помощи инструкции \TT{LDR}, это значение перемещается из стека в регистр \Reg{1}, 
чтобы быть переданным в \printf.}{Later, with the help of the \TT{LDR} instruction, this value is moved
from stack into the \Reg{1} register in order to be passed into \printf.}

\IFRU{Варианты, скомпилированные для ARM-режима процессора, а также варианты скомпилированные при 
помощи Xcode LLVM, не очень отличаются от этого, так что, мы можем пропустить их здесь.}
{Examples compiled for ARM-mode and also examples compiled with Xcode LLVM are not differ significantly
from what we saw here, so they are omitted.}



\subsection{\IFRU{Глобальные переменные}{Global variables}}
\index{\IFRU{Глобальные переменные}{Global variables}}

\IFRU
{А что если переменная \TT{x} из предыдущего примера будет глобальной переменной, а не локальной? 
Тогда к ней смогут обращаться из любого другого места, а не только из тела функции. 
Глобальные переменные считаются \glslink{anti-pattern}{анти-паттерном},
но ради примера мы можем себе это позволить.}
{What if \TT{x} variable from previous example will not be local but global variable? 
Then it will be accessible from any point, not only from function body. 
Global variables are considered as \gls{anti-pattern}, but for the sake of experiment we could do this.}

\lstinputlisting{patterns/04_scanf/ex2.c}

\subsubsection{x86}

\lstinputlisting{patterns/04_scanf/ex2_MSVC.asm}

\IFRU{Ничего особенного, в целом. Теперь \TT{x} объявлена в сегменте \TT{\_DATA}. 
Память для нее в стеке более не выделяется.
Все обращения к ней происходит не через стек, а уже напрямую. 
Неинициализированные глобальные переменные не занимают места в исполняемом файле
(и действительно, зачем в исполняемом файле
нужно выделять место под изначально нулевые переменные?), но тогда, когда к этому месту в памяти
кто-то обратится, \ac{OS} подставит туда блок состоящий из нулей\footnote{Так работает \ac{VM}}.}
{Now \TT{x} variable is defined in the \TT{\_DATA} segment. 
Memory in local stack is not allocated anymore. 
All accesses to it are not via stack but directly to process memory. 
Not initialized global variables takes no place in the executable file
(indeed, why we should allocate a place
in the executable file for initially zeroed variables?), but when someone will access this place
in memory, \ac{OS} will allocate a block of zeroes there\footnote{That is how \ac{VM} behaves}.}

\IFRU{Попробуем изменить объявление этой переменной:}
{Now let's assign value to variable explicitly:}

\begin{lstlisting}
int x=10; // default value
\end{lstlisting}

\IFRU{Выйдет в итоге:}{We got:}

\begin{lstlisting}
_DATA	SEGMENT
_x	DD	0aH

...
\end{lstlisting}

\IFRU{Здесь уже по месту этой переменной записано \TT{0xA} с типом DD (dword = 32 бита).}
{Here we see value \TT{0xA} of DWORD type (DD meaning DWORD = 32 bit).}

\IFRU{Если вы откроете скомпилированный .exe-файл в \IDA, то увидите что \IT{x} 
находится аккурат в начале сегмента \TT{\_DATA}, после этой переменной будут текстовые строки.}
{If you will open compiled .exe in \IDA, you will see the \IT{x} variable placed at the beginning of 
the \TT{\_DATA} segment, and after you'll see text strings.}

\IFRU{А вот если вы откроете в \IDA, .exe скомпилированный в прошлом примере, 
где значение \IT{x} не определено, то в IDA вы увидите:}
{If you will open compiled .exe in \IDA from previous example where \IT{x} value is not defined, 
you'll see something like this:}

\begin{lstlisting}
.data:0040FA80 _x              dd ?                    ; DATA XREF: _main+10
.data:0040FA80                                         ; _main+22
.data:0040FA84 dword_40FA84    dd ?                    ; DATA XREF: _memset+1E
.data:0040FA84                                         ; unknown_libname_1+28
.data:0040FA88 dword_40FA88    dd ?                    ; DATA XREF: ___sbh_find_block+5
.data:0040FA88                                         ; ___sbh_free_block+2BC
.data:0040FA8C ; LPVOID lpMem
.data:0040FA8C lpMem           dd ?                    ; DATA XREF: ___sbh_find_block+B
.data:0040FA8C                                         ; ___sbh_free_block+2CA
.data:0040FA90 dword_40FA90    dd ?                    ; DATA XREF: _V6_HeapAlloc+13
.data:0040FA90                                         ; __calloc_impl+72
.data:0040FA94 dword_40FA94    dd ?                    ; DATA XREF: ___sbh_free_block+2FE
\end{lstlisting}

\IFRU{\TT{\_x} обозначен как \TT{?}, наряду с другими переменными не требующими инициализации. 
Это означает, что при загрузке .exe в память, место под все это выделено будет. 
Но в самом .exe ничего этого нет. Неинициализированные переменные не занимают места в исполняемых файлах. Удобно для больших массивов, например.}
{\TT{\_x} marked as \TT{?} among other variables not required to be initialized. 
This means that after loading .exe to memory, a space for all these variables will be 
allocated and a random garbage will be here. 
But in an .exe file these not initialized variables are not occupy anything. 
E.g. it is suitable for large arrays.}

\index{ELF}
\IFRU{В Linux все также почти. За исключением того, что если значение \TT{x} не определено, 
то эта переменная будет находится в сегменте \TT{\_bss}.
В \ac{ELF} этот сегмент имеет такие атрибуты:}
{It is almost the same in Linux, except segment names and properties: 
not initialized variables are located in the \TT{\_bss} segment. 
In \ac{ELF} file format this segment has such attributes:}

\begin{lstlisting}
; Segment type: Uninitialized
; Segment permissions: Read/Write
\end{lstlisting}

\IFRU{Ну а если сделать статическое присвоение этой переменной какого-либо
значения, например, $10$, то она будет находится 
в сегменте \TT{\_data},
это сегмент с такими атрибутами:}
{If to statically assign a value to variable, e.g. $10$, it will be placed in the \TT{\_data} segment, 
this is segment with the following attributes:}

\begin{lstlisting}
; Segment type: Pure data
; Segment permissions: Read/Write
\end{lstlisting}


\subsubsection{ARM: \OptimizingKeil + \ThumbMode}

\begin{lstlisting}
.text:00000000 ; Segment type: Pure code
.text:00000000                 AREA .text, CODE
...
.text:00000000 main
.text:00000000                 PUSH    {R4,LR}
.text:00000002                 ADR     R0, aEnterX     ; "Enter X:\n"
.text:00000004                 BL      __2printf
.text:00000008                 LDR     R1, =x
.text:0000000A                 ADR     R0, aD          ; "%d"
.text:0000000C                 BL      __0scanf
.text:00000010                 LDR     R0, =x
.text:00000012                 LDR     R1, [R0]
.text:00000014                 ADR     R0, aYouEnteredD___ ; "You entered %d...\n"
.text:00000016                 BL      __2printf
.text:0000001A                 MOVS    R0, #0
.text:0000001C                 POP     {R4,PC}
...
.text:00000020 aEnterX         DCB "Enter X:",0xA,0    ; DATA XREF: main+2
.text:0000002A                 DCB    0
.text:0000002B                 DCB    0
.text:0000002C off_2C          DCD x                   ; DATA XREF: main+8
.text:0000002C                                         ; main+10
.text:00000030 aD              DCB "%d",0              ; DATA XREF: main+A
.text:00000033                 DCB    0
.text:00000034 aYouEnteredD___ DCB "You entered %d...",0xA,0 ; DATA XREF: main+14
.text:00000047                 DCB 0
.text:00000047 ; .text         ends
.text:00000047
...
.data:00000048 ; Segment type: Pure data
.data:00000048                 AREA .data, DATA
.data:00000048                 ; ORG 0x48
.data:00000048                 EXPORT x
.data:00000048 x               DCD 0xA                 ; DATA XREF: main+8
.data:00000048                                         ; main+10
.data:00000048 ; .data         ends
\end{lstlisting}

Итак, переменная \TT{x} теперь глобальная, и она расположена, почему-то, в другом сегменте данных (\IT{.data}). 
Можно спросить, почему текстовые строки расположены в сегменте кода (\IT{.text}) а \TT{x} нельзя было разместить
тут же? Потому что эта переменная, и как следует из определения, она может меняться. Сегмент кода нередко может 
быть расположен в ПЗУ микроконтроллера (не забывайте, мы сейчас имеем дело с embedded-микроэлектроникой),
а изменяемые переменные --- в ОЗУ.
Нередко, ОЗУ дороже чем ПЗУ, так что хранить в нем неизменяемые данные, когда в наличии есть ПЗУ, не экономно.

Далее, мы видим, в сегменте кода, хранится указатель на переменную \TT{x} (\TT{off\_2C}) и вообще, все операции 
с ним, происходят через этот указатель.
Это связано с тем что переменная \TT{x} может быть расположена где-то довольно далеко от данного участка кода
и её адрес нужно сохранить в переменной рядом с кодом.
Инструкция \TT{LDR} в thumb-режиме может адресовать только переменные в пределах вплоть до 1020 байт от места
где она находится. Эта же инструкция в ARM-режиме --- переменные в пределах $\pm{}4095$, таким образом, 
адрес глобальной переменной \TT{x} нужно иметь где-то рядом, ведь нет никакой гарантии, что саму переменную
получится хранить где-то рядом, она может быть даже в другом чипе памяти!

Еще одна вещь: если переменную объявить как \IT{const}, то компилятор Keil разместит её в сегменте \TT{.constdata}.
Должно быть, впоследствии, линкер и этот сегмент сможет разместить в ПЗУ.



% subsection here
\subsection{\IFRU{Проверка результата scanf()}{scanf() result checking}}

\IFRU {Как я уже упоминал, использовать \scanf в наше время это слегка старомодно. 
Но если уж жизнь заставила этим заниматься, нужно хотя бы проверять, сработал ли \scanf 
правильно или пользователь ввел вместо числа что-то другое, что \scanf не смог трактовать как число.}
{As I noticed before, it is slightly old-fashioned to use \scanf today. 
But if we have to, we need at least check if \scanf finished correctly without error.}

\lstinputlisting{patterns/04_scanf/ex3.c}

\IFRU{По стандарту}{By standard}, \scanf\footnote{\href{http://msdn.microsoft.com/en-us/library/9y6s16x1(VS.71).aspx}{MSDN: scanf, wscanf}} 
\IFRU{возвращает количество успешно полученных значений.}{function returns number of fields it successfully read.}

\IFRU{В нашем случае, если все успешно и пользователь ввел таки некое число, \scanf вернет 1. 
А если нет, то 0 или EOF.} 
{In our case, if everything went fine and user entered a number, 
\scanf will return 1 or 0 or EOF in case of error.}

\IFRU{Я добавил код проверяющий результат \scanf и в случае ошибки, он сообщает пользователю что-то другое.}
{I added C code for \scanf result checking and printing error message in case of error.}

\IFRU{Это работает предсказуемо}{This works predictably}:

\begin{lstlisting}
C:\...>ex3.exe
Enter X:
123
You entered 123...

C:\...>ex3.exe
Enter X:
ouch
What you entered? Huh?
\end{lstlisting}

\subsubsection{MSVC: x86}

\IFRU{Вот, что выходит на ассемблере}{What we got in assembly language} (MSVC 2010):

\lstinputlisting{patterns/04_scanf/ex3_MSVC.asm}

\index{x86!\Registers!EAX}
\IFRU{Для того чтобы вызывающая функция имела доступ к результату вызываемой функции, 
вызываемая функция (в нашем случае \scanf) оставляет это значение в регистре \EAX.}
{\Gls{caller} function (\main) must have access to the result of \gls{callee} function (\scanf), 
so \gls{callee} leaves this value in the \EAX register.}

\index{x86!\Instructions!CMP}
\IFRU{Мы проверяем его инструкцией \TT{CMP EAX, 1} (\IT{CoMPare}), то есть, 
сравниваем значение в \EAX с 1.}
{After, we check it with the help of instruction \TT{CMP EAX, 1} (\IT{CoMPare}),
in other words, we compare value in the \EAX register with $1$.} 

\index{x86!\Instructions!JNE}
\IFRU{Следующий за инструкцией \CMP: условный переход \JNE. 
Это означает \IT{Jump if Not Equal}, то есть, условный переход \IT{если не равно}.}
{\JNE conditional jump follows \CMP instruction. \JNE means \IT{Jump if Not Equal}.}

\IFRU{Итак, если \EAX не равен 1, то \JNE заставит перейти процессор 
по адресу указанном в операнде \JNE, у нас это \TT{\$LN2@main}.}
{So, if value in the \EAX register not equals to $1$, then the processor will pass execution to the 
address mentioned in operand of \JNE, in our case it is \TT{\$LN2@main}.}
\IFRU
{Передав управление по этому адресу, \ac{CPU} как раз начнет исполнять вызов \printf с 
аргументом \TT{``What you entered? Huh?''}.}
{Passing control to this address, \ac{CPU} will execute function \printf 
with argument \TT{``What you entered? Huh?''}.}
\IFRU
{Но если все нормально, перехода не случится, и исполнится другой \printf с двумя аргументами: 
\TT{'You entered \%d...'} и значением переменной \TT{x}.}
{But if everything is fine, conditional jump will not be taken, and another \printf call 
will be executed, with two arguments: \TT{'You entered \%d...'} and value of variable \TT{x}. }

\index{x86!\Instructions!XOR}
\index{\CLanguageElements!return}
\IFRU {А для того чтобы после этого вызова не исполнился сразу второй вызов \printf, 
после него имеется инструкция \JMP, безусловный переход, он отправит процессор на место аккурат 
после второго \printf и перед инструкцией \TT{XOR EAX, EAX}, которая собственно \TT{return 0}.}
{Since second subsequent \printf not needed to be executed, there is \JMP after (unconditional jump),
it will pass control to the point after second \printf and before \TT{XOR EAX, EAX} instruction, 
which implement \TT{return 0}.}

\index{x86!\Registers!\Flags}
\IFRU{Итак, можно сказать, что в подавляющих случаях сравнение какой-либо переменной с чем-то другим 
происходит при помощи пары инструкций \CMP и \Jcc, где \IT{cc} это \IT{condition code}.}
{So, it can be said that comparing a value with another is \IT{usually} implemented
by \CMP/\Jcc instructions pair, where \IT{cc} is \IT{condition code}.}
\IFRU{\CMP сравнивает два значения и выставляет 
флаги процессора\footnote{См. также о флагах x86-процессора: \url{http://en.wikipedia.org/wiki/FLAGS_register_(computing)}.}.}
{\CMP comparing two values and set 
processor flags\footnote{About x86 flags, see also: \url{http://en.wikipedia.org/wiki/FLAGS_register_(computing)}.}.}
\IFRU
{\Jcc проверяет нужные ему флаги и выполняет переход по указанному адресу (или не выполняет).}
{\Jcc check flags needed to be checked and pass control to mentioned address (or not pass).}

\index{x86!\Instructions!CMP}
\index{x86!\Instructions!SUB}
\label{CMPandSUB}
\IFRU{Но на самом деле, как это не парадоксально поначалу звучит, \CMP это почти то же самое что и 
инструкция \SUB, которая отнимает числа одно от другого.}
{But in fact, this could be perceived paradoxical, but \CMP instruction is in fact \SUB (subtract).}
\IFRU{Все арифметические инструкции также выставляют флаги в соответствии с результатом, не только \CMP.}
{All arithmetic instructions set processor flags too, not only \CMP.}
\IFRU{Если мы сравним 1 и 1, от единицы отнимется единица, получится $0$, и выставится флаг 
\ZF (\IT{zero flag}), означающий что последний полученный результат был $0$.}
{If we compare 1 and 1, $1-1$ will be $0$ in result, \ZF flag will be set (meaning the last result was $0$).}
\IFRU{Ни при каких других значениях \EAX, флаг \ZF выставлен не будет, кроме тех, когда операнды равны друг другу.}
{There is no any other circumstances when it is possible except when operands are equal.}
\index{x86!\Instructions!JNE}
\index{x86!\Registers!ZF}
\IFRU{Инструкция \JNE проверяет только флаг \ZF, и совершает переход только если флаг не поднят. 
Фактически, \JNE это синоним инструкции \JNZ (\IT{Jump if Not Zero}).}
{\JNE checks only \ZF flag and jumping only if it is not set. 
\JNE is in fact a synonym of \JNZ (\IT{Jump if Not Zero}) instruction.}
\IFRU{Ассемблер транслирует обе инструкции в один и тот же опкод.}
{Assembler translating both \JNE and \JNZ instructions into one single opcode.}
\IFRU
{Таким образом, можно \CMP заменить на \SUB и все будет работать также, но разница в том, что \SUB 
все-таки испортит значение в первом операнде. \CMP это \IT{SUB без сохранения результата}.}
{So, \CMP instruction can be replaced to \SUB instruction and almost everything will be fine,
but the difference is in 
the \SUB alter the value of the first operand.
\CMP is \IT{``SUB without saving result''}.}

\subsubsection{GCC: x86}

\IFRU{Код созданный при помощи GCC 4.4.1 в Linux практически такой же, если не считать мелких отличий, 
которые мы уже рассмотрели раннее.}
{Code generated by GCC 4.4.1 in Linux is almost the same, except differences we already considered.}

\input{patterns/04_scanf/checking_retval_x64}
\subsubsection{ARM: \OptimizingKeil + \ThumbMode}

\lstinputlisting{04_scanf/checking_retval_ARM_Keil_thumb_O3.asm}

\IFRU{Новые инструкции здесь для нас: \CMP и \TT{BEQ}.}
{New instructions here are \CMP and \TT{BEQ}.}

\CMP \IFRU{аналогична той что в x86, она отнимает один аргумент от второго и сохраняет флаги.}
{is similar to the x86 instruction, it subtracts one argument from another and save flags.}
% TODO: в мануале ARM $op1 + NOT(op2) + 1$ вместо вычитания

\TT{BEQ} (\IT{Branch Equal}) \IFRU{совершает переход по другому адресу, 
если операнды при сравнении были равны, 
либо если результат последнего вычисления был ноль, либо если флаг Z равен $1$.}
{is jumping to another address if operands while comparing were equal to each other, or,
if result of last computation was zero, or if Z flag is $1$.}
\IFRU{То же что и \JZ в}{Same thing as \JZ in} x86.

\IFRU{Всё остальное просто: исполнение разветвляется на две ветки, затем они сходятся там, 
где в \Rzero записывается $0$ как возвращаемое из функции значение и происходит выход из функции.}
{Everything else is simple: execution flow is forking into two branches, then the branches are 
converging at the place
where $0$ is written into \Rzero, as a value returned from the function, and then function finishing.}





