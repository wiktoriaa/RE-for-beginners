\clearpage
\subsection{MSVC + \olly}
\index{\olly}

\RU{Попробуем этот же пример в}\EN{Let's try this example in} \olly.
\RU{Загружаем, нажимаем F8 (\stepover) до тех пор, пока не окажемся в своем исполняемом файле,
а не в}\EN{Let's load it and keep pressing F8 (\stepover) until we reach our executable file
instead of} \TT{ntdll.dll}.
\RU{Прокручиваем вверх до тех пор, пока не найдем \main}\EN{Scroll up until \main appears}.
\RU{Щелкаем на первой инструкции (\TT{PUSH EBP}), нажимаем F2 (\IT{set a breakpoint}), 
затем F9 (\IT{Run}) и точка останова срабатывает на начале \main.}
\EN{Click on the first instruction (\TT{PUSH EBP}), press F2 (\IT{set a breakpoint}), 
then F9 (\IT{Run}).
The breakpoint will be triggered when \main begins.}

\RU{Трассируем до того места, где готовится адрес переменной $x$}%
\EN{Let's trace to the point where the address of the variable $x$ is calculated}:

\begin{figure}[H]
\centering
\includegraphics[scale=\FigScale]{patterns/04_scanf/1_simple/ex1_olly_1.png}
\caption{\olly: \RU{вычисляется адрес локальной переменной}\EN{The address of the local variable is calculated}}
\label{fig:scanf_ex1_olly_1}
\end{figure}

\RU{На \EAX в окне регистров можно нажать правой кнопкой и далее выбрать}
\EN{Right-click the \EAX in the registers window and then select} \q{Follow in stack}.
\RU{Этот адрес покажется в окне стека.}
\EN{This address will appear in the stack window.}
\RU{Смотрите, это переменная в локальном стеке. Там дорисована красная стрелка}%
\EN{The red arrow has been added, pointing to the variable in the local stack}.
\RU{И там сейчас какой-то мусор}\EN{At that moment this location contains some garbage} (\TT{0x6E494714}).
\RU{Адрес этого элемента стека сейчас, при помощи \PUSH запишется в этот же стек рядом}%
\EN{Now with the help of \PUSH instruction the address of this stack element is going to be stored to the same stack on the next position}.
\RU{Трассируем при помощи F8 вплоть до конца исполнения \scanf}\EN{Let's trace with F8 until the \scanf execution completes}.
\RU{А пока \scanf исполняется, в консольном окне, вводим, например, 123}%
\EN{During the \scanf execution, we input, for example, 123, in the console window}:

\begin{figure}[H]
\centering
\includegraphics[scale=\NormalScale]{patterns/04_scanf/1_simple/ex1_olly_2.png}
\caption{\RU{Ввод пользователя в консольном окне}\EN{User input in the console window}}
\label{fig:scanf_ex1_olly_2}
\end{figure}

\clearpage
\RU{Вот тут }\scanf \RU{отработал}\EN{completed its execution already}:

\begin{figure}[H]
\centering
\includegraphics[scale=\FigScale]{patterns/04_scanf/1_simple/ex1_olly_3.png}
\caption{\olly: \scanf \RU{исполнилась}\EN{executed}}
\label{fig:scanf_ex1_olly_3}
\end{figure}

\scanf \RU{вернул}\EN{returns} 1 \InENRU \EAX, \RU{что означает, что он успешно прочитал одно 
значение}\EN{which implies that it has read successfully one value}.
\RU{В наблюдаемом нами элементе стека теперь}\EN{If we look again at the stack element corresponding to the local variable it now contains} \TT{0x7B} (123).

\clearpage
\RU{Чуть позже это значение копируется из стека в регистр \ECX и передается в \printf}
\EN{Later this value is copied from the stack to the \ECX register and passed to \printf}:

\begin{figure}[H]
\centering
\includegraphics[scale=\FigScale]{patterns/04_scanf/1_simple/ex1_olly_4.png}
\caption{\olly: \RU{готовим значение для передачи в}\EN{preparing the value for passing to} \printf}
\label{fig:scanf_ex1_olly_4}
\end{figure}
