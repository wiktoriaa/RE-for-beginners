\subsection{Ein einfaches Beispiel}

\lstinputlisting[style=customc]{patterns/04_scanf/1_simple/ex1.c}

Es ist nicht ratsam \scanf heutzutage noch für User Interaktionen zu verwenden. Aber dennoch können wir hier die Übergabe eines Pointers an eine Variable vom Typ \Tint betrachten.

\subsubsection{Pointer}
\myindex{\CLanguageElements!\Pointers}
Pointer sind eines der fundamentalen Konzepte in der Informatik. Oft ist das Übergeben eines großen Array, eines Structs oder Objekts als Funktionsargument zu teuer, während die Übergabe der Adresse wesentlich billiger ist. 
Wenn man zum Beispiel einen Textstring auf der Konsole ausgeben möchte, ist es deutlich einfacher, nur dessen Adresse in den Kernel des \ac{OS} zu übergeben.

Wenn die aufgerufene Funktion außerdem das große Array oder Struct verändern muss und das gesamte Object zurückgeben muss, ist die Situation beinahe absurd. 
Das einfachste ist also die Adresse eines Arrays oder Structs an die aufgerufene Funktion zu übergeben und sie dann die notwendigen Veränderungen durchführen zu lassen.

Ein Pointer ist in \CCpp nichts anderes als die Adresse einer Speicherstelle.


\myindex{x86-64}
In x86 wird die Adresse als 32-Bit-Zahl dargestellt, d.h. sie benötigt 4 Byte, während in x86-64 eine Darstellung durch 64 Bit (d.h. 8 Byte) erfolgt.
Dies ist übrigens der Grund dafür, dass einige Leute den Wechsel zu x86-64 ablehnen--alle Pointer in der x64-Architektur erfordern doppelt soviel Speicherplatz, inklusive Speicher in Cache, der ein sehr teurer Speicher ist.

\myindex{\CStandardLibrary!memcpy()}
Es ist möglich lediglich mit untypisierten Pointern zu arbeiten, wenn man ein wenig zusätzlichen Aufwand betreibt; z.B. in der Standard-C-Funktion \TT{memcpy()}, die einen Datenblock von einer Speicherstelle zu einer anderen kopiert, werden zwei Pointer vom Typ \TT{void*} als Argumente verwendet, da es nicht vorhersagbar ist, welchen Datentyp die Funktion kopieren soll. Datentypen sind hier nicht wichtig, entscheidend ist hier nur die Größe des Speicherblocks.

Pointer werden außerdem häufig verwendet, wenn eine Funktion mehr als einen Wert zurückgeben muss. (Darauf kommen wir später in ~(\myref{label_pointers}) zurück.)

Die Funktion \IT{scanf()} ist solch ein Fall: Neben der Tatsache, dass die Funktion angeben muss wie viele Werte erfolgreich gelesen wurden, muss sie auch alle diese Werte zurückliefern.

In \CCpp wird der Pointertyp nur für Typüberprüfungen zur Compilezeit benötigt.

Intern steckt im kompilierten Code keinerlei Information über die Typen der enthaltenen Pointer.

\subsection{x86}

\subsubsection{MSVC}

\RU{Что получаем на ассемблере компилируя MSVC 2010:}
\EN{What we got after compiling in MSVC 2010:}

\lstinputlisting{patterns/04_scanf/1_simple/ex1_MSVC.asm.\LANG}

\RU{Переменная \TT{x} является локальной.}\EN{Variable \TT{x} is local.} 

\RU{По стандарту \CCpp она доступна только из этой же функции и ниоткуда более. 
Так получилось, что локальные переменные располагаются в стеке. 
Может быть, можно было бы использовать и другие варианты, но в x86 это традиционно так.}
\EN{\CCpp standard tell us it must be visible only in this function and not from any other point. 
Traditionally, local variables are placed in the local stack. 
Probably, there could be other ways, but in x86 it is so.}

\index{x86!\Instructions!PUSH}
\RU{Следующая после пролога инструкция \TT{PUSH ECX} не ставит своей целью сохранить 
значение регистра \ECX. 
(Заметьте отсутствие соответствующей инструкции \TT{POP ECX} в конце функции)}
\EN{Next instruction after function prologue, \TT{PUSH ECX}, has not a goal to save \ECX state 
(notice absence of corresponding \TT{POP ECX} at the function end).}

\RU{Она на самом деле выделяет в стеке 4 байта для хранения \TT{x} в будущем.} 
\EN{In fact, this instruction just allocates 4 bytes on the stack for \TT{x} variable storage.} 

\label{stack_frame}
\index{\Stack!\RU{Стековый фрейм}\EN{Stack frame}}
\index{x86!\Registers!EBP}
\RU{Доступ к \TT{x} будет осуществляться при помощи объявленного макроса \TT{\_x\$} 
(он равен -4) и регистра \EBP указывающего на текущий фрейм.}
\EN{\TT{x} will be accessed with the assistance of the \TT{\_x\$} macro 
(it equals to -4) and the \EBP register pointing to current frame.}

\RU{Вообще, во все время исполнения функции, \EBP указывает на текущий \glslink{stack frame}{фрейм} и через \TT{EBP+смещение}
можно иметь доступ как к локальным переменным функции, так и аргументам функции.} 
\EN{Over a span of function execution, \EBP is pointing to current \gls{stack frame} and it is possible 
to have an access to local variables and function arguments via \TT{EBP+offset}.}

\index{x86!\Registers!ESP}
\RU{Можно было бы использовать \ESP, но он во время исполнения функции часто меняется, а это не удобно. 
Так что можно сказать, что \EBP это \IT{замороженное состояние} \ESP на момент начала исполнения функции.}
\EN{It is also possible to use \ESP, but since it's changing too often, it's not very convenient.
So it can be said, the value of the \EBP is \IT{frozen state} of the value of the \ESP at the moment of function execution start.}

% FIXME это уже было в 02_stack?
\RU{Разметка типичного стекового \glslink{stack frame}{фрейма} в 32-битной среде}
\EN{A very typical \gls{stack frame} layout in 32-bit environment is}:

\begin{center}
\begin{tabular}{ | l | l | }
\hline
\dots & \dots \\
\hline
EBP-8 & \RU{локальная переменная}\EN{local variable} \#2, \MarkedInIDAAs{} \TT{var\_8} \\
\hline
EBP-4 & \RU{локальная переменная}\EN{local variable} \#1, \MarkedInIDAAs{} \TT{var\_4} \\
\hline
EBP & \RU{сохраненное значение}\EN{saved value of} \EBP \\
\hline
EBP+4 & \RU{адрес возврата}\EN{return address} \\
\hline
EBP+8 & \argument \#1, \MarkedInIDAAs{} \TT{arg\_0} \\
\hline
EBP+0xC & \argument \#2, \MarkedInIDAAs{} \TT{arg\_4} \\
\hline
EBP+0x10 & \argument \#3, \MarkedInIDAAs{} \TT{arg\_8} \\
\hline
\dots & \dots \\
\hline
\end{tabular}
\end{center}

\RU{У функции \scanf в нашем примере два аргумента.}\EN{Function \scanf in our example has two arguments.}

\RU{Первый ~--- указатель на строку содержащую \TT{``\%d''} и второй ~--- адрес переменной \TT{x}.} 
\EN{First is pointer to the string containing \TT{``\%d''} and second~---address of \TT{x} variable.} 

\index{x86!\Instructions!LEA}
\RU{Вначале адрес \TT{x} помещается в регистр \EAX при помощи инструкции \TT{lea eax, DWORD PTR \_x\$[ebp]}.}
\EN{First of all, address of the \TT{x} variable is placed into the \EAX register by \TT{lea eax, DWORD PTR \_x\$[ebp]} instruction}

\RU{Инструкция \LEA означает \IT{load effective address}, и часто используется для формирования адреса чего-либо}
\EN{\LEA meaning \IT{load effective address}, and is often used for forming address of something}
~(\ref{sec:LEA}).

\RU{Можно сказать, что в данном случае \LEA просто помещает в \EAX результат суммы значения в регистре 
\EBP и макроса \TT{\_x\$}.}
\EN{It can be said, \LEA here just stores sum of the value in the \EBP register and \TT{\_x\$} macro to the \EAX register.}

\RU{Это тоже что и}\EN{It is the same as} \TT{lea eax, [ebp-4]}.

\RU{Итак, от значения \EBP отнимается $4$ и помещается в \EAX.
Далее значение \EAX заталкивается в стек и вызывается \scanf.}
\EN{So, $4$ subtracting from value in the \EBP register and result is placed to the \EAX register.
And then value in the \EAX register is pushing into stack and \scanf is called.}

\RU{После этого вызывается \printf. Первый аргумент вызова которого, строка:} 
\EN{\printf is called after that. First argument is pointer to string:} \TT{``You entered \%d...\textbackslash{}n''}.

\RU{Второй аргумент: \TT{mov ecx, [ebp-4]}, эта инструкция помещает в \ECX не адрес переменной \TT{x}, 
а его значение, что там сейчас находится.}
\EN{Second argument is prepared as: \TT{mov ecx, [ebp-4]},
this instruction places to the \ECX not address of the \TT{x} variable, but its contents.}

\RU{Далее значение \ECX заталкивается в стек и вызывается последний \printf.}
\EN{After, value in the \ECX is placed on the stack and the last \printf called.}

\ifdefined\IncludeOlly
\clearpage
\subsection{MSVC + \olly}
\index{\olly}

\RU{Попробуем этот же пример в}\EN{Let's try this example in} \olly.
\RU{Загружаем, нажимаем F8 (\stepover) до тех пор, пока не окажемся в своем исполняемом файле,
а не в}\EN{Let's load, press F8 (\stepover) until we get into our executable file
instead of} \TT{ntdll.dll}.
\RU{Скроллим вверх, до тех пока не найдем \main}\EN{Scroll up until \main appears}.
\RU{Кликаем на первой инструкции}\EN{Let's click on the first instruction} (\TT{PUSH EBP}), 
\RU{нажимаем}\EN{press} F2, \RU{затем}\EN{then} F9 (Run) \RU{и брякпойнт срабатывает
на начале \main}\EN{and breakpoint triggers on the \main begin}.

\RU{Трассируем до того места, где готовится адрес переменной $x$}
\EN{Let's trace to the place where the address of $x$ variable is prepared}:

\begin{figure}[H]
\centering
\includegraphics[scale=\FigScale]{patterns/04_scanf/1_simple/ex1_olly_1.png}
\caption{\olly: \RU{вычисляется адрес локальной переменной}\EN{address of the local variable is computed}}
\label{fig:scanf_ex1_olly_1}
\end{figure}

\RU{На \EAX в окне регистров можно нажать правой кнопкой и далее}
\EN{It is possible to right-click on \EAX in registers window and then} ``Follow in stack''.
\RU{Этот адрес покажется в окне стека}\EN{This address will appear in stack window}.
\RU{Смотрите, это переменная в локальном стеке}\EN{Look, this is a variable in the local stack}.
\RU{Я нарисовал там красную стрелку}\EN{I drawed a red arrow there}.
\RU{И там сейчас какой-то мусор}\EN{And there are some garbage} (\TT{0x77D478}).
\RU{Адрес этого элемента стека сейчас, при помощи \PUSH, запишется в этот же стек, рядом}
\EN{Now address of the stack element, with the help of \PUSH, will be written to the same stack, nearly}.
\RU{Трассируем при помощи F8 вплоть до конца исполнения \scanf}\EN{Let's trace by F8 until \scanf 
execution finished}.
\RU{А пока \scanf исполняется, в консольном окне, вводим, например, 123}
\EN{During the moment of \scanf execution, we enter, for example, 123, in the console window}:

\begin{figure}[H]
\centering
\includegraphics[scale=0.66]{patterns/04_scanf/1_simple/ex1_olly_2.png}
\caption{\RU{Вывод в консоль}\EN{Console output}}
\label{fig:scanf_ex1_olly_2}
\end{figure}

\clearpage
\RU{Вот тут }\scanf \RU{отработал}\EN{executed here}:

\begin{figure}[H]
\centering
\includegraphics[scale=\FigScale]{patterns/04_scanf/1_simple/ex1_olly_3.png}
\caption{\olly: \scanf \RU{исполнилась}\EN{executed}}
\label{fig:scanf_ex1_olly_3}
\end{figure}

\scanf \RU{вернул}\EN{returns} $1$ \InENRU \EAX, \RU{что означает, что он успешно прочитал одно 
значение}\EN{which means, it have read one value successfully}.
\RU{В наблюдаемом нами элементе стека теперь}\EN{The element of 
stack of our attention now contain} \TT{0x7B} (123).

\clearpage
\RU{Чуть позже, это значение копируется из стека в регистр \ECX и передается в \printf}
\EN{Further, this value is copied from the stack to the \ECX register and passed into \printf}:

\begin{figure}[H]
\centering
\includegraphics[scale=\FigScale]{patterns/04_scanf/1_simple/ex1_olly_4.png}
\caption{\olly: \RU{готовим значение для передачи в}\EN{preparing the value for passing into} \printf}
\label{fig:scanf_ex1_olly_4}
\end{figure}
\fi

\subsubsection{GCC}

\RU{Попробуем тоже самое скомпилировать в Linux при помощи GCC 4.4.1:}
\EN{Let's try to compile this code in GCC 4.4.1 under Linux:}

\lstinputlisting{patterns/04_scanf/1_simple/ex1_GCC.asm}

\index{puts() \RU{вместо}\EN{instead of} printf()}
\RU{GCC заменил первый вызов \printf на \puts, почему это было сделано, 
уже было описано раннее~(\ref{puts}).}
\EN{GCC replaced first the \printf call to the \puts, it was already described~(\ref{puts})
why it was done.}

% TODO: rewrite
%\RU{Почему \scanf переименовали в \TT{\_\_\_isoc99\_scanf}, я честно говоря, пока не знаю.}
%\EN{Why \scanf is renamed to \TT{\_\_\_isoc99\_scanf}, I do not know yet.}

\RU{Далее все как и прежде ~--- параметры заталкиваются через стек при помощи \MOV.}
\EN{As before~---arguments are placed on the stack by \MOV instruction.}

\subsection{x64}

\index{x86-64}
\RU{Всё то же самое, только используются регистры вместо стека для передачи аргументов функций}%
\EN{The picture here is similar with the difference that the registers, rather than the stack, are used for arguments passing}.

\subsubsection{MSVC}

\lstinputlisting[caption=MSVC 2012 x64]{patterns/04_scanf/1_simple/ex1_MSVC_x64.asm.\LANG}

\ifdefined\IncludeGCC
\subsubsection{GCC}

\lstinputlisting[caption=\Optimizing GCC 4.4.6 x64]{patterns/04_scanf/1_simple/ex1_GCC_x64.s.\LANG}
\fi

\subsection{ARM}

\subsubsection{\OptimizingKeilVI (\ThumbMode)}

\begin{lstlisting}
.text:00000042             scanf_main
.text:00000042
.text:00000042             var_8           = -8
.text:00000042
.text:00000042 08 B5                       PUSH    {R3,LR}
.text:00000044 A9 A0                       ADR     R0, aEnterX     ; "Enter X:\n"
.text:00000046 06 F0 D3 F8                 BL      __2printf
.text:0000004A 69 46                       MOV     R1, SP
.text:0000004C AA A0                       ADR     R0, aD          ; "%d"
.text:0000004E 06 F0 CD F8                 BL      __0scanf
.text:00000052 00 99                       LDR     R1, [SP,#8+var_8]
.text:00000054 A9 A0                       ADR     R0, aYouEnteredD___ ; "You entered %d...\n"
.text:00000056 06 F0 CB F8                 BL      __2printf
.text:0000005A 00 20                       MOVS    R0, #0
.text:0000005C 08 BD                       POP     {R3,PC}
\end{lstlisting}

\index{\CLanguageElements!\Pointers}
\RU{Чтобы \scanf мог вернуть значение, ему нужно передать указатель на переменную типа \Tint.}
\EN{In order for \scanf to be able to read item it needs a parameter---pointer to an \Tint.}
\Tint\RU{~--- 32-битное значение, для его хранения нужно только 4 байта, и оно помещается в 
32-битный регистр.}
\EN{is 32-bit, so we need 4 bytes to store it somewhere in memory, and it fits exactly 
in a 32-bit register.}
\index{IDA!var\_?}
\RU{Место для локальной переменной \TT{x} выделяется в стеке, \IDA наименовала её \IT{var\_8}. 
Впрочем, место для неё выделять не обязательно, т.к. \glslink{stack pointer}{указатель стека} \ac{SP} уже указывает на место, 
свободное для использования.}\EN{A place for the local variable \TT{x} is allocated in the stack and \IDA
has named it \IT{var\_8}. It is not necessary, however, to allocate a such since \ac{SP} (\gls{stack pointer}) is already pointing to that space and it can be used directly.}
\RU{Так что значение указателя \ac{SP} копируется в регистр \Reg{1}, и вместе с format-строкой, 
передается в \scanf.}
\EN{So, \ac{SP}'s value is copied to the \Reg{1} register and, together with the format-string, passed
to \scanf.}
\index{ARM!\Instructions!LDR}
\RU{Позже, при помощи инструкции \TT{LDR}, это значение перемещается из стека в регистр \Reg{1}, 
чтобы быть переданным в \printf.}\EN{Later, with the help of the \TT{LDR} instruction, this value is moved
from the stack to the \Reg{1} register in order to be passed to \printf.}

\subsubsection{ARM64}

\lstinputlisting[caption=\NonOptimizing GCC 4.9.1 ARM64,numbers=left]{patterns/04_scanf/1_simple/ARM64_GCC491_O0.s.\LANG}

\RU{Под стековый фрейм выделяется 32 байта, что больше чем нужно. Вероятно, это связано с выравниваем по границе памяти?}%
\EN{There is 32 bytes are allocated for stack frame, which is bigger than it needed. Perhaps, some memory aligning issue?}
\RU{Самая интересная часть~--- это поиск места под переменную $x$ в стековом фрейме (строка 22).}
\EN{The most interesting part is finding space for the $x$ variable in the stack frame (line 22).}
\RU{Почему 28? Почему-то, компилятор решил расположить эту переменную в конце стекового фрейма, а не в начале.}%
\EN{Why 28? Somehow, compiler decided to place this variable at the end of stack frame instead of beginning.}
\RU{Адрес потом передается в \scanf, которая просто сохраняет значение, введенное пользователем, в памяти
по этому адресу.}
\EN{The address is passed to \scanf, which just stores the user input value in the memory at that address.}
\RU{Это 32-битное значение типа \Tint}\EN{This is 32-bit value of type \Tint}.
\RU{Значение загружается в строке 27 и затем передается в \printf.}
\EN{The value is fetched at line 27 and then passed to \printf.}


\ifx\RUSSIAN\undefined
\subsection{MIPS}

A place in the local stack is allocated for $x$ value (will be referred as $\$sp+24$).
\index{MIPS!\Instructions!LW}
It's address is passed to scanf(), a value read from user is loaded using LW (``Load Word'') instruction
and then is's passed to \printf.

\lstinputlisting[caption=\Optimizing GCC 4.4.5 (\assemblyOutput)]{patterns/04_scanf/1_simple/MIPS/ex1.O3.s}

IDA does stack layout:

\lstinputlisting[caption=\Optimizing GCC 4.4.5 (IDA)]{patterns/04_scanf/1_simple/MIPS/ex1.O3.IDA.lst}

% TODO non-optimized version?
\fi


