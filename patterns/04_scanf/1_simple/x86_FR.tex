\subsubsection{x86}

\myparagraph{MSVC}

Voici ce que l'on obtient après avoir compilé avec MSVC 2010:

\lstinputlisting[style=customasmx86]{patterns/04_scanf/1_simple/ex1_MSVC_FR.asm}

\TT{x} est une variable locale.

D'après le standard \CCpp elle ne doit être visible que dans cette fonction et dans
aucune autre portée.
Traditionnellement, les variables locales sont stockées dans la pile.
Il y a probablement d'autres moyens de les allouer, mais en x86, c'est la façon de faire.

\myindex{x86!\Instructions!PUSH}
Le but de l'instruction suivant le prologue de la fonction, \TT{PUSH ECX}, n'est
pas de sauver l'état de \ECX (noter l'absence d'un \TT{POP ECX} à la fin de la
fonction).

En fait, cela alloue 4 octets sur la pile pour stocker la variable \TT{x}.

\label{stack_frame}
\myindex{\Stack!Stack frame}
\myindex{x86!\Registers!EBP}
\TT{x} est accèdée à l'aide de la macro \TT{\_x\$} (qui vaut -4) et du registre \EBP
qui pointe sur la structure de pile courante.

Pendant la durée de l'exécution de la fonction, \EBP pointe sur la \glslink{stack frame}{structure locale de pile}
courante, rendant possible l'accès aux variables locales et aux arguments de la
fonction via \TT{EBP+offset}.

\myindex{x86!\Registers!ESP}
Il est aussi possible d'utiliser \ESP dans le même but, bien que ça ne soit pas
très commode, car il change fréquemment.
La valeur de \EBP peut être perçue comme un \IT{état figé} de la valeur de \ESP
au début de l'exécution de la fonction.

% FIXME1 это уже было в 02_stack?
Voici une \glslink{stack frame}{strucutre de pile} typique dans un environnement 32-bit:

\begin{center}
\begin{tabular}{ | l | l | }
\hline
\dots & \dots \\
\hline
EBP-8 & variable locale \#2, \MarkedInIDAAs{} \TT{var\_8} \\
\hline
EBP-4 & variable locale \#1, \MarkedInIDAAs{} \TT{var\_4} \\
\hline
EBP & valeur sauvée de \EBP \\
\hline
EBP+4 & adresse de retour \\
\hline
EBP+8 & \argument \#1, \MarkedInIDAAs{} \TT{arg\_0} \\
\hline
EBP+0xC & \argument \#2, \MarkedInIDAAs{} \TT{arg\_4} \\
\hline
EBP+0x10 & \argument \#3, \MarkedInIDAAs{} \TT{arg\_8} \\
\hline
\dots & \dots \\
\hline
\end{tabular}
\end{center}

La fonction \scanf de notre exemple a deux arguments.

Le premier est un pointeur sur la chaîne contenant \TT{\%d} et le second est l'adresse
de la variable \TT{x}.

\myindex{x86!\Instructions!LEA}
Tout d'abord, l'adresse de la variable \TT{x} est chargée dans le registre \EAX
par l'instruction \\ \TT{lea eax, DWORD PTR \_x\$[ebp]}.

\LEA signifie \IT{load effective address} (charger l'adresse effective) et est souvent
utilisée pour composer une adresse ~(\myref{sec:LEA}).

Nous pouvons dire que dans ce cas, \LEA stocke simplement la somme de la valeur du
registre \EBP et de la macro \TT{\_x\$} dans le registre \EAX.

C'est la même chose que \INS{lea eax, [ebp-4]}.

Donc, 4 est soustrait de la valeur du registre \EBP et le résultat est chargé dans
le registre \EAX.
Ensuite, la valeur du registre \EAX est poussée sur la pile et \scanf est appelée.

\printf est appelée ensuite avec son premier argument --- un pointeur sur la chaîne:
\TT{You entered \%d...\textbackslash{}n}.

Le second argument est préparé avec: \TT{mov ecx, [ebp-4]}.
L'instruction stocke la valeur de la variable \TT{x} et non son adresse, dans le
registre \ECX.

Puis, la valeur de \ECX est stockée sur la pile et le dernier appel à \printf
est effectué.

\clearpage
\subsection{MSVC + \olly}
\index{\olly}

\RU{Попробуем этот же пример в}\EN{Let's try this example in} \olly.
\RU{Загружаем, нажимаем F8 (\stepover) до тех пор, пока не окажемся в своем исполняемом файле,
а не в}\EN{Let's load, press F8 (\stepover) until we get into our executable file
instead of} \TT{ntdll.dll}.
\RU{Скроллим вверх, до тех пока не найдем \main}\EN{Scroll up until \main appears}.
\RU{Кликаем на первой инструкции}\EN{Let's click on the first instruction} (\TT{PUSH EBP}), 
\RU{нажимаем}\EN{press} F2, \RU{затем}\EN{then} F9 (Run) \RU{и брякпойнт срабатывает
на начале \main}\EN{and breakpoint triggers on the \main begin}.

\RU{Трассируем до того места, где готовится адрес переменной $x$}
\EN{Let's trace to the place where the address of $x$ variable is prepared}:

\begin{figure}[H]
\centering
\includegraphics[scale=\FigScale]{patterns/04_scanf/1_simple/ex1_olly_1.png}
\caption{\olly: \RU{вычисляется адрес локальной переменной}\EN{address of the local variable is computed}}
\label{fig:scanf_ex1_olly_1}
\end{figure}

\RU{На \EAX в окне регистров можно нажать правой кнопкой и далее}
\EN{It is possible to right-click on \EAX in registers window and then} ``Follow in stack''.
\RU{Этот адрес покажется в окне стека}\EN{This address will appear in stack window}.
\RU{Смотрите, это переменная в локальном стеке}\EN{Look, this is a variable in the local stack}.
\RU{Я нарисовал там красную стрелку}\EN{I drew a red arrow there}.
\RU{И там сейчас какой-то мусор}\EN{And there are some garbage} (\TT{0x6E494714}).
\RU{Адрес этого элемента стека сейчас, при помощи \PUSH, запишется в этот же стек, рядом}
\EN{Now address of the stack element, with the help of \PUSH, will be written to the same stack, nearly}.
\RU{Трассируем при помощи F8 вплоть до конца исполнения \scanf}\EN{Let's trace by F8 until \scanf 
execution finished}.
\RU{А пока \scanf исполняется, в консольном окне, вводим, например, 123}
\EN{During the moment of \scanf execution, we enter, for example, 123, in the console window}:

\begin{figure}[H]
\centering
\includegraphics[scale=\NormalScale]{patterns/04_scanf/1_simple/ex1_olly_2.png}
\caption{\RU{Вывод в консоль}\EN{Console output}}
\label{fig:scanf_ex1_olly_2}
\end{figure}

\clearpage
\RU{Вот тут }\scanf \RU{отработал}\EN{executed here}:

\begin{figure}[H]
\centering
\includegraphics[scale=\FigScale]{patterns/04_scanf/1_simple/ex1_olly_3.png}
\caption{\olly: \scanf \RU{исполнилась}\EN{executed}}
\label{fig:scanf_ex1_olly_3}
\end{figure}

\scanf \RU{вернул}\EN{returns} $1$ \InENRU \EAX, \RU{что означает, что он успешно прочитал одно 
значение}\EN{which means, it have read one value successfully}.
\RU{В наблюдаемом нами элементе стека теперь}\EN{The element of 
stack of our attention now contain} \TT{0x7B} (123).

\clearpage
\RU{Чуть позже, это значение копируется из стека в регистр \ECX и передается в \printf}
\EN{Further, this value is copied from the stack to the \ECX register and passed into \printf}:

\begin{figure}[H]
\centering
\includegraphics[scale=\FigScale]{patterns/04_scanf/1_simple/ex1_olly_4.png}
\caption{\olly: \RU{готовим значение для передачи в}\EN{preparing the value for passing into} \printf}
\label{fig:scanf_ex1_olly_4}
\end{figure}


\myparagraph{GCC}

Compilons ce code avec GCC 4.4.1 sous Linux:

\lstinputlisting[style=customasmx86]{patterns/04_scanf/1_simple/ex1_GCC.asm}

\myindex{puts() instead of printf()}
GCC a remplacé l'appel à \printf avec un appel à \puts. La raison de cela a été
expliquée dans ~(\myref{puts}).

% TODO: rewrite
%\RU{Почему \scanf переименовали в \TT{\_\_\_isoc99\_scanf}, я честно говоря, пока не знаю.}
%\EN{Why \scanf is renamed to \TT{\_\_\_isoc99\_scanf}, I do not know yet.}
% 
% Apparently it has to do with the ISO c99 standard compliance. By default GCC allows specifying a standard to adhere to.
% For example if you compile with -std=c89 the outputted assmebly file will contain scanf and not __isoc99__scanf. I guess current GCC version adhares to c99 by default.
% According to my understanding the two implementations differ in the set of suported modifyers (See printf man page)

Comme dans l'exemple avec MSVC---les arguments sont placés dans la pile avec l'instruction
\MOV.

\myparagraph{By the way}

Á propos, ce simple exemple est la démonstration du fait que le compilateur traduit
une liste d'expression en bloc-\CCpp en une liste séquentielle d'instructions.
% TODO FIXME: better translation / clarify ?
Il n'y a rien entre les expressions en \CCpp, et le résultat en code machine,
il n'y a rien entre le déroulement du flux de contrôle d'une expression à la suivante.
