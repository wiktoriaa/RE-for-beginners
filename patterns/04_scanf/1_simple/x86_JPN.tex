\subsubsection{x86}

\myparagraph{MSVC}

MSVC 2010でコンパイルした後に得られるものは次のとおりです。

\lstinputlisting[style=customasmx86]{patterns/04_scanf/1_simple/ex1_MSVC_JPN.asm}

\TT{x}はローカル変数です。

\CCpp 標準によれば、この関数でのみ表示でき、他の外部スコープでは表示できません。
従来、ローカル変数はスタックに格納されていました。
それらを割り当てる方法はおそらく他にもありますが、それはx86の方法です。

\myindex{x86!\Instructions!PUSH}
関数プロローグ、\TT{PUSH ECX}に続く命令の目的は、 \ECX 状態を保存することではありません
(関数の最後に対応する\TT{POP ECX}が存在しないことに注意してください)。

実際、\TT{x}変数を格納するためにスタックに4バイトを割り当てます。

\label{stack_frame}
\myindex{\Stack!Stack frame}
\myindex{x86!\Registers!EBP}
\TT{x}は、\TT{\_x\$} マクロ (-4に等しい)と現在のフレームを指す \EBP レジスタの助けを借りてアクセスされます。

関数の実行の範囲にわたって、 \EBP は現在の\gls{stack frame}を指しており、 \TT{EBP+オフセット}
を介してローカル変数と関数引数にアクセスすることができます。

\myindex{x86!\Registers!ESP}
同じ目的で \ESP を使用することもできますが、 \ESP は頻繁に変更されるためあまり便利ではありません。 
\EBP の値は、関数の実行開始時に \ESP の値が固定された状態として認識される可能性があります。

% FIXME1 это уже было в 02_stack?
32ビット環境での典型的な\gls{stack frame}レイアウトを次に示します。

\begin{center}
\begin{tabular}{ | l | l | }
\hline
\dots & \dots \\
\hline
EBP-8 & local variable \#2, \MarkedInIDAAs{} \TT{var\_8} \\
\hline
EBP-4 & local variable \#1, \MarkedInIDAAs{} \TT{var\_4} \\
\hline
EBP & saved value of \EBP \\
\hline
EBP+4 & return address \\
\hline
EBP+8 & \argument \#1, \MarkedInIDAAs{} \TT{arg\_0} \\
\hline
EBP+0xC & \argument \#2, \MarkedInIDAAs{} \TT{arg\_4} \\
\hline
EBP+0x10 & \argument \#3, \MarkedInIDAAs{} \TT{arg\_8} \\
\hline
\dots & \dots \\
\hline
\end{tabular}
\end{center}

この例の \scanf 関数には2つの引数があります。

最初のものは\TT{\%d}を含む文字列へのポインタで、2番目のものは\TT{x}変数のアドレスです。

\myindex{x86!\Instructions!LEA}
最初に、\TT{x}変数のアドレスが\TT{lea eax, DWORD PTR \_x\$[ebp]}命令によって \EAX レジスタにロードされます。

\LEA は\IT{ロード実効アドレス}の略で、アドレスを形成するためによく使用されます(~(\myref{sec:LEA}))。

この場合、\LEA は単に \EBP レジスタ値と\TT{\_x\$}マクロの合計を \EAX レジスタに格納すると言うことができます。

これは\INS{lea eax, [ebp-4]}と同じです。

したがって、 \EBP レジスタ値から4が減算され、その結果が \EAX レジスタにロードされます。
次に、 \EAX レジスタの値がスタックにプッシュされ、 \scanf が呼び出されます。

\printf は最初の引数で呼び出されています。文字列へのポインタ:
\TT{You entered \%d...\textbackslash{}n}

2番目の引数は\TT{mov ecx, [ebp-4]}で準備されています。
命令は、 \ECX レジスタにそのアドレスではなく\TT{x}変数値を格納します。

次に、 \ECX の値がスタックに格納され、最後の \printf が呼び出されます。

\clearpage
\subsection{MSVC + \olly}
\index{\olly}

\RU{Попробуем этот же пример в}\EN{Let's try this example in} \olly.
\RU{Загружаем, нажимаем F8 (\stepover) до тех пор, пока не окажемся в своем исполняемом файле,
а не в}\EN{Let's load, press F8 (\stepover) until we get into our executable file
instead of} \TT{ntdll.dll}.
\RU{Скроллим вверх, до тех пока не найдем \main}\EN{Scroll up until \main appears}.
\RU{Кликаем на первой инструкции}\EN{Let's click on the first instruction} (\TT{PUSH EBP}), 
\RU{нажимаем}\EN{press} F2, \RU{затем}\EN{then} F9 (Run) \RU{и брякпойнт срабатывает
на начале \main}\EN{and breakpoint triggers on the \main begin}.

\RU{Трассируем до того места, где готовится адрес переменной $x$}
\EN{Let's trace to the place where the address of $x$ variable is prepared}:

\begin{figure}[H]
\centering
\includegraphics[scale=\FigScale]{patterns/04_scanf/1_simple/ex1_olly_1.png}
\caption{\olly: \RU{вычисляется адрес локальной переменной}\EN{address of the local variable is computed}}
\label{fig:scanf_ex1_olly_1}
\end{figure}

\RU{На \EAX в окне регистров можно нажать правой кнопкой и далее}
\EN{It is possible to right-click on \EAX in registers window and then} ``Follow in stack''.
\RU{Этот адрес покажется в окне стека}\EN{This address will appear in stack window}.
\RU{Смотрите, это переменная в локальном стеке}\EN{Look, this is a variable in the local stack}.
\RU{Я нарисовал там красную стрелку}\EN{I drew a red arrow there}.
\RU{И там сейчас какой-то мусор}\EN{And there are some garbage} (\TT{0x6E494714}).
\RU{Адрес этого элемента стека сейчас, при помощи \PUSH, запишется в этот же стек, рядом}
\EN{Now address of the stack element, with the help of \PUSH, will be written to the same stack, nearly}.
\RU{Трассируем при помощи F8 вплоть до конца исполнения \scanf}\EN{Let's trace by F8 until \scanf 
execution finished}.
\RU{А пока \scanf исполняется, в консольном окне, вводим, например, 123}
\EN{During the moment of \scanf execution, we enter, for example, 123, in the console window}:

\begin{figure}[H]
\centering
\includegraphics[scale=\NormalScale]{patterns/04_scanf/1_simple/ex1_olly_2.png}
\caption{\RU{Вывод в консоль}\EN{Console output}}
\label{fig:scanf_ex1_olly_2}
\end{figure}

\clearpage
\RU{Вот тут }\scanf \RU{отработал}\EN{executed here}:

\begin{figure}[H]
\centering
\includegraphics[scale=\FigScale]{patterns/04_scanf/1_simple/ex1_olly_3.png}
\caption{\olly: \scanf \RU{исполнилась}\EN{executed}}
\label{fig:scanf_ex1_olly_3}
\end{figure}

\scanf \RU{вернул}\EN{returns} $1$ \InENRU \EAX, \RU{что означает, что он успешно прочитал одно 
значение}\EN{which means, it have read one value successfully}.
\RU{В наблюдаемом нами элементе стека теперь}\EN{The element of 
stack of our attention now contain} \TT{0x7B} (123).

\clearpage
\RU{Чуть позже, это значение копируется из стека в регистр \ECX и передается в \printf}
\EN{Further, this value is copied from the stack to the \ECX register and passed into \printf}:

\begin{figure}[H]
\centering
\includegraphics[scale=\FigScale]{patterns/04_scanf/1_simple/ex1_olly_4.png}
\caption{\olly: \RU{готовим значение для передачи в}\EN{preparing the value for passing into} \printf}
\label{fig:scanf_ex1_olly_4}
\end{figure}


\myparagraph{GCC}

Linux上のGCC 4.4.1でこのコードをコンパイルしようとしましょう。

\lstinputlisting[style=customasmx86]{patterns/04_scanf/1_simple/ex1_GCC.asm}

\myindex{puts() instead of printf()}
GCCは \printf 呼び出しを \puts の呼び出しで置き換えました。 この理由は、~(\myref{puts})で説明されました。

% TODO: rewrite
%\RU{Почему \scanf переименовали в \TT{\_\_\_isoc99\_scanf}, я честно говоря, пока не знаю.}
%\EN{Why \scanf is renamed to \TT{\_\_\_isoc99\_scanf}, I do not know yet.}
% 
% Apparently it has to do with the ISO c99 standard compliance. By default GCC allows specifying a standard to adhere to.
% For example if you compile with -std=c89 the outputted assmebly file will contain scanf and not __isoc99__scanf. I guess current GCC version adhares to c99 by default.
% According to my understanding the two implementations differ in the set of suported modifyers (See printf man page)

MSVCの例のように、引数は \MOV 命令を使用してスタックに配置されます。

\myparagraph{ところで}

ところで、この単純な例は、コンパイラが \CCpp ブロックの式のリストを命令の連続したリストに
変換するという事実のデモンストレーションです。
\CCpp の式の間には何もないので、結果のマシンコードには、
ある式から次の式への制御フローの間には何もありません。
