\subsubsection{x86}

\myparagraph{MSVC}

Tutaj znajduje się wynik kompilacji programu w MSVC 2010:

\lstinputlisting[style=customasmx86]{patterns/04_scanf/1_simple/ex1_MSVC_EN.asm}

\TT{x} jest zmienną lokalną.

Według standardu \CCpp zmienna lokalna może być widoczna tylko w konkretnej funkcji. Tradycyjnie zmienne lokalne są przechowywane na stosie. Prawdopodnie są inne moliwości przechowywania tych zmiennych, ale tak akurat jest w x86.

\myindex{x86!\Instructions!PUSH}
Zadaniem instrukcji rozpoczynającej funkcję, \TT{PUSH ECX}, nie jest zapisanie stanu \ECX (można zauważyć brak odpowiadającej instrukcji POP ECX na końcu funkcji).

Tak naprawdę instrukcja ta alokuje 4 bajty na stosie do przechowania zmiennej x.

\label{stack_frame}
\myindex{\Stack!Stack frame}
\myindex{x86!\Registers!EBP}
Dostęp do \TT{x} odbywa się z asystującym makrem \TT{\_x\$} (równym -4) i rejestrem \EBP rwskazującym bieżącą ramkę.

We fragmencie wykonującym funkcje, \EBP wskazuje bieżącą \gls{stack frame}
umożliwiając dostęp do zmiennych lokalnych i argumentów funkcji poprzez \TT{EBP+offset}.

\myindex{x86!\Registers!ESP}
Możliwe jest także użycie \ESP w takim samym celu, le nie jest to zbyt wygodne, ponieważ wartość tego rejestru często się zmienia.
Wartość \EBP może być postrzegana jako \IT{frozen state} wartości w \ESP z początku wykonania funkcji.

% FIXME1 это уже было в 02_stack?
Tutaj znajduje się typowa ramka stosu w układzie środowiska 32-bitowego:

\begin{center}
\begin{tabular}{ | l | l | }
\hline
\dots & \dots \\
\hline
EBP-8 & zmienna lokalna \#2, \MarkedInIDAAs{} \TT{var\_8} \\
\hline
EBP-4 & zmienna lokalna \#1, \MarkedInIDAAs{} \TT{var\_4} \\
\hline
EBP & zapisana wartość \EBP \\
\hline
EBP+4 & adres powrotu \\
\hline
EBP+8 & \argument \#1, \MarkedInIDAAs{} \TT{arg\_0} \\
\hline
EBP+0xC & \argument \#2, \MarkedInIDAAs{} \TT{arg\_4} \\
\hline
EBP+0x10 & \argument \#3, \MarkedInIDAAs{} \TT{arg\_8} \\
\hline
\dots & \dots \\
\hline
\end{tabular}
\end{center}

Funkcja \scanf w naszym przykładzie ma dwa argumenty.

Pierwszy jest wskaźnikiem na string \TT{\%d} a drugi jest adresem zmiennej \TT{x}.

\myindex{x86!\Instructions!LEA}
Na początku adres zmiennej \TT{x} jest ładowany do rejestru \EAX przy pomocy instrukcji \\
\TT{lea eax, DWORD PTR \_x\$[ebp]}.

\LEA oznacza \IT{load effective address} i jest często używana do formowania adresów ~(\myref{sec:LEA}).

Można powiedzieć, że w tym przypadku \LEA po prostu umieszcza sumę rejestru \EBP i makra \TT{\_x\$} w rejestrze \EAX.

To jest to samo co \INS{lea eax, [ebp-4]}.

Więc od rejestru \EBP jest odejmowane 4 i wynik zostaje umieszczony w rejestrze \EAX.
Następnie wartość rejestru \EAX jest odkładana na stosie i funkcja \scanf zostaje wywołana.

\printf wywołuje się z pierwszym argumentem- wskaźnikiem na string:
\TT{You entered \%d...\textbackslash{}n}.

Drugi argument jest przygotowywany za pomocą: \TT{mov ecx, [ebp-4]}.
Instrukcja kopiuje zmienną \TT{x} (nie jej adres) do rejestru \ECX.

Następnie wartość z \ECX jest odkładana na stos, a na koniec zostaje wywołana funkcja  \printf.

\clearpage
\subsection{MSVC + \olly}
\index{\olly}

\RU{Попробуем этот же пример в}\EN{Let's try this example in} \olly.
\RU{Загружаем, нажимаем F8 (\stepover) до тех пор, пока не окажемся в своем исполняемом файле,
а не в}\EN{Let's load, press F8 (\stepover) until we get into our executable file
instead of} \TT{ntdll.dll}.
\RU{Скроллим вверх, до тех пока не найдем \main}\EN{Scroll up until \main appears}.
\RU{Кликаем на первой инструкции}\EN{Let's click on the first instruction} (\TT{PUSH EBP}), 
\RU{нажимаем}\EN{press} F2, \RU{затем}\EN{then} F9 (Run) \RU{и брякпойнт срабатывает
на начале \main}\EN{and breakpoint triggers on the \main begin}.

\RU{Трассируем до того места, где готовится адрес переменной $x$}
\EN{Let's trace to the place where the address of $x$ variable is prepared}:

\begin{figure}[H]
\centering
\includegraphics[scale=\FigScale]{patterns/04_scanf/1_simple/ex1_olly_1.png}
\caption{\olly: \RU{вычисляется адрес локальной переменной}\EN{address of the local variable is computed}}
\label{fig:scanf_ex1_olly_1}
\end{figure}

\RU{На \EAX в окне регистров можно нажать правой кнопкой и далее}
\EN{It is possible to right-click on \EAX in registers window and then} ``Follow in stack''.
\RU{Этот адрес покажется в окне стека}\EN{This address will appear in stack window}.
\RU{Смотрите, это переменная в локальном стеке}\EN{Look, this is a variable in the local stack}.
\RU{Я нарисовал там красную стрелку}\EN{I drew a red arrow there}.
\RU{И там сейчас какой-то мусор}\EN{And there are some garbage} (\TT{0x6E494714}).
\RU{Адрес этого элемента стека сейчас, при помощи \PUSH, запишется в этот же стек, рядом}
\EN{Now address of the stack element, with the help of \PUSH, will be written to the same stack, nearly}.
\RU{Трассируем при помощи F8 вплоть до конца исполнения \scanf}\EN{Let's trace by F8 until \scanf 
execution finished}.
\RU{А пока \scanf исполняется, в консольном окне, вводим, например, 123}
\EN{During the moment of \scanf execution, we enter, for example, 123, in the console window}:

\begin{figure}[H]
\centering
\includegraphics[scale=\NormalScale]{patterns/04_scanf/1_simple/ex1_olly_2.png}
\caption{\RU{Вывод в консоль}\EN{Console output}}
\label{fig:scanf_ex1_olly_2}
\end{figure}

\clearpage
\RU{Вот тут }\scanf \RU{отработал}\EN{executed here}:

\begin{figure}[H]
\centering
\includegraphics[scale=\FigScale]{patterns/04_scanf/1_simple/ex1_olly_3.png}
\caption{\olly: \scanf \RU{исполнилась}\EN{executed}}
\label{fig:scanf_ex1_olly_3}
\end{figure}

\scanf \RU{вернул}\EN{returns} $1$ \InENRU \EAX, \RU{что означает, что он успешно прочитал одно 
значение}\EN{which means, it have read one value successfully}.
\RU{В наблюдаемом нами элементе стека теперь}\EN{The element of 
stack of our attention now contain} \TT{0x7B} (123).

\clearpage
\RU{Чуть позже, это значение копируется из стека в регистр \ECX и передается в \printf}
\EN{Further, this value is copied from the stack to the \ECX register and passed into \printf}:

\begin{figure}[H]
\centering
\includegraphics[scale=\FigScale]{patterns/04_scanf/1_simple/ex1_olly_4.png}
\caption{\olly: \RU{готовим значение для передачи в}\EN{preparing the value for passing into} \printf}
\label{fig:scanf_ex1_olly_4}
\end{figure}


\myparagraph{GCC}

Tak wygląda skompilowany kod w GCC 4.4.1 w systemie Linux:

\lstinputlisting[style=customasmx86]{patterns/04_scanf/1_simple/ex1_GCC.asm}

\myindex{puts() instead of printf()}
GCC zamienia wywołanie funkcji \printf na wywołanie funkcji \puts. Powód tego został wyjaśniony w ~(\myref{puts}).

% TODO: rewrite
%\RU{Почему \scanf переименовали в \TT{\_\_\_isoc99\_scanf}, я честно говоря, пока не знаю.}
%\EN{Why \scanf is renamed to \TT{\_\_\_isoc99\_scanf}, I do not know yet.}
% 
% Apparently it has to do with the ISO c99 standard compliance. By default GCC allows specifying a standard to adhere to.
% For example if you compile with -std=c89 the outputted assmebly file will contain scanf and not __isoc99__scanf. I guess current GCC version adhares to c99 by default.
% According to my understanding the two implementations differ in the set of suported modifyers (See printf man page)

Jak w przykładzie MSVC---argumenty funkcji są umieszczane na stosie przy użyciu instrukcji \MOV.

\myparagraph{By the way}

Ten prosty przykład pokazuje jak faktycznie kompilatory tłumaczą
listy wyrażeń w \CCpp-block na sekwencyjne listy instrukcji.
Nie ma nic pomiędzy wyrażeniami w \CCpp a wynikowym kodem maszynowym.
