\subsection{x86}

\subsubsection{MSVC}

\RU{Что получаем на ассемблере, компилируя в MSVC 2010:}%
\EN{Here is what we get after compiling with MSVC 2010:}%
\PTBR{Aqui está o que o resultado depois de se compilar com o MSVC 2010:}%

\lstinputlisting{patterns/04_scanf/1_simple/ex1_MSVC.asm.\LANG}

\RU{Переменная \TT{x} является локальной.}\EN{\TT{x} is a local variable.}\PTBR{\TT{x} é uma variável local.}

\RU{По стандарту \CCpp она доступна только из этой же функции и нигде более. 
Так получилось, что локальные переменные располагаются в стеке. 
Может быть, можно было бы использовать и другие варианты, но в x86 это традиционно так.}%
\EN{According to the \CCpp standard it must be visible only in this function and not from any other external scope. 
Traditionally, local variables are stored on the stack. 
There are probably other ways to allocate them, but in x86 that is the way it is.}%
\PTBR{De acordo com os padrões de \CCpp ela só deve ser visível nessa função e não além dela.
Tradicionalmente, variáveis locais são guardadas na pilha.
Provavelmente há outras maneiras de alocá-las, mas no x86 é assim que é feito.}%

\index{x86!\Instructions!PUSH}
\RU{Следующая после пролога инструкция \TT{PUSH ECX} не ставит своей целью сохранить 
значение регистра \ECX. 
(Заметьте отсутствие соответствующей инструкции \TT{POP ECX} в конце функции).}%
\EN{The goal of the instruction following the function prologue, \TT{PUSH ECX}, is not to save the \ECX state 
(notice the absence of corresponding \TT{POP ECX} at the function's end).}%
\PTBR{O objetivo da intrução que se segue após o cabeçalho da função, \INS{PUSH ECX},
não é para salvar o valor de \ECX
(perceba que não há a instrução \INS{POP ECX} no fim da função).}%

\RU{Она на самом деле выделяет в стеке 4 байта для хранения \TT{x} в будущем.}%
\EN{In fact it allocates 4 bytes on the stack for storing the \TT{x} variable.}%
\PTBR{Na verdade, esse PUSH aloca 4 bytes na pilha para guardar a variável \TT{x}.}%

\label{stack_frame}
\index{\Stack!\RU{Стековый фрейм}\EN{Stack frame}}
\index{x86!\Registers!EBP}
\RU{Доступ к \TT{x} будет осуществляться при помощи объявленного макроса \TT{\_x\$} (он равен -4) и регистра \EBP указывающего на текущий фрейм.}%
\EN{\TT{x} is to be accessed with the assistance of the \TT{\_x\$} macro (it equals to -4) and the \EBP register pointing to the current frame.}%
\PTBR{\TT{x} é para ser acessada com a ajuda do macro \TT{\_x\$} (que é igual a -4) e o registrador \EBP apontando para a posição atual.}%

\RU{Во всё время исполнения функции \EBP указывает на текущий \glslink{stack frame}{фрейм} и через \TT{EBP+смещение}
можно получить доступ как к локальным переменным функции, так и аргументам функции.}%
\EN{Over the span of the function's execution, \EBP is pointing to the current \gls{stack frame}
making it possible to access local variables and function arguments via \TT{EBP+offset}.}%
\PTBR{Conforme a execução da função avança, \EBP está apontando para a posição atual da pilha,
sendo possível acessar variáveis locais e argumentos da função via \TT{EBP+offset}.}%

\index{x86!\Registers!ESP}
\RU{Можно было бы использовать \ESP, но он во время исполнения функции часто меняется, а это не удобно. 
Так что можно сказать, что \EBP это \IT{замороженное состояние} \ESP на момент начала исполнения функции.}%
\EN{It is also possible to use \ESP for the same purpose, although that is not very convenient since it changes frequently.
The value of the \EBP could be perceived as a \IT{frozen state} of the value in \ESP at the start of the function's execution.}%
\PTBR{Também é possível usar \ESP para o mesmo objetivo, mas não é muito conveniente pois ele se altera com frequência.
O valor de \EBP pode ser visto como uma cópia do valor de \ESP no começo da execução da função.}%

% FIXME1 это уже было в 02_stack?
\RU{Разметка типичного стекового \glslink{stack frame}{фрейма} в 32-битной среде:}%
\EN{Here is a typical \gls{stack frame} layout in 32-bit environment:}%
\PTBR{Aqui está a aparência típica de uma pilha em um ambiente de 32-bits:}%

\begin{center}
\begin{tabular}{ | l | l | }
\hline
\dots & \dots \\
\hline
EBP-8 & \RU{локальная переменная}\EN{local variable}\PTBR{variável local} \#2, \MarkedInIDAAs{} \TT{var\_8} \\
\hline
EBP-4 & \RU{локальная переменная}\EN{local variable}\PTBR{variável local} \#1, \MarkedInIDAAs{} \TT{var\_4} \\
\hline
EBP & \RU{сохраненное значение}\EN{saved value of}\PTBR{valor salvo de} \EBP \\
\hline
EBP+4 & \RU{адрес возврата}\EN{return address}\PTBR{Endereço de retorno} \\
\hline
EBP+8 & \argument \#1, \MarkedInIDAAs{} \TT{arg\_0} \\
\hline
EBP+0xC & \argument \#2, \MarkedInIDAAs{} \TT{arg\_4} \\
\hline
EBP+0x10 & \argument \#3, \MarkedInIDAAs{} \TT{arg\_8} \\
\hline
\dots & \dots \\
\hline
\end{tabular}
\end{center}

\RU{У функции \scanf в нашем примере два аргумента.}\EN{The \scanf function in our example has two arguments.}\PTBR{A função \scanf no nosso exemplo tem dois argumentos.}

\RU{Первый~--- указатель на строку, содержащую \TT{\%d} и второй~--- адрес переменной \TT{x}.}%
\EN{The first one is a pointer to the string containing \TT{\%d} and the second is the address of the \TT{x} variable.}%
\PTBR{O primeiro é um ponteiro para a string contendo \TT{\%d} e a segunda é o endereço da variável \TT{x}.}%

\index{x86!\Instructions!LEA}
\RU{Вначале адрес \TT{x} помещается в регистр \EAX при помощи инструкции \TT{lea eax, DWORD PTR \_x\$[ebp]}.}%
\EN{First, the \TT{x} variable's address is loaded into the \EAX register by the \TT{lea eax, DWORD PTR \_x\$[ebp]} instruction.}%
\PTBR{Primeiro, o endereço da variável \TT{x} é carregado no registrador \EAX pela instrução \TT{lea eax, DWORD PTR \_x\$[ebp].}.}%

\ifx\LITE\undefined
\RU{Инструкция \LEA означает \IT{load effective address}, и часто используется для формирования адреса чего-либо}
\EN{\LEA stands for \IT{load effective address}, and is often used for forming an address}
~(\myref{sec:LEA}).
\fi

\RU{Можно сказать, что в данном случае \LEA просто помещает в \EAX результат суммы значения в регистре \EBP и макроса \TT{\_x\$}.}%
\EN{We could say that in this case \LEA simply stores the sum of the \EBP register value and the \TT{\_x\$} macro in the \EAX register.}%
\PTBR{Nós podemos dizer que nesse caso, \LEA simplesmente armazena a soma do valor em \EBP e o macro \TT{\_x\$} no registrador \EAX.}%

\RU{Это тоже что и}\EN{This is the same as}\PTBR{É a mesma coisa que} \TT{lea eax, [ebp-4]}.

\RU{Итак, от значения \EBP отнимается 4 и помещается в \EAX.
Далее значение \EAX заталкивается в стек и вызывается \scanf.}%
\EN{So, 4 is being subtracted from the \EBP register value and the result is loaded in the \EAX register.
Next the \EAX register value is pushed into the stack and \scanf is being called.}%
\PTBR{Então, 4 está sendo subtraido do registrador \EBP e o resultado é carregado no registrador \EAX.
Depois, o valor em \EAX é empurrado para dentro da pilha e o \scanf é chamado.}%

\RU{После этого вызывается \printf. Первый аргумент вызова строка:}%
\EN{\printf is being called after that with its first argument --- a pointer to the string:}%
\PTBR{\printf é chamado depois disso com seu primeiro argumento --- um ponteiro para a string:}%
\TT{You entered \%d...\textbackslash{}n}.

\RU{Второй аргумент: \TT{mov ecx, [ebp-4]}.
Эта инструкция помещает в \ECX не адрес переменной \TT{x}, а её значение.}%
\EN{The second argument is prepared with: \TT{mov ecx, [ebp-4]}.
The instruction stores the \TT{x} variable value and not its address, in the \ECX register.}%
\PTBR{O segundo argumento é preparado com: \TT{mov ecx, [ebp-4]}.
A instrução armazena o valor de x e não o seu endereço, no registrador \ECX.}%

\RU{Далее значение \ECX заталкивается в стек и вызывается \printf.}%
\EN{Next the value in the \ECX is stored on the stack and the last \printf is being called.}%
\PTBR{Depois, o valor em \ECX é armazenado na pilha e o último \printf é chamado.}%

\ifdefined\IncludeOlly
\clearpage
\subsection{MSVC + \olly}
\index{\olly}

\RU{Попробуем этот же пример в}\EN{Let's try this example in} \olly.
\RU{Загружаем, нажимаем F8 (\stepover) до тех пор, пока не окажемся в своем исполняемом файле,
а не в}\EN{Let's load, press F8 (\stepover) until we get into our executable file
instead of} \TT{ntdll.dll}.
\RU{Скроллим вверх, до тех пока не найдем \main}\EN{Scroll up until \main appears}.
\RU{Кликаем на первой инструкции}\EN{Let's click on the first instruction} (\TT{PUSH EBP}), 
\RU{нажимаем}\EN{press} F2, \RU{затем}\EN{then} F9 (Run) \RU{и брякпойнт срабатывает
на начале \main}\EN{and breakpoint triggers on the \main begin}.

\RU{Трассируем до того места, где готовится адрес переменной $x$}
\EN{Let's trace to the place where the address of $x$ variable is prepared}:

\begin{figure}[H]
\centering
\includegraphics[scale=\FigScale]{patterns/04_scanf/1_simple/ex1_olly_1.png}
\caption{\olly: \RU{вычисляется адрес локальной переменной}\EN{address of the local variable is computed}}
\label{fig:scanf_ex1_olly_1}
\end{figure}

\RU{На \EAX в окне регистров можно нажать правой кнопкой и далее}
\EN{It is possible to right-click on \EAX in registers window and then} ``Follow in stack''.
\RU{Этот адрес покажется в окне стека}\EN{This address will appear in stack window}.
\RU{Смотрите, это переменная в локальном стеке}\EN{Look, this is a variable in the local stack}.
\RU{Я нарисовал там красную стрелку}\EN{I drew a red arrow there}.
\RU{И там сейчас какой-то мусор}\EN{And there are some garbage} (\TT{0x6E494714}).
\RU{Адрес этого элемента стека сейчас, при помощи \PUSH, запишется в этот же стек, рядом}
\EN{Now address of the stack element, with the help of \PUSH, will be written to the same stack, nearly}.
\RU{Трассируем при помощи F8 вплоть до конца исполнения \scanf}\EN{Let's trace by F8 until \scanf 
execution finished}.
\RU{А пока \scanf исполняется, в консольном окне, вводим, например, 123}
\EN{During the moment of \scanf execution, we enter, for example, 123, in the console window}:

\begin{figure}[H]
\centering
\includegraphics[scale=\NormalScale]{patterns/04_scanf/1_simple/ex1_olly_2.png}
\caption{\RU{Вывод в консоль}\EN{Console output}}
\label{fig:scanf_ex1_olly_2}
\end{figure}

\clearpage
\RU{Вот тут }\scanf \RU{отработал}\EN{executed here}:

\begin{figure}[H]
\centering
\includegraphics[scale=\FigScale]{patterns/04_scanf/1_simple/ex1_olly_3.png}
\caption{\olly: \scanf \RU{исполнилась}\EN{executed}}
\label{fig:scanf_ex1_olly_3}
\end{figure}

\scanf \RU{вернул}\EN{returns} $1$ \InENRU \EAX, \RU{что означает, что он успешно прочитал одно 
значение}\EN{which means, it have read one value successfully}.
\RU{В наблюдаемом нами элементе стека теперь}\EN{The element of 
stack of our attention now contain} \TT{0x7B} (123).

\clearpage
\RU{Чуть позже, это значение копируется из стека в регистр \ECX и передается в \printf}
\EN{Further, this value is copied from the stack to the \ECX register and passed into \printf}:

\begin{figure}[H]
\centering
\includegraphics[scale=\FigScale]{patterns/04_scanf/1_simple/ex1_olly_4.png}
\caption{\olly: \RU{готовим значение для передачи в}\EN{preparing the value for passing into} \printf}
\label{fig:scanf_ex1_olly_4}
\end{figure}

\fi

\ifdefined\IncludeGCC
\subsubsection{GCC}

\RU{Попробуем тоже самое скомпилировать в Linux при помощи GCC 4.4.1:}
\EN{Let's try to compile this code in GCC 4.4.1 under Linux:}

\lstinputlisting{patterns/04_scanf/1_simple/ex1_GCC.asm}

\index{puts() \RU{вместо}\EN{instead of} printf()}
\RU{GCC заменил первый вызов \printf на \puts. Почему это было сделано, 
уже было описано ранее~(\myref{puts}).}
\EN{GCC replaced the \printf call with call to \puts. The reason for this was explained in ~(\myref{puts}).}

% TODO: rewrite
%\RU{Почему \scanf переименовали в \TT{\_\_\_isoc99\_scanf}, я честно говоря, пока не знаю.}
%\EN{Why \scanf is renamed to \TT{\_\_\_isoc99\_scanf}, I do not know yet.}
% 
% Apparently it has to do with the ISO c99 standard compliance. By default GCC allows specifying a standard to adhere to.
% For example if you compile with -std=c89 the outputted assmebly file will contain scanf and not __isoc99__scanf. I guess current GCC version adhares to c99 by default.
% According to my understanding the two implementations differ in the set of suported modifyers (See printf man page)


\RU{Далее всё как и прежде~--- параметры заталкиваются через стек при помощи \MOV.}
\EN{As in the MSVC example---the arguments are placed on the stack using the \MOV instruction.}
\fi

\subsubsection{\RU{Кстати}\EN{By the way}}

\RU{Кстати, этот простой пример иллюстрирует то обстоятельство, что компилятор преобразует
список выражений в \CCpp-блоке просто в последовательный набор инструкций.
Между выражениями в \CCpp ничего нет, и в итоговом машинном коде между ними тоже ничего нет, 
управление переходит от одной инструкции к следующей за ней.}%
\EN{By the way, this simple example is a demonstration of the fact that compiler translates
list of expressions in \CCpp-block into sequential list of instructions.
There are nothing between expressions in \CCpp, and so in resulting machine code, 
there are nothing between, control flow slips from one expression to the next one.}%
\PTBR{A propósito, esse simples exemplo é a demonstração do fato de que o compilador traduz a lista de expressões em \CCpp em uma lista sequêncial de instruções.
Não há nada entre as expressões em \CCpp, como também no código de máquina resultante, não há nada, o controle de fluxo passa de uma expressão para a outra diretamente.}%

