\subsection{ARM}

\subsubsection{\OptimizingKeil + \ThumbMode}

\begin{lstlisting}
.text:00000042             scanf_main
.text:00000042
.text:00000042             var_8           = -8
.text:00000042
.text:00000042 08 B5                       PUSH    {R3,LR}
.text:00000044 A9 A0                       ADR     R0, aEnterX     ; "Enter X:\n"
.text:00000046 06 F0 D3 F8                 BL      __2printf
.text:0000004A 69 46                       MOV     R1, SP
.text:0000004C AA A0                       ADR     R0, aD          ; "%d"
.text:0000004E 06 F0 CD F8                 BL      __0scanf
.text:00000052 00 99                       LDR     R1, [SP,#8+var_8]
.text:00000054 A9 A0                       ADR     R0, aYouEnteredD___ ; "You entered %d...\n"
.text:00000056 06 F0 CB F8                 BL      __2printf
.text:0000005A 00 20                       MOVS    R0, #0
.text:0000005C 08 BD                       POP     {R3,PC}
\end{lstlisting}

\index{\CLanguageElements!\Pointers}
\IFRU{Чтобы \scanf мог вернуть значение, ему нужно передать указатель на переменную типа \Tint.}
{A pointer
to a \Tint-typed variable must be passed to a \scanf so it can return value via it.}
\Tint \IFRU{~--- 32-битное значение, для его хранения нужно только 4 байта, и оно помещается в 
32-битный регистр.}
{is 32-bit value, so we need 4 bytes for storing it somewhere in memory, and it fits exactly 
in 32-bit register.}
\index{IDA!var\_?}
\IFRU{Место для локальной переменной \TT{x} выделяется в стеке, \IDA наименовала её \IT{var\_8}, 
впрочем, место для нее выделять не обязательно, т.к., \glslink{stack pointer}{указатель стека} \SP уже указывает на место, 
свободное для использования сразу же.}{A place for the local variable \TT{x} is allocated in the stack and \IDA
named it \IT{var\_8}, however, it is not necessary to allocate it since \SP \gls{stack pointer}
is already pointing to the space may be used instantly.}
\IFRU{Так что значение указателя \SP копируется в регистр \Reg{1}, и вместе с format-строкой, 
передается в \scanf.}
{So, \SP \gls{stack pointer} value is copied to the \Reg{1} register and, together with format-string, passed
into \scanf.}
\index{ARM!\Instructions!LDR}
\IFRU{Позже, при помощи инструкции \TT{LDR}, это значение перемещается из стека в регистр \Reg{1}, 
чтобы быть переданным в \printf.}{Later, with the help of the \TT{LDR} instruction, this value is moved
from stack into the \Reg{1} register in order to be passed into \printf.}

\IFRU{Варианты, скомпилированные для ARM-режима процессора, а также варианты скомпилированные при 
помощи Xcode LLVM, не очень отличаются от этого, так что, мы можем пропустить их здесь.}
{Examples compiled for ARM-mode and also examples compiled with Xcode LLVM are not differ significantly
from what we saw here, so they are omitted.}

