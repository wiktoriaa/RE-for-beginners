\subsubsection{ARM}

\myparagraph{ARM: \OptimizingKeilVI (\ThumbMode)}

\lstinputlisting[caption=\OptimizingKeilVI (\ThumbMode),style=customasmARM]{patterns/04_scanf/3_checking_retval/ex3_ARM_Keil_thumb_O3.asm}

\myindex{ARM!\Instructions!CMP}
\myindex{ARM!\Instructions!BEQ}

ここでの新しい命令は \CMP と\ac{BEQ}です。

\CMP は同じ名前のx86命令に似ていますが、他の引数から引数の1つを減算し、必要に応じて条件フラグを更新します。
% TODO: в мануале ARM $op1 + NOT(op2) + 1$ вместо вычитания

\myindex{ARM!\Registers!Z}
\myindex{x86!\Instructions!JZ}
オペランドが互いに等しい場合、
または最後の計算の結果が0の場合、またはZフラグが1の場合、
\ac{BEQ}は別のアドレスにジャンプします。
これはx86では \JZ として動作します。

それ以外はすべてシンプルです。実行フローが2つの分岐に分岐した後、
関数の戻り値として0が\Reg{0}に書き込まれた時点で分岐が収束し、関数が終了します。

\myparagraph{ARM64}

\lstinputlisting[caption=\NonOptimizing GCC 4.9.1 ARM64,numbers=left,style=customasmARM]{patterns/04_scanf/3_checking_retval/ARM64_GCC491_O0_JPN.s}

\myindex{ARM!\Instructions!CMP}
\myindex{ARM!\Instructions!Bcc}
この場合のコードフローは、\INS{CMP}/\INS{BNE}(Branch if Not Equal)命令のペアを使用して分岐します。
