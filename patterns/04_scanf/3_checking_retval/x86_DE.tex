\subsubsection{MSVC: x86}
Wir erhalten den folgenden Assembleroutput (MSVC 2010):

\lstinputlisting[style=customasmx86]{patterns/04_scanf/3_checking_retval/ex3_MSVC_x86.asm}

\myindex{x86!\Registers!EAX}
Die aufrufende Funktion (\main) benötigt das Ergebnis der aufgerufenen Funktion (\scanf), weshalb diese es über das
Register \EAX zurückgibt.

\myindex{x86!\Instructions!CMP}
Wir prüfen mithilfe des Befehls \TT{CMP EAX, 1} (\IT{CoMPare}). Mit anderen Worten, wir vergleichen den Wert in \EAX mit
1.

\myindex{x86!\Instructions!JNE}
Ein bedingter \JNE Sprung folgt auf den \CMP Befehl. \JNE steht für \IT{Jump if Not Equal}.

Wenn also der Wert in \EAX ungleich 1 ist, wird die \ac{CPU} die Ausführung an der Stelle fortsetzen, die im Operanden
von \JNE steht, in unserem Fall \TT{\$LN2@main}.
Den Control Flow an diese Adresse zu übergeben hat zur Folge, dass die Funktion \printf mit dem Argument \TT{What you
entered? Huh?} aufgerufen wird.
Wenn aber alles funktioniert und der bedingte Sprung nicht ausgeführt wird, wird ein anderer Aufruf von \printf mit zwei
Argumenten ausgeführt:\\
\TT{'You entered \%d...'} und dem Wert von \TT{x}.


\myindex{x86!\Instructions!XOR}
\myindex{\CLanguageElements!return}
Da in diesem Fall das zweite \printf nicht ausgeführt werden darf, befindet sich davor ein \JMP (unbedingter Sprung).
Dieser gibt den Control Flow ab an den Punkt nach dem zweiten \printf, genau vor dem \TT{XOR EAX EAX} Befehl, welcher
die Rückgabe von 0 implementiert.

% FIXME internal \ref{} to x86 flags instead of wikipedia
\myindex{x86!\Registers!\Flags}
Man kann also festhalten, dass der Vergleich von zwei Werten gewöhnlich durch ein \CMP/\Jcc Befehlspaar implementiert
wird, wobei \IT{cc} für \IT{condition code}, also Sprungbedingung, steht. 
\CMP vergleicht zwei Werte und setzt die Flags des Prozessors\footnote{zu x86 Flags, siehe auch:
\href{http://go.yurichev.com/17120}{wikipedia}.}.
\Jcc prüft diese Flags und entscheidet entweder den Control Flow an die angegebene Adresse zu übergeben oder nicht.

\myindex{x86!\Instructions!CMP}
\myindex{x86!\Instructions!SUB}
\myindex{x86!\Instructions!JNE}
\myindex{x86!\Registers!ZF}
\label{CMPandSUB}
Es klingt möglicherweise paradox, aber der \CMP Befehl ist tatsächlich ein \SUB (subtract).
Alle arithmetischen Befehle setzen die Flags des Prozessors, nicht nur \CMP.
Wenn wir 1 und 1 vergleichen, ist $1-1=0$ und daher wird das \ZF Flag gesetzt (gleichbedeutend damit, dass das Ergebnis
der letzten Berechnung 0 ergeben hat).
\ZF kann nur durch diesen Umstand gesetzt werden, nämlich, dass zwei Operanden gleich sind.
\JNE prüft nur das \ZF Flag und springt nur, wenn dieses nicht gesetzt ist. \JNE ist daher ein Synonym für \JNZ
(\IT{Jump if Not Zero}).
Der Assembler übersetzt \JNE und \JNZ in den gleichen Opcode.
Der \CMP Befehl kann also durch ein \SUB ersetzt werden, aber mit dem Unterschied, dass \SUB den Wert des ersten
Operanden verändert. \CMP bedeutet also \IT{SUB ohne Speichern des Ergebnisses, aber mit Setzen der Flags}.

\subsubsection{MSVC: x86: IDA}

\myindex{IDA}
Es ist an der Zeit \IDA auszuprobieren und etwas damit zu machen.
Für Anfänger ist es übrigens eine gute Idee, die \TT{/MD} Option in MSVC zu verwenden, da diese bewirkt, dass alle
Standardfunktionen nicht mit der ausführbaren Datei verlinkt werden, sondern aus der Datei \TT{MSVCR*.DLL} importiert
werden. Dadurch ist es einfacher zu erkennen, welche Standardfunktionen verwendet werden und wo dies geschieht.

Bei der Codeanalyse in \IDA ist es hilfreich Notizen für sich selbst (und andere) zu hinterlassen.
Bei der Analyse dieses Beispiels sehen wir, dass \TT{JNZ} im Falle eines Fehlers ausgeführt wird.
Es ist nun möglich den Cursor auf das Label zu setzen, \q{n} zu drücken und es in \q{error} umzubenennen.
Wir erstellen noch ein Label---in \q{exit}.
Hier ist das Ergebnis:

\lstinputlisting[style=customasmx86]{patterns/04_scanf/3_checking_retval/ex3.lst}
So ist es etwas einfacher den Code zu verstehen. Natürlich ist es aber auch keine gute Idee, jeden Befehl zu
kommentieren.

% FIXME draw button?
Man kann Teile einer Funktion in \IDA auch einklappen. Um dies zu tun, markiert man den Block und drückt dann \q{--} auf
dem Zahlenblock der Tastatur und gibt den stattdessen anzuzeigenden Text ein.

Verstecken wir zwei Blöcke und geben ihnen Namen:

\lstinputlisting[style=customasmx86]{patterns/04_scanf/3_checking_retval/ex3_2.lst}

% FIXME draw button?
Um eingeklappte Teile des Code wieder auszuklappen, verwendet man \q{+} auf dem Zahlenblock der Tastatur.

\clearpage
Durch Drücken von der \q{Leertaste} sehen wir, wie \IDA die Funktion als Graph darstellt:

\begin{figure}[H]
\centering
\myincludegraphics{patterns/04_scanf/3_checking_retval/IDA.png}
\caption{Graph Modus in IDA}
\label{fig:ex3_IDA_1}
\end{figure}
Es gibt hinter jedem bedingten Sprung zwei Pfeile: einen grünen und einen roten.
Der grüne Pfeil zeigt auf den Codeblock der ausgeführt wird, wenn der Sprung ausgeführt wird und der rote den Codeblock,
der ausgeführt wird, falls nicht gesprungen wird.

\clearpage
In diesem Modus ist es möglich, Knoten einzuklappen und ihnen auch Namen zu geben (\q{group nodes}).
Wir probieren das mit 3 Blöcken aus:

\begin{figure}[H]
\centering
\myincludegraphics{patterns/04_scanf/3_checking_retval/IDA2.png}
\caption{Graph Modus in IDA mit 3 eingeklappten Knoten}
\label{fig:ex3_IDA_2}
\end{figure}

Das ist sehr nützlich.
Man kann sagen, dass ein großer Teil der Arbeit eines Reverse Engineers (und eines jeden anderen Forsches) darin
besteht, die Menge der zur Verfügung stehenden Informationen zu reduzieren.

\clearpage
\subsubsection{MSVC: x86 + \olly}
Laden wir unser Programm in \olly und zwingen es dazu zu glauben, dass \scanf stets ohne Fehler arbeitet.
Wenn die Adresse einer lokalen Variablen an \scanf übergeben wird, enthält die Variable zu Beginn einen zufälligen Wert,
in diesem Fall \TT{0x6E494714}:

\begin{figure}[H]
\centering
\myincludegraphics{patterns/04_scanf/3_checking_retval/olly_1.png}
\caption{\olly: Adresse der Variablen an \scanf übergeben}
\label{fig:scanf_ex3_olly_1}
\end{figure}

\clearpage
Während \scanf ausgeführt wird, geben wir in der Konsole etwas ein, das definitiv keine Zahl ist, z.B. \q{asdasd}.
\scanf beendet sich mit 0 in \EAX, was anzeigt, dass ein Fehler aufgetreten ist:

\begin{figure}[H]
\centering
\myincludegraphics{patterns/04_scanf/3_checking_retval/olly_2.png}
\caption{\olly: \scanf Rückgabewert bei Fehler}
\label{fig:scanf_ex3_olly_2}
\end{figure}
Wir können auch die lokale Variable auf dem Stack überprüfen und stellen fest, dass sie sich nicht verändert hat.
Was könnte \scanf hier auch hineinschreiben? Die Funktion hat nichts getan außer 0 zurückzugeben.

Versuchen wir unser Programm zu modifizieren, d.i. zu \q{hacken}.
Rechtsklick auf \EAX, in den Optionen finden wir \q{Set to 1}. Das ist was wir brauchen.

Wir haben jetzt 1 in \EAX, sodass die folgende Überprüfung wie gewünscht ausgeführt wird und \printf den Wert der
Variablen auf dem Stack ausgibt.

Wenn wir das Programm laufen lassen (F9), sehen wir das Folgende im Konsolenfenster:

\lstinputlisting[caption=console window]{patterns/04_scanf/3_checking_retval/console.txt}

Und tatsächlich ist 1850296084 die dezimale Darstellung der Zahl auf dem Stack (\TT{0x6E494714})!


\clearpage
\subsubsection{MSVC: x86 + Hiew}
\myindex{Hiew}
Unser Programm kann auch als einfaches Beispel für das Patchen einer Executable dienen.
Wir könnten versuchen, die Executable so zu patchen, dass das Programm unabhängig vom Input diesen stets auszugeben.

Angenommen, dass die Executable mit externer \TT{MSVCR*.DLL} (d.h. mit der Option \TT{MD}) kompiliert
wurde\footnote{dieser Vorgang wird auch \q{dynamisches Verlinken genannt}}, finden wir die Funktion \main am Anfang des
\TT{.text} Segments. 
Öffnen wir die Executable in Hiew und schauen uns den Anfang des \TT{.text} Segments an (Enter, F8, F6, Enter, Enter).

Wir sehen das Folgende:

\begin{figure}[H]
\centering
\myincludegraphics{patterns/04_scanf/3_checking_retval/hiew_1.png}
\caption{Hiew: \main Funktion}
\label{fig:scanf_ex3_hiew_1}
\end{figure}

Hiew erkennt \ac{ASCIIZ} Strings und die Namen importierter Funktionen und zeigt diese an.

\clearpage
Setzen wir den Cursor auf die Adresse \TT{.00401027}, an der sich der \TT{JNZ} Befehl, den wir umgehen müssen, befindet,
drücken F3 und fügen dann \q{9090} (zwei \ac{NOPS}s) ein. 

\begin{figure}[H]
\centering
\myincludegraphics{patterns/04_scanf/3_checking_retval/hiew_2.png}
\caption{Hiew: ersetzen von \TT{JNZ} durch zwei \ac{NOP}s}
\label{fig:scanf_ex3_hiew_2}
\end{figure}
Wir drücken F9 (update). Die Executable wird gespeichert und verhält sich wie gewünscht.

Zwei \ac{NOP}s sind wahrscheinlich nicht der ästhetischte Ansatz. Ein anderer Weg die Executable zu patchen besteht
darin, das zweite Byte des Opcodes (den \gls{jump offset}) auf 0 zu setzen, sodass \TT{JNZ} immer zum nächsten Befehl
springt.

Wir könnten auch das Gegenteil tun: das erste Byte durch \TT{EB} ersetzen und das zweite (\gls{jump offset})
unangetastet lassen. Wir würde einen unbedingten Sprung erhalten, der stets ausgeführt wird.
In diesem Fall würde unabhängig vom Input stets die Fehlermeldung ausgegeben.
