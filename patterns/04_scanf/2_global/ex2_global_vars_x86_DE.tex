\subsubsection{MSVC: x86}

\lstinputlisting[style=customasmx86]{patterns/04_scanf/2_global/ex2_MSVC.asm}
In diesem Falle wird die Variable \TT{x} im \TT{\_DATA} Segment definiert und kein Speicher auf dem lokalen Stack wird
reserviert. Es erfolgt ein direkter Zugriff ohne Umweg über den Stack.
Nicht initialisierte globale Variablen verbrauchen keinen Platz in der ausführbaren Datei (und wirklich: warum sollte
man Speicher reservieren, wenn die Variable auf Anfang auf 0 gesetzt wird?), aber wenn jemand auf ihre Adresse zugreift,
wird das \ac{OS} einen Block Nullen reservieren\footnote{That is how a \ac{VM} behaves}.

Weisen wir nun der Variablen ausdrücklich einen Wert zu:

\lstinputlisting[style=customc]{patterns/04_scanf/2_global/default_value_DE.c}

Wir erhalten:

\begin{lstlisting}[style=customasmx86]
_DATA	SEGMENT
_x	DD	0aH

...
\end{lstlisting}

Wir sehen hier einen Wert \TT{0xA} vom Typ DWORD (DD steht für DWORD = 32 Bit) für diese Variable.

Wenn man die kompilierte .exe in \IDA öffnet, sieht man, dass die Variable $x$ am Anfang des \TT{\_DATA} Segments
angelegt wird und danach sehen wir Textstrings. 

Here we see a value \TT{0xA} of DWORD type (DD stands for DWORD = 32 bit) for this variable.

Wenn man die kompilierte .exe aus dem vorangegangenen Beispiel in \IDA öffnet, in dem der Wert von \TT{x} nicht gesetzt
wurde, sieht man etwa das Folgende:

\lstinputlisting[caption=\IDA,style=customasmx86]{patterns/04_scanf/2_global/IDA.lst}

\TT{\_x} ist mit \TT{?} markiert zusammen mit dem Rest der Variablen, die nicht initialisiert werden müssen.
Daraus folgt, dass nachdem die .exe in den Speicher geladen wurde, Platz für alle diese Variablen angelegt und mit
Nullen gefüllt werden muss \InSqBrackets{\CNineNineStd 6.7.8p10}.
Aber in der .exe Datei selbst, belegen diese nicht initialisierten Variablen keinerlei Platz. 
Dies ist beispielsweise für große Arrays üblich.

\clearpage
\subsection{MSVC: x86 + \olly}
\index{\olly}

\RU{Тут даже проще}\EN{Things are even simpler here}:

\begin{figure}[H]
\centering
\includegraphics[scale=\FigScale]{patterns/04_scanf/2_global/ex2_olly_1.png}
\caption{\olly: \RU{после исполнения \scanf}\EN{after \scanf execution}}
\label{fig:scanf_ex2_olly_1}
\end{figure}

\RU{Переменная хранится в сегменте данных.}
\EN{The variable is located in the data segment.}
\RU{Кстати, после исполнения инструкции \PUSH (заталкивающей адрес $x$) адрес появится в стеке, 
и на этом элементе можно нажать правой кнопкой, выбрать \q{Follow in dump}.}
\EN{After the \PUSH instruction (pushing the address of $x$) gets executed, 
the address appears in the stack window. Right-click on that row and select \q{Follow in dump}.}
\RU{И в окне памяти слева появится эта переменная.}
\EN{The variable will appear in the memory window on the left.}

\RU{После того как в консоли введем 123, здесь появится}\EN{After we have entered 123 in the console,} 
\TT{0x7B}\EN{ appears in the memory window (see the highlighted screenshot regions)}.

\RU{Почему самый первый байт это}\EN{But why is the first byte} \TT{7B}?
\RU{По логике вещей, здесь должно было бы быть}\EN{Thinking logically,} \TT{00 00 00 7B}\EN{ should be
there}.
\RU{Это называется}\EN{The cause for this is referred as } \gls{endianness}, \RU{и в x86 принят формат }\EN{and x86 uses }\IT{little-endian}.
\RU{Это означает, что в начале записывается самый младший байт, а заканчивается самым старшим байтом}%
\EN{This implies that the lowest byte is written first, and the highest written last}.
\RU{Больше об этом}\EN{Read more about it at}: \myref{sec:endianness}.

\RU{Позже из этого места в памяти 32-битное значение загружается в \EAX и передается в}
\EN{Back to the example, the 32-bit value is loaded from this memory address into \EAX and passed to} \printf.

\RU{Адрес переменной $x$ в памяти}\EN{The memory address of $x$ is} \TT{0x00C53394}.

\clearpage
\RU{В \olly{} мы можем посмотреть карту памяти процесса (Alt-M) и увидим, что этот адрес
внутри PE-сегмента \TT{.data} нашей программы}%
\EN{In \olly we can review the process memory map (Alt-M)
and we can see that this address is inside the \TT{.data} PE-segment of our program}:

\begin{figure}[H]
\centering
\includegraphics[scale=\FigScale]{patterns/04_scanf/2_global/ex2_olly_2.png}
\caption{\olly: \RU{карта памяти процесса}\EN{process memory map}}
\label{fig:scanf_ex2_olly_2}
\end{figure}


\subsubsection{GCC: x86}

\myindex{ELF}
Unter Linux zeigt sich ein ähnliches Bild, mit dem Unterschied, dass die nicht initialisierten Variablen im \TT{\_bss}
Segment gehalten werden. 
I der \ac{ELF} Datei hat dieses Segment die folgenden Attribute: 

\begin{lstlisting}
; Segment type: Uninitialized
; Segment permissions: Read/Write
\end{lstlisting}

Wenn man nun der Variablen einen Wert zuweist, z.B. 10, muss diese im \TT{\_data} Segment abgelegt werden, dass die
folgenden Attribute aufweist:

\begin{lstlisting}
; Segment type: Pure data
; Segment permissions: Read/Write
\end{lstlisting}

\subsubsection{MSVC: x64}

\lstinputlisting[caption=MSVC 2012 x64,style=customasmx86]{patterns/04_scanf/2_global/ex2_MSVC_x64_DE.asm}
Der Code ist beinahe der gleich wie in x86.
Man beachte, dass die Adresse der Variable $x$ mittels des Befehls \LEA an die Funktion \TT{scanf()} übergeben wird,
wohingegen der Wert der Variablen an das zweite \printf mit einem \MOV Befehl übergeben wird.
\TT{DWORD PTR}---ist ein Teil der Assemblersprache (mit keinerlei Verbindung zum Maschinencode) und zeigt an, dass die
Variablengröße 32 Bit beträgt und der \MOV Befehl entsprechend aufgebaut sein muss.


