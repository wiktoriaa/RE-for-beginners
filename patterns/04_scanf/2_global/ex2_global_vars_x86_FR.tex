\subsubsection{MSVC: x86}

\lstinputlisting[style=customasmx86]{patterns/04_scanf/2_global/ex2_MSVC.asm}

Dans ce cas, la variable \TT{x} est définie dans la section \TT{\_DATA} et il n'y
a pas de mémoire allouée sur la pile locale. Elle est accédée directement, pas par
la pile.
Les variables globales non initialisées ne prennent pas de place dans le fichier
exécutable (en effet, pourquoi aurait-on besoin d'allouer de l'espace pour des variables
initialement mises à zéro ?), mais lorsque quelqu'un accède à leur adresse, l'\ac{OS}
va y allouer un bloc de zéros\footnote{cwC'est comme ça que secomporte les \ac{VM}}.

Maintenant, assignons explicitement une valeur à la variable:

\lstinputlisting[style=customc]{patterns/04_scanf/2_global/default_value_FR.c}

Nous obtenons:

\begin{lstlisting}[style=customasmx86]
_DATA	SEGMENT
_x	DD	0aH

...
\end{lstlisting}

Ici nous voyons une valeur \TT{0xA} de type DWORD (DD signifie DWORD = 32 bit) pour
cette variable.

Si vous ouvrez le .exe compilé dans \IDA, vous pouvez voir la variable \IT{x} placée
au début du segment \TT{\_DATA}, et après elle vous pouvez voir la chaîne de texte.

Si vous ouvrez le .exe compilé de l'exemple précédent dans \IDA, oú la valeur de
\IT{x} n'était pas mise, vous verrez quelque chose comme ça:

\lstinputlisting[caption=\IDA,style=customasmx86]{patterns/04_scanf/2_global/IDA.lst}

\TT{\_x} est marquée avec \TT{?} avec le reste des variables qui ne doivent pas être
initialisées.
Ceci implique qu'après avoir chargé le .exe en mémoire, de l'espace pour toutes ces
variables doit être alloué et rempli avec des zéros \InSqBrackets{\CNineNineStd 6.7.8p10}.
Mais dans le fichier .exe, ces variables non initialisées n'occupent rien du tout.
C'est pratique pour les gros tableaux, par exemple.

\clearpage
\subsection{MSVC: x86 + \olly}
\index{\olly}

\RU{Тут даже проще}\EN{Things are even simpler here}:

\begin{figure}[H]
\centering
\includegraphics[scale=\FigScale]{patterns/04_scanf/2_global/ex2_olly_1.png}
\caption{\olly: \RU{после исполнения \scanf}\EN{after \scanf execution}}
\label{fig:scanf_ex2_olly_1}
\end{figure}

\RU{Переменная хранится в сегменте данных.}
\EN{The variable is located in the data segment.}
\RU{Кстати, после исполнения инструкции \PUSH (заталкивающей адрес $x$) адрес появится в стеке, 
и на этом элементе можно нажать правой кнопкой, выбрать \q{Follow in dump}.}
\EN{After the \PUSH instruction (pushing the address of $x$) gets executed, 
the address appears in the stack window. Right-click on that row and select \q{Follow in dump}.}
\RU{И в окне памяти слева появится эта переменная.}
\EN{The variable will appear in the memory window on the left.}

\RU{После того как в консоли введем 123, здесь появится}\EN{After we have entered 123 in the console,} 
\TT{0x7B}\EN{ appears in the memory window (see the highlighted screenshot regions)}.

\RU{Почему самый первый байт это}\EN{But why is the first byte} \TT{7B}?
\RU{По логике вещей, здесь должно было бы быть}\EN{Thinking logically,} \TT{00 00 00 7B}\EN{ should be
there}.
\RU{Это называется}\EN{The cause for this is referred as } \gls{endianness}, \RU{и в x86 принят формат }\EN{and x86 uses }\IT{little-endian}.
\RU{Это означает, что в начале записывается самый младший байт, а заканчивается самым старшим байтом}%
\EN{This implies that the lowest byte is written first, and the highest written last}.
\RU{Больше об этом}\EN{Read more about it at}: \myref{sec:endianness}.

\RU{Позже из этого места в памяти 32-битное значение загружается в \EAX и передается в}
\EN{Back to the example, the 32-bit value is loaded from this memory address into \EAX and passed to} \printf.

\RU{Адрес переменной $x$ в памяти}\EN{The memory address of $x$ is} \TT{0x00C53394}.

\clearpage
\RU{В \olly{} мы можем посмотреть карту памяти процесса (Alt-M) и увидим, что этот адрес
внутри PE-сегмента \TT{.data} нашей программы}%
\EN{In \olly we can review the process memory map (Alt-M)
and we can see that this address is inside the \TT{.data} PE-segment of our program}:

\begin{figure}[H]
\centering
\includegraphics[scale=\FigScale]{patterns/04_scanf/2_global/ex2_olly_2.png}
\caption{\olly: \RU{карта памяти процесса}\EN{process memory map}}
\label{fig:scanf_ex2_olly_2}
\end{figure}


\subsubsection{GCC: x86}

\myindex{ELF}
Le schéma sous Linux est presque le même, avec la différence que les variables
non initialisées se trouvent dans le segment \TT{\_bss}.
Dans un fichier \ac{ELF} ce segment possède les attributs suivants:

\begin{lstlisting}
; Segment type: Uninitialized
; Segment permissions: Read/Write
\end{lstlisting}

Si toutefois vous initialisez la variable avec une velauer quelconque, e.g. 10,
elle sera placée dans le segment \TT{\_data}, qui possède les attributs suivants:

\begin{lstlisting}
; Segment type: Pure data
; Segment permissions: Read/Write
\end{lstlisting}

\subsubsection{MSVC: x64}

\lstinputlisting[caption=MSVC 2012 x64,style=customasmx86]{patterns/04_scanf/2_global/ex2_MSVC_x64_FR.asm}

Le code est presque le même qu'en x86.
Notez toutefois que l'adresse de la variable $x$ est passée à \TT{scanf()} en utilisant
une instruction \LEA, tandis que la valeur de la variable est passée au second \printf
en utilisant une instruction \MOV.
\TT{DWORD PTR}---fait partie du langage d'assemblage (aucune relation avec le code machine),
indique que la taille de la variable est 32-bit et que l'instruction \MOV doit être
encodée en conséquence.

