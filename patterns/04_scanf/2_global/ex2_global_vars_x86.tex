\subsection{MSVC: x86}

\lstinputlisting{patterns/04_scanf/2_global/ex2_MSVC.asm}

\RU{В целом ничего особенного. Теперь \TT{x} объявлена в сегменте \TT{\_DATA}. 
Память для неё в стеке более не выделяется.
Все обращения к ней происходит не через стек, а уже напрямую. 
Неинициализированные глобальные переменные не занимают места в исполняемом файле
(и действительно, зачем в исполняемом файле
нужно выделять место под изначально нулевые переменные?), но тогда, когда к этому месту в памяти
кто-то обратится, \ac{OS} подставит туда блок, состоящий из нулей\footnote{Так работает \ac{VM}}.}%
\EN{In this case the \TT{x} variable is defined in the \TT{\_DATA} segment and no memory is allocated in the local stack. It is accessed directly, not through the stack. 
Uninitialized global variables take no space in the executable file
(indeed, why one needs to allocate space for variables initially set to zero?), 
but when someone accesses their address, 
the \ac{OS} will allocate a block of zeroes there\footnote{That is how a \ac{VM} behaves}.}%
\PTBR{Nesse caso, a variável \TT{x} é definida no segmento \TT{\_DATA} e nenhuma memória é alocada na pilha local.
Ela é acessada diretamente, não através da pilha.
Variáveis globais não inicialiadas não ocupam espaço no arquivo executável 
(realmente, ninguém precisa alocar espaço para uma variável inicialmente valendo zero), 
mas quando alguém acessa o endereço delas, o sistema operacional vai alocar um bloco contendo somente zeros nele.
\footnote{\ac{TBT}: That is how a \ac{VM} behaves}.}%

\RU{Попробуем изменить объявление этой переменной:}%
\EN{Now let's explicitly assign a value to the variable:}%
\PTBR{Agora vamos definir um valor para a variável:}%

\lstinputlisting{patterns/04_scanf/2_global/default_value.c.\LANG}

\RU{Выйдет в итоге:}\EN{We got:}\PTBR{Nós temos:}

\begin{lstlisting}
_DATA	SEGMENT
_x	DD	0aH

...
\end{lstlisting}

\RU{Здесь уже по месту этой переменной записано \TT{0xA} с типом DD (dword = 32 бита).}%
\EN{Here we see a value \TT{0xA} of DWORD type (DD stands for DWORD = 32 bit) for this variable.}%
\PTBR{Aqui nós vemos um valor \TT{0xA} do tipo DWORD (DD significa DWORD = 32 bits) para essa variável.}%

\RU{Если вы откроете скомпилированный .exe-файл в \IDA, то увидите, что \IT{x} 
находится в начале сегмента \TT{\_DATA}, после этой переменной будут текстовые строки.}%
\EN{If you open the compiled .exe in \IDA, you can see the \IT{x} variable placed at the beginning of 
the \TT{\_DATA} segment, and after it you can see text strings.}%
\PTBR{Se você abrir o .exe compilado no \IDA, você pode ver a variável \IT{x} colocada no começo do segmento \TT{\_DATA},
e depois disso você pode ver as strings.}%

\RU{А вот если вы откроете в \IDA .exe скомпилированный в прошлом примере, где значение \IT{x} не определено, то вы увидите:}%
\EN{If you open the compiled .exe from the previous example in \IDA, where the value of \IT{x} was not set, you would see something like this:}%
\PTBR{Se você abrir o .exe compilado no exemplo anterior no \IDA, onde o valor de x não foi declarado, você poderá ver algo assim:}%

\begin{lstlisting}
.data:0040FA80 _x              dd ?                    ; DATA XREF: _main+10
.data:0040FA80                                         ; _main+22
.data:0040FA84 dword_40FA84    dd ?                    ; DATA XREF: _memset+1E
.data:0040FA84                                         ; unknown_libname_1+28
.data:0040FA88 dword_40FA88    dd ?                    ; DATA XREF: ___sbh_find_block+5
.data:0040FA88                                         ; ___sbh_free_block+2BC
.data:0040FA8C ; LPVOID lpMem
.data:0040FA8C lpMem           dd ?                    ; DATA XREF: ___sbh_find_block+B
.data:0040FA8C                                         ; ___sbh_free_block+2CA
.data:0040FA90 dword_40FA90    dd ?                    ; DATA XREF: _V6_HeapAlloc+13
.data:0040FA90                                         ; __calloc_impl+72
.data:0040FA94 dword_40FA94    dd ?                    ; DATA XREF: ___sbh_free_block+2FE
\end{lstlisting}

\RU{\TT{\_x} обозначен как \TT{?}, наряду с другими переменными не требующими инициализации. 
Это означает, что при загрузке .exe в память, место под всё это выделено будет и будет заполнено
нулевыми байтами \cite[6.7.8p10]{C99TC3}. 
Но в самом .exe ничего этого нет. Неинициализированные переменные не занимают места в исполняемых файлах. 
Это удобно для больших массивов, например.}%
\EN{\TT{\_x} is marked with \TT{?} with the rest of the variables that do not need to be initialized. 
This implies that after loading the .exe to the memory, a space for all these variables is to be 
allocated and filled with zeroes \cite[6.7.8p10]{C99TC3}. 
But in the .exe file these uninitialized variables do not occupy anything.
This is convenient for large arrays, for example.}%
\PTBR{\TT{\_x} está marcada com \TT{?} juntamente com o resto das variáveis que não precisam ser inicializadas.
Isso implica que após carregar o .exe para a memória, um espaço para todas essas variáveis será alocado e preenchido com zeros \cite[6.7.8p10]{C99TC3}.
Mas no arquivo .exe essas variáveis não inicializadas não ocupam nenhum espaço.
Isso é conveniente para arrays grandes, por exemplo.}%

\ifdefined\IncludeOlly
\clearpage
\subsection{MSVC: x86 + \olly}
\index{\olly}

\RU{Тут даже проще}\EN{Things are even simpler here}:

\begin{figure}[H]
\centering
\includegraphics[scale=\FigScale]{patterns/04_scanf/2_global/ex2_olly_1.png}
\caption{\olly: \RU{после исполнения \scanf}\EN{after \scanf execution}}
\label{fig:scanf_ex2_olly_1}
\end{figure}

\RU{Переменная хранится в сегменте данных.}
\EN{The variable is located in the data segment.}
\RU{Кстати, после исполнения инструкции \PUSH (заталкивающей адрес $x$) адрес появится в стеке, 
и на этом элементе можно нажать правой кнопкой, выбрать \q{Follow in dump}.}
\EN{After the \PUSH instruction (pushing the address of $x$) gets executed, 
the address appears in the stack window. Right-click on that row and select \q{Follow in dump}.}
\RU{И в окне памяти слева появится эта переменная.}
\EN{The variable will appear in the memory window on the left.}

\RU{После того как в консоли введем 123, здесь появится}\EN{After we have entered 123 in the console,} 
\TT{0x7B}\EN{ appears in the memory window (see the highlighted screenshot regions)}.

\RU{Почему самый первый байт это}\EN{But why is the first byte} \TT{7B}?
\RU{По логике вещей, здесь должно было бы быть}\EN{Thinking logically,} \TT{00 00 00 7B}\EN{ should be
there}.
\RU{Это называется}\EN{The cause for this is referred as } \gls{endianness}, \RU{и в x86 принят формат }\EN{and x86 uses }\IT{little-endian}.
\RU{Это означает, что в начале записывается самый младший байт, а заканчивается самым старшим байтом}%
\EN{This implies that the lowest byte is written first, and the highest written last}.
\RU{Больше об этом}\EN{Read more about it at}: \myref{sec:endianness}.

\RU{Позже из этого места в памяти 32-битное значение загружается в \EAX и передается в}
\EN{Back to the example, the 32-bit value is loaded from this memory address into \EAX and passed to} \printf.

\RU{Адрес переменной $x$ в памяти}\EN{The memory address of $x$ is} \TT{0x00C53394}.

\clearpage
\RU{В \olly{} мы можем посмотреть карту памяти процесса (Alt-M) и увидим, что этот адрес
внутри PE-сегмента \TT{.data} нашей программы}%
\EN{In \olly we can review the process memory map (Alt-M)
and we can see that this address is inside the \TT{.data} PE-segment of our program}:

\begin{figure}[H]
\centering
\includegraphics[scale=\FigScale]{patterns/04_scanf/2_global/ex2_olly_2.png}
\caption{\olly: \RU{карта памяти процесса}\EN{process memory map}}
\label{fig:scanf_ex2_olly_2}
\end{figure}

\fi

\ifdefined\IncludeGCC
\subsection{GCC: x86}

\myindex{ELF}
\RU{В Linux всё почти также. За исключением того, что если значение \TT{x} не определено, 
то эта переменная будет находится в сегменте \TT{\_bss}.
В \ac{ELF} этот сегмент имеет такие атрибуты:}
\EN{The picture in Linux is near the same, with the difference that the uninitialized variables are located in the \TT{\_bss} segment. 
In \ac{ELF} file this segment has the following attributes:}

\begin{lstlisting}
; Segment type: Uninitialized
; Segment permissions: Read/Write
\end{lstlisting}

\RU{Ну а если сделать статическое присвоение этой переменной какого-либо
значения, например, 10, то она будет находится 
в сегменте \TT{\_data},
это сегмент с такими атрибутами:}
\EN{If you, however, initialise the variable with some value e.g. 10, 
it is to be placed in the \TT{\_data} segment, which has the following attributes:}

\begin{lstlisting}
; Segment type: Pure data
; Segment permissions: Read/Write
\end{lstlisting}
\fi

\subsection{MSVC: x64}

\lstinputlisting[caption=MSVC 2012 x64]{patterns/04_scanf/2_global/ex2_MSVC_x64.asm.\LANG}

\RU{Почти такой же код как и в x86.
Обратите внимание что для \TT{scanf()} адрес переменной $x$ передается
при помощи инструкции \LEA, а во второй \printf передается само значение переменной при помощи \MOV.
\TT{DWORD PTR} --- это часть языка ассемблера (не имеющая отношения к машинным кодам) показывающая, что тип переменной в памяти именно 32-битный, 
и инструкция \MOV должна быть здесь закодирована соответственно.}%
\EN{The code is almost the same as in x86.
Please note that the address of the $x$ variable is passed to \TT{scanf()} using a \LEA instruction,
while the variable's value is passed to the second \printf using a \MOV instruction.
\TT{DWORD PTR}---is a part of the assembly language (no relation to the machine code),
indicating that the variable data size is 32-bit and the \MOV instruction has to be encoded accordingly.}%
\PTBR{O código é quase o mesmo que no x86.
Por favor, perceba que o endereço da variável x é passado para \TT{scanf()} usando uma instrução \LEA,
enquanto os valores das variáveis são passadas para o segundo \printf usando uma instrução \MOV.
\TT{DWORD PTR} é uma parte da linguagem assembly (sem relação com o código de máquina),
indicando que o tamanho da informação da variável é de 32-bits e que a instrução \MOV tem de ser codificada de acordo.}%

