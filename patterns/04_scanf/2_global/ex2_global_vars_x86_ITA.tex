\subsubsection{MSVC: x86}

\lstinputlisting[style=customasmx86]{patterns/04_scanf/2_global/ex2_MSVC.asm}

In questo caso la variabile \TT{x} è definita nel segmento \TT{\_DATA} e per essa non viene allocata alcuna memoria nello stack locale. Viene acceduta direttamente, non attraverso lo stack.
Le variabili globali non inizializzate non occupano spazio nel file eseguibile 
(che motivo ci sarebbe di allocare spazio per varibaili inizialmente settate a zero?), 
ma quanto qualcuno accede al loro indirizzo, l' \ac{OS} allocherà un bloco di zeri al loro posto \footnote{Questo è il modo in cui vunziona una \ac{VM} }.

Adesso assegnamo esplicitamente un valore alla variabile:

\lstinputlisting[style=customc]{patterns/04_scanf/2_global/default_value_EN.c}

Otteniamo:

\begin{lstlisting}[style=customasmx86]
_DATA	SEGMENT
_x	DD	0aH

...
\end{lstlisting}

Vediamo qui un valore \TT{0xA} di tipo DWORD (DD sta per DWORD = 32 bit) per questa variabile.

Analizzando con \IDA il file .exe compilato, notiamo che la variabile \IT{x} è collocata all'inizio del segmento \TT{\_DATA}, e dopo di essa vediamo le stringhe testuali.

Analizzando l'eseguibile dell'esempio precedente con \IDA, vedremo qualcosa del genere dove il valore di \IT{x} non era stato impostato:

\lstinputlisting[caption=\IDA,style=customasmx86]{patterns/04_scanf/2_global/IDA.lst}

\TT{\_x} è contrassegnata con il simbolo \TT{?} insiema lresto delle variabili che non necessitano di essere inizializzate.
Ciò implica che dopo il caricamento del .exe in memoria, verrà allocato spazio riempito di zeri per tutte queste variabili \InSqBrackets{\CNineNineStd 6.7.8p10}.
Ma nel file .exe tutte le variabili non inizializzate non occupano alcuno spazio.
Questo risulta molto conveniente ad esempio nel caso di array molto grandi.

\clearpage
\subsection{MSVC: x86 + \olly}
\index{\olly}

\RU{Тут даже проще}\EN{Things are even simpler here}:

\begin{figure}[H]
\centering
\includegraphics[scale=\FigScale]{patterns/04_scanf/2_global/ex2_olly_1.png}
\caption{\olly: \RU{после исполнения \scanf}\EN{after \scanf execution}}
\label{fig:scanf_ex2_olly_1}
\end{figure}

\RU{Переменная хранится в сегменте данных.}
\EN{The variable is located in the data segment.}
\RU{Кстати, после исполнения инструкции \PUSH (заталкивающей адрес $x$) адрес появится в стеке, 
и на этом элементе можно нажать правой кнопкой, выбрать \q{Follow in dump}.}
\EN{After the \PUSH instruction (pushing the address of $x$) gets executed, 
the address appears in the stack window. Right-click on that row and select \q{Follow in dump}.}
\RU{И в окне памяти слева появится эта переменная.}
\EN{The variable will appear in the memory window on the left.}

\RU{После того как в консоли введем 123, здесь появится}\EN{After we have entered 123 in the console,} 
\TT{0x7B}\EN{ appears in the memory window (see the highlighted screenshot regions)}.

\RU{Почему самый первый байт это}\EN{But why is the first byte} \TT{7B}?
\RU{По логике вещей, здесь должно было бы быть}\EN{Thinking logically,} \TT{00 00 00 7B}\EN{ should be
there}.
\RU{Это называется}\EN{The cause for this is referred as } \gls{endianness}, \RU{и в x86 принят формат }\EN{and x86 uses }\IT{little-endian}.
\RU{Это означает, что в начале записывается самый младший байт, а заканчивается самым старшим байтом}%
\EN{This implies that the lowest byte is written first, and the highest written last}.
\RU{Больше об этом}\EN{Read more about it at}: \myref{sec:endianness}.

\RU{Позже из этого места в памяти 32-битное значение загружается в \EAX и передается в}
\EN{Back to the example, the 32-bit value is loaded from this memory address into \EAX and passed to} \printf.

\RU{Адрес переменной $x$ в памяти}\EN{The memory address of $x$ is} \TT{0x00C53394}.

\clearpage
\RU{В \olly{} мы можем посмотреть карту памяти процесса (Alt-M) и увидим, что этот адрес
внутри PE-сегмента \TT{.data} нашей программы}%
\EN{In \olly we can review the process memory map (Alt-M)
and we can see that this address is inside the \TT{.data} PE-segment of our program}:

\begin{figure}[H]
\centering
\includegraphics[scale=\FigScale]{patterns/04_scanf/2_global/ex2_olly_2.png}
\caption{\olly: \RU{карта памяти процесса}\EN{process memory map}}
\label{fig:scanf_ex2_olly_2}
\end{figure}


\subsubsection{GCC: x86}

\myindex{ELF}
La situazione in Linux è pressoché identica, con la differenza che le variabili non inizializzate sono collocate nel segmento \TT{\_bss}. 
In un file \ac{ELF} questo segmento ha i seguenti attributi:

\begin{lstlisting}
; Segment type: Uninitialized
; Segment permissions: Read/Write
\end{lstlisting}

Se invece si inizializza la variabile con un qualunque valore, es. 10, 
sarà collocata nel segmento \TT{\_data}, che ha i seguenti attributi:

\begin{lstlisting}
; Segment type: Pure data
; Segment permissions: Read/Write
\end{lstlisting}

\subsubsection{MSVC: x64}

\lstinputlisting[caption=MSVC 2012 x64,style=customasmx86]{patterns/04_scanf/2_global/ex2_MSVC_x64_EN.asm}

Il codice è pressoché identico a quello in x86.
Si noti che l'indirizzo della variabile $x$ è passato a \TT{scanf()} usando un'istruzione \LEA ,
mentre il valore della variabile è passato alla seconda \printf usando un'istruzione \MOV.
\TT{DWORD PTR}--- è parte del linguaggio assembly (non ha a che vedere con il codice macchina),
indica che la dimensione del dato della variabile è 32-bit e l'istruzione \MOV deve essere codificata in accordo alla dimensione.

