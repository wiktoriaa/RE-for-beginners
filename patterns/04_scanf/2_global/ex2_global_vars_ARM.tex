\subsection{ARM: \OptimizingKeilVI (\ThumbMode)}

\begin{lstlisting}
.text:00000000 ; Segment type: Pure code
.text:00000000                 AREA .text, CODE
...
.text:00000000 main
.text:00000000                 PUSH    {R4,LR}
.text:00000002                 ADR     R0, aEnterX     ; "Enter X:\n"
.text:00000004                 BL      __2printf
.text:00000008                 LDR     R1, =x
.text:0000000A                 ADR     R0, aD          ; "%d"
.text:0000000C                 BL      __0scanf
.text:00000010                 LDR     R0, =x
.text:00000012                 LDR     R1, [R0]
.text:00000014                 ADR     R0, aYouEnteredD___ ; "You entered %d...\n"
.text:00000016                 BL      __2printf
.text:0000001A                 MOVS    R0, #0
.text:0000001C                 POP     {R4,PC}
...
.text:00000020 aEnterX         DCB "Enter X:",0xA,0    ; DATA XREF: main+2
.text:0000002A                 DCB    0
.text:0000002B                 DCB    0
.text:0000002C off_2C          DCD x                   ; DATA XREF: main+8
.text:0000002C                                         ; main+10
.text:00000030 aD              DCB "%d",0              ; DATA XREF: main+A
.text:00000033                 DCB    0
.text:00000034 aYouEnteredD___ DCB "You entered %d...",0xA,0 ; DATA XREF: main+14
.text:00000047                 DCB 0
.text:00000047 ; .text         ends
.text:00000047
...
.data:00000048 ; Segment type: Pure data
.data:00000048                 AREA .data, DATA
.data:00000048                 ; ORG 0x48
.data:00000048                 EXPORT x
.data:00000048 x               DCD 0xA                 ; DATA XREF: main+8
.data:00000048                                         ; main+10
.data:00000048 ; .data         ends
\end{lstlisting}

\RU{Итак, переменная \TT{x} теперь глобальная, и она расположена, почему-то, в другом сегменте, а именно сегменте данных}
\EN{So, the \TT{x} variable is now global and for this reason located in another segment, namely the data segment} (\IT{.data}).
\RU{Можно спросить, почему текстовые строки расположены в сегменте кода (\IT{.text}), а \TT{x} нельзя было разместить тут же?}
\EN{One could ask, why are the text strings located in the code segment (\IT{.text}) and \TT{x} is located right here?}
\RU{Потому что эта переменная, и как следует из определения, она может меняться. И может быть, меняться часто.}
\EN{Because it is a variable and by definition its value could change. Moreover it could possibly change often.}
\RU{Ну а текстовые строки имеют тип констант, они не будут меняться, поэтому они располагаются в сегменте \IT{.text}.}
\EN{While text strings has constant type, they will not be changed, so they are located in the \IT{.text} segment.}
\index{\RAM}
\index{\ROM}
\RU{Сегмент кода иногда может быть расположен в ПЗУ микроконтроллера (не забывайте, 
мы сейчас имеем дело с embedded-микроэлектроникой, где дефицит памяти~--- обычное дело),
а изменяемые переменные~--- в ОЗУ.}
\EN{The code segment might sometimes be located in a \ac{ROM} chip (remember, we now deal
with embedded microelectronics, and memory scarcity is common here), and changeable 
variables~---in \ac{RAM}.}
\RU{Хранить в ОЗУ неизменяемые данные, когда в наличии есть ПЗУ, не экономно.}
\EN{It is not very economical to store constant variables in RAM when you have ROM.}
\RU{К тому же, сегмент данных в ОЗУ с константами нужно инициализировать перед работой,
ведь, после включения ОЗУ, очевидно, она содержит в себе случайную информацию.}
\EN{Furthermore, constant variables in RAM must be initialized, because after powering on, the RAM, obviously, contains random information.}

\index{\RU{Компоновщик}\EN{Linker}}
\RU{Далее мы видим в сегменте кода хранится указатель на переменную \TT{x} (\TT{off\_2C}) и 
все операции с переменной происходят через этот указатель.}
\EN{Moving forward, we see a pointer to the \TT{x} (\TT{off\_2C}) variable in the code segment, and that all
operations with the variable occur via this pointer.}
\RU{Это связано с тем, что переменная \TT{x} может быть расположена где-то довольно далеко от 
данного участка кода, так что её адрес нужно сохранить в непосредственной близости к этому коду.}
\EN{That is because the \TT{x} variable could be located somewhere far from this particular code fragment, so its address
must be saved somewhere in close proximity to the code.}
\index{ARM!\Instructions!LDR}
\RU{Инструкция \TT{LDR} в Thumb-режиме может адресовать только переменные в пределах вплоть 
до 1020 байт от своего местоположения.}
\EN{The \TT{LDR} instruction in Thumb mode can only address variables in a range of 1020 bytes from its location, }
\RU{Эта же инструкция в ARM-режиме~--- переменные в пределах $\pm{}4095$ байт.}
\EN{and in in ARM-mode~---variables in range of $\pm{}4095$ bytes.}
\RU{Таким образом,
адрес глобальной переменной \TT{x} нужно расположить в непосредственной близости, ведь нет никакой гарантии, 
что компоновщик\footnote{linker в англоязычной литературе} сможет разместить саму переменную где-то рядом, 
она может быть даже в другом чипе памяти!}
\EN{And so the address of the \TT{x} variable
must be located somewhere in close proximity, because there is no guarantee that the linker would be able to accommodate the variable somewhere nearby the code, it may well be even in an external memory chip!}

\index{\CLanguageElements!const}
\index{\ROM}
\RU{Ещё одна вещь: если переменную объявить как \IT{const}, то компилятор Keil разместит её в 
сегменте \TT{.constdata}.}
\EN{One more thing: if a variable is declared as \IT{const}, the Keil compiler allocates it in 
the \TT{.constdata} segment.}
\RU{Должно быть, впоследствии компоновщик и этот сегмент сможет разместить в ПЗУ вместе
с сегментом кода.}
\EN{Perhaps, thereafter, the linker could place this segment in ROM too, along with the code segment.}

\subsection{ARM64}

\lstinputlisting[caption=\NonOptimizing GCC 4.9.1 ARM64,numbers=left]{patterns/04_scanf/2_global/ARM64_GCC491_O0.s.\LANG}

\index{ARM!\Instructions!ADRP/ADD pair}
\RU{Теперь $x$ это глобальная переменная, и её адрес вычисляется при помощи пары инструкций ADRP/ADD 
(строки 21 и 25).}
\EN{In this case the $x$ variable is declared as global and its address is calculated using 
the ADRP/ADD instruction pair (lines 21 and 25).}
