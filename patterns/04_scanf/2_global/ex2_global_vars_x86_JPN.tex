\subsubsection{MSVC: x86}

\lstinputlisting[style=customasmx86]{patterns/04_scanf/2_global/ex2_MSVC.asm}

この場合、\IT{x}変数は\TT{\_DATA}セグメントに定義され、ローカルスタックにはメモリは割り当てられません。 スタックからではなく、直接アクセスされます。 
初期化されていないグローバル変数は、実行可能ファイルにスペースを入れません
(なぜ、最初に変数をゼロに設定する必要があるのでしょうか?)。
しかし、誰かが自分のアドレスにアクセスすると、\ac{OS}は0で初期化されたブロック\footnote{これが\ac{VM}の動作です}を割り当てます。

変数に明示的に値を割り当てましょう:

\lstinputlisting[style=customc]{patterns/04_scanf/2_global/default_value_EN.c}

以下を得ます。

\begin{lstlisting}[style=customasmx86]
_DATA	SEGMENT
_x	DD	0aH

...
\end{lstlisting}

ここでは、この変数のDWORDタイプの値\TT{0xA}(DDはDWORD = 32ビットを表します)が表示されます。

\IDA にコンパイルされた.exeを開くと、\TT{\_DATA}セグメントの先頭に\IT{x}変数が配置されていて、
その後にテキスト文字列が表示されます。

\IT{x}の値が設定されていない前の例のコンパイル済み.exeを \IDA で開くと、次のように表示されます。

\lstinputlisting[caption=\IDA,style=customasmx86]{patterns/04_scanf/2_global/IDA.lst}

\TT{\_x}に\TT{?}がマークされていると、残りの変数は初期化する必要はありません。 
これは、メモリに.exeをロードした後、これらすべての変数のための領域が割り当てられ、
0で満たされる\InSqBrackets{\CNineNineStd 6.7.8p10}ことを意味します。 
しかし、.exeファイルでは、これらの初期化されていない変数は何も占有しません。 
これは、例えば、大きな配列の場合に便利です。

\clearpage
\subsection{MSVC: x86 + \olly}
\index{\olly}

\RU{Тут даже проще}\EN{Things are even simpler here}:

\begin{figure}[H]
\centering
\includegraphics[scale=\FigScale]{patterns/04_scanf/2_global/ex2_olly_1.png}
\caption{\olly: \RU{после исполнения \scanf}\EN{after \scanf execution}}
\label{fig:scanf_ex2_olly_1}
\end{figure}

\RU{Переменная хранится в сегменте данных.}
\EN{The variable is located in the data segment.}
\RU{Кстати, после исполнения инструкции \PUSH (заталкивающей адрес $x$) адрес появится в стеке, 
и на этом элементе можно нажать правой кнопкой, выбрать \q{Follow in dump}.}
\EN{After the \PUSH instruction (pushing the address of $x$) gets executed, 
the address appears in the stack window. Right-click on that row and select \q{Follow in dump}.}
\RU{И в окне памяти слева появится эта переменная.}
\EN{The variable will appear in the memory window on the left.}

\RU{После того как в консоли введем 123, здесь появится}\EN{After we have entered 123 in the console,} 
\TT{0x7B}\EN{ appears in the memory window (see the highlighted screenshot regions)}.

\RU{Почему самый первый байт это}\EN{But why is the first byte} \TT{7B}?
\RU{По логике вещей, здесь должно было бы быть}\EN{Thinking logically,} \TT{00 00 00 7B}\EN{ should be
there}.
\RU{Это называется}\EN{The cause for this is referred as } \gls{endianness}, \RU{и в x86 принят формат }\EN{and x86 uses }\IT{little-endian}.
\RU{Это означает, что в начале записывается самый младший байт, а заканчивается самым старшим байтом}%
\EN{This implies that the lowest byte is written first, and the highest written last}.
\RU{Больше об этом}\EN{Read more about it at}: \myref{sec:endianness}.

\RU{Позже из этого места в памяти 32-битное значение загружается в \EAX и передается в}
\EN{Back to the example, the 32-bit value is loaded from this memory address into \EAX and passed to} \printf.

\RU{Адрес переменной $x$ в памяти}\EN{The memory address of $x$ is} \TT{0x00C53394}.

\clearpage
\RU{В \olly{} мы можем посмотреть карту памяти процесса (Alt-M) и увидим, что этот адрес
внутри PE-сегмента \TT{.data} нашей программы}%
\EN{In \olly we can review the process memory map (Alt-M)
and we can see that this address is inside the \TT{.data} PE-segment of our program}:

\begin{figure}[H]
\centering
\includegraphics[scale=\FigScale]{patterns/04_scanf/2_global/ex2_olly_2.png}
\caption{\olly: \RU{карта памяти процесса}\EN{process memory map}}
\label{fig:scanf_ex2_olly_2}
\end{figure}


\subsubsection{GCC: x86}

\myindex{ELF}
Linuxの画像はほぼ同じですが、初期化されていない変数は\TT{\_bss}セグメントにあります。 
\ac{ELF}ファイルでは、このセグメントには次の属性があります。

\begin{lstlisting}
; Segment type: Uninitialized
; Segment permissions: Read/Write
\end{lstlisting}

ただし、変数をある値で初期化してください。 
10の場合、次の属性を持つ\TT{\_data}セグメントに配置されます。

\begin{lstlisting}
; Segment type: Pure data
; Segment permissions: Read/Write
\end{lstlisting}

\subsubsection{MSVC: x64}

\lstinputlisting[caption=MSVC 2012 x64,style=customasmx86]{patterns/04_scanf/2_global/ex2_MSVC_x64_EN.asm}

コードはx86とほとんど同じです。 
$x$変数のアドレスは、 \LEA 命令を使用して \scanf に渡され、
変数の値は \MOV 命令を使用して2番目の \printf に渡されることに注意してください。 
\TT{DWORD PTR}はアセンブリ言語の一部であり(マシンコードと無関係)、
可変データサイズが32ビットであり、 \MOV 命令がそれに応じてエンコードされなければならないことを示します。
