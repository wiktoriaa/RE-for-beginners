\subsubsection{MSVC: x86}

\lstinputlisting[style=customasmx86]{patterns/04_scanf/2_global/ex2_MSVC.asm}

In this case the \TT{x} variable is defined in the \TT{\_DATA} segment and no memory is allocated in the local stack. It is accessed directly, not through the stack. 
Uninitialized global variables take no space in the executable file
(indeed, why one needs to allocate space for variables initially set to zero?), 
but when someone accesses their address, 
the \ac{OS} will allocate a block of zeros there\footnote{That is how a \ac{VM} behaves}.

Now let's explicitly assign a value to the variable:

\lstinputlisting[style=customc]{patterns/04_scanf/2_global/default_value_EN.c}

We got:

\begin{lstlisting}[style=customasmx86]
_DATA	SEGMENT
_x	DD	0aH

...
\end{lstlisting}

Here we see a value \TT{0xA} of DWORD type (DD stands for DWORD = 32 bit) for this variable.

If you open the compiled .exe in \IDA, you can see the \IT{x} variable placed at the beginning of 
the \TT{\_DATA} segment, and after it you can see text strings.

If you open the compiled .exe from the previous example in \IDA, where the value of \IT{x} hasn't been set, you would see something like this:

\lstinputlisting[caption=\IDA,style=customasmx86]{patterns/04_scanf/2_global/IDA.lst}

\TT{\_x} is marked with \TT{?} with the rest of the variables that do not need to be initialized. 
This implies that after loading the .exe to the memory, a space for all these variables is to be 
allocated and filled with zeros \InSqBrackets{\CNineNineStd 6.7.8p10}.
But in the .exe file these uninitialized variables do not occupy anything.
This is convenient for large arrays, for example.

\clearpage
\subsection{MSVC: x86 + \olly}
\index{\olly}

\RU{Тут даже проще}\EN{Things are even simpler here}:

\begin{figure}[H]
\centering
\includegraphics[scale=\FigScale]{patterns/04_scanf/2_global/ex2_olly_1.png}
\caption{\olly: \RU{после исполнения \scanf}\EN{after \scanf execution}}
\label{fig:scanf_ex2_olly_1}
\end{figure}

\RU{Переменная хранится в сегменте данных.}
\EN{The variable is located in the data segment.}
\RU{Кстати, после исполнения инструкции \PUSH (заталкивающей адрес $x$) адрес появится в стеке, 
и на этом элементе можно нажать правой кнопкой, выбрать \q{Follow in dump}.}
\EN{After the \PUSH instruction (pushing the address of $x$) gets executed, 
the address appears in the stack window. Right-click on that row and select \q{Follow in dump}.}
\RU{И в окне памяти слева появится эта переменная.}
\EN{The variable will appear in the memory window on the left.}

\RU{После того как в консоли введем 123, здесь появится}\EN{After we have entered 123 in the console,} 
\TT{0x7B}\EN{ appears in the memory window (see the highlighted screenshot regions)}.

\RU{Почему самый первый байт это}\EN{But why is the first byte} \TT{7B}?
\RU{По логике вещей, здесь должно было бы быть}\EN{Thinking logically,} \TT{00 00 00 7B}\EN{ should be
there}.
\RU{Это называется}\EN{The cause for this is referred as } \gls{endianness}, \RU{и в x86 принят формат }\EN{and x86 uses }\IT{little-endian}.
\RU{Это означает, что в начале записывается самый младший байт, а заканчивается самым старшим байтом}%
\EN{This implies that the lowest byte is written first, and the highest written last}.
\RU{Больше об этом}\EN{Read more about it at}: \myref{sec:endianness}.

\RU{Позже из этого места в памяти 32-битное значение загружается в \EAX и передается в}
\EN{Back to the example, the 32-bit value is loaded from this memory address into \EAX and passed to} \printf.

\RU{Адрес переменной $x$ в памяти}\EN{The memory address of $x$ is} \TT{0x00C53394}.

\clearpage
\RU{В \olly{} мы можем посмотреть карту памяти процесса (Alt-M) и увидим, что этот адрес
внутри PE-сегмента \TT{.data} нашей программы}%
\EN{In \olly we can review the process memory map (Alt-M)
and we can see that this address is inside the \TT{.data} PE-segment of our program}:

\begin{figure}[H]
\centering
\includegraphics[scale=\FigScale]{patterns/04_scanf/2_global/ex2_olly_2.png}
\caption{\olly: \RU{карта памяти процесса}\EN{process memory map}}
\label{fig:scanf_ex2_olly_2}
\end{figure}


\subsubsection{GCC: x86}

\myindex{ELF}
The picture in Linux is near the same, with the difference that the uninitialized variables are located in the \TT{\_bss} segment. 
In \ac{ELF} file this segment has the following attributes:

\begin{lstlisting}
; Segment type: Uninitialized
; Segment permissions: Read/Write
\end{lstlisting}

If you, however, initialize the variable with some value e.g. 10, 
it is to be placed in the \TT{\_data} segment, which has the following attributes:

\begin{lstlisting}
; Segment type: Pure data
; Segment permissions: Read/Write
\end{lstlisting}

\subsubsection{MSVC: x64}

\lstinputlisting[caption=MSVC 2012 x64,style=customasmx86]{patterns/04_scanf/2_global/ex2_MSVC_x64_EN.asm}

The code is almost the same as in x86.
Please note that the address of the $x$ variable is passed to \TT{scanf()} using a \LEA instruction,
while the variable's value is passed to the second \printf using a \MOV instruction.
\TT{DWORD PTR}---is a part of the assembly language (no relation to the machine code),
indicating that the variable data size is 32-bit and the \MOV instruction has to be encoded accordingly.

