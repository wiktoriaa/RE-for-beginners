\subsubsection{MSVC: x86}

\lstinputlisting[style=customasmx86]{patterns/04_scanf/2_global/ex2_MSVC.asm}

В целом ничего особенного. Теперь \TT{x} объявлена в сегменте \TT{\_DATA}. 
Память для неё в стеке более не выделяется.
Все обращения к ней происходит не через стек, а уже напрямую. 
Неинициализированные глобальные переменные не занимают места в исполняемом файле
(и действительно, зачем в исполняемом файле
нужно выделять место под изначально нулевые переменные?), но тогда, когда к этому месту в памяти
кто-то обратится, \ac{OS} подставит туда блок, состоящий из нулей\footnote{Так работает \ac{VM}}.

Попробуем изменить объявление этой переменной:

\lstinputlisting[style=customc]{patterns/04_scanf/2_global/default_value_RU.c}

Выйдет в итоге:

\begin{lstlisting}[style=customasmx86]
_DATA	SEGMENT
_x	DD	0aH

...
\end{lstlisting}

Здесь уже по месту этой переменной записано \TT{0xA} с типом DD (dword = 32 бита).

Если вы откроете скомпилированный .exe-файл в \IDA, то увидите, что \IT{x} 
находится в начале сегмента \TT{\_DATA}, после этой переменной будут текстовые строки.

А вот если вы откроете в \IDA .exe скомпилированный в прошлом примере, где значение \IT{x} не определено, то вы увидите:

\lstinputlisting[caption=\IDA,style=customasmx86]{patterns/04_scanf/2_global/IDA.lst}

\TT{\_x} обозначен как \TT{?}, наряду с другими переменными не требующими инициализации. 
Это означает, что при загрузке .exe в память, место под всё это выделено будет и будет заполнено
нулевыми байтами \InSqBrackets{\CNineNineStd 6.7.8p10}.
Но в самом .exe ничего этого нет. Неинициализированные переменные не занимают места в исполняемых файлах. 
Это удобно для больших массивов, например.

\clearpage
\subsection{MSVC: x86 + \olly}
\index{\olly}

\RU{Тут даже проще}\EN{Things are even simpler here}:

\begin{figure}[H]
\centering
\includegraphics[scale=\FigScale]{patterns/04_scanf/2_global/ex2_olly_1.png}
\caption{\olly: \RU{после исполнения \scanf}\EN{after \scanf execution}}
\label{fig:scanf_ex2_olly_1}
\end{figure}

\RU{Переменная хранится в сегменте данных.}
\EN{The variable is located in the data segment.}
\RU{Кстати, после исполнения инструкции \PUSH (заталкивающей адрес $x$) адрес появится в стеке, 
и на этом элементе можно нажать правой кнопкой, выбрать \q{Follow in dump}.}
\EN{After the \PUSH instruction (pushing the address of $x$) gets executed, 
the address appears in the stack window. Right-click on that row and select \q{Follow in dump}.}
\RU{И в окне памяти слева появится эта переменная.}
\EN{The variable will appear in the memory window on the left.}

\RU{После того как в консоли введем 123, здесь появится}\EN{After we have entered 123 in the console,} 
\TT{0x7B}\EN{ appears in the memory window (see the highlighted screenshot regions)}.

\RU{Почему самый первый байт это}\EN{But why is the first byte} \TT{7B}?
\RU{По логике вещей, здесь должно было бы быть}\EN{Thinking logically,} \TT{00 00 00 7B}\EN{ should be
there}.
\RU{Это называется}\EN{The cause for this is referred as } \gls{endianness}, \RU{и в x86 принят формат }\EN{and x86 uses }\IT{little-endian}.
\RU{Это означает, что в начале записывается самый младший байт, а заканчивается самым старшим байтом}%
\EN{This implies that the lowest byte is written first, and the highest written last}.
\RU{Больше об этом}\EN{Read more about it at}: \myref{sec:endianness}.

\RU{Позже из этого места в памяти 32-битное значение загружается в \EAX и передается в}
\EN{Back to the example, the 32-bit value is loaded from this memory address into \EAX and passed to} \printf.

\RU{Адрес переменной $x$ в памяти}\EN{The memory address of $x$ is} \TT{0x00C53394}.

\clearpage
\RU{В \olly{} мы можем посмотреть карту памяти процесса (Alt-M) и увидим, что этот адрес
внутри PE-сегмента \TT{.data} нашей программы}%
\EN{In \olly we can review the process memory map (Alt-M)
and we can see that this address is inside the \TT{.data} PE-segment of our program}:

\begin{figure}[H]
\centering
\includegraphics[scale=\FigScale]{patterns/04_scanf/2_global/ex2_olly_2.png}
\caption{\olly: \RU{карта памяти процесса}\EN{process memory map}}
\label{fig:scanf_ex2_olly_2}
\end{figure}


\subsubsection{GCC: x86}

\myindex{ELF}
В Linux всё почти также. За исключением того, что если значение \TT{x} не определено, 
то эта переменная будет находится в сегменте \TT{\_bss}.
В \ac{ELF} этот сегмент имеет такие атрибуты:

\begin{lstlisting}
; Segment type: Uninitialized
; Segment permissions: Read/Write
\end{lstlisting}

Ну а если сделать статическое присвоение этой переменной какого-либо
значения, например, 10, то она будет находится 
в сегменте \TT{\_data},
это сегмент с такими атрибутами:

\begin{lstlisting}
; Segment type: Pure data
; Segment permissions: Read/Write
\end{lstlisting}

\subsubsection{MSVC: x64}

\lstinputlisting[caption=MSVC 2012 x64,style=customasmx86]{patterns/04_scanf/2_global/ex2_MSVC_x64_RU.asm}

Почти такой же код как и в x86.
Обратите внимание что для \TT{scanf()} адрес переменной $x$ передается
при помощи инструкции \LEA, а во второй \printf передается само значение переменной при помощи \MOV.
\TT{DWORD PTR} --- это часть языка ассемблера (не имеющая отношения к машинным кодам) показывающая, что тип переменной в памяти именно 32-битный, 
и инструкция \MOV должна быть здесь закодирована соответственно.

