\section{\Example{} \#6}

\lstinputlisting{patterns/win16/ex6.c}

\lstinputlisting{patterns/win16/ex6.lst}

\index{\CStandardLibrary!time()}
\index{\CStandardLibrary!localtime()}
\RU{Время в формате UNIX это 32-битное значение, так что оно возвращается в паре регистров \TT{DX:AX} и сохраняется
в двух локальны 16-битных переменных}
\EN{UNIX time is 32-bit value, so it is returned in \TT{DX:AX} register pair and stored into two local 16-bit variables}.
\RU{Потом указатель на эту пару передается в ф-цию}\EN{Then a pointer to the pair is passed to}
\TT{localtime()}\EN{ function}.
\RU{Ф-ция}\EN{The} \TT{localtime()} \RU{имеет структуру}\EN{function has} \TT{struct tm} \RU{расположенную у себя
где-то внутри, так что только указатель на нее возвращается}
\EN{allocated somewhere in guts of the C library, so only pointer to it is returned}. 
\RU{Кстати, это также означает что функцию нельзя вызывать еще раз, пока её результаты не были использованы}
\EN{By the way, this is also means that the function cannot be called again until its results are used}.

\RU{Для ф-ций}\EN{For the} \TT{time()} \AndENRU \TT{localtime()} \RU{используется
Watcom-соглашение о вызовах: первые четыре аргумента передаются через регистры}
\EN{functions, a Watcom calling convention is used here:
first four arguments are passed in} \TT{AX}, \TT{DX}, \TT{BX} \AndENRU \TT{CX}, \RU{а остальные аргументы через стек}
\EN{registers, all the rest arguments are via stack}.
\RU{Ф-ции, использующие это соглашение, маркируется символом подчеркивания в конце имени}
\EN{Functions used this convention are also marked by underscore at the end of name}.

\RU{Для вызова ф-ции }\TT{sprintf()} \RU{используется обычное соглашение \IT{cdecl} (\ref{cdecl}) вместо 
\TT{PASCAL} или Watcom, так что аргументы передаются привычным образом}
\EN{does not use \TT{PASCAL} calling convention, nor Watcom one,
so the arguments are passed in usual \IT{cdecl} way (\ref{cdecl})}.

\subsection{\RU{Глобальные переменные}\EN{Global variables}}

\RU{Это тот же пример, только переменные теперь глобальные}
\EN{This is the same example, but now these variables are global}:

\lstinputlisting{patterns/win16/ex6_global.c}

\lstinputlisting{patterns/win16/ex6_global.lst}

\TT{t} \RU{не будет использоваться, но компилятор создал код, записывающий в эту переменную}
\EN{will not be used, but compiler emitted the code which stores the value}.
\RU{Потому что он не уверен, может быть это значение где-то еще будет прочитано}
\EN{Because it is not sure, maybe that value will be eventually used somewhere}.
