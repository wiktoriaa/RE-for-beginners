\ifx\RUSSIAN\undefined
\section{MIPS}

\lstinputlisting[caption=\Optimizing GCC 4.4.5]{patterns/05_passing_arguments/MIPS_O3_IDA.lst}

First four function arguments are passed in four registers prefixed by A-.

\index{MIPS!\Instructions!MULT}
There are two special registers in MIPS: HI and LO which are filled by 64-bit result of multiplication while
execution of \TT{MULT} instruction.
\index{MIPS!\Instructions!MFLO}
\index{MIPS!\Instructions!MFHI}
Registers are accessible only using \TT{MFLO} and \TT{MFHI} instructions.
\TT{MFLO} here is taking result of multiplication and putting it into \$v0.

So high 32-bit part of multiplication result is dropped (contents of HI register is not used).
Indeed: we work with 32-bit \Tint data type here.

\index{MIPS!\Instructions!ADDU}
Finally, \TT{ADDU} (``Add Unsigned'') adds value of the third argument to the result.

\index{MIPS!\Instructions!ADD}
\index{MIPS!\Instructions!ADDU}
\index{Ada}
\index{Integer overflow}
There are two different addition instructions in MIPS: \TT{ADD} and \TT{ADDU}.
In fact, it's not about signedness, but about exceptions: \TT{ADD} can raise exception on overflow,
which is sometimes useful\footnote{\url{http://blog.regehr.org/archives/1154}} and supported in Ada, for instance.
\TT{ADDU} do not raise exceptions on overflow.
Since, \CCpp doesn't support this, here we see \TT{ADDU}.

32-bit result is leaved in \$v0.

\index{MIPS!\Instructions!JAL}
\index{MIPS!\Instructions!JALR}
There are new instruction for us in \main: \TT{JAL} (``Jump and Link''). 
Difference between JAL and JALR is that relative offset is encoded in first instruction, 
while JALR jumping to the absolute address stored in register (``Jump and Link Register'').
Both \ttf and \main functions are located in the same object file, so relative address of \ttf 
is known and fixed.

\fi
