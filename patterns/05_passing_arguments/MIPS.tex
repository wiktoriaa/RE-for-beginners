\ifx\RUSSIAN\undefined
\section{MIPS}

\lstinputlisting[caption=\Optimizing GCC 4.4.5]{patterns/05_passing_arguments/MIPS_O3_IDA.lst}

First four function arguments are passed in four registers prefixed by A-.

\index{MIPS!\Instructions!mult}
There are two special registers in MIPS: HI and LO which are filled by 64-bit result of multiplication while
execution of \TT{mult} instruction.
\index{MIPS!\Instructions!mflo}
\index{MIPS!\Instructions!mfhi}
Registers are accessible only using \TT{mflo} and \TT{mfhi} instructions.
\TT{mflo} here is taking result of multiplication and putting it into \$v0.

So high 32-bit part of multiplication result is dropped (contents of HI register is not used).
Indeed: we work with 32-bit \Tint data type here.

\index{MIPS!\Instructions!addu}
Finally, \TT{addu} (``Add Unsigned'') adds value of the third argument to the result.

\index{MIPS!\Instructions!add}
\index{MIPS!\Instructions!addu}
\index{Ada}
\index{Integer overflow}
There are two different addition instructions in MIPS: \TT{add} and \TT{addu}.
In fact, it's not about signedness, but about exceptions: \TT{add} can raise exception on overflow,
which is sometimes useful\footnote{\url{http://blog.regehr.org/archives/1154}} and supported in Ada, for instance.
\TT{addu} do not raise exceptions on overflow.
Since, \CCpp doesn't support this, here we see \TT{addu}.

32-bit result is leaved in \$v0.

\fi
