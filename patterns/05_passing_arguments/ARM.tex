\section{ARM}

\subsection{\NonOptimizingKeilVI + \ARMMode}

\begin{lstlisting}
.text:000000A4 00 30 A0 E1                 MOV     R3, R0
.text:000000A8 93 21 20 E0                 MLA     R0, R3, R1, R2
.text:000000AC 1E FF 2F E1                 BX      LR
...
.text:000000B0             main
.text:000000B0 10 40 2D E9                 STMFD   SP!, {R4,LR}
.text:000000B4 03 20 A0 E3                 MOV     R2, #3
.text:000000B8 02 10 A0 E3                 MOV     R1, #2
.text:000000BC 01 00 A0 E3                 MOV     R0, #1
.text:000000C0 F7 FF FF EB                 BL      f
.text:000000C4 00 40 A0 E1                 MOV     R4, R0
.text:000000C8 04 10 A0 E1                 MOV     R1, R4
.text:000000CC 5A 0F 8F E2                 ADR     R0, aD_0        ; "%d\n"
.text:000000D0 E3 18 00 EB                 BL      __2printf
.text:000000D4 00 00 A0 E3                 MOV     R0, #0
.text:000000D8 10 80 BD E8                 LDMFD   SP!, {R4,PC}
\end{lstlisting}

\RU{В функции \main просто вызываются две функции, в первую (\TT{f}) передается три значения.}
\EN{In \main function, two other functions are simply called, and three values are passed to the 
first one (\TT{f}).}

\RU{Как я уже упоминал, первые 4 значения, в ARM обычно передаются в первых 4-х регистрах}
\EN{As I mentioned before, in ARM, first 4 values are usually passed in first 4 registers} (\Reg{0}-\Reg{3}).

\RU{Функция }\TT{f}\RU{, как видно, использует три первых регистра (\Reg{0}-\Reg{2}) как аргументы.}
\EN{function, as it seems, use first 3 registers (\Reg{0}-\Reg{2}) as arguments.}

\index{ARM!\Instructions!MLA}
\RU{Инструкция }\TT{MLA} (\IT{Multiply Accumulate}) \RU{перемножает два первых операнда (\Reg{3} и \Reg{1}), 
прибавляет к произведению
третий операнд (\Reg{2}) и помещает результат в нулевой операнд (\Reg{0}), через который, по стандарту, 
возвращаются значения функций.}
\EN{instruction multiplicates two first operands (\Reg{3} and \Reg{1}), adds third operand (\Reg{2}) to product and places
result into zeroth operand (\Reg{0}), via which, by standard, values are returned from functions.}

\index{Fused multiply–add}
\RU{Умножение и сложение одновременно}\EN{Multiplication and addition at once}\footnote{\WPMAO} 
(\IT{Fused multiply–add}) \RU{это много где применяемая операция, кстати, аналогичной
инструкции в x86 нет}\EN{is very useful operation, by the way, there is no such instruction in x86}, 
\RU{если не считать новых FMA-инструкций}\EN{if not to count new FMA-instruction}\footnote{\url{https://en.wikipedia.org/wiki/FMA_instruction_set}} \InENRU SIMD.

\RU{Самая первая инструкция}\EN{The very first} \TT{MOV R3, R0}, \RU{по-видимому, избыточна (можно было бы обойтись только одной инструкцией \TT{MLA})}
\EN{instruction, apparently, redundant (single \TT{MLA} instruction could be used here instead)}, 
\RU{компилятор не оптимизировал её, ведь, это компиляция без оптимизации}\EN{compiler was not optimized it,
since this is non-optimizing compilation}.

\index{ARM!\RU{Переключение режимов}\EN{Mode switching}}
\index{ARM!\Instructions!BX}
\RU{Инструкция \TT{BX} возвращает управление по адресу записанному в \ac{LR} и, если нужно, 
переключает режимы процессора с thumb на ARM или наоборот.}
\EN{\TT{BX} instruction returns control to the address stored in the \ac{LR} register and, if it is necessary, 
switches processor mode from thumb to ARM or vice versa.}
\RU{Это может быть необходимым потому, что, как мы видим, 
функции \TT{f} неизвестно, из какого кода она будет вызываться, из ARM или thumb.}
\EN{This can be necessary since, as we can see, \TT{f} function is not aware, from which code it may be
called, from ARM or thumb.}
\RU{Поэтому, если она будет вызываться из кода thumb, \TT{BX} не только вернет
управление в вызывающую функцию, но также переключит процессор в режим thumb.}
\EN{This, if it will be called from thumb code, \TT{BX} will not only return control to the calling function,
but also will switch processor mode to thumb mode.}
\RU{Либо не переключит, если функция вызывалась из кода для режима ARM.}
\EN{Or not switch, if the function was called from ARM code.}

\subsection{\OptimizingKeilVI + \ARMMode}

\begin{lstlisting}[label=ARM_leaf_example1]
.text:00000098             f
.text:00000098 91 20 20 E0                 MLA     R0, R1, R0, R2
.text:0000009C 1E FF 2F E1                 BX      LR
\end{lstlisting}

\RU{А вот и функция \TT{f} скомпилированная компилятором Keil в режиме полной оптимизации}
\EN{And here is \TT{f} function compiled by Keil compiler in full optimization mode} (\Othree).
\RU{Инструкция \MOV была соптимизирована и теперь \TT{MLA} использует все входящие регистры 
и помещает результат в \Reg{0}, как раз, где вызываемая функция будет его читать и использовать.}
\EN{\MOV instruction was optimized (or reduced) and now \TT{MLA} uses all 
input registers and also places result right into \Reg{0},
exactly where calling function will read it and use.}

\subsection{\OptimizingKeilVI + \ThumbMode}

\begin{lstlisting}[label=ARM_leaf_example2]
.text:0000005E 48 43                       MULS    R0, R1
.text:00000060 80 18                       ADDS    R0, R0, R2
.text:00000062 70 47                       BX      LR
\end{lstlisting}

\RU{В режиме thumb, инструкция \TT{MLA} недоступна, так что компилятору пришлось сгенерировать код, 
делающий обе операции по отдельности.}
\EN{\TT{MLA} instruction is not available in thumb mode, so, compiler generates the code doing these two 
operations separately.}
\index{ARM!\Instructions!MULS}
\index{ARM!\Instructions!ADDS}
\RU{Первая инструкция \TT{MULS} умножает \Reg{0} на \Reg{1} оставляя результат в \Reg{1}.}
\EN{First \TT{MULS} instruction multiply \Reg{0} by \Reg{1} leaving result in the \Reg{1} register.}
\RU{Вторая (\TT{ADDS}) складывает результат и \Reg{2}, оставляя результат в \Reg{0}.}
\EN{Second (\TT{ADDS}) instruction adds result and \Reg{2} leaving result in the \Reg{0} register.}

