\subsubsection{ARM64}

\myparagraph{\Optimizing GCC (Linaro) 4.9}

\myindex{Fused multiply–add}
\myindex{ARM!\Instructions!MADD}
Appare tutto molto semplice.
\TT{MADD} e' semplicemente un'istruzione che fa una moltiplicazione/addizione combinata (simile alla \TT{MLA} vista in precedenza).
Tutti i 3 argomenti sono passati tramite le parti a 32-bit dei registri X-.
Infatti gli argomenti sono tutti di tipo \IT{int} a 32-bit.
Il risultato e' restituito in \TT{W0}.

\lstinputlisting[caption=\Optimizing GCC (Linaro) 4.9,style=customasmARM]{patterns/05_passing_arguments/ARM/ARM64_O3_EN.s}

Estendiamo anche questo esempio usando il tipo \TT{uint64\_t} a 64-bit e vediamo che succede:

\lstinputlisting[style=customc]{patterns/05_passing_arguments/ex64.c}

\begin{lstlisting}[style=customasmARM]
f:
	madd	x0, x0, x1, x2
	ret
main:
	mov	x1, 13396
	adrp	x0, .LC8
	stp	x29, x30, [sp, -16]!
	movk	x1, 0x27d0, lsl 16
	add	x0, x0, :lo12:.LC8
	movk	x1, 0x122, lsl 32
	add	x29, sp, 0
	movk	x1, 0x58be, lsl 48
	bl	printf
	mov	w0, 0
	ldp	x29, x30, [sp], 16
	ret

.LC8:
	.string	"%lld\n"
\end{lstlisting}

La funzione \ttf{} e' rimasta invariata, ma adesso i registri a 64-bit X- sono utilizzati nella loro interezza.
I valori grandi a 64-bit sono caricati nei registri per parti, come descritto anche qui: \myref{ARM_big_constants_loading}.

\myparagraph{\NonOptimizing GCC (Linaro) 4.9}

L'output del compilatore non ottimizzante e' piu' ridondante:

\begin{lstlisting}[style=customasmARM]
f:
	sub	sp, sp, #16
	str	w0, [sp,12]
	str	w1, [sp,8]
	str	w2, [sp,4]
	ldr	w1, [sp,12]
	ldr	w0, [sp,8]
	mul	w1, w1, w0
	ldr	w0, [sp,4]
	add	w0, w1, w0
	add	sp, sp, 16
	ret
\end{lstlisting}

Il codice salva gli argomenti in input nello stack locale, nel caso in cui qualcuno (o qualcosa) in questa funzione abbia necessita' 
di usare i registri \TT{W0...W2} 
Questo previene l'eventualita' che gli argomenti originali siano sovrascritti, nel caso in cui servano nuovamente nel corso della funzione.

Cio' e' detto \IT{Register Save Area.} (\ARMPCS) ed e' vagamente simile allo \q{Shadow Space}: \myref{shadow_space}.
La funzione chiamata non e' comunque obbligata a salvarli.

Perche' GCC 4.9 ottimizzante ha eliminato questa porzione di codice che salva gli argomenti?
Lo ha fatto perche', a seguito di ulteriore ottimizzazione, ha concluso che gli argomenti di questa funzione non sono riutilizzati in futuro e
che i registri \TT{W0...W2} non saranno utilizzati.

\myindex{ARM!\Instructions!MUL}
\myindex{ARM!\Instructions!ADD}

Notiamo anche una coppia di istruzioni \TT{MUL}/\TT{ADD} al posto della singola \TT{MADD}.
