\subsection{MIPS}

\lstinputlisting[caption=GCC 4.4.5 \Optimizing,style=customasmMIPS]{patterns/05_passing_arguments/MIPS_O3_IDA_FR.lst}

Les quatre premiers arguments de la fonction sont passés par quatres registres préfixés
par A-.

\myindex{MIPS!\Instructions!MULT}

Il y a deux registres spéciaux en MIPS: HI et LO qui sont remplis avec le résultat
64-bit de la multiplication lors de l'exécution d'une instruction \TT{MULT}.
\myindex{MIPS!\Instructions!MFLO}
\myindex{MIPS!\Instructions!MFHI}

Ces registres sont accessibles seulement en utilisant les instructions \TT{MFLO} et \TT{MFHI}.
Ici \TT{MFLO} prend la partie basse du résultat de la multiplication et le stocke
dans \$V0.
Donc la partie haute du résultat de la multiplication est abandonnée (le contenu
du registre HI n'est pas utilisé).
Effectivement: nous travaillons avec des types de données \Tint 32-bit ici.

\myindex{MIPS!\Instructions!ADDU}

Enfin, \TT{ADDU} (\q{Add Unsigned} addition non signée) ajoute la valeur du troisième
argument au résultat.

\myindex{MIPS!\Instructions!ADD}
\myindex{MIPS!\Instructions!ADDU}
\myindex{Ada}
\myindex{Integer overflow}

Il y a deux instructions différentes pour l'addition en MIPS: \TT{ADD} et \TT{ADDU}.
La différence entre les deux n'est pas relative au fait d'être signé, mais aux
exceptions. \TT{ADD} peut déclencher une exception lors d'un débordement, ce qui
est parfois utile\footnote{\url{http://go.yurichev.com/17326}} et supporté en ADA
\ac{PL}, par exemple.
\TT{ADDU} ne déclenche pas d'exception lors d'un débordement.
Comme \CCpp ne supporte pas ceci, dans notre exemple nous voyons \TT{ADDU} au lieu
de \TT{ADD}.

Le résultat 32-bit est laissé dans \$V0.

\myindex{MIPS!\Instructions!JAL}
\myindex{MIPS!\Instructions!JALR}

Il y a une instruction nouvelle pour nous dans \main: \TT{JAL} (\q{Jump and Link}).

La différence entre \INS{JAL} et \INS{JALR} est qu'un offset relatif est encodé
dans la première instruction, tandis que \INS{JALR} saute à l'adresse absolue stockée
dans un registre (\q{Jump and Link Register}).

Les deux fonctions \ttf et \main sont stockées dans le même fichoer objet, donc
l'adresse relative de \ttf est connue et fixée.
