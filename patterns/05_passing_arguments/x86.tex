\section{x86}

\subsection{MSVC}

\RU{Имеем в итоге}\EN{What we have after compilation} (MSVC 2010 Express):

\lstinputlisting[label=src:passing_arguments_ex_MSVC_cdecl,caption=MSVC 2010 Express]{patterns/05_passing_arguments/msvc.asm.\LANG}

\index{x86!\Registers!EBP}
\RU{Итак, здесь видно: в функции \main заталкиваются три числа в стек и вызывается 
функция \TT{f(int,int,int)}.}
\EN{What we see is the 3 numbers are pushing to stack in function \main and \TT{f(int,int,int)} 
is called then.}
\RU{Внутри \ttf, доступ к аргументам, также как и к локальным переменным, происходит через макросы: 
\TT{\_a\$ = 8}, но разница в том, что эти смещения со знаком \IT{плюс}, 
таким образом если прибавить макрос \TT{\_a\$} к указателю на \EBP, то адресуется \IT{внешняя} 
часть \glslink{stack frame}{фрейма} стека относительно \EBP.}
\EN{Argument access inside \ttf is organized with the help of macros like: \TT{\_a\$ = 8}, 
in the same way as local variables accessed,
but the difference in that these offsets are positive 
(addressed with \IT{plus} sign).
So, adding \TT{\_a\$} macro to the value in the \EBP register, \IT{outer} side of \gls{stack frame} is addressed.}

\index{x86!\Instructions!IMUL}
\index{x86!\Instructions!ADD}
\RU{Далее все более-менее просто: значение a помещается в \EAX. 
Далее \EAX умножается при помощи инструкции \IMUL на то что лежит в \TT{\_b}, 
так в \EAX остается \glslink{product}{произведение} этих двух значений.}
\EN{Then $a$ value is stored into \EAX. After \IMUL instruction execution, value in the \EAX is 
a \gls{product} of value in \EAX and what is stored in \TT{\_b}.}
\RU{Далее к регистру \EAX прибавляется то что лежит в \TT{\_c}.}
\EN{After \IMUL execution, \ADD is 
summing value in \EAX and what is stored in \TT{\_c}.}
\RU{Значение из \EAX никуда не нужно перекладывать, оно уже лежит где надо. 
Возвращаем управление вызываемой 
функции ~--- она возьмет значение из \EAX и отправит его в \printf.}
\EN{Value in the \EAX is not needed to be moved: it is already in place it must be.
Now return to \gls{caller}~---it will take value from the \EAX and used it as \printf argument.}

\ifdefined\IncludeOlly
\subsection{MSVC + \olly}
\index{\olly}
\RU{Проиллюстрируем всё это в}\EN{Let's illustrate this in} \olly.
\RU{Когда мы протрассируем до первой инструкции в \ttf, которая использует какой-то из аргументов
(первый), мы увидим, что \EBP указывает на \glslink{stack frame}{фрейм стека}. Он выделен красным прямоугольником.}%
\EN{When we trace to the first instruction in \ttf that uses one of the arguments 
(first one), we see that \EBP is pointing to the \gls{stack frame}, 
which is marked with a red rectangle.}
\RU{Самый первый элемент \glslink{stack frame}{фрейма стека}~--- это сохраненное значение \EBP, 
затем \ac{RA}. Третий элемент это первый аргумент функции, затем второй аргумент и третий.}
\EN{The first element of the \gls{stack frame} is the saved value of \EBP, 
the second one is \ac{RA}, the third is the first function argument, then the second and third ones.}
\RU{Для доступа к первому аргументу функции нужно прибавить к \EBP 8 (2 32-битных слова).}
\EN{To access the first function argument, one needs to add exactly 8 (2 32-bit words) to \EBP.}

\olly \EN{is aware about this, so it has added comments to the stack elements like}
\RU{в курсе этого, так что он добавил комментарии к элементам стека вроде}
\q{RETURN from} \AndENRU \q{Arg1 = \dots}, \etc{}.

N.B.: \EN{Function arguments are not members of the function's stack frame, they are rather
members of the stack frame of the \gls{caller} function.}
\RU{аргументы функции являются членами фрейма стека вызывающей функции, а не текущей.}
\EN{Hence, \olly marked \q{Arg} elements as members of another stack frame.}
\RU{Поэтому \olly отметил элементы \q{Arg} как члены другого фрейма стека.}

\begin{figure}[H]
\centering
\includegraphics[scale=\FigScale]{patterns/05_passing_arguments/olly.png}
\caption{\olly: \RU{внутри функции}\EN{inside of} \ttf{}\EN{ function}}
\label{fig:passing_arguments_olly}
\end{figure}

\fi

\subsection{GCC}

\RU{Скомпилируем то же в GCC 4.4.1 и посмотрим результат в \IDA:}
\EN{Let's compile the same in GCC 4.4.1 and let's see results in \IDA:}

\lstinputlisting[caption=GCC 4.4.1]{patterns/05_passing_arguments/gcc.asm.\LANG}

\RU{Практически то же самое, если не считать мелких отличий описанных раннее.}
\EN{Almost the same result.}

\RU{После вызова обоих ф-ций, \glslink{stack pointer}{указатель стека} не возвращается назад, потому что предпоследняя
инструкция}\EN{The \gls{stack pointer} is not returning back after both function exeuction, because penultimate}
\TT{LEAVE} (\ref{x86_ins:LEAVE}) \RU{сделает это за один раз, в конце исполнения}\EN{instruction will do this,
at the end}.

