\subsection{ARM + \NonOptimizingXcodeIV + \ThumbTwoMode}
\label{FPU_passing_floats_ARM}

\lstinputlisting{patterns/12_FPU/2_passing_floats/Xcode_thumb_O0.asm}

\RU{Как я уже писал, 64-битные числа с плавающей точкой передаются в парах R-регистров.}
\EN{As I wrote before, 64-bit floating pointer numbers passing in R-registers pairs.}
\RU{Этот код слегка избыточен (наверное, потому что не включена оптимизация), ведь, можно было бы 
загружать значения напрямую в R-регистры минуя загрузку в D-регистры.}
\EN{This is code is redundant for a little (certainly because optimization is turned off), because,
it is actually possible to load values into R-registers straightforwardly without touching D-registers.}

\RU{Итак, видно, что функция}\EN{So, as we see,} \TT{\_pow} \RU{получает первый аргумент в}
\EN{function receiving first argument in} \Reg{0} \AndENRU \Reg{1}, \RU{а второй в}\EN{and the second one in} 
\Reg{2} \AndENRU \Reg{3}. 
\RU{Функция оставляет результат в}\EN{Function leaves result in} \Reg{0} \AndENRU \Reg{1}.
\RU{Результат работы}\EN{Result of} \TT{\_pow} \RU{перекладывается в}\EN{is moved into} \TT{D16}, 
\RU{затем в пару}\EN{then in} \Reg{1} \AndENRU \Reg{2}\EN{ pair}, \RU{откуда}\EN{from where} 
\printf \RU{будет читать это число}\EN{will take this number}.

\subsection{ARM + \NonOptimizingKeilVI + \ARMMode}

\lstinputlisting{patterns/12_FPU/2_passing_floats/Keil_ARM_O0.asm}

\RU{Здесь не используются D-регистры, используются только пары R-регистров.}
\EN{D-registers are not used here, only R-register pairs are used.}

\subsection{ARM64 + \Optimizing GCC (Linaro) 4.9}

\begin{lstlisting}
f:
	stp	x29, x30, [sp, -16]!
	add	x29, sp, 0
	ldr	d1, .LC1 ; load 1.54 into D1
	ldr	d0, .LC0 ; load 32.01 into D0
	bl	pow
; result of pow() in D0
	adrp	x0, .LC2
	add	x0, x0, :lo12:.LC2
	bl	printf
	mov	w0, 0
	ldp	x29, x30, [sp], 16
	ret
.LC0:
; 32.01 in IEEE 754 format
	.word	-1374389535
	.word	1077936455
.LC1:
; 1.54 in IEEE 754 format
	.word	171798692
	.word	1073259479
.LC2:
	.string	"32.01 ^ 1.54 = %lf\n"
\end{lstlisting}

\RU{Константы загружаются в}\EN{Constants are loaded into} \RegD{0} \AndENRU \RegD{1}: 
\RU{ф-ция }pow() \RU{возьмет их оттуда}\EN{function will take them there}.
\RU{Результат в}\EN{Result is in} \RegD{0} \RU{после исполнения}\EN{after execution of} pow().
\RU{Он пропускается в}\EN{It is passed into} \printf \RU{без всякой модификации и перемещений}\EN{without 
any modification and moving}, 
\RU{потому что}\EN{because} \printf \RU{берет аргументы \glslink{integral type}{интегральных типов} и указатели 
из X-регистров,
а аргументы типа плавающей точки из D-регистров}\EN{takes argumens of \glslink{integral type}{integral types} 
and pointers from X-registers, and floating pointer arguments from D-registers}.
