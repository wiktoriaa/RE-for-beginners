\subsection{MIPS}

\index{MIPS!\Registers!FCCR}
\EN{The co-processor of the most popular MIPS processor has only one condition bit which can be set in the FPU 
and checked in the CPU.}
\RU{В сопроцессоре наиболее популярных MIPS есть только один бит результата, который устанавливается в FPU и 
проверяется в CPU.}
\EN{Earlier MIPS-es have only one condition bit (called FCC0), later ones have 8 (called FCC7-FCC0).}
\RU{Ранние MIPS имели только один бит (с названием FCC0), у поздних их 8 (с названием FCC7-FCC0).}
\RU{Эти биты находятся в регистре с названием FCCR.}
\EN{These bits are located in the register called FCCR.}

\lstinputlisting[caption=\Optimizing GCC 4.4.5 (IDA)]{patterns/12_FPU/3_comparison/MIPS_O3_IDA.lst.\LANG}

\index{MIPS!\Instructions!C.LT.D}
``C.LT.D'' \EN{compares two values}\RU{сравнивает два значения}. 
``LT'' \EN{is the condition}\RU{это условие} ``Less Than''\RU{ (меньше чем)}.
``D'' \EN{implies values of type}\RU{означает переменные типа} \Tdouble.
\EN{Depending on the result of the comparison, the FCC0 condition bit is either set or cleared.}
\RU{В зависимости от результата сравнения, бит FCC0 устанавливается или очищается.}

\index{MIPS!\Instructions!BC1T}
\index{MIPS!\Instructions!BC1F}
``BC1T'' \EN{checks the FCC0 bit and jumps if the bit is set}\RU{проверяет бит FCC0 и делает переход, если бит выставлен}.
``T'' \EN{mean that the jump is to be taken if the bit is set}\RU{означает что переход произойдет если бит выставлен} (``True'').
\EN{There is also the instruction}\RU{Имеется также инструкция} ``BC1F'' \EN{which jumps if the bit is cleared}\RU{которая сработает, если бит сброшен} (``False'').

\RU{В зависимости от перехода, один из аргументов ф-ции помещается в регистр \$F0.}
\EN{Depending on the jump, one of function arguments is placed into \$F0.}
