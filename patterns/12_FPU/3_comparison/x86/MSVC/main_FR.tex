\myparagraph{MSVC \NonOptimizing}

MSVC 2010 génère ce qui suit:

\lstinputlisting[caption=MSVC 2010 \NonOptimizing,style=customasmx86]{patterns/12_FPU/3_comparison/x86/MSVC/MSVC_FR.asm}

\myindex{x86!\Instructions!FLD}

Ainsi, \FLD charge \GTT{\_b} dans \ST{0}.

\label{Czero_etc}
\newcommand{\Czero}{\GTT{C0}\xspace}
\newcommand{\Ctwo}{\GTT{C2}\xspace}
\newcommand{\Cthree}{\GTT{C3}\xspace}
\newcommand{\CThreeBits}{\Cthree/\Ctwo/\Czero}

\myindex{x86!\Instructions!FCOMP}

\FCOMP compare la valeur dans \ST{0} avec ce qui est dans \GTT{\_a} et met les bits
\CThreeBits du mot registre d'état du FPU, suivant le résultat.
Ceci est un registre 16-bit qui reflète l'état courant du FPU.

Après que les bits ont été mis, l'instruction \FCOMP dépile une variable depuis la
pile.
C'est ce qui la différencie de \FCOM, qui compare juste les valeurs, laissant la
pile dans le même état.

Malheureusement, les CPUs avant les Intel P6\footnote{Intel P6 comprend les Pentium
Pro, Pentium II, etc.} ne possèdent aucune instruction de saut conditionnel qui teste
les bits \CThreeBits.
Peut-être est-ce une raison historique (rappel: le FPU était une puce séparée dans
le passé).\\Les CPU modernes, à partir des Intel P6 possèdent les instructions \FCOMI/\FCOMIP/\FUCOMI/\FUCOMIP~---qui
font la même chose, mais modifient les flags \ZF/\PF/\CF du CPU.

\myindex{x86!\Instructions!FNSTSW}

L'instruction \FNSTSW copie le le mot du registre d'état du FPU dans \AX.
Les bits \CThreeBits sont placés aux positions 14/10/8, ils sont à la même position
dans le registre \AX et tous sont placés dans la partie haute de \AX{}~---\AH{}.

\begin{itemize}
\item Si $b>a$ dans notre exemple, alors les bits \CThreeBits sont mis comme ceci: 0, 0, 0.
\item Si $a>b$, alors les bits sont: 0, 0, 1.
\item Si $a=b$, alors les bits sont: 1, 0, 0.

Si le résultat n'est pas ordonné (en cas d'erreur), alors les bits sont: 1, 1, 1.
\end{itemize}
% TODO: table here?

Voici comment les bits \CThreeBits sont situés dans le registre \AX:

\begin{center}
\begin{bytefield}[endianness=big,bitwidth=0.03\linewidth]{16}
\bitheader{14,10,9,8} \\
\bitbox{1}{} & 
\bitbox{1}{\TT{C3}} & 
\bitbox{3}{} & 
\bitbox{1}{\TT{C2}} & 
\bitbox{1}{\TT{C1}} & 
\bitbox{1}{\TT{C0}} &
\bitbox{8}{}
\end{bytefield}
\end{center}


Voici comment les bits \CThreeBits sont situés dans le registre \AH:

\begin{center}
\ifdefined\ebook
\begin{bytefield}[endianness=big,bitwidth=0.06\linewidth]{8}
\else
\begin{bytefield}[endianness=big,bitwidth=0.03\linewidth]{8}
\fi
\bitheader{6,2,1,0} \\
\bitbox{1}{} & 
\bitbox{1}{C3} & 
\bitbox{3}{} & 
\bitbox{1}{C2} & 
\bitbox{1}{C1} & 
\bitbox{1}{C0}
\end{bytefield}
\end{center}


Après l'exécution de \INS{test ah, 5}\footnote{5=101b}, seul les bits \Czero et \Ctwo
(en position 0 et 2) sont considérés, tous les autres bits sont simplement ignorés.

\label{parity_flag}
\myindex{x86!\Registers!\Flags!Parity flag}

Parlons maintenant du \IT{parity flag} (flag de parité), un autre rudiment historique
remarquable.

Ce flag est mis à 1 si le nombre de un dans le résultat du dernier calcul est pair,
et à 0 s'il est impair.

Regardons sur Wikipédia\footnote{\href{http://go.yurichev.com/17131}{wikipedia.org/wiki/Parity\_flag}}:

\begin{framed}
\begin{quotation}
Une raison commune de tester le bit de parité n'a rien à voir avec la parité. Le FPU
possède quatre flags de condition (C0 à C3), mais ils ne peuvent pas être testés
directement, et doivent d'abord être copiés dans le registre d'états.
Lorsque ça se produit, C0 est mis dans le flag de retenue, C2 dans le flag
de parité et C3 dans le flag de zéro.
Le flag C2 est mis lorsque e.g. des valeurs en virgule flottantes incomparable
(NaN ou format non supporté) sont comparées avec l'instruction \FUCOM.
\end{quotation}
\end{framed}

Comme indiqué dans Wikipédia, le flag de parité est parfois utilisé dans du code
FPU, voyons comment.

\myindex{x86!\Instructions!JP}

Le flag \PF est mis à 1 si à la fois \Czero et \Ctwo sont mis à 0 ou si les deux
sont à 1, auquel cas le \JP (\IT{jump if PF==1}) subséquent est déclenché.
Si l'on se rappelle les valeurs de \CThreeBits pour différents cas, nous pouvons
voir que le saut conditionnel \JP est déclenché dans deux cas: si $b>a$ ou $a=b$
(le bit \Cthree n'est pris en considération ici, puisqu'il a été mis à 0 par l'instruction
\INS{test ah, 5}).

C'est très simple ensuite.
Si le saut conditionnel a été déclenché, \FLD charge la valeur de \GTT{\_b} dans
\ST{0}, et sinon, la valeur de \GTT{\_a} est chargée ici.

\mysubparagraph{Et à propos du test de \Ctwo?}

Le flag \Ctwo est mis en cas d'erreur (\gls{NaN}, etc.), mais notre code ne le teste
pas.

Si le programmeur veut prendre en compte les erreurs FPU, il doit ajouter des tests
supplémentaires.

\clearpage
\mysubparagraph{Premier exemple sous \olly: a=1.2 et b=3.4}
\myindex{\olly}

Chargeons l'exemple dans \olly:

\begin{figure}[H]
\centering
\myincludegraphics{patterns/12_FPU/3_comparison/x86/MSVC/olly1_1.png}
\caption{\olly: la première instruction \FLD a été exécutée}
\label{fig:FPU_comparison_case1_olly1}
\end{figure}

Arguments courants de la fonction: $a=1.2$ et $b=3.4$ (Nous pouvons les voir dans
la pile: deux paires de valeurs 32-bit).
$b$ (3.4) est déjà chargé dans \ST{0}.
Maintenant \FCOMP est train d'être exécutée.
\olly montre le second argument de \FCOMP, qui se trouve sur la pile à ce moment.

\clearpage
\FCOMP a été exécutée:

\begin{figure}[H]
\centering
\myincludegraphics{patterns/12_FPU/3_comparison/x86/MSVC/olly1_2.png}
\caption{\olly: \FCOMP a été exécutée}
\label{fig:FPU_comparison_case1_olly2}
\end{figure}

Nous voyons l'état des flags de condition du \ac{FPU}: tous à zéro.
La valeur dépilée est vue ici comme \ST{7}, la raison a été décrite ici:
\myref{FPU_is_rather_circular_buffer}.

\clearpage
\FNSTSW a été exécutée:
\begin{figure}[H]
\centering
\myincludegraphics{patterns/12_FPU/3_comparison/x86/MSVC/olly1_3.png}
\caption{\olly: \FNSTSW a été exécutée}
\label{fig:FPU_comparison_case1_olly3}
\end{figure}

Nous voyons que le registre \GTT{AX} contient des zéro: en effet, tous les flags
de condition sont à zéro.
(\olly désassemble l'instruction \FNSTSW comme \INS{FSTSW}---elles sont synonymes).

\clearpage
\TEST a été exécutée:

\begin{figure}[H]
\centering
\myincludegraphics{patterns/12_FPU/3_comparison/x86/MSVC/olly1_4.png}
\caption{\olly: \TEST a été exécutée}
\label{fig:FPU_comparison_case1_olly4}
\end{figure}

Le flag \GTT{PF} est mis à 1.

En effet: le nombre de bit mis à 0 est 0 et 0 est un nombre pair.
olly désassemble l'instruction \INS{JP} comme \ac{JPE}---elles sont synonymes.
Et elle va maintenant se déclencher.

\clearpage
\ac{JPE} déclenchée, \FLD charge la valeur de $b$ (3.4) dans \ST{0}:

\begin{figure}[H]
\centering
\myincludegraphics{patterns/12_FPU/3_comparison/x86/MSVC/olly1_5.png}
\caption{\olly: la seconde instruction \FLD a été exécutée}
\label{fig:FPU_comparison_case1_olly5}
\end{figure}

La fonction a fini son travail.

\clearpage
\mysubparagraph{Second exemple sous \olly: a=5.6 et b=-4}

Chargeons l'exemple dans \olly:

\begin{figure}[H]
\centering
\myincludegraphics{patterns/12_FPU/3_comparison/x86/MSVC/olly2_1.png}
\caption{\olly: premier \FLD exécutée}
\label{fig:FPU_comparison_case2_olly1}
\end{figure}

Arguments de la fonction courante: $a=5.6$ et $b=-4$.
$b$ (-4) est déjà chargé dans \ST{0}.
\FCOMP va s'exécuter maintenant.
\olly montre le second argument de \FCOMP, qui est sur la pile juste maintenant.

\clearpage
\FCOMP a été exécutée:

\begin{figure}[H]
\centering
\myincludegraphics{patterns/12_FPU/3_comparison/x86/MSVC/olly2_2.png}
\caption{\olly: \FCOMP exécutée}
\label{fig:FPU_comparison_case2_olly2}
\end{figure}

Nous voyons l'état des flags de condition du \ac{FPU}: tous à zéro sauf \Czero.

\clearpage
\FNSTSW a été exécutée:

\begin{figure}[H]
\centering
\myincludegraphics{patterns/12_FPU/3_comparison/x86/MSVC/olly2_3.png}
\caption{\olly: \FNSTSW exécutée}
\label{fig:FPU_comparison_case2_olly3}
\end{figure}

Nous voyons que le registre \GTT{AX} contient \GTT{0x100}: le flag \Czero est au 8ième bit.

\clearpage
\TEST a été exécutée:

\begin{figure}[H]
\centering
\myincludegraphics{patterns/12_FPU/3_comparison/x86/MSVC/olly2_4.png}
\caption{\olly: \TEST exécutée}
\label{fig:FPU_comparison_case2_olly4}
\end{figure}

Le flag \GTT{PF} est mis à zéro.
En effet:

le nombre de bit mis à 1 dans \GTT{0x100} est 1, et 1 est un nombre impair.
\ac{JPE} est sautée maintenant.

\clearpage
\ac{JPE} n'a pas été déclenchée, donc \FLD charge la valeur de $a$ (5.6) dans \ST{0}:

\begin{figure}[H]
\centering
\myincludegraphics{patterns/12_FPU/3_comparison/x86/MSVC/olly2_5.png}
\caption{\olly: second \FLD exécutée}
\label{fig:FPU_comparison_case2_olly5}
\end{figure}

La fonction a fini son travail.

