\myparagraph{GCC 4.4.1}

\lstinputlisting[caption=GCC 4.4.1,style=customasmx86]{patterns/12_FPU/3_comparison/x86/GCC_EN.asm}

\myindex{x86!\Instructions!FUCOMPP}

\FUCOMPP{} is almost like \FCOM, but pops both values from the stack and handles
\q{not-a-numbers} differently.

\myindex{Non-a-numbers (NaNs)}
A bit about \IT{not-a-numbers}.

\newcommand{\NANFN}{\footnote{\href{http://go.yurichev.com/17130}{wikipedia.org/wiki/NaN}}}

The FPU is able to deal with special values which are \IT{not-a-numbers} or \gls{NaN}s\NANFN. 
These are infinity, result of division by 0, etc.
Not-a-numbers can be \q{quiet} and \q{signaling}. It is possible to continue to work with \q{quiet} NaNs, 
but if one tries to do any operation with \q{signaling} NaNs, an exception is to be raised.

\myindex{x86!\Instructions!FCOM}
\myindex{x86!\Instructions!FUCOM}

\FCOM raising an exception if any operand is \gls{NaN}. 
\FUCOM raising an exception only if any operand is a signaling \gls{NaN} (SNaN).

\myindex{x86!\Instructions!SAHF}
\label{SAHF}

The next instruction is \SAHF (\IT{Store AH into Flags})~---this is a rare 
instruction in code not related to the FPU. 
8 bits from AH are moved into the lower 8 bits of the CPU flags in the following order:

\begin{center}
\begin{bytefield}[endianness=big,bitwidth=0.03\linewidth]{8}
\bitheader{7,6,4,2,0} \\
\bitbox{1}{SF} & 
\bitbox{1}{ZF} & 
\bitbox{1}{} & 
\bitbox{1}{AF} & 
\bitbox{1}{} & 
\bitbox{1}{PF} & 
\bitbox{1}{} & 
\bitbox{1}{CF}
\end{bytefield}
\end{center}


\myindex{x86!\Instructions!FNSTSW}

Let's recall that \FNSTSW moves the bits that interest us (\CThreeBits) into \AH 
and they are in positions 6, 2, 0 of the \AH register:

\begin{center}
\ifdefined\ebook
\begin{bytefield}[endianness=big,bitwidth=0.06\linewidth]{8}
\else
\begin{bytefield}[endianness=big,bitwidth=0.03\linewidth]{8}
\fi
\bitheader{6,2,1,0} \\
\bitbox{1}{} & 
\bitbox{1}{C3} & 
\bitbox{3}{} & 
\bitbox{1}{C2} & 
\bitbox{1}{C1} & 
\bitbox{1}{C0}
\end{bytefield}
\end{center}


In other words, the \INS{fnstsw  ax / sahf} instruction pair moves \CThreeBits into \ZF, \PF and \CF.

Now let's also recall the values of \CThreeBits in different conditions:

\begin{itemize}
\item If $a$ is greater than $b$ in our example, then \CThreeBits are to be set to: 0, 0, 0.
\item if $a$ is less than $b$, then the bits are to be set to: 0, 0, 1.
\item If $a=b$, then: 1, 0, 0.
\end{itemize}
% TODO: table?

In other words, these states of the CPU flags are possible
after three \\
\FUCOMPP/\FNSTSW/\SAHF instructions:

\begin{itemize}
\item If $a>b$, the CPU flags are to be set as: \GTT{ZF=0, PF=0, CF=0}.
\item If $a<b$, then the flags are to be set as: \GTT{ZF=0, PF=0, CF=1}.
\item And if $a=b$, then: \GTT{ZF=1, PF=0, CF=0}.
\end{itemize}
% TODO: table?

\myindex{x86!\Instructions!SETcc}
\myindex{x86!\Instructions!JNBE}

Depending on the CPU flags and conditions, \SETNBE stores 1 or 0 to AL. 
It is almost the counterpart of \JNBE, with the exception that \SETcc 
\footnote{\IT{cc} is \IT{condition code}} stores 1 or 0 in \AL, 
but \Jcc does actually jump or not. 
\SETNBE stores 1 only if \GTT{CF=0} and \GTT{ZF=0}. 
If it is not true, 0 is to be stored into \AL.

Only in one case both \CF and \ZF are 0: if $a>b$.

Then 1 is to be stored to \AL, the subsequent \JZ is not to be triggered and the function will return {\_a}. 
In all other cases, {\_b} is to be returned.

