\ifx\RUSSIAN\undefined
\subsection{MIPS}

MIPS may support several coprocessors (up to 4), 
zeroth of which is a special control coprocessor,
and first coprocessor is FPU one.

As in ARM, MIPS coprocessor is not stack machine, it has 32 32-bit registers (\$F0-\$F31): 
\ref{MIPS_FPU_registers}.
When one need to work with 64-bit double values, a pair of 32-bit F-registers is used.

\lstinputlisting[caption=\Optimizing GCC 4.4.5 (IDA)]{patterns/12_FPU/1_simple/MIPS_O3_IDA.lst}

New instructions here are:

\begin{itemize}

\index{MIPS!\Instructions!LWC1}
\item LWC1 that loads a 32-bit word into register of first coprocessor (hence 1 in instruction name).
\index{MIPS!\Pseudoinstructions!L.D}
Pair of LWC1 instructions may be coalesced into L.D pseudoinstruction.

\index{MIPS!\Instructions!DIV.D}
\index{MIPS!\Instructions!MUL.D}
\index{MIPS!\Instructions!ADD.D}
\item DIV.D, MUL.D, ADD.d do division/multiplication/addition 
(D in suffix mean double precision, S would mean float precision)

\end{itemize}

\index{MIPS!\Instructions!LUI}
\index{\CompilerAnomaly}
There are also weird kind of compiler anomaly: LUI instructions marked by with question marks.
It's hard for me to understand why to load part of 64-bit double constant into \$V0 register.
These instructions has no effect.
If someone knows more about it, please drop me email\footnote{\EMAIL}.

\fi
