\subsection{Limites de chaînes}

\myindex{win32!GetOpenFileName}
Il est intéressant de noter comment les paramètres sont passés à la fonction win32
\IT{GetOpenFileName()}. Afin de l'appeler, il faut définir une liste des extensions
de fichier autorisées:

\begin{lstlisting}[style=customc]
	OPENFILENAME *LPOPENFILENAME;
	...
	char * filter = "Text files (*.txt)\0*.txt\0MS Word files (*.doc)\0*.doc\0\0";
	...
	LPOPENFILENAME = (OPENFILENAME *)malloc(sizeof(OPENFILENAME));
	...
	LPOPENFILENAME->lpstrFilter = filter;
	...

	if(GetOpenFileName(LPOPENFILENAME))
	{
		...
\end{lstlisting}

Ce qui se passe ici, c'est que la liste de chaînes est passée à \IT{GetOpenFileName()}.
Ce n'est pas un problème de l'analyser: à chaque fois que l'on rencontre un octet
nul, c'est un élément.
Quand on rencontre deux octets nul, c'est la fin de la liste.
Si vous passez cette chaîne à \printf, elle traitera le premier élément comme une
simple chaîne.

Donc, ceci est un chaîne, ou...?
Il est plus juste de dire que c'est un buffer contenants plusieurs chaînes-C terminées
par zéro, qui peut être stocké et traité comme un tout.

\myindex{\CStandardLibrary!strtok}
Un autre exemple est la fonction \IT{strtok()}. Elle prend une chaîne et y écrit
des octets nul.
C'est ainsi qu'elle transforme la chaîne d'entrée en une sorte de buffer, qui contient
plusieurs chaînes-C terminées par zéro.

