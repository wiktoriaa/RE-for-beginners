\subsection{MIPS}

\lstinputlisting[caption=\Optimizing GCC 4.4.5 (IDA)]{patterns/10_strings/1_strlen/MIPS_O3_IDA.lst.\LANG}

\index{MIPS!\Instructions!NOR}
\index{MIPS!\Pseudoinstructions!NOT}
\RU{В MIPS нет инструкции ``НЕ'', но есть NOR, это операция ``ИЛИ + НЕ''.}
\EN{MIPS lacks NOT instruction, but has NOR which is OR + NOT operation.}
\RU{Эта операция широко применяется в цифровой электронике\footnote{ИЛИ-НЕ называют ``универсальным элементом''.
Например, космический компьютер Apollo Guidance Computer использовавшийся в программе ``Аполлон'' был
построен исключительно на 5600 элементах ИЛИ-НЕ: \cite{Eickhoff}.}, но не очень популярна в программировании.}
\EN{This operation is widely used in digital electronics\footnote{NOR is called ``universal gate''.
For example, space Apollo Guidance Computer used in Apollo program, 
was built using only 5600 NOR gates: \cite{Eickhoff}.}, but not quite popular in computer programming.}
\RU{Так что операция ``НЕ'' реализована здесь как}\EN{So, NOT operation is implemented here as} 
``NOR DST, \$ZERO, SRC''.

\RU{Из фундаментальных знаний \ref{sec:signednumbers}, мы можем знать что побитовое инвертирование знакового
числа это то же что и смена его знака, а также вычитание 1 из результата.}
\EN{From fundamendals \ref{sec:signednumbers}, we may know that bitwise inverting a signed number is the same 
as changing its sign and also subtracting 1 from result.}
\RU{Так что, то что делает здесь NOT это берет значение $str$ и трансформирует его в $-str-1$.}
\EN{So what NOT doing here is taking $str$ value and transforms it into $-str-1$.}
\RU{Следующая операция сложения готовит результат}\EN{Next addition operation prepares result}.
