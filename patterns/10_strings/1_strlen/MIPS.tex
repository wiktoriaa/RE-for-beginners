\ifx\RUSSIAN\undefined
\subsection{MIPS}

\lstinputlisting[caption=\Optimizing GCC 4.4.5 (IDA)]{patterns/10_strings/1_strlen/MIPS_O3_IDA.lst}

\index{MIPS!\Instructions!NOR}
\index{MIPS!\Pseudoinstructions!NOT}
MIPS lacks NOT instruction, but has NOR which is OR + NOT operation.
This operation is widely used in digital electronics\footnote{NOR is called ``universal gate''.
For example, space Apollo Guidance Computer used in Apollo program, 
was built using only 5600 NOR gates: \cite{Eickhoff}.}, but not quite popular in computer programming.
So, NOT operation is implemented here as ``NOR DST, \$ZERO, SRC''.

From fundamendals (\ref{sec:signednumbers}), we may know that bitwise inverting a signed number is the same 
as changing its sign and also subtracting 1 from it.
So what NOT doing here is taking $str$ value and transforms it into $-str-1$.
Next addition operation prepares result.

\fi
