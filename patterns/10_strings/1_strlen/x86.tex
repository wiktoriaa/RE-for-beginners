\subsection{x86}

\subsubsection{\NonOptimizing MSVC}

\RU{Итак, компилируем:}\EN{Let's compile:}

\lstinputlisting{patterns/10_strings/1_strlen/10_1_msvc.asm.\LANG}

\index{x86!\Instructions!MOVSX}
\index{x86!\Instructions!TEST}
\RU{Здесь две новых инструкции: \MOVSX и \TEST.}
\EN{Two new instructions here: \MOVSX and \TEST.}

\label{MOVSX}
\RU{О первой: \MOVSX предназначен для того чтобы взять байт из какого-либо места в памяти и положить его, 
в нашем случае, в регистр \EDX. 
Но регистр \EDX ~--- 32-битный. \MOVSX означает \IT{MOV with Sign-Extend}. 
Оставшиеся биты с 8-го по 31-й \MOVSX сделает единицей, если исходный байт в памяти имеет знак \IT{минус}, 
или заполнит нулями, если знак \IT{плюс}.}
\EN{About first: \MOVSX is intended to take byte from a point in memory and store value in a 32-bit register. 
\MOVSX meaning \IT{MOV with Sign-Extend}. 
Rest bits starting at 8th till 31th \MOVSX will set to $1$ if source byte in memory has \IT{minus} 
sign or to 0 if \IT{plus}.}

\RU{И вот зачем все это.}\EN{And here is why all this.}

\RU{По умолчанию, в MSVC и GCC, тип \Tchar ~--- знаковый. Если у нас есть две переменные, одна \Tchar, а другая \Tint 
(\Tint тоже знаковый), и если в первой переменной лежит $-2$ (что кодируется как \TT{0xFE}) и мы просто 
переложим это в \Tint, 
то там будет \TT{0x000000FE}, а это, с точки зрения \Tint, даже знакового, будет $254$, но никак не $-2$. 
$-2$ в переменной \Tint кодируется как \TT{0xFFFFFFFE}. И для того чтобы значение \TT{0xFE} из переменной типа 
\Tchar переложить 
в знаковый \Tint с сохранением всего, нужно узнать его знак, и затем заполнить остальные биты. 
Это делает \MOVSX.}
\EN{By default, \Tchar type as signed in MSVC and GCC. If we have two values, one is \Tchar 
and another is \Tint, (\Tint is signed too), and if first value contain $-2$ (it is coded as \TT{0xFE}) 
and we just copying this byte into \Tint container, there will be \TT{0x000000FE}, and this, 
from the point of signed \Tint view is $254$, but not $-2$. In signed int, $-2$ is coded as \TT{0xFFFFFFFE}. 
So if we need to transfer \TT{0xFE} value from variable of \Tchar type to \Tint, 
we need to identify its sign and extend it. That is what \MOVSX does.}

\RU{См. также об этом раздел}
\EN{Also read about it in} ``\IT{\SignedNumbersSectionName}''\EN{ section}~(\ref{sec:signednumbers}).

\RU{Хотя, конкретно здесь, компилятору вряд ли была особая надобность хранить значение \Tchar в регистре \EDX 
а не его восьмибитной части, скажем, \DL. Но получилось, как получилось: должно быть, 
\gls{register allocator} компилятора сработал именно так.}
\EN{I'm not sure if the compiler needs to store \Tchar variable in the \EDX, it could take 8-bit register part 
(let's say \DL). Apparently, compiler's \gls{register allocator} works like that.}

\index{ARM!\Instructions!TEST}
\RU{Позже выполняется \TT{TEST EDX, EDX}. 
Об инструкции \TEST читайте в разделе о битовых полях~(\ref{sec:bitfields}).
Но конкретно здесь, эта инструкция просто проверяет состояние регистра \EDX на $0$.}
\EN{Then we see \TT{TEST EDX, EDX}. 
About \TEST instruction, read more in section about bit fields~(\ref{sec:bitfields}).
But here, this instruction just checking value in the \EDX, if it is equals to $0$.}

\subsubsection{\NonOptimizing GCC}

\RU{Попробуем}\EN{Let's try} GCC 4.4.1:

\lstinputlisting{patterns/10_strings/1_strlen/10_3_gcc.asm}

\label{movzx}
\index{x86!\Instructions!MOVZX}
\RU{Результат очень похож на MSVC, вот только здесь используется \MOVZX а не \MOVSX. 
\MOVZX означает \IT{MOV with Zero-Extend}. Эта инструкция перекладывает какое-либо значение 
в регистр и остальные биты выставляет в $0$.
Фактически, преимущество этой инструкции только в том, что она позволяет 
заменить две инструкции сразу: \TT{xor eax, eax / mov al, [...]}.}
\EN{The result almost the same as MSVC did, but here we see \MOVZX instead of \MOVSX. 
\MOVZX means \IT{MOV with Zero-Extend}. 
This instruction copies 8-bit or 16-bit value into 32-bit register and sets the rest bits to $0$. 
In fact, this instruction is convenient only since it enable us to replace two instructions at once: 
\TT{xor eax, eax / mov al, [...]}.}

\RU{С другой стороны, нам очевидно, что здесь можно было бы написать вот так: 
\TT{mov al, byte ptr [eax] / test al, al} ~--- это тоже самое, хотя старшие биты \EAX будут ``замусорены''. 
Но, будем считать, что это погрешность компилятора ~--- 
он не смог сделать код более экономным или более понятным. 
Строго говоря, компилятор вообще не нацелен на то чтобы генерировать понятный (для человека) код.}
\EN{On the other hand, it is obvious to us the compiler could produce the code: 
\TT{mov al, byte ptr [eax] / test al, al}~---it is almost the same, however, 
the highest \EAX register bits will contain random noise. 
But let's think it is compiler's drawback~---it cannot produce more understandable code. 
Strictly speaking, compiler is not obliged to emit understandable (to humans) code at all.}

\index{x86!\Instructions!SETcc}
\RU{Следующая новая инструкция для нас ~--- \SETNZ. В данном случае, если в \AL был не ноль, 
то \TT{test al, al} выставит флаг \ZF в $0$, а \SETNZ, если \TT{ZF==0} 
(\IT{NZ} значит \IT{not zero}) выставит $1$ в \AL. 
Смысл этой процедуры в том, что, если говорить человеческим языком, 
\IT{если AL не ноль, то выполнить переход на} \TT{loc\_80483F0}.
Компилятор выдал немного избыточный код, но не будем забывать, что оптимизация выключена.}
\EN{Next new instruction for us is \SETNZ. Here, if \AL contain not zero, \TT{test al, al} 
will set $0$ to the \ZF flag, but \SETNZ, if \TT{ZF==0} (\IT{NZ} means \IT{not zero}) will set $1$ to the \AL.
Speaking in natural language, \IT{if \AL is not zero, let's jump to loc\_80483F0}. 
Compiler emitted slightly redundant code, but let's not forget the optimization is turned off.}

\subsubsection{\Optimizing MSVC}
\label{strlen_MSVC_Ox}

\RU{Теперь скомпилируем все то же самое в MSVC 2012, но с включенной оптимизацией (\Ox)}
\EN{Now let's compile all this in MSVC 2012, with optimization turned on (\Ox)}:

\lstinputlisting[caption=\Optimizing MSVC 2012 /Ob0]{patterns/10_strings/1_strlen/10_2.asm.\LANG}

\RU{Здесь все попроще стало. Но следует отметить, что компилятор обычно может так хорошо использовать регистры 
только на не очень больших функциях с не очень большим количеством локальных переменных.}
\EN{Now it is all simpler.
But it is needless to say the compiler could use registers such efficiently 
only in small functions with small number of local variables.}

\index{x86!\Instructions!INC}
\index{x86!\Instructions!DEC}
\INC/\DEC\EMDASH\RU{это инструкции \glslink{increment}{инкремента}-\glslink{decrement}{декремента}, попросту говоря: 
увеличить на единицу или уменьшить.}
\EN{are \gls{increment}/\gls{decrement} instruction, in other words: add 1 to variable or subtract.}

\ifdefined\IncludeOlly
\EN{\clearpage
\myparagraph{\Optimizing MSVC + \olly}
\myindex{\olly}

We can try this (optimized) example in \olly.  Here is the first iteration:

\begin{figure}[H]
\centering
\myincludegraphics{patterns/10_strings/1_strlen/olly1.png}
\caption{\olly: first iteration start}
\label{fig:strlen_olly_1}
\end{figure}

We see that \olly found a loop and, for convenience, \IT{wrapped} its instructions in brackets.
By clicking the right button on \EAX, we can choose 
\q{Follow in Dump} and the memory window scrolls to the right place.
Here we can see the string \q{hello!} in memory.
There is at least
one zero byte after it and then random garbage.

If \olly sees a register with a valid address in it, that points to some string, 
it is shown as a string.

\clearpage
Let's press F8 (\stepover) a few times, to get to the start of the body of the loop:

\begin{figure}[H]
\centering
\myincludegraphics{patterns/10_strings/1_strlen/olly2.png}
\caption{\olly: second iteration start}
\label{fig:strlen_olly_2}
\end{figure}

We see that \EAX contains the address of the second character in the string.

\clearpage

We have to press F8 enough number of times in order to escape from the loop:

\begin{figure}[H]
\centering
\myincludegraphics{patterns/10_strings/1_strlen/olly3.png}
\caption{\olly: pointers difference to be calculated now}
\label{fig:strlen_olly_3}
\end{figure}

We see that \EAX now contains the address of zero byte that's right after the string.
Meanwhile, \EDX hasn't changed,
so it still pointing to the start of the string.

The difference between these two addresses is being calculated now.

\clearpage
The \SUB instruction just got executed:

\begin{figure}[H]
\centering
\myincludegraphics{patterns/10_strings/1_strlen/olly4.png}
\caption{\olly: \EAX to be decremented now}
\label{fig:strlen_olly_4}
\end{figure}

The difference of pointers is in the \EAX register now---7.
Indeed, the length of the \q{hello!} string is 6, 
but with the zero byte included---7.
But \TT{strlen()} must return the number of non-zero characters in the string.
So the decrement executes and then the function returns.
}
\RU{\clearpage
\myparagraph{\Optimizing MSVC + \olly}
\myindex{\olly}

Можем попробовать этот (соптимизированный) пример в \olly.  Вот самая первая итерация:

\begin{figure}[H]
\centering
\myincludegraphics{patterns/10_strings/1_strlen/olly1.png}
\caption{\olly: начало первой итерации}
\label{fig:strlen_olly_1}
\end{figure}

Видно, что \olly обнаружил цикл и, для удобства, \IT{свернул} инструкции тела цикла в скобке.

Нажав правой кнопкой на \EAX, можно выбрать \q{Follow in Dump} 
и позиция в окне памяти будет как раз там, где надо.

Здесь мы видим в памяти строку \q{hello!}.
После неё имеется как минимум 1 нулевой байт, затем случайный мусор.
Если \olly видит, что в регистре содержится адрес какой-то строки, он показывает эту строку.

\clearpage
Нажмем F8 (\stepover) столько раз, чтобы текущий адрес снова был в начале тела цикла:

\begin{figure}[H]
\centering
\myincludegraphics{patterns/10_strings/1_strlen/olly2.png}
\caption{\olly: начало второй итерации}
\label{fig:strlen_olly_2}
\end{figure}

Видно, что \EAX уже содержит адрес второго символа в строке.

\clearpage
Будем нажимать F8 достаточное количество раз, чтобы выйти из цикла:

\begin{figure}[H]
\centering
\myincludegraphics{patterns/10_strings/1_strlen/olly3.png}
\caption{\olly: сейчас будет вычисление разницы указателей}
\label{fig:strlen_olly_3}
\end{figure}

Увидим, что \EAX теперь содержит адрес нулевого байта, следующего сразу за строкой.

А \EDX так и не менялся~--- он всё ещё указывает на начало строки.
Здесь сейчас будет вычисляться разница между этими двумя адресами.

\clearpage
Инструкция \SUB исполнилась:

\begin{figure}[H]
\centering
\myincludegraphics{patterns/10_strings/1_strlen/olly4.png}
\caption{\olly: сейчас будет декремент \EAX}
\label{fig:strlen_olly_4}
\end{figure}

Разница указателей сейчас в регистре \EAX~--- 7.

Действительно, длина строки \q{hello!}~--- 6, 
но вместе с нулевым байтом --- 7.
Но \TT{strlen()} должна возвращать количество ненулевых символов в строке.
Так что сейчас будет исполняться декремент и выход из функции.

}


\fi

\subsubsection{\Optimizing GCC}

\RU{Попробуем GCC 4.4.1 с включенной оптимизацией (ключ \Othree:}
\EN{Let's check GCC 4.4.1 with optimization turned on (\Othree key):}

\lstinputlisting{patterns/10_strings/1_strlen/10_3_gcc_O3.asm}

\RU{Здесь GCC не очень отстает от MSVC за исключением наличия \MOVZX.} 
\EN{Here GCC is almost the same as MSVC, except of \MOVZX presence.}

\RU{Впрочем, \MOVZX здесь явно можно заменить на}
\EN{However, \MOVZX could be replaced here to} \TT{mov dl, byte ptr [eax]}.

\RU{Но, возможно, компилятору GCC просто проще помнить, что у него под переменную типа \Tchar отведен целый 
32-битный регистр \EDX и быть уверенным в том, что старшие биты регистра не будут замусорены.}
\EN{Probably, it is simpler for GCC compiler's code generator to \IT{remember} 
the whole 32-bit \EDX register 
is allocated for \Tchar variable and it can be sure the highest bits will not contain any noise 
at any point.}

\label{strlen_NOT_ADD}
\index{x86!\Instructions!NOT}
\index{x86!\Instructions!XOR}
\RU{Далее мы видим новую для нас инструкцию \NOT. Эта инструкция инвертирует все биты в операнде. 
Можно сказать, что здесь это синонимично инструкции \TT{XOR ECX, 0ffffffffh}. 
\NOT и следующая за ней инструкция \ADD вычисляют разницу указателей и отнимают от результата единицу. 
Только происходит это слегка по-другому. Сначала \ECX, где хранится указатель на \IT{str}, 
инвертируется и от него отнимается единица.}
\EN{After, we also see new instruction \NOT. This instruction inverts all bits in operand. 
It can be said, it is synonym to the \TT{XOR ECX, 0ffffffffh} instruction. 
\NOT and following \ADD calculating pointer difference and subtracting 1. 
At the beginning \ECX, where pointer to \IT{str} is stored, inverted and 1 is subtracted from it.}

\RU{См. также раздел:}\EN{See also:} ``\SignedNumbersSectionName''~(\ref{sec:signednumbers}).

\RU{Иными словами, в конце функции, после цикла, происходит примерно следующее:} 
\EN{In other words, at the end of function, just after loop body, these operations are executed:}

\begin{lstlisting}
ecx=str;
eax=eos;
ecx=(-ecx)-1; 
eax=eax+ecx
return eax
\end{lstlisting}

\dots \RU{что эквивалентно}\EN{and this is effectively equivalent to}:

\begin{lstlisting}
ecx=str;
eax=eos;
eax=eax-ecx;
eax=eax-1;
return eax
\end{lstlisting}

\RU{Но почему GCC решил, что так будет лучше? Снова не берусь сказать. Но я не сомневаюсь, 
что эти оба варианта работают примерно равноценно в плане эффективности и скорости.}
\EN{Why GCC decided it would be better? I cannot be sure. 
But I'm sure the both variants are effectively equivalent in efficiency sense.}
