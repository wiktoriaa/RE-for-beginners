\section[Линейный конгруэнтный генератор]{Линейный конгруэнтный генератор как генератор псевдослучайных чисел}
\myindex{\CStandardLibrary!rand()}
\label{LCG_simple}

Линейный конгруэнтный генератор, пожалуй, самый простой способ генерировать псевдослучайные числа.

Он не в почете в наше время\footnote{Вихрь Мерсенна куда лучше}, но он настолько прост
(только одно умножение, одно сложение и одна операция \q{И}),
что мы можем использовать его в качестве примера.

\lstinputlisting[style=customc]{patterns/145_LCG/rand_RU.c}

Здесь две функции: одна используется для инициализации внутреннего состояния, а вторая
вызывается собственно для генерации псевдослучайных чисел.

Мы видим, что в алгоритме применяются две константы.
Они взяты из
[William H. Press and Saul A. Teukolsky and William T. Vetterling and Brian P. Flannery, \IT{Numerical Recipes}, (2007)].
Определим их используя выражение \CCpp \TT{\#define}. Это макрос.

Разница между макросом в \CCpp и константой в том, что все макросы заменяются на значения препроцессором
\CCpp и они не занимают места в памяти как переменные.

А константы, напротив, это переменные только для чтения.

Можно взять указатель (или адрес) переменной-константы, но это невозможно сделать с макросом.

Последняя операция \q{И} нужна, потому что согласно стандарту Си \TT{my\_rand()} должна возвращать значение в пределах
0..32767.

Если вы хотите получать 32-битные псевдослучайные значения, просто уберите последнюю операцию \q{И}.

\subsection{x86}

\lstinputlisting[caption=\Optimizing MSVC 2013,style=customasmx86]{patterns/145_LCG/rand_MSVC_2013_x86_Ox.asm}

Вот мы это и видим: обе константы встроены в код.

Память для них не выделяется.
Функция \TT{my\_srand()} просто копирует входное значение во внутреннюю переменную \TT{rand\_state}.

\TT{my\_rand()} берет её, вычисляет следующее состояние \TT{rand\_state}, 
обрезает его и оставляет в регистре EAX.

Неоптимизированная версия побольше:

\lstinputlisting[caption=\NonOptimizing MSVC 2013,style=customasmx86]{patterns/145_LCG/rand_MSVC_2013_x86.asm}

\subsection{x64}

Версия для x64 почти такая же, и использует 32-битные регистры вместо 64-битных
(потому что мы работаем здесь с переменными типа \Tint).

Но функция \TT{my\_srand()} берет входной аргумент из регистра \ECX, а не из стека:

\lstinputlisting[caption=\Optimizing MSVC 2013 x64,style=customasmx86]{patterns/145_LCG/rand_MSVC_2013_x64_Ox_RU.asm}

GCC делает почти такой же код.

\subsection{32-bit ARM}

\lstinputlisting[caption=\OptimizingKeilVI (\ARMMode),style=customasmARM]{patterns/145_LCG/rand.s_Keil_ARM_O3_RU.s}

В ARM инструкцию невозможно встроить 32-битную константу, так что Keil-у приходится размещать их отдельно и дополнительно загружать.
Вот еще что интересно: константу 0x7FFF также нельзя встроить.
Поэтому Keil сдвигает \TT{rand\_state} влево на 17 бит и затем сдвигает вправо на 17 бит.
Это аналогично \CCpp{}-выражению $(rand\_state \ll 17) \gg 17$.
Выглядит как бессмысленная операция, но тем не менее, что она делает это очищает старшие 17 бит, оставляя младшие 15 бит нетронутыми, и это наша цель, в конце концов. \\
\\
\Optimizing Keil для режима Thumb делает почти такой же код.

\subsection{MIPS}

\lstinputlisting[caption=\Optimizing GCC 4.4.5 (IDA),style=customasmMIPS]{patterns/145_LCG/MIPS_O3_IDA_RU.lst}

Ух, мы видим здесь только одну константу (0x3C6EF35F или 1013904223).
Где же вторая (1664525)?

Похоже, умножение на 1664525 сделано только при помощи сдвигов и прибавлений!

Проверим эту версию:

\lstinputlisting[style=customc]{patterns/145_LCG/test.c}

\lstinputlisting[caption=\Optimizing GCC 4.4.5 (IDA),style=customasmMIPS]{patterns/145_LCG/test_O3_MIPS.lst}

Действительно!

\subsubsection{Перемещения в MIPS (\q{relocs})}

Ещё поговорим о том, как на самом деле происходят операции загрузки из памяти и запись в память.

Листинги здесь были сделаны в IDA, которая убирает немного деталей.

Запустим objdump дважды: чтобы получить дизассемблированный листинг и список перемещений:

\lstinputlisting[caption=\Optimizing GCC 4.4.5 (objdump)]{patterns/145_LCG/MIPS_O3_objdump.txt}

Рассмотрим два перемещения для функции \TT{my\_srand()}.

Первое, для адреса 0, имеет тип \TT{R\_MIPS\_HI16}, и второе, для адреса 8, имеет тип \TT{R\_MIPS\_LO16}.

Это значит, что адрес начала сегмента .bss будет записан в инструкцию по адресу 0 (старшая часть адреса)
и по адресу 8 (младшая часть адреса).

Ведь переменная \TT{rand\_state} находится в самом начале сегмента .bss.

Так что мы видим нули в операндах инструкций \LUI и \SW потому что там пока ничего нет~--- 
компилятор не знает, что туда записать.

Линкер это исправит и старшая часть адреса будет записана в операнд инструкции \LUI и младшая часть адреса~---
в операнд инструкции \SW.

\SW просуммирует младшую часть адреса и то что находится в регистре \$V0 (там старшая часть).

Та же история и с функцией my\_rand(): перемещение R\_MIPS\_HI16 указывает линкеру записать старшую часть
адреса сегмента .bss в инструкцию \LUI.

Так что старшая часть адреса переменной rand\_state находится в регистре \$V1.

Инструкция \LW по адресу 0x10 просуммирует старшую и младшую часть и загрузит значение переменной 
rand\_state в \$V0.

Инструкция \SW по адресу 0x54 также просуммирует и затем запишет новое значение в глобальную переменную
rand\_state.

IDA обрабатывает перемещения при загрузке, и таким образом эти детали скрываются.

Но мы должны о них помнить.

% TODO add example of compiled binary, GDB example, etc...


\subsection{Версия этого примера для многопоточной среды}

Версия примера для многопоточной среды будет рассмотрена позже: \myref{LCG_TLS}.

