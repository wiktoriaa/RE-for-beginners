\chapter{\RU{Линейный конгруэнтный генератор как генератор псевдослучайных чисел}\EN{Linear congruential generator as pseudorandom number generator}}
\index{\CStandardLibrary!rand()}
\label{LCG_simple}

\RU{Линейный конгруэнтный генератор, пожалуй, самый просто способ генерировать псевдослучайные числа.}
\EN{Linear congruential generator is probably simplest possible way to generate random numbers.}
\RU{Он не в почете в наше время\footnote{Вихрь Мерсенна куда лучше}, но он настолько прост
(только одно умножение, одно сложение и одна операция ``И''),
что мы можем использовать его в качестве примера.}
\EN{It's not in favour in modern times\footnote{Mersenne twister is better}, but it's so simple 
(just one multiplication, one addition and one AND operation), 
so we can use it as an example.}

\lstinputlisting{patterns/145_LCG/rand.c.\LANG}

\RU{Так что здесь две ф-ции: одна используется для инициализации внутреннего состояния и вторая
вызывается собственно для генерации псевдослучайных чисел.}
\EN{So there are two functions: one is used for internal state initialization, and the second one is called
for pseudorandom numbers generating.}

\RU{Мы видим что в алгоритме применяются две константы}\EN{We see two constants here used in algorithm}.
\RU{Они взяты из}\EN{They are taken from} \cite{Numerical}.
\RU{Я определил их используя выражение \CCpp \TT{\#define}. Это макрос.}
\EN{I defined it using \TT{\#define} \CCpp statement. It's a macro.}
\RU{Разница между макросом в \CCpp и константой в том что все макросы заменяются на значения препроцессором
\CCpp, и они не занимают места в памяти как переменные.}
\EN{The difference between \CCpp macro and constant is that all macros are replaced 
by its value by \CCpp preprocessor,
and it's not holding any value in the memory like variables.}
\RU{А константы, напротив, это переменные только для чтения.}
\EN{By contrast, constant is read-only variable.}
\RU{Можно взять указатель (или адрес) переменной-константы, но это невозможно сделать с макросом.}
\EN{It's possible to take a pointer (or address) of constant variable, but impossible to do so with macro.}

\RU{Последняя операция ``И'' нужно из-за стандарта Си, \TT{my\_rand()} должна возвращать значение в пределах
$0..32767$.}
\EN{The last AND operation is needed because by C-standard, \TT{my\_rand()} should return value in 
$0..32767$ range.}
\RU{Если вы хотите получать 32-битные псевдослучайные значения, просто уберите последнюю операцию ``И''.}
\EN{If you want to get 32-bit pseudorandom values, just the last omit AND operation.}

\section{x86}

\lstinputlisting[caption=\Optimizing MSVC 2013]{patterns/145_LCG/rand_MSVC_2013_x86_Ox.asm}

\RU{Вот мы это и видим: обе константы встроены в код.}
\EN{Here we see it: both constants are embedded into the code.}
\RU{Память для них не выделяется.}\EN{There are no memory allocated for them.}
\RU{Ф-ция \TT{my\_srand()} просто копирует входное значение во внутреннюю переменную \TT{rand\_state}.}
\EN{\TT{my\_srand()} function just copies input value into the internal \TT{rand\_state} variable.}

\RU{\TT{my\_rand()} берет её, вычисляет следующее состояние \TT{rand\_state}, 
обрезает его и оставляет в регистре EAX.}
\EN{\TT{my\_rand()} takes it, calculate next \TT{rand\_state}, cuts it and leaves it in the EAX register.}

\RU{Неоптимизирующая версия побольше}\EN{Non-Optimized version is more verbose}:

\lstinputlisting[caption=\NonOptimizing MSVC 2013]{patterns/145_LCG/rand_MSVC_2013_x86.asm}

\section{x64}

\RU{Версия для x64 почти такая же и использует 32-битные регистры вместо 64-битных
(потому что мы работаем здесь с переменными типа \Tint).}
\EN{x64 version is mostly the same and uses 32-bit registers instead of 64-bit ones 
(because we work with \Tint values here).}
\RU{Но ф-ция \TT{my\_srand()} берет входной аргумент из регистра ECX а не из стека:}
\EN{But \TT{my\_srand()} function takes input argument from the ECX register rather than from stack:}

\lstinputlisting[caption=\Optimizing MSVC 2013 x64]{patterns/145_LCG/rand_MSVC_2013_x64_Ox.asm.\LANG}

\RU{GCC делает почти такой же код}\EN{GCC compiler do mostly the same code}.

\ifdefined\IncludeARM
\section{32-bit ARM}

\lstinputlisting[caption=\OptimizingKeilVI (\ARMMode)]{patterns/145_LCG/rand.s_Keil_ARM_O3.s.\LANG}

\RU{В ARM инструкцию невозможно встроить 32-битную константу, так что Keil-у приходится размещать
их отдельно и дополнительно загружать.}
\EN{It's not possible to embed 32-bit constants into ARM instructions, so Keil is ought to place them externally
and load them additionally.}

\RU{Вот еще что интересно: константу 0x7FFF также нельзя встроить.}
\EN{One interesting thing is that it's not possible to embed 0x7FFF constant as well.}
\RU{Так что то что делает Keil это сдвигает \TT{rand\_state} влево на 17 бит и затем сдвигает вправо на 17 бит.}
\EN{So what Keil does is shifting \TT{rand\_state} left by 17 bits and then shifting it right by 17 bits.}
\RU{Это аналогично \CCpp{}-выражению $(rand\_state \ll 17) \gg 17$.}
\EN{This is analogous to $(rand\_state \ll 17) \gg 17$ statement in \CCpp.}
\RU{Выглядит как бессмысленная операция, но тем не менее, что она делает это очищает старшие 17 бит, оставляя
младшие 15 бит нетронутыми, и это наша цель, в конце концов.}
\EN{Seems to be useless operation, but nevertheless,
what it does is clearing high 17 bits, leaving low 15 bits intact, and that's our goal after all.}\\
\\
\Optimizing Keil \RU{для режима Thumb делает почти такой же код}\EN{for Thumb mode do mostly the same code}.
\fi

\ifdefined\IncludeMIPS
\section{MIPS}

\lstinputlisting[caption=\Optimizing GCC 4.4.5 (IDA)]{patterns/145_LCG/MIPS_O3_IDA.lst.\LANG}

\RU{Ух, мы видим здесь только одну константу}
\EN{Wow, we see here only one constant} (0x3C6EF35F \OrENRU 1013904223).
\RU{Где же вторая}\EN{Where is another one} (1664525)?

\RU{Похоже, умножение на 1664525 сделано только при помощи сдвигов и прибавлений!}
\EN{It seems, multiplication by 1664525 is done using just shifts and additions!}
\RU{Проверим эту версию}\EN{Let's check this assumption}:

\lstinputlisting{patterns/145_LCG/test.c}

\lstinputlisting[caption=\Optimizing GCC 4.4.5 (IDA)]{patterns/145_LCG/test_O3_MIPS.lst}

\RU{Действительно}\EN{Indeed}!

\subsection{\RU{Релоки в MIPS}\EN{MIPS relocations}}

\RU{Еще поговорим о том, как на самом деле происходят операции загрузки из памяти и запись в память.}
\EN{I would also focus on how such operations as load from memory and store to memory are actually works.}
\RU{Листинги здесь были сделаны в IDA, которая убирает немного деталей.}
\EN{The listings here are produced by IDA, which hiding some details.}

\RU{Я запущу objdump дважды, чтобы получить дизассемблированный листинг и еще список релоков:}
\EN{I'll run objdump twice to get disassembled listing and also relocations list:}

\lstinputlisting[caption=\Optimizing GCC 4.4.5 (objdump)]{patterns/145_LCG/MIPS_O3_objdump.txt}

\RU{Рассмотрим два релока для ф-ции \TT{my\_srand()}.}
\EN{Let's consider two relocations for the \TT{my\_srand()} function.}
\RU{Первый, для адреса 0, имеет тип \TT{R\_MIPS\_HI16}, и второй, для адреса 8, имеет тип \TT{R\_MIPS\_LO16}.}
\EN{First for address 0 has type of \TT{R\_MIPS\_HI16} and second for address 8 has types \TT{R\_MIPS\_LO16}.}
\RU{Это значит, что адрес начала сегмента .bss будет записан в инструкцию по адресу 0 (старшая часть адреса)
и по адресу 8 (младшая асть адреса).}
\EN{That means that address of beginning of .bss segment will be written into instructions at
address of 0 (high part of address) and 8 (low part of address).}

\RU{Ведь переменная \TT{rand\_state} находится в самом начале сегмента .bss.}
\EN{\TT{rand\_state} variable is at the very start of .bss segment.}

\RU{Так что мы видим нули в операндах инструкций LUI и SW потому что там пока ничего нет --- 
компилятор не знает, что туда записать.}
\EN{So we see zeroes in the operands of instructions LUI and SW, because nothing is there yet --- 
compiler don't know what to write there.}
\RU{Линкер это исправит и старшая часть адреса будет записана в операнд инструкции LUI и младшая часть адреса ---
в операнд инструкции SW.}
\EN{Linker will fix this and high part of address will be written into LUI instruction operand and
low part of address --- to operand of SW instruction.}
\RU{SW просуммирует младшую асть адреса и то что находится в регистре \$V0 (там старшая часть).}
\EN{SW will sum up low part of address and what is in \$V0 register (high part is there).}

\RU{Та же история и с ф-цией my\_rand(): релок R\_MIPS\_HI16 указывает линкеру записать старшую часть
адреса сегмента .bss в инструкцию LUI.}
\EN{The same story about my\_rand() function: R\_MIPS\_HI16 relocation instructs linker to write high part
of .bss segment address into LUI instruction.}
\RU{Так что старшая часть адреса переменной rand\_state будет находится в регистре \$V1.}
\EN{So, high part of rand\_state variable address will reside in \$V1 register.}
\RU{Инструкция LW по адресу 0x10 просуммирует старшую и младшую часть и загрузит значение переменной 
rand\_state в \$V1.}
\EN{LW instruction at address 0x10 will sum up high and low part and load value of rand\_state 
variable into \$V1.}
\RU{Инструкция SW по адресу 0x54 также просуммирует и затем запишет новое значение в глобальную переменную
rand\_state.}
\EN{SW instruction at address 0x54 will also do the summing and then will store new value 
to rand\_state global variable.}

\RU{Так что IDA обрабатывает релоки при загрузке, и таким образом эти детали скрываются.}
\EN{So, IDA processes relocations while loading, thus hiding these details.}
\RU{Но мы должны о них помнить.}\EN{But we should remember about them.}

% TODO add example of compiled binary, GDB example, etc...

\fi

\section{\RU{Версия этого примера для многопоточной среды}\EN{Thread-safe version of the example}}

\RU{Версия примера для многопоточной среды будет рассмотрена позже}
\EN{Thread-safe version of the example is to be demonstrated later}: \ref{LCG_TLS}.
