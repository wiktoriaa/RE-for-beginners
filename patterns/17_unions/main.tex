\section{\RU{Объединения (union)}\EN{Unions}\DE{Unions}}

\EN{\CCpp \IT{union} is mostly used for interpreting a variable (or memory block) of one data type as a variable of another data type.}
\DE{Die \\Cpp \IT{union} wird hauptsächlich verwendet um eine Variable (oder einen Speicherblock) eines Datentyps als
Variable eines anderen Datentyps zu interpretieren.}
\RU{\IT{union} в \CCpp используется в основном для интерпретации переменной (или блока памяти) одного типа как переменной другого типа.}

% sections
\section{\RU{Пример генератора случайных чисел}\EN{Pseudo-random number generator example}}
\label{FPU_PRNG}

\RU{Если нам нужны случайные значения с плавающей запятой в интервале от 0 до 1, самое простое это взять
\ac{PRNG} вроде Mersenne twister.
Он выдает случайные 32-битные числа в виде DWORD.
Затем мы можем преобразовать это число в \Tfloat и затем разделить на \TT{RAND\_MAX} (\TT{0xFFFFFFFF} в данном случае)\EMDASH{}
полученное число будет в интервале от 0 до 1.}
\EN{If we need float random numbers between 0 and 1, the simplest thing is to use a \ac{PRNG} like
the Mersenne twister. 
It produces random 32-bit values in DWORD form. 
Then we can transform this value to \Tfloat and then
divide it by \TT{RAND\_MAX} (\TT{0xFFFFFFFF} in our case)\EMDASH{}
we getting a value in the 0..1 interval.}

\RU{Но как известно, операция деления\EMDASH{}это медленная операция. 
Да и вообще хочется избежать лишних операций с FPU.
Сможем ли мы избежать деления?}
\EN{But as we know, division is slow.
Also, we would like to issue as few FPU operations as possible.
Can we get rid of the division?}

\index{IEEE 754}
\RU{Вспомним состав числа с плавающей запятой: это бит знака, биты мантиссы и биты экспоненты. 
Для получения случайного числа, нам нужно просто заполнить случайными битами все биты мантиссы!}
\EN{Let's recall what a floating point number consists of: sign bit, significand bits and exponent bits.
We just need to store random bits in all significand bits to get a random float number!}

\RU{Экспонента не может быть нулевой (иначе число с плавающей точкой будет денормализованным), 
так что в эти биты мы запишем \TT{01111111}\EMDASH{}
это будет означать что экспонента равна единице. Далее заполняем мантиссу случайными битами, 
знак оставляем в виде 0 (что значит наше число положительное), и вуаля. 
Генерируемые числа будут в интервале от 1 до 2, так что нам еще нужно будет отнять единицу.}
\EN{The exponent cannot be zero (the floating number is denormalized in this case), so we are storing \TT{01111111} 
to exponent\EMDASH{}this means that the exponent is 1. 
Then we filling the significand with random bits, set the sign bit to
0 (which means a positive number) and voilà.
The generated numbers is to be between 1 and 2, so we must also subtract 1.}

\newcommand{\URLXOR}{\url{http://go.yurichev.com/17308}}

\RU{В моем примере\footnote{идея взята здесь: \URLXOR} 
применяется очень простой линейный конгруэнтный генератор случайных чисел, выдающий 32-битные числа.
Генератор инициализируется текущим временем в стиле UNIX.}
\EN{A very simple linear congruential random numbers generator is used in my 
example\footnote{the idea was taken from: \URLXOR}, it produces 32-bit numbers. 
The \ac{PRNG} is initialized with the current time in UNIX timestamp format.}

\RU{Далее, тип \Tfloat представляется в виде \IT{union}\EMDASH{}это конструкция \CCpp позволяющая 
интерпретировать часть памяти по-разному. В нашем случае, мы можем создать переменную типа \TT{union} 
и затем обращаться к ней как к \Tfloat или как к \IT{uint32\_t}. Можно сказать, что это хак, причем грязный.}
\EN{Here we represent the \Tfloat type as an \IT{union}\EMDASH{}it is the \CCpp construction that enables us
to interpret a piece of memory as different types.
In our case, we are able to create a variable
of type \TT{union} and then access to it as it is \Tfloat or as it is \IT{uint32\_t}. 
It can be said, it is just a hack. A dirty one.}

% WTF?
\RU{Код целочисленного \ac{PRNG} точно такой же, как мы уже рассматривали ранее:}
\EN{The integer \ac{PRNG} code is the same as we already considered:} \myref{LCG_simple}.
\RU{Так что и в скомпилированном виде этот код будет опущен.}
\EN{So this code in compiled form is omitted.}

\lstinputlisting{patterns/17_unions/FPU_PRNG/FPU_PRNG.cpp.\LANG}

\subsection{x86}

\lstinputlisting[caption=\Optimizing MSVC 2010]{patterns/17_unions/FPU_PRNG/MSVC2010_Ox_Ob0.asm.\LANG}

\EN{Function names are so strange here because this example was compiled as C++ and this is name mangling in C++,
we will talk about it later:}%
\RU{Имена функций такие странные, потому что этот пример был скомпилирован как Си++, и это манглинг имен в Си++, 
мы будем рассматривать это позже:} \myref{namemangling}.

\RU{Если скомпилировать это в MSVC 2012, компилятор будет использовать SIMD-инструкции для FPU, читайте об этом
здесь:}
\EN{If we compile this in MSVC 2012, it uses the SIMD instructions for the FPU, read more about it here:}
\myref{FPU_PRNG_SIMD}.

\subsection{MIPS}

\lstinputlisting[caption=\Optimizing GCC 4.4.5]{patterns/17_unions/FPU_PRNG/MIPS_O3_IDA.lst.\LANG}

\EN{There is also an useless LUI instruction added for some weird reason.}
\RU{Здесь снова зачем-то добавлена инструкция LUI, которая ничего не делает.}
\EN{We considered this artifact earlier:}
\RU{Мы уже рассматривали этот артефакт ранее:} \myref{MIPS_FPU_LUI}.

\subsection{ARM (\ARMMode)}

\lstinputlisting[caption=\Optimizing GCC 4.6.3 (IDA)]{patterns/17_unions/FPU_PRNG/raspberry_GCC_O3_IDA.lst.\LANG}

\index{objdump}
\index{binutils}
\index{IDA}
\RU{Мы также сделаем дамп в objdump и увидим что FPU-инструкции имеют немного другие имена чем в \IDA.}%
\EN{We'll also make a dump in objdump and we'll see that the FPU instructions have different names than in \IDA.}
\EN{Apparently, IDA and binutils developers used different manuals?}
\RU{Наверное, разработчики IDA и binutils пользовались разной документацией?}
\EN{Perhaps, it would be good to know both instruction name variants.}
\RU{Должно быть, будет полезно знать оба варианта названий инструкций.}

\lstinputlisting[caption=\Optimizing GCC 4.6.3 (objdump)]{patterns/17_unions/FPU_PRNG/raspberry_GCC_O3_objdump.lst}

\EN{The instructions at 0x5c in float\_rand() and at 0x38 in main() are random noise.}
\RU{Инструкции по адресам 0x5c в float\_rand() и 0x38 в main() это случайный мусор.}

\ifdefined\RUSSIAN
\else
\section{Calculating machine epsilon}

\subsection{x86}

Machine epsilon is a smallest possible granule \ac{FPU} can work with\RU{\footnote{В русскоязычной
литературе встречается также термин ``машинный ноль''.}}.
The more bits allocated for floating point number, the smaller machine epsilon.
It is $2^{-23} = 1.19e-07$ for \Tfloat and $2^{-52} = 2.22e-16$ for double.

It's interesting, how easy it's possible to calculate machine epsilon:

\lstinputlisting{patterns/17_unions/epsilon/float.c}

What we do here is just treating fraction part of IEE 754 number as integer and adding 1 to it.
Resulting number will be $starting\_value+machine\_epsilon$, so we just need to subtract
starting value (using floating point arithmetics) to measure, what number one bit reflects
in the single precision (\Tfloat).

union serves here as a way to access IEEE 754 number as a regular integer.
Adding 1 to it is in fact adds 1 to \IT{fraction} part of number, however, needless to say,
overflow is possible, which will add yet another bit to exponent part.

\lstinputlisting[caption=\Optimizing MSVC 2010]{patterns/17_unions/epsilon/float_MSVC_2010_Ox.asm}

Second FST instruction is redundant: there are no need to store input value to the same
place (compiler decided to allocate $v$ variable at the same point of local stack as input 
argument).

Then it is incremented with INC, as it is usual integer variable.
Then it is loaded into FPU as it is 32-bit IEEE 754 number, FSUBR do the job and resulting
value is in the ST0.

Two last FSTP/FLD instruction pair is redundant, but compiler didn't optimized them.

\ifdefined\IncludeARM
\subsection{ARM64}

Let's extend our example to 64-bit:

\lstinputlisting[label=machine_epsilon_double_c]{patterns/17_unions/epsilon/double.c}

ARM64 has no instruction which can add a number to FPU D-register, 
so input value (came in D0) is first copied into GPR,
incremented, copied to FPU register D1, then subtraction occurred.

\lstinputlisting[caption=\Optimizing GCC 4.9 ARM64]{patterns/17_unions/epsilon/double_GCC49_ARM64_O3.s}

See also this example compiled for x64 with SIMD instructions: \ref{machine_epsilon_x64_and_SIMD}.
\fi

\subsection{Conclusion}

It's hard to say, whether someone will need this trickery in real-world code, 
but as I write many times in this book, this example is serving well 
for explaining IEEE 754 format and union feature of \CCpp.
\fi

\EN{\section{FSCALE replacement}
\myindex{x86!\Instructions!FSCALE}

Agner Fog in his \IT{Optimizing subroutines in assembly language / An optimization guide for x86 platforms} work
\footnote{\url{http://www.agner.org/optimize/optimizing_assembly.pdf}} states that \INS{FSCALE} \ac{FPU} instruction
(calculating $2^n$) may be slow on many CPUs, and he offers faster replacement.

Here is my translation of his assembly code to \CCpp:

\lstinputlisting[style=customc]{patterns/17_unions/FSCALE.c}

\INS{FSCALE} instruction may be faster in your environment, but still, it's a good example of \IT{union}'s and the fact
that exponent is stored in $2^n$ form,
so an input $n$ value is shifted to the exponent in IEEE 754 encoded number.
Then exponent is then corrected with addition of 0x3f800000 or 0x3ff0000000000000.

The same can be done without shift using \IT{struct}, but internally, shift operations still occurred.

}
\DE{\section{FSCALE Ersatz}
\myindex{x86!\Instructions!FSCALE}
Agner Fog schreibt in seiner Abhandlung \IT{Optimizing subroutines in assembly language / An optimization guide for x86
platforms} \footnote{\url{http://www.agner.org/optimize/optimizing_assembly.pdf}} , dass der Befehl \INS{FSCALE}
\ac{FPU} (der $2^n$ berechnet) auf vielen CPUs langsam ist und bietet einen schnelleren Ersatz an.

Hier ist meine Übersetzung von seinem Assemblercode in \CCpp:

\lstinputlisting[style=customc]{patterns/17_unions/FSCALE.c}
Der Befehl \INS{FSCALE} kann zwar in bestimmten Umgebungen schneller sein, ist aber vor allem ein gutes Beispiel für
\IT{unions} und die Tatsache, dass der Exponent in der Form $2^n$ gespeichert wird, sodass ein Eingabewert $n$ zum
Exponenten nach IEEE 754 Standard verschoben wird.
Der Exponent wird dann durch Addition von 0x3f800000 oder 0x3ff0000000000000 korrigiert.

Das gleiche kann ohne Verschiebung durch ein \IT{struct} erreicht werden, aber intern werden stets Schiebebefehle
verwendet.
}

\subsection{\RU{Быстрое вычисление квадратного корня}\EN{Fast square root calculation}\DE{Schnelle Berechnung der
Quadratwurzel}}

\RU{Вот где еще можно на практике применить трактовку типа \Tfloat как целочисленного, это быстрое вычисление квадратного корня.}%
\EN{Another well-known algorithm where \Tfloat is interpreted as integer is fast calculation of square root.}
\DE{Ein anderer bekannter Algorithmus, in dem \Tfloat als \Tint interpretiert wird, ist die schnelle Berechnung einer
Quadratwurzel.}

\begin{lstlisting}[caption=DE{Quellcode stammt aus der Wikipedia}\EN{The source code is taken from
Wikipedia}\RU{Исходный код взят из Wikipedia}:
\url{http://go.yurichev.com/17364},style=customc] /* Assumes that float is in the IEEE 754 single precision floating point format
 * and that int is 32 bits. */
float sqrt_approx(float z)
{
    int val_int = *(int*)&z; /* Same bits, but as an int */
    /*
     * To justify the following code, prove that
     *
     * ((((val_int / 2^m) - b) / 2) + b) * 2^m = ((val_int - 2^m) / 2) + ((b + 1) / 2) * 2^m)
     *
     * where
     *
     * b = exponent bias
     * m = number of mantissa bits
     *
     * .
     */
 
    val_int -= 1 << 23; /* Subtract 2^m. */
    val_int >>= 1; /* Divide by 2. */
    val_int += 1 << 29; /* Add ((b + 1) / 2) * 2^m. */
 
    return *(float*)&val_int; /* Interpret again as float */
}
\end{lstlisting}

\ifdefined\RUSSIAN
В качестве упражнения, вы можете попробовать скомпилировать эту функцию и разобраться, как она работает. \\
\\
Имеется также известный алгоритм быстрого вычисления $\frac{1}{\sqrt{x}}$.
\myindex{Quake III Arena}
Алгоритм стал известным, вероятно потому, что был применен в Quake III Arena.

Описание алгоритма есть в Wikipedia: \url{http://go.yurichev.com/17361}.
\fi % RUSSIAN

\ifdefined\ENGLISH
As an exercise, you can try to compile this function and to understand, how it works. \\
\\
There is also well-known algorithm of fast calculation of $\frac{1}{\sqrt{x}}$.
\myindex{Quake III Arena}
Algorithm became popular, supposedly, because it was used in Quake III Arena.

Algorithm description can be found in Wikipedia: \url{http://go.yurichev.com/17360}.
\fi % ENGLISH

\ifdefined\GERMAN
Versuchen Sie als Übung, diese Funktion zu kompilieren und zu verstehen wie sie funktioniert.\\\\
Es gibt auch einen bekannten Algorithmus zur schnellen Berechnung von $\frac{1}{\sqrt{x}}$.
\myindex{Quake III Arena}
Der Algorithmus wurde vermutlich so populär, weil er in Quake III Arena verwendet wurde.
Eine Beschreibung des Algorithmus' findet man bei Wikipedia: \url{http://go.yurichev.com/17360}.
\fi % GERMAN

