\ifdefined\RUSSIAN
\else
\section{Calculating machine epsilon}

\subsection{x86}

Machine epsilon is a smallest possible granule \ac{FPU} can work with\RU{\footnote{В русскоязычной
литературе встречается также термин ``машинный ноль''.}}.
The more bits allocated for floating point number, the smaller machine epsilon.
It is $2^{-23} = 1.19e-07$ for \Tfloat and $2^{-52} = 2.22e-16$ for double.

It's interesting, how easy it's possible to calculate machine epsilon:

\lstinputlisting{patterns/17_unions/epsilon/float.c}

What we do here is just treating fraction part of IEE 754 number as integer and adding 1 to it.
Resulting number will be $starting\_value+machine\_epsilon$, so we just need to subtract
starting value (using floating point arithmetics) to measure, what number one bit reflects
in the single precision (\Tfloat).

union serves here as a way to access IEEE 754 number as a regular integer.
Adding 1 to it is in fact adds 1 to \IT{fraction} part of number, however, needless to say,
overflow is possible, which will add yet another bit to exponent part.

\lstinputlisting[caption=\Optimizing MSVC 2010]{patterns/17_unions/epsilon/float_MSVC_2010_Ox.asm}

Second FST instruction is redundant: there are no need to store input value to the same
place (compiler decided to allocate $v$ variable at the same point of local stack as input 
argument).

Then it is incremented with INC, as it is usual integer variable.
Then it is loaded into FPU as it is 32-bit IEEE 754 number, FSUBR do the job and resulting
value is in the ST0.

Two last FSTP/FLD instruction pair is redundant, but compiler didn't optimized them.

\ifdefined\IncludeARM
\subsection{ARM64}

Let's extend our example to 64-bit:

\lstinputlisting[label=machine_epsilon_double_c]{patterns/17_unions/epsilon/double.c}

ARM64 has no instruction which can add a number to FPU D-register, 
so input value (came in D0) is first copied into GPR,
incremented, copied to FPU register D1, then subtraction occurred.

\lstinputlisting[caption=\Optimizing GCC 4.9 ARM64]{patterns/17_unions/epsilon/double_GCC49_ARM64_O3.s}

See also this example compiled for x64 with SIMD instructions: \ref{machine_epsilon_x64_and_SIMD}.
\fi

\subsection{Conclusion}

It's hard to say, whether someone will need this trickery in real-world code, 
but as I write many times in this book, this example is serving well 
for explaining IEEE 754 format and union feature of \CCpp.
\fi
