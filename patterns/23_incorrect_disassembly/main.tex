\section{\IFRU{Неверно дизассемблированный код}{Incorrectly disassembled code}}

\IFRU{Практикующие reverse engineer-ы часто сталкиваются с неверно дизассемблированным кодом}
{Practicing reverse engineers often dealing with incorrectly disassembled code}.

\subsection{\IFRU{Дизассемблирование началось в неверном месте}{Disassembling started incorrectly} (x86)}

\IFRU{В отличие от ARM и MIPS (где у каждой инструкции длина или 2 или 4 байта), x86-инструкции имеют переменную длину,
так что, любой дизассемблер, начиная работу с середины x86-инструкции, может выдать неверные результаты.}
{Unlike ARM and MIPS (where any instruction has length of 2 or 4 bytes), x86 instructions has variable size,
so, any disassembler, starting at the middle of x86 instruction, may produce incorrect results.}

\IFRU{Как пример}{As an example}:

\lstinputlisting{patterns/23_incorrect_disassembly/x86_wrong_start.asm}

\IFRU{В начале мы видим неверно дизассемблированные инструкции, но потом, так или иначе, дизассемблер находит верный след}
{There are incorrectly disassembled instructions at the beginning, but eventually, disassembler finds right 
track}.

\subsection{\IFRU{Как выглядят случайные данные в дизассемблированном виде}{How random noise looks disassembled}?}

\IFRU{Общее, что можно сразу заметить, это}{Common properties which can be easily spotted are}:

\begin{itemize}
\item \IFRU{Необычно большой разброс инструкций}{Unusually big instruction dispersion}.
\IFRU{Самые частые x86-инструкции это}{Most frequent x86 instructions are} \PUSH{}, \MOV{}, \CALL{}, \IFRU{но здесь мы видим
инструкции из любых групп: \ac{FPU}-инструкции, инструкции \TT{IN}/\TT{OUT}, редкие и системные инструкции, всё друг с другом смешано 
в одном месте}{but here we will see
instructions from any instruction group: \ac{FPU} instructions, \TT{IN}/\TT{OUT} instructions, rare and system instructions,
everything messed up in one single place}.

\item \IFRU{Большие и случайные значения, смещения,}{Big and random values, offsets and} immediates.

\item \IFRU{Переходы с неверными смещениями часто имеют адрес перехода в середину другой инструкции}
{Jumps having incorrect offsets often jumping into the middle of another instructions}.
\end{itemize}

\lstinputlisting[caption=\randomNoise{} (x86)]{patterns/23_incorrect_disassembly/x86.asm}

\lstinputlisting[caption=\randomNoise{} (x86-64)]{patterns/23_incorrect_disassembly/x64.asm}

\index{ARM}
\lstinputlisting[caption=\randomNoise{} (ARM \IFRU{в режиме ARM}{in ARM mode})]{patterns/23_incorrect_disassembly/ARM.asm}

\lstinputlisting[caption=\randomNoise{} (ARM \IFRU{в режиме Thumb}{in Thumb mode})]{patterns/23_incorrect_disassembly/ARM_thumb.asm}

\index{MIPS}
\lstinputlisting[caption=\randomNoise (MIPS little endian)]{patterns/23_incorrect_disassembly/MIPS.asm}

\IFRU{Также важно помнить, что хитрым образом написанный код для распаковки и дешифровки (включая самомодифицирующийся),
также может выглядеть как случайный шум, тем не менее, он исполняется корректно}{It is also important to keep in mind that 
cleverly constructed unpacking and decrypting code 
(including self-modifying) may looks like noise as well, nevertheless, it executes correctly}.

\subsection{\IFRU{Информационная энтропия среднестатистического кода}{Information entropy of average code}}

\index{\IFRU{Информационная энтропия}{Information entropy}}
\IFRU{Результаты работы утилиты \IT{ent}}{\IT{ent} utility results}\footnote{\url{http://www.fourmilab.ch/random/}}.

(\IFRU{Энтропия идеально сжатого (или зашифрованного) файла\EMDASH{}8 бит на байт; файла с нулями любой длины\EMDASH{}0 бит на байт.}
{Entropy of ideally compressed (or encrypted) file is 8 bits per byte; of zero file of arbitrary size if 0 bits per byte.})

\IFRU{Здесь видно что код для CPU с 4-байтными инструкциями (ARM в режиме ARM и MIPS) наименее экономичны в этом смысле.}
{Here we can see that a code for CPU with 4-byte instructions (ARM in ARM mode and MIPS) is least effective in this sense.}

\subsubsection{x86}

\IFRU{Секция \TT{.text} файла \TT{ntoskrnl.exe} из}
{\TT{.text} section of \TT{ntoskrnl.exe} file from} Windows 2003:

\begin{lstlisting}
Entropy = 6.662739 bits per byte.

Optimum compression would reduce the size
of this 593920 byte file by 16 percent.
...
\end{lstlisting}

\IFRU{Секция \TT{.text} файла}{\TT{.text} section of} \TT{ntoskrnl.exe} \IFRU{из}{from} Windows 7 x64:

\begin{lstlisting}
Entropy = 6.549586 bits per byte.

Optimum compression would reduce the size
of this 1685504 byte file by 18 percent.
...
\end{lstlisting}

\subsubsection{ARM (Thumb)}
\index{ARM}

AngryBirds Classic:

\begin{lstlisting}
Entropy = 7.058766 bits per byte.

Optimum compression would reduce the size
of this 3336888 byte file by 11 percent.
...
\end{lstlisting}

\subsubsection{ARM (\IFRU{режим ARM}{ARM mode})}

Linux Kernel 3.8.0:

\begin{lstlisting}
Entropy = 6.036160 bits per byte.

Optimum compression would reduce the size
of this 6946037 byte file by 24 percent.
...
\end{lstlisting}

\subsubsection{MIPS (little endian)}
\index{MIPS}

\IFRU{Секция \TT{.text} файла}{\TT{.text} section of} \TT{user32.dll} \IFRU{из}{from} Windows NT 4:

\begin{lstlisting}
Entropy = 6.098227 bits per byte.

Optimum compression would reduce the size
of this 433152 byte file by 23 percent.
....
\end{lstlisting}

