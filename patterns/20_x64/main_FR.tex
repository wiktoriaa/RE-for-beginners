\section{64 bits}

\subsection{x86-64}
\myindex{x86-64}
\label{x86-64}

Il s'agit d'une extension à 64 bits de l'architecture x86.

Pour l'ingénierie inverse, les changements les plus significatifs sont:

\myindex{\CLanguageElements!\Pointers}
\begin{itemize}

\item

La plupart des registres (à l'exception des registres FPU et SIMD) ont été étendus à 64 bits et 
leur nom préfixé de la lettre R. 8 registres ont également été ajoutés.
Les \ac{GPR} sont donc désormais: \RAX, \RBX, \RCX, \RDX, 
\RBP, \RSP, \RSI, \RDI, \Reg{8}, \Reg{9}, \Reg{10}, 
\Reg{11}, \Reg{12}, \Reg{13}, \Reg{14}, \Reg{15}. 

Les \IT{anciens} registres restent accessibles de la manière habituelle. Ainsi, l'utilisation de 
\EAX donne accès aux 32 bits de poids faible du registre \RAX:

\RegTableOne{RAX}{EAX}{AX}{AH}{AL}

Les nouveaux registres \GTT{R8-R15} possèdent eux aussi des sous-parties : \GTT{R8D-R15D} 
(pour les 32 bits de poids faible), \GTT{R8W-R15W} (16 bits de poids faible), \GTT{R8L-R15L} 
(8 bits de poids faible).

\RegTableFour{R8}{R8D}{R8W}{R8L}

Les registres SIMDont vu leur nombre passer de 8 à 16: \XMM{0}-\XMM{15}.

\item

En environnement Win64, la convention d'appel de fonctions est légèrement différente et ressemble 
à la convention fastcall (\myref{fastcall}).
Les 4 premiers arguments sont stockés dans les registres \RCX, \RDX, \Reg{8} et \Reg{9}. Les 
arguments suivants~---sur la pile.
La fonction \glslink{caller}{appelante} doit également allouer 32 octets pour utilisation par la fonction
\glslink{callee}{appelée} qui pourra y sauvegarder les registres contenant les 4 premiers arguments.
Les fonctions les plus simples peuvent utiliser les arguments directement depuis les registres. 
En revanche, les fonctions plus complexes peuvent sauvegarder ces registres sur la pile.

L'ABI System V AMD64 (Linux, *BSD, \MacOSX)\SysVABI ressemble elle aussi à la convention fastcall.
Elle utilise 6 registres \RDI, \RSI, \RDX, \RCX, \Reg{8}, \Reg{9} pour les 6 premiers arguments. 
Tous les suivants sont passés sur la pile.

Référez-vous également à la section sur les conventions d'appel~(\myref{sec:callingconventions}).

\item
Pour des raisons de compatibilité, le type \CCpp \Tint conserve sa taille de 32bits.

\item
Tous les pointeurs sont désormais sur 64 bits.

\end{itemize}

\myindex{Register allocation}

Dans la mesure où le nombre de registres a doublé, les compilateurs disposent de plus de marge 
de manoeuvre en matière d'\glslink{register allocator}{allocation des registres}. 
Pour nous, il en résulte que le code généré contient moins de variables locales.

\myindex{DES}

Par exemple, la fonction qui calcule la première S-box de l'algorithme de chiffrage DES traite 
en une fois au moyen de la méthode DES bitslice des valeurs de 32/64/128/256 bits (en fonction 
du type \GTT{DES\_type} (uint32, uint64, SSE2 or AVX)).
Pour en savoir plus sur cette technique, voyez ~(\myref{bitslicedes}):

\lstinputlisting[style=customc]{patterns/20_x64/19_1.c}

Cette fonction contient de nombreuses variables locales, mais toutes ne se retrouveront pas dans 
la pile. Compilons ce fichier avec MSVC 2008 et l'option \GTT{/Ox}:

\lstinputlisting[caption=MSVC 2008 \Optimizing,style=customasmx86]{patterns/20_x64/19_2_msvc_Ox.asm}

Seules 5 variables ont été allouées dans la pile par le compilateur.

Essayons maintenant une compilation avec la version 64 bits de MSVC 2008:

\lstinputlisting[caption=MSVC 2008 \Optimizing,style=customasmx86]{patterns/20_x64/19_3_msvc_x64.asm}

Le compilateur n'a pas eu besoin d'allouer de l'espace sur la pile. \GTT{x36} est synonyme de \GTT{a5}.

\iffalse
% FIXME1 невнятно

Nous observons également que les registres \RCX et \RDX ont été sauvegardés dans l'espace alloué 
par l'\gls{appelant} et que les registres \Reg{8} et \Reg{9} sont utilisés directement, sans avoir 
à être sauvegardés.
\fi

Il existe cependant des CPUs qui possèdent beaucoup plus de \ac{GPR}. Itanium possède ainsi 128 
registres.

\subsection{ARM}

Les instructions 64 bits sont apparues avec ARMv8.

\subsection{Nombres flottants}

Le traitement des nombres flottants en environnement x86-64 est expliqué ici: \myref{floating_SIMD}.

\subsection{Critiques concernant l'architecture 64 bits}

Certains se sont parfois irrité du doublement de la taille des pointeurs, en particulier dans le 
cache puisque les \ac{CPU}s x64 ne peuvent de toute manière adresser que des adresses \ac{RAM} 
sur 48 bits.

\begin{framed}
\begin{quotation}
Les pointeurs ont perdu mes faveurs au point que j'en viens à les injurier. Si je cherche vraiment
à utiliser au mieux les capacités de mon ordinateur 64 bits, j'en conclus que je ferais mieux de
ne pas utiliser de pointeurs. Les registres de mon ordinateur sont sur 64 bits, mais je n'ai que
2Go de RAM. Les pointeurs n'ont donc jamais plus de 32 bits significatifs. Et pourtant, chaque
fois que j'utilise un pointeur, il me coûte 64 bits ce qui double la taille de ma structure de
données. Pire, ils atterrissent dans mon cache et en gaspillent la moitié et cela me coûte car le
cache est cher.

Donc, si je cherche à grappiller, j'en viens à utiliser des tableaux au lieu de pointeurs. Je
rédige des macros compliquées qui peuvent laisser l'impression à tort que j'utilise des pointeurs.
\end{quotation}
\end{framed}

( Donald Knuth dans ``Coders at Work: Reflections on the Craft of Programming ''. )

Certains en sont venus à fabriquer leurs propres allocateurs de mémoire.
\myindex{CryptoMiniSat}

Il est intéressant de se pencher sur le cas de CryptoMiniSat\footnote{\url{https://github.com/msoos/cryptominisat/}}.
Ce programme qui utilise rarement plus de 4Go de \ac{RAM}, fait un usage intensif des pointeurs.
Il nécessite donc moins de mémoire sur les architectures 32 bits que sur celles à 64 bits.
Pour remédier à ce problème, l'auteur a donc programmé son propre allocateur 
(dans \IT{clauseallocator.(h|cpp)}), qui lui permet d'allouer de la mémoire en utilisant des 
identifiants sur 32 bits à la place de pointeurs sur 64 bits.

