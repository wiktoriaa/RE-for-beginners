\chapter{\EN{More about}\RU{Еще немного о} MIPS}

\section{Instructions}

There are 3 kinds of instructions:

\begin{itemize}

\item R-type: those which has 3 registers.
R-instruction are often has the following form:

\begin{lstlisting}
instruction destination, source1, source2
\end{lstlisting}

One important thing to remember is that when first and second register is the same, 
IDA may show it in shorter form:

\begin{lstlisting}
instruction destination/source1, source2
\end{lstlisting}

That somewhat reminds us Intel-syntax of x86 assembly language.

\item I-type: those which has 16-bit immediate value.
\item J-type: jump/branch instructions, has 26 bits for offset encoding.

\end{itemize}

% sections
\ifx\RUSSIAN\undefined
\section{\RU{Загрузка констант в регистр}\EN{Loading constants into register}}

\begin{lstlisting}
unsigned int f()
{
	return 0x12345678;
};
\end{lstlisting}

MIPS, just like ARM, has all instructions with size of 32-bit, so it's not possible to
embed 32-bit constant into one instruction.
\index{MIPS!\Instructions!LI}
\index{MIPS!\Instructions!ORI}
So this translates to at least two instructions: 
first loads high part of 32-bit number and the second
one apply OR operation which effectively sets low 16-bit part of the target register:

\begin{lstlisting}[caption=GCC 4.4.5 -O3 (assembly output)]
        li      $2,305397760                    # 0x12340000
        j       $31
        ori     $2,$2,0x5678
\end{lstlisting}

\IDA is fully aware of such code patterns, so it shows the last ORI instruction as LI pseudoinstruction
allegedly it loads full 32-bit number into V0 register --- for convenience.

\index{MIPS!\Instructions!LUI}

\begin{lstlisting}[caption=GCC 4.4.5 -O3 (IDA)]
         lui     $v0, 0x1234
         jr      $ra
         li      $v0, 0x12345678
\end{lstlisting}

GCC assembly output has LI instruction, but in fact, LUI (``Load Upper Imeddiate'') there,
which stores 16-bit value into high part of register.

\fi


\section{Further reading about MIPS}

\cite{MIPSRun}.
