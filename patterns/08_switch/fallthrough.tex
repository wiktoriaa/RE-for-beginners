\section{Fallthrough}

\RU{Еще одно популярное использование}\EN{Another very popular usage of} \TT{switch()} 
\EN{is fallthrough}\RU{это т.н. ``fallthrough'' (``проваливаясь'')}.
\RU{Вот простой пример}\EN{Here is a small example}:

\lstinputlisting[numbers=left]{patterns/08_switch/fallthrough.c}

\RU{Если}\EN{If} $type=1$ (R), $r$ \RU{будет выставлен в}\EN{to be set to} $1$, \RU{если}\EN{if} 
$type=2$ (W), $w$ \RU{будет выставлен в}\EN{is to be set to} $2$.
\RU{В случае}\EN{In case of} $type=3$ (RW), \RU{обе}\EN{both} $r$ \AndENRU $w$ \RU{будут 
выставлены в}\EN{are to be set to} $1$.

\RU{Фрагмент кода на строке 14 будет исполнен в двух случаях: если}\EN{The piece of code at 
line 14 is executed in two cases: if} $type=RW$ \RU{или если}\EN{or if} $type=W$.
\RU{Там нет ``break'' для ``case RW'', и это нормально}\EN{There are no ``break'' 
for ``case RW'', that's OK}.

\subsection{MSVC x86}

\lstinputlisting[caption=MSVC 2012]{patterns/08_switch/fallthrough_MSVC.asm}

\RU{Код почти полностью повторяет то что в исходнике.}
\EN{The code is mostly resembles what is in source.}
\RU{Там нет переходов между метками}\EN{There are no jumps between labels} \TT{\$LN4@f} \AndENRU 
\TT{\$LN3@f}: \RU{так что когда управление (code flow) находится на}\EN{so when code flow are at} 
\TT{\$LN4@f}, $r$ \RU{в начале выставляется в $1$, затем}\EN{value are first set to $1$, then} $w$.
\EN{This is probably why it's called fallthrough: code flow is falling through one piece of code
(setting $r$) to another (setting $w$).}
\RU{Наверное поэтому всё это и называется ``проваливаться'': управление проваливается через
один фрагмент кода (выставляющий $r$) в другой (выставляющий $w$).}
\RU{Если}\EN{If} $type=W$, \RU{мы оказываемся на}\EN{we are landing at} \TT{\$LN3@f}, 
\RU{так что код выставляющий $r$ в $1$ не исполнится}\EN{so no code setting $r$ to $1$ 
is executing}.

\subsection{ARM64}

\lstinputlisting[caption=GCC (Linaro) 4.9]{patterns/08_switch/fallthrough_ARM64.s}

\RU{Почти то же самое}\EN{Merely the same thing}.
\RU{Здесь нет переходов между метками}\EN{There are no jumps between labels} \TT{.L4} 
\AndENRU \TT{.L3}.
