\subsection{Fall-through}

Un autre usage très répandu de l'opérateur \TT{switch()} est ce qu'on appelle
un \q{fallthrough} (passer à travers).
Voici un exemple simple\footnote{Copié/collé depuis \url{https://github.com/azonalon/prgraas/blob/master/prog1lib/lecture_examples/is_whitespace.c}}:

\lstinputlisting[numbers=left,style=customc]{patterns/08_switch/4_fallthrough/fallthrough1.c}

Légèrement plus difficile, tiré du noyau Linux\footnote{Copié/collé depuis \url{https://github.com/torvalds/linux/blob/master/drivers/media/dvb-frontends/lgdt3306a.c}}:

\lstinputlisting[numbers=left,style=customc]{patterns/08_switch/4_fallthrough/fallthrough2.c}

\lstinputlisting[caption=GCC 5.4.0 x86 \Optimizing,numbers=left,style=customasmx86]{patterns/08_switch/4_fallthrough/fallthrough2.s}

Nous atteignons le label \TT{.L5} si la fonction a reçue le nombre 3250 en entrée.
Mais nous pouvons atteindre ce label d'une autre façon:
nous voyons qu'il n'y a pas de saut entre l'appel à \printf et le label \TT{.L5}.

Nous comprenons maintenant pourquoi la déclaration \IT{switch()} est parfois une
source de bug:
un \IT{break} oublié va transformer notre déclaration \IT{switch()} en un \IT{fallthrough},
et plusieurs blocs seront exécutés au lieu d'un seul.

