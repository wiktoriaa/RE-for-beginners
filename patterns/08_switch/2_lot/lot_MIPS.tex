\subsection{MIPS}

\lstinputlisting[caption=\Optimizing GCC 4.4.5 (IDA)]{patterns/08_switch/2_lot/MIPS_O3_IDA.lst.\LANG}

\index{MIPS!\Instructions!SLTIU}
\RU{Новая для нас инструкция здесь это SLTIU (\q{Set on Less Than Immediate Unsigned}~--- установить,
если меньше чем значение, беззнаковое сравнение).}
\EN{The new instruction for us is SLTIU (\q{Set on Less Than Immediate Unsigned}).}
\index{MIPS!\Instructions!SLTU}
\RU{На самом деле, это то же что и SLTU (\q{Set on Less Than Unsigned}), но \q{I} означает \q{immediate},
т.е. число может быть задано в самой инструкции.}
\EN{This is the same as SLTU (\q{Set on Less Than Unsigned}), but \q{I} stands for \q{immediate}, 
i.e., a number has to be specified in the instruction itself.}

\index{MIPS!\Instructions!BNEZ}
BNEZ \RU{это}\EN{is} \q{Branch if Not Equal to Zero}\RU{ (переход если не равно нулю)}.

\RU{Код очень похож на код для других \ac{ISA}.}\EN{Code is very close to the other \ac{ISA}s.}
\index{MIPS!\Instructions!SLL}
SLL (\q{Shift Word Left Logical}\RU{~--- логический сдвиг влево}) 
\RU{совершает умножение на 4}\EN{does multiplication by 4}.
\RU{MIPS всё-таки это 32-битный процессор, так что все адреса в таблице переходов 
(\IT{jumptable}) 32-битные.}
\EN{MIPS is a 32-bit CPU after all, so all addresses in the \IT{jumptable} are 32-bit ones.}
