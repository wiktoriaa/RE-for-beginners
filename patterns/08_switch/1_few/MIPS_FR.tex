\subsubsection{MIPS}

\lstinputlisting[caption=GCC 4.4.5 \Optimizing (IDA),style=customasmMIPS]{patterns/08_switch/1_few/MIPS_O3_IDA_FR.lst}

\myindex{MIPS!\Instructions!JR}

La fonction se termine toujours en appelant \puts, donc nous voyons un saut à \puts
(\INS{JR}: \q{Jump Register}) au lieu de \q{jump and link}.
Nous avons parlé de ceci avant: \myref{JMP_instead_of_RET}.

\myindex{MIPS!Load delay slot}
Nous voyons aussi souvent l'instruction \INS{NOP} après \INS{LW}.
Ceci est le slot de délai de chargement (\q{load delay slot}): un autre slot de
délai (\IT{delay slot}) en MIPS.
\myindex{MIPS!\Instructions!LW}

Une instruction suivant \INS{LW} peut s'exécuter pendant que \INS{LW} charge une
valeur depuis la mémoire.

Toutefois, l'instruction suivante ne doit pas utiliser le résultat de \INS{LW}.

Les CPU MIPS modernes ont la capacité d'attendre si l'instruction suivante utilise
le résultat de \INS{LW}, donc ceci est un peu démodé, mais GCC ajoute toujours
des NOPs pour les anciens CPU MIPS.
En général, ça peut être ignoré.
