\subsection{ARM: \OptimizingKeilVI (\ARMMode)}
\index{\CLanguageElements!switch}

\lstinputlisting{patterns/08_switch/1_few/few_ARM_ARM_O3.asm}

\RU{Мы снова не сможем сказать, глядя на этот код, был ли в оригинальном исходном коде switch() 
либо же несколько if()-в.}
\EN{Again, by investigating this code, we cannot say, was it switch() in the original source code, 
or pack of if() statements.}

\index{ARM!\Instructions!ADRcc}
\RU{Так или иначе, мы снова видим здесь инструкции с предикатами, например, \ADREQ (\IT{(Equal)}), 
которая будет исполняться только
если $R0=0$, и тогда, в \Reg{0} будет загружен адрес строки \IT{<<zero\textbackslash{}n>>}.}
\EN{Anyway, we see here predicated instructions again (like \ADREQ (\IT{Equal}))
which will be triggered only in $R0=0$ case, and the, address of the \IT{<<zero\textbackslash{}n>>}
string will be loaded into the \Reg{0}.}
\index{ARM!\Instructions!BEQ}
\RU{Следующая инструкция}\EN{The next instruction} \ac{BEQ}
\RU{перенаправит исполнение на}\EN{will redirect control flow to} \TT{loc\_170}, \RU{если}\EN{if} $R0=0$.
\RU{Кстати, наблюдательный читатель может спросить, сработает ли \ac{BEQ} нормально,
ведь \ADREQ перед ним уже заполнила регистр \Reg{0} чем-то другим.}
\EN{By the way, astute reader may ask, will \ac{BEQ} triggered right since \ADREQ before it
is already filled the \Reg{0} register with another value.}
\RU{Сработает, потому что \ac{BEQ} проверяет флаги, установленные инструкцией \CMP, 
а \ADREQ флаги никак не модифицирует.}
\EN{Yes, it will since \ac{BEQ} checking flags set by \CMP instruction, and \ADREQ not modifying flags
at all.}

\RU{Кстати, в ARM имеется также для некоторых инструкций суффикс \IT{-S}, указывающий, 
что эта инструкция будет модифицировать флаги, а при отсутствии суффикса ~--- не будет.}
\EN{By the way, there is \IT{-S} suffix for some instructions in ARM,
indicating the instruction will set the flags according to the result, and without 
it~---the flags will not be touched.}
\index{ARM!\Instructions!ADD}
\index{ARM!\Instructions!ADDS}
\index{ARM!\Instructions!CMP}
\RU{Например, инструкция}\EN{For example} \TT{ADD} \RU{в отличие от}\EN{unlike} \TT{ADDS}
\RU{сложит два числа, но флаги не изменит}
\EN{will add two numbers, but flags will not be touched}.
\RU{Такие инструкции удобно использовать
между \CMP где выставляются флаги и, например, инструкциями перехода, где флаги используются.}
\EN{Such instructions are convenient to use between \CMP where flags are set and, 
e.g. conditional jumps, where flags are used.}

\RU{Далее всё просто и знакомо.}\EN{Other instructions are already familiar to us.} 
\RU{Вызов}\EN{There is only one call to} \printf \RU{один, и в самом конце, 
мы уже рассматривали подобный трюк здесь}\EN{, at the end, and we already examined this trick here}
~(\ref{ARM_B_to_printf}).
\RU{К}\EN{There are three paths to} \printf{}\RU{-у в конце ведут три пути}\EN{at the end}.

\RU{Обратите внимание на то что происходит если $a=2$ и если $a$ не попадает под сравниваемые константы.}
\EN{Also pay attention to what is going on if $a=2$ and if $a$ is not in range of constants it is comparing against.}
\index{ARM!\Instructions!ADRcc}
\index{ARM!\Instructions!CMP}
\RU{Инструкция }\TT{``CMP R0, \#2''} \RU{нужна чтобы узнать $a=2$ или нет}\EN{instruction is needed here
to know, if $a=2$ or not}.
\RU{Если это не так, то при помощи \ADRNE (\IT{Not Equal}) в \Reg{0} будет загружен указатель на 
строку \IT{<<something unknown \textbackslash{}n>>}, ведь $a$ уже было проверено на $0$ и $1$ до этого, 
и здесь $a$ точно не попадает под эти константы.}
\EN{If it is not true, then \ADRNE will load pointer to the string \IT{<<something unknown \textbackslash{}n>>} 
into \Reg{0} since $a$ was already
checked before to be equal to $0$ or $1$,
so we can be assured the $a$ variable is not equal to these numbers
at this point.}
\RU{Ну а если}\EN{And if} $R0=2$, \RU{в \Reg{0} будет загружен указатель на строку}\EN{a pointer to string} 
\IT{<<two\textbackslash{}n>>} 
\RU{при помощи инструкции \ADREQ}\EN{will be loaded by \ADREQ into \Reg{0}}.

\subsection{ARM: \OptimizingKeilVI (\ThumbMode)}

\lstinputlisting{patterns/08_switch/1_few/few_ARM_thumb_O3.asm}

\RU{Как я уже писал, в thumb-режиме нет возможности \IT{присоединять} предикаты к большинству инструкций,
так что thumb-код вышел похожим на код x86, вполне понятный.}
\EN{As I already mentioned, there is no feature of \IT{connecting} predicates to majority of instructions in thumb
mode, so the thumb-code here is somewhat similar to the easily understandable x86 \ac{CISC}-code.}

\subsection{ARM64: \NonOptimizing GCC (Linaro) 4.9}

\lstinputlisting{patterns/08_switch/1_few/ARM64_GCC_O0.lst}

\RU{Входное значение имеет тип \Tint поэтому для него используется регистр \RegW{0},
а не целая часть регистра \RegX{0}.}
\EN{Input value has \Tint type, hence \RegW{0} register is used as input value instead of the whole
\RegX{0} register.}
\RU{Указатели на строки передаются в \puts, как я и показывал в примере}\EN{String pointers are passed to 
\puts just like I showed in} ``\HelloWorldSectionName''\EN{ example}: \ref{pointers_ADRP_and_ADD}.

\subsection{ARM64: \Optimizing GCC (Linaro) 4.9}

\lstinputlisting{patterns/08_switch/1_few/ARM64_GCC_O3.lst}

\RU{Фрагмент кода более оптимизированный}\EN{Better optimized piece of code}.
\RU{Инструкция }\TT{CBZ} (\IT{Compare and Branch on Zero}\RU{ (сравнить и перейти если ноль)}) 
\RU{совершает переход если}\EN{instruction do jump if} \RegW{0} \RU{ноль}\EN{is zero}.
\RU{Здесь также прямой переход на}\EN{There is also direct jump to} \puts \RU{вместо вызова}\EN{instead 
of calling it}: \ref{JMP_instead_of_RET}.
