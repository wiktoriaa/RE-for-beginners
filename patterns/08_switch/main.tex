\section{\SwitchCaseDefaultSectionName}
\myindex{\CLanguageElements!switch}

% sections
\section{\RU{Если вариантов мало}\EN{Few number of cases}}

\lstinputlisting{patterns/08_switch/1_few/few.c}

\subsection{x86}

\subsubsection{\NonOptimizing MSVC}

\RU{Это дает в итоге}\EN{Result} (MSVC 2010):

\lstinputlisting[caption=MSVC 2010]{patterns/08_switch/1_few/few_msvc.asm}

\RU{Наша функция со switch()-ем, с небольшим количеством вариантов, 
это практически аналог подобной конструкции:}
\EN{Our function with a few cases in switch(), in fact, is analogous to this construction:}

\lstinputlisting[label=switch_few_ifelse]{patterns/08_switch/1_few/few_analogue.c}

\index{\CLanguageElements!switch}
\index{\CLanguageElements!if}
\RU{Когда вариантов немного, и мы видим подобный код, невозможно сказать с уверенностью, был ли
в оригинальном исходном коде switch(), либо просто набор if()-ов.}
\EN{If we work with switch() with a few cases, it is impossible to be sure, was it
real switch() in source code, or just pack of if() statements.}
\index{\SyntacticSugar}
\RU{То есть, switch() это синтаксический сахар для большого количества вложенных проверок 
при помощи if().}
\EN{This means, switch() is like syntactic sugar for large number of nested checks constructed using if().}

\RU{В самом выходном коде, в принципе, ничего особо нового для нас здесь, 
за исключением того, что компилятор зачем-то 
перекладывает входящую переменную ($a$) во временную в локальном стеке \TT{v64}
\footnote{Локальные переменные в стеке с префиксом \TT{tv} --- 
так MSVC называет внутренние переменные для своих нужд}.}
\EN{Nothing especially new to us in generated code,
with the exception the compiler moving 
input variable 
$a$ to temporary local variable \TT{tv64}
\footnote{Local variables in stack prefixed with \TT{tv} --- 
that's how MSVC names internal variables for its needs}.}

\RU{Если скомпилировать это при помощи GCC 4.4.1, то будет почти то же самое, даже с максимальной оптимизацией 
(ключ \Othree).}
\EN{If to compile the same in GCC 4.4.1, we'll get almost the same, even with maximal optimization 
turned on (\Othree option).}

\subsubsection{\Optimizing MSVC}

\RU{Попробуем, включить оптимизацию кодегенератора}
\EN{Now let's turn on optimization in} MSVC (\Ox): \TT{cl 1.c /Fa1.asm /Ox}

\label{JMP_instead_of_RET}
\lstinputlisting[caption=MSVC]{patterns/08_switch/1_few/few_msvc_Ox.asm}

\RU{Вот здесь уже все немного по-другому, причем не без грязных хаков.}
\EN{Here we can see some dirty hacks.}

\index{x86!\Instructions!JZ}
\index{x86!\Instructions!JE}
\index{x86!\Instructions!SUB}
\RU{Первое: \TT{а} помещается в \EAX и от него отнимается 0. Звучит абсурдно, но нужно это для того, чтобы проверить, 
0 ли в \EAX был до этого? Если да, то выставится флаг \ZF (что означает что результат отнимания $0$ от числа 
стал $0$) и первый условный переход \JE (\IT{Jump if Equal} или его синоним \JZ ~--- \IT{Jump if Zero}) 
сработает на метку \TT{\$LN4@f}, где выводится сообщение \TT{'zero'}.
Если первый переход не сработал, от значения отнимается по единице, 
и если на какой-то стадии образуется в результате $0$, то сработает соответствующий переход.}
\EN{First: the value of the $a$ variable is placed into \EAX and $0$ subtracted from it. Sounds absurd, but it may needs to check if 
$0$ was in the \EAX register before? If yes, flag \ZF will be set (this also means that subtracting from $0$ is $0$) 
and first conditional jump \JE (\IT{Jump if Equal} or synonym \JZ~---\IT{Jump if Zero}) will be triggered 
and control flow passed to the \TT{\$LN4@f} label, where \TT{'zero'} message is being printed. 
If first jump was not triggered, $1$ subtracted from the input value and if at some stage $0$ will be resulted, 
corresponding jump will be triggered.}

\RU{И в конце концов, если ни один из условных переходов не сработал, управление передается \printf
со строковым аргументом \TT{'something unknown'}.}
\EN{And if no jump triggered at all, control flow passed to the \printf with argument \TT{'something unknown'} string.}

\label{jump_to_last_printf}
\index{\Stack}
\RU{Второе: мы видим две, мягко говоря, необычные вещи: указатель на сообщение помещается в переменную $a$, 
и затем \printf вызывается не через \CALL, а через \JMP. Объяснение этому простое. 
Вызывающая функция заталкивает в стек некоторое значение и через \CALL вызывает нашу функцию. 
\CALL в свою очередь заталкивает в стек адрес возврата (\ac{RA}) и делает безусловный переход на адрес нашей функции. 
Наша функция в самом начале (да и в любом её месте, потому что в теле функции нет ни одной инструкции, 
которая меняет что-то в стеке или в \ESP) имеет следующую разметку стека:}
\EN{Second: we see unusual thing for us: string pointer is placed into the $a$ variable, and 
then \printf is called not via \CALL, but via \JMP. This could be explained simply. 
\Gls{caller} pushing to stack a value and calling our function via \CALL. 
\CALL itself pushing returning address (\ac{RA}) to stack and do unconditional jump to our function address. 
Our function at any point of execution (since it do not contain any instruction moving stack 
pointer) has the following stack layout:}

\begin{itemize}
\item\ESP\EMDASH\RU{хранится}\EN{pointing to} \ac{RA}
\item\TT{ESP+4}\EMDASH\RU{хранится значение $a$}\EN{pointing to the $a$ variable} 
\end{itemize}

\RU{С другой стороны, чтобы вызвать \printf нам нужна почти такая же разметка стека, 
только в первом аргументе нужен указатель на строку. Что, собственно, этот код и делает.}
\EN{On the other side, when we need to call \printf here, we need exactly the same stack 
layout, except of first \printf argument pointing to string. 
And that is what our code does.}

\RU{Он заменяет свой первый аргумент на адрес строки, и затем передает управление \printf, как если бы вызвали не 
нашу функцию \ttf, а сразу \printf. 
\printf выводит некую строку на \gls{stdout}, затем исполняет инструкцию \RET, 
которая из стека достает \ac{RA} и управление передается в ту функцию, 
которая вызывала \ttf, минуя при этом конец ф-ции \ttf.}
\EN{It replaces function's first argument to address of the string and 
jumping to the \printf, as if not our function \ttf was called firstly, but immediately \printf.
\printf printing a string to \gls{stdout} and then execute \RET instruction, which POPping 
\ac{RA} from stack and control flow is returned not to \ttf but rather to the \ttf's \gls{callee}, 
bypassing end of \ttf function.}

\index{\CStandardLibrary!longjmp()}
\newcommand{\URLSJ}{\url{http://go.yurichev.com/17121}}
\RU{Все это возможно потому что \printf вызывается в \ttf в самом конце. 
Все это чем-то даже похоже на \TT{longjmp()}\footnote{\URLSJ}.
И все это, разумеется, сделано для экономии времени исполнения.}
\EN{All this is possible since \printf is called right at the end of the \ttf function in any case. 
In some way, it is all similar to the \TT{longjmp()}\footnote{\URLSJ} function.
And of course, it is all done for the sake of speed.}

\ifdefined\IncludeARM
\RU{Похожая ситуация с компилятором для ARM описана в секции}
\EN{Similar case with ARM compiler described in} ``\PrintfSeveralArgumentsSectionName'', 
\RU{здесь}\EN{section, here}~(\ref{ARM_B_to_printf}).
\fi

\ifdefined\IncludeOlly
\clearpage
\myparagraph{\olly}

\RU{Так как этот пример немного запутанный, попробуем оттрассировать его в}\EN{Since this example is tricky, 
let's trace it in} \olly.\\
\\
\olly \RU{может распознавать подобные switch()-конструкции, так что он добавляет полезные комментарии}\EN{can 
detect such switch() constructs, so its add some useful comments}.
\EAX \RU{в начале}\EN{is} $2$\EN{ at start}, \RU{это входное значение ф-ции}\EN{that's function's input value}: 

\begin{figure}[H]
\centering
\includegraphics[scale=\FigScale]{patterns/08_switch/1_few/few_olly1.png}
\caption{\olly: \EAX \RU{содержит первый (и единственный) аргумент ф-ции}
\EN{now contain first (and sole) function argument}}
\label{fig:switch_few_olly1}
\end{figure}

\clearpage
$0$ \RU{отнимается от}\EN{is subtracted from} $2$ \InENRU \EAX. 
\RU{Конечно же}\EN{Of course}, \EAX \RU{все еще содержит}\EN{is still contain} $2$.
\RU{Но флаг}\EN{But} \ZF \RU{теперь}\EN{flag is now} $0$, \RU{что означает что последнее вычисленное значение
не было нулевым}\EN{indicating that resulting value is non-zero}:

\begin{figure}[H]
\centering
\includegraphics[scale=\FigScale]{patterns/08_switch/1_few/few_olly2.png}
\caption{\olly: \SUB \RU{исполнилась}\EN{executed}}
\label{fig:switch_few_olly2}
\end{figure}

\clearpage
\DEC \RU{исполнилась и}\EN{is executed and} \EAX \RU{теперь содержит}\EN{now contain} $1$. 
\RU{Но}\EN{But} $1$ \RU{не ноль, так что флаг}\EN{is non-zero, so the} \ZF \RU{все еще}\EN{flag is still} $0$:

\begin{figure}[H]
\centering
\includegraphics[scale=\FigScale]{patterns/08_switch/1_few/few_olly3.png}
\caption{\olly: \RU{первая}\EN{first} \DEC \RU{исполнилась}\EN{executed}}
\label{fig:switch_few_olly3}
\end{figure}

\clearpage
\RU{Следующая}\EN{Next} \DEC \RU{исполнилась}\EN{is executed}. 
\EAX \RU{наконец}\EN{is finally} $0$ \RU{и флаг}\EN{and} \ZF \RU{выставлен, потому что результат --- ноль}\EN{flag
is set, because the result is zero}:

\begin{figure}[H]
\centering
\includegraphics[scale=\FigScale]{patterns/08_switch/1_few/few_olly4.png}
\caption{\olly: \RU{вторая}\EN{second} \DEC \RU{исполнилась}\EN{executed}}
\label{fig:switch_few_olly4}
\end{figure}

\olly \RU{показывает, что условный переход сейчас сработатет}\EN{shows that this jump will be taken now}.

\clearpage
\RU{Указатель на строку}\EN{A pointer to the string} ``two'' \RU{сейчас будет записан в стек}\EN{will now be 
written into the stack}:

\begin{figure}[H]
\centering
\includegraphics[scale=\FigScale]{patterns/08_switch/1_few/few_olly5.png}
\caption{\olly: \RU{указатель на строку сейчас запишется на место первого аргумента}
\EN{pointer to the string is to be written at the place of first argument}}
\label{fig:switch_few_olly5}
\end{figure}

\RU{Обратите внимание: текущий аргумент ф-ции это $2$ и $2$ прямо сейчас в стеке по адресу}\EN{Please note: 
current argument of the function is $2$ and $2$ is now in the stack at the address} \TT{0x0020FA44}.

\clearpage
\MOV \RU{записывает указатель на строку по адресу}\EN{wrote pointer to the string at the address} 
\TT{0x0020FA44} (\RU{см. окно стека}\EN{see stack window}).
\RU{Переход сработал}\EN{Jump is happen}.
\RU{Это самая первая инструкция ф-ции}\EN{This is the first instruction of} \printf \RU{в}\EN{function in} 
MSVCR100.DLL (\RU{я скомпилировал этот пример с опцией /MD}\EN{I compiled the example with /MD switch}): 

\begin{figure}[H]
\centering
\includegraphics[scale=\FigScale]{patterns/08_switch/1_few/few_olly6.png}
\caption{\olly: \RU{первая инструкция в}\EN{first instruction of} \printf \InENRU MSVCR100.DLL}
\label{fig:switch_few_olly6}
\end{figure}

\RU{Теперь}\EN{Now the} \printf \RU{будет считать строку на}\EN{will treat the string at} \TT{0x0020FA44} 
\RU{как свой единственный аргумент и выведет строку}\EN{as its sole argument and will print the string}.

\clearpage
\RU{Это самая последняя инструкция ф-ции}\EN{This is the very last instruction of} \printf:

\begin{figure}[H]
\centering
\includegraphics[scale=\FigScale]{patterns/08_switch/1_few/few_olly7.png}
\caption{\olly: \RU{последняя инструкция в}\EN{last instruction of} \printf \InENRU MSVCR100.DLL}
\label{fig:switch_few_olly7}
\end{figure}

\RU{Строка }``two'' \RU{была только что выведена в консоли}\EN{string was just printed to the console window}.

\clearpage
\RU{Нажмем}\EN{Let's press} F7 \OrENRU F8 (\stepover) \RU{и вернемся}\EN{and we will return}\dots
\RU{нет, не в ф-цию}\EN{not to} \ttf \RU{но в}\EN{function, but rather to the} \main:

\begin{figure}[H]
\centering
\includegraphics[scale=\FigScale]{patterns/08_switch/1_few/few_olly8.png}
\caption{\olly: \RU{возврат в}\EN{return to} \main}
\label{fig:switch_few_olly8}
\end{figure}

\RU{Да, это прямой переход из внутренностей}\EN{Yes, the jump was direct, from the guts of} \printf 
\RU{в}\EN{to} \main.
\RU{Потому как}\EN{Because} \ac{RA} \RU{в стеке указывает не на какое-то место в ф-ции}\EN{in the stack pointed 
not to some place in} \ttf \RU{а в}\EN{function, but rather to} \main.
\RU{И}\EN{And} \CALL \TT{0x01201000} \RU{это инструкция вызывающая ф-цию}\EN{was the actual instruction which called} 
\ttf\EN{ function}.

\fi

\subsection{ARM: \OptimizingKeilVI (\ARMMode)}
\index{\CLanguageElements!switch}

\lstinputlisting{patterns/08_switch/1_few/few_ARM_ARM_O3.asm}

\RU{Мы снова не сможем сказать, глядя на этот код, был ли в оригинальном исходном коде switch() 
либо же несколько if()-в.}
\EN{Again, by investigating this code, we cannot say, was it switch() in the original source code, 
or pack of if() statements.}

\index{ARM!\Instructions!ADRcc}
\RU{Так или иначе, мы снова видим здесь инструкции с предикатами, например, \ADREQ (\IT{(Equal)}), 
которая будет исполняться только
если $R0=0$, и тогда, в \Reg{0} будет загружен адрес строки \IT{<<zero\textbackslash{}n>>}.}
\EN{Anyway, we see here predicated instructions again (like \ADREQ (\IT{Equal}))
which will be triggered only in $R0=0$ case, and the, address of the \IT{<<zero\textbackslash{}n>>}
string will be loaded into the \Reg{0}.}
\index{ARM!\Instructions!BEQ}
\RU{Следующая инструкция}\EN{The next instruction} \ac{BEQ}
\RU{перенаправит исполнение на}\EN{will redirect control flow to} \TT{loc\_170}, \RU{если}\EN{if} $R0=0$.
\RU{Кстати, наблюдательный читатель может спросить, сработает ли \ac{BEQ} нормально,
ведь \ADREQ перед ним уже заполнила регистр \Reg{0} чем-то другим.}
\EN{By the way, astute reader may ask, will \ac{BEQ} triggered right since \ADREQ before it
is already filled the \Reg{0} register with another value.}
\RU{Сработает, потому что \ac{BEQ} проверяет флаги, установленные инструкцией \CMP, 
а \ADREQ флаги никак не модифицирует.}
\EN{Yes, it will since \ac{BEQ} checking flags set by \CMP instruction, and \ADREQ not modifying flags
at all.}

\RU{Кстати, в ARM имеется также для некоторых инструкций суффикс \IT{-S}, указывающий, 
что эта инструкция будет модифицировать флаги, а при отсутствии суффикса ~--- не будет.}
\EN{By the way, there is \IT{-S} suffix for some instructions in ARM,
indicating the instruction will set the flags according to the result, and without 
it~---the flags will not be touched.}
\index{ARM!\Instructions!ADD}
\index{ARM!\Instructions!ADDS}
\index{ARM!\Instructions!CMP}
\RU{Например, инструкция}\EN{For example} \TT{ADD} \RU{в отличие от}\EN{unlike} \TT{ADDS}
\RU{сложит два числа, но флаги не изменит}
\EN{will add two numbers, but flags will not be touched}.
\RU{Такие инструкции удобно использовать
между \CMP где выставляются флаги и, например, инструкциями перехода, где флаги используются.}
\EN{Such instructions are convenient to use between \CMP where flags are set and, 
e.g. conditional jumps, where flags are used.}

\RU{Далее всё просто и знакомо.}\EN{Other instructions are already familiar to us.} 
\RU{Вызов}\EN{There is only one call to} \printf \RU{один, и в самом конце, 
мы уже рассматривали подобный трюк здесь}\EN{, at the end, and we already examined this trick here}
~(\ref{ARM_B_to_printf}).
\RU{К}\EN{There are three paths to} \printf{}\RU{-у в конце ведут три пути}\EN{at the end}.

\RU{Обратите внимание на то что происходит если $a=2$ и если $a$ не попадает под сравниваемые константы.}
\EN{Also pay attention to what is going on if $a=2$ and if $a$ is not in range of constants it is comparing against.}
\index{ARM!\Instructions!ADRcc}
\index{ARM!\Instructions!CMP}
\RU{Инструкция }\TT{``CMP R0, \#2''} \RU{нужна чтобы узнать $a=2$ или нет}\EN{instruction is needed here
to know, if $a=2$ or not}.
\RU{Если это не так, то при помощи \ADRNE (\IT{Not Equal}) в \Reg{0} будет загружен указатель на 
строку \IT{<<something unknown \textbackslash{}n>>}, ведь $a$ уже было проверено на $0$ и $1$ до этого, 
и здесь $a$ точно не попадает под эти константы.}
\EN{If it is not true, then \ADRNE will load pointer to the string \IT{<<something unknown \textbackslash{}n>>} 
into \Reg{0} since $a$ was already
checked before to be equal to $0$ or $1$,
so we can be assured the $a$ variable is not equal to these numbers
at this point.}
\RU{Ну а если}\EN{And if} $R0=2$, \RU{в \Reg{0} будет загружен указатель на строку}\EN{a pointer to string} 
\IT{<<two\textbackslash{}n>>} 
\RU{при помощи инструкции \ADREQ}\EN{will be loaded by \ADREQ into \Reg{0}}.

\subsection{ARM: \OptimizingKeilVI (\ThumbMode)}

\lstinputlisting{patterns/08_switch/1_few/few_ARM_thumb_O3.asm}

\RU{Как я уже писал, в thumb-режиме нет возможности \IT{присоединять} предикаты к большинству инструкций,
так что thumb-код вышел похожим на код x86, вполне понятный.}
\EN{As I already mentioned, there is no feature of \IT{connecting} predicates to majority of instructions in thumb
mode, so the thumb-code here is somewhat similar to the easily understandable x86 \ac{CISC}-code.}

\subsection{ARM64: \NonOptimizing GCC (Linaro) 4.9}

\lstinputlisting{patterns/08_switch/1_few/ARM64_GCC_O0.lst}

\RU{Входное значение имеет тип \Tint поэтому для него используется регистр \RegW{0},
а не целая часть регистра \RegX{0}.}
\EN{Input value has \Tint type, hence \RegW{0} register is used as input value instead of the whole
\RegX{0} register.}
\RU{Указатели на строки передаются в \puts, как я и показывал в примере}\EN{String pointers are passed to 
\puts just like I showed in} ``\HelloWorldSectionName''\EN{ example}: \ref{pointers_ADRP_and_ADD}.

\subsection{ARM64: \Optimizing GCC (Linaro) 4.9}

\lstinputlisting{patterns/08_switch/1_few/ARM64_GCC_O3.lst}

\RU{Фрагмент кода более оптимизированный}\EN{Better optimized piece of code}.
\RU{Инструкция }\TT{CBZ} (\IT{Compare and Branch on Zero}\RU{ (сравнить и перейти если ноль)}) 
\RU{совершает переход если}\EN{instruction do jump if} \RegW{0} \RU{ноль}\EN{is zero}.
\RU{Здесь также прямой переход на}\EN{There is also direct jump to} \puts \RU{вместо вызова}\EN{instead 
of calling it}: \ref{JMP_instead_of_RET}.


\EN{\subsection{A lot of cases}

If a \TT{switch()} statement contains a lot of cases, it is not very convenient for the compiler to emit too large code
with a lot \JE/\JNE instructions.

\lstinputlisting[label=switch_lot_c,style=customc]{patterns/08_switch/2_lot/lot.c}

\subsubsection{x86}

\myparagraph{\NonOptimizing MSVC}

We get (MSVC 2010):

\lstinputlisting[caption=MSVC 2010,style=customasmx86]{patterns/08_switch/2_lot/lot_msvc_EN.asm}

\myindex{jumptable}

What we see here is a set of \printf calls with various arguments. 
All they have not only addresses in the memory of the process, but also internal symbolic labels assigned 
by the compiler. 
All these labels are also mentioned in the \TT{\$LN11@f} internal table.

At the function start, if $a$ is greater than 4, control flow is passed to label 
\TT{\$LN1@f}, where \printf with argument \TT{'something unknown'} is called.

But if the value of $a$ is less or equals to 4, then it gets multiplied by 4 and added with the \TT{\$LN11@f} 
table address. That is how an address inside the table is constructed, pointing exactly to the 
element we need. For example, let's say $a$ is equal to 2. $2*4 = 8$ (all table elements 
are addresses in a 32-bit process and that is why all elements are 4 bytes wide). 
The address of the \TT{\$LN11@f} table + 8 is the table element where the \TT{\$LN4@f} label is stored.
\JMP fetches the \TT{\$LN4@f} address from the table and jumps to it.

This table is sometimes called \IT{jumptable} or \IT{branch table}\footnote{The whole method was once called 
\IT{computed GOTO} in early versions of Fortran:
\href{http://go.yurichev.com/17122}{wikipedia}.
Not quite relevant these days, but what a term!}.

Then the corresponding \printf is called with argument \TT{'two'}.\\
Literally, the \TT{jmp DWORD PTR \$LN11@f[ecx*4]} instruction implies
\IT{jump to the DWORD that is stored at address} \TT{\$LN11@f + ecx * 4}.

\TT{npad} (\myref{sec:npad}) is an assembly language macro that align the next label so that it will be stored at an address aligned on a 4 byte
(or 16 byte) boundary.
This is very suitable for the processor since it is able to fetch 32-bit values from memory through the memory bus,
cache memory, etc., in a more effective way if it is aligned.

\input{patterns/08_switch/2_lot/olly_EN}

\myparagraph{\NonOptimizing GCC}
\label{switch_lot_GCC}

Let's see what GCC 4.4.1 generates:

\lstinputlisting[caption=GCC 4.4.1,style=customasmx86]{patterns/08_switch/2_lot/lot_gcc.asm}

\myindex{x86!\Registers!JMP}

It is almost the same, with a little nuance: argument \TT{arg\_0} is multiplied by 4 by
shifting it to left by 2 bits (it is almost the same as multiplication by 4)~(\myref{SHR}).
Then the address of the label is taken from the \TT{off\_804855C} array, stored in 
\EAX, and then \TT{JMP EAX} does the actual jump.


\subsubsection{ARM: \OptimizingKeilVI (\ARMMode)}
\label{sec:SwitchARMLot}

\lstinputlisting[caption=\OptimizingKeilVI (\ARMMode),style=customasmARM]{patterns/08_switch/2_lot/lot_ARM_ARM_O3.asm}

This code makes use of the ARM mode feature in which all instructions have a fixed size of 4 bytes.

Let's keep in mind that the maximum value for $a$ is 4 and any greater value will cause
\IT{<<something unknown\textbackslash{}n>>} string to be printed.

\myindex{ARM!\Instructions!CMP}
\myindex{ARM!\Instructions!ADDCC}
The first \TT{CMP R0, \#5} instruction compares the input value of $a$ with 5.

\footnote{ADD---addition}
The next \TT{ADDCC PC, PC, R0,LSL\#2} instruction is being executed only if $R0 < 5$ (\IT{CC=Carry clear / Less than}). 
Consequently, if \TT{ADDCC} does not trigger (it is a $R0 \geq 5$ case), a jump to \IT{default\_case} label will occur.

But if $R0 < 5$ and \TT{ADDCC} triggers, the following is to be happen:

The value in \Reg{0} is multiplied by 4.
In fact, \TT{LSL\#2} at the instruction's suffix stands for \q{shift left by 2 bits}.
But as we will see later~(\myref{division_by_shifting}) in section \q{\ShiftsSectionName}, 
shift left by 2 bits is equivalent to multiplying by 4.

Then we add $R0*4$ to the current value in \ac{PC}, 
thus jumping to one of the \TT{B} (\IT{Branch}) instructions located below.

At the moment of the execution of \TT{ADDCC}, the value in \ac{PC} is 8 bytes ahead (\TT{0x180})
than the address at which the \TT{ADDCC} instruction is located (\TT{0x178}), 
or, in other words, 2 instructions ahead.

\myindex{ARM!Pipeline}

This is how the pipeline in ARM processors works: when \TT{ADDCC} is executed,
the processor at the moment
is beginning to process the instruction after the next one,
so that is why \ac{PC} points there. This has to be memorized.

If $a=0$, then is to be added to the value in \ac{PC},
and the actual value of the \ac{PC} will be written into \ac{PC} (which is 8 bytes ahead)
and a jump to the label \IT{loc\_180} will happen,
which is 8 bytes ahead of the point where the \TT{ADDCC} instruction is.

If $a=1$, then $PC+8+a*4 = PC+8+1*4 = PC+12 = 0x184$ will be written to \ac{PC},
which is the address of the \IT{loc\_184} label.

With every 1 added to $a$, the resulting \ac{PC} is increased by 4.

4 is the instruction length in ARM mode and also, the length of each \TT{B} instruction,
of which there are 5 in row.

Each of these five \TT{B} instructions passes control further, to what was programmed in the \IT{switch()}.

Pointer loading of the corresponding string occurs there, etc.

\subsubsection{ARM: \OptimizingKeilVI (\ThumbMode)}

\lstinputlisting[caption=\OptimizingKeilVI (\ThumbMode),style=customasmARM]{patterns/08_switch/2_lot/lot_ARM_thumb_O3.asm}

\myindex{ARM!\ThumbMode}
\myindex{ARM!\ThumbTwoMode}

One cannot be sure that all instructions in Thumb and Thumb-2 modes has the same size.
It can even be said that in these modes the instructions have variable lengths, just like in x86.

\myindex{jumptable}

So there is a special table added that contains information about how much cases are there (not including 
default-case), and an offset for each with a label to which control must be passed in 
the corresponding case.

\myindex{ARM!Mode switching}
\myindex{ARM!\Instructions!BX}

A special function is present here in order to deal with the table and pass control, \\
named \IT{\_\_ARM\_common\_switch8\_thumb}. 
It starts with \TT{BX PC}, whose function is to switch the processor to ARM-mode.
Then you see the function for table processing. 

It is too advanced to describe it here now, so let's omit it.
% TODO explain it...

\myindex{ARM!\Registers!Link Register}

It is interesting to note that the function uses the \ac{LR} register as a pointer to the table.

Indeed, after calling of this function, \ac{LR} contains the address after\\
\TT{BL \_\_ARM\_common\_switch8\_thumb} instruction, where the table starts.

It is also worth noting that the code is generated as a separate function in order to reuse it, 
so the compiler doesn't generate the same code for every switch() statement.

\IDA successfully perceived it as a service function and a table, and added comments to the labels like\\
\TT{jumptable 000000FA case 0}.


\subsubsection{MIPS}

\lstinputlisting[caption=\Optimizing GCC 4.4.5 (IDA),style=customasmMIPS]{patterns/08_switch/2_lot/MIPS_O3_IDA_EN.lst}

\myindex{MIPS!\Instructions!SLTIU}

The new instruction for us is \INS{SLTIU} (\q{Set on Less Than Immediate Unsigned}).
\myindex{MIPS!\Instructions!SLTU}

This is the same as \INS{SLTU} (\q{Set on Less Than Unsigned}), but \q{I} stands for \q{immediate}, 
i.e., a number has to be specified in the instruction itself.

\myindex{MIPS!\Instructions!BNEZ}
\INS{BNEZ} is \q{Branch if Not Equal to Zero}.

Code is very close to the other \ac{ISA}s.
\myindex{MIPS!\Instructions!SLL}
\INS{SLL} (\q{Shift Word Left Logical}) does multiplication by 4.

MIPS is a 32-bit CPU after all, so all addresses in the \IT{jumptable} are 32-bit ones.



\subsubsection{\Conclusion{}}

Rough skeleton of \IT{switch()}:

% TODO: ARM, MIPS skeleton
\lstinputlisting[caption=x86,style=customasmx86]{patterns/08_switch/2_lot/skel1_EN.lst}

The jump to the address in the jump table may also be implemented using this instruction: \\
\TT{JMP jump\_table[REG*4]}.
Or \TT{JMP jump\_table[REG*8]} in x64.

A \IT{jumptable} is just array of pointers, like the one described later: \myref{array_of_pointers_to_strings}.
}
\RU{\subsection{И если много}

Если ветвлений слишком много, то генерировать слишком длинный код с многочисленными \JE/\JNE 
уже не так удобно.

\lstinputlisting[label=switch_lot_c,style=customc]{patterns/08_switch/2_lot/lot.c}

\subsubsection{x86}

\myparagraph{\NonOptimizing MSVC}

Рассмотрим пример, скомпилированный в (MSVC 2010):

\lstinputlisting[caption=MSVC 2010,style=customasmx86]{patterns/08_switch/2_lot/lot_msvc_RU.asm}

\myindex{jumptable}
Здесь происходит следующее: в теле функции есть набор вызовов \printf с разными аргументами. 
Все они имеют, конечно же, адреса, а также внутренние символические метки, которые присвоил им компилятор.
Также все эти метки указываются во внутренней таблице \TT{\$LN11@f}.

В начале функции, если $a$ больше 4, то сразу происходит переход на метку \TT{\$LN1@f}, 
где вызывается \printf с аргументом \TT{'something unknown'}.

А если $a$ меньше или равно 4, то это значение умножается на 4 и прибавляется адрес таблицы 
с переходами (\TT{\$LN11@f}). 
Таким образом, получается адрес внутри таблицы, где лежит нужный адрес внутри тела функции. 
Например, возьмем $a$ равным 2. $2*4 = 8$ (ведь все элементы таблицы~--- это адреса внутри 32-битного процесса, 
таким образом, каждый элемент занимает 4 байта). 8 прибавить к \TT{\$LN11@f}~--- это будет элемент таблицы,
где лежит \TT{\$LN4@f}. \JMP вытаскивает из таблицы адрес \TT{\$LN4@f} и делает безусловный переход туда.

Эта таблица иногда называется \IT{jumptable} или
\IT{branch table}\footnote{Сам метод раньше назывался 
\IT{computed GOTO} В ранних версиях Фортрана:
\href{http://go.yurichev.com/17122}{wikipedia}.
Не очень-то и полезно в наше время, но каков термин!}.

А там вызывается \printf с аргументом \TT{'two'}. 
Дословно, инструкция \TT{jmp DWORD PTR \$LN11@f[ecx*4]} 
означает \IT{перейти по DWORD, который лежит по адресу} \TT{\$LN11@f + ecx * 4}.

\TT{npad} (\myref{sec:npad}) это макрос ассемблера, выравнивающий начало таблицы, 
чтобы она располагалась по адресу кратному 4 (или 16).
Это нужно для того, чтобы процессор мог эффективнее загружать 32-битные 
значения из памяти через шину с памятью, кэш-память, итд.

\input{patterns/08_switch/2_lot/olly_RU}

\myparagraph{\NonOptimizing GCC}
\label{switch_lot_GCC}

Посмотрим, что сгенерирует GCC 4.4.1:

\lstinputlisting[caption=GCC 4.4.1,style=customasmx86]{patterns/08_switch/2_lot/lot_gcc.asm}

\myindex{x86!\Registers!JMP}
Практически то же самое, за исключением мелкого нюанса: аргумент из \TT{arg\_0} умножается на 4 
при помощи сдвига влево на 2 бита (это почти то же самое что и умножение на 4)~(\myref{SHR}).
Затем адрес метки внутри функции берется из массива \TT{off\_804855C} и адресуется при помощи 
вычисленного индекса.


\subsubsection{ARM: \OptimizingKeilVI (\ARMMode)}
\label{sec:SwitchARMLot}

\lstinputlisting[caption=\OptimizingKeilVI (\ARMMode),style=customasmARM]{patterns/08_switch/2_lot/lot_ARM_ARM_O3.asm}

В этом коде используется та особенность режима ARM, 
что все инструкции в этом режиме имеют фиксированную длину 4 байта.

Итак, не будем забывать, что максимальное значение для $a$ это 4: всё что выше, должно вызвать
вывод строки \IT{<<something unknown\textbackslash{}n>>}.

\myindex{ARM!\Instructions!CMP}
\myindex{ARM!\Instructions!ADDCC}
Самая первая инструкция \TT{CMP R0, \#5} сравнивает входное значение в $a$ c 5.

\footnote{ADD---складывание чисел}
Следующая инструкция \TT{ADDCC PC, PC, R0,LSL\#2} сработает только в случае если $R0 < 5$ (\IT{CC=Carry clear / Less than}). 
Следовательно, если \TT{ADDCC} не сработает (это случай с $R0 \geq 5$), выполнится переход на метку 
\IT{default\_case}.

Но если $R0 < 5$ и \TT{ADDCC} сработает, то произойдет следующее.

Значение в \Reg{0} умножается на 4.
Фактически, \TT{LSL\#2} в суффиксе инструкции означает \q{сдвиг влево на 2 бита}.

Но как будет видно позже~(\myref{division_by_shifting}) в секции \q{\ShiftsSectionName}, 
сдвиг влево на 2 бита, это эквивалентно его умножению на 4.

Затем полученное $R0*4$ прибавляется к текущему значению \ac{PC}, 
совершая, таким образом, переход на одну из расположенных ниже инструкций \TT{B} (\IT{Branch}).

На момент исполнения \TT{ADDCC},
содержимое \ac{PC} на 8 байт больше (\TT{0x180}), чем адрес по которому расположена сама инструкция \TT{ADDCC} (\TT{0x178}), 
либо, говоря иным языком, на 2 инструкции больше.

\myindex{ARM!Конвейер}
Это связано с работой конвейера процессора ARM:
пока исполняется инструкция \TT{ADDCC}, процессор уже начинает обрабатывать инструкцию после следующей, 
поэтому \ac{PC} указывает туда. Этот факт нужно запомнить.

Если $a=0$, тогда к \ac{PC} ничего не будет прибавлено и 
в \ac{PC} запишется актуальный на тот момент \ac{PC} (который больше на 8) 
и произойдет переход на метку \IT{loc\_180}. 
Это на 8 байт дальше места, где находится инструкция \TT{ADDCC}.

Если $a=1$, тогда в \ac{PC} запишется 
$PC+8+a*4 = PC+8+1*4 = PC+12 = 0x184$. Это адрес метки \IT{loc\_184}.

При каждой добавленной к $a$ единице итоговый \ac{PC} увеличивается на 4.

4 это длина инструкции в режиме ARM и одновременно с этим, 
длина каждой инструкции \TT{B}, их здесь следует 5 в ряд.

Каждая из этих пяти инструкций \TT{B} передает управление дальше, где собственно и происходит то, 
что запрограммировано в операторе \IT{switch()}.
Там происходит загрузка указателя на свою строку, итд.

\subsubsection{ARM: \OptimizingKeilVI (\ThumbMode)}

\lstinputlisting[caption=\OptimizingKeilVI (\ThumbMode),style=customasmARM]{patterns/08_switch/2_lot/lot_ARM_thumb_O3.asm}

\myindex{ARM!\ThumbMode}
\myindex{ARM!\ThumbTwoMode}
В режимах Thumb и Thumb-2 уже нельзя надеяться на то, что все инструкции имеют одну длину.

Можно даже сказать, что в этих режимах инструкции переменной длины, как в x86.

\myindex{jumptable}
Так что здесь добавляется специальная таблица, содержащая информацию о том, как много вариантов здесь,
не включая варианта по умолчанию, и смещения, для каждого варианта. Каждое смещение кодирует метку, куда нужно передать
управление в соответствующем случае.

\myindex{ARM!Переключение режимов}
\myindex{ARM!\Instructions!BX}
Для того чтобы работать с таблицей и совершить переход, вызывается служебная функция

\IT{\_\_ARM\_common\_switch8\_thumb}. 
Она начинается с инструкции \TT{BX PC}, чья функция~--- переключить процессор в ARM-режим.

Далее функция, работающая с таблицей. 
Она слишком сложная для рассмотрения в данном месте, так что пропустим это.

% TODO explain it...

\myindex{ARM!\Registers!Link Register}
Но можно отметить, что эта функция использует регистр \ac{LR} как указатель на таблицу.

Действительно, после вызова этой функции, в \ac{LR} был записан адрес после инструкции

\TT{BL \_\_ARM\_common\_switch8\_thumb}, а там как раз и начинается таблица.

Ещё можно отметить, что код для этого выделен в отдельную функцию для того, 
чтобы не нужно было каждый раз генерировать 
точно такой же фрагмент кода для каждого выражения switch().

\IDA распознала эту служебную функцию и таблицу автоматически дописала комментарии к меткам вроде \\
\TT{jumptable 000000FA case 0}.


\subsubsection{MIPS}

\lstinputlisting[caption=\Optimizing GCC 4.4.5 (IDA),style=customasmMIPS]{patterns/08_switch/2_lot/MIPS_O3_IDA_RU.lst}

\myindex{MIPS!\Instructions!SLTIU}
Новая для нас инструкция здесь это \INS{SLTIU} (\q{Set on Less Than Immediate Unsigned}~--- установить,
если меньше чем значение, беззнаковое сравнение).

\myindex{MIPS!\Instructions!SLTU}
На самом деле, это то же что и \INS{SLTU} (\q{Set on Less Than Unsigned}), но \q{I} означает \q{immediate},
т.е. число может быть задано в самой инструкции.

\myindex{MIPS!\Instructions!BNEZ}
\INS{BNEZ} это \q{Branch if Not Equal to Zero} (переход если не равно нулю).

Код очень похож на код для других \ac{ISA}.
\myindex{MIPS!\Instructions!SLL}
\INS{SLL} (\q{Shift Word Left Logical}~--- логический сдвиг влево) совершает умножение на 4.
MIPS всё-таки это 32-битный процессор, так что все адреса в таблице переходов (\IT{jumptable}) 32-битные.



\subsubsection{\Conclusion{}}

Примерный скелет оператора \IT{switch()}:

% TODO: ARM, MIPS skeleton
\lstinputlisting[caption=x86,style=customasmx86]{patterns/08_switch/2_lot/skel1_RU.lst}

Переход по адресу из таблицы переходов может быть также реализован такой инструкцией: \\
\TT{JMP jump\_table[REG*4]}. Или \TT{JMP jump\_table[REG*8]} в x64.

Таблица переходов (\IT{jumptable}) это просто массив указателей, как это будет вскоре описано: \myref{array_of_pointers_to_strings}.
}
\DE{\subsection{Viele Fälle}
Wenn ein \TT{switch()} Ausdruck viele Fälle enthält, ist es für den Compiler nicht günstig sehr großen Code mit vielen
\JE/\JNE Befehlen zu erzeugen.

\lstinputlisting[label=switch_lot_c,style=customc]{patterns/08_switch/2_lot/lot.c}

\subsubsection{x86}

\myparagraph{\NonOptimizing MSVC}

Wir erhalten (MSVC 2010):

\lstinputlisting[caption=MSVC 2010,style=customasmx86]{patterns/08_switch/2_lot/lot_msvc_DE.asm}

\myindex{jumptable}
Was wir hier sehen ist eine Ansammlung von Aufrufen von \printf mit diversen Argumenten.
Alle haben nicht Adressen im Speicher des Prozesses, sondern auch interne symbolische Labels, die ihnen vom Compiler
zugewiesen werden.
Alle diese Labels werden auch in der internen Tabelle \TT{\$LN11@f} aufgeführt. 

Zu Beginn der Funktion wird der Control Flow an das Label \TT{\$LN1@f} abgegeben, wenn $a$ größer ist als 4. An diesem
Label wird \printf mit dem Argument \TT{'something unknown'} aufgerufen.

Wenn aber der Wert von $a$ kleiner gleich 4 ist, dann wird dieser mit 4 multipliziert und mit der Tabellenadresse
\TT{\$LN11@f} addiert. Auf diese Weise wird die Adresse innerhalb der Tabelle konstruiert und zeigt genau auf das
gewünschte Element. Nehmen wir zum Beispiel an, dass $a$ gleich 2 ist. $2\cdot 4=8$ (alle Tabellenelemente sind
Adressen in einem 32-Bit-Prozess und haben daher eine Breite von 4 Bytes).
Die Adresse an der Stelle \TT{\$LN11@f} + 8 ist das Tabellenelement, an dem das Label \TT{\$LN4@f} gespeichert ist.
\JMP holt die Adresse \TT{\$LN4@f} aus der Tabelle und springt dorthin.

Diese Tabelle wird manchmal \IT{Jumptable} oder \IT{Verzweigungstabelle} genannt\footnote{Die ganze Methode wurde
in früheren Versionen von Fortran \IT{berechnetes GOTO} genannt:
\href{http://go.yurichev.com/17122}{wikipedia}.
Heutzutage zwar nicht mehr relevant, aber welch ein Ausdruck!}.

Dann wird das zugehörige \printf mit dem Argument \TT{'two'} aufgerufen.\\
Der Befehl TT{jmp DWORD PTR \$LN11@f[ecx*4]} bedeutet dabei \IT{springe zum an dieser Stelle gespeicherten
DWORD}\TT{\$LN11@f + ecx * 4}.

\TT{npad} (\myref{sec:npad}) ist ein Assemblermakro, dass das nächste Label so angeordnet, dass es an einer 4 Byte
(oder 16 Bit) Adressgrenze gespeichert wird. Das ist für den Prozessor sehr praktisch, da er die 32-Bit-Werte aus dem
Speicher durch den Speicherbus, den Cache, etc. in effektiverer Weise laden kann.

\input{patterns/08_switch/2_lot/olly_DE}

\myparagraph{\NonOptimizing GCC}
\label{switch_lot_GCC}

Schauen wir was GCC 4.4.1 erzeugt:

\lstinputlisting[caption=GCC 4.4.1,style=customasmx86]{patterns/08_switch/2_lot/lot_gcc.asm}

\myindex{x86!\Registers!JMP}
Es ist bis auf eine Nuance das gleiche: das Argument \TT{arg\_0} wird mit 4 multipliziert durch eine Verschiebung von 2
Bits nach links (dies entspricht einer Multiplikation mit 4)~(\myref{SHR}).
Dann wird die Adresse des Labels vom \TT{off\_804855C} genommen, die in \EAX gespeichert wird, und dann wird mit
\TT{JMP EAX} der eigentliche Sprung durchgeführt.



\subsubsection{ARM: \OptimizingKeilVI (\ARMMode)}
\label{sec:SwitchARMLot}

\lstinputlisting[caption=\OptimizingKeilVI (\ARMMode),style=customasmARM]{patterns/08_switch/2_lot/lot_ARM_ARM_O3.asm}
Dieser Code verwendet das ARM mode Feature, das alle Befehle eine feste Länge von 4 Byte haben.

Vergessen wir nicht, dass der Maximalwert für $a$ 4 beträgt und jeder größere Wert zur Ausgabe des \IT{<<something
unknown\textbackslash{}n>>} Strings führt.

\myindex{ARM!\Instructions!CMP}
\myindex{ARM!\Instructions!ADDCC}
Der erste \TT{CMP R0, \#5} Befehl vergleich den Eingabewert $a$ mit 5.

\footnote{ADD---Addition}
Der nächste \TT{ADDCC PC, PC, R0,LSL\#2} Befehl wird nur ausgeführt, falls $R0 < 5$ (\IT{CC=Carry clear / kleiner als}).
Wenn \TT{ADDCC} nicht ausgeführt wird (d.h. $R0\geq 5$), wird ein Sprung zum \IT{default\_case} Label ausgeführt.

Aber wenn $R0 < 5$ und \TT{ADDCC} ausgeführt wird, wird das Folgende geschehen:

Der Wert in \Reg{0} wird mit 4 multipliziert.
Der Suffix \TT{LSL2} am Befehl steht dabei für \q{shift left by 2 bits}.
Aber wie wir später~(\myref{division_by_shifting}) im Abschnitt \q{\ShiftsSectionName} sehen werden, ist eine
Verschiebung um 2 Bits nach links äquivalent zu einer Multiplikation mit 4.

Danach addieren wir $R0\cdot 4$ zum aktuellen Wert in \ac{PC} und springen dadurch zu einem der unteren \TT{B}
(\IT{Branch}) Befehle.

Im Moment der Ausführung von\TT{ADDCC} ist der Wert von \ac{PC} (\TT{0x180}) 8 Bytes - oder mit anderen Worten: 2
Befehle - größer als die Adresse, an der sich der \TT{ADDCC} Befehl befindet (\TT{0x178})

\myindex{ARM!Pipeline}
So funktioniert die Pipeline in ARM Prozessoren: wenn \TT{ADDCC} ausgeführt wird, beginnt der Prozessor den Befehl
nach dem nächsten abzuarbeiten und deshalb zeigt \ac{PC} hierher. Das müssen wir im Kopf behalten.

Wenn $a=0$, dann wird dies zum Wert in \ac{PC} addiert und der aktuelle Wert des \ac{PC} wird nach \ac{PC} geschrieben
(welcher 8 Byte größer ist) und es wird zum Label \IT{loc\_180} gesprungen, welches 8 Byte größer ist als die Adresse
des \TT{ADDCC} Befehls.

Wenn $a=1$, dann wird $PC+8+a\cdot 4 = PC+8+1\cdot 4 = PC+12 = 0x184$ nach \ac{PC} geschrieben,was der Adresse des
\IT{loc\_184} Labels entspricht.

Jedes Mal wenn $a$ um 1 erhöht wird, erhöht sich der \ac{PC} um 4.

Dabei ist 4 die Länge eines Befehls im ARM mode und auch die Länge jedes \TT{B} Befehls, von denen sich hier 5 befinden.

Jeder dieser fünf \TT{B} Befehle gibt den Control Flow weiter so wie es im \IT{switch()} Ausdruck programmiert wurde.

Hier werden jeweils die Pointer auf die zugehörigen Strings geladen, etc.

\subsubsection{ARM: \OptimizingKeilVI (\ThumbMode)}

\lstinputlisting[caption=\OptimizingKeilVI (\ThumbMode),style=customasmARM]{patterns/08_switch/2_lot/lot_ARM_thumb_O3.asm}

\myindex{ARM!\ThumbMode}
\myindex{ARM!\ThumbTwoMode}
Man kann sich nicht sicher sein, dass alle Befehle im Thumb und Thumb-2 mode dieselbe Größe haben.
Man kann sogar sagen, dass die Befehle hier genau wie in x86 variable Längen haben.

\myindex{jumptable}
Deshalb wird hier eine spezielle Tabelle verwendet, die Informationen darüber enthält wie viele Fälle vorliegen (ohne
den Default-Case) und es wird für jeden Fall ein Label mit einem Offset für den Control Flow im zugehörigen Fall
angegeben.


\myindex{ARM!Mode switching}
\myindex{ARM!\Instructions!BX}
Hier taucht eine spezielle Funktion namens \IT{\_\_ARM\_common\_switch8\_thumb} auf, die mit der Tabelle und der
Übergabe des Control Flows umgeht.
Sie beginnt mit \TT{BX PC}, dessen Aufgabe es ist, den Prozessor in den ARM mode zu versetzen.
Danach finden wir die Funktion für den Umgang mit der Tabelle.

Es ist hier zu fortgeschritten um weiter ins Details zu gehen, daher lassen wir es für den Moment hierbei bewenden. 

% TODO explain it...

\myindex{ARM!\Registers!Link Register}
Ist ist interessant festzustellen, dass die Funktion das \ac{LR} Register als Pointer auf die Tabelle verwendet.

Tatsächlich enthält \ac{LR} nach dem Aufruf der Funktion die Adresse nach dem Befehl\\
\TT{BL \_\_ARM\_common\_switch8\_thumb}, an dem die Tabelle beginnt.

Es ist auch bemerkenswert, dass der Code als eine separate Funktion erzeugt wird, um wiederverwendet werden zu können,
sodass der Compiler nicht für jeden switch() Ausdruck den gleichen Code erzeugen muss.

\IDA hat erfolgreich ermittelt, dass es sich um eine Servicefunktion und eine Tabelle handelt und hat Kommentare wie
etwa \TT{jumptable 000000FA case 0} zu den Labels hinzugefügt.



\subsubsection{MIPS}

\lstinputlisting[caption=\Optimizing GCC 4.4.5 (IDA),style=customasmMIPS]{patterns/08_switch/2_lot/MIPS_O3_IDA_DE.lst}

\myindex{MIPS!\Instructions!SLTIU}
Der für uns neue Befehl ist \INS{SLTIU} (\q{Set on Less Than Immediate Unsigned}).

\myindex{MIPS!\Instructions!SLTU}
Dies ist das gleiche wie \INS{SLTU} (\q{Set on Less Than Unsigned}); das \q{I} steht dabei für \q{immediate}, d.h.
für den Befehl muss eine Zahl angegeben werden. 

\myindex{MIPS!\Instructions!BNEZ}
\INS{BNEZ} ist \q{Branch if Not Equal to Zero}.

Der Code ist den anderen \ac{ISA}s sehr ähnlich.
\myindex{MIPS!\Instructions!SLL}
\INS{SLL} (\q{Shift Word Left Logical}) führt eine Multiplikation mit 4 durch.
Da MIPS eine 32-Bit CPU ist, sind auch die Adressen in der \IT{Jumptable} 32 Bit groß.

\subsubsection{\Conclusion{}}

Das grobe Gerüst eines \IT{switch()}:

% TODO: ARM, MIPS skeleton
\lstinputlisting[caption=x86,style=customasmx86]{patterns/08_switch/2_lot/skel1_DE.lst}
Der Sprung zur Adresse in der Jumptable kann auch durch den folgenden Befehl realisiert werden:\\
\TT{JMP jump\_table[REG*4]}
oder \TT{JMP jump\_table[REG*8]} in x64.

Eine \IT{Jumptable} ist nur ein Array von Pointern, genau wie das hier beschriebene:
\myref{array_of_pointers_to_strings}.
}
\FR{\subsection{De nombreux cas}

Si une déclaration \TT{switch()} contient beaucoup de cas, il n'est pas très pratique
pour le compilateur de générer un trop gros code avec de nombreuses instructions
\JE/\JNE.

\lstinputlisting[label=switch_lot_c,style=customc]{patterns/08_switch/2_lot/lot.c}

\subsubsection{x86}

\myparagraph{MSVC \NonOptimizing}

Nous obtenons (MSVC 2010):

\lstinputlisting[caption=MSVC 2010,style=customasmx86]{patterns/08_switch/2_lot/lot_msvc_FR.asm}

\myindex{jumptable}

Ce que nous voyons ici est un ensemble d'appels à \printf avec des arguments variés.
Ils ont tous, non seulement des adresses dans la mémoire du procesus, mais aussi
des labels symboliques internes assignés par le compilateur.
Tous ces labels ont aussi mentionnés dans la table interne \TT{\$LN11@f}.

Au début de la fonctions, si $a$ est supérieur à 4, l'exécution est passée au
labal \TT{\$LN1@f}, oú \printf est appelé avec l'argument \TT{'something unknown'}.

Mais si la valeur de $a$ est inférieure ou égale à 4, elle est alors multipliée
par 4 et ajoutée à l'adresse de la table \TT{\$LN11@f}. C'est ainsi qu'une adresse
à l'intérieur de la table est construite, pointant exactement sur l'élément dont
nous avons besoin. Par exemple, supposons que $a$ soit égale à 2. $2*4 = 8$ (tous
les éléments de la table sont adressés dans un processus 32-bit et c'est pourquoi
les éléments ont une taille de 4 octets).
L'adresse de la table \TT{\$LN11@f} + 8 est celle de l'élément de la table oú
le label \TT{\$LN4@f} est stocké.
\JMP prend l'adresse de \TT{\$LN4@f} dans la table et y saute.

Cette table est quelquefois appelée \IT{jumptable} (table de saut) ou \IT{branch table}
(table de branchement)\footnote{L'ensemble de la méthode était appelé \IT{computed
GOTO} (GOTO calculés) dans les premières versions de ForTran:
\href{http://go.yurichev.com/17122}{wikipedia}.
Pas très pertinent de nos jours, mais quel terme!}.

Le \printf correspondant est appelé avec l'argument \TT{'two'}.\\
Littéralement, l'instruction \TT{jmp DWORD PTR \$LN11@f[ecx*4]} signifie
\IT{sauter au DWORD qui est stocké à l'adresse} \TT{\$LN11@f + ecx * 4}.

\TT{npad} (\myref{sec:npad}) est une macro du langage d'assemblage qui aligne le
label suivant de telle sorte qu'il soit stocké à une adresse alignée sur une limite
de 4 octets (ou 16 octets).
C'est très adapté pour le processeur puisqu'il est capable d'aller chercher des
valeurs 32-bit dans la mémoire à travers le bus mémoire, la mémoire cache, etc.,
de façons beaucoup plus éfficace si c'est aligné.

\input{patterns/08_switch/2_lot/olly_FR}

\myparagraph{GCC \NonOptimizing}
\label{switch_lot_GCC}

Voyons ce que GCC 4.4.1 génère:

\lstinputlisting[caption=GCC 4.4.1,style=customasmx86]{patterns/08_switch/2_lot/lot_gcc.asm}

\myindex{x86!\Registers!JMP}

C'est presque la même chose, avec une petite nuance: l'argument \TT{arg\_0} est multiplié
par 4 en décalant de 2 bits vers la gauche (c'est preque comme multiplier par 4)~(\myref{SHR}).
Ensuite l'adresse du label est prise depuis le tableau \TT{off\_804855C}, stockée
dans \EAX, et ensuite \TT{JMP EAX} effectue le saut réel.


\subsubsection{ARM: \OptimizingKeilVI (\ARMMode)}
\label{sec:SwitchARMLot}

\lstinputlisting[caption=\OptimizingKeilVI (\ARMMode),style=customasmARM]{patterns/08_switch/2_lot/lot_ARM_ARM_O3.asm}

Ce code utilise les caractéristiques du mode ARM dans lequel toutes les instructions
ont une taille fixe de 4 octets.

Gardons à l'esprit que la valeur maximale de $a$ est 4 et que toute autre valeur
supérieure provoquera l'affichage de la chaîne \IT{<<something unknown\textbackslash{}n>>}

\myindex{ARM!\Instructions!CMP}
\myindex{ARM!\Instructions!ADDCC}
La première instruction \TT{CMP R0, \#5} compare la valeur entrée dans $a$ avec 5.

\footnote{ADD---addition}
L'instruction suivante, \TT{ADDCC PC, PC, R0,LSL\#2}, est exécutée si et seulement
si $R0 < 5$ (\IT{CC=Carry clear / Less than} retenue vide, inférieur à).
Par conséquent, si \TT{ADDCC} n'est pas exécutée (c'est le cas $R0 \geq 5$), un
saut au label \IT{default\_case} se produit.

Mais si $R0 < 5$ et que \TT{ADDCC} est exécuté, voici ce qui se produit:

La valeur dans \Reg{0} est multipliée par 4.
En fait, le suffixe de l'instruction \TT{LSL\#2} signifie \q{décalage à gauche de 2 bits}.
Mais comme nous le verrons plus tard~(\myref{division_by_shifting}) dans la section
\q{\ShiftsSectionName}, décaler de 2 bits vers la gauche est équivalent à multiplier
par 4.

Puis, nous ajoutons $R0*4$ à la valeur courante du \ac{PC}, et sautons à l'une
des instructions \TT{B} (\IT{Branch}) situées plus bas.

Au moment de l'exécution de \TT{ADDCC}, la valeur du \ac{PC} est en avance de 8
octets (\TT{0x180}) sur l'adresse à laquelle l'instruction \TT{ADDCC} se trouve
(\TT{0x178}), ou, autrement dit, en avance de 2 instructions.

\myindex{ARM!Pipeline}

C'est ainsi que le pipeline des processeurs ARM fonctionne: lorsque \TT{ADDCC} est
exécutée, le processeur, à ce moment, commence à préparer les instructions après
la suivante, c'est pourquoi \ac{PC} pointe ici. Cela doit être mémorisé.

Si $a=0$, elle sera ajoutée à la valeur de \ac{PC}, et la valeur courante de \ac{PC}
sera écrite dans \ac{PC} (qui est 8 octets en avant) et un saut au label \IT{loc\_180}
sera effectué, qui est 8 octets en avant du point où l'instruction se trouve.

Si $a=1$, alors $PC+8+a*4 = PC+8+1*4 = PC+12 = 0x184$ sera écrit dans \ac{PC}, qui
est l'adresse du label \IT{loc\_184}.

A chaque fois que l'on ajoute 1 à $a$, le \ac{PC} résultant est incrémenté de
4.

4 est la taille des instructions en mode ARM, et donc, la longueur de chaque instruction
\TT{B} desquelles il y a 5 à la suite.

Chacune de ces cinq instructions \TT{B} passe le contrôle plus loin, à ce qui a
été programmé dans le \IT{switch()}.

Le chargement du pointeur sur la chaîne correspondante se produit ici, etc.

\subsubsection{ARM: \OptimizingKeilVI (\ThumbMode)}

\lstinputlisting[caption=\OptimizingKeilVI (\ThumbMode),style=customasmARM]{patterns/08_switch/2_lot/lot_ARM_thumb_O3.asm}

\myindex{ARM!\ThumbMode}
\myindex{ARM!\ThumbTwoMode}

On ne peut pas être sûr que toutes ces instructions en mode Thumb et Thumb-2 ont
la même taille. On peut même dire que les intructions dans ces modes ont une longueur
variable, tout comme en x86.

\myindex{jumptable}

Donc, une table spéciale est ajoutée, qui contient des informations sur le nombre
de cas (sans inclure celui par défaut), et un offset pour chaque label auquel le
contrôle doit être passé dans chaque cas.

\myindex{ARM!Mode switching}
\myindex{ARM!\Instructions!BX}

Une fonction spéciale est présente ici qui s'occupe de la table et du passage du
contrôle, appelée \IT{\_\_ARM\_common\_switch8\_thumb}.
Elle commence avec \TT{BX PC}, dont la fonction est de passer le mode du processeur
en ARM.
Ensuite, vous voyez la fonction pour le traitement de la table.

C'est trop avancé pour être détaillé ici, donc passons cela.
% TODO explain it...

\myindex{ARM!\Registers!Link Register}

Il est intéressant de noter que la fonction utilise le regsitre \ac{LR} comme un
pointeur sur la table.

En effet, après l'appel de cette fonction, \ac{LR} contient l'adresse après\\
l'instruciton \TT{BL \_\_ARM\_common\_switch8\_thumb}, oú la table commence.

Il est intéressant de noter que le code est généré comme une fonction indépendante
afin de la ré-utiliser, donc le compilateur ne génèrera pas le même code pour chaque
déclaration switch().

\IDA l'a correctement identifié comme une fonction de service et une table, et a
ajouté un commentaire au label comme\\
\TT{jumptable 000000FA case 0}.


\subsubsection{MIPS}

\lstinputlisting[caption=GCC 4.4.5 \Optimizing (IDA),style=customasmMIPS]{patterns/08_switch/2_lot/MIPS_O3_IDA_FR.lst}

\myindex{MIPS!\Instructions!SLTIU}

La nouvelle instruction pour nous est \INS{SLTIU} (\q{Set on Less Than Immediate Unsigned}
Mettre si inférieur à la valeur immédiate non signée).
\myindex{MIPS!\Instructions!SLTU}

Ceci est la même que \INS{SLTU} (\q{Set on Less Than Unsigned}), mais \q{I} signifie
\q{immediate}, i.e., un nombre doit être spécifié dans l'instruction elle-même.

\myindex{MIPS!\Instructions!BNEZ}
\INS{BNEZ} est \q{Branch if Not Equal to Zero}.

Le code est très proche de l'autre \ac{ISA}s.
\myindex{MIPS!\Instructions!SLL}
\INS{SLL} (\q{Shift Word Left Logical}) effectue une multiplication par 4.

MIPS est un CPU 32-bit après tout, donc toutes les adresses de la \IT{jumtable}
sont 32-bits.



\subsubsection{\Conclusion{}}

Squelette grossier d'un \IT{switch()}:

% TODO: ARM, MIPS skeleton
\lstinputlisting[caption=x86,style=customasmx86]{patterns/08_switch/2_lot/skel1_FR.lst}

Le saut a une adresse de la table de saut peut aussi être implémenté en utilisant
cette instruction: \\
\TT{JMP jump\_table[REG*4]}.
Ou \TT{JMP jump\_table[REG*8]} en x64.

Une table de saut est juste un tableau de pointeurs, comme celle décrite plus
loin: \myref{array_of_pointers_to_strings}. 
}


% TODO What's the difference between 3 and 4? Seems to be the same...
% it is fallthrough from 3 to 4 :) --DY
\section{\RU{Когда много \IT{case} в одном блоке}
\EN{When there are several \IT{case} in one block}}

\RU{Вот очень часто используемая конструкция: несколько \IT{case} может быть использовано в одном блоке:}
\EN{Here is also a very often used construction: several \IT{case} statements may be used in single block:}

\lstinputlisting{patterns/08_switch/3_several_cases/several_cases.c}

\RU{Слишком расточительно генерировать каждый блок для каждого случая, поэтому обычно
каждый блок генерируется плюс диспетчер.}
\EN{It's too wasteful to generate each block for each possible case,
so what is usually done, is each block generated plus some kind of dispatcher.}

\subsection{MSVC}

\lstinputlisting[caption=\Optimizing MSVC 2010,numbers=left]{patterns/08_switch/3_several_cases/several_cases_MSVC_2010_Ox.asm}

\RU{Здесь видим две таблицы}\EN{We see two tables here}: 
\RU{первая таблица}\EN{the first table} (\TT{\$LN10@f}) \RU{это таблица индексов}\EN{is index table},
\RU{и вторая таблица}\EN{and the second table} (\TT{\$LN11@f}) \RU{это массив указателей на блоки}\EN{is 
an array of pointers to blocks}.

\RU{В начале, входное значение используется как индекс в таблице индексов}\EN{First, input value 
is used as index in index table} (\LineENRU 13). 

\RU{Вот краткое описание значений в таблице}\EN{Here is short legend for values in the table}: 
0 \RU{это первый блок \IT{case}}\EN{is first \IT{case} block} (\RU{для значений}\EN{for values} 1, 2, 7, 10),
1 \RU{это второй}\EN{is second} (\RU{для значений}\EN{for values} 3, 4, 5),
2 \RU{это третий}\EN{is third} (\RU{для значений}\EN{for values} 8, 9, 21),
3 \RU{это четвертый}\EN{is fourth} (\RU{для значений}\EN{for value} 22),
4 \RU{это для default-блока}\EN{is for default block}.

\RU{Мы получаем индекс для второй таблицы указателей на блоки и переходим туда}\EN{We get there index for 
the second table of block pointers and we we jump there} (\LineENRU 14).

\EN{What is also worth to note that there are no case for input value $0$.}
\RU{Что еще нужно отметить, так это то что здесь нет случая для нулевого входного значения.}
\EN{Hence, we see \DEC instruction at line 10, and the table is beginning at $a=1$. 
Because there are no need to allocate table element for $a=0$.}
\RU{Поэтому мы видим инструкцию \DEC на строке 10 и таблица начинается с $a=1$.
Потому что незачем выделять в таблице элемент для $a=0$.}

\RU{Это очень часто используемый шаблон}\EN{This is very often used pattern}.

\RU{В чем же экономия}\EN{So where economy is}?
\RU{Почему нельзя сделать так, как уже обсуждалось}\EN{Why it's not possible to make it as it was 
already discussed} (\ref{switch_lot_GCC}), \RU{используя только одну таблицу, содержащую указатели на 
блоки}\EN{just with one table, consisting of block pointers}?
\RU{Причина в том что элементы в таблице индексов занимают только по 8-битному байту, поэтому всё это более 
компактно}\EN{The reason is because elements in index table has 8-bit byte type, hence it's all more compact}.

\subsection{GCC}

GCC \RU{делает так, как уже обсуждалось}\EN{do the job like it was already discussed} 
(\ref{switch_lot_GCC}), \RU{используя просто таблицу указателей}\EN{using just one table of pointers}.

\section{Fall-through}

\RU{Ещё одно популярное использование оператора}\EN{Another very popular usage of} \TT{switch()} 
\EN{is the fall-through}\RU{это т.н. \q{fallthrough} (\q{провал})}.
\RU{Вот простой пример}\EN{Here is a small example}:

\lstinputlisting[numbers=left]{patterns/08_switch/4_fallthrough/fallthrough.c}

\RU{Если}\EN{If} $type=1$ (R), $read$ \RU{будет выставлен в}\EN{is to be set to} 1, \RU{если}\EN{if} 
$type=2$ (W), $write$ \RU{будет выставлен в}\EN{is to be set to} 2.
\RU{В случае}\EN{In case of} $type=3$ (RW), \RU{обе}\EN{both} $read$ \AndENRU $write$ \RU{будут 
выставлены в}\EN{is to be set to} 1.

\RU{Фрагмент кода на строке 14 будет исполнен в двух случаях: если}\EN{The code at 
line 14 is executed in two cases: if} $type=RW$ \RU{или если}\EN{or if} $type=W$.
\RU{Там нет \q{break} для \q{case RW}, и это нормально}\EN{There is no \q{break} 
for \q{case RW}x and that's OK}.

\subsection{MSVC x86}

\lstinputlisting[caption=MSVC 2012]{patterns/08_switch/4_fallthrough/fallthrough_MSVC.asm}

\RU{Код почти полностью повторяет то, что в исходнике.}
\EN{The code mostly resembles what is in the source.}
\RU{Там нет переходов между метками}\EN{There are no jumps between labels} \TT{\$LN4@f} \AndENRU 
\TT{\$LN3@f}: \RU{так что когда управление (code flow) находится на}\EN{so when code flow is at} 
\TT{\$LN4@f}, $read$ \RU{в начале выставляется в 1, затем}\EN{is first set to 1, then} $write$.
\EN{This is why it's called fall-through: code flow falls through one piece of code
(setting $read$) to another (setting $write$).}
\RU{Наверное, поэтому всё это и называется \q{проваливаться}: управление проваливается через
один фрагмент кода (выставляющий $read$) в другой (выставляющий $write$).}
\RU{Если}\EN{If} $type=W$, \RU{мы оказываемся на}\EN{we land at} \TT{\$LN3@f}, 
\RU{так что код выставляющий $read$ в 1 не исполнится}\EN{so no code setting $read$ to 1 
is executed}.

\ifdefined\IncludeARM
\subsection{ARM64}

\lstinputlisting[caption=GCC (Linaro) 4.9]{patterns/08_switch/4_fallthrough/fallthrough_ARM64.s.\LANG}

\RU{Почти то же самое}\EN{Merely the same thing}.
\RU{Здесь нет переходов между метками}\EN{There are no jumps between labels} \TT{.L4} 
\AndENRU \TT{.L3}.
\fi


\subsection{\Exercises}

\subsubsection{\Exercise \#1}
\label{exercise_switch_1}

\RU{Вполне возможно переделать пример на Си в листинге \myref{switch_lot_c} так, чтобы при компиляции
получалось даже ещё меньше кода, но работать всё будет точно так же.
Попробуйте этого добиться.}
\EN{It's possible to rework the C example in \myref{switch_lot_c} in such way that the compiler
can produce even smaller code, but will work just the same.
Try to achieve it.}
\DE{Der C-Code des Beispiels in \myref{switch_lot_c} soll so neu geschrieben werden, dass der Compiler die gleiche
Funktionalität in noch kürzerem Code erreichen kann.}
\FR{Il est possible de modifier l'exemple en C de \myref{switch_lot_c} de telle sorte
que le compilateur produise un code plus concis, mais qui fonctionne toujours pareil.}

% \RU{Подсказка}\EN{Hint}: \printf \EN{may be called only from a single place}\RU{вполне может 
% вызываться только из одного места}.
