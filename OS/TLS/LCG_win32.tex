\subsection{Win32}

\subsubsection{Uninitialized \ac{TLS} data}

One solution is to add \TT{\_\_declspec( thread )} modifier to the global variable, now it will be allocated
in \ac{TLS} (line 9):

\lstinputlisting{OS/TLS/win32/rand_uninit.c}

Hiew shows us that there are new PE section in the executable file: \TT{.tls}.

\lstinputlisting[caption=\Optimizing MSVC 2013 x86]{OS/TLS/win32/rand_x86_uninit.asm}

\TT{rand\_state} is now in \ac{TLS} segment, and it is unique for each thread.
Here is how it's accessed: load address of \ac{TIB} from FS:2Ch, then add additional index (if needed),
then calculate address of \ac{TLS} segment.

Then it's possible to access rand\_state variable through ECX register, which points to unique area
in each thread.

\index{x86!\Registers!FS}
\TT{FS:} selector is familiar to any reverse engineer, it is specially used to walys point to \ac{TIB},
so it would be fast to load thread-specific data.

\index{x86!\Registers!GS}
\TT{GS:} selector used in Win64 and address of \ac{TLS} is 0x58:

\lstinputlisting[caption=\Optimizing MSVC 2013 x64]{OS/TLS/win32/rand_x64_uninit.asm}

\subsubsection{Initialized \ac{TLS} data}

Let's say, we want to set some fixed value to rand\_state so in case of forgetfulness of programmer,
the \TT{rand\_state} variable would be initialized to some constant anyway (line 9):

\lstinputlisting{OS/TLS/win32/rand_init.c}

The code is no differ from what we already saw, but what we see in IDA:

\lstinputlisting{OS/TLS/win32/rand_init_IDA.lst}

1234 is there and while any new thread starting, new TLS will be allocated for the new thread, and all this
data, including 1234, will be copied.

This is typical scenario:

\begin{itemize}
\item Thread A is started. TLS is created for it, 1234 is copied to rand\_state.
\item my\_rand() function called several times in thread A. rand\_state is different from 1234.
\item Thread B is started. TLS is created for it, 1234 is copied to rand\_state, 
while thread A has some other value in this variable.
\end{itemize}

\subsubsection{Understanding \ac{TLS} callbacks}
\index{TLS!Callbacks}

But what if TLS variables should be filled with some data that must be prepared in some unusual way?
Let's say, we've got the following task:
programmer may forget to call my\_srand() function to initialize \ac{PRNG}, but it should be initialized
at start with something truly random, rather than 1234.
Here is a moment when \ac{TLS} callbacks can be used.

The following code is not very portable due to the hack, but nevertheless, you've got the idea.
What we do here is define a function (tls\_callback()) which will be called \IT{before} process and/or thread 
start.
The function will initialize \ac{PRNG} with GetTickCount() value.

\lstinputlisting{OS/TLS/win32/rand_TLS_callback.c}

Let's see it in IDA:

\lstinputlisting[caption=\Optimizing MSVC 2013]{OS/TLS/win32/rand_TLS_callback.lst}

TLS callback functions are sometimes used in unpacking routines to obscure its processings.
Some people may be confused and be in the dark that some code was already executed right before \ac{OEP}.
