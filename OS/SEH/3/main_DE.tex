\subsubsection{Windows x64}

\label{SEH_win64}

Wie man sich vielleicht denken kann, ist es nicht sehr schnell bei jedem Funktionsprolog
einen SEH-Frame aufzubauen.
Ein weiteres Geschwindigkeitsproblem ist das häufige Ändern des \IT{previous try level}-Werts
während der Ausführung einer Funktion.

Also haben sich die Dinge in x64 komplett geändert: alle Zeiger auf einen \TT{try}-Block,
Filter und Handler-Funktionen sind im einem PE-Segment \TT{.pdata} gesichert.
Von hier nimmt die \ac{OS}-Ausnahmebehandlung alle Informationen.

Hier sind zwei Beispiele aus dem letzten Abschnitt, für x64 kompiliert:

\lstinputlisting[caption=MSVC 2012,style=customasmx86]{OS/SEH/3/2_x64.asm}

\lstinputlisting[caption=MSVC 2012,style=customasmx86]{OS/SEH/3/3_x64.asm}

In \IgorSkochinsky gibt es eine Reihe weiterer, detaillierte Information über dieses Thema.

Neben den Ausnahme-Informationen, beinhaltet \TT{.pdata} die Adressen von fast allen
Funktionsbeginn- und enden, da dies für Tools die für automatische Analysen nützlich
sein kann.
