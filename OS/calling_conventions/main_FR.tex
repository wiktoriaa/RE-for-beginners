\section{Méthodes de transmission d'arguments (calling conventions)}
\label{sec:callingconventions}

\subsection{cdecl}
\myindex{cdecl}
\label{cdecl}

Il s'agit de la méthode la plus courante pour passer des arguments à une fonction en langage \CCpp.

Après que le \gls{callee} ait rendu la main, le gls{caller} doit ajuster la valeur du \gls{stack pointer}
(\ESP) pour lui redonner la valeur qu'elle avait avant l'appel du \gls{callee}.

\begin{lstlisting}[caption=cdecl,style=customasmx86]
push arg3
push arg2
push arg1
call function
add esp, 12 ; returns ESP
\end{lstlisting}

\subsection{stdcall}
\label{sec:stdcall}
\myindex{stdcall}

\newcommand{\SIZEOFINT}{La taille d'une variable de type \Tint est de 4 octets sur les systèmes x86 et de 8 octets sur les systèmes x64}

Similaire à \IT{cdecl}, sauf que c'est le \gls{callee} qui doit réinitialise \ESP à sa valeur d'origine
en utilisant l'instruction \TT{RET x} et non pas \RET, \\
avec \TT{x = nb arguments * sizeof(int)\footnote{\SIZEOFINT}}.
Après le retour du \gls{callee}, le \gls{caller} ne modifie pas le \gls{stack pointer}. Il ny' a donc pas d'instruction \TT{add esp, x}.

\begin{lstlisting}[caption=stdcall,style=customasmx86]
push arg3
push arg2
push arg1
call function

function:
... do something ...
ret 12
\end{lstlisting}

Cette méthode est omniprésente dans les librairies standard win32, mais pas dans celles win64 (voir ci-dessous pour win64).

\par Prenons par exemple la fonction \myref{src:passing_arguments_ex} et modifions la légèrement en utilisant la convention \TT{\_\_stdcall} :

\begin{lstlisting}[style=customc]
int __stdcall f2 (int a, int b, int c)
{
	return a*b+c;
};
\end{lstlisting}

Le résultat de la compilation sera quasiment identique à celui de \myref{src:passing_arguments_ex_MSVC_cdecl}, 
mais vous constatez la présence de \TT{RET 12} au lieu de \TT{RET}.
\acLe \gls{caller} ne met pas à jour {SP} après l'appel.

Avec la convention \TT{\_\_stdcall}, on peut facilement déduire le nombre d'arguments de la fonction 
appelée à partir de l'instruction \TT{RETN n}: divisez $n$ par 4.

\lstinputlisting[caption=MSVC 2010,style=customasmx86]{OS/calling_conventions/stdcall_ex.asm}

\subsubsection{Fonctions à nombre d'arguments variables}

Les fonctions du style \printf sont un des rares cas de fonctions à nombre d'arguments variables 
en \CCpp. Elles permettent d'illustrer une différence importante entre les conventions \IT{cdecl} et \IT{stdcall}. 
Partons du principe que le compilateur connait le nombre d'arguments utilisés à chaque fois que la 
fonction \printf est appelée.

En revanche, la fonction \printf est déjà compilée dans MSVCRT.DLL (si l'on parle de Windows) et ne
possède aucune information sur le nombre d'arguments qu'elle va recevoir. Elle peut cependant le
deviner à partir du contenu du paramètre format.

Si la convention \IT{stdcall} était utilisé pour la fonction \printf, elle devrait réajuster le 
\gls{stack pointer} à sa valeur initiale en comptant le nombre d'arguments dans la chaîne de format.
Cette situation serait dangereuse et pourrait provoquer un crash du programme à la moindre faute de
frappe du programmeur.
La convention \IT{stdcall} n'est donc pas adaptée à ce type de fonction. La convention \IT{cdecl} est préférable.

\subsection{fastcall}
\label{fastcall}
\myindex{fastcall}

Il s'agit d'un nom générique pour désigner les conventions qui passent une partie des paramètres dans
des registres et le reste sur la pile.
Historiquement, cette méthode était plus performante que les conventions \IT{cdecl}/\IT{stdcall} - 
car elle met moins de pression sur la pile. Ce n'est cependant probablement plus le cas sur les
processeurs actuels qui sont beaucoup plus complexes.

Cette convention n'est pas standardisée. Les compilateurs peuvent donc l'implémenter à leur guise.
Prenez une DLL qui en utilise une autre compilée avec une interprétation différente de la convention 
\IT{fastcall}. Vous avez un cas d'école et des problèmes en perspective.

Les compilateurs MSVC et GCC passent les deux premiers arguments dans \ECX et \EDX, et le reste des
arguments sur la pile.

Le \gls{stack pointer} doit être restauré à sa valeur initiale par le \gls{callee} 
(comme pour la convention \IT{stdcall}).

\begin{lstlisting}[caption=fastcall,style=customasmx86]
push arg3
mov edx, arg2
mov ecx, arg1
call function

function:
.. do something ..
ret 4
\end{lstlisting}

Prenons par exemple la fonction \myref{src:passing_arguments_ex} et modifions la légèrement en utilisant la convention \TT{\_\_fastcall} :

\begin{lstlisting}[style=customc]
int __fastcall f3 (int a, int b, int c)
{
	return a*b+c;
};
\end{lstlisting}

Le résultat de la compilation est le suivant:

\lstinputlisting[caption=\MSVC 2010 optimisé/Ob0,style=customasmx86]{OS/calling_conventions/fastcall_ex.asm}

Nous voyons que le \gls{callee} ajuste \ac{SP} en utilisant l'instruction \TT{RETN} suivie d'un opérande.

Le nombre d'arguments peut, encore une fois, être facilement déduit.

\subsubsection{GCC regparm}

\newcommand{\URLREGPARMM}{\url{http://go.yurichev.com/17040}}

Il s'agit en quelque sorte d'une évolution de la convention \IT{fastcall}\footnote{\URLREGPARMM}.
L'option \TT{-mregparm} permet de définir le nombre d'arguments (3 au maximum) qui doivent être passés 
dans les registres. \EAX, \EDX et \ECX sont alors utilisés.

Bien entendu si le nombre d'arguments est inférieur à 3, seuls les premiers registres sont utilisés.

C'est le \gls{caller} qui effectue l'ajustement du \gls{stack pointer}.

Pour un exemple, voire (\myref{regparm}).

\subsubsection{Watcom/OpenWatcom}
\myindex{OpenWatcom}

Dans le cas de ce compilateur, on parle de \q{register calling convention}.
Les 4 premiers arguments sont passés dans les registres \EAX, \EDX, \EBX et \ECX.
Les paramètres suivant sont passés sur la pile.

Un suffixe constitué d'un caractère de soulignement est ajouté par le compilateur au nom de la
fonction afin de les distinguer de celles qui utilisent une autre convention d'appel.

\subsection{thiscall}
\myindex{thiscall}

This is passing the object's \ITthis pointer to the function-method, in \Cpp.

Le compilateur MSVC, passe généralement le pointeur \ITthis dans le registre \ECX.

Le compilateur GCC passe le pointeur \ITthis comme le premier argument de la fonction.
% BlueSkeye : don't undertand the underlying idea.
% BlueSkeye : at most this is true for instance method. Class level methods don't have a this pointer.
Thus it will be very visible that internally: all function-methods have an extra argument.

Pour un example, voir (\myref{thiscall}).

\subsection{x86-64}
\myindex{x86-64}

\subsubsection{Windows x64}
\label{sec:callingconventions_win64}

The method of for passing arguments in Win64 somewhat resembles \TT{fastcall}.
The first 4 arguments are passed via \RCX, \RDX, \Reg{8} and \Reg{9}, the rest---via the stack.
The \gls{caller} also must prepare space for 32 bytes or 4 64-bit values,
so then the \gls{callee} can save there the first 4 arguments.
Short functions may use the arguments' values just from the registers,
but larger ones may save their values for further use.

The \gls{caller} also must return the \gls{stack pointer} into its initial state.

This calling convention is also used in Windows x86-64 system DLLs 
(instead of \IT{stdcall} in win32).

Example:

\lstinputlisting[style=customc]{OS/calling_conventions/x64.c}

\lstinputlisting[caption=MSVC 2012 /0b,style=customasmx86]{OS/calling_conventions/x64_MSVC_Ob.asm}

\myindex{Scratch space}

Here we clearly see how 7 arguments are passed: 4 via registers and the remaining 3 via the stack.

The code of the f1() function's prologue saves the arguments in the \q{scratch space}---a space in the stack
intended exactly for this purpose.

This is arranged so because the compiler cannot be sure that there will be enough registers to use without these 4,
which will otherwise be occupied by the arguments until the function's execution end.

The \q{scratch space} allocation in the stack is the caller's duty.

\lstinputlisting[caption=\Optimizing MSVC 2012 /0b,style=customasmx86]{OS/calling_conventions/x64_MSVC_Ox_Ob.asm}

If we compile the example with optimizations, it is to be almost the same, 
but the \q{scratch space} will not be used, because it won't be needed.

\myindex{x86!\Instructions!LEA}
\label{using_MOV_and_pack_of_LEA_to_load_values}

Also take a look on how MSVC 2012 optimizes the loading of primitive values into registers by using \LEA (\myref{sec:LEA}).
\INS{MOV} would be 1 byte longer here (5 instead of 4).

Another example of such thing is: \myref{TaskMgr_LEA}.

\myparagraph{Windows x64: Passing \ITthis (\CCpp)}

The \ITthis pointer is passed in \RCX, the first argument of the method is in \RDX, etc.
For an example see: \myref{simple_CPP_MSVC_x64}.
 
\subsubsection{Linux x64}

The way arguments are passed in Linux for x86-64 is almost the same as in Windows, but 6 registers are
used instead of 4 (\RDI, \RSI, \RDX, \RCX, \Reg{8}, \Reg{9}) and there is no \q{scratch space}, 
although the \gls{callee} may save the register values in the stack, if it needs/wants to.

\lstinputlisting[caption=\Optimizing GCC 4.7.3,style=customasmx86]{OS/calling_conventions/x64_linux_O3.s}

\myindex{AMD}

N.B.: here the values are written into the 32-bit parts of the registers (e.g., EAX) but not in the whole 64-bit 
register (RAX).
This is because each write to the low 32-bit part of a register automatically clears the high 32 bits.
Supposedly, it was decided in AMD to do so to simplify porting code to x86-64.

\subsection{Return values of \Tfloat and \Tdouble type}
\myindex{float}
\myindex{double}

In all conventions except in Win64, the values of type \Tfloat or \Tdouble are returned via the FPU register \ST{0}.

In Win64, the values of \Tfloat and \Tdouble types are returned 
in the low 32 or 64 bits of the \XMM{0} register.

\subsection{Modifying arguments}

Sometimes, \CCpp{} programmers (not limited to these \ac{PL}s, though),
may ask, what can happen if they modify the arguments?

The answer is simple: the arguments are stored in the stack, 
that is where the modification takes place.

The calling functions is not using them after the \gls{callee}'s exit (the author of these lines has never seen any such case in his practice).

\lstinputlisting[style=customc]{OS/calling_conventions/change_arguments.c}

\lstinputlisting[caption=MSVC 2012,style=customasmx86]{OS/calling_conventions/change_arguments.asm}

% TODO (OllyDbg) пример как в стеке меняется $a$

So yes, one can modify the arguments easily.
Of course, if it is not \IT{references} in \Cpp{} (\myref{cpp_references}),
and if you don't modify data to which a pointer points to, 
then the effect will not propagate outside the current function.

Theoretically, after the \gls{callee}'s return, 
the \gls{caller} could get the modified argument and use it somehow.
Maybe if it is written directly in assembly language.

For example, code like this will be generated by usual \CCpp compiler:

\begin{lstlisting}[style=customasmx86]
	push	456	; will be b
	push	123	; will be a
	call	f	; f() modifies its first argument
	add	esp, 2*4
\end{lstlisting}

We can rewrite this code like:

\begin{lstlisting}[style=customasmx86]
	push	456	; will be b
	push	123	; will be a
	call	f	; f() modifies its first argument
	pop	eax
	add	esp, 4
	; EAX=1st argument of f() modified in f()
\end{lstlisting}

Hard to imagine, why anyone would need this, but this is possible in practice.
Nevertheless, the \CCpp languages standards don't offer any way to do so.

% sections
\subsection{Taking a pointer to function argument}
\label{pointer_to_argument}

\dots even more than that, it's possible to take a pointer to the function's argument and pass
it to another function:

\lstinputlisting[style=customc]{OS/calling_conventions/ptr_to_argument/10.c}

It's hard to understand how it works until we can see the code:

\lstinputlisting[caption=\Optimizing MSVC 2010,style=customasmx86]{OS/calling_conventions/ptr_to_argument/MSVC_2010_O3.asm}

The address of the place in the stack where $a$ has been passed is just passed to another function.
It modifies the value addressed by the pointer and then \printf prints the modified value.

\par The observant reader might ask, what about calling conventions where the function's arguments are
passed in registers?

That's a situation where the \IT{Shadow Space} is used.

The input value is copied from the register
to the \IT{Shadow Space} in the local stack, and then this address is passed to the other function:

\lstinputlisting[caption=\Optimizing MSVC 2012 x64,style=customasmx86]{OS/calling_conventions/ptr_to_argument/MSVC_2012_x64_O3.asm}

GCC also stores the input value in the local stack:

\lstinputlisting[caption=\Optimizing GCC 4.9.1 x64,style=customasmx86]{OS/calling_conventions/ptr_to_argument/GCC491_x64_O3.s}

GCC for ARM64 does the same, but this space is called \IT{Register Save Area} here:

\lstinputlisting[caption=\Optimizing GCC 4.9.1 ARM64,style=customasmARM]{OS/calling_conventions/ptr_to_argument/GCC49_ARM64_O3.s}

By the way, a similar usage of the \IT{Shadow Space} is also considered here: \myref{variadic_arith_registers}.



