\ifdefined\RUSSIAN
\else
\section{Taking a pointer to function argument}

Even more than that, it's possible to take a pointer to the function argument and pass
it to another function:

\lstinputlisting{OS/calling_conventions/ptr_to_argument/10.c}

It's hard to understand how it works until we will see the code:

\lstinputlisting[caption=\Optimizing MSVC 2010]{OS/calling_conventions/ptr_to_argument/MSVC_2010_O3.asm}

The address of the place in stack where $a$ was passed is just passed to another function.
It will modify it and then \printf will print modified value.

Observant reader might ask, what about calling conventions where function arguments are
passed in registers?

That's a situation when \IT{Shadow Space} is used. 
So the input value is copied from register
to the \IT{Shadow Space} in the local stack, and then its address is passed into another function:

\lstinputlisting[caption=\Optimizing MSVC 2012 x64]{OS/calling_conventions/ptr_to_argument/MSVC_2012_x64_O3.asm}

GCC is also store input value in the local stack:

\lstinputlisting[caption=\Optimizing GCC 4.9.1 x64]{OS/calling_conventions/ptr_to_argument/GCC491_x64_O3.s}

GCC for ARM64 doing the same, but this space is called \IT{Register Save Area} here:

\lstinputlisting[caption=\Optimizing GCC 4.9.1 ARM64]{OS/calling_conventions/ptr_to_argument/GCC49_ARM64_O3.s}
\fi
