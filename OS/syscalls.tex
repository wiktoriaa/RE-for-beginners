\chapter{\RU{Системные вызовы (syscall-ы)}\EN{System calls (syscall-s)}}

\label{syscalls}
\index{syscall}

\index{kernel space}
\index{user space}
\RU{Как известно, все работающие процессы в \ac{OS} делятся на две категории}\EN{As we know, all running processes
inside \ac{OS} are divided into two categories}:
\RU{имеющие полный доступ ко всему ``железу''}\EN{those having all access to the hardware} (``kernel space'') 
\RU{и не имеющие}\EN{and those have not} (``user space'').

\RU{В первой категории ядро \ac{OS} и, обычно, драйвера}
\EN{There are \ac{OS} kernel and usually drivers in the first category}.

\RU{Во второй категории всё прикладное ПО}\EN{All applications are usually in the second category}.

\RU{Это разделение очень важно для безопасности \ac{OS}:
очень важно чтобы никакой процесс не мог испортить что-то в других процессах
или даже в самом ядре \ac{OS}}
\EN{This separation is crucial for \ac{OS} safety: it is very important not to give to any process possibility to screw up
something in other processes or even in \ac{OS} kernel}.
\index{kernel panic}
\index{BSoD}
\RU{С другой стороны, падающий драйвер или ошибка внутри ядра \ac{OS} обычно приводит к}
\EN{On the other hand, failing driver or error inside \ac{OS} kernel usually lead to} kernel panic \OrENRU \ac{BSOD}.

\RU{Защита x86-процессора устроена так что возможно разделить всё на 4 слоя защиты (rings), но и в Linux,
и в Windows, используются только 2}
\EN{x86-processor protection allows to separate everything into 4 levels of protection (rings), but both in Linux
and in Windows only two are used}: ring0 (``kernel space'') \AndENRU ring3 (``user space'').

\RU{Системные вызовы}\EN{System calls} (syscall-\RU{ы}\EN{s})
\RU{это точка где соединяются вместе оба эти пространства}\EN{is a point where these two areas are connected}.
\RU{Это, можно сказать, самое главное \ac{API} предоставляемое прикладному ПО}\EN{It can be said, this is the most
principal \ac{API} providing to application software}.

\RU{В \gls{Windows NT} таблица сисколлов находится в \ac{SSDT}}
\EN{As in \gls{Windows NT}, syscalls table reside in \ac{SSDT}}.

\index{shellcode}
\RU{Работа через syscall-ы популярна у авторов шеллкодов в вирусов,
потому что там обычно бывает трудно определить адреса нужных ф-ций в системных библиотеках,
а syscall-ами проще пользоваться, хотя и придется писать больше
кода из-за более низкого уровня абстракции этого \ac{API}}
\EN{Usage of syscalls is very popular among shellcode and computer viruses authors, 
because it is hard to determine the addresses of
needed functions in the system libraries, while it is easier to use syscalls, however, much more code should be
written due to lower level of abstraction of the \ac{API}}.
\RU{Также нельзя еще забывать, что номера syscall-ов, например, в Windows, могут отличаться от версии к версии}
\EN{It is also worth noting that the numbers of syscalls e.g. in Windows,
may be different from version to version}.

\section{Linux}

\index{int 0x80}
\RU{В Linux вызов syscall-а обычно происходит через}\EN{In Linux, syscall is usually called via} \TT{int 0x80}.
\RU{В регистре}\EN{Call number is passed in the} \EAX \RU{передается номер вызова,
в остальных регистрах ~---- параметры}\EN{register, and any other parameters~---in the other registers}.

\lstinputlisting[caption=\RU{Простой пример использования пары syscall-ов}\EN{Simple example of two syscalls usage}]
{OS/linux_syscall.s}

\RU{Компиляция}\EN{Compilation}:

\begin{lstlisting}
nasm -f elf32 1.s
ld 1.o
\end{lstlisting}

\RU{Полный список syscall-ов в}\EN{The full list of syscalls in} Linux: \url{http://syscalls.kernelgrok.com/}.

\RU{Для перехвата и трассировки системных вызовов в Linux, можно применять}
\EN{For system calls intercepting and tracing in Linux,} strace(\ref{strace})\EN{ can be used}.

\section{Windows}

\index{int 0x2e}
\index{x86!\Instructions!SYSENTER}

\RU{Вызов происходит через}\EN{They are called by} \TT{int 0x2e} 
\RU{либо используя специальную x86-инструкцию}\EN{or using special x86 instruction} \TT{SYSENTER}.

\RU{Полный список syscall-ов в}\EN{The full list of syscalls in} Windows: \url{http://j00ru.vexillium.org/ntapi/}.

\RU{Смотрите также}\EN{Further reading}:

``Windows Syscall Shellcode'' by Piotr Bania:\\
\url{http://www.symantec.com/connect/articles/windows-syscall-shellcode}.

