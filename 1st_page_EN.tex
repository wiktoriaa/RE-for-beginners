% To translators: don't bother to translate this... english-only version.
\vspace*{\fill}

\normalsize \textbf{Subscribe to news about my other articles and blog posts:}

\bigskip
\bigskip
\bigskip

\begin{itemize}

\item \url{https://twitter.com/yurichev}

\item \url{https://www.facebook.com/dennis.yurichev.5}

\end{itemize}

\bigskip
\bigskip
\bigskip
% ---------------------------------------
\huge Reverse engineering services
\normalsize

\bigskip
\bigskip
\bigskip

I tried many jobs in my life, but, surprisingly (even to myself),
the job I'm the most proud of is rewriting large piece(s) of compiled code back to C/C++.
This is an extremely boring and slow process, I once spent more than a year on rewriting 100KB DLL to pure C,
and it was like full-time job.
And this is also expensive.

A lot of tricks have been added to this book as a result of this work.

I assume such a service could be interesting to those who inherited some compiled code with no source code.

You must be a legal owner of the software product.

\bigskip

I could also try (binary) code audit.
I can try to find vulnerabilities in your software, before others will do it.
This is like penetration testing.
I can try to work with binary code without source code.

You must also be a legal owner of the software product.

E-Mail: \GTT{\EMAIL}.

\bigskip
\bigskip
\bigskip

\huge Please donate
\normalsize

\bigskip
\bigskip
\bigskip

\dots to this project so I can continue to work on the book and other articles: \\
\url{https://yurichev.com/donate.html}.

\bigskip
\bigskip
\bigskip

\huge Attention: Opinion Poll
\normalsize

\bigskip
\bigskip
\bigskip

I have an idea to replace all the OllyDbg examples in the book with examples using some other debugger.
I have nothing against OllyDbg, but it has a GUI and uses small fonts, and the screenshots are somewhat unsuitable for the book.

Maybe I could use GDB, rada.re, WinDbg, or maybe some other console debugger?

What do you think about it?
Should I leave OllyDbg examples, or would radare examples would be OK?

E-Mail: \GTT{\EMAIL}.

\vspace*{\fill}
\vfill
