\section{\RU{Предисловие}\EN{Preface}}

\RU{Здесь (будет) немного моих заметок о \gls{reverse engineering} на русском языке для начинающих, 
для тех кто хочет научиться понимать создаваемый \CCpp компиляторами код для x86 (коего, 
практически, больше всего остального) и ARM.}
\EN{Here are some of my notes about \gls{reverse engineering} in English language for 
those beginners who would like to learn to understand x86 (which accounts for almost 
all executable software in the world) and ARM code created by \CCpp compilers.}

\RU{У термина ``\gls{reverse engineering}'' несколько популярных значений: 
1) исследование скомпилированных
программ; 2) сканирование трехмерной модели для последующего копирования;
3) восстановление структуры СУБД. Настоящий сборник заметок
связан с первым значением}
\EN{There are several popular meanings of the term ``\gls{reverse engineering}'': 
1) reverse engineering of software: researching of compiled programs;
2) 3D model scanning and reworking in order to make a copy of it;
3) recreating \ac{DBMS} structure.
These notes are related to the first meaning.}

\subsection*{\RU{Рассмотренные темы}\EN{Topics discussed}}

x86, ARM.

\subsection*{\RU{Затронутые темы}\EN{Topics touched}}

\oracle (\ref{oracle}),
Itanium (\ref{itanium}),
\RU{донглы для защиты от копирования}\EN{copy-protection dongles} (\ref{dongles}), 
LD\_PRELOAD (\ref{ld_preload}),
\RU{переполнение стека}\EN{stack overflow}, 
\ac{ELF},
\RU{формат файла PE в win32}\EN{win32 PE file format} (\ref{win32_pe}),
x86-64 (\ref{x86-64}),
\RU{критические секции}\EN{critical sections} (\ref{critical_sections}),
\RU{сисколлы}\EN{syscalls} (\ref{syscalls}), 
\ac{TLS},
\RU{адресно-независимый код}\EN{position-independent code} (\ac{PIC}) (\ref{sec:PIC}), 
profile-guided optimization (\ref{PGO}),
C++ STL (\ref{cpp_STL}),
OpenMP (\ref{openmp}),
SEH (\label{sec:SEH}).

\subsection*{\RU{Мини-}\EN{Mini-}\ac{FAQ}}

\newcommand{\FNURLREDDIT}{\footnote{\url{http://www.reddit.com/r/ReverseEngineering/}}}

\begin{itemize}
\item
Q: \RU{Нужно ли учиться понимать язык ассемблера в наше время?}
\EN{Should one learn to understand assembly language these days?} \\
A: \RU{Да: ради того чтобы понимать лучше внутреннее устройство, отлаживать код лучше и быстрее.}
\EN{Yes: in order to have deeper understanding of the internals and to debug your software better and faster.}

\item
Q: \RU{Нужно ли учиться писать на языке ассемблера в наше время?}
\EN{Should one learn to write in assembly language these days?} \\
A: \RU{Пожалуй, нет, если только не писать низкоуровневый код для \ac{OS}.}
\EN{Unless one writes low-level \ac{OS} code, probably no.}

\item
Q: \RU{Но для написания очень оптимизированных процедур?}
\EN{But what about writing highly optimized routines?} \\
A: \RU{Нет, современные компиляторы \CCpp делают это лучше.}
\EN{No, modern \CCpp compilers do this job better.}

\item
Q: \RU{Нужно ли знать внутреннее устройство микропроцессоров}
\EN{Should I learn microprocessor internals}? \\
A: \RU{Современные}\EN{Modern} \ac{CPU}\RU{ очень сложные}\EN{-s are very complex}.
\RU{Если вы не собираетесь писать очень оптимизированный код
или не работаете над кодегенератором компилятора,
тогда устройство CPU можно изучать только в общих чертах}
\EN{If you do not plan to write highly optimized code or if you do not work on compiler's code generator
then you may still learn internals in bare outlines.}
\footnote{\RU{Очень хороший текст на эту тему}\EN{Very good text about it}: \cite{AgnerFog}}.
\RU{В то же время для понимания и анализа кода
достаточно только знать \ac{ISA}, назначения регистров, т.е., ``внешнюю'' часть \ac{CPU}, доступную
для прикладного программиста}
\EN{At the same time, in order to understand and analyze compiled code it is enough to know
only \ac{ISA}, register's descriptions, i.e., the ``outside'' part of a \ac{CPU} that is available to
an application programmer}.

\item
Q: \RU{И всё-таки, зачем мне учить ассемблер}\EN{So why should I learn assembly language anyway}? \\
A: \RU{В основном для лучшего понимания происходящего во время отладки и для исследования
программ без наличия исходных кодов, включая зловреды (или вредоносы)
\footnote{современные (2013) русскоязычные термины для malware}}
\EN{Mostly to better understand what is going on while debugging and for \gls{reverse engineering} without source code, including, but not limited to, malware}.

\item
Q: \RU{Как можно найти работу reverse engineer-а}\EN{How would I search for a reverse engineering job}? \\
A: \RU{На reddit, посвященному RE\FNURLREDDIT, время от времени бывают hiring thread}
\EN{There are hiring threads that appear from time to time on reddit devoted to RE\FNURLREDDIT}
(\href{http://www.reddit.com/r/ReverseEngineering/comments/1hywvr/rreverseengineerings_q3_2013_hiring_thread/}{2013 Q3}, 
\href{http://www.reddit.com/r/ReverseEngineering/comments/1vui22/rreverseengineerings_2014_hiring_thread/}{2014}).
\RU{Посмотрите там}\EN{Try to take a look there}.
\end{itemize}

\subsection*{\RU{Об авторе}\EN{About the author}}

\begin{tabularx}{\textwidth}{ l X }

\raisebox{-\totalheight}{
\includegraphics[scale=0.60]{Dennis_Yurichev.jpg}
}

&
\RU{Денис Юричев ~--- опытный reverse engineer и программист.
С его резюме можно ознакомиться на его сайте}
\EN{Dennis Yurichev is an experienced reverse engineer and programmer.
His CV is available on his website}\footnote{\url{http://yurichev.com/Dennis_Yurichev.pdf}}. \\
% FIXME: no link. \tablefootnote doesn't work
\end{tabularx}

\subsection*{\RU{Благодарности}\EN{Thanks}}

\HERMIT, \RU{Слава ''Avid'' Казаков, Станислав ''Beaver'' Бобрицкий, Александр Лысенко, 
Александр ''Lstar'' Черненький, Андрей Зубинский, Владимир Ботов}
\EN{Slava ''Avid'' Kazakov, Stanislav ''Beaver'' Bobrytskyy, Alexander Lysenko, 
Alexander ''Lstar'' Chernenkiy, Andrew Zubinski, Vladimir Botov}, \RU{Марк}\EN{Mark} ``Logxen'' \RU{Купер}\EN{Cooper},
Shell Rocket, Yuan Jochen Kang, Arnaud Patard (rtp \RU{на}\EN{on} \#debian-arm IRC), 
\RU{и всем тем на github.com кто присылал замечания и коррективы}\EN{and all the folks on github.com
who have contributed notes and corrections}.

\RU{Было использовано множество пакетов \LaTeX{}: их авторов я также хотел бы поблагодарить}
\EN{A lot of \LaTeX{} packages were used: I would thank their authors as well}.

\subsection*{\RU{Отзывы об этой книге}\EN{Praise for \IT{\TITLE}}}

\begin{itemize}
\item ``It's very well done .. and for free .. amazing.''\footnote{\url{https://twitter.com/daniel_bilar/status/436578617221742593}} Daniel Bilar, Siege Technologies, LLC.

\item ``...excellent and free''\footnote{\url{https://twitter.com/petefinnigan/status/400551705797869568}} Pete Finnigan, \RU{гуру по безопасности }\oracle\EN{ security guru}.

\item ``... book is interesting, great job!'' Michael Sikorski, \RU{автор книги}\EN{author of} \IT{Practical Malware Analysis: The Hands-On Guide to Dissecting Malicious Software}.

\item ``... my compliments for the very nice tutorial!'' Herbert Bos, \RU{профессор университета}\EN{full professor at the} Vrije Universiteit Amsterdam.

\item ``... It is amazing and unbelievable.'' Luis Rocha, CISSP / ISSAP, Technical Manager, Network \& Information Security at Verizon Business.

\item ``Thanks for the great work and your book.'' Joris van de Vis, SAP Netweaver \& Security specialist.

\item ``... reasonable intro to some of the techniques.''\footnote{\url{http://www.reddit.com/r/IAmA/comments/24nb6f/i_was_a_professional_password_cracker_who_taught/}} (Mike Stay, teacher at the Federal Law Enforcement Training Center, Georgia, US.)

\end{itemize}

\section{\IFRU{Поддержите автора}{Donate}}

\IFRU{Эта книга является свободной, находится в свободном доступе, и доступна в виде исходных кодов}
{This book is free, available freely and available in source code form}\footnote{\url{https://github.com/dennis714/RE-for-beginners}} (LaTeX), 
\IFRU{и всегда будет оставаться таковой}{and it will be so forever}.

\IFRU{Если вы хотите поддержать мою работу, чтобы я мог продолжать регулярно дополнять её и далее, 
вы можете рассмотреть идею пожертвования}{If you want to support my work, so that I can continue to add things
to it regularly, you may consider donation}.

\IFRU{Со способами пожертвований можно ознакомиться на странице}
{Ways to donate are available on the page:} \url{http://yurichev.com/donate.html}


\subsection*{\RU{Об иллюстрациях}\EN{About illustrations}}

\RU{Читатели, привыкшие читать интернет-страницы, вероятно привыкли к тому что иллюстрации
находятся там же, где и должны}\EN{Those readers who are used to read a lot in the Internet, expects
seeing illustrations at the places where they should be}.
\RU{Это потому что там нет разбивок на страницы, там она только одна}\EN{It's because there 
are no pages at all, only single one}.
\RU{В книгах же, иллюстрации далеко не всегда удается вставить в том контексте где нужно}
\EN{It's not possible to place illustrations in the book at the suitable context}.
\RU{Так что, здесь бывает так, что они все находятся в конце секции,
и по тексту на них ставятся ссылки вроде}\EN{So, in this book, illustrations can be at the end of section,
and a referenceses in the text may be present, like}
``\figname{}1.1''.
\ifdefined\ebook

\RU{Это версия формата A5 для электронных читалок}\EN{This is A5-format version for e-book readers}. 
\RU{Хотя, тут всё то же самое, но иллюстрации уменьшены и не очень хорошо читаемы}
\EN{Although, it mostly the same, but illustrations are resized and probably not readable}. 
\RU{Извините}\EN{Sorry for this}! \RU{Вы всегда можете посмотреть их в версии формата A4 здесь}
\EN{You can always view them in A4-format version here}: \url{http://yurichev.com/RE-book.html}.
\fi

% {\RU{Целевая аудитория}\EN{Target audience}}
