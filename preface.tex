\section{\RU{Предисловие}\EN{Preface}}

\RU{Здесь (будет) немного моих заметок о \gls{reverse engineering} на русском языке для начинающих, 
для тех кто хочет научиться понимать создаваемый \CCpp компиляторами код для x86 (коего, 
практически, больше всего остального) и ARM.}
\EN{Here are some of my notes about \gls{reverse engineering} in English language for 
those beginners who would like to learn to understand x86 (which accounts for almost 
all executable software in the world) and ARM code created by \CCpp compilers.}

\RU{У термина ``\gls{reverse engineering}'' несколько популярных значений: 
1) исследование скомпилированных
программ; 2) сканирование трехмерной модели для последующего копирования;
3) восстановление структуры СУБД. Настоящий сборник заметок
связан с первым значением}
\EN{There are several popular meanings of the term ``\gls{reverse engineering}'': 
1) reverse engineering of software: researching of compiled programs;
2) 3D model scanning and reworking in order to make a copy of it;
3) recreating \ac{DBMS} structure.
These notes are related to the first meaning.}

\subsection*{\RU{Рассмотренные темы}\EN{Topics discussed}}

x86, ARM.

\subsection*{\RU{Затронутые темы}\EN{Topics touched}}

\oracle (\ref{oracle}),
Itanium (\ref{itanium}),
\RU{донглы для защиты от копирования}\EN{copy-protection dongles} (\ref{dongles}), 
LD\_PRELOAD (\ref{ld_preload}),
\RU{переполнение стека}\EN{stack overflow}, 
\ac{ELF},
\RU{формат файла PE в win32}\EN{win32 PE file format} (\ref{win32_pe}),
x86-64 (\ref{x86-64}),
\RU{критические секции}\EN{critical sections} (\ref{critical_sections}),
\RU{сисколлы}\EN{syscalls} (\ref{syscalls}), 
\ac{TLS},
\RU{адресно-независимый код}\EN{position-independent code} (\ac{PIC}) (\ref{sec:PIC}), 
profile-guided optimization (\ref{PGO}),
C++ STL (\ref{cpp_STL}),
OpenMP (\ref{openmp}),
SEH (\label{sec:SEH}).

\subsection*{\RU{Еще кое-что}\EN{Couple of things}}

\newcommand{\FNURLREDDIT}{\footnote{\url{http://www.reddit.com/r/ReverseEngineering/}}}

\EN{Why one should learn assembly language these days?}
\RU{Зачем в наше время нужно изучать язык ассемблера?}
\EN{Unless you are OS developer, you probably don't need write in assembly: 
modern compiler do optimizations much better than humans}
\RU{Если вы не разработчик OS, вам наверное не нужно писать на ассемблере:
современные компиляторы оптимизируют код намного лучше человека}
\footnote{\RU{Очень хороший текст на эту тему}\EN{Very good text about it}: \cite{AgnerFog}}.
\EN{Modern \ac{CPU}s are very complex devices as well, and assembly knowledge would 
not help one to understand its internals.}
\RU{Современные \ac{CPU} это также крайне сложные устройства, и знание ассемблера врядли
поможет узнать его внутренности.}
\EN{But there are still at least two areas where good assembly understanding may help:
1) security/malware research; 2) better understanding of your compiled code while debugging.}
\RU{Но все-таки остается по крайней мере две области, где знание ассемблера может хорошо
помочь:
1) исследование malware (\IT{зловредов}) в целях security research; 2) лучшее понимание
вашего скомпилированного кода в процессе отладки.}

\EN{Hence, this book is intended to those who wants to understand assembly language rather 
than to write in it, and that's why here a lot of examples connected to compilers output.}
\RU{Таким образом, эта книга предназначена для тех, кто хочет скорее понимать ассемблер,
нежели писать на нем, и вот почему здесь масса примеров связанных с результатами
работы компиляторов.}

\RU{Как можно найти работу reverse engineer-а}\EN{How would one search for a reverse engineering job}? \\
\RU{На reddit, посвященному RE\FNURLREDDIT, время от времени бывают hiring thread}
\EN{There are hiring threads that appear from time to time on reddit devoted to RE\FNURLREDDIT}
(\href{http://www.reddit.com/r/ReverseEngineering/comments/1hywvr/rreverseengineerings_q3_2013_hiring_thread/}{2013 Q3}, 
\href{http://www.reddit.com/r/ReverseEngineering/comments/1vui22/rreverseengineerings_2014_hiring_thread/}{2014}).
\RU{Посмотрите там}\EN{Try to take a look there}.
\EN{Somewhat adjacent hiring thread is in ``netsec'' subreddit}\RU{В смежном субреддите ``netsec'' имеется похожий тред}: 
\href{http://www.reddit.com/r/netsec/comments/221xxu/rnetsecs_q2_2014_information_security_hiring}{2014 Q2}.

\subsection*{\RU{Об авторе}\EN{About the author}}

\begin{tabularx}{\textwidth}{ l X }

\raisebox{-\totalheight}{
\includegraphics[scale=0.60]{Dennis_Yurichev.jpg}
}

&
\RU{Денис Юричев ~--- опытный reverse engineer и программист.
С его резюме можно ознакомиться на его сайте}
\EN{Dennis Yurichev is an experienced reverse engineer and programmer.
His CV is available on his website}\footnote{\url{http://yurichev.com/Dennis_Yurichev.pdf}}. \\
% FIXME: no link. \tablefootnote doesn't work
\end{tabularx}

\subsection*{\RU{Благодарности}\EN{Thanks}}

\RU{Тем, кто много помогал мне отвечая на массу вопросов}\EN{To those who answered all my
questions patiently}:
\HERMIT, \RU{Слава ''Avid'' Казаков}\EN{Slava ''Avid'' Kazakov}.

\RU{Тем, кто присылал замечания об ошибках и неточностях}\EN{To those who had sent me a notes
about mistakes and inaccuracies}:
\RU{Станислав ''Beaver'' Бобрицкий, Александр Лысенко}
\EN{Stanislav ''Beaver'' Bobrytskyy, Alexander Lysenko}, Shell Rocket, Zhu Ruijin.

\RU{Просто помогали разными способами}\EN{To those who just helped me in other ways}:
\RU{Андрей Зубинский}\EN{Andrew Zubinski}, 
Arnaud Patard (rtp \RU{на}\EN{on} \#debian-arm IRC).

\RU{Переводчику на китайский язык}\EN{To translator to Chinese simplified}: Xian Chi.

\RU{Корректорам}\EN{To proofreaders}:
\RU{Александр ''Lstar'' Черненький}\EN{Alexander ''Lstar'' Chernenkiy},
\RU{Владимир Ботов}\EN{Vladimir Botov},
\RU{Марк}\EN{Mark} ``Logxen'' \RU{Купер}\EN{Cooper},
Yuan Jochen Kang.

\RU{И еще всем тем на github.com кто присылал замечания и коррективы}
\EN{And all the folks on github.com who have contributed notes and corrections}.

\RU{Было использовано множество пакетов \LaTeX\: их авторов я также хотел бы поблагодарить}
\EN{A lot of \LaTeX\ packages were used: I would thank their authors as well}.

\section{\IFRU{Поддержите автора}{Donate}}

\IFRU{Эта книга является свободной, находится в свободном доступе, и доступна в виде исходных кодов}
{This book is free, available freely and available in source code form}\footnote{\url{https://github.com/dennis714/RE-for-beginners}} (LaTeX), 
\IFRU{и всегда будет оставаться таковой}{and it will be so forever}.

\IFRU{Если вы хотите поддержать мою работу, чтобы я мог продолжать регулярно дополнять её и далее, 
вы можете рассмотреть идею пожертвования}{If you want to support my work, so that I can continue to add things
to it regularly, you may consider donation}.

\IFRU{Со способами пожертвований можно ознакомиться на странице}
{Ways to donate are available on the page:} \url{http://yurichev.com/donate.html}


\subsection*{\RU{Об иллюстрациях}\EN{About illustrations}}

\RU{Читатели, привыкшие читать интернет-страницы, вероятно привыкли к тому что иллюстрации
находятся там же, где и должны}\EN{Those readers who are used to read a lot in the Internet, expects
seeing illustrations at the places where they should be}.
\RU{Это потому что там нет разбивок на страницы, там она только одна}\EN{It's because there 
are no pages at all, only single one}.
\RU{В книгах же, иллюстрации далеко не всегда удается вставить в том контексте где нужно}
\EN{It's not possible to place illustrations in the book at the suitable context}.
\RU{Так что, здесь бывает так, что они все находятся в конце секции,
и по тексту на них ставятся ссылки вроде}\EN{So, in this book, illustrations can be at the end of section,
and a referenceses in the text may be present, like}
``\figname{}1.1''.
\ifdefined\ebook

\RU{Это версия формата A5 для электронных читалок}\EN{This is A5-format version for e-book readers}. 
\RU{Хотя, тут всё то же самое, но иллюстрации уменьшены и не очень хорошо читаемы}
\EN{Although, it mostly the same, but illustrations are resized and probably not readable}. 
\RU{Извините}\EN{Sorry for this}! \RU{Вы всегда можете посмотреть их в версии формата A4 здесь}
\EN{You can always view them in A4-format version here}: \url{http://yurichev.com/RE-book.html}.
\fi

% {\RU{Целевая аудитория}\EN{Target audience}}
