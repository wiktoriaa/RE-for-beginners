\part{Tools}

\chapter{Disassembler}

\section{IDA}

\label{IDA}
An older freeware version is available for download
\footnote{\href{http://go.yurichev.com/17031}{hex-rays.com/products/ida/support/download\_freeware.shtml}}.

\ShortHotKeyCheatsheet: \myref{sec:IDA_cheatsheet}

\chapter{Debugger}

\section{\olly}
\myindex{\olly}

Very popular user-mode win32 debugger: \href{http://go.yurichev.com/17032}{ollydbg.de}.

\ShortHotKeyCheatsheet: \myref{sec:Olly_cheatsheet}

\section{GDB}
\myindex{GDB}

Not very popular debugger among reverse engineers, but very comfortable nevertheless.

Some commands: \myref{sec:GDB_cheatsheet}.

\section{tracer}

\myindex{tracer}
\label{tracer}
The author often uses
\IT{tracer}
\footnote{\href{http://go.yurichev.com/17338}{yurichev.com}}
instead of a debugger.

The author of these lines stopped using a debugger eventually, since all he needs from it is to spot function arguments while
executing, or registers state at some point.
Loading a debugger each time is too much, so a small utility called \IT{tracer} was born.
It works from command line, allows intercepting function execution,
setting breakpoints at arbitrary places, reading and changing registers state, etc.

However, for learning purposes it is highly advisable to trace code in a debugger manually, watch how the registers state
changes (e.g. classic SoftICE, OllyDbg, WinDbg highlight changed registers), flags, data, change them
manually, watch the reaction, etc.

\chapter{System calls tracing}

\label{strace}
\myindex{strace}
\myindex{dtruss}
\subsection{strace / dtruss}

\myindex{syscall}
It shows which system calls (syscalls(\myref{syscalls})) are called by a process right now.

For example:

\begin{lstlisting}
# strace df -h

...

access("/etc/ld.so.nohwcap", F_OK)      = -1 ENOENT (No such file or directory)
open("/lib/i386-linux-gnu/libc.so.6", O_RDONLY|O_CLOEXEC) = 3
read(3, "\177ELF\1\1\1\0\0\0\0\0\0\0\0\0\3\0\3\0\1\0\0\0\220\232\1\0004\0\0\0"..., 512) = 512
fstat64(3, {st_mode=S_IFREG|0755, st_size=1770984, ...}) = 0
mmap2(NULL, 1780508, PROT_READ|PROT_EXEC, MAP_PRIVATE|MAP_DENYWRITE, 3, 0) = 0xb75b3000
\end{lstlisting}

\myindex{\MacOSX}
\MacOSX has dtruss for doing the same.

\myindex{Cygwin}
Cygwin also has strace, but as far as it's known, it works only for .exe-files
compiled for the cygwin environment itself.

\chapter{Decompilers}

There is only one known, publicly available, high-quality decompiler to C code: Hex-Rays:
\href{http://go.yurichev.com/17033}{hex-rays.com/products/decompiler/}

% TODO Java, .NET, VB, etc

\chapter{Other tools}

\begin{itemize}
\item Microsoft Visual Studio Express
\footnote{\href{http://go.yurichev.com/17034}{visualstudio.com/en-US/products/visual-studio-express-vs}}:
Stripped-down free version of Visual Studio, convenient for simple experiments.

Some useful options: \myref{sec:MSVC_options}.

\item
\label{Hiew}
Hiew\footnote{\href{http://go.yurichev.com/17035}{hiew.ru}}:
for small modifications of code in binary files.
	
\item
\myindex{binary grep}
binary grep: 
a small utility for searching any byte sequence in a big pile of files, 
including non-executable ones: \BGREPURL.
\end{itemize}

