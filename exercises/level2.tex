\chapter{\RU{Уровень}\EN{Level} 2}

\RU{Для решения задач второго уровня, вам вероятно понадобится текстовый редактор или тетрадка с ручкой}
\EN{For solving exercises of level 2, you probably will need text editor or paper with pencil}.

\section{\Exercise 2.1}
% toupper()

\index{OpenWatcom}
\RU{Это стандартная функция из библиотек Си. Исходник взят из OpenWatcom}
\EN{This is standard C library function. Source code taken from OpenWatcom}.

\subsection{MSVC 2010}

\lstinputlisting{exercises/1_1_msvc.asm}

\subsection{GCC 4.4.1 + \Othree}

\lstinputlisting{exercises/1_1_gcc.asm}

\subsection{Keil (ARM) + \Othree}

\lstinputlisting{exercises/1_1_ARM.s}

\subsection{Keil (thumb) + \Othree}

\lstinputlisting{exercises/1_1_thumb.s}

\section{\Exercise 2.2}
% atoi()

\index{OpenWatcom}
\RU{Это так же стандартная функция из библиотек Си. Исходник взят из OpenWatcom и немного переделан}. 
\EN{This is also standard C library function. Source code is taken from OpenWatcom and modified slightly}.

\RU{Эта функция использует стандартные функции Си:}
\EN{This function also use these standard C functions:} isspace() \AndENRU isdigit().

\subsection{MSVC 2010 + \Ox}

\lstinputlisting{exercises/1_2_msvc.asm}

\subsection{GCC 4.4.1}

\RU{Задача немного усложняется тем, что GCC представил isspace() и isdigit() 
как inline-функции и вставил их тела прямо в код.}
\EN{This exercise is slightly harder since GCC compiled isspace() and isdigit()
functions as inline-functions and inserted their bodies right into the code.}

\lstinputlisting{exercises/1_2_gcc.asm}

\subsection{Keil (ARM) + \Othree}

\lstinputlisting{exercises/1_2_ARM.s}

\subsection{Keil (thumb) + \Othree}

\lstinputlisting{exercises/1_2_thumb.s}

\section{\Exercise 2.3}
% rand()/srand()

\RU{Это так же стандартная функция из библиотек Си, а вернее, две функции, работающие в паре. 
Исходник взят из MSVC 2010 и немного переделан.}
\EN{This is standard C function too, actually, two functions working in pair.
Source code taken from MSVC 2010 and modified slightly.}

\RU{Суть переделки в том, что эта функция может корректно работать в мульти-тредовой среде, 
а я, для упрощения (или запутывания) убрал поддержку этого.}
\EN{The matter of modification is that this function can work properly in multi-threaded environment,
and I removed its support for simplification (or for confusion).}

\subsection{MSVC 2010 + \Ox}

\lstinputlisting{exercises/1_3_msvc.asm}

\subsection{GCC 4.4.1}

\lstinputlisting{exercises/1_3_gcc.asm}

\subsection{Keil (ARM) + \Othree}

\lstinputlisting{exercises/1_3_ARM.s}

\subsection{Keil (thumb) + \Othree}

\lstinputlisting{exercises/1_3_thumb.s}

\section{\Exercise 2.4}
% strstr()

\RU{Это стандартная функция из библиотек Си. Исходник взят из MSVC 2010.}
\EN{This is standard C library function. Source code taken from MSVC 2010.}

\subsection{MSVC 2010 + \Ox}

\lstinputlisting{exercises/1_4_msvc.asm}

\subsection{GCC 4.4.1}

\lstinputlisting{exercises/1_4_gcc.asm}

\subsection{Keil (ARM) + \Othree}

\lstinputlisting{exercises/1_4_ARM.s}

\subsection{Keil (thumb) + \Othree}

\lstinputlisting{exercises/1_4_thumb.s}

\section{\Exercise 2.5}
% Pentium FDIV bug

\RU{Задача, скорее, на эрудицию, нежели на чтение кода.}
\EN{This exercise is rather on knowledge than on reading code.}

\index{OpenWatcom}
\RU{Функция взята из OpenWatcom}.
\EN{The function is taken from OpenWatcom}.

\subsection{MSVC 2010 + \Ox}

\lstinputlisting{exercises/1_5_msvc.asm}

\section{\Exercise 2.6}
% TEA

\subsection{MSVC 2010 + \Ox}

\lstinputlisting{exercises/1_6_msvc.asm}

\subsection{Keil (ARM) + \Othree}

\lstinputlisting{exercises/1_6_ARM.s}

\subsection{Keil (thumb) + \Othree}

\lstinputlisting{exercises/1_6_thumb.s}

\section{\Exercise 2.7}
% bitrev.c

\RU{Это взята функция из ядра Linux 2.6.}\EN{This function is taken from Linux 2.6 kernel.}

\subsection{MSVC 2010 + \Ox}

\lstinputlisting{exercises/1_7_msvc.asm}

\subsection{Keil (ARM) + \Othree}

\lstinputlisting{exercises/1_7_ARM.s}

\subsection{Keil (thumb) + \Othree}

\lstinputlisting{exercises/1_7_thumb.s}

% 2.8
% 2.9
% 2.10

\section{\Exercise 2.11}

\RU{В рамках шутки, ``обманите'' ваш Windows Task Manager чтобы он показывал
больше процессоров/ядер процессоров чем есть в вашем компьютере на самом деле}
\EN{As a practical joke, ``fool'' your Windows Task Manager 
to show much more CPUs/CPU cores than your machine actually has}:

\begin{figure}[H]
\centering
\includegraphics[scale=\FigScale]{exercises/taskmgr_64cpu_crop.png}
\caption{\RU{Обманутый}\EN{Fooled} Windows Task Manager}
\end{figure}

\section{\Exercise 2.12}
% ROT13

\RU{Это довольно известный алгоритм}\EN{This is a well-known algorithm}.
\RU{Как он называется}\EN{How it's called}?

\subsection{MSVC 2012 x64 + \Ox}

\lstinputlisting{exercises/1_12_MSVC_x64.asm}

\subsection{Keil (ARM)}

\lstinputlisting{exercises/1_12_ARM.s}

\subsection{Keil (thumb)}

\lstinputlisting{exercises/1_12_thumb.s}

\section{\Exercise 2.13}
% LFSR

\RU{Это довольно известный криптоалгоритм прошлого}\EN{This is a well-known cryptoalgorithm of the past}.
\RU{Как он называется}\EN{How it's called}?

\subsection{MSVC 2012 + \Ox}

\begin{lstlisting}
_in$ = 8						; size = 2
_f	PROC
	movzx	ecx, WORD PTR _in$[esp-4]
	lea	eax, DWORD PTR [ecx*4]
	xor	eax, ecx
	add	eax, eax
	xor	eax, ecx
	shl	eax, 2
	xor	eax, ecx
	and	eax, 32					; 00000020H
	shl	eax, 10					; 0000000aH
	shr	ecx, 1
	or	eax, ecx
	ret	0
_f	ENDP
\end{lstlisting}

\subsection{Keil (ARM)}

\begin{lstlisting}
f PROC
        EOR      r1,r0,r0,LSR #2
        EOR      r1,r1,r0,LSR #3
        EOR      r1,r1,r0,LSR #5
        AND      r1,r1,#1
        LSR      r0,r0,#1
        ORR      r0,r0,r1,LSL #15
        BX       lr
        ENDP
\end{lstlisting}

\subsection{Keil (thumb)}

\begin{lstlisting}
f PROC
        LSRS     r1,r0,#2
        EORS     r1,r1,r0
        LSRS     r2,r0,#3
        EORS     r1,r1,r2
        LSRS     r2,r0,#5
        EORS     r1,r1,r2
        LSLS     r1,r1,#31
        LSRS     r0,r0,#1
        LSRS     r1,r1,#16
        ORRS     r0,r0,r1
        BX       lr
        ENDP
\end{lstlisting}

\section{\Exercise 2.14}
% GCD

\RU{Еще один хорошо известный алгоритм. Ф-ция берет на вход 2 значения и возвращает одно.}
\EN{Another well-known algorithm. The function takes two variables and returning one.}

\subsection{MSVC 2012}

\index{ARM!\Instructions!CLZ}
\lstinputlisting{exercises/2/GCD_MSVC_2012_Ox.asm}

\subsection{Keil (ARM mode)}

\index{ARM!\Instructions!CLZ}
\lstinputlisting{exercises/2/GCD_Keil_ARM_O3.s}

\subsection{GCC 4.6.3 for Raspberry Pi (ARM mode)}

\index{x86!\Instructions!BSF}
\lstinputlisting{exercises/2/GCD_ARM_pi_GCC_4.6.3_O3.s}

\section{\Exercise 2.15}
% Monte Carlo

\RU{И снова известный алгоритм. Что он делает?}\EN{Well-known algorithm again. What it does?}

\RU{Обратите внимание, что код для x86 использует FPU, а для x64 --- SIMD-инструкции. Это нормально}
\EN{Take also notice that the code for x86 uses FPU, but SIMD-instructions are used instead in x64 code.
That's OK}: \ref{floating_SIMD}.

\subsection{MSVC 2012 x64 /Ox}

\lstinputlisting{exercises/2/monte_MSVC_2012_Ox_x64.asm}

\subsection{GCC 4.4.6 \Othree x64}

\lstinputlisting{exercises/2/monte_GCC_4.4.6_O3_x64.s}

\subsection{GCC 4.8.1 \Othree x86}

\lstinputlisting{exercises/2/monte_GCC_4.8.1_O3_x86.s}

\subsection{Keil (ARM mode): \RU{для процессора Cortex-R4F}\EN{Cortex-R4F CPU as target}}

\lstinputlisting{exercises/2/monte_Keil_ARM_Cortex.s}

\section{\Exercise 2.16}
% Ackermann function

\RU{Известная функция. Что она вычисляет? Почему стек переполняется если на вход подать
числа 4 и 2? Есть ли здесь какая-то ошибка?}\EN{Well-known function. What it computes? 
Why stack overflows if 4 and 2 are supplied at input? Are there any error?}

\subsection{MSVC 2012 x64 /Ox}

\lstinputlisting{exercises/2/ack_MSVC_Ox_x64.asm}

\subsection{Keil (ARM) \Othree}

\lstinputlisting{exercises/2/ack_ARM_O3.s}

\subsection{Keil (thumb) \Othree}

\lstinputlisting{exercises/2/ack_thumb_O3.s}

\section{\Exercise 2.17}
% Rule 110

\RU{Эта программа выдает в \IT{stdout} какую-то информацию, каждый раз --- разную}\EN{This program
prints some information to \IT{stdout}, each time different}.
\RU{Что это}\EN{What is it}?

\RU{Скомпилированные бинарные файлы}\EN{Compiled binaries}:

\begin{itemize}
\item Linux x64: \url{http://yurichev.com/RE-exercises/2/17/17_Linux_x64.tar}
\item \MacOSX: \url{http://yurichev.com/RE-exercises/2/17/17_MacOSX_x64.tar}
\item Win32: \url{http://yurichev.com/RE-exercises/2/17/17_win32.exe}
\item Win64: \url{http://yurichev.com/RE-exercises/2/17/17_win64.exe}
\end{itemize}

\RU{Для версий под Windows, возможно, нужно будет установить}
\EN{As of Windows versions, you may need to install} 
\href{http://www.microsoft.com/en-us/download/details.aspx?id=30679}{MSVC 2012 redist}.

\section{\Exercise 2.18}

\RU{Эта программа запрашивает пароль}\EN{This program requires password}.
\RU{Найдите его}\EN{Find it}.

\RU{Кстати, не только один пароль может подойти}\EN{By the way, multiple passwords may work}. 
\RU{Попробуйте найти еще}\EN{Try to find more}.

\RU{Как дополнительное упражнение, попробуйте изменить пароль модифицируя исполняемый файл}
\EN{As an additional exercise, try to change the password by patching executable file}.

\begin{itemize}
\item Win32: \url{http://yurichev.com/RE-exercises/2/18/password2.exe}
\item Linux x86: \url{http://yurichev.com/RE-exercises/2/18/password2_Linux_x86.tar}
\item \MacOSX: \url{http://yurichev.com/RE-exercises/2/18/password2_MacOSX64.tar}
\end{itemize}

\section{\Exercise 2.19}

\RU{То же что и в упражнении}\EN{The same as in exercise} 2.18.

\begin{itemize}
\item Win32: \url{http://yurichev.com/RE-exercises/2/19/password3.exe}
\item Linux x86: \url{http://yurichev.com/RE-exercises/2/19/password3_Linux_x86.tar}
\item \MacOSX: \url{http://yurichev.com/RE-exercises/2/19/password3_MacOSX64.tar}
\end{itemize}

