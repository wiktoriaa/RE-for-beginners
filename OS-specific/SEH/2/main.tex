\subsection{\IFRU{Теперь вспомним MSVC}{Now let's get back to MSVC}}

\IFRU{Должно быть, программистам Microsoft были нужны исключения в Си, но не в Си++, так что они добавили
нестандартное расширение Си в MSVC}
{Supposedly, Microsoft programmers need exceptions in C, but not in C++, so they added a non-standard C extension
to MSVC}\footnote{\url{http://msdn.microsoft.com/en-us/library/swezty51.aspx}}.
\IFRU{Оно не связано с исключениями в Си++}{It is not related to C++ \ac{PL} exceptions}.

\begin{lstlisting}
__try
{
    ...
}
__except(filter code)
{
    handler code
}
\end{lstlisting}

\IFRU{Блок ``finally'' может присутствовать вместо код обработчика}{``Finally'' block may be instead of handler code}:

\begin{lstlisting}
__try
{
    ...
}
__finally
{
    ...
}
\end{lstlisting}

\IFRU{Код-фильтр это выражение, отвечающее на вопрос, соответствует ли код этого обработчика к поднятому исключению}
{The filter code is an expression, telling whether this handler code is coressponding to the exception raised}.
\IFRU{Если ваш код слишком большой и не помещается в одно выражение, отдельная функция-фильтр может быть определена}
{If your code is too big and cannot be fitted into one expression, a separate filter function can be defined}.

% a lot in WRK...

\IFRU{Вот несколько примеров из ядра Windows}{Here is couple of examples from Windows kernel} (Windows Research Kernel):

\lstinputlisting[caption=WRK-v1.2/base/ntos/ob/obwait.c]{OS-specific/SEH/2/wrk_ex1.c}

\lstinputlisting[caption=WRK-v1.2/base/ntos/cache/cachesub.c]{OS-specific/SEH/2/wrk_ex2.c}

\IFRU{Вот пример кода-фильтра}{Here is also filter code example}:

\lstinputlisting[caption=WRK-v1.2/base/ntos/cache/copysup.c]{OS-specific/SEH/2/wrk_ex3.c}

\IFRU{Внутри, SEH это расширение исключений поддерживаемых OS}{Internally, SEH is an extension of OS-supported exceptions}.
\IFRU{Но функция обработчик теперь или}{But the handler function is} \TT{\_except\_handler3} (\ForENRU SEH3) 
\OrENRU \TT{\_except\_handler4} (\ForENRU SEH4).
\IFRU{Код обработчика от MSVC, расположен в его библиотеках, или же в}
{The code of this handler is MSVC-related, it is located in its libraries, or in} msvcr*.dll.
\IFRU{Очень важно понимать что SEH это специфичное для MSVC}{It is very important to know that SEH is MSVC-specific thing}.
\IFRU{Другие компиляторы могут предлагать что-то совершенно другое}
{Other compilers may offer something completely different}.

\subsubsection{SEH3}

SEH3 \IFRU{имеет}{has} \TT{\_except\_handler3} \IFRU{как функцию-обработчик, и расширяет структуру}
{as handler functions, and extends} \TT{\_EXCEPTION\_REGISTRATION} \IFRU{добавляя указатель на \IT{scope table}
и переменную \IT{previous try level}}{table, adding
a pointer to the \IT{scope table} and \IT{previous try level} variable}.
SEH4 \IFRU{расширяет}{extends} \IT{scope table} \IFRU{добавляя еще 4 значения связанных с защитой от переполнения буфера}
{by 4 values for buffer overflow protection}.

\IT{Scope table} \IFRU{это таблица, состоящая из указателей на код фильтра и обработчика, для каждого уровня вложенности
\IT{try/except}}{is a table consisting of pointers to the filter and handler codes, for each level of \IT{try/except}
nestedness}.

\begin{center}
\begin{tikzpicture}[thick,scale=0.85, every node/.style={scale=0.85}]
	\tikzstyle{every path}=[thick]
	\tikzstyle{undefined}=[draw,rectangle,minimum height=1cm, minimum width=3.5cm, text width=3.5cm]
	\tikzstyle{undefinedw}=[draw,rectangle,minimum height=1cm, minimum width=4.5cm, text width=4.5cm]
	\tikzstyle{node}=[draw,rectangle,minimum height=1cm, minimum width=3.5cm, text width=3.5cm, fill=gray!20]
	\tikzstyle{nodew}=[draw,rectangle,minimum height=1cm, minimum width=4.5cm, text width=4.5cm, fill=gray!20]
	\tikzstyle{text_as_rect}=[rectangle,minimum height=1cm, minimum width=3.5cm, text width=3.5cm]
	\tikzstyle{point}=[inner sep=0pt]

	\node[node] (fs) [minimum width=1.5cm, text width=1.5cm] {FS:0};

\begin{scope}[start chain=1 going below, node distance=0cm]
	\node[on chain=1,node] (tib1) [right=1.5cm of fs] {+0: \_\_except\_list};
	\node[on chain=1,undefined] {+4: \dots};
	\node[on chain=1,undefined] {+8: \dots};
\end{scope}
	\node (tib_text) [above of=tib1] {TIB};
	
	\draw [->] (fs.east) -- (tib1.west);

\begin{scope}[start chain=3 going below, node distance=0cm]
	\node[on chain=3,undefined] (u1) [text centered, right=2.5cm of tib1] {\dots};
	%\node[on chain=3,node] (n1prev) {Prev=0xFFFFFFFF (1)};
	\node[on chain=3,node] (n1prev) {Prev=0xFFFFFFFF};
	\node[on chain=3,node] (n1handler) {Handle};
	\node[on chain=3,undefined] [text centered] {\dots};
	\node[on chain=3,node] (n2prev) {Prev};
	\node[on chain=3,node] (n2handler) {Handle};
	\node[on chain=3,undefined] [text centered] {\dots};
	\node[on chain=3,node] (n3prev) {Prev};
	\node[on chain=3,node] (n3handler) {Handle};
	\ifx\SEHfour\undefined	
	\node[on chain=3,node] (n3scopetable) {\ScopeTable};
	\else
	\node[on chain=3,node] (n3scopetable) {\ScopeTable$\oplus$security\_cookie};
	\fi
	\node[on chain=3,node] {\RU{предыдущий уровень try}\EN{previous try level}};
	\node[on chain=3,node] (n3ebp) {EBP};
	\ifx\SEHfour\undefined
	\else
	\node[on chain=3,undefined] [text centered] {\dots};
	\node[on chain=3,node] (AnotherCookie) {EBP$\oplus$security\_cookie};
	\node[on chain=3,undefined] [text centered] {\dots};
	\fi
\end{scope}
	
	\node [node] (n1handler_text) [right=1cm of n1handler] {\HandlerFunction};
	\draw [->] (n1handler.east) -- (n1handler_text.west);
	\node [node] (n2handler_text) [right=1cm of n2handler] {\HandlerFunction};
	\draw [->] (n2handler.east) -- (n2handler_text.west);

	\ifx\SEHfour\undefined	
	\node [node] (n3handler_text) [right=1cm of n3handler] {\_except\_handler3};
	\else
	\node [node] (n3handler_text) [right=1cm of n3handler] {\_except\_handler4};
	\fi

	\draw [->] (n3handler.east) -- (n3handler_text.west);
	\node[undefined] (u4) [text centered, below of=n3ebp] {\dots};
	\node (stack_text) [above of=u1] {Stack};
	
	\node[point] (n2block_pt2) [above left=0cm and 0.8cm of n2prev] {};
	\draw [->] (n3prev.west) .. controls +(left:0.8cm) and (n2block_pt2) .. (n2prev.north west);

	\node[point] (n1block_pt2) [above left=0cm and 0.8cm of n1prev] {};
	\draw [->] (n2prev.west) .. controls +(left:0.8cm) and (n1block_pt2) .. (n1prev.north west);
	
	\node[point] (n3block_pt2) [above left=0cm and 1.25cm of n3prev] {};
	\draw [->] (tib1.east) .. controls +(right:1.25cm) and (n3block_pt2) .. (n3prev.north west);

\begin{scope}[start chain=2 going below, node distance=0cm]
	% SEH3
	\ifx\SEHfour\undefined	
	\node[on chain=2,nodew] (scope_tbl) [left=3.5cm of n2handler] {0xFFFFFFFF (-1)};
	\else
	% SEH4 header
	\node[on chain=2,nodew] (scope_tbl) [left=3.5 cm of n2handler] {\RU{смещение GS Cookie}\EN{GS Cookie Offset}};
	\node[on chain=2,nodew] {\RU{смещение GS Cookie XOR}\EN{GS Cookie XOR Offset}};
	\node[on chain=2,nodew] (EHCookieOffset) {\RU{смещение EH Cookie}\EN{EH Cookie Offset}};
	\node[on chain=2,nodew] {\RU{смещение EH Cookie XOR}\EN{EH Cookie XOR Offset}};
	\node[on chain=2,minimum height=0.3cm] {};
	\node[on chain=2,nodew] {0xFFFFFFFF (-1)};
	\fi

	\node[on chain=2,nodew] (scope_tbl_filter1) {\FilterFunction};
	\node[on chain=2,nodew] {\HandlerFinallyFunction};
	\node[on chain=2,minimum height=0.3cm] {};
	\node[on chain=2,nodew] {0};
	\node[on chain=2,nodew] (scope_tbl_filter2) {\FilterFunction};
	\node[on chain=2,nodew] {\HandlerFinallyFunction};
	\node[on chain=2,minimum height=0.3cm] {};
	\node[on chain=2,nodew] {1};
	\node[on chain=2,nodew] (scope_tbl_filter3) {\FilterFunction};
	\node[on chain=2,nodew] {\HandlerFinallyFunction};
	\node[on chain=2,minimum height=0.3cm] {};
	\node[on chain=2,undefinedw] [text centered] {\dots \MoreEntries \dots};

\end{scope}
	
	\node[text_as_rect,left=0.5cm of scope_tbl_filter1] {\RU{информация для первого блока try/except}\EN{information about first try/except block}};
	\node[text_as_rect,left=0.5cm of scope_tbl_filter2] {\RU{информация для второго блока try/except}\EN{information about second try/except block}};
	\node[text_as_rect,left=0.5cm of scope_tbl_filter3] {\RU{информация для третьего блока try/except}\EN{information about third try/except block}};

	\node (scope_text) [above of=scope_tbl] {\ScopeTable};
	\node[point] (scope_table_pt2) [right=1.75cm of scope_tbl] {};
	\draw [->] (n3scopetable.west) .. controls +(left:1.75cm) and (scope_table_pt2) .. (scope_tbl.north east);

	\ifx\SEHfour\undefined
	\else
	\node[point] (AnotherCookie_pt2) [left=1.75cm of AnotherCookie] {};
	\draw [->] (EHCookieOffset.east) .. controls +(right:1.75cm) and (AnotherCookie_pt2) .. (AnotherCookie.west);
	\fi

\end{tikzpicture}
\end{center}


\IFRU{И снова, очень важно понимать, что OS заботится только о полях}
{Again, it is very important to understand that OS take care only of} \IT{prev/handle}
\IFRU{, и больше ничего}{ fields, and nothing more}.
\IFRU{Это работа функции}{It is job of} \TT{\_except\_handler3} \IFRU{читать другие поля, читать \IT{scope table}
и решать, какой обработчик исполнять и когда}{function to read other fields, read \IT{scope table}, and decide,
which handler to execute and when}.

\IFRU{Исходный код ф-ции}{The source code of} \TT{\_except\_handler3} \IFRU{закрыт}{function is closed}.
\IFRU{Хотя, Sanos OS, имеющая слой совместимости с win32, имеет некоторые ф-ции написанные заново, которые
в каком-то смысле эквивалентны тем что в Windows}
{However, Sanos OS, which have win32 compatibility layer, has the same
functions redeveloped, which are somewhat equivalent to those in Windows}
\footnote{\url{https://code.google.com/p/sanos/source/browse/src/win32/msvcrt/except.c}}.
\IFRU{Другие попытки реализации имеются в}{Another reimplementations are present in} 
Wine\footnote{\url{https://github.com/mirrors/wine/blob/master/dlls/msvcrt/except_i386.c}}
\AndENRU 
ReactOS\footnote{\url{http://doxygen.reactos.org/d4/df2/lib_2sdk_2crt_2except_2except_8c_source.html}}.

\IFRU{Если указатель}{If the} \IT{filter} \IFRU{ноль}{pointer is zero}, \IT{handler} \IFRU{указывает на код \IT{finally}}
{pointer is the pointer to a \IT{finally} code}.

\IFRU{Во время исполнения, значение}{During execution,} \IT{previous try level} \IFRU{в стеке меняется,
чтобы ф-ция}{value in the stack is changing, so the} \TT{\_except\_handler3} \IFRU{знала о текущем уровне вложенности,
чтобы знать, какой элемент таблицы}{will know about current
state of nestedness, in order to know which} \IT{scope table} \IFRU{использовать}{entry to use}.

\subsubsection{SEH3: \IFRU{пример с одним блоком try/except}{one try/except block example}}

\lstinputlisting{OS-specific/SEH/2/2.c}

\lstinputlisting[caption=MSVC 2003]{OS-specific/SEH/2/2_SEH3.asm}

\IFRU{Здесь мы видим как структура SEH конструируется в стеке}{Here we see how SEH frame is being constructed in the stack}.
\IT{Scope table} \IFRU{расположена в сегменте}{is located in the} \TT{CONST}\IFRU{}{ segment}\EMDASH{}
\IFRU{действительно, эти поля не будут меняться}{indeed, these fields will not be changed}.
\IFRU{Интересно, как меняется переменная}{An interesting thing is how} \IT{previous try level}\IFRU{}{ variable is changed}.
\IFRU{Исходное значение}{Initial value is} \TT{0xffffffff} ($-1$).
\IFRU{Момент, когда тело}{The moment when body of} \TT{try} \IFRU{открывается, обозначен инструкцией, записывающей}
{statement is opened is marked as an instruction writing} $0$ \IFRU{в эту переменную}{to the variable}.
\IFRU{В момент, когда тело}{The moment when body of} \TT{try} \IFRU{закрывается}{statement is closed}, $-1$ 
\IFRU{возвращается в нее назад}{is returned back to it}.
\IFRU{Мы так же видим адреса кода фильтра и обработчика}{We also see addresses of filter and handler code}.
\IFRU{Так мы можем легко увидеть структуру конструкций \IT{try/except} в ф-ции}
{Thus we can easily see the structure of \IT{try/except} constructs in the function}.

\IFRU{Так как код инициализации SEH-структур в прологе ф-ций может быть общим для нескольких ф-ций, иногда компилятор
вставляет в прологе вызов ф-ции}{Since the SEH setup code in the function prologue may be shared between many of functions,
sometimes compiler inserts a call to} \TT{SEH\_prolog()}\IFRU{, которая всё это делает}
{ function in the prologue, which do that}.

\IFRU{Запустим этот пример в}{Let's try to run this example in} \tracer{}:

\begin{lstlisting}
tracer.exe -l:2.exe --dump-seh
\end{lstlisting}

\begin{lstlisting}[caption=tracer.exe output]
EXCEPTION_ACCESS_VIOLATION at 2.exe!main+0x44 (0x401054) ExceptionInformation[0]=1
EAX=0x00000000 EBX=0x7efde000 ECX=0x0040cbc8 EDX=0x0008e3c8
ESI=0x00001db1 EDI=0x00000000 EBP=0x0018feac ESP=0x0018fe80
EIP=0x00401054
FLAGS=AF IF RF
* SEH frame at 0x18fe9c prev=0x18ff78 handler=0x401204 (2.exe!_except_handler3)
SEH3 frame. previous trylevel=0
scopetable entry[0]. previous try level=-1, filter=0x401070 (2.exe!main+0x60) handler=0x401088 (2.exe!main+0x78)
* SEH frame at 0x18ff78 prev=0x18ffc4 handler=0x401204 (2.exe!_except_handler3)
SEH3 frame. previous trylevel=0
scopetable entry[0]. previous try level=-1, filter=0x401531 (2.exe!mainCRTStartup+0x18d) handler=0x401545 (2.exe!mainCRTStartup+0x1a1)
* SEH frame at 0x18ffc4 prev=0x18ffe4 handler=0x771f71f5 (ntdll.dll!__except_handler4)
SEH4 frame. previous trylevel=0
SEH4 header:	GSCookieOffset=0xfffffffe GSCookieXOROffset=0x0
		EHCookieOffset=0xffffffcc EHCookieXOROffset=0x0
scopetable entry[0]. previous try level=-2, filter=0x771f74d0 (ntdll.dll!___safe_se_handler_table+0x20) handler=0x771f90eb (ntdll.dll!_TppTerminateProcess@4+0x43)
* SEH frame at 0x18ffe4 prev=0xffffffff handler=0x77247428 (ntdll.dll!_FinalExceptionHandler@16)
\end{lstlisting}

\IFRU{Мы видим что цепочка SEH состоит из 4-х обработчиков}{We that SEH chain consisting of 4 handlers}. 

\IFRU{Первые два расположены в нашем примере}{First two are located in out example}. \IFRU{Два}{Two}?
\IFRU{Но ведь мы же сделали только один}{But we made only one}?
\IFRU{Да, второй был установлен в \ac{CRT}-функции}{Yes, another one is setting up in \ac{CRT} function} 
\TT{\_mainCRTStartup()}, \IFRU{и судя по всему, он обрабатывает как минимум исключения связанные с \ac{FPU}}
{and as it seems, it handles at least \ac{FPU} exceptions}.
\IFRU{Его код можно посмотреть в инсталляции MSVC}{Its source code can found in MSVS installation}: \TT{crt/src/winxfltr.c}.

\IFRU{Третий это SEH4 в}{Third is SEH4 frame in} ntdll.dll, \IFRU{и четвертый это обработчик, не имеющий отношения к MSVC,
расположенный в}
{and the fourth handler is not MSVC-related located in} ntdll.dll,
\IFRU{имеющий ``говорящее'' название ф-ции}{and it has self-describing function name}.

\IFRU{Как видно, в цепочке присутствуют обработчики трех типов}{As you can see, there are 3 types of handlers in one chain}:
\IFRU{один не связан с MSVC вообще (последний) и два связанных с MSVC}
{one is not related to MSVC at all (the last one) and two MSVC-related}: SEH3 \AndENRU SEH4.

\subsubsection{SEH3: \IFRU{пример с двумя блоками try/except}{two try/except blocks example}}

\lstinputlisting{OS-specific/SEH/2/3.c}

\IFRU{Теперь здесь два блока \TT{try}}{Now there are two \TT{try} blocks}.
\IFRU{Так что}{So the} \IT{scope table} \IFRU{теперь содержит два элемента, один элемент на каждый блок}
{now have two entries, each entry for each block}.
\IT{Previous try level} \IFRU{меняется вместе с тем, как исполнение доходит до очередного \TT{try}-блока, либо
выходит из него}{is changing as execution flow entering or exiting \TT{try} block}.

\lstinputlisting[caption=MSVC 2003]{OS-specific/SEH/2/3_SEH3.asm}

\IFRU{Если установить брякпоинт на ф-цию}{If to set a breakpoint on} \printf{} \IFRU{вызываемую из обработчика,
мы можем увидеть что добавился еще один SEH-обработчик}{function which is called from the handler, 
we may also see how yet another SEH handler is added}.
\IFRU{Наверное, это еще какая-то дополнительная механика скрытая внутри процесса обработки исключений}
{Perhaps, yet another machinery inside of SEH handling process}.
\IFRU{Тут мы также видим}{Here we also see our} \IT{scope table} \IFRU{состояющую из двух элементов}
{consisting of 2 entries}.

\begin{lstlisting}
tracer.exe -l:3.exe bpx=3.exe!printf --dump-seh
\end{lstlisting}

\begin{lstlisting}[caption=tracer.exe output]
(0) 3.exe!printf
EAX=0x0000001b EBX=0x00000000 ECX=0x0040cc58 EDX=0x0008e3c8
ESI=0x00000000 EDI=0x00000000 EBP=0x0018f840 ESP=0x0018f838
EIP=0x004011b6
FLAGS=PF ZF IF
* SEH frame at 0x18f88c prev=0x18fe9c handler=0x771db4ad (ntdll.dll!ExecuteHandler2@20+0x3a)
* SEH frame at 0x18fe9c prev=0x18ff78 handler=0x4012e0 (3.exe!_except_handler3)
SEH3 frame. previous trylevel=1
scopetable entry[0]. previous try level=-1, filter=0x401120 (3.exe!main+0xb0) handler=0x40113b (3.exe!main+0xcb)
scopetable entry[1]. previous try level=0, filter=0x4010e8 (3.exe!main+0x78) handler=0x401100 (3.exe!main+0x90)
* SEH frame at 0x18ff78 prev=0x18ffc4 handler=0x4012e0 (3.exe!_except_handler3)
SEH3 frame. previous trylevel=0
scopetable entry[0]. previous try level=-1, filter=0x40160d (3.exe!mainCRTStartup+0x18d) handler=0x401621 (3.exe!mainCRTStartup+0x1a1)
* SEH frame at 0x18ffc4 prev=0x18ffe4 handler=0x771f71f5 (ntdll.dll!__except_handler4)
SEH4 frame. previous trylevel=0
SEH4 header:	GSCookieOffset=0xfffffffe GSCookieXOROffset=0x0
		EHCookieOffset=0xffffffcc EHCookieXOROffset=0x0
scopetable entry[0]. previous try level=-2, filter=0x771f74d0 (ntdll.dll!___safe_se_handler_table+0x20) handler=0x771f90eb (ntdll.dll!_TppTerminateProcess@4+0x43)
* SEH frame at 0x18ffe4 prev=0xffffffff handler=0x77247428 (ntdll.dll!_FinalExceptionHandler@16)
\end{lstlisting}

\subsubsection{SEH4}

\IFRU{Во время атаки переполнения буфера}{During buffer overflow} (\ref{subsec:bufferoverflow})
\IFRU{адрес}{attack, address of the} \IT{scope table} \IFRU{может быть перезаписан, так что начиная с}
{can be rewritten, so starting at} MSVC 2005, SEH3 \IFRU{был дополнен защитой от переполнения буфера, до SEH4}
{was upgraded to SEH4 in order to have buffer overflow protection}.
\IFRU{Указатель на}{The pointer to} \IT{scope table} \IFRU{теперь}{is now} \glslink{xoring}{\IFRU{про-XOR-ен}{xored}} 
\IFRU{с}{with} \gls{security cookie}.
\IT{Scope table} \IFRU{расширена, теперь имеет заголовок содержающий 2 указателя на}
{extended to have a header, consisting of two pointers to} \IT{security cookies}.
\IFRU{Каждый элемент имеет смешение внутри стека на другое значение: это адрес фрейма (EBP) также}
{Each element have an offset inside of stack of another value: 
this is address of stack frame (EBP)} \glslink{xoring}{\IFRU{про-XOR-еный}{xored}} \IFRU{с}{with} 
\TT{security\_cookie} \IFRU{расположенный в стеке}{as well, placed in the stack}.
\IFRU{Это значение будет прочитано во время обработки исключения и проверено на правильность}
{This value will be read during exception handling and checked, if it is correct}.
\IT{Security cookie} \IFRU{в стеке случайное каждый раз, так что атакующий, как мы надеемся, не мог предсказать его}
{in the stack is random each time, so remote attacker, hopefully, will not be able to predict it}.

\IFRU{Изначальное значение}{Initial} \IT{previous try level} \IFRU{это}{is} $-2$ \InENRU SEH4 \IFRU{вместо}{instead of} $-1$.

\def\SEHfour{1}
\begin{center}
\begin{tikzpicture}[thick,scale=0.85, every node/.style={scale=0.85}]
	\tikzstyle{every path}=[thick]
	\tikzstyle{undefined}=[draw,rectangle,minimum height=1cm, minimum width=3.5cm, text width=3.5cm]
	\tikzstyle{undefinedw}=[draw,rectangle,minimum height=1cm, minimum width=4.5cm, text width=4.5cm]
	\tikzstyle{node}=[draw,rectangle,minimum height=1cm, minimum width=3.5cm, text width=3.5cm, fill=gray!20]
	\tikzstyle{nodew}=[draw,rectangle,minimum height=1cm, minimum width=4.5cm, text width=4.5cm, fill=gray!20]
	\tikzstyle{text_as_rect}=[rectangle,minimum height=1cm, minimum width=3.5cm, text width=3.5cm]
	\tikzstyle{point}=[inner sep=0pt]

	\node[node] (fs) [minimum width=1.5cm, text width=1.5cm] {FS:0};

\begin{scope}[start chain=1 going below, node distance=0cm]
	\node[on chain=1,node] (tib1) [right=1.5cm of fs] {+0: \_\_except\_list};
	\node[on chain=1,undefined] {+4: \dots};
	\node[on chain=1,undefined] {+8: \dots};
\end{scope}
	\node (tib_text) [above of=tib1] {TIB};
	
	\draw [->] (fs.east) -- (tib1.west);

\begin{scope}[start chain=3 going below, node distance=0cm]
	\node[on chain=3,undefined] (u1) [text centered, right=2.5cm of tib1] {\dots};
	%\node[on chain=3,node] (n1prev) {Prev=0xFFFFFFFF (1)};
	\node[on chain=3,node] (n1prev) {Prev=0xFFFFFFFF};
	\node[on chain=3,node] (n1handler) {Handle};
	\node[on chain=3,undefined] [text centered] {\dots};
	\node[on chain=3,node] (n2prev) {Prev};
	\node[on chain=3,node] (n2handler) {Handle};
	\node[on chain=3,undefined] [text centered] {\dots};
	\node[on chain=3,node] (n3prev) {Prev};
	\node[on chain=3,node] (n3handler) {Handle};
	\ifx\SEHfour\undefined	
	\node[on chain=3,node] (n3scopetable) {\ScopeTable};
	\else
	\node[on chain=3,node] (n3scopetable) {\ScopeTable$\oplus$security\_cookie};
	\fi
	\node[on chain=3,node] {\RU{предыдущий уровень try}\EN{previous try level}};
	\node[on chain=3,node] (n3ebp) {EBP};
	\ifx\SEHfour\undefined
	\else
	\node[on chain=3,undefined] [text centered] {\dots};
	\node[on chain=3,node] (AnotherCookie) {EBP$\oplus$security\_cookie};
	\node[on chain=3,undefined] [text centered] {\dots};
	\fi
\end{scope}
	
	\node [node] (n1handler_text) [right=1cm of n1handler] {\HandlerFunction};
	\draw [->] (n1handler.east) -- (n1handler_text.west);
	\node [node] (n2handler_text) [right=1cm of n2handler] {\HandlerFunction};
	\draw [->] (n2handler.east) -- (n2handler_text.west);

	\ifx\SEHfour\undefined	
	\node [node] (n3handler_text) [right=1cm of n3handler] {\_except\_handler3};
	\else
	\node [node] (n3handler_text) [right=1cm of n3handler] {\_except\_handler4};
	\fi

	\draw [->] (n3handler.east) -- (n3handler_text.west);
	\node[undefined] (u4) [text centered, below of=n3ebp] {\dots};
	\node (stack_text) [above of=u1] {Stack};
	
	\node[point] (n2block_pt2) [above left=0cm and 0.8cm of n2prev] {};
	\draw [->] (n3prev.west) .. controls +(left:0.8cm) and (n2block_pt2) .. (n2prev.north west);

	\node[point] (n1block_pt2) [above left=0cm and 0.8cm of n1prev] {};
	\draw [->] (n2prev.west) .. controls +(left:0.8cm) and (n1block_pt2) .. (n1prev.north west);
	
	\node[point] (n3block_pt2) [above left=0cm and 1.25cm of n3prev] {};
	\draw [->] (tib1.east) .. controls +(right:1.25cm) and (n3block_pt2) .. (n3prev.north west);

\begin{scope}[start chain=2 going below, node distance=0cm]
	% SEH3
	\ifx\SEHfour\undefined	
	\node[on chain=2,nodew] (scope_tbl) [left=3.5cm of n2handler] {0xFFFFFFFF (-1)};
	\else
	% SEH4 header
	\node[on chain=2,nodew] (scope_tbl) [left=3.5 cm of n2handler] {\RU{смещение GS Cookie}\EN{GS Cookie Offset}};
	\node[on chain=2,nodew] {\RU{смещение GS Cookie XOR}\EN{GS Cookie XOR Offset}};
	\node[on chain=2,nodew] (EHCookieOffset) {\RU{смещение EH Cookie}\EN{EH Cookie Offset}};
	\node[on chain=2,nodew] {\RU{смещение EH Cookie XOR}\EN{EH Cookie XOR Offset}};
	\node[on chain=2,minimum height=0.3cm] {};
	\node[on chain=2,nodew] {0xFFFFFFFF (-1)};
	\fi

	\node[on chain=2,nodew] (scope_tbl_filter1) {\FilterFunction};
	\node[on chain=2,nodew] {\HandlerFinallyFunction};
	\node[on chain=2,minimum height=0.3cm] {};
	\node[on chain=2,nodew] {0};
	\node[on chain=2,nodew] (scope_tbl_filter2) {\FilterFunction};
	\node[on chain=2,nodew] {\HandlerFinallyFunction};
	\node[on chain=2,minimum height=0.3cm] {};
	\node[on chain=2,nodew] {1};
	\node[on chain=2,nodew] (scope_tbl_filter3) {\FilterFunction};
	\node[on chain=2,nodew] {\HandlerFinallyFunction};
	\node[on chain=2,minimum height=0.3cm] {};
	\node[on chain=2,undefinedw] [text centered] {\dots \MoreEntries \dots};

\end{scope}
	
	\node[text_as_rect,left=0.5cm of scope_tbl_filter1] {\RU{информация для первого блока try/except}\EN{information about first try/except block}};
	\node[text_as_rect,left=0.5cm of scope_tbl_filter2] {\RU{информация для второго блока try/except}\EN{information about second try/except block}};
	\node[text_as_rect,left=0.5cm of scope_tbl_filter3] {\RU{информация для третьего блока try/except}\EN{information about third try/except block}};

	\node (scope_text) [above of=scope_tbl] {\ScopeTable};
	\node[point] (scope_table_pt2) [right=1.75cm of scope_tbl] {};
	\draw [->] (n3scopetable.west) .. controls +(left:1.75cm) and (scope_table_pt2) .. (scope_tbl.north east);

	\ifx\SEHfour\undefined
	\else
	\node[point] (AnotherCookie_pt2) [left=1.75cm of AnotherCookie] {};
	\draw [->] (EHCookieOffset.east) .. controls +(right:1.75cm) and (AnotherCookie_pt2) .. (AnotherCookie.west);
	\fi

\end{tikzpicture}
\end{center}


\IFRU{Оба примера скомпилированные в}{Here is both examples compiled in} MSVC 2012 \IFRU{с}{with} SEH4:

\lstinputlisting[caption=MSVC 2012: one try block example]{OS-specific/SEH/2/2_SEH4.asm}

\lstinputlisting[caption=MSVC 2012: two try blocks example]{OS-specific/SEH/2/3_SEH4.asm}

\IFRU{Вот значение}{Here is a meaning of} \IT{cookies}: \TT{Cookie Offset} 
\IFRU{это разница между адресом записанного в стеке значения EBP и значения}
{is a difference between address of saved EBP value in stack
and the} $EBP \oplus security\_cookie$ \IFRU{в стеке}{value in the stack}.
\TT{Cookie XOR Offset} \IFRU{это дополнительная разница между значением}{is additional difference between} 
$EBP \oplus security\_cookie$ \IFRU{и тем что записано в стеке}{value and what is
stored in the stack}.
\IFRU{Если это уравнение не верно, то процесс остановится из-за разрушения стека}
{If this equation is not true, a process will be stopped due to stack corruption}:

\begin{center}
$security\_cookie \oplus (CookieXOROffset + address of saved EBP) == stack[address of saved EBP + CookieOffset]$
\end{center}

\IFRU{Если}{If} \TT{Cookie Offset} \IFRU{равно}{is} $-2$, \IFRU{это значит что оно не присутствует}{it is not present}.

\IFRU{Проверка \IT{cookies} также реализована в моем}{\IT{Cookies} checking is also implemented in my} \tracer{},
\IFRU{смотрите}{see} \url{https://github.com/dennis714/tracer/blob/master/SEH.c} \IFRU{для деталей}{for details}. 

\IFRU{Возможность переключиться назад на SEH3 все еще присутствует в компиляторах после (и включая) MSVC 2005, нужно включить
опцию \TT{/GS-}, впрочем, \ac{CRT}-код будет продолжать использовать SEH4}
{It is still possible to fall back to SEH3 in the compilers after (and including) MSVC 2005 by setting \TT{/GS-} option,
however, \ac{CRT} code will use SEH4 anyway}.

