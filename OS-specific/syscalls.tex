\section{\IFRU{Системные вызовы (syscall-ы)}{System calls (syscall-s)}}

\index{syscall}

\index{kernel space}
\index{user space}
Как известно, все работающие процессы в \ac{OS} делятся на две категории: имеющие полный доступ
ко всему ``железу'' (kernel space) и не имеющие (user space).

В первой категории ядро \ac{OS} и, обычно, драйвера.

Во второй категории всё прикладное ПО.

Это разделение существует для того, чтобы падающий процесс не мог испортить что-то в других процессах
или даже в самом ядре \ac{OS}.
\index{kernel panic}
\index{BSoD}
С другой стороны, падающий драйвер или ошибка внутри OS обычно приводит к kernel panic или BSoD.

Защита x86-процессора устроена так что возможно разделить всё на 4 слоя защиты (rings), но и в Linux,
и в Windows, используются только 2: ring0 (kernel space) и ring3 (user space).

Системные вызовы (syscall-ы) это место где сходятся вместе оба эти пространства.
Это, можно сказать, самое главное API предоставляемое прикладному ПО.

Работа через syscall-ы популярна у авторов шеллкодов, потому что в шеллкоде обычно бывает трудно определить
адреса нужных ф-ций в системных библиотеках, а syscall-ами проще пользоваться, хотя и придется писать больше
кода. Также нельзя еще забывать, что номера syscall-ов, например, в Windows, могут отличаться от версии к
версии.

\subsection{Linux}

\index{int 0x80}
В Linux вызов syscall-а обычно происходит через \TT{int 0x80}. В регистре \EAX передается номер вызова,
в остальных регистрах ~---- параметры.

\lstinputlisting[caption=Простой пример использования пары syscall-ов]{OS-specific/linux_syscall.s}

Компиляция:

\begin{lstlisting}
nasm -f elf32 1.s
ld 1.o
\end{lstlisting}

Полный список syscall-ов в Linux: \url{http://syscalls.kernelgrok.com/}.

\subsection{Windows}

\index{int 0x2e}
\index{x86!\Instructions!SYSENTER}

Вызов происходит через \TT{int 0x2e} либо используя специальную инструкцию \TT{SYSENTER}.

Полный список syscall-ов в Windows: \url{http://j00ru.vexillium.org/ntapi/}.

\IFRU{Смотрите также}{Further reading}:

\href{http://www.symantec.com/connect/articles/windows-syscall-shellcode}
{``Windows Syscall Shellcode'' by Piotr Bania}.

