\subsection{Win32 PE}

\acs{PE} \IFRU{это формат исполняемых файлов, принятый в Windows}{is a executable file format used in
Windows}.

\IFRU{Разница между .exe, .dll, и .sys в том, что у .exe и .sys обычно нет экспортов, только импорты}
{The difference between .exe, .dll and .sys is that .exe and .sys usually does not have exports, only imports}.

\index{OEP}
\IFRU{У \ac{DLL}, как и у всех PE-файлов, есть точка входа (\ac{OEP})
(там располагается ф-ция DllMain()), но обычно эта ф-ция ничего не делает.}
{A \ac{DLL}, just as any other PE-file, has entry point (\ac{OEP}) (the function DllMain() is located at it) 
but usually this function does nothing.}

.sys \IFRU{это обычно драйвера устройств}{is usually device driver}.

\IFRU{Для драйверов, Windows требует чтобы контрольная сумма в PE-файле была проставлена
и была верной}
{As of drivers, Windows require the checksum is present in PE-file and must be correct}
\footnote{\IFRU{Например}{For example}, Hiew(\ref{Hiew}) \IFRU{умеет её подсчитывать}{can calculate it}}.

\index{Windows!Windows Vista}
\IFRU{А начиная с}{Starting at} Windows Vista, 
\IFRU{PE-файлы-драйвера должны быть также подписаны при помощи электронной подписи, 
иначе они не будут загружаться.}
{driver PE-files must be also signed by digital signature. It will fail to load without signature.}

\index{MS-DOS}
\IFRU{В начале всякого PE-файла есть крохотная DOS-программа,
выводящая на консоль сообщение вроде}{Any PE-file begins with tiny DOS-program, printing a
message like} ``This program cannot be run in DOS mode.'' --- 
\IFRU{если запустить эту программу в DOS либо Windows 3.1, выведется это сообщение}
{if to run this program in DOS or Windows 3.1, this message will be printed}.

\subsubsection{\IFRU{Терминология}{Terminology}}

\begin{itemize}
\item
\IFRU{Модуль}{Module} --- \IFRU{это отдельный файл}{is a separate file}, .exe \OrENRU .dll.

\item
\IFRU{Процесс}{Process} --- \IFRU{это некая загруженная в память и работающая программа}{a program
loaded into memory and running}.
\IFRU{Как правило состоит из одного .exe-файла и массы .dll-файлов}{Commonly consisting of 
one .exe-file and bunch of .dll-files}.

\item
\IFRU{Память процесса}{Process memory} --- \IFRU{память с которой работает процесс}{the memory a process
works with}.
\IFRU{У каждого процесса --- своя}{Each process has its own}.
\IFRU{Там обычно имеются загруженные модули, память стека, кучи (heap), итд}{There can usually be 
loaded modules, memory of the stack, heap(s), etc}.

\item
\index{VA}
\ac{VA} --- \IFRU{это адрес, который будет использоваться в самой программе}{is address which will
be used in program}.

\item
\index{\IFRU{Базовый адрес}{Base address}}
\IFRU{Базовый адрес}{Base address} --- \IFRU{это адрес, по которому модуль будет загружен 
в пространство процесса}{is the address within a process memory at which a module will be loaded}.

\item
\index{RVA}
\ac{RVA} --- \IFRU{это}{is a} \ac{VA}-\IFRU{адрес минус базовый адрес}{address minus base address}.
\IFRU{Многие адреса в таблицах PE-файла используют именно}{Many addresses in PE-file tables
uses exactly} \ac{RVA}-\IFRU{адреса}{addresses}.

\item
\index{IAT}
\ac{IAT} --- \IFRU{самая главная таблица описывающая импорты}{most important table which describes
imports}\footnote{\cite{Pietrek1}}.
\end{itemize}

\subsubsection{\IFRU{Базовый адрес}{Base address}}

\IFRU{Дело в том что несколько авторов модулей могут готовить DLL-файлы для других, и нет возможности договориться о том, какие адреса и кому будут отведены.}
{The fact is that several module authors may prepare DLL-files for others and there is no possibility
to reach agreement, which addresses will be assigned to whose modules.}

\IFRU{Поэтому, если у двух необходимых для загрузки процесса DLL одинаковые базовые адреса,
одна из них будет загружена по этому базовому адресу, 
а вторая ~--- по другому свободному месту в памяти процесса, и все виртуальные адреса
во второй DLL будут скорректированы.}
{So that is why if two necessary for process loading DLLs has the same base addresses,
one of which will be loaded at this base address, and another ~--- at the other spare space in process memory,
and each virtual addresses in the second DLL will be corrected.} \\
\\
\IFRU{Очень часто линкер в}{Often, linker in} \ac{MSVC} \IFRU{генерирует .exe-файлы с базовым адресом}
{generates an .exe-files with the base address} \TT{0x400000},
\IFRU{и с секцией кода начинающейся с}{and with the code section started at} \TT{0x401000}.
\IFRU{Это значит что}{This mean} \ac{RVA} \IFRU{начала секции кода ~---}{of code section begin is} \TT{0x1000}.
\IFRU{А \ac{DLL} часто генерируются этим линкером с базовым адресом}
{DLLs are often generated by this linked with the base address} \TT{0x10000000}
\footnote{\IFRU{Это можно изменять опцией /BASE в линкере}{This can be changed by /BASE linker option}}.

\index{ASLR}
\IFRU{Помимо всего прочего, есть еще одна причина намеренно загружать модули по разным адресам, а точнее, 
по случайным}
{There is also another reason to load modules at different base addresses, rather at random ones}.

\IFRU{Это}{It is} \ac{ASLR}
\footnote{\IFRU{\url{https://ru.wikipedia.org/wiki/Address_Space_Layout_Randomization}}
{\url{https://en.wikipedia.org/wiki/Address_space_layout_randomization}}}.

\index{shellcode}
\IFRU{Дело в том, что некий шеллкод, пытающийся исполниться на зараженной системе, 
должен вызывать какие-то системные ф-ции}{The fact is that a shellcode trying to be executed on a compromised
system must call a system functions}.

\IFRU{И в старых}{In older} \ac{OS} (\IFRU{в линейке Windows NT: до}{in Windows NT line: before} Windows Vista),
\IFRU{системные}{system} DLL (\IFRU{такие как}{like} kernel32.dll, user32.dll) \IFRU{загружались все время
по одним и тем же адресам}{were always loaded at the known addresses}, 
\IFRU{а если еще и вспомнить, что версии этих DLL редко менялись}{and also if to recall
that its versions were rarely changed}, \IFRU{то адреса отдельных
ф-ций, можно сказать, фиксированы и шеллкод может вызывать их напрямую}{an addresses of functions were
fixed and shellcode can call it directly}.

\IFRU{Чтобы избежать этого, методика}{In order to avoid this,} \ac{ASLR}
\IFRU{загружает и вашу программу, и все модули ей необходимые, по случайным адресам, разным при каждом запуске}
{method loads your program and all modules it needs at random base addresses, each time different}.

\subsubsection{Subsystem}

\IFRU{Имеется также поле \IT{subsystem}, обычно это}{There is also \IT{subsystem} field, usually it is}
native (.sys-\IFRU{драйвер}{driver}), 
console (\IFRU{консольное приложение}{console application}) \OrENRU 
\ac{GUI} (\IFRU{не консольное}{non-console}).

\subsubsection{\IFRU{Версия ОС}{OS version}}

\IFRU{В PE-файле имеется минимальный номер версии Windows, необходимый для загрузки модуля.}
{A PE-file also has minimal Windows version needed in order to load it.}
\IFRU{Соответствие номеров версий в файле и кодовых наименований Windows, можно посмотреть}
{The table of version numbers stored in PE-file and corresponding Windows codenames is}
\href{https://en.wikipedia.org/wiki/Windows_NT#Releases}{\IFRU{здесь}{here}}.

\index{Windows!Windows NT4}
\index{Windows!Windows 2000}
\IFRU{Например}{For example}, \ac{MSVC} 2005 \IFRU{еще компилирует .exe-файлы запускающиеся на}{compiles
.exe-files running on} Windows NT4 (\IFRU{версия}{version} 4.00),
\IFRU{а вот}{but} \ac{MSVC} 2008 \IFRU{уже нет}{is not} 
(\IFRU{генерируемые файлы имеют версию}{files generated has version} 5.00, 
\IFRU{для запуска необходима как минимум Windows 2000}{at least Windows 2000 is needed to run them}).

\index{Windows!Windows XP}
\ac{MSVC} 2012 \IFRU{по умолчанию генерирует .exe-файлы версии}{by default generates .exe-files of version} 6.00, 
\IFRU{для запуска нужна как минимум Windows Vista}{targeting at least Windows Vista}, \IFRU{хотя}{however, by} 
\href{http://blogs.msdn.com/b/vcblog/archive/2012/10/08/10357555.aspx}{\IFRU{изменив настройки компиляции}
{by changing compiler's options}}, 
\IFRU{можно заставить генерировать и под Windows XP}{it is possible to force it to compile for Windows XP}.

\subsubsection{\IFRU{Секции}{Sections}}

\IFRU{Разделение на секции присутствует, по видимому, во всех форматах исполняемых файлов}{Division by sections,
as it seems, are present in all execuable file formats}.

\IFRU{Сделано это для того, чтобы отделить код от данных, а данные ~--- от константных данных.}
{It is done in order to separate code from data, and data ~--- from constant data.}

\begin{itemize}
\item
\IFRU{На секции кода будет стоять флаг}{There will be flag} 
\IT{IMAGE\_SCN\_CNT\_CODE} \OrENRU \IT{IMAGE\_SCN\_MEM\_EXECUTE}\IFRU{}{ on code section}
~--- \IFRU{это исполняемый код}{this is executable code}.

\item
\IFRU{На секции данных}{On data section} ~--- \IFRU{флаги }{}\IT{IMAGE\_SCN\_CNT\_INITIALIZED\_DATA}, 
\IT{IMAGE\_SCN\_MEM\_READ} \AndENRU \IT{IMAGE\_SCN\_MEM\_WRITE}\IFRU{}{ flags}.

\item
\IFRU{На пустой секции с неинициализированными данными}{On an empty section with uninitialized data} ~--- 
\IT{IMAGE\_SCN\_CNT\_UNINITIALIZED\_DATA}, \IT{IMAGE\_SCN\_MEM\_READ} \AndENRU \IT{IMAGE\_SCN\_MEM\_WRITE}.

\item
\IFRU{А на секции с константными данными, то есть, защищенными от записи}{On a constant data section, in other words,
protected from writing}, \IFRU{могут быть флаги}
{there are may be flags} \\
\IT{IMAGE\_SCN\_CNT\_INITIALIZED\_DATA} \AndENRU \IT{IMAGE\_SCN\_MEM\_READ} \IFRU{без}{without} \IT{IMAGE\_SCN\_MEM\_WRITE}. 
\IFRU{Если попытаться записать что-то в эту секцию, процесс упадет}{A process will crash if it would try to write to this
section}.
\end{itemize}

\IFRU{В PE-файле можно задавать название для секции, но это не важно}{Each section in PE-file may have a name, however,
it is not very important}.
\IFRU{Часто (но не всегда)}{Often (but not always)} \IFRU{секция кода называется}{code section have the name} \TT{.text}, 
\index{TLS}
\index{BSS}
\IFRU{секция данных}{data section} --- \TT{.data}, \IFRU{константных данных}{constant data section} --- \TT{.rdata} 
\IT{(readable data)}.
\IFRU{Еще популярные имена секций}{Other popular section names are}: 
\TT{.idata} (\IFRU{секция импортов}{imports section}),
\TT{.edata} (\IFRU{секция экспортов}{exports section}),
\TT{.reloc} (\IFRU{секция релоков}{relocs section}),
\TT{.bss} (\IFRU{неинициализированные данные}{uninitialized data (\ac{BSS})}),
\TT{.tls} ~--- thread local storage (\ac{TLS}),
\TT{.rsrc} (\IFRU{ресурсы}{resources}).

\IFRU{Запаковщики/зашифровщики PE-файлов часто затирают имена секций, или меняют на свои}
{PE-file packers/encryptors are often garble section names or replacing names to their own}.

\IFRU{В \ac{MSVC} можно объявлять данные в произвольно названной секции}
{\ac{MSVC} allows to declare data in arbitrarily named section}
\footnote{\url{http://msdn.microsoft.com/en-us/library/windows/desktop/cc307397.aspx}}.

\IFRU{Некоторые компиляторы и линкеры могут добавлять также секцию с отладочными символами 
и вообще отладочной информацией}
{Some compilers and linkers can add a section with debugging symbols and other debugging information}
(\IFRU{например}{for example}, MinGW).
\index{PDB}
\IFRU{Хотя это не так в современных версиях}{However it is not so in modern versions of} \ac{MSVC} 
(\IFRU{там принято отладочную информацию сохранять в отдельных PDB-файлах}
{a separate PDB-files are used there for this purpose}).

\subsubsection{\IFRU{Релоки}{Relocations (relocs)}}

\IFRU{Так же известны как FIXUP-ы}{\ac{AKA} FIXUP-s}.

\IFRU{Это также присутствует почти во всех форматах загружаемых и исполняемых файлов}
{This is also present in almost all executable file formats}
\footnote{\IFRU{Даже .exe-файлы в}{Even .exe-files in} MS-DOS}.

\IFRU{Как видно, модули могут загружаться по другим базовым адресам,
но как же тогда работать с глобальными переменными, например?}
{Obviously, modules are can be loaded on different base addresses, but how to work with global variables,
for example?}
\IFRU{Ведь нужно обращаться к ним по адресу}{They must be accessed by an address}.
\IFRU{Одно из решений это}{One solution is} \PICcode(\ref{sec:PIC}).
\IFRU{Но это далеко не всегда удобно}{But it is not always suitable}.

\IFRU{Поэтому имеется таблица релоков. 
Там просто перечислены адреса мест в модуле подлежащими коррекции при загрузке
по другому базовому адресу.}
{That is why relocations table is present.
The addresses of points needs to be corrected in case of loading on another base address 
are just enumerated in the table.}

\IFRU{Например, по}{For example, there is a global variable at the address}
\TT{0x410000} 
\IFRU{лежит некая глобальная переменная, и вот как обеспечивается её чтение}
{and this is how it is accessed}:

\begin{lstlisting}
A1 00 00 41 00         mov         eax,[000410000]
\end{lstlisting}

\IFRU{Базовый адрес модуля}{Base address of module is} \TT{0x400000},
\IFRU{а }{}\ac{RVA} \IFRU{глобальной переменной}{of global variable is} \TT{0x10000}.

\IFRU{Если загружать модуль по базовому адресу}{If the module is loading on the base address}
\TT{0x500000}, \IFRU{нужно чтобы адрес этой переменной в этой инструкции стал}{the factual address
of the global variable must be} \TT{0x510000}.

\index{x86!\Instructions!MOV}
\IFRU{Как видно, адрес переменной закодирован в самой инструкции}
{As we can see, address of variable is encoded in the instruction} \TT{MOV}, 
\IFRU{после байта}{after the byte} \TT{0xA1}.

\IFRU{Поэтому адрес четырех байт}{That is why address of 4 bytes}, \IFRU{после}{after} \TT{0xA1},
\IFRU{записыватся в таблицу релоков}{is written into relocs table}.

\IFRU{Если модуль загружается по другому базовому адресу},
\IFRU{загрузчик \ac{OS} обходит все адреса в таблице}{\ac{OS}-loader enumerates all addresses in table}, 
\IFRU{находит каждое 32-битное слово по этому адресу}
{finds each 32-bit word the address points on},
\IFRU{отнимает от него настоящий, оригинальный базовый адрес}{subtracts real, original base address of it}
(\IFRU{в итоге получается}{we getting} \ac{RVA}\IFRU{}{ here}),
\IFRU{и прибавляет к нему новый базовый адрес}{and adds new base address to it}.

\IFRU{А если модуль загружается по своему оригинальному базовому адресу, ничего не происходит}
{If module is loading on original base address, nothing happens}.

\IFRU{Так можно обходиться со всеми глобальными переменными}
{All global variables may be treated like that}.

\IFRU{Релоки могут быть разных типов}{Relocs may have various types}, 
\IFRU{однако в Windows для x86-процессоров, тип обычно}
{however, in Windows, for x86 processors, the type is usually} \\
\IT{IMAGE\_REL\_BASED\_HIGHLOW}.

\subsubsection{\IFRU{Экспорты и импорты}{Exports and imports}}

\IFRU{Как известно}{As all we know}, 
\IFRU{любая исполняемая программа должна как-то пользоваться сервисами \ac{OS} и прочими DLL-библиотеками}
{any executable program must use \ac{OS} services and other DLL-libraries somehow}.

\IFRU{Можно сказать, что нужно связывать функции из одного модуля (обычно DLL) и места их вызовов в 
другом модуле (.exe-файл или другая DLL)}
{It can be said, functions from one module (usually DLL) must be connected somehow to a points of their
calls in other module (.exe-file or another DLL)}.

\IFRU{Для этого, у каждой DLL есть ``экспорты'', это таблица ф-ций плюс их адреса в модуле}
{Each DLL has ``exports'' for this, this is table of functions plus its addresses in a module}.

\IFRU{А у .exe-файла, либо DLL, есть ``импорты'', (\ac{IAT}) это таблица ф-ций требующихся для исполнения 
включая список имен DLL-файлов}
{Each .exe-file or DLL has ``imports'', (\ac{IAT}) this is a table of functions it needs for execution including
list of DLL filenames}.

\IFRU{Загрузчик \ac{OS}, после загрузки основного .exe-файла, проходит по \ac{IAT}:
загружает дополнительные DLL-файлы, 
находит имена ф-ций среди экспортов в DLL и прописывает их адреса в \ac{IAT} в головном .exe-модуле}
{After loading main .exe-file, \ac{OS}-loader, processes \ac{IAT}: 
it loads additional DLL-files, finds function names
among DLL exports and writes their addresses down in an \ac{IAT} of main .exe-module}.

\index{Ordinal}
\IFRU{Как видно, во время загрузки, загрузчику нужно много сравнивать одни имена ф-ций с другими, а сравнение строк
это не очень быстрая процедура, так что,
имеется также поддержка ``ординалов'' или 
``hint''-ов, это когда в таблице импортов проставлены номера ф-ций вместо их имен}
{As we can notice, during loading, loader must compare a lot of function names, but strings comparison is not a very
fast procedure, so, there is a support of ``ordinals'' or ``hints'',
that is a function numbers stored in the table instead of their names}.

\IFRU{Так их быстрее находить в загружаемой DLL}
{That is how they can be located faster in loading DLL}.
\IFRU{В таблице экспортов ординалы присутствуют всегда}{Ordinals are always present in ``export'' table}.

\index{MFC}
\IFRU{К примеру}{For example}, \IFRU{программы использующие библиотеки}{program using} 
\ac{MFC}\IFRU{, обычно загружают mfc*.dll по ординалам}{ library usually loads mfc*.dll by ordinals},
\IFRU{и в таких программах, в \ac{IAT}, нет имен ф-ций \ac{MFC}}
{and in such programs there are no \ac{MFC} function names in \ac{IAT}}.

\IFRU{При загрузке такой программы в \IDA, она спросит у вас путь к файлу mfc*.dll,
чтобы установить имена ф-ций}{While loading such program in \IDA, it will asks for a path to mfs*.dll files,
in order to determine function names}.
\IFRU{Если в \IDA не указать путь к этой DLL, то вместо имен ф-ций будет что-то вроде}
{If not to tell \IDA path to this DLL, they will look like}
\IT{mfc80\_123}\IFRU{}{ instead of function names}.

\paragraph{\IFRU{Секция импортов}{Imports section}}

\IFRU{Под \ac{IAT} и всё что с ней связано иногда отводится отдельная секция (с названием вроде \TT{.idata}),
но это не обязательно}
{Often a separate section is allocated for \ac{IAT} table and everything related to it (with name like \TT{.idata}),
however, it is not a strict rule}.

\IFRU{Импорты это запутанная тема еще и из-за терминологической путанницы. Попробуем собрать всё в одно место.}
{Imports is also confusing subject because of terminological mess. Let's try to collect all information in one place.}

\begin{figure}[ht!]
\centering
\includegraphics[scale=0.66]{OS-specific/PE/unnamed0.png}
\caption{\IFRU{схема, объеденяющая все структуры в PE-файлы, связанные с импортами}
{The scheme, uniting all PE-file structures related to imports}}
\end{figure}

\IFRU{Самая главная структура, это массив}{Main structure is the array of} \IT{IMAGE\_IMPORT\_DESCRIPTOR}.
\IFRU{Каждый элемент на каждую импортируемую DLL}{Each element for each DLL being imported}.

\IFRU{У каждого элемента есть}{Each element holds} \ac{RVA}-\IFRU{адрес}{address} 
\IFRU{текстовой строки (имя DLL)}{of text string (DLL name)} (\IT{Name}).

\IT{OriginalFirstThink} \IFRU{это}{is a} \ac{RVA}-\IFRU{адрес}{address} \IFRU{массива}{of array of} 
\ac{RVA}-\IFRU{адресов}{addresses},
\IFRU{каждый из которых указывает на текстовую строку где записано имя ф-ции}
{each of which points to the text string with function name}. 
\IFRU{Каждую строку предваряет 16-битное число}{Each string is prefixed by 16-bit integer} (``hint'') ~--- 
\IFRU{``ординал'' ф-ции}{``ordinal'' of function}.

\IFRU{Если при загрузке удается найти ф-цию по ординалу, тогда сравнение текстовых строк не будет происходить.
Массив оканчивается нулем}{While loading, if it is possible to find function by ordinal,
then strings comparison will not occur. Array is terminated by zero}.
\IFRU{Есть также указатель с названием}
{There is also a pointer with the name} \IT{FirstThunk},
\IFRU{это просто}{it is just} \ac{RVA}-\IFRU{адрес}{address} 
\IFRU{места, где загрузчик будет проставлять адреса найденных ф-ций}{of the place where loader will
write addresses of functions resolved}.

\IFRU{Места где загрузчик проставляет адреса, \IDA именует их так}{The points where
loader writes addresses, \IDA marks like}: \IT{\_\_imp\_CreateFileA}, \IFRU{итд}{etc}.

\IFRU{Есть по крайней мере два способа использовать адреса проставленные загрузчиком}
{There are at least two ways to use addresses written by loader}.

\begin{itemize}
\index{x86!\Instructions!CALL}
\item
\IFRU{В коде будут просто инструкции вроде}{The code will have instructions like} 
\IT{call \_\_imp\_CreateFileA}, 
\IFRU{а так как, поле с адресом импортируемой ф-ции это как бы глобальная переменная}
{and since the field with the address of function imported is a global variable in some sense}, 
\IFRU{то в таблице релоков добавляется адрес (плюс 1 или 2) в инструкции \IT{call}}
{the address of \IT{call} instruction (plus 1 or 2) will be added to relocs table},
\IFRU{на случай если модуль будет загружен по другому базовому адресу}
{for the case if module will be loaded on different base address}.

\IFRU{Но как видно, это приводит к увеличению таблицы релоков}{But, obviously, this may enlarge
relocs table significantly}.
\IFRU{Ведь вызовов импортируемой ф-ции у вас в модуле может быть очень много}
{Because there are might be a lot of calls to imported functions in the module}.
\IFRU{К тому же, чем больше таблица релоков, тем дольше загрузка}
{Furthermore, large relocs table slowing down the process of module loading}.

\index{x86!\Instructions!JMP}
\index{thunk-\IFRU{функции}{functions}}
\item
\IFRU{На каждую импортируемую ф-цию выделяется только один переход на импортируемую ф-цию используя
инструкцию \JMP плюс релок на эту инструкцию}
{For each imported function, there is only one jump allocated, using \JMP instruction 
plus reloc to this instruction}.
\IFRU{Такие места-``переходники'' называются также ``thunk''-ами}{Such points are also called ``thunks''}.
\IFRU{А все вызовы импортируемой ф-ции это просто инструкция \CALL на соответствующий ``thunk''}
{All calls to the imported functions are just \CALL instructions to the corresponding ``thunk''}.
\IFRU{В данном случае, дополнительные релоки не нужны, потому что эти CALL-ы имеют относительный адрес,
и корректировать их не надо}{In this case, additional relocs are not necessary because these CALL-s
has relative addresses, they are not to be corrected}.
\end{itemize}

\IFRU{Оба этих два метода могут комбинироваться}{Both of these methods can be combined}.
\IFRU{Вероятно, линкер создает отдельный ``thunk'', если вызовов слишком много, но по умолчанию --- не создает}
{Apparently, linker creates individual ``thunk'' if there are too many calls to the functions,
but by default it is not to be created}. \\
\\
\IFRU{Кстати, массив адресов ф-ций, на который указывает FirstThunk,
не обязательно может быть в секции \ac{IAT}}{By the way, an array of function addresses to which FirstThunk is
pointing is not necessary to be located in \ac{IAT} section}.
\IFRU{К примеру, я написал утилиту}{For example, I once wrote the}
PE\_add\_import\footnote{\url{http://yurichev.com/PE_add_import.html}} 
\IFRU{для добавления импорта в уже существующий .exe-файл}{utility for adding import to an existing .exe-file}.
\IFRU{На месте ф-ции, вместо которой вы хотите подставить вызов в другую DLL,
моя утилита вписывает такой код}{At the place of the function you want to substitute by call to another DLL,
the following code my utility writes}:

\begin{lstlisting}
MOV EAX, [yourdll.dll!function]
JMP EAX
\end{lstlisting}

\IFRU{При этом, FirstThunk указывает прямо на первую инструкцию.
Иными словами, загрузчик, загружая yourdll.dll, 
прописывает адрес ф-ции \IT{function} прямо в коде.}
{FirstThunk points to the first instruction. In other words, while loading yourdll.dll,
loader writes address of the \IT{function} function right in the code.}

\IFRU{Надо также отметить что обычно секция кода защищена от записи}
{It also worth noting a
code section is usually write-protected}, \IFRU{так что, моя утилита
добавляет флаг}{so my utility adds} \IT{IMAGE\_SCN\_MEM\_WRITE} 
\IFRU{для секции кода. Иначе при загрузке такой программы, она упадет с ошибкой}
{flag for code section. Otherwise, the program will crash while loading with the error code}
5 (access denied). \\
\\
\IFRU{Может возникнуть вопрос: а что если я поставляю программу с набором DLL,
которые никогда не будут меняться, может как-то можно ускорить процесс загрузки?}
{One might ask: what if I supply a program with the DLL files set which are not supposed to change,
is it possible to speed up loading process?}

\IFRU{Да, можно прописать адреса импортируемых ф-ций в массивы FirstThunk зараннее}
{Yes, it is possible to write addresses of the functions to be imported into FirstThunk arrays in advance}.
\IFRU{Для этого в структуре}{The \IT{Timestamp} field is present in the}
\IT{IMAGE\_IMPORT\_DESCRIPTOR} \IFRU{имеется поле \IT{Timestamp}}{structure}.
\IFRU{И если там присутствует какое-то значение, то загрузчик сверяет это значение с датой-временем DLL-файла}
{If a value is present there, then loader compare this value with date-time of the DLL file}.
\IFRU{И если они равны, то загрузчик больше ничего не делает, и загрузка будет происходить быстрее}
{If the values are equal to each other, then the loader is not do anything, and loading process will be faster}.\IFRU{Это называется}{This is what called} ``old-style binding''
\footnote{\url{http://blogs.msdn.com/b/oldnewthing/archive/2010/03/18/9980802.aspx}.
\IFRU{Существует также}{There is also} ``new-style binding'',
\IFRU{про него напишу позже}{I will write about it in future}}.
\index{BIND.EXE}
\IFRU{В Windows SDK для этого имеется утилита BIND.EXE}
{There is the BIND.EXE utility in Windows SDK for this}.
\IFRU{Для ускорения загрузки вашей программы}{For speeding up of loading of your program}, 
Matt Pietrek \InENRU \cite{Pietrek1}, \IFRU{предлагает делать binding сразу после инсталляции
вашей программы на компьютере конечного пользователя}{offers to do binding shortly after your program
installation on the computer of the end user}. \\
\\
\IFRU{Запаковщики/зашифровщики PE-файлов могут также сжимать/шифровать \ac{IAT}}
{PE-files packers/encryptors may also comporess/encrypt \ac{IAT}}.
\IFRU{В этом случае, загрузчик Windows, конечно же, не загрузит все нужные DLL}
{In this case, Windows loader, of course, will not load all necessary DLLs}.
\index{Windows!LoadLibrary}
\index{Windows!GetProcAddress}
\IFRU{Поэтому распаковщик/расшифровщик делает это сам, при помощи вызовов}
{Therefore, packer/encryptor do this on its own, with the help of} 
\IT{LoadLibrary()} \AndENRU \IT{GetProcAddress()}\IFRU{}{ functions}. \\
\\
\IFRU{В стандартных DLL входящих в состав Windows, часто, \ac{IAT} находится в самом начале PE-файла}
{In the standard DLLs from Windows installation, often, \ac{IAT} is located right in the beginning of PE-file}.
\IFRU{Возможно это для оптимизации}{Supposedly, it is done for optimization}.
\IFRU{Ведь .exe-файл при загрузке не загружается в память весь 
(вспомните что инсталляторы огромного размера подозрительно быстро запускаются), он ``мапится'' (map), 
и подгружается в память частями по мере обращения к этой памяти.}
{While loading, .exe file is not loaded into memory as a whole (recall huge install programs which are
started suspiciously fast), it is ``mapped'', and loaded into memory by parts as they are accessed.}
\IFRU{И возможно в Microsoft решили что так будет быстрее}
{Probably, Microsoft developers decided it will be faster}.

\subsubsection{\IFRU{Ресурсы}{Resources}}

\label{PEresources}
\IFRU{Ресурсы в PE-файле это набор иконок, картинок, текстовых строк, описаний диалогов}
{Resources in a PE-file is just a set of icons, pictures, text strings, dialog descriptions}.
\IFRU{Возможно, их в свое время решили отделить от основного кода, чтобы все эти вещи были мультиязычными,
и было проще выбирать текст или картинку того языка, который установлен в \ac{OS}}
{Perhaps, they were separated from the main code, so all these things could be multilingual,
and it would be simpler to pick text or picture for the language that is currently set in \ac{OS}}. \\
\\
\IFRU{В качестве побочного эффекта, их легко редактировать и сохранять обратно в исполняемый файл,
даже не обладая специальными знаниями,
например, редактором}{As a side effect, they can be edited easily and saved back to the executable file,
even, if one does not have special knowledge, for example, using} ResHack\IFRU{}{ editor}(\ref{ResHack}).

\subsubsection{.NET}

\index{.NET}
\IFRU{Программы на .NET компилируются не в машинный код, а в свой собственный байткод}
{.NET programs are compiled not into machine code but into special bytecode}.
\index{OEP}
\IFRU{Собственно, в .exe-файлы байткод вместо обычного кода, однако, точка входа (\ac{OEP}) 
указывает на крохотный фрагмент x86-кода}{Strictly speaking, there is bytecode instead of usual x86-code
in the .exe-file, however, entry point (\ac{OEP}) is pointing to the tiny fragment of x86-code}:

\begin{lstlisting}
jmp         mscoree.dll!_CorExeMain
\end{lstlisting}

\IFRU{А в mscoree.dll и находится .NET-загрузчик, который уже сам будет работать с PE-файлом}
{.NET-loader is located in mscoree.dll, it will process the PE-file}.
\index{Windows!Windows XP}
\IFRU{Так было в \ac{OS} до Windows XP. Начиная с XP, загрузчик \ac{OS} уже сам определяет что это
.NET-файл и запускает его не исполняя этой инструкции \JMP}
{It was so in pre-Windows XP \ac{OS}. Starting from XP, \ac{OS}-loader able to detect the .NET-file
and run it without execution of that \JMP instruction}
\footnote{url{http://msdn.microsoft.com/en-us/library/xh0859k0(v=vs.110).aspx}}.

\index{TLS}
\subsubsection{TLS}

\IFRU{Эта секция содержит в себе инициализированные данные для}{This section holds initialized
data for} \ac{TLS}(\ref{TLS}) (\IFRU{если нужно}{if needed}).
\IFRU{При старте нового треда, его}{When new thread starting, its} 
\ac{TLS}-\IFRU{данные инициализируются данными из этой секции}
{data is initialized by the data from this section}. \\
\\
\index{C++!C++11}
\IFRU{В}{In the} C++11 \IFRU{ввели модификатор}{standard, a new} \IT{thread\_local} 
\IFRU{показывающий что каждый тред будет иметь свою версию этой переменной}
{modifier was added, showing that each thread will have its own version of the variable},
\IFRU{и её можно инициализировать, и она расположена в}{it can be initialized, and it is located in the} \ac{TLS}
\footnote{
\index{C11}
\IFRU{В C11 также есть поддержка тредов, хотя и опциональная}
{C11 also has thread support, optional though}}:

\begin{lstlisting}[caption=C++11]
#include <iostream>
#include <thread>

thread_local int tmp=3;

int main()
{
	std::cout << tmp << std::endl;
};
\end{lstlisting}
\footnote{\IFRU{Компилируется в}{Compiled in} MinGW GCC 4.8.1, \IFRU{но не в}{but not in} MSVC 2012}

\IFRU{В исполняемом файле значение}{In the resulting executable file, the} \IT{tmp} 
\IFRU{будет именно в}{variable will be stored in the} \ac{TLS}.
\\
\index{TLS!Callbacks}
\IFRU{Помимо всего прочего, спецификация PE-файла предусматривает инициализацию}
{Aside from that, PE-file specification also provides initialization of}
\ac{TLS}-\IFRU{секции, т.н.}{section, so called}, TLS callbacks.
\IFRU{Если они присутствуют, то они будут вызваны перед тем как передать управление на главную точку входа}
{If they are present, they will be called before control passing to the main entry point} (\ac{OEP}).
\IFRU{Это широко используется запаковщиками/защифровщиками PE-файлов}
{This is used widely in the PE-file packers/encryptors}.

\subsubsection{\IFRU{Инструменты}{Tools}}

\begin{itemize}
\item
\index{objdump}
\index{cygwin}
objdump (\IFRU{из}{from} cygwin) \IFRU{для вывода всех структур PE-файла}{for dumping all PE-file structures}.

\item
\index{Hiew}
Hiew(\ref{Hiew}) \IFRU{как редактор}{as editor}.

\item
pefile --- Python-\IFRU{библиотека для работы с PE-файлами}{library for PE-file processings}
\footnote{\url{https://code.google.com/p/pefile/}}.

\item
\label{ResHack}
ResHack \acs{AKA} Resource Hacker --- \IFRU{редактор ресурсов}{resources editor}
\footnote{\url{http://www.angusj.com/resourcehacker/}}.
\end{itemize}

\subsubsection{Further reading}

% FIXME: bibliography per chapter or section
\begin{itemize}
\item
Daniel Pistelli --- The .NET File Format \footnote{\url{http://www.codeproject.com/Articles/12585/The-NET-File-Format}}
\end{itemize}

