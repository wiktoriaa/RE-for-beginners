\chapter{Endianness\RU{ (порядок байт)}}
\label{sec:endianness}

\RU{Endianness (порядок байт) это способ представления чисел в памяти}
\EN{Endianness is a way of representing values in memory}.

\section{Big-endian\RU{ (от старшего к младшему)}}

\RU{Число}\EN{A} \TT{0x12345678} \RU{будет представлено в памяти так}\EN{value will be represented in memory as}:

\begin{center}
\begin{tabular}{ | l | l | }
\hline
\cellcolor{blue!25} \RU{адрес в памяти}\EN{address in memory} & \cellcolor{blue!25} \RU{значение байта}\EN{byte value} \\
\hline
+0 & 0x12 \\
\hline
+1 & 0x34 \\
\hline
+2 & 0x56 \\
\hline
+3 & 0x78 \\
\hline
\end{tabular}
\end{center}

\RU{CPU с таким порядком включают в себя}\EN{Big-endian CPUs include} Motorola 68k, IBM POWER.

\section{Little-endian\RU{ (от младшего к старшему)}}

\RU{Число}\EN{A} \TT{0x12345678} \RU{будет представлено в памяти так}\EN{value will be represented in memory as}:

\begin{center}
\begin{tabular}{ | l | l | }
\hline
\cellcolor{blue!25} \RU{адрес в памяти}\EN{address in memory} & \cellcolor{blue!25} \RU{значение байта}\EN{byte value} \\
\hline
+0 & 0x78 \\
\hline
+1 & 0x56 \\
\hline
+2 & 0x34 \\
\hline
+3 & 0x12 \\
\hline
\end{tabular}
\end{center}

\RU{CPU с таким порядком байт включают в себя}\EN{Little-endian CPUs include} Intel x86.

\ifdefined\RUSSIAN
\else
\section{Example}

I've got big-endianness MIPS Linux installed and ready for QEMU
\footnote{I've got it here: \url{https://people.debian.org/~aurel32/qemu/mips/}}.

And I compiled simple example:

\begin{lstlisting}
#include <stdio.h>

int main()
{
	int v, i;

	v=123;

	printf ("%02X %02X %02X %02X\n", 
		*(char*)&v,
		*(((char*)&v)+1),
		*(((char*)&v)+2),
		*(((char*)&v)+3));
};
\end{lstlisting}

And run it:

\begin{lstlisting}
root@debian-mips:~# ./a.out 
00 00 00 7B
\end{lstlisting}

That is.
0x7B is 123 in decimal.
In little-endianness architecture, 7B will be the first byte (you can check on x86 or x86-64), 
but here it is the last one, because highest goes first.

That's why there are separate Linux distributions exist for MIPS 
(``mips'' (big-endian) and ``mipsel'' (little-endian)).
It is impossible for binary compiled for one endianness to work on the OS with different endianness.

\fi

\section{Bi-endian\RU{ (переключаемый порядок)}}

\RU{CPU поддерживающие оба порядка, и его можно переключать, включают в себя}
\EN{CPUs which may switch between endianness are} ARM, PowerPC, SPARC, MIPS, \ac{IA64}, \RU{итд}\EN{etc}.

\section{\RU{Конвертирование}\EN{Converting data}}

\index{TCP/IP}
\RU{Сетевые пакеты TCP/IP используют соглашение big-endian, вот почему программа работающая на little-endian архитектуре
должна конвертировать значения используя ф-ции}
\EN{TCP/IP network data packets use big-endian conventions, so that is why a program working on little-endian architecture
should convert values using} \TT{htonl()} \AndENRU \TT{htons()}\EN{ functions}.

\RU{Порядок байт big-endian в среде TCP/IP также называется}
\EN{In TCP/IP, big-endian is also called} ``network byte order'',
\RU{а}\EN{while} little-endian\EMDASH{}``host byte order''.

\index{x86!\Instructions!BSWAP}
\RU{Инструкция }{The }\TT{BSWAP} \RU{также может использоваться для конвертирования}
\EN{instruction can also be used for conversion}.

