\subsection{Zobrist hashing / tabulation hashing}
\myindex{Chess}
\myindex{Zobrist hashing}
\myindex{Tabulation hashing}

If you work on a chess engine, you traverse a game tree many times per second, and often, you can encounter
the same position, which has already been processed.

So you have to use a method to store already calculated positions somewhere.
But chess position can require a lot of memory, and a hash function would be used instead.

Here is a way to compress a chess position into 64-bit value, called Zobrist hashing:

\begin{lstlisting}[style=customc]
// we have 8*8 board and 12 pieces (6 for white side and 6 for black)

uint64_t table[12][8][8]; // filled with random values

int position[8][8]; // for each square on board. 0 - no piece. 1..12 - piece

uint64_t hash;

for (int row=0; row<8; row++)
	for (int col=0; col<8; col++)
	{
		int piece=position[row][col];

		if (piece!=0)
			hash=hash^table[piece][row][col];
	};

return hash;
\end{lstlisting}

Now the most interesting part: if the next (modified) chess position differs only by one (moved) piece,
you don't need to recalculate hash for the whole position, all you need is:

\begin{lstlisting}[style=customc]
hash=...; // (already calculated)

// subtract information about the old piece:
hash=hash^table[old_piece][old_row][old_col];

// add information about the new piece:
hash=hash^table[new_piece][new_row][new_col];
\end{lstlisting}
