\newcommand{\HashFuncChapterName}{%
	\RU{Хеш-функции}%
	\EN{Hash functions}%
	\ES{Funciones hash}%
	\PTBRph{}%
	\DEph{}\PLph{}%
	\ITAph{}%
}
\chapter{\HashFuncChapterName}
\label{hash_func}

\index{\HashFuncChapterName}
\index{CRC32}
\RU{Простейший пример это CRC32, алгоритм \q{более мощный} чем простая контрольная сумма,
для проверки целостности данных.}%
\EN{A very simple example is CRC32, an algorithm that provides \q{stronger} checksum for integrity checking purposes.}%
\ES{Un ejemplo muy sencillo es CRC32, un algoritmo que provee una \q{fuerte} suma de
comprobaci\'on para prop\'ositos de comprobaci\'on de integridad}%
\PTBRph{}%
\DEph{}\PLph{}%
\ITAph{}
\RU{Невозможно восстановить оригинальный текст из хеша, там просто меньше информации: ведь текст
может быть очень длинным, но результат CRC32 всегда ограничен 32 битами.}%
\EN{it is impossible to restore the original text from the hash value, it has much less information:
the input can be long, but CRC32's result is always limited to 32 bits.}%
\ES{Es imposible reestablecer el texto original a partir de su valor hash, tiene mucho menos informaci\'on:
la entrada puede ser larga, pero el resultado de CRC32 siempre est\'a limitado a 32 bits.}%
\PTBRph{}%
\DEph{}\PLph{}%
\ITAph{}
\RU{Но CRC32 не надежна в криптографическом смысле: известны методы как изменить текст таким образом,
чтобы получить нужный результат.}%
\EN{But CRC32 is not cryptographically secure: it is known how to alter a text in a way that the resulting
CRC32 hash value will be the one we need.}%
\ES{Pero CRC32 no es criptogr\'aficamente seguro: es sabido c\'omo alterar un texto de tal modo que el valor
del hash CRC32 resultante sea el que necesitemos.}%
\PTBRph{}%
\DEph{}\PLph{}%
\ITAph{}
\RU{Криптографические хеш-функции защищены от этого.}%
\EN{Cryptographic hash functions are protected from this.}%
\ES{Las funciones hash criptogr\'aficas est\'an protegidas de esto.}%
\PTBRph{}%
\DEph{}\PLph{}%
\ITAph{}
\\
\\
\index{MD5}
\index{SHA1}
\RU{Такие функции как MD5, SHA1, \etc{}., широко используются для хеширования паролей
для хранения их в базе.}%
\EN{Such functions are MD5, SHA1, \etc{}, and they are widely used to hash user passwords in order to store them in a database.}%
\ES{Tales funciones son MD5, SHA1, \etc{}, y son utilizadas ampliamente para obtener el hash de contrase\~nas de usuarios para
almacenarlas en las bases de datos.}%
\PTBRph{}%
\DEph{}\PLph{}%
\ITAph{}
\RU{Действительно: БД форума в интернете может и не хранить пароли 
(иначе злоумышленник получивший доступ к БД сможет узнать все пароли), а только хеши.}%
\EN{Indeed: an internet forum database may not contain user passwords 
(a stolen database can compromise all user's passwords) but only hashes 
(a cracker can't reveal passwords).}%
\ES{De hecho, la base de datos de un foro en internet no puede contener las contrase\~nas de los usuarios
(una base de datos robado puede comprometer las contrase\~nas de todos los usuarios) sino \'unicamente
hashes (un cracker no puede recuperar las contrase\~nas).}%
\PTBRph{}%
\DEph{}\PLph{}%
\ITAph{}
\RU{К тому же, скрипту интернет-форума вовсе не обязательно знать ваш пароль, он только должен
сверить его хеш с тем что лежит в БД, и дать вам доступ если cверка проходит.}%
\EN{Besides, an internet forum engine is not aware of your password, it has only to check if its hash
is the same as the one in the database, and give you access if they match.}%
\ES{Adem\'as, el motor de un foro de internet no es consciente de tu contrase\~na, s\'olo debe comprobar
si su hash es el mismo que aquel almacenado en la base de datos, y darte acceso si concuerdan.}%
\PTBRph{}%
\DEph{}\PLph{}%
\ITAph{}
\RU{Один из самых простых способов взлома --- это просто перебирать все пароли и ждать пока
результат будет такой же как тот что нам нужен.}%
\EN{One of the simplest password cracking methods is just to try hashing all possible passwords in order
to see which is matching the resulting value that we need.}%
\ES{Uno de los m\'etodos de cracking de contrase\~nas m\'as simple es tratar de obtener los hashes de todas
las contrase\~nas posibles para ver cu\'al concuerda con el resultado que necesitamos.}%
\PTBRph{}%
\DEph{}\PLph{}%
\ITAph{}
\RU{Другие методы намного сложнее.}%
\EN{Other methods are much more complex.}%
\ES{Otros m\'etodos son mucho m\'as complejos.}%
\PTBRph{}%
\DEph{}\PLph{}%
\ITAph{}
% TODO1 add about Rainbow tables

\section{%
	\RU{Как работает односторонняя функция?}%
	\EN{How one-way function works?}%
	\ES{?`C\'omo trabajan las funciones de una v\'ia?}%
	\PTBRph{}%
	\DEph{}\PLph{}%
	\ITAph{}%
}

\RU{Односторонняя функция, это функция, которая способна превратить из одного значения другое,
при этом невозможно (или трудно) проделать обратную операцию.}%
\EN{A one-way function is a function which is able to transform one value into another,
while it is impossible (or very hard) to reverse it back.}%
\ES{Las funciones de una v\'ia son funciones }capces de transformar un valor en otro,
a la vez que es imposible (o muy dif\'icil revertirlo.)%
\PTBRph{}%
\DEph{}\PLph{}%
\ITAph{}
\RU{Некоторые люди имеют трудности с пониманием, как это возможно.}%
\EN{Some people have difficulties while understanding how it's possible at all.}%
\ES{Algunas personas tienen dificultad entendiendo c\'omo puede ser esto posible.}%
\PTBRph{}%
\PLph{}%
\ITAph{}
\RU{Рассмотрим очень простой пример.}%
\EN{Let's consider simple demonstration.}%
\ES{}%
\PTBRph{}%
\DEph{}\PLph{}%
\ITAph{Consideremos una demostraci\'on simple.}

\RU{У нас есть ряд из 10-и чисел в пределах 0..9, каждое встречается один раз, например:}%
\EN{We've got a vector of 10 numbers in range 0..9, each is present only once, for example:}%
\ES{Tenemos un vector de 10 n\'umeros en el rango 0..9, cada una presente una sola vez, por ejemplo:}%
\PTBRph{}%
\DEph{}\PLph{}%
\ITAph{}

\begin{lstlisting}
4 6 0 1 3 5 7 8 9 2
\end{lstlisting}

\RU{Алгоритм простейшей односторонней функции выглядит так:}%
\EN{The algorithm for the simplest possible one-way function is:}%
\ES{El algoritmo de la funci\'on de una v\'ia m\'as simple es:}%
\PTBRph{}%
\DEph{}\PLph{}%
\ITAph{}

\begin{itemize}
\item
	\RU{возьми число на нулевой позиции (у нас это 4);}%
	\EN{take the number at zeroth position (4 in our case);}%
	\ES{toma el n\'umero en la posici\'on cero (4 en nuestro caso);}%
	\PTBRph{}%
	\DEph{}\PLph{}%
	\ITAph{}
\item
	\RU{возьми число на первой позиции (у нас это 6);}%
	\EN{take the number at first position (6 in our case);}%
	\ES{toma el n\'umero en la primera posici\'on (6 en nuestro caso);}%
	\PTBRph{}%
	\DEph{}\PLph{}%
	\ITAph{}
\item
	\RU{обменяй местами числа на позициях 4 и 6.}%
	\EN{swap numbers at positions of 4 and 6.}%
	\ES{intercambia los n\'umeros en las posiciones 4 y 6.}%
	\PTBRph{}%
	\DEph{}\PLph{}%
	\ITAph{}
\end{itemize}

\RU{Отметим числа на позициях 4 и 6:}%
\EN{Let's mark numbers on positions of 4 and 6:}%
\ES{Marquemos los n\'umeros en las posiciones 4 y 6:}%
\PTBRph{}%
\DEph{}\PLph{}%
\ITAph{}

\begin{lstlisting}
4 6 0 1 3 5 7 8 9 2
        ^   ^
\end{lstlisting}

\RU{Меняем их местами и получаем результат:}%
\EN{Let's swap them and we've got the result:}%
\ES{Intercambi\'emolos y tenemos el resultado:}%
\PTBRph{}%
\DEph{}\PLph{}%
\ITAph{}

\begin{lstlisting}
4 6 0 1 7 5 3 8 9 2
\end{lstlisting}

\RU{Глядя на результат, и даже зная алгоритм функции, мы не можем однозначно восстановить изначальное
положение чисел.
Ведь первые два числа могли быть 0 и/или 1, и тогда именно они могли бы участвовать в обмене.}
\EN{While looking at the result, and even if we know algorithm, we can't enumerate unambiguously the initial
number set because the first two numbers could be 0 and/or 1, and then they could participate in the swapping procedure.}
\ES{Mientras vemos el resultado, incluso si conocemos el algoritmo, no podemos enumerar sin ambig\"uedad el conjunto inicial
porque los primeros dos n\'umeros puedieron haber sido 0 y/o 1, y puedieron haber participado en el proceso de intercambio.}%
\PTBRph{}%
\DEph{}\PLph{}%
\ITAph{}

\RU{Это крайне упрощенный пример для демонстрации, настоящие односторонние функции могут быть значительно
сложнее.}%
\EN{This is an utterly simplified example for demonstration. Real one-way functions may be much more
complex.}%
\ES{Este ejemplo fue demasiado simplificado para efectos de desmostraci\'on. Las funciones de una v\'ia reales
pueden llegar a ser muy complejas.}%
\PTBRph{}%
\DEph{}\PLph{}%
\ITAph{}
