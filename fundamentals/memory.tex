\chapter{\RU{Память}\EN{Memory}}

\RU{Есть три основных типа памяти:}
\EN{There are 3 main types of memory:}

\begin{itemize}
\item \RU{Глобальная память}\EN{Global memory}.
\ac{AKA} ``static memory allocation''.
\RU{Нет нужды явно выделять, выделение происходит просто при объявлении переменных/массивов 
глобально.}
\EN{No need to allocate explicitly, allocation is done just by declaring variables/arrays 
globally.}
\RU{Это глобальные переменные расположенные в сегменте данных или констант.}
\EN{This is global variables residing in data or constant segments.}
\RU{Доступны глобально (поэтому считаются \glslink{anti-pattern}{анти-паттерном}).}
\EN{Available globally (hence, considered as \gls{anti-pattern}).}
\RU{Не удобны для буферов/массивов, потому что должны иметь фиксированный размер.}
\EN{Not convenient for buffers/arrays, because must have fixed size.}
\RU{Переполнения буфера случающиеся здесь, обычно перезаписывают переменные или буферы
расположеные рядом в памяти.}
\EN{Buffer overflows occuring here usually overwriting variable or buffer residing next in memory.}
\RU{Пример в этой книге}\EN{Example in this book}: \ref{scanf_global_variable}.

\item \RU{Стек}\EN{Stack}.
\ac{AKA} ``allocate on stack''.
\RU{Выделение происходит просто при объявлении переменных/массивов локально в ф-ции.}
\EN{Allocation is done just by declaring variables/arrays locally in the function.}
\RU{Обычно это локальные для ф-ции переменные}\EN{This is usually local to function variables}.
\RU{Иногда эти локальные переменные также доступны и для нисходящих ф-ций (если ф-ция передает
указатель на переменную в следующую ф-цию).}
\EN{Sometimes these local variable are also available to descending functions (if one passing
pointer to variable to the function to be executed).}
\RU{Выделение и освобождение очень быстрое, достаточно просто сдвига \ac{SP}.}
\EN{Allocation and deallocation are very fast, \ac{SP} only needs to be shifted.}
\index{\CStandardLibrary!alloca()}
\EN{But also not convenient for buffers/arrays, because buffer size should be fixed at some length,
unless \TT{alloca()} (\ref{alloca}) (or variable-length array) is used.}
\RU{Но так же не удобно для буферов/массивов, потому что размер буфера фиксирован,
если только не используется \TT{alloca()} (\ref{alloca}) (или массив с переменной длиной).}
\RU{Переполнение буфера обычно перезаписывает важные структуры стека}\EN{Buffer overflow usually 
overwrites important stack structures}: \ref{subsec:bufferoverflow}.

\index{\CStandardLibrary!malloc()}
\index{\CStandardLibrary!free()}
\item \RU{Куча (\IT{heap}}\EN{Heap}. 
\ac{AKA} ``dynamic memory allocation''.
\RU{Выделение происходит при помощи вызова}\EN{Allocation is done by calling} 
\TT{malloc()/free()} \OrENRU \TT{new/delete} \InENRU \Cpp.
\RU{Самый удобный метод: размер блока может быть задан во время исполнения}\EN{Most convenient 
method: block size may be set in runtime}.
\index{\CStandardLibrary!realloc()}
\RU{Изменение размера возможно (при помощи \TT{realloc()}), но может быть медленным.}
\EN{Resizing is possible  (using \TT{realloc()}), but may be slow.}
\RU{Это самый медленный метод выделения памяти: аллокатор памяти должен поддерживать и обновлять
все управляющие структуры во время выделения и освобождения.}
\EN{This is slowest way to allocate memory: 
memory allocator must support and update all control structures while
allocating and deallocating.}
\RU{Переполнение буфера обычно перезаписывает все эти структуры}\EN{Buffer overflows are usually 
overwrites these structures}.
\RU{Выделения в куче также ведут к проблеме утечек памяти: каждый выделенный блок должен быть
явно освобожден, но кто-то может забыть об этом, или делать это неправильно.}
\EN{Heap allocations is also source of memory leak problem: each memory block should be deallocated
explicitly, but one may forgot about it, or do it incorrectly.}
\index{\CStandardLibrary!free()}
\RU{Еще одна проблема это ``использовать после освобождения'' --- использовать блок памяти после
того как \TT{free()} был вызван на нем, это тоже очень опасно.}
\EN{Another problem is ``use after free''---using a memory block after \TT{free()} was called on it,
which is very dangerous.}
\RU{Пример в этой книге}\EN{Example in this book}: \ref{struct_malloc_example}.

\end{itemize}
