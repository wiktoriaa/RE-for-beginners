\mysection{Comptage de population}
\label{POPCNT}

L'instruction \INS{POPCNT} est \q{population count} (comptage de population) (\ac{AKA}
poids de Hamming). Elle compte simplement les nombres de bits mis dans une valeur
d'entrée.

Par effet de bord, l'instruction \INS{POPCNT} (ou opération) peut-être utilisée pour
déterminer si la valeur est de la forme $2^n$. Puisqu'un nombre de la forme $2^n$
a un seul bit à 1, le résultat de \INS{POPCNT} (ou opération) sera toujours 1.

\myindex{base64scanner}
Par exemple, j'ai écrit une fois un scanner de chaînes en base64 pour chercher des
choses intéressantes dans les fichiers binaires\footnote{\url{https://github.com/DennisYurichev/base64scanner}}.
Et il y  beaucoup de déchets et de faux-positifs, donc j'ai ajouté une option pour
filtrer les blocs de données ayant une taille de $2^n$ octets (i.e., 256 octets,
512, 1024, etc.).
La taille du bloc est testée simplement comme ceci:

\begin{lstlisting}[style=customc]
if (popcnt(size)==1)
	// OK
...
\end{lstlisting}

Cette instruction est aussi connue en tant qu'\q{instruction \ac{NSA}} à cause de
rumeurs:

% Cryptographie appliquée : protocoles, algorithmes et codes source en C / Bruce Schneier ; trad. de Laurent Viennot
% RDRL registres à décalage à rétroaction linéaire, traduction de LFSR
\begin{framed}
\begin{quotation}
Cette branche de la cryptographie croît très rapidement et est très influencée politiquement.
La plupart des conceptions sont secrètes ; la majorité des systèmes de chiffrement
militaire utilisés aujourd'hui est basée sur les RDRL. De fait, la plupart des ordinateurs
CRAY (Cray 1, Cray X-MP, Cray Y-MP) possèdent une instruction curieuse du
nom de \q{comptage de la population}.
Elle compte les 1 dans un registre et peut être
utilisée à la fois pour calculer efficacement la distance de Hamming entre deux mots
binaires et pour réaliser une version vectorielle d'un RDRL. Certains la nomment
l'instruction canonique de la NSA, elle est demandée sur presque tous les contrats
d'ordinateur.
\end{quotation}
\end{framed}
\InSqBrackets{\Schneier{}}\footnote{NDT: traduit en français par Laurent Viennot}\\
Traduction française: \InSqBrackets{Cryptographie appliquée : protocoles, algorithmes
et codes source en C / Bruce Schneier ; traduction de Laurent Viennot}\footnote{La
traduction de la citation est extraite de ce livre.}

