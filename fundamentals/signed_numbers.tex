\chapter{\SignedNumbersSectionName}
\label{sec:signednumbers}
\index{Signed numbers}

\newcommand{\URLS}{\href{http://go.yurichev.com/17117}{wikipedia}}

\RU{Методов представления чисел с знаком \q{плюс} или \q{минус} несколько\footnote{\URLS}, 
но в компьютерах обычно применяется метод \q{дополнительный код} или \q{two's complement}.}
\EN{There are several methods for representing signed numbers\footnote{\URLS}, 
but \q{two's complement} is the most popular one in computers.}

\EN{Here is a table for some byte values:}
\RU{Вот таблица некоторые значений байтов:}

\begin{center}
\begin{tabular}{ | l | l | l | l | }
\hline
\cellcolor{blue!25} \RU{двоичное}\EN{binary} & 
\cellcolor{blue!25} \RU{шестнадцатеричное}\EN{hexadecimal} & 
\cellcolor{blue!25} \RU{беззнаковое}\EN{unsigned} &
\cellcolor{blue!25} \RU{знаковое}\EN{signed} (\RU{дополнительный код}\EN{2's complement}) \\
\hline
01111111 & 0x7f & 127 & 127 \\
\hline
01111110 & 0x7e & 126 & 126 \\
\hline
\multicolumn{4}{ |c| }{...} \\
\hline
00000110 & 0x6 & 6 & 6 \\
\hline
00000101 & 0x5 & 5 & 5 \\
\hline
00000100 & 0x4 & 4 & 4 \\
\hline
00000011 & 0x3 & 3 & 3 \\
\hline
00000010 & 0x2 & 2 & 2 \\
\hline
00000001 & 0x1 & 1 & 1 \\
\hline
00000000 & 0x0 & 0 & 0 \\
\hline
11111111 & 0xff & 255 & -1 \\
\hline
11111110 & 0xfe & 254 & -2 \\
\hline
11111101 & 0xfd & 253 & -3 \\
\hline
11111100 & 0xfc & 252 & -4 \\
\hline
11111011 & 0xfb & 251 & -5 \\
\hline
11111010 & 0xfa & 250 & -6 \\
\hline
\multicolumn{4}{ |c| }{...} \\
\hline
10000010 & 0x82 & 130 & -126 \\
\hline
10000001 & 0x81 & 129 & -127 \\
\hline
10000000 & 0x80 & 128 & -128 \\
\hline
\end{tabular}
\end{center}

\index{x86!\Instructions!JA}
\index{x86!\Instructions!JB}
\index{x86!\Instructions!JL}
\index{x86!\Instructions!JG}
\RU{Разница в подходе к знаковым/беззнаковым числам, собственно, нужна потому что, например, 
если представить \TT{0xFFFFFFFE} и \TT{0x0000002} как беззнаковое, то первое число (4294967294) больше второго (2). 
Если их оба представить как знаковые, то первое будет $-2$, которое, разумеется, меньше чем второе (2).
Вот почему инструкции для условных переходов~(\myref{sec:Jcc}) представлены в обоих версиях ~--- 
и для знаковых сравнений (например, \JG, \JL) и для беззнаковых (\JA, \JB).}
\EN{The difference between signed and unsigned numbers is that if we represent \TT{0xFFFFFFFE} and \TT{0x0000002} 
as unsigned, then the first number (4294967294) is bigger than the second one (2). 
If we represent them both as signed, the first one is to be $-2$, and it is smaller than the second (2). 
That is the reason why conditional jumps~(\myref{sec:Jcc}) are present both for signed (e.g. \JG, \JL) 
and unsigned (\JA, \JB) operations.}\PTBRph{}\ESph{}\PLph{}\ITAph{} \\
\\
\RU{Для простоты, вот что нужно знать}\EN{For the sake of simplicity, that is what one need to know}:
\begin{itemize}
\item \RU{Числа бывают знаковые и беззнаковые}\EN{Numbers can be signed or unsigned}.

\item \RU{Знаковые типы в \CCpp}\EN{\CCpp signed types}:
  \begin{itemize}
    \item \TT{int64\_t} (-9,223,372,036,854,775,808..9,223,372,036,854,775,807) (-~9.2..~9.2 \EN{quintillions}\RU{квинтиллионов}) \OrENRU \\
                \TT{0x8000000000000000..0x7FFFFFFFFFFFFFFF}),
    \item \Tint (-2,147,483,648..2,147,483,647 (-~2.15..~2.15Gb) \OrENRU \TT{0x80000000..0x7FFFFFFF}),
    \item \Tchar (-128..127 \OrENRU \TT{0x80..0x7F}),
    \item \TT{ssize\_t}.
   \end{itemize}

	\RU{Беззнаковые}\EN{Unsigned}: 
  \begin{itemize}
	  \item \TT{uint64\_t} (0..18,446,744,073,709,551,615 (~18 \EN{quintillions}\RU{квинтиллионов}) \OrENRU \TT{0..0xFFFFFFFFFFFFFFFF}),
   \item \TT{unsigned int} (0..4,294,967,295 (~4.3Gb) \OrENRU \TT{0..0xFFFFFFFF}),
   \item \TT{unsigned char} (0..255 \OrENRU \TT{0..0xFF}), 
   \item \TT{size\_t}.
  \end{itemize}

\item \RU{У знаковых чисел знак определяется самым старшим битом: 1 означает \q{минус}, 0 означает \q{плюс}}
	\EN{Signed types have the sign in the most significant bit: 1 mean \q{minus}, 0 mean \q{plus}}.

\item \EN{Promoting to a larger data types is simple:}
      \RU{Преобразование в б\'{о}льшие типы данных обходится легко:} \myref{subsec:sign_extending_32_to_64}.

\label{sec:signednumbers:negation}
\item \EN{Negation is simple: just invert all bits and add 1.}
\RU{Изменить знак легко: просто инвертируйте все биты и прибавьте 1.}
\EN{We can remember that a number of inverse sign is located 
on the opposite side at the same proximity from zero.}
\RU{Мы можем заметить, что число другого знака находится на другой стороне на том же расстоянии от нуля.}
\RU{Прибавление единицы необходимо из-за присутствия нуля посредине.}
\EN{The addition of one is needed because zero is present in the middle.}

\index{x86!\Instructions!IDIV}
\index{x86!\Instructions!DIV}
\index{x86!\Instructions!IMUL}
\index{x86!\Instructions!MUL}
\index{x86!\Instructions!CBW}
\index{x86!\Instructions!CWD}
\index{x86!\Instructions!CWDE}
\index{x86!\Instructions!CDQ}
\index{x86!\Instructions!CDQE}
\index{x86!\Instructions!MOVSX}
\index{x86!\Instructions!SAR}
\item \RU{Инструкции сложения и вычитания работают одинаково хорошо и для знаковых и для беззнаковых значений}
	\EN{The addition and subtraction operations work well for both signed and unsigned values}.
	\RU{Но для операций умножения и деления, в x86 имеются разные инструкции}
	\EN{But for multiplication and division operations, x86 has different instructions}:
	\TT{IDIV}/\TT{IMUL} \RU{для знаковых}\EN{for signed}
	\AndENRU \TT{DIV}/\TT{MUL} \RU{для беззнаковых}\EN{for unsigned}.
\ifx\LITE\undefined
\item \RU{Еще инструкции работающие с знаковыми числами}\EN{Here are some more instructions that work with signed numbers}:
	\TT{CBW/CWD/CWDE/CDQ/CDQE} (\myref{ins:CBW_CWD_etc}), \TT{MOVSX} (\myref{MOVSX}), \TT{SAR} (\myref{ins:SAR}).
\fi
\end{itemize}

\iffalse
% TODO rework!
\section{\RU{Переполнение integer}\EN{Integer overflow}}

\RU{Бывает так, что ошибки представления знаковых/беззнаковых могут привести к уязвимости 
\IT{переполнение integer}.}
\EN{It is worth noting that the incorrect representation of a number can lead to integer overflow vulnerabilities.}

\RU{Например, есть некий сервис, который принимает по сети некие пакеты. 
В пакете есть заголовок где указана длина пакета. Это 32-битное значение. 
В процессе приема пакета, 
сервис проверяет это значение и сверяет, больше ли оно чем максимальный размер пакета, скажем, константа
\TT{MAX\_PACKET\_SIZE} (например, 10 килобайт), и если да, то пакет отвергается как некорректный. 
Сравнение знаковое. Злоумышленник подставляет значение \TT{0xFFFFFFFF}. Это число трактуется как знаковое $-1$ 
и оно меньше чем 10000. Проверка проходит. Продолжаем дальше и копируем этот пакет куда-нибудь себе 
в сегмент данных\dots вызов функции \TT{memcpy (dst, src, 0xFFFFFFFF)} скорее всего, 
затрет много чего внутри процесса.}
\EN{For example, we have a network service and it receives network packets. 
In the packets there is a field where the subpacket length is encoded. 
It is a 32-bit value. 
After the receiving of the network packet, the service checks the field and if it is larger than 
some \TT{MAX\_PACKET\_SIZE} (let's say, 10 kilobytes), the packet is rejected as incorrect.
The comparison is signed. The intruder set this value to \TT{0xFFFFFFFF}.
While comparing, this number is considered as signed $-1$ and it is less than 10 kilobytes. 
No error here. 
The service would then like to copy the subpacket to another place in memory and calls the 
\TT{memcpy (dst, src, 0xFFFFFFFF)} function: this operation, rapidly garbles a lot of 
of process memory.}

\RU{Немного подробнее}\EN{More about it}: \cite{Phrack3C0A}.
% TODO example!
\fi
