\chapter*{\IFRU{Послесловие}{Afterword}}
\addcontentsline{toc}{chapter}{\IFRU{Послесловие}{Afterword}}

\section{\IFRU{Поддержите автора}{Donate}}

\IFRU{Эта книга является свободной, находится в свободном доступе, и доступна в виде исходных кодов}
{This book is free, available freely and available in source code form}\footnote{\url{https://github.com/dennis714/RE-for-beginners}} (LaTeX), 
\IFRU{и всегда будет оставаться таковой}{and it will be so forever}.

\IFRU{Если вы хотите поддержать мою работу, чтобы я мог продолжать регулярно дополнять её и далее, 
вы можете рассмотреть идею пожертвования}{If you want to support my work, so that I can continue to add things
to it regularly, you may consider donation}.

\newcommand{\bitcoinfn}{\footnote{\url{\IFRU{http://ru.wikipedia.org/wiki/Bitcoin}{http://en.wikipedia.org/wiki/Bitcoin}}}}

\IFRU{Вы можете сделать небольшое (или большое) пожертвование на адрес в bitcoin}
{You may donate by sending small (or not small) donation to bitcoin}\bitcoinfn \\
\TT{1HRGTRdFNH1cE81zxWQg6jTtkLzAiGU9Lp}

\begin{figure}[ht!]
\centering
\includegraphics[scale=0.66]{donate_bitcoin.png}
\caption{\IFRU{Счет в bitcoin}{bitcoin address}}
\end{figure}

\IFRU{С другими способами пожертвований можно ознакомиться на странице}
{Other ways to donate are available on the page:} \url{http://yurichev.com/donate.html}

\IFRU{Основные благотворители будут упомянуты прямо здесь}{Major benefactors will be mentioned right here}.

\section{\IFRU{Вопросы?}{Questions?}}

\IFRU{Совершенно по любым вопросам, вы можете не раздумывая писать автору}
{Do not hesitate to mail any questions to the author}: \TT{<\EMAIL>}
 
\IFRU{Пожалуйста, присылайте мне информацию о замеченных ошибках 
(включая грамматические), итд.}
{Please, also do not hesitate to send me any corrections 
(including grammar ones (you see how horrible my English is?)), etc.}
