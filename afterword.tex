\part*{\RU{Послесловие}\EN{Afterword}}
\addcontentsline{toc}{part}{\RU{Послесловие}\EN{Afterword}}

\chapter{\RU{Вопросы?}\EN{Questions?}}

\RU{Совершенно по любым вопросам вы можете не раздумывая писать автору}%
\EN{Do not hesitate to mail any questions to the author}: \TT{<\EMAIL>}

\EN{Any suggestions what also should be added to my book?}%
\RU{Есть идеи о том, что ещё можно добавить в эту книгу?}
 
\RU{Пожалуйста, присылайте мне информацию о замеченных ошибках (включая грамматические),}
\EN{Please, do not hesitate to send me any corrections (including grammar (you see how horrible my English is?)),}\etc.\\
\\
\RU{Автор много работает над книгой, поэтому номера страниц, листингов, \etc. очень часто меняются.}%
\EN{The author is working on the book a lot, so the page and listing numbers, \etc. are changing very rapidly.}
\RU{Пожалуйста, в своих письмах мне не ссылайтесь на номера страниц и листингов.}%
\EN{Please, do not refer to page and listing numbers in your emails to me.}
\RU{Есть метод проще: сделайте скриншот страницы, затем в графическом редакторе подчеркните место, где вы видите
ошибку, и отправьте автору. Так он может исправить её намного быстрее.}%
\EN{There is a much simpler method: make a screenshot of the page, in a graphics editor underline the place where you see the error,
and send it to me. He'll fix it much faster.}
\RU{Ну а если вы знакомы с git и \LaTeX, вы можете исправить ошибку прямо в исходных текстах:}\EN{And if you familiar with git and \LaTeX\, you can fix the error right in the source code:}\\
\href{http://go.yurichev.com/17089}{GitHub}.\\
\\
\EN{Do not worry to bother me while writing me about any petty mistakes you found, even if you are not very confident.
I'm writing for beginners, after all, so beginners' opinions and comments are crucial for my job.}
\RU{Не бойтесь побеспокоить меня написав мне о какой-то мелкой ошибке, даже если вы не очень уверены.
Я всё-таки пишу для начинающих, поэтому мнение и коментарии именно начинающих очень важны для моей работы.}

