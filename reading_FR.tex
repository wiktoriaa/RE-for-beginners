% TODO resync with EN version
\chapter{Livres/blogs qui valent le détour}

\section{Livres et autres matériels}

\subsection{Rétro-ingénierie}

\begin{itemize}
\item Eldad Eilam, \IT{Reversing: Secrets of Reverse Engineering}, (2005)

\item Bruce Dang, Alexandre Gazet, Elias Bachaalany, Sebastien Josse, \IT{Practical Reverse Engineering: x86, x64, ARM, Windows Kernel, Reversing Tools, and Obfuscation}, (2014)

\item Michael Sikorski, Andrew Honig, \IT{Practical Malware Analysis: The Hands-On Guide to Dissecting Malicious Software}, (2012)

\item Chris Eagle, \IT{IDA Pro Book}, (2011)
\end{itemize}


\subsection{Windows}

\begin{itemize}
\item \Russinovich
\end{itemize}

\EN{Blogs}\ES{Blogs}\RU{Блоги}\FR{Blogs}\DE{Blogs}:

\begin{itemize}
\item \href{http://go.yurichev.com/17025}{Microsoft: Raymond Chen}
\item \href{http://go.yurichev.com/17026}{nynaeve.net}
\end{itemize}



\subsection{\CCpp}

\input{CCppBooks}

\subsection{Architecture x86 / x86-64}

\label{x86_manuals}
\begin{itemize}
\item Manuels Intel\footnote{\AlsoAvailableAs \url{http://www.intel.com/content/www/us/en/processors/architectures-software-developer-manuals.html}}

\item Manuels AMD\footnote{\AlsoAvailableAs \url{http://developer.amd.com/resources/developer-guides-manuals/}}

\item \AgnerFog{}\footnote{\AlsoAvailableAs \url{http://agner.org/optimize/microarchitecture.pdf}}

\item \AgnerFogCC{}\footnote{\AlsoAvailableAs \url{http://www.agner.org/optimize/calling_conventions.pdf}}

\item \IntelOptimization

\item \AMDOptimization
\end{itemize}

Quelque peu vieux, mais toujours intéressant à lire :

\MAbrash\footnote{\AlsoAvailableAs \url{https://github.com/jagregory/abrash-black-book}}
(il est connu pour son travail sur l'optimisation bas niveau pour des projets tels que Windows NT 3.1 et id Quake).

\subsection{ARM}

\begin{itemize}
\item Manuels ARM\footnote{\AlsoAvailableAs \url{http://infocenter.arm.com/help/index.jsp?topic=/com.arm.doc.subset.architecture.reference/index.html}}

\item \ARMSevenRef

\item \ARMSixFourRefURL

\item \ARMCookBook\footnote{\AlsoAvailableAs \url{http://go.yurichev.com/17273}}
\end{itemize}

\subsection{Java}

\JavaBook.

\subsection{UNIX}

\TAOUP

% subsection:
\subsection{\EN{Cryptography}\ES{Criptograf\'ia}\ITA{Crittografia}\RU{Криптография}\FR{Cryptographie}\DE{Kryptografie}}
\label{crypto_books}

\begin{itemize}
\item \Schneier{}

\item (Free) lvh, \IT{Crypto 101}\footnote{\AlsoAvailableAs \url{https://www.crypto101.io/}}

\item (Free) Dan Boneh, Victor Shoup, \IT{A Graduate Course in Applied Cryptography}\footnote{\AlsoAvailableAs \url{https://crypto.stanford.edu/~dabo/cryptobook/}}.
\end{itemize}



\section{Autre}

\HenryWarren.
% TODO! shouldn't be here!
Il existe deux excellents subreddits liés à la \ac{RE} (rétro-ingénierie) sur reddit.com :
\href{http://go.yurichev.com/17027}{reddit.com/r/ReverseEngineering/} et
\href{http://go.yurichev.com/17028}{reddit.com/r/remath}
(en lien avec la liaison de la \ac{RE} et des mathématiques).

Il y a également une section sur l'\ac{RE} sur le site Stack Exchange :

\par \href{http://go.yurichev.com/17029}{reverseengineering.stackexchange.com}.

Sur IRC, il y a un channel dédié au \#\#\'reverse\' sur
FreeNode\footnote{\href{http://go.yurichev.com/17030}{freenode.net}}.

