\chapter{\IFRU{Паттерны компиляторов}{Compiler's patterns}}

\IFRU
{Когда я учил Си, а затем Си++, я просто писал небольшие фрагменты кода, компилировал и смотрел что 
получилось на ассемблере. Так намного проще было понять. Я делал это такое количество раз, 
что связь между кодом на \CCpp и тем что генерирует компилятор вбилась мне в подсознание достаточно 
глубоко, поэтому я могу глядя на код на ассемблере сразу понимать, в общих чертах, что там было написано 
на Си. Возможно это поможет кому-то еще, попробую описать некоторые примеры.}
{When I first learned C and then C++ I was just writing small pieces of code, compiling it, 
and seeing what 
was produced in assembly language. That was easy for me. I did it many times and the relation 
between \CCpp code and what the compiler produced was imprinted in my mind so deep that 
I can quickly understand what was in C code when I look at produced x86 code. 
Perhaps this method may be helpful for someone else so I will try to describe some examples here.}

\section{\HelloWorldSectionName}
\label{sec:helloworld}

\IFRU{Начнем с знаменитого примера из книги}{Let's start with that famous example from the book}
``The C programming Language''\cite{Kernighan:1988:CPL:576122}:

\lstinputlisting{01_helloworld/1_1.c}

\subsection{x86}

\subsubsection{MSVC}

\IFRU{Компилируем в}{Let's compile it in} MSVC 2010: \TT{cl 1.cpp /Fa1.asm}

\IFRU
{(Ключ /Fa означает сгенерировать листинг на ассемблере)}
{(/Fa option mean generate assembly listing file)}

\begin{lstlisting}[caption=MSVC 2010]
CONST	SEGMENT
$SG3830	DB	'hello, world', 00H
CONST	ENDS
PUBLIC	_main
EXTRN	_printf:PROC
; Function compile flags: /Odtp
_TEXT	SEGMENT
_main	PROC
	push	ebp
	mov	ebp, esp
	push	OFFSET $SG3830
	call	_printf
	add	esp, 4
	xor	eax, eax
	pop	ebp
	ret	0
_main	ENDP
_TEXT	ENDS
\end{lstlisting}

\IFRU{MSVC выдает листинки в Intel-овском синтаксисе.}{MSVC produces assembly listings in Intel-syntax.} 
\IFRU{Разница между Intel-синтаксисом и AT\&T будет рассмотрена немного позже.}{The difference between 
Intel-syntax and AT\&T-syntax will be discussed below.}

\IFRU{Компилятор сгенерировал файл \TT{1.obj}, который впоследствии будет слинкован линкером в \TT{1.exe}.} 
{Compiler generated \TT{1.obj} file which will be linked into \TT{1.exe}.}

\IFRU{В нашем случае, этот файл состоит из двух сегментов: \TT{CONST} (для данных-констант) и \TT{\_TEXT} (для кода).}
{In our case, the file contain two segments: \TT{CONST} (for data constants) and \TT{\_TEXT} (for code).} 

\index{\CLanguageElements!const}
\IFRU{Строка \TT{``hello, world''} в \CCpp имеет тип \TT{const char*}, однако не имеет имени.}
{The string \TT{``hello, world''} in \CCpp has type \TT{const char*}, however hasn't its own name.}

\IFRU{Но компилятору нужно как-то с ней работать, так что он дает ей внутреннее имя \TT{\$SG3830}.}
{But compiler need to work with the string somehow, so it define internal name \TT{\$SG3830} for it.}

\IFRU{Как видно, строка заканчивается нулевым байтом ~--- это требования стандарта \CCpp для строк.}
{As we can see, the string is terminated by zero byte ~--- it's \CCpp standard for strings.}

\IFRU{В сегменте кода \TT{\_TEXT} находится пока только одна функция ~--- \main.}
{In the code segment \TT{\_TEXT} there are only one function so far ~--- \main.}

\IFRU{Функция \main, как и практически все функции, начинается с пролога и заканчивается эпилогом.}
{Function \main starting with prologue code and ending with epilogue code, like almost any function.}

\IFRU{Об этом смотрите подробнее в разделе о прологе и эпилоге функции}
{Read more about it in section about function prolog and epilog}
~\ref{sec:prologepilog}.

\index{x86!\Instructions!CALL}
\IFRU{Далее следует вызов функции \printf}
{After function prologue we see a function \printf call}: \TT{CALL \_printf}. 

\index{x86!\Instructions!PUSH}
\IFRU
{Перед этим вызовом, адрес строки (или указатель на нее) с нашим приветствием при помощи инструкции \PUSH помещается в стек.}
{Before the call, string address (or pointer to it) containing our greeting is placed into stack with help of \PUSH instruction.}

\IFRU{После того как функция \printf возвращает управление в функцию \main, адрес строки (или указатель на нее) все еще лежит в стеке.}
{When \printf function returning control flow to \main function, string address (or pointer to it) is still in stack.}

\IFRU{Так как он больше не нужен, то указатель стека (регистр \ESP) корректируется.} 
{Because we do not need it anymore, stack pointer (\ESP register) is to be corrected.}

\index{x86!\Instructions!ADD}
\TT{ADD ESP, 4} \IFRU{означает прибавить 4 к значению в регистре \ESP.}
{mean add 4 to the value in \ESP register.}

\IFRU
{Почему 4? Так как, это 32-битный код, для передачи адреса нужно аккурат 4 байта. В x64-коде это 8 байт.}
{Why 4? Since it is 32-bit code, we need exactly 4 bytes for address passing through the stack. 
It's 8 bytes in x64-code}

\TT{``ADD ESP, 4''} \IFRU{эквивалентно \TT{``POP регистр''}, но без использования какого-либо регистра\footnote{Флаги
процессора, впрочем, модифицируются}.}
{is equivalent to \TT{``POP register''} but without any register usage\footnote{CPU flags, however, modified}.}

\index{Intel C++}
\index{Oracle RDBMS}
\index{x86!\Instructions!POP}
\IFRU{Некоторые компиляторы, например Intel C++ Compiler, в этой же ситуации, могут вместо 
\ADD сгенерировать \TT{POP ECX} (подобное можно встретить например в коде \oracle{}, им скомпилированном), 
что почти то же самое, только портится значение в регистре \ECX.}
{Some compilers like Intel C++ Compiler, at the same point, could emit \TT{POP ECX} 
instead of \ADD (for example, such pattern can be observed in \oracle{} code, compiled by Intel C++ compiler), 
and this instruction has almost the same effect, but \ECX register contents will be rewritten.}

\IFRU
{Возможно, компилятор применяет \TT{POP ECX} потому что эта инструкция короче (1 байт против 3).}
{Probably, Intel C++ compiler using \TT{POP ECX} because this instruction's opcode is shorter then 
\TT{ADD ESP, x} (1 byte against 3).}

\IFRU{О стеке можно прочитать в соответствующем разделе}{Read more about stack in relevant section}~\ref{sec:stack}.

\index{\CLanguageElements!return}
\IFRU{После вызова \printf, в оригинальном коде на \CCpp указано \TT{return 0} ~--- вернуть 0 
в качестве результата функции \main.} 
{After \printf call, in original \CCpp code was \TT{return 0} ~--- return zero as a \main function result.} 

\index{x86!\Instructions!XOR}
\IFRU{В сгенерированном коде это обеспечивается инструкцией}
{In the generated code this is implemented by instruction} \TT{XOR EAX, EAX} 

\index{x86!\Instructions!MOV}
\IFRU{\XOR, на самом деле, как легко догадаться, ``исключающее ИЛИ''}
{\XOR, in fact, just ``eXclusive OR''}
\footnote{\url{http://en.wikipedia.org/wiki/Exclusive_or}}, 
\IFRU{но компиляторы часто используют его вместо простого}
{but compilers using it often instead of}
\TT{MOV EAX, 0} ~--- 
\IFRU
{потому что снова опкод короче (2 байта против 5).}
{slightly shorter opcode again (2 bytes against 5).}

\index{x86!\Instructions!SUB}
\IFRU{Бывает так, что некоторые компиляторы генерируют}{Some compilers emitting} 
\TT{SUB EAX, EAX}, 
\IFRU
{что значит, \IT{отнять значение \EAX от \EAX}, в любом случае это даст 0 в результате.}
{which mean \IT{SUBtract \EAX value from \EAX}, which is in any case will result zero.}

\index{x86!\Instructions!RET}
\IFRU{Самая последняя инструкция \RET возвращает управление в вызывающую функцию.
Обычно, это код \CCpp CRT\footnote{C Run-Time Code}, который, в свою очередь, 
вернет управление операционной системе.}
{Last instruction \RET returning control flow to calling function.
Usually, it's \CCpp CRT\footnote{C Run-Time Code} code, which, in turn, 
return control to operation system.}

\subsubsection{GCC}

\IFRU{Теперь скомпилируем то же самое компилятором GCC 4.4.1 в Linux}
{Now let's try to compile the same \CCpp code in GCC 4.4.1 compiler in Linux}: \TT{gcc 1.c -o 1}

\IFRU{Затем при помощи \IDA. посмотрим как создалась функция \main.}
{After, with the \IDA disassembler assistance, let's see how \main function was created.} 

(\IDA, \IFRU{как и MSVC, показывает код в Intel-синтаксисе}{as MSVC, showing code in Intel-syntax}).

\IFRU{Замечание: мы также можем заставить GCC генерировать листинги в этом формате при помощи ключа}
{Note: we could also switch GCC to produce assembly listings in Intel-syntact by applying option} 
\TT{-S -masm=intel}

\begin{lstlisting}[caption=GCC]
main            proc near

var_10          = dword ptr -10h

                push    ebp
                mov     ebp, esp
                and     esp, 0FFFFFFF0h
                sub     esp, 10h
                mov     eax, offset aHelloWorld ; "hello, world"
                mov     [esp+10h+var_10], eax
                call    _printf
                mov     eax, 0
                leave
                retn
main            endp
\end{lstlisting}

\index{Function prologue}
\index{x86!\Instructions!AND}
\IFRU{Почти то же самое. 
Адрес строки ``hello, world'' лежащей в сегменте данных, в начале сохраняется в \EAX, затем записывается в стек.
А еще в прологе функции мы видим \TT{AND ESP, 0FFFFFFF0h} ~--- 
эта инструкция выравнивает значение в \ESP по 16-байтной границе, делая все значения 
в стеке также выровненными по этой границе (процессор более эффективно работает с переменными расположенными
в памяти по адресам кратным 4 или 16)\footnote{\URLWPDA}.}
{Almost the same.
Address of ``hello world'' string (stored in data segment) is saved in \EAX register first, then it stored into stack.
Also, in function prologue we see \TT{AND ESP, 0FFFFFFF0h} ~--- 
this instruction aligning \ESP value on 16-byte border, resulting all values in stack aligned too
(CPU performing better if values it working with are located in memory at addresses aligned by 
4 or 16 byte border)\footnote{\URLWPDA}.}

\index{x86!\Instructions!SUB}
\TT{SUB ESP, 10h} \IFRU{выделяет в стеке 16 байт, хотя, как будет видно далее, здесь достаточно только 4.}
{allocate 16 bytes in stack, although, as we could see below, only 4 need here.} 

\IFRU{Это происходит потому что количество выделяемого места в локальном стеке тоже выровнено по 
16-байтной границе.}{This is because the size of allocated stack is also aligned on 16-byte border.}

% TODO: rewrite.
\index{x86!\Instructions!PUSH}
\IFRU{Адрес строки (или указатель на строку) затем записывается прямо в стек без помощи инструкции \PUSH.
\IT{var\_10} по совместительству ~--- и локальная переменная и одновременно аргумент для \printf{}. Подробнее об этом будет ниже.}
{String address (or pointer to string) is then writing directly into stack space without \PUSH instruction use.
\IT{var\_10} ~--- is local variable, but also argument for \printf{}. Read below about it.}

\IFRU{Затем вызывается \printf.}{Then \printf function is called.}

\IFRU{В отличие от MSVC, GCC в компиляции без включенной оптимизации генерирует \TT{MOV EAX, 0} вместо 
более короткого опкода.}{Unlike MSVC, GCC while compiling without optimization turned on, 
emitting \TT{MOV EAX, 0} instead of shorter opcode.}

\index{x86!\Instructions!LEAVE}
\IFRU{Последняя инструкция \LEAVE ~--- это аналог команд \TT{MOV ESP, EBP} и \TT{POP EBP} ~--- 
то есть возврат указателя стека и регистра \EBP в первоначальное состояние.} 
{The last instruction \LEAVE ~--- is \TT{MOV ESP, EBP} and \TT{POP EBP} instructions pair equivalent ~--- 
in other words, this instruction setting back stack pointer (\ESP) and \EBP register to its initial state.} 

\IFRU{Это необходимо, т.к., в начале функции мы модифицировали регистры \ESP и \EBP (при помощи}
{This is necessary because we modified these register values (\ESP and \EBP) at the function start (executing}
\TT{\MOV EBP, ESP} / \TT{AND ESP, ...}).

\subsubsection{GCC: \ATTSyntax}

\IFRU{Попробуем посмотреть, как выглядит то же самое в AT\&T-синтаксисе языка ассемблера.}
{Let's see how this can be represented in AT\&T syntax of assembly language.}
\IFRU{Этот синтаксис больше распространен в UNIX-мире.}
{This syntax is much more popular in UNIX-world.}

\begin{lstlisting}[caption=\IFRU{компилируем в}{let's compile in} GCC 4.7.3]
gcc -S 1_1.c
\end{lstlisting}

\IFRU{Получим такой файл:}{We got this:}

\lstinputlisting[caption=GCC 4.7.3]{01_helloworld/1_1.s}

\IFRU{Здесь много макросов (начинающихся с точки), которые пока нам не интересны.}
{There are a lot of macros (started with dot), which are not very interesting to us so far.}
\IFRU{Пока что, ради упрощения, мы можем
их игнорировать и впредь (кроме макроса \IT{.string}, при помощи которого кодируется последовательность символов 
оканчивающихся нулем, такие же строки как в Си) и тогда получится следующее}
{For now, for the sake of simplification, we can ignore them (except \IT{.string} macro, which
encode null-terminated characters sequence, just like C-strings) and then we'll see this}
\footnote{\IFRU{Кстати, для уменьшения генерации ``лишних'' макросов, можно использовать такой ключ GCC}
{By the way, for eliminating ``unnecessary'' macros, this GCC option can be used}: 
\IT{-fno-asynchronous-unwind-tables}}:

\lstinputlisting[caption=GCC 4.7.3]{01_helloworld/1_1_refined.s}

\index{\ATTSyntax}
\index{\IntelSyntax}
\IFRU{Основные отличия синтаксиса Intel и AT\&T следующие:}{Major differences between Intel and AT\&T syntax are:}

\begin{itemize}

\item
\IFRU{Операнды записываются наоборот.}{Operands are written backwards.}

\IFRU{В Intel-синтаксисе: <инструкция> <операнд назначения> <операнд-источник>.}
{In Intel-syntax: <instruction> <destination operand> <source operand>.}

\IFRU{В AT\&T-синтаксисе: <инструкция> <операнд-источник> <операнд назначения>.}
{In AT\&T syntax: <instruction> <source operand> <destination operand>.}

\IFRU{Чтобы легче понимать разницу, можно запомнить следующее}
{Here is a thing can be memorized for easier difference understanding}: \IFRU{когда вы работаете с Intel-синтаксисом, можете в уме ставить знак равенства ($=$) между операндами}
{when you work with Intel-syntax, you can put equality sign ($=$) in your mind between operands}, 
\IFRU{а когда с AT\&T-синтаксисом, мысленно ставьте стрелку направо}{and when with AT\&T-syntax, put right arrow} 
($\rightarrow$)
\footnote{
\index{\CStandardLibrary!memcpy()}
\index{\CStandardLibrary!strcpy()}
\IFRU{Кстати, в некоторые стандартных функциях библиотеки Си (например, memcpy(), strcpy()) также применяется 
расстановка аргументов как в Intel-синтаксисе: в начале указатель в памяти на блок назначения, 
затем указатель на блок-источник.}{By the way, in some C standard functions (e.g., memcpy(), strcpy()), arguments
are listed in the same way as in Intel-syntax: pointer to destination memory block at the beginning and then
pointer to source memory block.}}.

\item
\IFRU{Перед именами регистров ставится знак процента (\%), а перед числами знак доллара (\$).}
{Before registers names, percent sign should be written (\%), and dollar sign (\$) before numbers.}
\IFRU{Вместо квадратных скобок применяются круглые.}{Parentheses are used instead of brackets.}

\item
\IFRU{К каждой инструкции добавляется специальный символ, определяющий тип данных:}
{A special symbol is to be added to each instruction, defining type of data:}

\begin{itemize}
\item l --- long (32 \IFRU{бита}{bits})
\item w --- word (16 \IFRU{бит}{bits})
\item b --- byte (8 \IFRU{бит}{bits})
\end{itemize}

\end{itemize}

\IFRU{Возвращаясь к результату компиляции: он идентичен тому, который мы посмотрели в \IDA.}
{Let's return back to compilation result: it is identical to which we saw in \IDA.}
\IFRU{Одна мелочь}{One small difference}: \TT{0FFFFFFF0h} \IFRU{записывается как}{is written as} \TT{\$-16}.
\IFRU{Это тоже самое}{It is the same}: \TT{16} \IFRU{в десятичной системе это}{in decimal system is} \TT{0x10} 
\IFRU{в шестнадцатеричной}{in hexadecimal}. 
\TT{-0x10} \IFRU{будет как раз}{is exactly} \TT{0xFFFFFFF0} 
(\IFRU{в рамках 32-битных чисел}{within 32-bit data type}).




\subsection{ARM}
\label{sec:hw_ARM}

\index{\idevices}
\index{Xcode}
\index{LLVM}
\index{Keil}
\IFRU{Для экспериментов с процессором ARM, я выбрал два компилятора}{For my experiments with ARM CPU I choose two compilers}: \IFRU{популярный в embedded-среде}{popular in embedded area} Keil Release 6/2013 
\IFRU{и среду разработки}{and} Apple Xcode 4.6.3 \IFRU{}{IDE} (\IFRU{с компилятором}{with} LLVM-GCC 4.2 \IFRU{}{compiler}), \IFRU{генерирующую код для ARM-совместимых процессоров и}{producing code for ARM-compatible processors and} \IFRU{SoC}{SoCs}\footnote{system on chip} \IFRU{в}{in} \idevices, 
\IFRU{планшетных компьютеров для Windows 8 и Windows RT}{Windows 8 and Window RT tables}\footnote{\url{http://en.wikipedia.org/wiki/List_of_Windows_8_and_RT_tablet_devices}} 
\IFRU{и таких устройствах как}{and also such devices as} Raspberry Pi.

\subsubsection{\NonOptimizingKeil + \ARMMode}

\IFRU{Для начала, скомпилируем наш пример в Keil}{Let's start by compiling our example in Keil}:

\begin{lstlisting}
armcc.exe --arm --c90 -O0 1.c 
\end{lstlisting}

\IFRU{Компилятор \IT{armcc} генерирует листинг на ассемблере}{\IT{armcc} compiler producing assembly listing}, 
\IFRU{но он содержит некоторые высокоуровневые макросы связанные с ARM}{but it has some high-level ARM-processor related macros}\footnote{
\IFRU{например, он показывает инструкции \PUSH/\POP отсутствующие в режиме ARM}{for example, ARM mode lacks 
\PUSH/\POP instructions}}, 
\IFRU{а нам важнее увидеть инструкции ``как есть'', так что посмотрим скомпилированный результат в \IDA}{but it's more important for us to see instructions ``as is'', so let's see compiled results in \IDA}.

\begin{lstlisting}[caption=\NonOptimizingKeil + \ARMMode + \IDA]
.text:00000000             main
.text:00000000 10 40 2D E9                 STMFD   SP!, {R4,LR}
.text:00000004 1E 0E 8F E2                 ADR     R0, aHelloWorld ; "hello, world"
.text:00000008 15 19 00 EB                 BL      __2printf
.text:0000000C 00 00 A0 E3                 MOV     R0, #0
.text:00000010 10 80 BD E8                 LDMFD   SP!, {R4,PC}

.text:000001EC 68 65 6C 6C+aHelloWorld     DCB "hello, world",0    ; DATA XREF: main+4
\end{lstlisting}

\index{ARM!\ARMMode}
\index{ARM!\ThumbMode}
\index{ARM!\ThumbTwoMode}
\IFRU{Вот чуть-чуть фактов о процессоре ARM, которые желательно знать}{Here is couple of ARM-related facts we should know in order to proceed}.
\IFRU{Процессор ARM имеет по крайней мере два основных режима: режим ARM и thumb}{ARM processor has at least two major modes: ARM mode and thumb}. 
\IFRU{В первом (ARM) режиме доступны все инструкции и каждая имеет размер 32 бита (или 4 байта)}{In first (ARM) mode all instructions are enabled and each has 32-bit (4 bytes) size}. 
\IFRU{Во втором режиме (thumb) каждая инструкция имеет размер 16 бит (или 2 байта)}{In second (thumb) mode each instruction has 16-bit (or 2 bytes) size}\footnote{\IFRU{Кстати, инструкции фиксированного размера удобны тем, что всегда можно легко узнать адрес предыдущей инструкции, или следующей}{NOTTRANSLATED}}. 
\IFRU{Режим thumb может выглядеть привлекательнее тем, что программа на нем может быть 1) компактнее; 2) эффективнее исполняться на микроконтроллере с 16-битной шиной данных}{Thumb mode may look attractive because program in it may be 1) compact; 2) executing faster on microcontroller having 16-bit memory datapath}. 
\IFRU{Но за всё нужно платить: в режиме thumb куда меньше возможностей процессора, например, возможен доступ только к 8-и регистрам процессора, и чтобы совершить некоторые действия, выполнимые в режиме ARM одной инструкцией, нужны несколько thumb-инструкций}{Nothing come for free of charge, so, in thumb mode there are reduced instruction set, only 8 registers are accessible and one need several thumb instructions for doing some operations when in ARM mode you'll need just one}.
\IFRU{Начиная с ARMv7, имеется также поддержка инструкций thumb-2, это thumb расширенный до поддержки куда большего числа инструкций}{Starting at ARMv7, there are also thumb-2 instructions set present, this is a thumb extended to support much bigger instructions set}.
\IFRU{Распространено заблуждение что thumb-2 это смесь ARM и thumb. Это не верно. Просто thumb-2 был дополен до
более полной поддержки возможностей процессора, что теперь может легко конкурировать с режимом ARM.}
{There is a common misconception that thumb-2 is a mix of ARM and thumb. It's not correct. 
But rather thumb-2 was extended to support processor features so fully,
so now it can compete with ARM mode.}
\IFRU{Программа для процессора ARM может представлять смесь процедур скомпилированных для обоих режимов}{A program for ARM processor may be mix of procedures compiled for both modes}.
\IFRU{Основное количество приложений для \idevices скомпилировано для набора инструкций thumb-2, потому что Xcode
делает так по умолчанию}{Majority of \idevices applications are compiled for thumb-2 instructions set, because Xcode do this by default}.

\IFRU{В вышеприведененном примере можно легко увидеть что каждая инструкция имеет размер 4 байта}{In example we see here we can easily see that each instruction has size of 4 bytes}.
\IFRU{Действительно, ведь мы же компилировали наш код для режима ARM а не thumb}{Indeed, we compiled our code for ARM mode, but for thumb}.

\index{ARM!\Instructions!STMFD}
\index{ARM!\Instructions!POP}
\IFRU{Самая первая инструкция}{The very first instruction} \TT{''STMFD SP!, \{R4,LR\}''}\footnote{Store Multiple Full Descending} \IFRU{работает как инструкция}{works here as} \PUSH \IFRU{в}{in} x86, \IFRU{записывает значения двух регистров}{instruction, writing values of two} (\TT{R4} \IFRU{и}{and} \LR) \IFRU{в стек}{registers into stack}. 
\IFRU{Действительно, в выдаваемом листинге на ассемблере, компилятор \IT{armcc}, для упрощения, указывает здесь инструкцию}{Indeed, in output listing, \IT{armcc} compiler, for the sake of simplification, showing here} \TT{''PUSH \{r4,lr\}''}\IFRU{}{instruction}.
\IFRU{Но это не совсем точно, инструкция \PUSH доступна только в режиме thumb, поэтому, во избежания путанницы, я предложил работать в \IDA}{But it's not quite correct, \PUSH instruction available only in thumb mode, so, to make things less messy, I offered to work in \IDA}.

\IFRU{Итак, эта инструкция записывает значения регистров \TT{R4} и \LR по адресу в памяти, на который указывает регистр \SPwithfootnote, затем уменьшает \TT{SP}, чтобы он указывал на место в стеке, доступное для новых записей}{So this instruction writes values of \TT{R4} and \LR registers at the address in memory to which \SPwithfootnote pointing, then decrements \SP so it will points to a place in stack free for new entries}.

\IFRU{Эта инструкция, как и инструкция \PUSH в режиме thumb, может сохранить в стеке одновременно несколько значений регистров, что может быть очень удобно}{This instruction, like \PUSH instruction in thumb mode, is able save several register values at once and this may be useful}. 
\IFRU{Кстати, такого в x86 нет}{By the way, there is no such thing in x86}. 
\IFRU{Так же следует заметить, что \TT{STMFD} ~--- генерализация инструкции \PUSH (то есть, расширяет её возможности), потому что может работать с любым регистром а не только с \SP, это тоже может быть очень удобно}{It's also can be noted that \TT{STMFD} ~--- generalization of \PUSH instruction (extending its features), because it can work with any register, not just with \SP and this can be very useful}.

\index{\PICcode}
\index{ARM!\Instructions!ADR}
\IFRU{Инструкция}{} \TT{''ADR R0, aHelloWorld''} \IFRU{прибавляет значение регистра \PC к смещению, где хранится строка}{instruction adding \PC register value to the offset, where the} \IT{``hello, world''} \IFRU{}{string is located}. 
\IFRU{Причем здесь \PC, можно спросить}{How \TT{PC} register used here, one might ask}?
\IFRU{Притом, что это так называемый ``\PICcode''}{This is so called ``\PICcode''}
\footnote{\IFRU{Читайте больше об этом в соответствующем разделе}{Read more about it in relevant section}~\ref{sec:PIC}}, 
\IFRU{он предназначен для исполнения будучи не привязанным к каким-либо адресам в памяти}{it is intended to be executed not to be fixed to any addresses in memory}.
\IFRU{В опкоде инструкции \TT{ADR} указывается разница между адресом этой инструкции и местом, где хранится строка}{In the opcode of \TT{ADR} instruction, here is encoded a difference between address of this instruction and the place where the string is located}.
\IFRU{Эта разница всегда будет постоянной, вне зависимости от того, куда был загружен операционной системой наш код}{Difference will always be constant, without any dependence to the address where that code being loaded, by operation system, presumably}. 
\IFRU{Поэтому всё что нужно это прибавить адрес текущей инструкции (из \PC) чтобы получить текущий абсолютный адрес нашей Си-строки}{That's why all we need is to add address of current instruction (from \PC) in order to get absolute address of our C-string in memory}.

\index{ARM!\Registers!Link Register}
\index{ARM!\Instructions!BL}
\IFRU{Инструкция}{} \TT{''BL \_\_2printf''}\footnote{Branch with Link} \IFRU{вызывает функцию \printf}{instruction calling \printf function}. 
\IFRU{Работа этой инструкции состоит из двух фаз}{That's how this instruction works}: 
\begin{itemize}
\item
\IFRU{записать адрес после инструкции \TT{BL} ($0xC$) в регистр \LRwithfootnote}
{write address after \TT{BL} instruction ($0xC$) into \LRwithfootnote register};
\item
\IFRU{затем собственно передать управление в \printf, записав адрес этой функции в регистр \PCwithfootnote}
{then pass control flow into \printf by writing its address into \PCwithfootnote register}.
\end{itemize}

\IFRU{Ведь, когда функция \printf закончит работу, нужно знать, куда вернуть управление, поэтому закончив работу, всякая функция передает управление по адресу записанному в регистре \LR}
{Because, when \printf finishes its work, it should have information, where it should return control, that's why each function passes control to the address stored in \LR register}.

\IFRU{В этом разница между ``чистыми'' RISC-процессорами вроде ARM и x86, где адрес возврата записывается в стек}{That is the difference between ``pure'' RISC-processors like ARM and x86, where address of return is stored in stack}\footnote{\IFRU{Подробнее об этом будет описано в следующей главе}{Read more about this in next section}~\ref{sec:stack}}.

\IFRU{Кстати, 32-битный абсолютный адрес, либо же смещение, невозможно закодировать в 32-битной инструкции \TT{BL}, в ней есть место только для 24-х бит}
{By the way, absolute 32-bit address or offset cannot be encoded in 32-bit \TT{BL} instruction, because it has space only for 24 bits}.
\IFRU{Так же следует отметить, что из-за того что все инструкции в режиме ARM имеют длину 4 байта (32 бита), и инструкции могут находится только по адресам кратным 4, то последние 2 бита (всегда нулевых) можно не кодировать.}
{It's also worth to note that all ARM mode instructions has size 4 bytes (32 bits), hence they all can be located only on 4-byte boundary addresses. This mean, last 2 bits of instruction address (always zero bits) may be omitted.}
\IFRU{В итоге имеем 26 бит, при помощи которых можно закодировать смещение}
{In summary, we have 26 bit for offset encoding, this is enough to represent offset} $\pm{}\approx{}32M$.

\index{ARM!\Instructions!MOV}
\IFRU{Следующая инструкция}{Next} \TT{''MOV R0, \#0''}\footnote{MOVe} \IFRU{просто записывает $0$ в регистр \Rzero}{instruction just writes $0$ into \Rzero register}.
\IFRU{Ведь наша Си-функция возвращает $0$ а возвращаемое значение всякая функция оставляет в \Rzero}{That's because our C-function returning $0$ and returning value is to be placed in \Rzero}.

\index{ARM!\Registers!Link Register}
\index{ARM!\Instructions!LDMFD}
\index{ARM!\Instructions!POP}
\IFRU{Последняя инструкция}{The last instruction} \TT{''LDMFD SP!, {R4,PC}''}\footnote{\LDMFDDESC} \IFRU{это инструкция обратная от}{is an inversive instruction of} \TT{STMFD}, \IFRU{она загружает из стека значения для сохранения их в \TT{R4} и \PC, увеличивая указатель стека \SP}{it loads values from stack for saving them into \TT{R4} and \PC, incremeting stack pointer \SP}.
\IFRU{Это, в каком-то смысле, аналог \POP}{It can be said, it is similar to \POP}. 
\IFRU{Обратите внимание: самая первая инструкция \TT{STMFD} сохранила в стеке \TT{R4} и \LR, а \IT{восстанавливаются} \TT{R4} и \PC}{Note: the very first instruction \TT{STMFD} saved \TT{R4} and \LR into stack, but \TT{R4} and \PC are \IT{restored}}.
\IFRU{Как я уже описывал, в регистре \LRwithfootnote обычно сохраняется адрес места, куда нужно всякой функции вернуть управление}{As I wrote before, in \LRwithfootnote register address of place saved, to where each function should return control}.
\IFRU{Самая первая инструкция сохраняет это значение в стеке, потому что наша функция \main позже будет сама пользоваться этим регистром, в момент вызова \printf}{The very first function saving its value in stack because our \main function will use that register in order to call \printf}.
\IFRU{А затем, в конце функции, это значение можно сразу записать в \PC, таким образом, передав управление туда, откуда была вызвана наша функция}{And then, in the function end this value can be written to \PC, thus, by passing control to where our function was called}.
\IFRU{Так как функция \main обычно самая главная в \CCpp, вероятно, управление будет возвращено в загрузчик операционной системы, либо куда-то в runtime функции Си, или что-то в этом роде}
{Since our \main function is usually primary function in \CCpp, apparently, control will be returned to operation system loader or to some place in runtime C functions, or something like that}.

\index{ARM!DCB}
\TT{DCB} ~--- \IFRU{директива ассемблера, описывающая массивы байт или ASCII-строк, аналог директивы DB в 
x86-ассемблере}
{assembly language directive, defining array of bytes or ASCII-strings, similar to DB directive 
in x86-assembly language}.

\subsubsection{\NonOptimizingKeil: \ThumbMode}

\IFRU{Скомпилируем тот же пример в Keil для режима thumb}{Let's compile the same example in Keil in thumb mode}:

\begin{lstlisting}
armcc.exe --thumb --c90 -O0 1.c 
\end{lstlisting}

\IFRU{Получим (в \IDA)}{We will get (in \IDA)}:

\begin{lstlisting}[caption=\NonOptimizingKeil + \ThumbMode + \IDA]
.text:00000000             main
.text:00000000 10 B5                       PUSH    {R4,LR}
.text:00000002 C0 A0                       ADR     R0, aHelloWorld ; "hello, world"
.text:00000004 06 F0 2E F9                 BL      __2printf
.text:00000008 00 20                       MOVS    R0, #0
.text:0000000A 10 BD                       POP     {R4,PC}

.text:00000304 68 65 6C 6C+aHelloWorld     DCB "hello, world",0    ; DATA XREF: main+2
\end{lstlisting}

\IFRU{Сразу бросаются в глаза двухбайтные (16-битные) опкоды, это, как я уже упоминал, thumb}{We can easily spot 2-byte (16-bit) opcodes, this is, as I mentioned, thumb}.
\index{ARM!\Instructions!BL}
\IFRU{Кроме инструкции \TT{BL}}{Except \TT{BL} instruction}.
\IFRU{Но на самом деле, она состоит из двух 16-битных инструкций}{In fact, it consisted in two 16-bit instructions}.
\IFRU{Это потому что загрузить в \PC смещение, по которому находится функция \printf, используя так мало места в одном 16-битном опкоде, очевидно, нельзя}{That's because it's not possible to load offset to \printf function into \PC when using so small space in one 16-bit opcode, obviously}.
\IFRU{Поэтому первая 16-битная инструкция загружает старшие 10 бит смещения, а вторая ~--- младшие 11 бит смещения}{That's why first 16-bit instruction loads higher 10 bits of offset and second ~--- loads 11 lower bits of offset}.
\IFRU{Как я уже упоминал, все инструкции в thumb-режиме имеют длину 2 байта или 16 бит}{As I mentioned, all instructions in thumb mode has size of 2 bytes or 16 bits}.
\IFRU{Поэтому невозможна такая ситуация, когда thumb-инструкция начинается по нечетному адресу}
{This mean, it's not possible for thumb-instruction to be on odd address whatsoever}.
\IFRU{Следовательно, последний бит адреса можно не кодировать}{Considering this, last address bit may be omitted while instruction encoding}.
\IFRU{Таким образом, в итоге, в thumb-инструкции \TT{BL} кодируется смещение}{Summarizing, in \TT{BL} thumb-instruction,} $\pm{}\approx{}2M$ \IFRU{от текущего адреса}{can be encoded as offset from current address}.

\IFRU{Остальные инструкции в функции: \PUSH и \POP работают почти так же как и описанные \TT{STMFD}/\TT{LDMFD}, только регистр \SP здесь не указывается явно}{Other instructions in functions are: \PUSH and \POP works just like described \TT{STMFD}/\TT{LDMFD}, but \SP register not mentioned explicitely here}.
\TT{ADR} \IFRU{работает также как и в предыдущем примере}{works just like in previous example}.
\TT{MOVS} \IFRU{записывает $0$ в регистр \Rzero для возврата нуля}{writes $0$ in \Rzero register to zero returning}.

\subsubsection{\OptimizingXcode + \ARMMode}

Xcode 4.6.3 \IFRU{без включенной оптимизации выдает слишком много лишнего кода, поэтому остановимся на той версии, где как можно меньше инструкций}{without optimization turned on, produces a lot of redundant code, so we'll study that version where instruction count as small as possible}: \Othree.

\begin{lstlisting}[caption=\OptimizingXcode + \ARMMode]
__text:000028C4             _hello_world
__text:000028C4 80 40 2D E9                 STMFD           SP!, {R7,LR}
__text:000028C8 86 06 01 E3                 MOV             R0, #0x1686
__text:000028CC 0D 70 A0 E1                 MOV             R7, SP
__text:000028D0 00 00 40 E3                 MOVT            R0, #0
__text:000028D4 00 00 8F E0                 ADD             R0, PC, R0
__text:000028D8 C3 05 00 EB                 BL              _puts
__text:000028DC 00 00 A0 E3                 MOV             R0, #0
__text:000028E0 80 80 BD E8                 LDMFD           SP!, {R7,PC}

__cstring:00003F62 48 65 6C 6C+aHelloWorld_0   DCB "Hello world!",0
\end{lstlisting}

\IFRU{Инструкции}{Instructions} \TT{STMFD} \IFRU{и}{and} \TT{LDMFD} \IFRU{нам уже знакомы}{are familiar to us}.

\IFRU{Инструкция \MOV просто записывает число $0x1686$ в регистр \Rzero, это смещение указывающее на строку ``Hello world!''}{\MOV instruction just writes $0x1686$ number into \Rzero register, this is offset pointing to the ``Hello world!'' string}.

\IFRU{Регистр \TT{R7}, по стандарту принятому в}{\TT{R7} register, as it is standardized in}\cite{IOSABI}
\IFRU{это}{is} frame pointer, \IFRU{о нем будет рассказано позже}{more on it below}.

\index{ARM!\Instructions!MOVT}
\IFRU{Инструкция}{} \TT{MOVT R0, \#0} \IFRU{записывает 0 в старшие 16 бит регистра}{instruction writes 0 into higher 16 bit of register}.
\IFRU{Дело в том, что обычная инструкция \MOV в режиме ARM может записывать какое-либо значение только в младшие 16 бит регистра, ведь, больше нельзя закодировать в ней}{The issue is here in that generic \MOV instruction in ARM mode may writes only lower 16 bit of register}.
\IFRU{Помните, что в режиме ARM опкоды всех инструкций ограничены длиной в 32 бита. Конечно, это ограничение не касается перемещений между регистрами.}{Remember, all instruction's opcodes in ARM mode are limited in size to 32 bits. Of course, this limitation is not related to moving between registers.}
\IFRU{Поэтому для записи в старшие биты (от 16-го по 31-го включительно) существует дополнительная команда \TT{MOVT}}{So that's why additional instruction \TT{MOVT} exist for writing into higher bits (from 16 to 31 inclusive)}.
\IFRU{Впрочем, здесь её использование избыточно, потому что инструкция \TT{''MOV R0, \#0x1686''} выше итак обнулила старшую часть регистра}{However, its usage here is redundant, because \TT{''MOV R0, \#0x1686''} instruction above cleared higher part of register}. \IFRU{Возможно, это недочет компилятора}{Probably, it's compiler's shortcoming}.

\index{ARM!\Instructions!ADD}
\IFRU{Инструкция} \TT{''ADD R0, PC, R0''} \IFRU{прибавляет \PC к \Rzero, для вычисления действительного адреса строки ``Hello world!'', как нам уже известно, это ``\PICcode'', поэтому такая корректива необходима}
{instruction adding \PC to \Rzero, for calculating absolute address of ``Hello world!'' string, 
and as we already know that, it's ``\PICcode'', so this corrective is essential here}.

\IFRU{Инструкция \TT{BL} вызывает \puts вместо \printf}{\TT{BL} instruction calling \puts instead of \printf}.

\label{puts}
\index{puts() \IFRU{вместо}{instead of} printf()}
\IFRU{Компилятор заменил вызов \printf на \puts. 
Действительно, \printf с одним агрументом это почти аналог \puts.}
{GCC replaced first \printf call to \puts. 
Indeed: \printf with sole argument is almost analogous to \puts.} 

\IFRU{\IT{Почти}, если принять условие что в строке не будет управляющих символов \printf 
начинающихся со знака процента. Тогда эффект от работы этих двух функций будет разным.}
{\IT{Almost}, because we need to be sure that this string will not contain printf-control 
statements starting with \IT{\%}: then effect of these two functions will be different.}

\IFRU{Зачем компилятор заменил один вызов на другой? Потому что \puts() работает быстрее}
{Why compiler replaced \printf to \puts? Because \puts() work faster}
\footnote{\url{http://www.ciselant.de/projects/gcc_printf/gcc_printf.html}}. 

\IFRU{Видимо потому, что \puts проталкивает символы в stdout не сравнивая каждый со знаком процента.}
{\puts working faster because it just passes characters to stdout not comparing each with \IT{\%} symbol.}

\IFRU{Далее уже знакомая инструкция}{Next, we see familiar to us} \TT{''MOV R0, \#0''}\IFRU{, служащая для установки в 0 возвращаемого значения функции}{instruction, intended to set 0 to \Rzero register}.

\subsubsection{\OptimizingXcode + \ThumbTwoMode}

\IFRU{По умолчанию}{By default}, Xcode 4.6.3 \IFRU{генерирует код для режима thumb-2, примерно в такой манере}{generating code for thumb-2 in such manner}:

\begin{lstlisting}[caption=\OptimizingXcode + \ThumbTwoMode]
__text:00002B6C                   _hello_world
__text:00002B6C 80 B5                             PUSH            {R7,LR}
__text:00002B6E 41 F2 D8 30                       MOVW            R0, #0x13D8
__text:00002B72 6F 46                             MOV             R7, SP
__text:00002B74 C0 F2 00 00                       MOVT.W          R0, #0
__text:00002B78 78 44                             ADD             R0, PC
__text:00002B7A 01 F0 38 EA                       BLX             _puts
__text:00002B7E 00 20                             MOVS            R0, #0
__text:00002B80 80 BD                             POP             {R7,PC}

...

__cstring:00003E70 48 65 6C 6C 6F 20+aHelloWorld     DCB "Hello world!",0xA,0
\end{lstlisting}

\index{ThumbTwoMode}
\IFRU{Инструкции \TT{BL} и \TT{BLX} в thumb, как мы помним, кодируются как пара 16-битных инструкций, 
а в thumb-2 эти \IT{суррогатные} опкоды расширены так, что новые инструкции кодируются здесь как 
32-битные инструкции}{\TT{BL} and \TT{BLX} instructions in thumb mode, as we remember, encoded as pair
of 16-bit instructions and in thumb-2, these \IT{surrogate} opcodes extended in such way so that new instruction
may be encoded here as 32-bit instructions}.
\IFRU{Это можно заметить по тому что опкоды thumb-2 инструкций всегда начинаются с $0xFx$ либо с $0xEx$}{That's
easily observable ~--- opcodes of thumb-2 instructions are also beginning with $0xFx$ or $0xEx$}.
\IFRU{Но в листинге \IDA, первый байт опкода стоит вторым, это из-за того что в ARM инструкции кодируются так:
в начале последний байт, потом первый (для thumb и thumb-2 режима), либо, 
(для инструкций в режиме ARM) в начале четвертый байт, затем третий, второй и первый}{But in \IDA listings,
first byte of opcode is at the place of second, that's because instructions here encoded as follows: last byte and then first one (for thumb and thumb-2 modes), or, (for instructions in ARM mode): fourth byte, then third, then second and first}.
\index{ARM!\Instructions!MOVW}
\index{ARM!\Instructions!MOVT.W}
\index{ARM!\Instructions!BLX}
\IFRU{Так что мы видим здесь что инструкции \TT{MOVW}, \TT{MOVT.W} и \TT{BLX} начинаются с}{So as we see, \TT{MOVW}, \TT{MOVT.W} and \TT{BLX} instructions are beginning with} $0xFx$.

\IFRU{Одна из thumb-2 инструкций это}{One of thumb-2 instructions is} \TT{``MOVW R0, \#0x13D8''} ~--- \IFRU{она записывает 16-битное число в младшую часть регистра \Rzero}{it writes 16-bit value into lower part of \Rzero register}.

\IFRU{Еще}{Also} \TT{``MOVT.W R0, \#0''} ~--- \IFRU{эта инструкция работает так же как и}{this instruction works just like} 
\TT{MOVT} \IFRU{из предыдущего примера, но она работает в}{from previous example, but it works in} thumb-2.

\index{ARM!\IFRU{переключение режимов}{mode switching}}
\index{ARM!\Instructions!BLX}
\IFRU{Помимо прочих отличий, здесь используется инструкция}{Among other differences, here is} \TT{BLX} \IFRU{вместо}{instruction used instead of} \TT{BL}.
\IFRU{Отличие в том, что помимо сохранения адреса возврата в регистре \LR и передаче управления в функцию \puts, происходит смена режима процессора с thumb на ARM, либо наоборот}{Difference in that way that beside saving of return address in \LR register and passing control to \puts function, processor is switching from thumb mode to ARM or back}.
\IFRU{Здесь это нужно потому что инструкция, куда ведет переход, выглядит так (она закодирована в режиме ARM)}{This instruction in place here because the instruction to which control is passed looks like (it's encoded in ARM mode)}:

\begin{lstlisting}
__symbolstub1:00003FEC _puts           ; CODE XREF: _hello_world+E
__symbolstub1:00003FEC 44 F0 9F E5     LDR  PC, =__imp__puts
\end{lstlisting}

\IFRU{Итак, внимательный читатель может задать справделивый вопрос: почему бы не вызывать \puts сразу в 
том же месте кода, где он нужен?}
{So, observant reader may ask: why not to call \puts right at the place of code where it needed?}

\IFRU{Но это не очень выгодно (в плане экономия места) и вот почему}{But that's not very space-efficient, and that's why}.

\index{\IFRU{Динамически подгружаемые библиотеки}{Dynamically loaded libraries}}
\IFRU{Практически любая программа использует внешние динамические библиотеки, будь то DLL в Windows, .so в *NIX 
либо .dylib в Mac OS X}{Almost any program uses external dynamic libraries, like DLL in Windows, .so in *NIX or .dylib in Mac OS X}. 
\IFRU{В динамических библиотеках находятся часто используемые библиотечные функции, в том числе стандартная функция Си \puts}
{Often used library functions are stored in dynamic libraries, including standard C-function \puts}.

\index{Relocation}
\IFRU{В исполняемом бинарном файле}{In executable binary file} (Windows PE .exe, ELF \IFRU{либо}{or} Mach-O) \IFRU{имеется секция импортов, список символов (функций либо глобальных переменных) импортируемых из внешних модулей, а также названия самих модулей}{a section of imports is present, that is list of symbols (functions or global variables) being imported from external modules and also names of these modules}.

\IFRU{Загрузчик операционной системы загружает необходимые модули и, перебирая импортируемые символы в основном модуле, проставляет правильные адреса каждого символа}{Operation system loader loads all modules need and, while enumerating importing symbols in primary module, sets correct addresses of each symbol}.

\IFRU{В нашем случае}{In our case}, \IT{\_\_imp\_\_puts} \IFRU{это 32-битная переменная, куда загрузчик ОС запишет правильный адрес этой же функции во внешней библиотеке}{is 32-bit variable where OS loader will write correct address of that function in external library}. 
\IFRU{Так что инструкция \TT{LDR} просто берет 32-битное значение из этой переменной и, записывая его в регистр \PC, просто передает туда управление}{So that \TT{LDR} instruction just takes 32-bit value from this variable and, writing it into \PC register, just passing control to it}.

\IFRU{Чтобы уменьшить время работы загрузчика ОС, нужно чтобы ему пришлось записать адрес каждого символа только один раз, в соответствующее для них место}{So to readuce a time OS loader needs for doing this procedure, it's good idea for it to write address of each symbol only once, to special place for it}.

\index{thunk-\IFRU{функции}{functions}}
\IFRU{К тому же, как мы уже убедились, нельзя одной инструкцией загрузить в регистр 32-битное число без обращений к памяти}{Besides, as we already figured out, it's not possible to load 32-bit value into register 
using only one instruction, without memory access}.
\IFRU{Так что, наиболее оптимально, выделить отдельную функцию, работающую в режиме ARM, 
чья единственная цель ~--- передавать управление дальше, в динамическую библиотеку}
{So, it is optimal to allocate separate function working in ARM mode with only one goal ~--- 
to pass control to dynamic library}. 
\IFRU{И затем ссылаться на эту короткую функцию из одной инструкции (так называемую thunk-функцию) из thumb-кода}{And then to jump to this short one-instruction function (so called thunk-function) from thumb-code}.

\index{ARM!\Instructions!BL}
\IFRU{Кстати, в предыдущем примере (скомпилированном для режима ARM), переход при помощи инструкции \TT{BL} ведет 
на такую же thunk-функцию, однако режим процессора не переключается (отсюда, отсутствие ``X'' в мнемонике инструкции)}{By the way, in previous example (compiled for ARM mode) control passing by \TT{BL} instruction is going to the same thunk-function, however, processor mode is not switched (hence, absence of ``X'' in instruction mnemonic)}.



\section{\Stack}
\label{sec:stack}
\index{\Stack}

\IFRU{Стек в компьютерных науках ~--- это одна из наиболее фундаментальных вещей}
{Stack ~--- is one of the most fundamental things in computer science.}\footnote{\url{http://en.wikipedia.org/wiki/Call_stack}}.

\IFRU{Технически, это просто блок памяти в памяти процесса + регистр \ESP или \RSP в x86, либо \SP в ARM, который указывает где-то в пределах этого блока.}
{Technically, it's just a memory block in process memory + \ESP or \RSP register in x86, or \SP register in ARM, as a pointer within this block.}

\index{ARM!\Instructions!PUSH}
\index{ARM!\Instructions!POP}
\index{x86!\Instructions!PUSH}
\index{x86!\Instructions!POP}
\IFRU{Часто используемые инструкции для работы со стеком это \PUSH и \POP (в x86 и thumb-режиме ARM). 
\PUSH уменьшает \ESP/\RSP/\SP на $4$, затем записывает по адресу на который указывает \ESP/\RSP/\SP содержимое своего единственного операнда.}
{Most frequently used stack access instructions are \PUSH and \POP (both in x86 and ARM thumb-mode). 
\PUSH subtracting \ESP/\RSP/\SP by $4$ and then writing contents of its sole operand to the memory address pointing by \ESP/\RSP/\SP.} 

\IFRU{\POP это обратная операция ~--- сначала достает из \ESP/\RSP/\SP значение и помещает его в операнд 
(который очень часто является регистром) и затем увеличивает \ESP/\RSP/\SP на $4$. 
Конечно, это для 32-битной среды. В x64-среде это будет $8$ а не $4$.}
{\POP is reverse operation: get a data from memory pointing by \ESP/\RSP/\SP, put it to operand
(often register) and then add $4$ to \ESP/\RSP/\SP. 
Of course, this is for 32-bit environment. $8$ will be here instead of $4$ in x64 environment.}

\IFRU{В самом начале, регистр-указатель указывает на конец стека.}{After stack allocation, stack pointer pointing to the end of stack.}
\IFRU{\PUSH уменьшает регистр-указатель, а \POP ~--- увеличивает.}{\PUSH increasing stack pointer, and \POP decreasing.}
\IFRU{Конец стека находится в начале блока памяти выделенного под стек. Это странно, но это так.}
{The end of stack is actually at the beginning of allocated for stack memory block. 
It seems strange, but it is so.}

\IFRU{В процессоре ARM, тем не менее, есть поддержка стеков растущих как в сторону уменьшения, так и в
сторону увеличения}{Nevertheless, ARM has instructions supporting ascending stacks, but also descending stacks}. 
\index{ARM!\Instructions!STMFD}
\index{ARM!\Instructions!LDMFD}
\index{ARM!\Instructions!STMED}
\index{ARM!\Instructions!LDMED}
\index{ARM!\Instructions!STMFA}
\index{ARM!\Instructions!LDMFA}
\index{ARM!\Instructions!STMEA}
\index{ARM!\Instructions!LDMEA}
\IFRU{Например, инструкции}{For example,} 
STMFD\footnote{\STMFDdesc}/LDMFD\footnote{\LDMFDDESC}, 
STMED\footnote{\STMEDdesc}/LDMED\footnote{\LDMEDdesc} 
\IFRU{предназначены для descending-стека, т.е., уменьшающегося}{instructions are intended for work with 
descending stack}.
\IFRU{Инструкции}{}
STMFA\footnote{\STMFAdesc}/LMDFA\footnote{\LDMFAdesc}, 
STMEA\footnote{\STMEAdesc}/LDMEA\footnote{\LDMEAdesc} 
\IFRU{предназначены для ascending-стека, т.е., увеличивающегося}{instructions are intended for work with 
ascending stack}.

\IFRU{Для чего используется стек?}{What stack is used for?}

\subsubsection{\IFRU{Сохранение адреса куда должно вернуться управление после вызова функции}
{Save the return address where a function should return control after execution}}

\paragraph{x86}

\index{x86!\Instructions!CALL}
\IFRU{При вызове другой функции через \CALL, сначала в стек записывается адрес указывающий на место аккурат после 
инструкции \CALL, затем делается безусловный переход (почти как \TT{JMP}) на адрес указанный в операнде.} 
{While calling another function with a \CALL instruction the address of the point exactly after the \CALL instruction is saved 
to the stack and then an unconditional jump to the address in the CALL operand is executed.} 

\index{x86!\Instructions!PUSH}
\index{x86!\Instructions!JMP}
\IFRU{\CALL это аналог пары инструкций \TT{PUSH address\_after\_call / JMP}.}
{The \CALL instruction is equivalent to a \TT{PUSH address\_after\_call / JMP operand} instruction pair}.

\index{x86!\Instructions!RET}
\index{x86!\Instructions!POP}
\IFRU{\RET вытаскивает из стека значение и передает управление по этому адресу ~--- 
это аналог пары инструкций \TT{POP tmp / JMP tmp}.}
{\RET fetches a value from the stack and jumps to it ~--- it is equivalent to a \TT{POP tmp / JMP tmp} instruction pair.}

\index{\Stack!\IFRU{Переполнение стека}{Stack overflow}}
\index{\Recursion}
\IFRU{Крайне легко устроить переполнение стека запустив бесконечную рекурсию:}
{Overflow the stack is simple. Just run eternal recursion:}

\begin{lstlisting}
void f()
{
	f();
};
\end{lstlisting}

\IFRU{MSVC 2008 предупреждает о проблеме:}{MSVC 2008 reports the problem:}

\begin{lstlisting}
c:\tmp6>cl ss.cpp /Fass.asm
Microsoft (R) 32-bit C/C++ Optimizing Compiler Version 15.00.21022.08 for 80x86
Copyright (C) Microsoft Corporation.  All rights reserved.

ss.cpp
c:\tmp6\ss.cpp(4) : warning C4717: 'f' : recursive on all control paths, function will cause runtime stack overflow
\end{lstlisting}

\dots \IFRU{но тем не менее создает нужный код}{but generates the right code anyway}:

\begin{lstlisting}
?f@@YAXXZ PROC						; f
; File c:\tmp6\ss.cpp
; Line 2
	push	ebp
	mov	ebp, esp
; Line 3
	call	?f@@YAXXZ				; f
; Line 4
	pop	ebp
	ret	0
?f@@YAXXZ ENDP						; f
\end{lstlisting}

\dots \IFRU
{причем, если включить оптимизацию (\Ox), то будет даже интереснее, без переполнения стека, 
но работать будет \IT{корректно}\footnote{здесь ирония}:}
{Also if we turn on optimization (\Ox option) the optimized code will not overflow the stack 
but will work \IT{correctly}\footnote{irony here}:}

\begin{lstlisting}
?f@@YAXXZ PROC						; f
; File c:\tmp6\ss.cpp
; Line 2
$LL3@f:
; Line 3
	jmp	SHORT $LL3@f
?f@@YAXXZ ENDP						; f
\end{lstlisting}

\IFRU{GCC 4.4.1 генерирует точно такой же код в обоих случаях, хотя и не предупреждает о проблеме.}
{GCC 4.4.1 generating the same code in both cases, although not warning about problem.}

\paragraph{ARM}

\index{ARM!\Registers!Link Register}
\IFRU{Программы для ARM также используют стек для сохранения \ac{RA}, куда нужно вернуться, но несколько иначе}{ARM
programs also use the stack for saving return addresses, but differently}.
\IFRU{Как уже упоминалось в секции}{As it was mentioned in} ``\HelloWorldSectionName''~\ref{sec:hw_ARM}, 
\IFRU{\ac{RA} записывается в регистр}{the \ac{RA} is saved to the} \LR (\IT{link register}).
\IFRU{Но если есть необходимость вызывать какую-то другую функцию, и использовать регистр \LR еще
раз, его значение желательно сохранить}
{However, if one needs to call another function and use the \LR register
one more time its value should be saved}.
\index{Function prologue}
\IFRU{Обычно, это происходит в прологе функции, часто мы видим там инструкцию вроде}
{Usually it is saved in the function prologue. Often, we see instructions like}
\index{ARM!\Instructions!PUSH}
\index{ARM!\Instructions!POP}
\TT{``PUSH {R4-R7,LR}''} \IFRU{, а в эпилоге}{along with this instruction in epilogue} \TT{``POP {R4-R7,PC}''} ~--- 
\IFRU{так сохраняются регистры, которые будут использоваться в текущей функции, в том числе}
{thus register values
to be used in the function are saved in the stack, including} \LR.

\index{ARM!Leaf function}
\IFRU{Тем не менее, если некая функция не вызывает никаких более функций, в терминологии ARM она называется}
{Nevertheless, if a function never calls any other function, in ARM terminology it is called}
\IT{leaf function}\footnote{\url{http://infocenter.arm.com/help/index.jsp?topic=/com.arm.doc.faqs/ka13785.html}}. 
\IFRU{Как следствие, ``leaf''-функция не использует регистр \LR}
{As a consequence ``leaf'' functions do not use the \LR register}.
\IFRU{А если эта функция небольшая, использует мало регистров, она может не использовать стек вообще}
{And if this function is small and it uses a small number of registers it may not use stack at all}.
\IFRU{Таким образом, в ARM возможен вызов небольших ``leaf'' функций не используя стек}
{Thus, it is possible to call ``leaf'' functions without using stack}.
\IFRU{Это может быть быстрее чем в x86, ведь внешняя память для стека не используется}
{This can be faster than on x86 because external RAM is not used for the stack}
\footnote{\IFRU{Когда-то очень давно, на PDP-11 и VAX, на инструкцию CALL (вызов других функций) могло тратиться
вплоть до 50\% времени, возможно из-за работы с памятью, 
поэтому считалось что много небольших функций это анти-паттерн}
{Some time ago, on PDP-11 and VAX, CALL instruction (calling other functions) was expensive, up to 50\%
of execution time might be spent on it, so it was common sense that big number of small function is anti-pattern}\cite[Chapter 4, Part II]{Raymond:2003:AUP:829549}.}.
\IFRU{Либо, это может быть полезным для тех ситуаций, когда память для стека еще не выделена либо недоступна}
{It can be useful for such situations when memory for the stack is not yet allocated or not available}.


\subsection{\IFRU{Передача параметров для функции}{Function arguments passing}}

\begin{lstlisting}
push arg3
push arg2
push arg1
call f
add esp, 4*3
\end{lstlisting}

\IFRU{Вызываемая функция получает свои параметры также через указатель стека.}
{Callee{\footnote{Function being called}} function get its arguments via stack ponter.}

\IFRU{См.также в соответствующем разделе о способах передачи аргументов через стек}
{See also section about calling conventions}~\ref{sec:callingconventions}.

\IFRU{Важно отметить, что, в общем, никто не заставляет программистов передавать параметры именно через стек,
это не является требованием к исполняемому коду.}
{It is important to note that no one oblige programmers to pass arguments through stack, it is not prerequisite.}

\IFRU{Вы можете делать это совершенно иначе, не используя стек.}
{One could implement any other method not using stack.}

\IFRU{К примеру, можно выделять в куче\footnote{heap в англоязычной литературе} место для аргументов, 
заполнять их и передавать в функцию указатель на это место через \EAX. И это вполне будет работать}
{For example, it is possible to allocate a place for arguments in heap, fill it and pass to a function 
via pointer to this pack in \EAX register. And this will work}
\footnote{\IFRU{Например, в книге Дональда Кнута ``Искусство программирования'', в разделе 1.4.1 
посвященном подпрограммам\cite[раздел 1.4.1]{Knuth:1998:ACP:521463}, 
мы можем прочитать о возможности располагать параметры для вызываемой подпрограммы после инструкции \JMP
передающей управление подпрограмме. Кнут описывает что это было особенно удобно для компьютеров System/360.}
{For example, in ``The Art of Computer Programming'' book by Donald Knuth, 
in section 1.4.1 dedicated to subroutines\cite[section 1.4.1]{Knuth:1998:ACP:521463},
we can read about one way to supply arguments to subroutine is simply to list them after the \JMP instruction
passing control to subroutine. Knuth writes that this method was particularly convenient on System/360.}}.

\IFRU{Однако, так традиционно сложилось, что в x86 и ARM передача аргументов происходит именно через стек.}
{However, it is convenient tradition in x86 and ARM to use stack for this.}


\subsubsection{\IFRU{Хранение локальных переменных}{Local variable storage}}

\IFRU{Функция может выделить для себя некоторое место в стеке для локальных переменных просто отодвинув 
\glslink{stack pointer}{указатель стека} глубже к концу стека.}
{A function could allocate a space in the stack for its local variables just by shifting 
the \gls{stack pointer} towards stack bottom.}

\IFRU{Это снова не является необходимым требованием. Вы можете хранить локальные переменные где угодно. 
Но по традиции всё сложилось так.}
{It is also not a requirement. You could store local variables wherever you like. 
But traditionally it is so.}


\subsection{x86: \IFRU{Функция alloca()}{alloca() function}}
\label{alloca}
\index{\CStandardLibrary!alloca()}
\IFRU{Интересен случай с функцией \TT{alloca()}}
{It is worth noting \TT{alloca()} function.}\footnote{
\IFRU
{В MSVC, реализацию функции можно посмотреть в файлах}
{As of MSVC, function implementation can be found in} 
  \TT{alloca16.asm} 
  \IFRU{и}{and} 
  \TT{chkstk.asm} 
  \IFRU{в}{in} 
  \TT{C:\textbackslash{}Program Files (x86)\textbackslash{}Microsoft Visual Studio 10.0\textbackslash{}VC\textbackslash{}crt\textbackslash{}src\textbackslash{}intel}}. 

\IFRU{Эта функция работает как \TT{malloc()}, но выделяет память прямо в стеке.} 
{This function works like \TT{malloc()} but allocates memory just in stack.}

\IFRU{Память освобождать через \TT{free()} не нужно, так как эпилог функции~\ref{sec:prologepilog} 
вернет \ESP назад в изначальное состояние и выделенная память просто аyнулируется.}
{Allocated memory chunk is not needed to be freed via \TT{free()} function call since 
function epilogue~\ref{sec:prologepilog} shall return value of the \ESP back to initial state and 
allocated memory will be just annuled.} 

\IFRU{Интересна реализация функции \TT{alloca()}.}
{It is worth noting how \TT{alloca()} implemented.}

\IFRU{Эта функция, если упрощенно, просто сдвигает \ESP вглубь стека 
на столько байт сколько вам нужно и возвращает \ESP в качестве указателя на выделенный блок.}
{This function, if to simplify, just shifting \ESP deeply to stack bottom so much bytes you 
need and set \ESP as a pointer to that \IT{allocated} block.}
\IFRU{Попробуем:}{Let's try:}

\lstinputlisting{02_stack/2_1.c}

\IFRU{(Функция \TT{\_snprintf()} работает так же как и \printf, только вместо выдачи результата в 
stdout (т.е., на терминал или в консоль),
записывает его в буфер \TT{buf}. \puts выдает содержимое буфера \TT{buf} в stdout. Конечно, можно было бы
заменить оба этих вызова на один \printf, но мне нужно проиллюстрировать использование небольшого буфера.)}
{(\TT{\_snprintf()} function works just like \printf, but instead dumping result into stdout (e.g., to terminal or 
console), write it to the \TT{buf} buffer. \puts copies \TT{buf} contents to stdout. Of course, these two
function calls might be replaced by one \printf call, but I would like to illustrate small buffer usage.)}

\subsubsection{MSVC}

\IFRU{Компилируем}{Let's compile} (MSVC 2010):

\lstinputlisting[caption=MSVC 2010]{02_stack/2_2_msvc.asm}

\index{Compiler intrinsic}
\IFRU {Единственный параметр в \TT{alloca()} передается через \EAX, а не как обычно через стек}
{The sole \TT{alloca()} argument passed via \EAX (instead of pushing into stack)}
\footnote{\IFRU{Это потому что alloca() это не сколько функция, сколько т.е. compiler intrinsic}{It's because
alloca() is rather compiler intrinsic than usual function}}.
\IFRU{После вызова \TT{alloca()}, \ESP теперь указывает на блок в 600 байт который 
мы можем использовать под \TT{buf}.}
{After \TT{alloca()} call, \ESP is now pointing to the block of 600 bytes and we can 
use it as memory for \TT{buf} array.}

\subsubsection{GCC + \IntelSyntax}

\IFRU{А GCC 4.4.1 обходится без вызова других функций:}
{GCC 4.4.1 can do the same without calling external functions:}

\lstinputlisting[caption=GCC 4.7.3]{\IFRU{02_stack/2_1_gcc_intel_O3_ru.asm}{02_stack/2_1_gcc_intel_O3_en.asm}}

\subsubsection{GCC + \ATTSyntax}

\IFRU{Посмотрим на тот же код, только в синтаксисе AT\&T}{Let's see the same code, but in AT\&T syntax}:

\lstinputlisting[caption=GCC 4.7.3]{02_stack/2_1_gcc_ATT_O3.s}

\index{\ATTSyntax}
\IFRU{Всё то же самое что и в прошлом листинге.}{The same code as in previos listing.}

\IFRU{Обратите внимание что, например}{Please note that, for example}, \TT{movl \$3, 20(\%esp)} 
\IFRU{это аналог}{is analogous to} \TT{mov DWORD PTR [esp+20], 3} \IFRU{в Intel-синтаксисе}{in Intel-syntax} ~--- 
\IFRU{при адресации памяти в виде}{when addressing memory in form} \IT{\IFRU{регистр+смещение}{register+offset}}, 
\IFRU{это записывается в AT\&T синтаксисе как}{it's written in AT\&T syntax as} 
\TT{\IFRU{смещение}{offset}(\%\IFRU{регистр}{register})}.



\subsection{(Windows) SEH}
\index{Windows!Structured Exception Handling}

\IFRU{В стеке хранятся записи SEH (\IT{Structured Exception Handling}) для функции (если имеются)}
{SEH (\IT{Structured Exception Handling}) records are also stored in stack (if needed).}
\footnote{
\IFRU{О SEH: классическая статья Мэтта Питрека}{Classic Matt Pietrek article about SEH}: 
\url{http://www.microsoft.com/msj/0197/Exception/Exception.aspx}}.


\subsection{\RU{Защита от переполнений буфера}\EN{Buffer overflow protection}\PTBR{Proteção contra estouro de buffer}}

\RU{Здесь больше об этом}\EN{More about it here}\PTBR{Mais sobre aqui}~(\myref{subsec:bufferoverflow}).



\input{03_printf/printf}
\section{scanf()}

\IFRU{Теперь попробуем использовать scanf().}{Now let's use scanf().}

\begin{lstlisting}
int main() 
{
	int x;
	printf ("Enter X:\n");

	scanf ("%d", &x);

	printf ("You entered %d...\n", x);

	return 0;
};
\end{lstlisting}

\IFRU
{Да, согласен, использовать \scanf в наши времена для того чтобы спросить у юзера что-то: не самая хорошая идея.
Но я хотел проиллюстрировать передачу указателя на \Tint.}
{OK, I agree, it is not clever to use \scanf today. But I wanted to illustrate passing pointer to \Tint.}

\subsection{x86}

\IFRU{Что получаем на ассемблере компилируя MSVC 2010:}
{What we got after compiling in MSVC 2010:}

\lstinputlisting{04_scanf/4_1_msvc.asm}

\IFRU{Переменная \TT{x} является локальной.}{Variable \TT{x} is local.} 

\IFRU{По стандарту \CCpp она доступна только из этой же функции и ниоткуда более. 
Так получилось, что локальные переменные располагаются в стеке. 
Может быть, можно было бы использовать и другие варианты, но в x86 это традиционно так.}
{\CCpp standard tell us it must be visible only in this function and not from any other point. 
Traditionally, local variables are placed in the stack. 
Probably, there could be other ways, but in x86 it is so.}

\index{x86!\Instructions!PUSH}
\IFRU{Следующая после пролога инструкция \TT{PUSH ECX} не ставит своей целью сохранить 
значение регистра \ECX. 
(Заметьте отсутствие сооветствующей инструкции \TT{POP ECX} в конце функции)}
{Next instruction after function prologue, \TT{PUSH ECX}, hasn't goal to save \ECX state 
(notice absence of corresponding \TT{POP ECX} at the function end).}

\IFRU{Она на самом деле выделяет в стеке 4 байта для хранения \TT{x} в будущем.} 
{In fact, this instruction just allocates 4 bytes on the stack for \TT{x} variable storage.} 

\index{\Stack!\IFRU{Стековый фрейм}{Stack frame}}
\index{x86!\Registers!EBP}
\IFRU{Доступ к \TT{x} будет осуществляться при помощи объявленного макроса \TT{\_x\$} 
(он равен -4) и регистра \EBP указывающего на текущий фрейм.}
{\TT{x} will be accessed with the assistance of the \TT{\_x\$} macro 
(it equals to -4) and the \EBP register pointing to current frame.}

\IFRU{Вообще, во все время исполнения функции, \EBP указывает на текущий фрейм и через \TT{EBP+смещение}
можно иметь доступ как к локальным переменным функции, так и аргументам функции.} 
{Over a span of function execution, \EBP is pointing to current stack frame and it is possible 
to have an access to local variables and function arguments via \TT{EBP+offset}.}

\index{x86!\Registers!ESP}
\IFRU
{Можно было бы использовать \ESP, но он во время исполнения функции постоянно меняется. 
Так что можно сказать что \EBP это \IT{замороженное состояние} \ESP на момент начала исполнения функции.}
{It is also possible to use \ESP, but it's often changing and not very convenient.
So it can be said, the value of the \EBP is \IT{frozen state} of the value of the \ESP at the moment of function execution start.}

\IFRU
{У функции \scanf в нашем примере два аргумента.}{Function \scanf in our example has two arguments.}

\IFRU
{Первый ~--- указатель на строку содержащую \TT{``\%d''} и второй ~--- адрес переменной \TT{x}.} 
{First is pointer to the string containing \TT{``\%d''} and second ~--- address of variable \TT{x}.} 

\index{x86!\Instructions!LEA}
\IFRU{Вначале адрес \TT{x} помещается в регистр \EAX при помощи инструкции \TT{lea eax, DWORD PTR \_x\$[ebp]}.}
{First of all, address of the \TT{x} variable is placed into the \EAX register by \TT{lea eax, DWORD PTR \_x\$[ebp]} instruction}

\IFRU{Инструкция \LEA означает \IT{load effective address}, но со временем она изменила свою функцию}
{\LEA meaning \IT{load effective address} but over a time it changed its primary application}
~\ref{sec:LEA}.

\IFRU{Можно сказать что в данном случае \LEA просто помещает в \EAX результат суммы значения в регистре 
\EBP и макроса \TT{\_x\$}.}
{It can be said, \LEA here just stores sum of the value in the \EBP register and \TT{\_x\$} macro to the \EAX register.}

\IFRU{Это тоже что и}{It is the same as} \TT{lea eax, [ebp-4]}.

\IFRU{Итак, от значения \EBP отнимается $4$ и помещается в \EAX.
Далее значение \EAX заталкивается в стек и вызывается \scanf.}
{So, $4$ subtracting from value in the \EBP register and result is placed to the \EAX register.
And then value in the \EAX register is pushing into stack and \scanf is called.}

\IFRU{После этого вызывается \printf. Первый аргумент вызова которого, строка:} 
{After that, \printf is called. First argument is pointer to string:} \TT{``You entered \%d...\textbackslash{}n''}.

\IFRU{Второй аргумент: \TT{mov ecx, [ebp-4]}, эта инструкция помещает в \ECX не адрес переменной \TT{x}, 
а его значение, что там сейчас находится.}
{Second argument is prepared as: \TT{mov ecx, [ebp-4]},
this instruction places to the \ECX not address of the \TT{x} variable, but its contents.}

\IFRU{Далее значение \ECX заталкивается в стек и вызывается последний \printf.}
{After, value in the \ECX is placed on the stack and the last \printf called.}

\IFRU{Попробуем тоже самое скомпилировать в Linux при помощи GCC 4.4.1:}
{Let's try to compile this code in GCC 4.4.1 under Linux:}

\lstinputlisting{04_scanf/4_1_gcc.asm}

\index{puts() \IFRU{вместо}{instead of} printf()}
\IFRU{GCC заменил первый вызов \printf на \puts, почему это было сделано, 
уже было описано раннее~\ref{puts}.}
{GCC replaced first the \printf call to the \puts, it was already described~\ref{puts} 
why it was done.}

% TODO: rewrite
%\IFRU
%{Почему \scanf переименовали в \TT{\_\_\_isoc99\_scanf}, я честно говоря, пока не знаю.}
%{Why \scanf is renamed to \TT{\_\_\_isoc99\_scanf}, I do not know yet.}

\IFRU{Далее все как и прежде ~--- параметры заталкиваются через стек при помощи \MOV.}
{As before ~--- arguments are placed on the stack by \MOV instruction.}


\subsection{ARM}

\subsubsection{\OptimizingKeil + \ThumbMode}

\begin{lstlisting}
.text:00000042             scanf_main
.text:00000042
.text:00000042             var_8           = -8
.text:00000042
.text:00000042 08 B5                       PUSH    {R3,LR}
.text:00000044 A9 A0                       ADR     R0, aEnterX     ; "Enter X:\n"
.text:00000046 06 F0 D3 F8                 BL      __2printf
.text:0000004A 69 46                       MOV     R1, SP
.text:0000004C AA A0                       ADR     R0, aD          ; "%d"
.text:0000004E 06 F0 CD F8                 BL      __0scanf
.text:00000052 00 99                       LDR     R1, [SP,#8+var_8]
.text:00000054 A9 A0                       ADR     R0, aYouEnteredD___ ; "You entered %d...\n"
.text:00000056 06 F0 CB F8                 BL      __2printf
.text:0000005A 00 20                       MOVS    R0, #0
.text:0000005C 08 BD                       POP     {R3,PC}
\end{lstlisting}

Чтобы \scanf мог вернуть значение, нужно передать ему указатель на переменную типа \Tint. \Tint ~--- 32-битное 
значение, для его хранения нужно только 4 байта и оно помещается в регистр.
Локальная переменная \TT{x} выделяется в стеке, \IDA наименовала её \IT{var\_8}, место для нее выделять
не обязательно, т.к., указатель стека \SP уже указывает на место, свободное для использования.
Так что указатель \SP копируется в регистр \TT{R1} и вместе с format-строкой, передается в \scanf.
Позже, при помощи инструкции \TT{LDR}, это значение перемещается из стека в регистр R1, чтобы быть переданным
в \printf.

Варианты скомпилированные для ARM-режима процессора, а также варианты скомпилированные при помощи Xcode,
не очень отличаются от этого, так что, мы можем пропустить их здесь.



\subsection{\IFRU{Глобальные переменные}{Global variables}}
\index{\IFRU{Глобальные переменные}{Global variables}}
\subsubsection{x86}

\IFRU
{А что если переменная \TT{x} из предыдущего примера будет глобальной переменной а не локальной? 
Тогда к ней смогут обращаться из любого другого места, а не только из тела функции. 
Это снова не очень хорошая практика программирования, но ради примера мы можем себе это позволить.}
{What if \TT{x} variable from previous example will not be local but global variable? 
Then it will be accessible from any point, not only from function body. 
It is not very good programming practice, but for the sake of experiment we could do this.}

\lstinputlisting{04_scanf/4_2_msvc.asm}

\IFRU
{Ничего особенного, в целом. Теперь \TT{x} объявлена в сегменте \TT{\_DATA}. 
Память для нее в стеке более не выделяется. Все обращения к ней происходит не через стек, а уже напрямую. 
Её значение неопределено. 
Это означает, что память под нее будет выделена, но ни компилятор, ни \ac{ОС} не будет заботиться о том, 
что там будет лежать на момент старта функции \main.
В качестве домашнего задания, попробуйте объявить большой неопределенный массив и посмотреть 
что там будет лежать после загрузки.}
{Now \TT{x} variable is defined in the \TT{\_DATA} segment. 
Memory in local stack is not allocated anymore. 
All accesses to it are not via stack but directly to process memory. 
Its value is not defined. 
This means that memory will be allocated by \ac{OS}, but not compiler, 
neither \ac{OS} will not take care about its initial value at the moment of 
the \main function start.
As experiment, try to declare large array and see what will it contain after 
program loading.}

\IFRU{Попробуем изменить объявление этой переменной:}
{Now let's assign value to variable explicitly:}

\begin{lstlisting}
int x=10; // default value
\end{lstlisting}

\IFRU{Выйдет в итоге:}{We got:}

\begin{lstlisting}
_DATA	SEGMENT
_x	DD	0aH

...
\end{lstlisting}

\IFRU{Здесь уже по месту этой переменной записано \TT{0xA} с типом DD (dword = 32 бита).}
{Here we see value \TT{0xA} of DWORD type (DD meaning DWORD = 32 bit).}

\IFRU{Если вы откроете скомпилированный .exe-файл в \IDA, то увидите что \IT{x} 
находится аккурат в начале сегмента \TT{\_DATA}, после этой переменной будут текстовые строки.}
{If you will open compiled .exe in \IDA, you will see the \IT{x} variable placed at the beginning of 
the \TT{\_DATA} segment, and after you'll see text strings.}

\IFRU{А вот если вы откроете в \IDA, .exe скомплированный в прошлом примере, 
где значение \IT{x} неопределено, то в IDA вы увидите:}
{If you will open compiled .exe in \IDA from previous example where \IT{x} value is not defined, 
you'll see something like this:}

\begin{lstlisting}
.data:0040FA80 _x              dd ?                    ; DATA XREF: _main+10
.data:0040FA80                                         ; _main+22
.data:0040FA84 dword_40FA84    dd ?                    ; DATA XREF: _memset+1E
.data:0040FA84                                         ; unknown_libname_1+28
.data:0040FA88 dword_40FA88    dd ?                    ; DATA XREF: ___sbh_find_block+5
.data:0040FA88                                         ; ___sbh_free_block+2BC
.data:0040FA8C ; LPVOID lpMem
.data:0040FA8C lpMem           dd ?                    ; DATA XREF: ___sbh_find_block+B
.data:0040FA8C                                         ; ___sbh_free_block+2CA
.data:0040FA90 dword_40FA90    dd ?                    ; DATA XREF: _V6_HeapAlloc+13
.data:0040FA90                                         ; __calloc_impl+72
.data:0040FA94 dword_40FA94    dd ?                    ; DATA XREF: ___sbh_free_block+2FE
\end{lstlisting}

\IFRU{\TT{\_x} обозначен как \TT{?}, наряду с другими переменными не требующими инициализции. 
Это означает, что при загрузке .exe в память, место под все это выделено будет. 
Но в самом .exe ничего этого нет. Неинициализированные переменные не занимают места в исполняемых файлах. Удобно для больших массивов, например.}
{\TT{\_x} marked as \TT{?} among another variables not required to be initialized. 
This means that after loading .exe to memory, a space for all these variables will be 
allocated and some random garbage will be here. 
But in an .exe file these not initialized variables are not occupy anything. 
It is suitable for large arrays, for example.}

\index{ELF}
\IFRU{В Linux все также почти. За исключением того что если значение \TT{x} не определено, 
то эта переменная будет находится в сегменте \TT{\_bss}. В ELF\footnote{Формат исполняемых файлов, использующийся в Linux и некоторых других *NIX} этот сегмент имеет такие аттрибуты:}
{It is almost the same in Linux, except segment names and properties: 
not initialized variables are located in the \TT{\_bss} segment. 
In ELF\footnote{Executable file format widely used in *NIX system including Linux} 
file format this segment has such attributes:}

\begin{lstlisting}
; Segment type: Uninitialized
; Segment permissions: Read/Write
\end{lstlisting}

\IFRU{Ну а если сделать присвоение этой переменной значения $10$, то она будет находится 
в сегменте \TT{\_data},
это сегмент с такими аттрибутами:}
{If to assign some value to variable, e.g. $10$, it will be placed in the \TT{\_data} segment, 
this is segment with such attributes:}

\begin{lstlisting}
; Segment type: Pure data
; Segment permissions: Read/Write
\end{lstlisting}

\subsubsection{ARM: \OptimizingKeil + \ThumbMode}

\begin{lstlisting}
.text:00000000 ; Segment type: Pure code
.text:00000000                 AREA .text, CODE
...
.text:00000000 main
.text:00000000                 PUSH    {R4,LR}
.text:00000002                 ADR     R0, aEnterX     ; "Enter X:\n"
.text:00000004                 BL      __2printf
.text:00000008                 LDR     R1, =x
.text:0000000A                 ADR     R0, aD          ; "%d"
.text:0000000C                 BL      __0scanf
.text:00000010                 LDR     R0, =x
.text:00000012                 LDR     R1, [R0]
.text:00000014                 ADR     R0, aYouEnteredD___ ; "You entered %d...\n"
.text:00000016                 BL      __2printf
.text:0000001A                 MOVS    R0, #0
.text:0000001C                 POP     {R4,PC}
...
.text:00000020 aEnterX         DCB "Enter X:",0xA,0    ; DATA XREF: main+2
.text:0000002A                 DCB    0
.text:0000002B                 DCB    0
.text:0000002C off_2C          DCD x                   ; DATA XREF: main+8
.text:0000002C                                         ; main+10
.text:00000030 aD              DCB "%d",0              ; DATA XREF: main+A
.text:00000033                 DCB    0
.text:00000034 aYouEnteredD___ DCB "You entered %d...",0xA,0 ; DATA XREF: main+14
.text:00000047                 DCB 0
.text:00000047 ; .text         ends
.text:00000047
...
.data:00000048 ; Segment type: Pure data
.data:00000048                 AREA .data, DATA
.data:00000048                 ; ORG 0x48
.data:00000048                 EXPORT x
.data:00000048 x               DCD 0xA                 ; DATA XREF: main+8
.data:00000048                                         ; main+10
.data:00000048 ; .data         ends
\end{lstlisting}

Итак, переменная \TT{x} теперь глобальная, и она расположена, почему-то, в другом сегменте данных (\IT{.data}). 
Можно спросить, почему текстовые строки расположены в сегменте кода (\IT{.text}) а \TT{x} нельзя было разместить
тут же? Потому что эта переменная, и как следует из определения, она может меняться. Сегмент кода нередко может 
быть расположен в ПЗУ микроконтроллера (не забывайте, мы сейчас имеем дело с embedded-микроэлектроникой),
а изменяемые переменные --- в ОЗУ.
Нередко, ОЗУ дороже чем ПЗУ, так что хранить в нем неизменяемые данные, когда в наличии есть ПЗУ, не экономно.

Далее, мы видим, в сегменте кода, хранится указатель на переменную \TT{x} (\TT{off\_2C}) и вообще, все операции 
с ним, происходят через этот указатель.
Это связано с тем что переменная \TT{x} может быть расположена где-то довольно далеко от данного участка кода
и её адрес нужно сохранить в переменной рядом с кодом.
Инструкция \TT{LDR} в thumb-режиме может адресовать только переменные в пределах вплоть до 1020 байт от места
где она находится. Эта же инструкция в ARM-режиме --- переменные в пределах $\pm{}4095$, таким образом, 
адрес глобальной переменной \TT{x} нужно иметь где-то рядом, ведь нет никакой гарантии, что саму переменную
получится хранить где-то рядом, она может быть даже в другом чипе памяти!

Еще одна вещь: если переменную объявить как \IT{const}, то компилятор Keil разместит её в сегменте \TT{.constdata}.
Должно быть, впоследствии, линкер и этот сегмент сможет разместить в ПЗУ.





\subsection{\IFRU{Проверка результата scanf()}{scanf() result checking}}

\subsubsection{x86}

\IFRU {Как я уже упоминал, использовать \scanf в наше время это слегка старомодно. 
Но если уж жизнь заставила этим заниматься, нужно хотя бы проверять, сработал ли \scanf 
правильно или пользователь ввел вместо числа что-то другое, что \scanf не смог трактовать как число.}
{As I noticed before, it is slightly old-fashioned to use \scanf today. 
But if we have to, we need at least check if \scanf finished correctly without error.}

\lstinputlisting{04_scanf/retval_check.c}

\IFRU{По стандарту}{By standard}, \scanf\footnote{\href{http://msdn.microsoft.com/en-us/library/9y6s16x1(VS.71).aspx}{MSDN: scanf, wscanf}} 
\IFRU{возвращает количество успешно полученных значений.}{function returns number of fields it successfully read.}

\IFRU{В нашем случае, если все успешно и пользователь ввел таки некое число, \scanf вернет 1. 
А если нет, то 0 или EOF.} 
{In our case, if everything went fine and user entered a number, 
\scanf will return 1 or 0 or EOF in case of error.}

\IFRU{Я добавил код проверяющий результат \scanf и в случае ошибки, он сообщает пользователю что-то другое.}
{I added C code for \scanf result checking and printing error message in case of error.}

\IFRU{Вот, что выходит на ассемблере}{What we got in assembly language} (MSVC 2010):

\lstinputlisting{04_scanf/retval_check_MSVC.asm}

\index{x86!\Registers!EAX}
\IFRU{Для того чтобы вызывающая функция имела доступ к результату вызываемой функции, 
вызываемая функция (в нашем случае \scanf) оставляет это значение в регистре \EAX.}
{Caller function (\main) must have access to the result of callee function (\scanf), 
so callee leaves this value in the \EAX register.}

\index{x86!\Instructions!CMP}
\IFRU{Мы проверяем его инструкцией \TT{CMP EAX, 1} (\IT{CoMPare}), то есть, 
сравниваем значение в \EAX с 1.}
{After, we check it with the help of instruction \TT{CMP EAX, 1} (\IT{CoMPare}),
in other words, we compare value in the \EAX register with $1$.} 

\index{x86!\Instructions!JNE}
\IFRU{Следующий за инструкцией \CMP: условный переход \JNE. 
Это означает \IT{Jump if Not Equal}, то есть, условный переход \IT{если не равно}.}
{\JNE conditional jump follows \CMP instruction. \JNE means \IT{Jump if Not Equal}.}

\IFRU{Итак, если \EAX не равен 1, то \JNE заставит перейти процессор 
по адресу указанном в операнде \JNE, у нас это \TT{\$LN2@main}.}
{So, if value in the \EAX register not equals to $1$, then the processor will pass execution to the 
address mentioned in operand of \JNE, in our case it is \TT{\$LN2@main}.}
\IFRU
{Передав управление по этому адресу, процессор как раз начнет исполнять вызов \printf с 
аргументом \TT{``What you entered? Huh?''}.}
{Passing control to this address, microprocesor will execute function \printf 
with argument \TT{``What you entered? Huh?''}.}
\IFRU
{Но если все нормально, перехода не случится, и исполнится другой \printf с двумя аргументами: 
\TT{'You entered \%d...'} и значением переменной \TT{x}.}
{But if everything is fine, conditional jump will not be taken, and another \printf call 
will be executed, with two arguments: \TT{'You entered \%d...'} and value of variable \TT{x}. }

\index{x86!\Instructions!XOR}
\index{\CLanguageElements!return}
\IFRU {А для того чтобы после этого вызова не исполнился сразу второй вызов \printf, 
после него имеется инструкция \JMP, безусловный переход, он отправит процессор на место аккурат 
после второго \printf и перед инструкцией \TT{XOR EAX, EAX}, которая собственно \TT{return 0}.}
{Since second subsequent \printf not needed to be executed, there is \JMP after (unconditional jump),
it will pass control to the point after second \printf and before \TT{XOR EAX, EAX} instruction, 
which implement \TT{return 0}.}

\index{x86!\Registers!\Flags}
\IFRU{Итак, можно сказать, что в подавляющих случаях сравнение какой либо переменной с чем-то другим 
происходит при помощи пары инструкций \CMP и \Jcc, где \IT{cc} это \IT{condition code}.}
{So, it can be said that comparing a value with another is \IT{usually} implemented
by \CMP/\Jcc instructions pair, where \IT{cc} is \IT{condition code}.}
\IFRU{\CMP сравнивает два значения и выставляет 
флаги процессора\footnote{См.также о флагах x86-процессора: \url{http://en.wikipedia.org/wiki/FLAGS_register_(computing)}.}.}
{\CMP comparing two values and set 
processor flags\footnote{About x86 flags, see also: \url{http://en.wikipedia.org/wiki/FLAGS_register_(computing)}.}.}
\IFRU
{\Jcc проверяет нужные ему флаги и выполняет переход по указанному адресу (или не выполняет).}
{\Jcc check flags needed to be checked and pass control to mentioned address (or not pass).}

\index{x86!\Instructions!CMP}
\index{x86!\Instructions!SUB}
\label{CMPandSUB}
\IFRU{Но на самом деле, как это не парадоксально поначалу звучит, \CMP это почти то же самое что и 
инструкция \SUB, которая отнимает числа одно от другого.}
{But in fact, this could be perceived paradoxical, but \CMP instruction is in fact \SUB (subtract).}
\IFRU{Все арифметические инструкции также выставляют флаги в соответствии с результатом, не только \CMP.}
{All arithmetic instructions set processor flags too, not only \CMP.}
\IFRU{Если мы сравним 1 и 1, от единицы отнимется единица, получится $0$, и выставится флаг 
\ZF (\IT{zero flag}), означающий что последний полученный результат был $0$.}
{If we compare 1 and 1, $1-1$ will be $0$ in result, \ZF flag will be set (meaning the last result was $0$).}
\IFRU{Ни при каких других значениях \EAX, флаг \ZF выставлен не будет, кроме тех, когда операнды равны друг другу.}
{There is no any other circumstance when it is possible except when operands are equal.}
\index{x86!\Instructions!JNE}
\index{x86!\Registers!ZF}
\IFRU{Инструкция \JNE проверяет только флаг \ZF, и совершает переход только если флаг не поднят. 
Фактически, \JNE это синоним инструкции \JNZ (\IT{Jump if Not Zero}).}
{\JNE checks only \ZF flag and jumping only if it is not set. 
\JNE is in fact a synonym of \JNZ (\IT{Jump if Not Zero}) instruction.}
\IFRU{Ассемблер транслирует обе инструкции в один и тот же опкод.}
{Assembler translating both \JNE and \JNZ instructions into one single opcode.}
\IFRU
{Таким образом, можно \CMP заменить на \SUB и все будет работать также, но разница в том что \SUB 
все-таки испортит значение в первом операнде. \CMP это \IT{SUB без сохранения результата}.}
{So, \CMP instruction can be replaced to \SUB instruction and almost everything will be fine,
but the difference is in 
the \SUB alter the value of the first operand.
\CMP is \IT{``SUB without saving result''}.}

\IFRU
{Код созданный при помощи GCC 4.4.1 в Linux практически такой же, если не считать мелких отличий, 
которые мы уже рассмотрели раннее.}
{Code generated by GCC 4.4.1 in Linux is almost the same, except differences we already considered.}

\subsubsection{ARM: \OptimizingKeil + \ThumbMode}

\lstinputlisting{04_scanf/checking_retval_ARM_Keil_thumb_O3.asm}

\IFRU{Новые инструкции здесь для нас: \CMP и \TT{BEQ}.}
{New instructions here are \CMP and \TT{BEQ}.}

\CMP \IFRU{аналогична той что в x86, она отнимает один аргумент от второго и сохраняет флаги.}
{is similar to the x86 instruction, it subtracts one argument from another and save flags.}
% TODO: в мануале ARM $op1 + NOT(op2) + 1$ вместо вычитания

\TT{BEQ} (\IT{Branch Equal}) \IFRU{совершает переход по другому адресу, 
если операнды при сравнении были равны, 
либо если результат последнего вычисления был ноль, либо если флаг Z равен $1$.}
{is jumping to another address if operands while comparing were equal to each other, or,
if result of last computation was zero, or if Z flag is $1$.}
\IFRU{То же что и \JZ в}{Same thing as \JZ in} x86.

\IFRU{Всё остальное просто: исполнение разветвляется на две ветки, затем они сходятся там, 
где в \Rzero записывается $0$ как возвращаемое из функции значение и происходит выход из функции.}
{Everything else is simple: execution flow is forking into two branches, then the branches are 
converging at the place
where $0$ is written into \Rzero, as a value returned from the function, and then function finishing.}




\section{\IFRU{Передача параметров через стек}{Passing arguments via stack}}

\IFRU{Как мы уже успели заметить, вызывающая функция передает аргументы для вызываемой через стек. 
А как вызываемая функция имеет к ним доступ?}
{Now we figured out that caller function passing arguments to callee via stack. 
But how callee\footnote{function being called} access them?}

\lstinputlisting{05_passing_arguments/ex.c}

\subsection{x86: \IFRU{3 аргумента}{3 arguments}}

\subsubsection{MSVC}

\IFRU{Компилируем при помощи MSVC 2010 Express, и в итоге получим:}
{Let's compile it by MSVC 2010 Express and we got:}

\begin{lstlisting}
$SG3830	DB	'a=%d; b=%d; c=%d', 00H

...

	push	3
	push	2
	push	1
	push	OFFSET $SG3830
	call	_printf
	add	esp, 16					; 00000010H
\end{lstlisting}

\IFRU{Все почти то же, за исключением того, что теперь видно, что аргументы для \printf заталкиваются в стек в обратном порядке: самый первый аргумент заталкивается последним.}
{Almost the same, but now we can see the \printf arguments are pushing into stack in reverse order: and the first argument is pushing in as the last one.}

\IFRU{Кстати, вспомним что переменные типа \Tint в 32-битной системе, как известно, имеет ширину 32 бита, это 4 байта}
{By the way, variables of \Tint type in 32-bit environment has 32-bit width that is 4 bytes}.

\IFRU{Итак, у нас всего 4 аргумента. $4*4 = 16$ ~--- именно 16 байт занимают в стеке указатель на строку плюс еще 3 числа типа \Tint.}
{So, we got here 4 arguments. $4*4 = 16$~---they occupy exactly 16 bytes in the stack: 32-bit pointer to string and 3 number of \Tint type.}

\index{x86!\Instructions!ADD}
\index{x86!\Registers!ESP}
\index{cdecl}
\IFRU{Когда при помощи инструкции \TT{``ADD ESP, X''} корректируется \glslink{stack pointer}{указатель стека} \ESP 
после вызова какой-либо функции, зачастую можно сделать вывод о том, сколько аргументов 
у вызываемой функции было, разделив X на 4.}
{When \gls{stack pointer} (the \ESP register) is corrected by \TT{``ADD ESP, X''}
instruction after a function 
call, often, the number of function arguments could be deduced here: just divide X by 4.}

\IFRU{Конечно, это относится только к cdecl-методу передачи аргументов через стек.}
{Of course, this is related only to \IT{cdecl} calling convention.}

\IFRU{См. также в соответствующем разделе о способах передачи аргументов через стек}
{See also section about calling conventions}~(\ref{sec:callingconventions}).

\IFRU{Иногда бывает так, что подряд идут несколько вызовов разных функций, 
но стек корректируется только один раз, после последнего вызова:}
{It is also possible for compiler to merge several \TT{``ADD ESP, X''} instructions into one, after last call:}

\begin{lstlisting}
push a1
push a2
call ...
...
push a1
call ...
...
push a1
push a2
push a3
call ...
add esp, 24
\end{lstlisting}

\subsubsection{MSVC \AndENRU \olly}
\index{\olly}

\IFRU{Попробуем этот же пример в}{Now let's try to load this example in} \olly.
\IFRU{Это один из наиболее популярных win32-отладчиков user-режима}{It is one of the most 
popular user-land win32 debugger}.
\IFRU{Мы можем компилировать наш пример в}{We can try to compile our example in} MSVC 2012 
\IFRU{с опцией}{with} \TT{/MD} \IFRU{что означает, линковать с библиотекой}{option, meaning, to link 
against} \TT{MSVCR*.DLL},
\IFRU{чтобы импортируемые ф-ции были хорошо видны в отладчике}{so we will able to see imported 
functions clearly in debugger}.

\IFRU{Затем загружаем исполняемый файл в}{Then load executable in} \olly.
\IFRU{Самый первый брякпойнт в}{The very first breakpoint is in} \TT{ntdll.dll}, \IFRU{нажмите}{press} 
F9 (\IFRU{запустить}{run}).
\IFRU{Второй брякпойнт в}{The second breakpoint is in} \ac{CRT}-\IFRU{коде}{code}.
\IFRU{Теперь мы должны найти ф-цию}{Now we should find} \main\EN{ function}.

\IFRU{Найдите этот код скроллируя окно кода до самого верха (MSVC располагает ф-цию \main в самом начале
секции кода)}{Find this code by scrolling the code to the very bottom (MSVC allocates \main function at
the very beginning of the code section)}: 
\figname \ref{fig:printf3_olly_1}.

\IFRU{Кликните на инструкции}{Click on} \TT{PUSH EBP}\IFRU{, нажмите}{ instruction, press} F2 
(\IFRU{установка брякпойнта}{set breakpoint}) \IFRU{и нажмите}{and press} F9 (\IFRU{запустить}{run}).
\IFRU{Нам нужно произвести все эти манипуляции, чтобы пропустить \ac{CRT}-код, потому что нам он пока
не интересен}{We need to do these manupulations in order to skip \ac{CRT}-code, because, we don't really 
interesting in it yet}.

\IFRU{Нажмите}{Press} F8 (\stepover) 6 \IFRU{раз, т.е., пропустить
6 инструкций}{times, i.e., skip 6 instructions}: \figname \ref{fig:printf3_olly_2}.

\IFRU{Теперь}{Now the} \PC \IFRU{указывает на инструкцию}{points to the}
\TT{CALL printf}\EN{ instruction}.
\olly, \IFRU{как и другие отладчики, подсвечивает регистры со значениями, которые изменились}
{like other debuggers, highlights value of registers which were changed}.
\IFRU{Так что, каждый раз, когда мы нажимаем}{So each time you press F8}, \EIP 
\IFRU{изменяется и его значение подсвечивается красным}{is changing and its value looking red}.
\ESP \IFRU{также меняется, потому что значения заталкиваются в стек}{is changing as well, 
because values are pushed into the stack}.

\IFRU{Где находятся эти значения в стеке}{Where are the values in the stack}?
\IFRU{Посмотрите на правое/нижнее окно в отладчике}{Take a look into right/bottom window of debugger}:

\begin{figure}[H]
\centering
\includegraphics[scale=0.66]{patterns/03_printf/olly3_stack.png}
\caption{\olly: \IFRU{стек, после того как значения там сохранены}{stack after values pushed}
(\IFRU{я сделал здесь округлую красную пометку в графическом редакторе}{I made round red mark 
here in graphics editor})}
\end{figure}

\IFRU{Так что здесь видно 3 столбца: адрес в стеке, значение в стеке и еще дополнительный комментарий
от \olly}{So we can see there 3 columns: address in the stack, 
value in the stack and some additional \olly comments}. 
\olly \IFRU{понимает}{understands} \printf\IFRU{-строки}{-like strings}, 
\IFRU{так что он показывает здесь и строку и 3 значения \IT{привязанных} к ней}{so it reports the 
string here and 3 values \IT{attached} to it}.

\IFRU{Нажмите}{Press} F8 (\stepover).

\IFRU{В коносил мы видим вывод}{In the console we'll see the output}:

\begin{figure}[H]
\centering
\includegraphics[scale=0.66]{patterns/03_printf/olly3_console.png}
\caption{\RU{Ф-ция }\printf \IFRU{исполнилась}{function executed}}
\end{figure}

\IFRU{Посмотрим, как изменились регистры и состояние стека}{Let's see how registers and stack state 
are changed}: \figname \ref{fig:printf3_olly_3}.

\RU{Регистр }\EAX \IFRU{теперь содержит}{register now contains} \TT{0xD} (13).
That's correct, \printf returns number of characters printed.
\RU{Значение }\EIP \IFRU{изменилось: действительно, теперь здесь адрес инструкции после}
{value is changed: indeed, now there is address of the instruction after} \TT{CALL printf}.
\RU{Значения регистров }\ECX \AndENRU \EDX \IFRU{также изменились}{values are changed as well}.
\IFRU{Очевидно, внутренности ф-ции \printf используют их для каких-то своих нужд}{Apparently, 
\printf function's hidden machinery used them for its own needs}.

\IFRU{Очень важный момент в том что значение \ESP не изменилось. И состояние стека также!}
{A very important thing is that \ESP value is not changed. And stack state too!}
\IFRU{Мы ясно видим здесь и строку формата и соответствующие ей 3 значения, они все еще здесь}
{We clearly see that format string and corresponding 3 values are still there}.
\IFRU{Действительно, по соглашению вызовов \IT{cdecl}, вызывающая ф-ция не очищает аргументы из стека}
{Indeed, that's \IT{cdecl} calling convention, calling function doesn't clear arguments in stack}.
\IFRU{Это должна делать вызывающая ф-ция}{It's caller's duty to do so}.

\IFRU{Нажмите}{Press} F8 \IFRU{снова, чтобы исполнилась инструкция}{again to execute} 
\TT{ADD ESP, 10}\EN{ instruction}: \figname \ref{fig:printf3_olly_4}.

\ESP \IFRU{изменился, но значения все еще в стеке}{is changed, but values are still in the stack}!
\IFRU{Конечно, никому не нужно заполнять эти значения нулями или что-то в этом роде}{Yes, 
of course, no one needs to fill these values by zero or something like that}.
\IFRU{Потому что всё что выше указателя стека}{Because, everything above stack pointer} (\SP) 
\IFRU{это}{is} \IT{\IFRU{шум}{noise}} \OrENRU \IT{\IFRU{мусор}{garbage}}, \IFRU{это всё не имеет
особой ценности}{it has no value at all}.
\IFRU{Было бы очень затратно по времени очищать ненужные элементы стека, к тому же, никому это и не 
нужно}{It would be time consuming to clear unused stack entries, besides, no one really needs to}.

\begin{figure}[H]
\centering
\includegraphics[scale=0.66]{patterns/03_printf/olly3_1.png}
\caption{\olly: \IFRU{самое начало ф-ции}{the very start of the} \main\EN{ function}}
\label{fig:printf3_olly_1}
\end{figure}

\begin{figure}[H]
\centering
\includegraphics[scale=0.66]{patterns/03_printf/olly3_2.png}
\caption{\olly: \IFRU{перед исполнением}{before} \printf\EN{ execution}}
\label{fig:printf3_olly_2}
\end{figure}

\begin{figure}[H]
\centering
\includegraphics[scale=0.66]{patterns/03_printf/olly3_3.png}
\caption{\olly: \IFRU{после исполнения}{after} \printf\EN{ execution}}
\label{fig:printf3_olly_3}
\end{figure}

\begin{figure}[H]
\centering
\includegraphics[scale=0.66]{patterns/03_printf/olly3_4.png}
\caption{\olly: \IFRU{после исполнения инструкции}{after} \TT{ADD ESP, 10}\EN{ instruction execution}}
\label{fig:printf3_olly_4}
\end{figure}

\subsubsection{GCC}

\IFRU{Скомпилируем то же самое в Linux при помощи GCC 4.4.1 и посмотрим в \IDA что вышло:}
{Now let's compile the same in Linux by GCC 4.4.1 and take a look in \IDA what we got:}

\begin{lstlisting}
main            proc near

var_10          = dword ptr -10h
var_C           = dword ptr -0Ch
var_8           = dword ptr -8
var_4           = dword ptr -4

                push    ebp
                mov     ebp, esp
                and     esp, 0FFFFFFF0h
                sub     esp, 10h
                mov     eax, offset aADBDCD ; "a=%d; b=%d; c=%d"
                mov     [esp+10h+var_4], 3
                mov     [esp+10h+var_8], 2
                mov     [esp+10h+var_C], 1
                mov     [esp+10h+var_10], eax
                call    _printf
                mov     eax, 0
                leave
                retn
main            endp
\end{lstlisting}

\IFRU{Можно сказать, что этот короткий код, созданный GCC, отличается от кода MSVC только способом помещения 
значений в стек.
Здесь GCC снова работает со стеком напрямую без \PUSH/\POP.}
{It can be said, the difference between code by MSVC and GCC is only in method of placing arguments on the stack.
Here GCC working directly with stack without \PUSH/\POP.}


\subsection{ARM}

\subsubsection{\NonOptimizingKeil + \ARMMode}

\begin{lstlisting}
.text:000000A4 00 30 A0 E1                 MOV     R3, R0
.text:000000A8 93 21 20 E0                 MLA     R0, R3, R1, R2
.text:000000AC 1E FF 2F E1                 BX      LR
...
.text:000000B0             main
.text:000000B0 10 40 2D E9                 STMFD   SP!, {R4,LR}
.text:000000B4 03 20 A0 E3                 MOV     R2, #3
.text:000000B8 02 10 A0 E3                 MOV     R1, #2
.text:000000BC 01 00 A0 E3                 MOV     R0, #1
.text:000000C0 F7 FF FF EB                 BL      f
.text:000000C4 00 40 A0 E1                 MOV     R4, R0
.text:000000C8 04 10 A0 E1                 MOV     R1, R4
.text:000000CC 5A 0F 8F E2                 ADR     R0, aD_0        ; "%d\n"
.text:000000D0 E3 18 00 EB                 BL      __2printf
.text:000000D4 00 00 A0 E3                 MOV     R0, #0
.text:000000D8 10 80 BD E8                 LDMFD   SP!, {R4,PC}
\end{lstlisting}

\IFRU{В функции \main просто вызываются две функции, в первую (\TT{f}) передается три значения.}
{In \main function, two other functions are simply called, and three values are passed to the 
first one (\TT{f}).}

\IFRU{Как я уже упоминал, первые 4 значения, в ARM обычно передаются в первых 4-х регистрах}
{As I mentioned before, in ARM, first 4 values are usually passed in first 4 registers} (\Rzero-\Rthree).

\IFRU{Функция }{}\TT{f}\IFRU{, как видно, использует три первых регистра (\Rzero-\Rtwo) как аргументы.}
{function, as it seems, use first 3 registers (\Rzero-\Rtwo) as arguments.}

\IFRU{Инструкция }{}\TT{MLA} (\IT{Multiply Accumulate}) \IFRU{перемножает два первых операнда (\Rthree и \Rone), 
прибавляет к произведению
третий операнд (\Rtwo) и помещает результат в нулевой операнд (\Rzero), через который, по стандарту, 
возвращаются значения функций.}
{instruction multiplicates two first operands (\Rthree and \Rone), adds third operand (\Rtwo) to product and places
result into zeroth operand (\Rzero), via which, by standard, values are returned from functions.}

\IFRU{Умножение и сложение одновременно}{Multiplication and addition at once}\footnote{\WPMAO} 
(\IT{Fused multiply–add}) \IFRU{это много где применяемая операция, кстати, аналогичной
инструкции в x86 нет}{is very useful operation, by the way, there are no such instruction in x86}, 
\IFRU{если не считать новых FMA-инструкций}{if not to count new FMA-instruction}\footnote{\url{https://en.wikipedia.org/wiki/FMA_instruction_set}} \IFRU{в}{in} SIMD.

\IFRU{Самая первая инструкция}{The very first} \TT{MOV R3, R0}, \IFRU{по видимому, избыточна (можно было бы обойтись только одной инструкцией \TT{MLA})}
{instruction, as it seems, redundant (single \TT{MLA} instruction could be used here instead)}, 
\IFRU{компилятор не оптимизировал её, ведь, это компиляция без оптимизации}{compiler wasn't optimized it,
because, this is non-optimizing compilation}.

\IFRU{Инструкция \TT{BX} возвращает управление по адресу записанному в \LR и, если нужно, 
переключает режимы процессора с thumb на ARM или наоборот.}
{\TT{BX} instruction returns control to the address stored in \LR and, if need, switches processor mode from
thumb to ARM or vice versa.}
\IFRU{Это может быть необходимым потому, что, как мы видим, 
функции \TT{f} неизвестно, из какого кода она будет вызываться, из ARM или thumb.}
{This can be necessary because, as we can see, \TT{f} function is not aware, from which code it may be
called, from ARM or thumb.}
\IFRU{Поэтому, если она будет вызываться из кода thumb, \TT{BX} не только вернет
управление в вызывающую функцию, но также переключит процессор в режим thumb.}
{This, if it will be called from thumb code, \TT{BX} will not only return control to the calling function,
but also will switch processor mode to thumb mode.}
\IFRU{Либо не переключит, если функция вызывалась из кода для режима ARM.}
{Or not switch, if the function was called from ARM code.}

\subsubsection{\OptimizingKeil + \ARMMode}

\begin{lstlisting}
.text:00000098             f
.text:00000098 91 20 20 E0                 MLA     R0, R1, R0, R2
.text:0000009C 1E FF 2F E1                 BX      LR
\end{lstlisting}

\IFRU{А вот и функция \TT{f} скомпилированная компилятором Keil в режиме полной оптимизации}
{And here is \TT{f} function compiled by Keil compiler in full optimization mode} (\Othree).
\IFRU{Инструкция \MOV была соптимизирована и теперь \TT{MLA} использует все входящие регистры 
и помещает результат в \Rzero, как раз, где вызываемая функция будет его читать и использовать.}
{\MOV instruction was optimized and now \TT{MLA} uses all input registers and place result into \Rzero, 
exactly where calling function will read it and use.}

\subsubsection{\OptimizingKeil + \ThumbMode}

\begin{lstlisting}
.text:0000005E 48 43                       MULS    R0, R1
.text:00000060 80 18                       ADDS    R0, R0, R2
.text:00000062 70 47                       BX      LR
\end{lstlisting}

\IFRU{В режиме thumb, инструкция \TT{MLA} недоступна, так что компилятору пришлось сгенерировать код, делающий
обе операции по отдельности.}
{\TT{MLA} instruction is not available in thumb mode, so, compiler generates the code doing these two operations
separately.}
\IFRU{Первая инструкция \TT{MULS} умножает \Rzero на \Rone оставляя результат в \Rone.}
{First \TT{MULS} instruction multiply \Rzero by \Rone leaving result in \Rone.}
\IFRU{Вторая (\TT{ADDS}) складывает результат и \Rtwo, оставляя результат в \Rzero.}
{Second (\TT{ADDS}) instruction adds result and \Rtwo leaving result in \Rzero.}



\section{\IFRU{И еще немного о возвращаемых результатах}{One more word about results returning.}}

\newcommand{\MSDNURL}{\href{http://msdn.microsoft.com/en-us/library/7572ztz4.aspx}{MSDN: Return Values (C++)}}

\IFRU{Резльутат выполнения функции в x86 обычно возвращается\footnote{См.также: \MSDNURL} через регистр \EAX, 
а если результат имеет тип байт или символ (\IT{char}), 
то в самой младшей части \EAX ~--- \AL. Если функция возвращает число с плавающей запятой, 
то регистр FPU \STZERO будет использован.
В ARM обычно результат возвращается в регистре R0.}
{As of x86, function execution result is usually returned\footnote{See also: \MSDNURL} in 
\EAX register. 
If it's byte type or character (\IT{char}) ~--- then in lowest register \EAX part ~--- \AL. 
If function returning \Tfloat number, FPU register 
\STZERO will be used instead.
In ARM, result is usually returned in R0 register.}

\IFRU{Вот почему старые компиляторы Си не способны создавать функции возвращающие нечто большее нежели помещается 
в один регистр (обычно, тип \Tint), а когда нужно, приходится возвращать через указатели, указываемые 
в аргументах.}
{That is why old C compilers can't create functions capable of returning something not fitting in one 
register (usually type \Tint), but if one need it, one should return information via pointers passed 
in function arguments.}
\IFRU{Хотя, позже и стало возможным, вернуть, скажем, целую структуру, но этот метод до сих пор не очень популярен. 
Если функция должна вернуть структуру, вызывающая функция должна сама, скрыто и прозрачно для программиста, 
выделить место и передать указатель на него в качестве первого аргумента. Это почти то же самое 
что и сделать это вручную, но компилятор прячет это.

Небольшой пример:}
{Now it is possible, to return, let's say, whole structure, but its still not very popular. 
If function should return a large structure, caller must allocate it and pass pointer to it via first argument, 
hiddenly and transparently for programmer. 
That is almost the same as to pass pointer in first argument manually, but compiler hide this.

Small example:}

\lstinputlisting{06_return_results/6_1.c}

\dots \IFRU{получим}{what we got} (MSVC 2010 \Ox):

\lstinputlisting{06_return_results/6_1.asm}

\IFRU{Имя внутреннего макроса для передачи указателя на структуру здесь это \TT{\$T3853}.}
{Macro name for internal variable passing pointer to structure is \TT{\$T3853} here.}


\input{061_pointers/ptrs_and_refs}
\section{\IFRU{Условные переходы}{Conditional jumps}}
\label{sec:Jcc}

\IFRU{Об условных переходах.}{Now about conditional jumps.}

\lstinputlisting{07_jcc/7_1.c}

\subsection{x86: \IFRU{3 аргумента}{3 arguments}}

\subsubsection{MSVC}

\IFRU{Компилируем при помощи MSVC 2010 Express, и в итоге получим:}
{Let's compile it by MSVC 2010 Express and we got:}

\begin{lstlisting}
$SG3830	DB	'a=%d; b=%d; c=%d', 00H

...

	push	3
	push	2
	push	1
	push	OFFSET $SG3830
	call	_printf
	add	esp, 16					; 00000010H
\end{lstlisting}

\IFRU{Все почти то же, за исключением того, что теперь видно, что аргументы для \printf заталкиваются в стек в обратном порядке: самый первый аргумент заталкивается последним.}
{Almost the same, but now we can see the \printf arguments are pushing into stack in reverse order: and the first argument is pushing in as the last one.}

\IFRU{Кстати, вспомним что переменные типа \Tint в 32-битной системе, как известно, имеет ширину 32 бита, это 4 байта}
{By the way, variables of \Tint type in 32-bit environment has 32-bit width that is 4 bytes}.

\IFRU{Итак, у нас всего 4 аргумента. $4*4 = 16$ ~--- именно 16 байт занимают в стеке указатель на строку плюс еще 3 числа типа \Tint.}
{So, we got here 4 arguments. $4*4 = 16$~---they occupy exactly 16 bytes in the stack: 32-bit pointer to string and 3 number of \Tint type.}

\index{x86!\Instructions!ADD}
\index{x86!\Registers!ESP}
\index{cdecl}
\IFRU{Когда при помощи инструкции \TT{``ADD ESP, X''} корректируется \glslink{stack pointer}{указатель стека} \ESP 
после вызова какой-либо функции, зачастую можно сделать вывод о том, сколько аргументов 
у вызываемой функции было, разделив X на 4.}
{When \gls{stack pointer} (the \ESP register) is corrected by \TT{``ADD ESP, X''}
instruction after a function 
call, often, the number of function arguments could be deduced here: just divide X by 4.}

\IFRU{Конечно, это относится только к cdecl-методу передачи аргументов через стек.}
{Of course, this is related only to \IT{cdecl} calling convention.}

\IFRU{См. также в соответствующем разделе о способах передачи аргументов через стек}
{See also section about calling conventions}~(\ref{sec:callingconventions}).

\IFRU{Иногда бывает так, что подряд идут несколько вызовов разных функций, 
но стек корректируется только один раз, после последнего вызова:}
{It is also possible for compiler to merge several \TT{``ADD ESP, X''} instructions into one, after last call:}

\begin{lstlisting}
push a1
push a2
call ...
...
push a1
call ...
...
push a1
push a2
push a3
call ...
add esp, 24
\end{lstlisting}

\subsubsection{MSVC \AndENRU \olly}
\index{\olly}

\IFRU{Попробуем этот же пример в}{Now let's try to load this example in} \olly.
\IFRU{Это один из наиболее популярных win32-отладчиков user-режима}{It is one of the most 
popular user-land win32 debugger}.
\IFRU{Мы можем компилировать наш пример в}{We can try to compile our example in} MSVC 2012 
\IFRU{с опцией}{with} \TT{/MD} \IFRU{что означает, линковать с библиотекой}{option, meaning, to link 
against} \TT{MSVCR*.DLL},
\IFRU{чтобы импортируемые ф-ции были хорошо видны в отладчике}{so we will able to see imported 
functions clearly in debugger}.

\IFRU{Затем загружаем исполняемый файл в}{Then load executable in} \olly.
\IFRU{Самый первый брякпойнт в}{The very first breakpoint is in} \TT{ntdll.dll}, \IFRU{нажмите}{press} 
F9 (\IFRU{запустить}{run}).
\IFRU{Второй брякпойнт в}{The second breakpoint is in} \ac{CRT}-\IFRU{коде}{code}.
\IFRU{Теперь мы должны найти ф-цию}{Now we should find} \main\EN{ function}.

\IFRU{Найдите этот код скроллируя окно кода до самого верха (MSVC располагает ф-цию \main в самом начале
секции кода)}{Find this code by scrolling the code to the very bottom (MSVC allocates \main function at
the very beginning of the code section)}: 
\figname \ref{fig:printf3_olly_1}.

\IFRU{Кликните на инструкции}{Click on} \TT{PUSH EBP}\IFRU{, нажмите}{ instruction, press} F2 
(\IFRU{установка брякпойнта}{set breakpoint}) \IFRU{и нажмите}{and press} F9 (\IFRU{запустить}{run}).
\IFRU{Нам нужно произвести все эти манипуляции, чтобы пропустить \ac{CRT}-код, потому что нам он пока
не интересен}{We need to do these manupulations in order to skip \ac{CRT}-code, because, we don't really 
interesting in it yet}.

\IFRU{Нажмите}{Press} F8 (\stepover) 6 \IFRU{раз, т.е., пропустить
6 инструкций}{times, i.e., skip 6 instructions}: \figname \ref{fig:printf3_olly_2}.

\IFRU{Теперь}{Now the} \PC \IFRU{указывает на инструкцию}{points to the}
\TT{CALL printf}\EN{ instruction}.
\olly, \IFRU{как и другие отладчики, подсвечивает регистры со значениями, которые изменились}
{like other debuggers, highlights value of registers which were changed}.
\IFRU{Так что, каждый раз, когда мы нажимаем}{So each time you press F8}, \EIP 
\IFRU{изменяется и его значение подсвечивается красным}{is changing and its value looking red}.
\ESP \IFRU{также меняется, потому что значения заталкиваются в стек}{is changing as well, 
because values are pushed into the stack}.

\IFRU{Где находятся эти значения в стеке}{Where are the values in the stack}?
\IFRU{Посмотрите на правое/нижнее окно в отладчике}{Take a look into right/bottom window of debugger}:

\begin{figure}[H]
\centering
\includegraphics[scale=0.66]{patterns/03_printf/olly3_stack.png}
\caption{\olly: \IFRU{стек, после того как значения там сохранены}{stack after values pushed}
(\IFRU{я сделал здесь округлую красную пометку в графическом редакторе}{I made round red mark 
here in graphics editor})}
\end{figure}

\IFRU{Так что здесь видно 3 столбца: адрес в стеке, значение в стеке и еще дополнительный комментарий
от \olly}{So we can see there 3 columns: address in the stack, 
value in the stack and some additional \olly comments}. 
\olly \IFRU{понимает}{understands} \printf\IFRU{-строки}{-like strings}, 
\IFRU{так что он показывает здесь и строку и 3 значения \IT{привязанных} к ней}{so it reports the 
string here and 3 values \IT{attached} to it}.

\IFRU{Нажмите}{Press} F8 (\stepover).

\IFRU{В коносил мы видим вывод}{In the console we'll see the output}:

\begin{figure}[H]
\centering
\includegraphics[scale=0.66]{patterns/03_printf/olly3_console.png}
\caption{\RU{Ф-ция }\printf \IFRU{исполнилась}{function executed}}
\end{figure}

\IFRU{Посмотрим, как изменились регистры и состояние стека}{Let's see how registers and stack state 
are changed}: \figname \ref{fig:printf3_olly_3}.

\RU{Регистр }\EAX \IFRU{теперь содержит}{register now contains} \TT{0xD} (13).
That's correct, \printf returns number of characters printed.
\RU{Значение }\EIP \IFRU{изменилось: действительно, теперь здесь адрес инструкции после}
{value is changed: indeed, now there is address of the instruction after} \TT{CALL printf}.
\RU{Значения регистров }\ECX \AndENRU \EDX \IFRU{также изменились}{values are changed as well}.
\IFRU{Очевидно, внутренности ф-ции \printf используют их для каких-то своих нужд}{Apparently, 
\printf function's hidden machinery used them for its own needs}.

\IFRU{Очень важный момент в том что значение \ESP не изменилось. И состояние стека также!}
{A very important thing is that \ESP value is not changed. And stack state too!}
\IFRU{Мы ясно видим здесь и строку формата и соответствующие ей 3 значения, они все еще здесь}
{We clearly see that format string and corresponding 3 values are still there}.
\IFRU{Действительно, по соглашению вызовов \IT{cdecl}, вызывающая ф-ция не очищает аргументы из стека}
{Indeed, that's \IT{cdecl} calling convention, calling function doesn't clear arguments in stack}.
\IFRU{Это должна делать вызывающая ф-ция}{It's caller's duty to do so}.

\IFRU{Нажмите}{Press} F8 \IFRU{снова, чтобы исполнилась инструкция}{again to execute} 
\TT{ADD ESP, 10}\EN{ instruction}: \figname \ref{fig:printf3_olly_4}.

\ESP \IFRU{изменился, но значения все еще в стеке}{is changed, but values are still in the stack}!
\IFRU{Конечно, никому не нужно заполнять эти значения нулями или что-то в этом роде}{Yes, 
of course, no one needs to fill these values by zero or something like that}.
\IFRU{Потому что всё что выше указателя стека}{Because, everything above stack pointer} (\SP) 
\IFRU{это}{is} \IT{\IFRU{шум}{noise}} \OrENRU \IT{\IFRU{мусор}{garbage}}, \IFRU{это всё не имеет
особой ценности}{it has no value at all}.
\IFRU{Было бы очень затратно по времени очищать ненужные элементы стека, к тому же, никому это и не 
нужно}{It would be time consuming to clear unused stack entries, besides, no one really needs to}.

\begin{figure}[H]
\centering
\includegraphics[scale=0.66]{patterns/03_printf/olly3_1.png}
\caption{\olly: \IFRU{самое начало ф-ции}{the very start of the} \main\EN{ function}}
\label{fig:printf3_olly_1}
\end{figure}

\begin{figure}[H]
\centering
\includegraphics[scale=0.66]{patterns/03_printf/olly3_2.png}
\caption{\olly: \IFRU{перед исполнением}{before} \printf\EN{ execution}}
\label{fig:printf3_olly_2}
\end{figure}

\begin{figure}[H]
\centering
\includegraphics[scale=0.66]{patterns/03_printf/olly3_3.png}
\caption{\olly: \IFRU{после исполнения}{after} \printf\EN{ execution}}
\label{fig:printf3_olly_3}
\end{figure}

\begin{figure}[H]
\centering
\includegraphics[scale=0.66]{patterns/03_printf/olly3_4.png}
\caption{\olly: \IFRU{после исполнения инструкции}{after} \TT{ADD ESP, 10}\EN{ instruction execution}}
\label{fig:printf3_olly_4}
\end{figure}

\subsubsection{GCC}

\IFRU{Скомпилируем то же самое в Linux при помощи GCC 4.4.1 и посмотрим в \IDA что вышло:}
{Now let's compile the same in Linux by GCC 4.4.1 and take a look in \IDA what we got:}

\begin{lstlisting}
main            proc near

var_10          = dword ptr -10h
var_C           = dword ptr -0Ch
var_8           = dword ptr -8
var_4           = dword ptr -4

                push    ebp
                mov     ebp, esp
                and     esp, 0FFFFFFF0h
                sub     esp, 10h
                mov     eax, offset aADBDCD ; "a=%d; b=%d; c=%d"
                mov     [esp+10h+var_4], 3
                mov     [esp+10h+var_8], 2
                mov     [esp+10h+var_C], 1
                mov     [esp+10h+var_10], eax
                call    _printf
                mov     eax, 0
                leave
                retn
main            endp
\end{lstlisting}

\IFRU{Можно сказать, что этот короткий код, созданный GCC, отличается от кода MSVC только способом помещения 
значений в стек.
Здесь GCC снова работает со стеком напрямую без \PUSH/\POP.}
{It can be said, the difference between code by MSVC and GCC is only in method of placing arguments on the stack.
Here GCC working directly with stack without \PUSH/\POP.}


\subsection{ARM}

\subsubsection{\OptimizingKeil + \ARMMode}

\lstinputlisting[caption=\OptimizingKeil + \ARMMode]{07_jcc/ARM_O3_signed.asm}

\index{ARM!Condition codes}
\IFRU{Многие инструкции в режиме ARM могут быть исполнены только при некоторых выставленных флагах.}
{A lot of instructions in ARM mode can be executed only when specific flags are set.}
\IFRU{Это нередко используется для сравнения чисел, например.}
{This is often used while numbers comparing, for example.}

\index{ARM!\Instructions!ADD}
\index{ARM!\Instructions!ADDAL}
\IFRU{К примеру, инструкция \ADD на самом деле может быть представлена как \TT{ADDAL}, \TT{AL} означает 
\IT{Always}, то есть, исполнять всегда.}
{For instance, \ADD instruction is \TT{ADDAL} internally in fact, where \TT{AL} meaning
\IT{Always}, i.e., execute always.}
\IFRU{Предикаты кодируются в 4-х старших битах инструкции 32-битных ARM-инструкций}
{Predicates are encoded in 4 high bits of 32-bit ARM instructions} (\IT{condition field}).
\index{ARM!\Instructions!B}
\IFRU{Инструкция безусловного перехода \TT{B}, на самом деле условная и кодируется так же 
как и прочие инструкции условных переходов, но имеет \TT{AL} в \IT{condition field}, 
то есть, исполняется всегда, игнорируя флаги.}
{\TT{B} instruction of unconditional jump is in fact conditional and encoded just like any other
conditional jumps, but has \TT{AL} in the \IT{condition field}, and what it means, executing always, ignoring flags.}

\index{ARM!\Instructions!ADR}
\index{ARM!\Instructions!ADRGT}
\index{ARM!\Instructions!CMP}
\IFRU{Инструкция \TT{ADRGT} работает так же как и \TT{ADR}, но исполнится только в случае 
если предыдущая инструкция \CMP,
сравнивая два числа, обнаружила что одно из них больше второго}
{\TT{ADRGT} instructions works just like \TT{ADR} but will execute only in the case when previous \CMP
instruction, while comparing two numbers, found one number greater than another}
(\IT{Greater Than}).

\index{ARM!\Instructions!BL}
\index{ARM!\Instructions!BLGT}
\IFRU{Следующая инструкция \TT{BLGT} ведет себя так же как и \TT{BL} и сработает только если 
результат сравнения был такой же}{The next \TT{BLGT} instruction behaves exactly as \TT{BL} and will be
triggered only if result of comparison was the same} (\IT{Greater Than}). 
\TT{ADRGT} \IFRU{записывает в \Rzero указатель на строку}{writes a pointer to the string} 
\TT{``a>b\textbackslash{}n''}, 
\IFRU{а \TT{BLGT} вызывает}{into \Rzero and \TT{BLGT} calls} \printf.
\IFRU{Следовательно, эти инструкции с суффиксом \TT{-GT}, исполнятся только в том случае, если значение
в \Rzero (там $a$) было больше чем значение в \Rfour (там $b$).}
{Consequently, these instructions with \TT{-GT} suffix, will be executed only in the case when
value in the \Rzero ($a$ is there) was bigger than value in the \Rfour ($b$ is there).}

\index{ARM!\Instructions!ADREQ}
\index{ARM!\Instructions!BLEQ}
\IFRU{Далее мы увидим инструкции \TT{ADREQ} и \TT{BLEQ}.}
{Then we see \TT{ADREQ} and \TT{BLEQ} instructions.}
\IFRU{Они работают так же как и \TT{ADR} и \TT{BL}, но исполнятся только в случае если значения при сравнении были равны.}
{They behave just like \TT{ADR} and \TT{BL} but is to be executed only in the case when operands were equal to each
other while comparison.}
\IFRU{Перед ними еще один \CMP (ведь вызов \printf мог испортить состояние флагов).}
{Another \CMP is before them (since \printf call may tamper state of flags).}

\index{ARM!\Instructions!LDMGEFD}
\index{ARM!\Instructions!LDMFD}
\IFRU{Далее мы увидим \TT{LDMGEFD}, эта инструкция работает так же как и \TT{LDMFD}\footnote{\LDMFDDESC}, 
но сработает только в случае если в результате сравнения одно из значений было больше 
или равно второму}
{Then we see \TT{LDMGEFD}, this instruction works just like \TT{LDMFD}\footnote{\LDMFDDESC},
but will be triggered only in the case when one value was greater or equal to another while comparison}
(\IT{Greater or Equal}).

\IFRU{Смысл инструкции}{The sense of} \TT{``LDMGEFD SP!, \{R4-R6,PC\}''} 
\IFRU{в том, что это как бы эпилог функции, но он сработает только если $a>=b$, только тогда работа 
функции закончится.}
{instruction is that is like function epilogue, but it will be triggered only if $a>=b$, only then function 
execution will be finished.}
\index{Function epilogue}
\IFRU{Но если это не так, то есть $a<b$, то исполнение дойдет до следующей инструкции 
\TT{``LDMFD SP!, \{R4-R6,LR\}''}, это еще один эпилог функции, эта инструкция восстанавливает состояние регистров
\TT{R4-R6}, но и \LR вместо \PC, таким образом, пока что не делая возврата из функции.}
{But if it is not true, i.e., $a<b$, then control flow come to next \TT{``LDMFD SP!, \{R4-R6,LR\}''} instruction,
this is one more function epilogue, this instruction restores \TT{R4-R6} registers state, 
but also \LR instead of \PC, thus, it does not returns from function.}
\IFRU{Последние две инструкции вызывают}{Last two instructions calls} \printf 
\IFRU{со строкой}{with the string} <<a<b\textbackslash{}n>> \IFRU{в качестве единственного аргумента}{as 
sole argument}.
\IFRU{Безусловный переход на \printf вместо возврата из функции, это то что мы уже рассматривали в 
секции}{Unconditional jump to the \printf function instead of function return, is what we already examined in} <<\PrintfSeveralArgumentsSectionName>>\IFRU{, здесь}{ section, here}~(\ref{ARM_B_to_printf}).

\index{ARM!\Instructions!ADRHI}
\index{ARM!\Instructions!BLHI}
\index{ARM!\Instructions!LDMCSFD}
\IFRU{Функция }{}\TT{f\_unsigned} \IFRU{точно такая же, но там используются инструкции}{is likewise,
but } \TT{ADRHI}, \TT{BLHI}, \AndENRU \TT{LDMCSFD} \IFRU{эти предикаты}{instructions are
used there, these predicates}
(\IT{HI = Unsigned higher, CS = Carry Set (greater than or equal)}) 
\IFRU{аналогичны рассмотренным, но служат для работы с беззнаковыми значениями.}
{are analogical to those examined before, but serving for unsigned values.}

\IFRU{В функции \main ничего для нас нового нет:}
{There is not much new in the \main function for us:}

\lstinputlisting[caption=\main]{07_jcc/ARM_O3_main.asm}

\IFRU{Так, в режиме ARM можно обойтись без условных переходов.}
{That's how to get rid of conditional jumps in ARM mode.}

\index{\IFRU{Конвеер RISC}{RISC pipeline}}
\IFRU{Почему это хорошо?}{Why it is so good?}
\IFRU{Потому что ARM это RISC-процессор имеющий конвеер (pipeline) для исполнения инструкций.}
{Since ARM is RISC-processor with pipeline for instructions executing.}
\IFRU{Если говорить 
коротко, то процессору с конвеером тяжело даются переходы вообще, поэтому есть спрос на возможность 
предсказывания переходов.}
{In short, pipelined processor is not very good on jumps at all,
so that is why branch predictor units are
critical here.}
\IFRU{Очень хорошо если программа имеют как можно меньшее переходов, как условных, 
так и безусловных, поэтому, 
инструкции с добавленными предикатами, указывающими,
исполнять инструкцию или нет, могут избавить от некоторого количества условных переходов.}
{It is very good if the program has as few jumps as possible, conditional and unconditional, so that is why,
predicated instructions can help in reducing conditional jumps count.}

\index{x86!\Instructions!CMOVcc}
\IFRU{В x86 нет аналогичной возможности, если не считать инструкцию \TT{CMOVcc}, это то же что и \MOV, 
но она срабатывает
только при определенных выставленных флагах, обычно, выставленных при помощи \CMP во время сравнения.}
{There is no such feature in x86, if not to count \TT{CMOVcc} instruction, it is the same as \MOV,
but triggered only when specific flags are set, usually set while value comparison by \CMP.}

\subsubsection{\OptimizingKeil + \ThumbMode}

\lstinputlisting[caption=\OptimizingKeil + \ThumbMode]{07_jcc/ARM_thumb_signed.asm}

\index{ARM!\Instructions!BLE}
\index{ARM!\Instructions!BNE}
\index{ARM!\Instructions!BGE}
\index{ARM!\Instructions!BLS}
\index{ARM!\Instructions!BCS}
\index{ARM!\Instructions!B}
\index{ARM!\ThumbMode}
\IFRU{В режиме thumb, только инструкции \TT{B} могут быть дополнены условием исполнения (\IT{condition code}), 
так что, код для режима thumb выглядит привычнее.}
{Only \TT{B} instructions in thumb mode may be supplemented by \IT{condition codes}, so the thumb code 
looks more ordinary.}

\TT{BLE} \IFRU{это обычный переход с условием}{is usual conditional jump} \IT{Less than or Equal}, 
\TT{BNE} ~--- \IT{Not Equal}, 
\TT{BGE} ~--- \IT{Greater than or Equal}.

\IFRU{Функция }{}\TT{f\_unsigned} \IFRU{точно такая же, но для работы с беззнаковыми величинами, 
там используются 
инструкции}
{function is just likewise, but other instructions are used while working with unsigned values:}\TT{BLS} 
(\IT{Unsigned lower or same}) \AndENRU \TT{BCS} (\IT{Carry Set (Greater than or equal)}).



\section{\SwitchCaseDefaultSectionName}

\subsection{\IFRU{Если вариантов мало}{Few number of cases}}

\section{\RU{Если вариантов мало}\EN{Few number of cases}}

\lstinputlisting{patterns/08_switch/few.c}

\subsubsection{x86}

\IFRU{Это дает в итоге}{Result} (MSVC 2010):

\lstinputlisting[caption=MSVC 2010]{08_switch/8_2_msvc.asm}

\IFRU{Наша функция со switch()-ем, с небольшим количеством вариантов, 
это практически аналог подобной конструкции:}
{Out function with a few cases in switch(), in fact, is analogous to this construction:}

\lstinputlisting{08_switch/8_1_analogue.c}

\index{\CLanguageElements!switch}
\index{\CLanguageElements!if}
\IFRU{Когда вариантов немного и мы видим подобный код, невозможно сказать с уверенностью, был ли
в оригинальном исходном коде switch(), либо просто набор if()-ов.}
{When few cases in switch(), and we see such code, it's impossible to say with certainty, was it
switch() in source code, or just pack of if().}
\index{\SyntacticSugar}
\IFRU{То есть, switch() это синтаксический сахар для большого количества вложенных проверок 
при помощи if().}
{This mean, switch() is syntactic sugar for large number of nested checks constructed using if().}

\IFRU{В самом выходном коде, в принципе, ничего особо нового для нас здесь, 
за исключением того, что компилятор зачем-то 
перекладывает входящую переменную \TT{a} во временную в локальном стеке \TT{v64}.}
{Nothing specially new to us in generated code, with the exception that compiler moving 
input variable 
\TT{a} to temporary local variable \TT{tv64}.}

\IFRU{Если скомпилировать это при помощи GCC 4.4.1, то будет почти то же самое, даже с максимальной оптимизацией 
(ключ \Othree).}
{If to compile the same in GCC 4.4.1, we'll get alsmost the same, even with maximal optimization 
turned on (\Othree option).}

\IFRU{Попробуем, включить оптимизацию кодегенератора}
{Now let's turn on optimization in} MSVC (\Ox): \TT{cl 1.c /Fa1.asm /Ox}

\lstinputlisting[caption=MSVC]{08_switch/8_3_msvc.asm}

\IFRU{Вот здесь уже все немного по-другому, причем не без грязных хаков.}
{Here we can see even dirty hacks.}

\index{x86!\Instructions!JZ}
\index{x86!\Instructions!JE}
\index{x86!\Instructions!SUB}
\IFRU
{Первое: \TT{а} помещается в \EAX и от него отнимается 0. Звучит абсурдно, но нужно это для того, чтобы проверить, 
0 ли в \EAX был до этого? Если да, то выставится флаг \ZF (что означает что результат отнимания нуля от числа 
стал нулем) и первый условный переход \JE (\IT{Jump if Equal} или его синоним \JZ ~--- \IT{Jump if Zero}) 
сработает на метку \TT{\$LN4@f}, где выводится сообщение \TT{'zero'}.
Если первый переход не сработал, от значения отнимается по единице, 
и если на какой-то стадии образуется в результате $0$, то сработает соответствующий переход.}
{First: \TT{a} is placed into \EAX and $0$ subtracted from it. Sounds absurdly, but it may need to check if 
0 was in \EAX before? If yes, flag \ZF will be set (this also mean that subtracting from zero is zero) 
and first conditional jump \JE (\IT{Jump if Equal} or synonym \JZ ~--- \IT{Jump if Zero}) will be triggered 
and control flow passed to \TT{\$LN4@f} label, where \TT{'zero'} message is begin printed. 
If first jump was not triggered, 1 subtracted from input value and if at some stage 0 will be resulted, 
corresponding jump will be triggered.}

\IFRU{И в конце концов, если ни один из условных переходов не сработал, управление передается \printf
с агрументом \TT{'something unknown'}.}
{And if no jump triggered at all, control flow passed to \printf with argument \TT{'something unknown'}.}

\label{jump_to_last_printf}
\index{\Stack}
\IFRU
{Второе: мы видим две, мягко говоря, необычные вещи: указатель на сообщение помещается в переменную \TT{a}, 
и затем \printf вызывается не через \CALL, а через \JMP. Объяснение этому простое. 
Вызывающая функция заталкивает в стек некоторое значение и через \CALL вызывает нашу функцию. 
\CALL в свою очередь затакливает в стек адрес возврата и делает безусловный переход на адрес нашей функции. 
Наша функция в самом начале (да и в любом её месте, потому что в теле функции нет ни одной инструкции, 
которая меняет что-то в стеке или в \ESP) имеет следующую разметку стека:}
{Second: we see unusual thing for us: string pointer is placed into \TT{a} variable, and 
then \printf is called not via \CALL, but via \JMP. This could be explained simply. 
Caller pushing to stack some value and via \CALL calling our function. 
\CALL itself pushing returning address to stack and do unconditional jump to our function address. 
Our function at any place of its execution (since it do not contain any instruction moving stack 
pointer) has the following stack layout:}

\begin{itemize}
\item\ESP ~--- \IFRU{хранится адрес возврата}{pointing to return address} 
\item\TT{ESP+4} ~--- \IFRU{хранится значение \TT{a}}{pointing to \TT{a} variable} 
\end{itemize}

\IFRU{С другой стороны, чтобы вызвать \printf нам нужна почти такая же разметка стека, 
только в первом аргументе нужен указатель на строку. Что, собственно, этот код и делает.}
{On the other side, when we need to call \printf here, we need exactly the same stack 
layout, except of first \printf argument pointing to string. 
And that is what our code does.}

\IFRU{Он заменяет свой первый аргумент на другой и затем передает управление \printf, как если бы вызвали не 
нашу функцию \TT{f()}, а сразу \printf. 
\printf выводит некую строку на \TT{stdout}, затем исполняет инструкцию \RET, 
которая из стека достает адрес возврата и управление передается в ту функцию, 
которая вызывала \TT{f()}, минуя при этом саму \TT{f()}.}
{It replaces function's first argument to different and 
jumping to \printf, as if not our function \TT{f()} was called firstly, but immediately \printf.
\printf printing some string to \TT{stdout} and then execute \RET instruction, which POPping 
return address from stack and control flow is returned not to \TT{f()}, but to \TT{f()}'s callee, 
escaping \TT{f()}.}

\index{\CStandardLibrary!longjmp()}
\newcommand{\URLSJ}{\url{http://en.wikipedia.org/wiki/Setjmp.h}}
\IFRU{Все это возможно потому что \printf вызывается в \TT{f()} в самом конце. 
Все это чем-то даже похоже на \TT{longjmp()}\footnote{\URLSJ}.
И все это, разумеется, сделано для экономии времени исполнения.}
{All it's possible because \printf is called right at the end of \TT{f()} in any case. 
In some way, it's all similar to \TT{longjmp()}\footnote{\URLSJ}. 
And of course, it's all done for the sake of speed.}

\IFRU{Похожая ситуация с компилятором для ARM описана в секции}
{Similar case with ARM compiler described in} ``\PrintfSeveralArgumentsSectionName'', 
\IFRU{здесь}{section, here}~\ref{ARM_B_to_printf}.

%NOTTRANSLATED
\subsubsection{ARM: Оптимизирующий Keil + режим ARM}

\begin{lstlisting}
.text:0000014C             f1
.text:0000014C 00 00 50 E3                 CMP     R0, #0
.text:00000150 13 0E 8F 02                 ADREQ   R0, aZero       ; "zero\n"
.text:00000154 05 00 00 0A                 BEQ     loc_170
.text:00000158 01 00 50 E3                 CMP     R0, #1
.text:0000015C 4B 0F 8F 02                 ADREQ   R0, aOne        ; "one\n"
.text:00000160 02 00 00 0A                 BEQ     loc_170
.text:00000164 02 00 50 E3                 CMP     R0, #2
.text:00000168 4A 0F 8F 12                 ADRNE   R0, aSomethingUnkno ; "something unknown\n"
.text:0000016C 4E 0F 8F 02                 ADREQ   R0, aTwo        ; "two\n"
.text:00000170
.text:00000170             loc_170                                 ; CODE XREF: f1+8
.text:00000170                                                     ; f1+14
.text:00000170 78 18 00 EA                 B       __2printf
\end{lstlisting}

Мы снова не сможем сказать, глядя на этот код, был ли в оригинальном исходном коде switch() либо же несколько
if()-в.

Так или иначе, мы снова видим здесь инструкции с предикатами, например \TT{ADREQ} (\IT{(Equal)}), 
которая будет исполняться только
если $R0=0$, в таком случае, в R0 будет загружен адрес строки \IT{<<zero\textbackslash{}n>>}. 
Следующая инструкция \TT{BEQ} (\IT{(Branch Equal)}) перенаправит
исполнение на \TT{loc\_170}, если $R0=0$. 
Кстати, наблюдательный читатель может спросить, сработает ли \TT{BEQ} нормально,
ведь \TT{ADREQ} перед ним уже заполнила регистр \TT{R0} чем-то другим. 
Сработает, потому что \TT{BEQ} проверяет флаги установленные
инструкцией \CMP, а \TT{ADREQ} флаги никак не модифицирует.

%TODO -S instructions

Далее всё просто и знакомо. Вызов \printf один и в конце, мы уже рассматривали подобный трюк 
здесь~\ref{ARM_B_to_printf}. К \printf{}-у в конце ведут три пути. 

Обратите внимание на то что происходит
если $a=2$ и если $a$ не попадает под сравниваемые константы. Инструкция \TT{``CMP R0, \#2''} нужна чтобы узнать
$a=2$ или нет. Если это не так, то при помощи \TT{ADRNE} (\IT{Not Equal}) в \TT{R0} будет загружен указатель на 
строку \IT{<<something unknown \textbackslash{}n>>},
ведь $a$ уже было проверено на $0$ и $1$ до этого, и здесь $a$ точно не попадает под эти константы. 
Ну а если $R0=2$,
в R0 будет загружен указатель на строку \IT{<<two\textbackslash{}n>>} при помощи инструкции \TT{ADREQ}.




\subsection{\IFRU{И если много}{A lot of cases}}

\section{\RU{И если много}\EN{A lot of cases}}

\RU{А если ветвлений слишком много, то конечно генерировать слишком длинный код с многочисленными \JE/\JNE 
уже не так удобно.}
\EN{If \TT{switch()} statement contain a lot of case's, it is not very convenient for compiler to emit too large code
with a lot \JE/\JNE instructions.}

\lstinputlisting{patterns/08_switch/lot.c}

\subsection{x86}

\subsubsection{\NonOptimizing MSVC}

\RU{Имеем в итоге}\EN{We got} (MSVC 2010):

\lstinputlisting[caption=MSVC 2010]{patterns/08_switch/lot_8_5_msvc.asm}

\index{jumptable}
\RU{Здесь происходит следующее: в теле функции есть набор вызовов \printf с разными аргументами. 
Все они имеют, конечно же, адреса, а также внутренние символические метки, которые присвоил им компилятор.
Помимо всего прочего, все эти метки складываются во внутреннюю таблицу \TT{\$LN11@f}.}
\EN{OK, what we see here is: there is a set of the \printf calls with various arguments. 
All they has not only addresses in process memory, but also internal symbolic labels assigned 
by compiler. 
Besides, all these labels are also places into \TT{\$LN11@f} internal table.}

\RU{В начале функции, если \TT{a} больше 4, то сразу происходит переход на метку \TT{\$LN1@f}, 
где вызывается \printf с аргументом \TT{'something unknown'}.}
\EN{At the function beginning, if \TT{a} is greater than 4, control flow is passed to label 
\TT{\$LN1@f}, where \printf with argument \TT{'something unknown'} is called.}

\RU{А если \TT{a} меньше или равно 4, то это значение умножается на 4 и прибавляется адрес таблицы 
с переходами. 
Таким образом, получается адрес внутри таблицы, где лежит нужный адрес внутри тела функции. 
Например, возьмем \TT{a} равным 2. $2*4 = 8$ (ведь все элементы таблицы ~--- это адреса внутри 32-битного процесса, 
таким образом, каждый элемент занимает 4 байта). 8 прибавить к \TT{\$LN11@f} ~--- это будет элемент таблицы,
где лежит \TT{\$LN4@f}. \JMP вытаскивает из таблицы адрес \TT{\$LN4@f} и делает безусловный переход туда.}
\EN{And if \TT{a} value is less or equals to 4, let's multiply it by 4 and add \TT{\$LN1@f} 
table address. That is how address inside of table is constructed, pointing exactly to the 
element we need. For example, let's say \TT{a} is equal to 2. $2*4 = 8$ (all table elements 
are addresses within 32-bit process that is why all elements contain 4 bytes). 
Address of the \TT{\$LN11@f} table + 8~---it will be table element where \TT{\$LN4@f} label is stored.
\JMP fetches \TT{\$LN4@f} address from the table and jump to it.}

\RU{Эта таблица иногда называется}\EN{This table called sometimes} \IT{jumptable}.

\RU{А там вызывается \printf с аргументом \TT{'two'}. 
Дословно, инструкция \TT{jmp DWORD PTR \$LN11@f[ecx*4]} 
означает \IT{перейти по DWORD, который лежит по адресу} \TT{\$LN11@f + ecx * 4}.}
\EN{Then corresponding \printf is called with argument \TT{'two'}. 
Literally, \TT{jmp DWORD PTR \$LN11@f[ecx*4]} instruction means
\IT{jump to DWORD, which is stored at address} \TT{\$LN11@f + ecx * 4}.}

\TT{npad}~(\ref{sec:npad})
\RU{это макрос ассемблера, немного выровнять начало таблицы, 
дабы она располагалась по 
адресу кратному 4 (или 16). Это нужно для того чтобы процессор мог эффективнее загружать 32-битное 
значения из памяти, через шину с памятью, кэш-память, и т.д.}
\EN{is assembly language macro, aligning next label so that it will be stored at address aligned on a 4 byte
(or 16 byte) border.
This is very suitable for processor since it is able to fetch 32-bit values from memory through memory bus,
cache memory, etc, in much effective way if it is aligned.}

\subsubsection{\NonOptimizing GCC}

\RU{Посмотрим, что сгенерирует GCC 4.4.1}\EN{Let's see what GCC 4.4.1 generates}:

\lstinputlisting[caption=GCC 4.4.1]{patterns/08_switch/lot_8_6_gcc.asm}

\index{x86!\Registers!JMP}
\RU{Практически то же самое, за исключением мелкого нюанса: аргумент из \TT{arg\_0} умножается на 4 
при помощи сдвига влево на 2 бита (это почти то же самое что и умножение на 4)~(\ref{SHR}).
Затем адрес метки внутри функции берется из массива \TT{off\_804855C} и адресуется при помощи 
вычисленного индекса.}
\EN{It is almost the same, except little nuance: argument \TT{arg\_0} is multiplied by 4 with by
shifting it to left by 2 bits (it is almost the same as multiplication by 4)~(\ref{SHR}).
Then label address is taken from \TT{off\_804855C} array, address calculated and stored into 
\EAX, then \TT{``JMP EAX''} do actual jump.}


\subsubsection{ARM: \OptimizingKeil + \ARMMode}
\label{sec:SwitchARMLot}

\lstinputlisting{patterns/08_switch/lot_ARM_ARM_O3.asm}

\IFRU{В этом коде используется та особенность режима ARM, что все инструкции в этом режиме имеют длину 4 байта.}
{This code makes use of the ARM feature in which all instructions in the ARM mode has size of 4 bytes.}

\IFRU{Итак, не будем забывать, что максимальное значение для $a$ это $4$, всё что выше, должно вызвать
вывод строки}
{Let's keep in mind the maximum value for $a$ is $4$ and any greater value must cause}
\IT{<<something unknown\textbackslash{}n>>}
\IFRU{.}
{string printing.}

\index{ARM!\Instructions!CMP}
\index{ARM!\Instructions!ADDCC}
\IFRU{Самая первая инструкция}{The very first} \TT{``CMP R0, \#5''} 
\IFRU{сравнивает входное значение в $a$ c $5$.}
{instruction compares $a$ input value with $5$.}

\IFRU{Следующая инструкция}{The next} \TT{``ADDCC PC, PC, R0,LSL\#2''}
\footnote{ADD\IFRU{ ~--- }{~---}\IFRU{складывание чисел}{addition}}
\IFRU{сработает только в случае если}{instruction will execute only if} $R0 < 5$ (\IT{CC=Carry clear / Less than}). 
\IFRU{Следвательно, если}{Consequently, if} \TT{ADDCC} \IFRU{не сработает}{will not trigger} 
(\IFRU{это случай с}{it is a} $R0 \geq 5$\IFRU{}{ case}), 
\IFRU{выполнится переход на метку}{a jump to} 
\IT{default\_case}\IFRU{.}{label will be occurred.}

\IFRU{Но если}{But if} $R0 < 5$ \AndENRU \TT{ADDCC} \IFRU{сработает, то произойдет следующее:}
{will trigger, following events will happen:}

\IFRU{Значение в \Rzero умножается на $4$}{Value in the \Rzero is multiplied by $4$}.
\IFRU{Фактически}{In fact}, \TT{LSL\#2} \IFRU{в конце инструкции означает ``сдвиг влево на 2 бита''.}
{at the instruction's ending means ``shift left by 2 bits''.}
\IFRU{Но как будет видно позже}{But as we will see later}~(\ref{division_by_shifting}) \IFRU{в секции}{in} 
``\ShiftsSectionName'' \IFRU{}{section}, 
\IFRU{сдвиг влево на 2 бита это как раз эквивалентно его умножению на $4$.}
{shift left by 2 bits is just equivalently to multiplying by $4$.}

\IFRU{Затем полученное}{Then,} $R0*4$ \IFRU{прибавляется к текущему значению \PC}{value we got, is added to
current value in the \PC}, 
\IFRU{совершая, таким образом, переход на одну из расположенных ниже инструкций \TT{B} (\IT{Branch}).}
{thus jumping to one of \TT{B} (\IT{Branch}) instructions located below.}

\IFRU{На момент исполнения}{At the moment of} \TT{ADDCC} \IFRU{}{execution}, 
\IFRU{содержимое \PC на 8 байт больше}{value in the \PC is 8 bytes ahead} (\TT{0x180}) 
\IFRU{чем адрес по которому расположена сама инструкция} 
{than address at which} \TT{ADDCC} \IFRU{}{instruction is located} (\TT{0x178}), 
\IFRU{либо, говоря иным языком, на 2 инструкции больше.}
{or, in other words, 2 instructions ahead.}

\index{ARM!\IFRU{Конвеер}{Pipeline}}
\IFRU{Это связано с работой конвеера процессора ARM:
пока исполняется инструкция \TT{ADDCC}, процессор уже начинает обрабатывать инструкцию после следующей, 
поэтому \PC указывает туда.}
{This is how ARM processor pipeline works: when \TT{ADDCC} instruction is executed,
the processor at the moment
is beginning to process instruction after the next one,
so that is why \PC pointing there.}

\IFRU{В случае, если $a=0$, тогда к \PC ничего не будет прибавлено, 
в \PC запишется актуальный на тот момент \PC (который больше на 8) 
и произойдет переход на метку \IT{loc\_180}, 
это на 8 байт дальше от места где находится инструкция \TT{ADDCC}.}
{If $a=0$, then nothing will be added to the value in the \PC,
and actual value in the \PC is to be written into the \PC (which is 8 bytes ahead)
and jump to the label \IT{loc\_180} will happen,
this is 8 bytes ahead of the point where \TT{ADDCC} instruction is.}

\IFRU{В случае, если}{In case of} $a=1$, \IFRU{тогда в \PC запишется}{then} 
$PC+8+a*4 = PC+8+1*4 = PC+16 = 0x184$\IFRU{, это адрес метки \IT{loc\_184}}{will be written to the \PC,
this is the address of the \IT{loc\_184} label}.

\IFRU{При каждой добавленной к $a$ единице, итоговый \PC увеличивается на $4$.}
{With every $1$ added to $a$, resulting \PC increasing by $4$.}
\IFRU{$4$ это как раз длина инструкции  в режиме ARM и одновременно с этим, 
длина каждой инструкции \TT{B}, их здесь следует 5 в ряд.}
{$4$ is also instruction length in ARM mode and also, length of each \TT{B} instruction length,
there are 5 of them in row.}

\IFRU{Каждая из этих пяти инструкций \TT{B}, передает управление дальше, где собственно и происходит то, 
что запрограммировано в}
{Each of these five \TT{B} instructions passing control further, where something is going on, 
what was programmed in}
\IT{switch()}.
\IFRU{Там происходит загрузка указателя на свою строку, итд.}
{Pointer loading to corresponding string occurring there, etc.}

\subsubsection{ARM: \OptimizingKeil + \ThumbMode}

\lstinputlisting{patterns/08_switch/lot_ARM_thumb_O3.asm}

\index{ARM!\ThumbMode}
\index{ARM!\ThumbTwoMode}
\IFRU{В режимах thumb и thumb-2, уже нельзя надеятся на то что все инструкции будут иметь одну длину.}
{One cannot be sure all instructions in thumb and thumb-2 modes will have same size.}
\IFRU{Можно даже сказать что в этих режимах инструкции переменной длины, как в x86.}
{It is even can be said that in these modes instructions has variable length, just like in x86.}

\index{jumptable}
\IFRU{Так что здесь добавляется специальная таблица, содержащая информацию о том, как много вариантов здесь,
не включая default-варианта, и смещения, для каждого варианта, каждое кодирует метку, куда нужно передать
управление в соответствующем случае.}
{So there is a special table added, containing information about how much cases are there, not including 
default-case, and offset, for each, each encoding a label, to which control must be passed in 
corresponding case.}

\index{ARM!\IFRU{Переключение режимов}{Mode switching}}
\index{ARM!\Instructions!BX}
\IFRU{Для того чтобы работать с таблицей и совершить переход, вызывается служебная функция}
{A special function here present in order to deal with the table and pass control, named} \\
\IT{\_\_ARM\_common\_switch8\_thumb}. 
\IFRU{Она начинается с инструкции}{It is beginning with} \TT{``BX PC''}
\IFRU{, чья функция ~--- переключить процессор в ARM-режим.}
{instruction, which function is to switch processor to ARM-mode.}
\IFRU{Далее функция работающая с таблицей.}{Then you may see the function for table processing.} 
\IFRU{Она слишком сложная для 
рассмотрения в данном месте, так что я пропущу объяснения.}
{It is too complex for describing it here now, so I will omit elaborations.}
%TODO дописать когда-то?

\index{ARM!\Registers!Link Register}
\IFRU{Но можно отметить, что эта функция использует регистр \LR как указатель на таблицу.}
{But it is interesting to note the function uses \LR register as a pointer to the table.}
\IFRU{Действительно, после вызова этой функции, в \LR был записан
адрес после инструкции}{Indeed, after this function calling, \LR will contain address after} \\ 
\TT{``BL \_\_ARM\_common\_switch8\_thumb''}\IFRU{, а там как раз и начинается таблица.}
{ instruction, and the table is beginning right there.}

\IFRU{Еще можно отметить что код для этого выделен в отдельную функцию для того, чтобы и в других местах,
в похожих случаях, обрабатывались \IT{switch()}-и и не нужно было каждый раз генерировать во всех этих
местах такой фрагмент кода.}
{It is also worth noting the code is generated as a separate function in order to reuse it, in similar places,
in similar cases, for \IT{switch()} processing, so compiler will not generate same code at each point.}
% код выделен -> en?

\IDA 
\IFRU{распознала эту служебную функцию и таблицу автоматически, дописав комментарии к меткам вроде}
{successfully perceived it as a service function and table, automatically, and added commentaries to labels
like} \TT{jumptable 000000FA case 0}.




\section{\IFRU{Циклы}{Loops}}

\input{loops/loops_x86}

\subsection{ARM}

\subsubsection{\NonOptimizingKeil + режим ARM}

\lstinputlisting{loops/Keil_ARM_O0.asm}

Счетчик итераций \TT{i} будет храниться в регистре \TT{R4}.

Инструкция \TT{``MOV R4, \#2''} просто инициализирует \IT{i}.

Инструкции \TT{``MOV R0, R4''} и \TT{``BL f''} составляют тело цикла, первая инструкция готовит аргумент для
функции f и вторая собственно вызывает её.

Инструкция \TT{``ADD R4, R4, \#1''} прибавляет единицу к \IT{i} при каждой итерации.

\TT{``CMP R4, \#0xA''} сравнивает \TT{i} с $0xA$ ($10$). Следующая за ней инструкция \TT{BLT} 
(\IT{Branch Less Than}) совершит переход, если \IT{i} меньше чем $10$.

В противном случае, в \TT{R0} запишется $0$ (потому что наша функция возвращает 0) и произойдет выход из функции.

\subsubsection{\OptimizingKeil + режим thumb}

\lstinputlisting{loops/Keil_thumb_O3.asm}

Практически, всё то же самое.

\subsubsection{\OptimizingXcode + режим thumb}

\lstinputlisting{loops/xcode_thumb_O3.asm}

На самом деле, в моей функции f было такое:

\begin{lstlisting}
void f(int i)
{
    // do something here
    printf ("%d\n", i);
};
\end{lstlisting}

Так что, LLVM не только \IT{развернул} цикл, но также и представил мою очень простую функцию \TT{f} как inline-вую,
и вставил её тело вместо цикла 8 раз. Это возможно когда функция очень простая, как та что у меня, и когда
она вызывается не очень много раз, как здесь.



\subsection{\IFRU{Еще кое-что}{One more thing}}

По генерируемому коду мы видим следующее: после инициализации \IT{i}, тело цикла не исполняется, а исполняется сразу
проверка условия \IT{i}, а лишь затем исполняется тело цикла. Это правильно. Потому что если условие в самом начале
не выполняется, тело цикла исполнять нельзя. Так может быть, например, в таком случае:

\lstinputlisting{loops/loops_3_ru.c}

Если \IT{total\_entries\_to\_process} равно нулю, тело цикла не должно исполниться ни разу. Поэтому проверка
условия происходит перед тем как исполнить само тело.

Впрочем, оптимизирующий компилятор может переставить проверку условия и тело цикла местами, если компилятор уверен,
что описанная здесь ситуация невозможна, как в случае с нашим примером и компиляторами Keil, MSVC, GCC в режиме
оптимизации.

\section{strlen()}
\index{\CStandardLibrary!strlen()}
\index{\CLanguageElements!while}

\IFRU{Еще немного о циклах. Часто, функция \TT{strlen()}\footnote{подсчет длины строки в Си} 
реализуется при помощи \TT{while()}.}
{Now let's talk about loops one more time. Often, \TT{strlen()} 
function\footnote{counting characters in string in C language} is implemented using \TT{while()} 
statement.}
\IFRU{Например, как это сделано в стандартных библиотеках MSVC:}
{Here is how it's done in MSVC standard libraries:}

\begin{lstlisting}
int strlen (const char * str)
{
        const char *eos = str;

        while( *eos++ ) ;

        return( eos - str - 1 );
}
\end{lstlisting}

\input{10_strlen/strlen_x86}

\subsection{ARM}

\subsubsection{\NonOptimizingXcode + \ARMMode}

\lstinputlisting[caption=\NonOptimizingXcode + \ARMMode]{10_strlen/xcode_ARM_O0_en.asm}

\IFRU{Неоптимизирующий LLVM генерирует слишком много кода, зато на этом примере можно посмотреть, 
как функции работают с локальными переменными в стеке.}
{Non-optimizing LLVM generates too much code, however, here we can see how function works with local variables
in stack.}
\IFRU{В нашей функции только локальных переменных две, это два указателя}{There are only two
local variables in our function}, \IT{eos} \IFRU{и}{and} \IT{str}.

\IFRU{В этом листинге}{In this listing}, \IFRU{сгенерированном при помощи}{generated by} \IDA, 
\IFRU{я переименовал}{I renamed} \IT{var\_8} \IFRU{и}{and} \IT{var\_4} \IFRU{в}{into} \IT{eos} 
\IFRU{и}{and} \IT{str} \IFRU{вручную}{manually}.

\IFRU{Итак, первые несколько инструкций просто сохраняют входное значение в переменных}{So, 
first instructions are just saves input value in} \IT{str} \IFRU{и}{and} \IT{eos}.

\IFRU{Начиная с метки}{Loop body is beginning at} \IT{loc\_2CB8}\IFRU{, начинается тело цикла}{ label}.

\IFRU{Первые три инструкции в теле цикла}{First three instruction in loop body} (\TT{LDR}, \ADD, \TT{STR}) 
\IFRU{загружают значение}{loads} \IT{eos} \IFRU{в}{value into} \Rzero, 
\IFRU{затем происходит инкремент значения и оно сохраняется назад в локальной переменной \IT{eos} расположенной 
в стеке.}{then value is incremented and it's saving back into \IT{eos} local variable located in stack.}

\IFRU{Следующая инструкция}{The next} \TT{``LDRSB R0, [R0]''} (\IT{Load Register Signed Byte}) 
\IFRU{загружает байт из памяти по адресу \Rzero, расширяет его до 32-бит считая его знаковым (signed) 
и сохраняет в \Rzero}{instruction loading byte from memory at \Rzero address and sign-extends it to 32-bit}.
\IFRU{Это немного похоже на инструкцию}{This is similar to} \MOVSX \IFRU{в}{instruction in} x86.
\IFRU{Компилятор считает этот байт знаковым (signed), потому что тип \Tchar по стандарту Си ~--- знаковый.}
{The compiler treating this byte as signed because \Tchar type in C standard is signed.}
\IFRU{Об это я уже немного писал}{I already wrote about it}~\ref{MOVSX} \IFRU{в этой же секции, 
но посвященной x86}{in this section, but related to x86}.

\IFRU{Следует также заметить, что, в ARM нет возможности использовать 8-битную или 16-битную часть регистра, 
как это возможно в x86.}
{It's should be noted, there are to way to use 8-bit part or 16-bit part of 32-bit register in ARM, as it's
possible in x86.}
\IFRU{Вероятно, это связано с тем что за x86 тянется длинный шлейф совместимости со своими предками, такими как
16-битный 8086 и даже 8-битный 8080, а ARM разрабатывался с чистого листа как 32-битный RISC-процессор.}
{Apparently, it's because x86 has a huge history of compatibility with its ancestors like 16-bit 8086 
and even 8-bit 8080,
but ARM was developed from scratch as 32-bit RISC-processor.}
\IFRU{Следовательно, чтобы работать с отдельными байтами на ARM, так или иначе, придется использовать 
32-битные регистры.}
{Consequently, in order to process separate bytes in ARM, one have to use 32-bit registers anyway.}

\IFRU{Итак}{So}, \TT{LDRSB} \IFRU{загружает символ из строки в \Rzero, по одному.}{loads symbol from string
into \Rzero, one by one.}
\IFRU{Следующие инструкции}{Next} \CMP \IFRU{и}{and} \TT{BEQ} \IFRU{проверяют, является ли этот символ нулем.}
{instructions checks, if loaded symbol is zero.}
\IFRU{Если не ноль, то происходит переход на начало тела цикла.}{If not zero, control passing to loop body
begin.}
\IFRU{А если ноль, выходим из цикла.}{And if zero, loop is finishing.}

\IFRU{В конце функции вычисляется разница между}{At the end of function, a difference between} 
\IT{eos} \IFRU{и}{and} \IT{str}\IFRU{, вычитается еще единица и вычисленное 
значение возвращается через \Rzero.}{ is calculated, 1 is also subtracting, and resulting value is returned
via \Rzero.}

\IFRU{Кстати, обратите внимание, в этой функции не сохранялись регистры.}{By the way, please note, registers
wasn't saved in this function.}
\IFRU{Это потому что, по стандарту, регистры \Rzero-\Rthree называются также ``scratch registers'',
они предназначены для передачи аргументов, 
их значения не нужно восстанавливать при выходе из функции, потому что они больше не нужны в вызывающей функции.
Таким образом, их можно использовать как захочется}
{That's because by ARM calling convention, \Rzero-\Rthree registers are ``scratch registers'', 
they are intended for arguments passing,
its values may not be restored upon function exit, because calling function will not use them anymore.
Consequently, they may be used for anything we want.}
\IFRU{А так как никакие больше регистры не используются, то и сохранять нечего.}
{Other registers are not used here, so that's why we have nothing to save in stack.}
\IFRU{Поэтому, управление можно вернуть назад вызывающей функции 
простым переходом (\TT{BX}), по адресу в регистре \LR.}
{Thus, control may be returned back to calling function by simple jump (\TT{BX}), to address in \LR register.}

%\subsubsection{\NonOptimizingXcode + режим thumb}
%Практически, точно такой же код.

\subsubsection{\OptimizingXcode + \ThumbMode}

\lstinputlisting[caption=\OptimizingXcode + \ThumbMode]{10_strlen/xcode_thumb_O3.asm}

\IFRU{Оптимизирующий LLVM решил что под переменные \IT{eos} и \IT{str} выделять место в стеке не обязательно}
{As optimizing LLVM concludes, place in stack for \IT{eos} and \IT{str} may not be allocated},
\IFRU{и эти переменные можно хранить прямо в регистрах.}
{and these variables may always be stored right in registers.}
\IFRU{Перед началом тела цикла}{Before loop body beginning}, \IT{str} \IFRU{будет находиться в}{will always be in} 
\Rzero, \IFRU{а}{and} \IT{eos} ~--- \IFRU{в}{in} \Rone.

\IFRU{Инструкция }{}\TT{``LDRB.W R2, [R1],\#1''} \IFRU{загружает в \Rtwo байт из памяти по адресу \Rone, 
расширяя его как знаковый (signed), до 32-битного
значения, но не только это.}
{instruction loads byte from memory at the address \Rone into \Rtwo, sign-extending it to 32-bit value, but not
only that.}
\TT{\#1} \IFRU{в конце инструкции называется}{at the instruction's end calling} ``Post-indexed addressing'', 
\IFRU{это значит что после загрузки байта, к \Rone добавится единица.}{this mean, $1$ is to be added
to \Rone after byte load.}
\IFRU{Это очень удобно для работы с массивами.}{That's handy when accessing arrays.}

\IFRU{Такого режима адресации в x86 нет, но он есть в некоторых других процессорах, даже на PDP-11.}
{There are no such addressing mode in x86, but it's present in some other processors, even on PDP-11.}
\IFRU{Существует байка, что режимы пре-инкремента, пост-инкремента, 
пре-декремента и пост-декремента адреса в PDP-11}
{There is a legend that pre-increment, post-increment, pre-decrement and post-decrement modes in PDP-11},
\IFRU{были ``виновны'' в появлении таких конструктов языка Си (который разрабатывался на PDP-11) как}
{were ``guilty'' in appearance such C language (which developed on PDP-11) constructs as}
*ptr++, *++ptr, *ptr-{}-, *-{}-ptr. 
\IFRU{Кстати, это является труднозапоминаемой особенностью в Си.}
{By the way, this is one of hard to memorize C feature.}
\IFRU{Дела обстоят так:}{This is how it is:}

\begin{center}
\begin{tabular}{ | l | l | l | l | }
\hline                        
\cellcolor{blue!25} \IFRU{термин в Си}{C term} & 
\cellcolor{blue!25} \IFRU{термин в ARM}{ARM term} & 
\cellcolor{blue!25} \IFRU{выражение Си}{C statement} & 
\cellcolor{blue!25}\IFRU{как это работает}{how it works} \\
\hline                        
\IFRU{Пост-инкремент}{Post-increment} & 
post-indexed addressing & 
\TT{*ptr++} & 
\IFRU{использовать значение \TT{*ptr}}{use \TT{*ptr} value}, \\
& & & \IFRU{затем икремент указателя \TT{ptr}}{then increment \TT{ptr} pointer} \\
\hline                        
\IFRU{Пост-декремент}{Post-decrement} & 
post-indexed addressing & 
\TT{*ptr-{}-} & 
\IFRU{использовать значение \TT{*ptr}}{use \TT{*ptr} value}, \\
& & & \IFRU{затем декремент указателя \TT{ptr}}{then decrement \TT{ptr} pointer} \\
\hline                        
\IFRU{Пре-инкремент}{Pre-increment} & 
pre-indexed addressing & 
\TT{*++ptr} & 
\IFRU{инкремент указателя \TT{ptr}}{increment \TT{ptr} pointer}, \\
& & & \IFRU{затем использовать значение \TT{*ptr}}{then use \TT{*ptr} value} \\
\hline                        
\IFRU{Пре-декремент}{Pre-decrement} & 
post-indexed addressing & 
\TT{*-{}-ptr} & 
\IFRU{декремент указателя \TT{ptr}}{decrement \TT{ptr} pointer}, \\
& & & \IFRU{затем использовать значение \TT{*ptr}}{then use \TT{*ptr} value} \\
\hline  
\end{tabular}
\end{center}

\IFRU{Деннис Ритчи (один из создателей ЯП Си) указывал, что, это, вероятно, придумал Кен Томпсон 
(еще один создатель Си),
потому что подобная возможность процессора имелась еще в PDP-7}
{Dennis Ritchie (one of C language creators) mentioned that it's, probably, was invented by Ken Tompson
(another C creator) because this processor feature was present in PDP-7}
\cite{Ritchie:1986}\cite{Ritchie:1993:DCL:155360.155580}.
\IFRU{Таким образом, компиляторы с ЯП Си на тот процессор, где это есть, могут использовать это.}
{Thus, C language compilers may use it, if it's present in targer processor.}

Итак, далее в теле цикла \CMP и \TT{BNE} продолжают работу цикла, до тех пор, пока не будет встречен $0$.

После конца цикла \TT{MVNS}\footnote{MoVe Not} (инвертирование всех бит в значении, аналог \NOT на x86) 
и \ADD вычисляют $eos - str - 1$. 
На самом деле, эти две инструкции вычисляют $R0 = ~str + eos$, что эквивалентно тому, что было в исходном коде, 
а почему это так, я уже описывал чуть раньше, здесь~\ref{strlen_NOT_ADD}. 
Вероятно, LLVM, как и GCC, посчитал что так будет короче, или быстрее.

%\subsubsection{\OptimizingXcode + \ARMMode}
%Практически, точно такой же код.

\subsubsection{\OptimizingKeil{} + \ARMMode}

\lstinputlisting[caption=\OptimizingKeil + \ARMMode]{10_strlen/Keil_ARM_O3.asm}

Практически то же самое что мы уже видели, за тем исключением что выражение $str - eos - 1$ может быть вычислено
не в самом конце функции, а прямо в теле цикла. 
Суффикс \TT{-EQ}, как мы помним, означает что инструкция будет выполнена только
если операнды в исполненной перед этим инструкции \CMP были равны. 
Таким образом, если в \Rzero будет $0$, обе инструкции \TT{SUBEQ} исполнятся и результат останется в \Rzero.



\section{\DivisionByNineSectionName}
\label{sec:divisionbynine}

\IFRU{Простая функция:}{Very simple function:}

\begin{lstlisting}
int f(int a)
{
	return a/9;
};
\end{lstlisting}

\IFRU{Компилируется вполне предсказуемо:}{Is compiled in a very predictable way:}

\lstinputlisting{\IFRU{11_division_by_9/11_1_msvc_ru.asm}{11_division_by_9/11_1_msvc_en.asm}}

\IFRU{\IDIV делит 64-битное число хранящееся в паре регистров \TT{EDX:EAX} на значение в \ECX. 
В результате, \EAX будет содержать частное\footnote{результат деления}, а \EDX ~--- остаток от деления. 
Результат возвращается из функции через \EAX, так что после операции деления, 
это значение не перекладывается больше никуда, 
оно уже там где надо.}
{\IDIV divides 64-bit number stored in \TT{EDX:EAX} register pair by value in \ECX register. 
As a result, \EAX will contain quotient\footnote{result of division}, and \EDX ~--- remainder.
Result is returning from f() function in \EAX register, so, that value is not moved anymore after division 
operation, it is in right place already.}
\IFRU
{Из-за того что \IDIV требует пару регистров \TT{EDX:EAX}, то перед этим инструкция \TT{CDQ} 
расширяет \EAX до 64-битного значения учитывая знак, также как это делает \MOVSX.}
{Because \IDIV require value in \TT{EDX:EAX} register pair, \TT{CDQ} instruction (before \IDIV) extending 
\EAX value to 64-bit value taking value sign into account, just as \MOVSX does.}
\IFRU{Со включенной оптимизацией (\Ox) получается:}
{If we turn optimization on (\Ox), we got:}

\lstinputlisting{11_division_by_9/11_1_msvc_Ox.asm}

\newcommand{\URLMSDN}{\href{http://blogs.msdn.com/b/devdev/archive/2005/12/12/502980.aspx}
{MSDN: Integer division by constants}}
\newcommand{\URLN}{http://www.nynaeve.net/?p=115}

\IFRU{Это ~--- деление через умножение. Умножение конечно быстрее работает. 
Поэтому можно используя этот трюк
\footnote{Читайте подробнее о делении через умножение в книге 
``Генри Уоррен, мл. ~--- Алгоритмические трюки для программистов'' 
(глава 10 ~--- ``Целое деление на константы''): \URLMSDN, \url{\URLN}} 
создать код эквивалентный тому что мы хотим и работающий быстрее.}
{This is ~--- division using multiplication. Multiplication operation working much faster. 
And it is possible to use that trick
\footnote{Read more about division by multiplication in 
``Henry S. Warren Jr. ~--- Hacker's Delight'' book (chapter 10 ~--- ``Integer Division By Constants'') 
and: \URLMSDN, \url{\URLN}} 
to produce a code which is equivalent and faster.}
\IFRU
{GCC 4.4.1 даже без включенной оптимизации генерит примерно такой же код как и MSVC с оптимизацией:}
{GCC 4.4.1 even without optimization turned on, generate almost the same code as MSVC with optimization turned on:}

\lstinputlisting{11_division_by_9/11_2_gcc.asm}

\subsection{ARM}

\IFRU{В процессоре ARM, как и во многих других ``чистых'' (pure) RISC-процессорах нет инструкции деления,
да и инструкции умножения на 32-битную константу также нет.}{In ARM processor, just like in any other ''pure'' 
RISC-processors, there are no division instruction, instruction for multiplication by 32-bit constant 
is absent too.}
\IFRU{При некотором желании, можно обойтись только тремя действиями: сложением, вычитанием и 
битовыми сдвигами}{With some effort, it's possible to do division using only three instructions: addition,
subtraction and bit shifts}~\ref{sec:bitfields}.

\IFRU{Пример деления 32-битного числа на 10 из книги}{Here is an example of 32-bit number division by 10 from the} 
ARM Cookbook (1994)\footnote{\href{http://yurichev.com/ref/ARM\%20Cookbook\%20(1994)/cook3.txt}{ARM Cookbook (1994)}}. 
\IFRU{На выходе и частное и остаток}{Quotient and remainder on output}.

\begin{lstlisting}
; takes argument in a1
; returns quotient in a1, remainder in a2
; cycles could be saved if only divide or remainder is required
    SUB    a2, a1, #10             ; keep (x-10) for later
    SUB    a1, a1, a1, lsr #2
    ADD    a1, a1, a1, lsr #4
    ADD    a1, a1, a1, lsr #8
    ADD    a1, a1, a1, lsr #16
    MOV    a1, a1, lsr #3
    ADD    a3, a1, a1, asl #2
    SUBS   a2, a2, a3, asl #1      ; calc (x-10) - (x/10)*10
    ADDPL  a1, a1, #1              ; fix-up quotient
    ADDMI  a2, a2, #10             ; fix-up remainder
    MOV    pc, lr
\end{lstlisting}

\subsubsection{\OptimizingXcode + \ARMMode}

\begin{lstlisting}
__text:00002C58                         _f
__text:00002C58 39 1E 08 E3 E3 18 43 E3                 MOV             R1, 0x38E38E39
__text:00002C60 10 F1 50 E7                             SMMUL           R0, R0, R1
__text:00002C64 C0 10 A0 E1                             MOV             R1, R0,ASR#1
__text:00002C68 A0 0F 81 E0                             ADD             R0, R1, R0,LSR#31
__text:00002C6C 1E FF 2F E1                             BX              LR
\end{lstlisting}

Этот код почти тот же, что сгенерирован MSVC и GCC в режиме оптимизации. Должно быть, LLVM использует тот же
алгоритм для поиска констант.

Наблюдательный читатель может спросить, как \MOV записала в регистр сразу 32-битное число, ведь это невозможно.
Действительно невозможно, но как мы видим, здесь на инструкцию 8 байт вместо стандартных 4-х.
Первая инструкция загружает в младшие 16 бит регистра значение $0x8E39$, а вторая инструкция, 
на самом деле \TT{MOVT},
загружающая в старшие 16 бит регистра значение $0x383E$. \IDA распознала эту последовательность и для удобства
выдала только одну инструкцию.

Инструкция \TT{SMMUL} (\IT{Signed Most Significant Word Multiply}) умножает числа считая их знаковыми (signed)
и оставляет в R0 старшие 32 бита результата, не сохраняя младшие 32 бита.

Инструкция \TT{``MOV R1, R0,ASR\#1''} это арифметический сдвиг право на один бит.

\TT{``ADD R0, R1, R0,LSR\#31''} это $R0=R1 + R0>>31$

Дело в том что в режиме ARM нет отдельных инструкций для битовых сдвигов. 
Вместо этого, некоторые инструкции (\MOV, \ADD,
\SUB, \TT{RSB}) могут быть
дополнеты пометкой, сдвигать ли второй операнд и если да, то на сколько и как. 
\TT{ASR} означает \IT{Arithmetic Shift Right}, \TT{LSR} ~--- \IT{Logican Shift Right}.

\subsubsection{\OptimizingXcode + режим thumb}

\begin{lstlisting}
MOV             R1, 0x38E38E39
SMMUL.W         R0, R0, R1
ASRS            R1, R0, #1
ADD.W           R0, R1, R0,LSR#31
BX              LR
\end{lstlisting}

В режиме thumb отдельные инструкции для битовых сдвигов есть, и здесь применяется одна из них ~--- ASRS 
(арифметический сдвиг вправо).

\subsubsection{Неоптимизирующие Xcode (LLVM) и Keil}

Неоптимизирующий LLVM не занимается генерацией подобного кода а вместо этого просто вставляет вызов
библиотечной функции \IT{\_\_\_divsi3}. А Keil во всех случаях вставляет вызов функции \IT{\_\_aeabi\_idivmod}.


\input{12_FPU/FPU}
\section{\IFRU{Массивы}{Arrays}}
\label{arrays}

\IFRU{Массив это просто набор переменных в памяти, обязательно лежащих рядом, и обязательно одного типа.}
{Array is just a set of variables in memory, always lying next to each other, always has same type.}

\subsection{\IFRU{Простой пример}{Simple example}}

\lstinputlisting{arrays/simple.c}

\subsubsection{x86}

\IFRU{Компилируем}{Let's compile}:

\lstinputlisting{arrays/simple_msvc.asm}

\IFRU{Однако, ничего особенного, просто два цикла, один заполняет цикл, второй печатает его содержимое. 
Команда \TT{shl ecx, 1} используется для умножения \ECX на 2, об этом ниже~\ref{SHR}.}
{Nothing very special, just two loops: first is filling loop and second is printing loop.
\TT{shl ecx, 1} instruction is used for \ECX value multiplication by 2, more about below~\ref{SHR}.}

\IFRU{Под массив выделено в стеке 80 байт, это 20 элементов по 4 байта.}
{80 bytes are allocated in stack for array, that's 20 elements of 4 bytes.}

\IFRU{То что делает GCC 4.4.1:}{Here is what GCC 4.4.1 does:}

\lstinputlisting{arrays/simple_gcc.asm}

\IFRU{Кстати, переменная \IT{a} в нашем примере имеет тип \IT{int*} (то есть, указатель на \Tint{}) ~--- вы можете попробовать передать в другую функцию указатель на массив, но точнее было бы сказать что передается указатель на первый элемент массива (а адреса остальных элементов массива можно вычислить очевидным образом).}{By the way, \IT{a} variable has \IT{int*} type (that is pointer to \Tint{}) ~--- you can try to pass a pointer to array to another function, but it much correctly to say that pointer to the first array elemnt is passed (addresses of another element's places are calculated in obvious way).}
\IFRU{Если индексировать этот указатель как \IT{a[idx]}, \IT{idx} просто прибавляется к указателю и возвращается элемент, расположенный там, куда ссылается вычисленный указатель.}{If to index this pointer as \IT{a[idx]}, \IT{idx} just to be added to the pointer and the element placed there (to which calculated pointer is pointing) returned.}

\IFRU{Вот любопытный пример: строка символов вроде \IT{``string''} это массив из символов, и она имеет тип \IT{const char*}.}{An interesting example: string of characters like \IT{``string''} is array of characters and it has \IT{const char*} type.}\IFRU{К этому указателю также можно применять индекc.}{Index can be applied to this pointer.}
\IFRU{И поэтому можно написать даже так:  \TT{``string''[i]} ~--- это совершенно легальное выражение в \CCpp!}{And that's why it's possible to write like \TT{``string''[i]} ~--- this is correct \CCpp expression!}


\subsubsection{ARM + \NonOptimizingKeil + режим ARM}

\lstinputlisting{arrays/simple_Keil_ARM_O0_en.asm}

Тип \Tint требует 32 бита для хранения, или 4 байта, так что для хранения 20 переменных, нужно 80 (0x50) байт,
поэтому инструкция \TT{``SUB SP, SP, \#0x50''} в эпилоге функции выделяет в локальном стеке место под массив.

И в первом и во втором цикле, итератор цикла $i$ будет постоянно находится в регистре R4.

Число, которое нужно записать в массив, вычисляется так $i*2$, но это эквивалентно сдвигу на 1 бит влево, 
инструкция \TT{``MOV R0, R4,LSL\#1''} делает это.

\TT{``STR R0, [SP,R4,LSL\#2]''} записывает R0 в массив. Указатель на элемент массива вычисляется так: \SP указывает
на начало массива, R4 это $i$. Так что сдвигаем $i$ на 2 бита влево, что эквивалентно умножению на 4 (ведь каждый
элемент массива занимает 4 байта) и прибавляем к адресу начала массива.

Во втором цикле используется обратная инструкция \TT{``LDR R2, [SP,R4,LSL\#2]''}, она загружает из массива нужное
значение, и указатель на него вычисляется точно так же.

\subsubsection{ARM + \OptimizingKeil + режим thumb}

\lstinputlisting{arrays/simple_Keil_thumb_O3_en.asm}

Код для thumb очень похожий. В thumb имеются отдельные инструкции для битовых сдвигов (\TT{LSLS}), вычисляющие и
число для записи в массив и адрес каждого элемента массива.

Компилятор почему-то выделил в локальном стеке немного больше места, однако последние 4 байта не используются.




\section{\RU{Переполнение буфера}\EN{Buffer overflow}}
\label{subsec:bufferoverflow}
\index{\BufferOverflow}

\RU{Итак, индексация массива ~--- это просто \IT{массив\lbrack{}индекс\rbrack}.  % TODO как-то плохо отображаются []
Если вы присмотритесь к коду, в цикле печати значений массива через \printf вы 
не увидите проверок индекса, \IT{меньше ли он двадцати?} 
А что будет если он будет больше двадцати? 
Эта одна из особенностей \CCpp, за которую их, собственно, и ругают.}
\EN{So, array indexing is just \IT{array\lbrack{}index\rbrack}.
If you study generated code closely, you'll probably note missing index bounds checking,
which could check index, \IT{if it is less than 20}.
What if index will be greater than 20?
That's the one \CCpp feature it is often blamed for.}

\RU{Вот код который и компилируется и работает:}
\EN{Here is a code successfully compiling and working:}

\begin{lstlisting}
#include <stdio.h>

int main() 
{
	int a[20];
	int i;

	for (i=0; i<20; i++)
		a[i]=i*2;

	printf ("a[100]=%d\n", a[100]);

	return 0;
};
\end{lstlisting}

\RU{Вот в это}\EN{Compilation results} (MSVC 2010):

\lstinputlisting{patterns/13_arrays/BO2_msvc.asm}

\RU{У меня оно при запуске выдало вот это:}\EN{I'm running it, and I got:}

\begin{lstlisting}
a[100]=760826203
\end{lstlisting}

\RU{Это просто \IT{что-то}, что волею случая лежало в стеке рядом с массивом, 
через 400 байт от его первого элемента.}
\EN{It is just \IT{something}, occasionally lying in the stack near to array, 400 bytes from its first element.}

\RU{Действительно, а как могло бы быть иначе? Компилятор мог бы встроить какой-то код, 
каждый раз проверяющий индекс на соответствие пределам массива, как в языках программирования 
более высокого уровня\footnote{Java, Python, и т.д.}, что делало бы запускаемый код медленнее.}
\EN{Indeed, how it could be done differently?
Compiler may generate some additional code for checking index value to be always
in array's bound (like in higher-level programming languages\footnote{Java, Python, etc})
but this makes running code slower.}

\RU{Итак, мы прочитали какое-то число из стека явно \IT{нелегально}, а что если мы запишем?}
\EN{OK, we read some values from the stack \IT{illegally} but what if we could write something to it?}

\RU{Вот что мы пишем:}\EN{Here is what we will write:}

\begin{lstlisting}
#include <stdio.h>

int main() 
{
	int a[20];
	int i;

	for (i=0; i<30; i++)
		a[i]=i;

	return 0;
};
\end{lstlisting}

\RU{И вот что имеем на ассемблере:}\EN{And what we've got:}

\lstinputlisting{patterns/13_arrays/BO_\LANG.asm}

\RU{Запускаете скомпилированную программу, и она падает. Немудрено. Но давайте теперь узнаем, где именно.}
\EN{Run compiled program and its crashing. No wonder. Let's see, where exactly it is crashing.}

\index{tracer}
\RU{Отладчик я уже давно не использую, так как надоело для всяких мелких задач вроде подсмотреть состояние 
регистров, запускать что-то, двигать мышью, и т.д. 
Поэтому я написал очень минималистическую утилиту для себя, \tracer, коей обхожусь.}
\EN{I'm not using debugger anymore since I tried to run it each time, move mouse, etc, when I need just to
spot a register's state at the specific point.
That's why I wrote very minimalistic tool for myself, \tracer, which is enough for my tasks.}

\RU{Помимо всего прочего, я могу использовать мою утилиту просто чтобы посмотреть 
где и какое исключение произошло. 
Итак, пробую:}
\EN{I can also use it just to see, where \gls{debuggee} is crashed.
So let's see:}

\begin{lstlisting}
generic tracer 0.4 (WIN32), http://conus.info/gt

New process: C:\PRJ\...\1.exe, PID=7988
EXCEPTION_ACCESS_VIOLATION: 0x15 (<symbol (0x15) is in unknown module>), ExceptionInformation[0]=8
EAX=0x00000000 EBX=0x7EFDE000 ECX=0x0000001D EDX=0x0000001D
ESI=0x00000000 EDI=0x00000000 EBP=0x00000014 ESP=0x0018FF48
EIP=0x00000015
FLAGS=PF ZF IF RF
PID=7988|Process exit, return code -1073740791
\end{lstlisting}

\RU{Итак, следите внимательно за регистрами.}
\EN{Now please keep your eyes on registers.}

\RU{Исключение произошло по адресу 0x15. Это явно нелегальный адрес для кода ~--- по крайней мере, win32-кода! 
Мы там как-то очутились, причем, сами того не хотели. Интересен также тот факт, что в \EBP хранится 0x14, 
а в \ECX и \EDX ~--- 0x1D.}
\EN{Exception occurred at address 0x15. It is not legal address for code~---at least for win32 code!
We trapped there somehow against our will.
It is also interesting fact the \EBP register contain 0x14,
\ECX and \EDX{}~---0x1D.}

\RU{И еще немного изучим разметку стека.}\EN{Let's study stack layout more.}

\RU{После того как управление передалось в \main, в стек было сохранено значение \EBP. 
Затем, для массива + переменной \IT{i} было выделено $84$ байта. Это \TT{(20+1)*sizeof(int)}. 
\ESP сейчас указывает на переменную \TT{\_i} в локальном стеке и при исполнении следующего \TT{PUSH что-либо}, 
\IT{что-либо} появится рядом с \TT{\_i}.}
\EN{After control flow was passed into \TT{\main}, the value in the \EBP register was saved on the stack.
Then, $84$ bytes was allocated for array and \IT{i} variable.
That's \TT{(20+1)*sizeof(int)}.
The \ESP pointing now to the \TT{\_i} variable in the local stack and after execution of next \TT{PUSH something},
\IT{something} will be appeared next to \TT{\_i}.}

\RU{Вот так выглядит разметка стека пока управление находится внутри}
\EN{That's stack layout while control is inside} \main:

\begin{center}
\begin{tabular}{ | l | l | }
\hline
  \TT{ESP}    & \RU{4 байта для \IT{i}}\EN{4 bytes for \IT{i}} \\
  \TT{ESP+4}  & \RU{80 байт для массива \TT{a[20]}}\EN{80 bytes for \TT{a[20]} array} \\
  \TT{ESP+84} & \RU{сохраненное значение \EBP}\EN{saved \EBP value} \\
  \TT{ESP+88} & \RU{адрес возврата}\EN{returning address} \\
\hline
\end{tabular}
\end{center}

\RU{Команда \TT{a[19]=чего\_нибудь} записывает последний \Tint в пределах массива (пока что в пределах!)}
\EN{Instruction \TT{a[19]=something} writes last \Tint in array bounds (in bounds so far!)}

\RU{Команда \TT{a[20]=чего\_нибудь} записывает \IT{чего\_нибудь} на место где сохранено значение \EBP.}
\EN{Instruction \TT{a[20]=something} writes \IT{something} to the place where value from the \EBP is saved.}

\RU{Обратите внимание на состояние регистров на момент падения процесса. В нашем случае, 
в 20-й элемент записалось значение 20. 
И вот все дело в том, что заканчиваясь, эпилог функции восстанавливал значение \EBP. 
(20 в десятичной системе это как раз 0x14 в шестнадцатеричной). 
Далее выполнилась инструкция \RET, которая на самом деле эквивалентна \TT{POP EIP}.}
\EN{Please take a look at registers state at the crash moment. In our case,
number 20 was written to 20th element. 
By the function ending, function epilogue restores original \EBP value.
(20 in decimal system is 0x14 in hexadecimal).
Then, \RET instruction was executed, which is effectively equivalent to \TT{POP EIP} instruction.}

\RU{Инструкция \RET вытащила из стека адрес возврата (это адрес где-то внутри \ac{CRT}), 
которая вызвала \main), 
а там было записано 21 в десятичной системе, то есть 0x15 в шестнадцатеричной. 
И вот процессор оказался по адресу 0x15, но исполняемого кода там нет, так что случилось исключение.}
\EN{\RET instruction taking returning address from the stack (that is the address inside of \ac{CRT}),
which was called \main),
and 21 was stored there (0x15 in hexadecimal).
The CPU trapped at the address 0x15,
but there is no executable code, so exception was raised.}

\index{\RU{Переполнение буфера}\EN{Buffer overflow}}
\RU{Добро пожаловать! Это называется}
\EN{Welcome! It is called} \IT{buffer overflow}\footnote{\url{http://en.wikipedia.org/wiki/Stack_buffer_overflow}}.

\RU{Замените массив \Tint на строку (массив \Tchar), нарочно создайте слишком длинную строку, 
просуньте её в ту программу, 
в ту функцию, которая не проверяя длину строки скопирует её в слишком короткий буфер, 
и вы сможете указать программе, по какому именно адресу перейти. 
Не все так просто в реальности, конечно, но началось все с этого
\footnote{Классическая статья об этом: \cite{Phrack4914}}.}
\EN{Replace \Tint array by string (\Tchar array), create a long string deliberately,
and pass it to the program, to the function which is not checking string length and copies it to short buffer,
and you'll able to point to a program an address to which it must jump.
Not that simple in reality, but that is how it was emerged
\footnote{Classic article about it: \cite{Phrack4914}.}}

\RU{Попробуем то же самое в GCC 4.4.1. У нас выходит такое:}\EN{Let's try the same code in GCC 4.4.1. We got:}

\lstinputlisting{patterns/13_arrays/BO2_gcc.asm}

\RU{Запуск этого в Linux выдаст:}\EN{Running this in Linux will produce:} \TT{Segmentation fault}.

\index{GDB}
\RU{Если запустить полученное в отладчике GDB, получим:}
\EN{If we run this in GDB debugger, we getting this:}

\begin{lstlisting}
(gdb) r
Starting program: /home/dennis/RE/1 

Program received signal SIGSEGV, Segmentation fault.
0x00000016 in ?? ()
(gdb) info registers
eax            0x0	0
ecx            0xd2f96388	-755407992
edx            0x1d	29
ebx            0x26eff4	2551796
esp            0xbffff4b0	0xbffff4b0
ebp            0x15	0x15
esi            0x0	0
edi            0x0	0
eip            0x16	0x16
eflags         0x10202	[ IF RF ]
cs             0x73	115
ss             0x7b	123
ds             0x7b	123
es             0x7b	123
fs             0x0	0
gs             0x33	51
(gdb) 
\end{lstlisting}

\RU{Значения регистров немного другие чем в примере win32, это потому что разметка стека чуть другая.}
\EN{Register values are slightly different then in win32 example
since stack layout is slightly different too.}

\subsection{\IFRU{Защита от переполнения буфера}{Buffer overflow protection methods}}

\newcommand{\URLWPB}{\href{http://en.wikipedia.org/wiki/Buffer_overflow_protection}
{Wikipedia: \IFRU{описания защит, которые компилятор может вставлять в код}
{compiler-side buffer overflow protection methods}}}

\IFRU{В наше время пытаются бороться с этой напастью, не взирая на халатность программистов на \CCpp. 
В MSVC есть опции вроде\footnote{\URLWPB}:}
{There are several methods to protect against it, regardless of \CCpp programmers' negligence.
MSVC has options like\footnote{\URLWPB}:}

\begin{verbatim}
 /RTCs Stack Frame runtime checking
 /GZ Enable stack checks (/RTCs)
\end{verbatim}

\IFRU{Один из методов, это в прологе функции вставлять в область локальных переменных 
некоторое случайное значение 
и в эпилоге функции, перед выходом, это число проверять. 
И если проверка не прошла, то не выполнять инструкцию \RET а остановиться (или зависнуть). 
Процесс зависнет, но это лучше чем удаленная атака на ваш хост.}
{One of the methods is to write random value among local variables to stack at function prologue 
and to check it in function epilogue before function exiting.
And if value is not the same, do not execute last instruction \RET, but halt (or hang).
Process will hang, but that's much better then remote attack to your host.}

\subsection{\IFRU{Еще немного о массивах}{One more word about arrays}}

\IFRU{Теперь понятно, почему нельзя написать в исходном коде на \CCpp что-то вроде:
\footnote{GCC способен это сделать выделяя место под массив динамически в стеке (как alloca()~\ref{alloca}), 
но это расширение не является частью стандарта}}
{Now we understand, why it's not possible to write something like that in \CCpp code
\footnote{GCC can actually do this by allocating array dynammically in stack (like alloca()~\ref{alloca}), 
but it's not standard langauge extension}:}

\begin{lstlisting}
void f(int size)
{
    int a[size];
...
};
\end{lstlisting}

\IFRU{Все просто потому, чтобы выделять место под массив в локальном стеке или же сегменте данных 
(если массив глобальный), компилятору нужно знать его размер, чего он, на стадии компиляции, 
разумеется знать не может.}
{That's just because compiler should know exact array size to allocate place for it in local stack layout or
in data segment (in case of global variable) on compiling stage.}

\IFRU{Если вам нужен массив произвольной длины, то выделите столько, сколько нужно, через \TT{malloc()}, 
затем обращайтесь к выделенному блоку байт как к массиву того типа, который вам нужен.}
{If you need array of arbitrary size, allocate it by \TT{malloc()}, then access allocated memory block
as array of variables of type you need.}

\subsection{\IFRU{Многомерные массивы}{Multidimensional arrays}}

\IFRU{Многомерный массив выглядит внутри так же как и линейный.}
{Internally, multidimensional array is essentially the same thing as linear array.}

\IFRU{Попробуем:}{Let's see:}

\begin{lstlisting}
#include <stdio.h>

int a[10][20][30];

void insert(int x, int y, int z, int value)
{
	a[x][y][z]=value;
};
\end{lstlisting}

\IFRU{В итоге}{We got} (MSVC 2010):

\lstinputlisting{arrays/13_5_msvc.asm}

\IFRU{В принципе, ничего удивительного. В \TT{insert()} для индексирования нужного элемента массива, 
три входных аргумента перемножаются так, чтобы представить массив трехмерным.}
{Nothing special. For index calculation, three input arguments are multiplying 
in such way to represent array as multidimensional.}

GCC 4.4.1:

\lstinputlisting{arrays/13_5_gcc.asm}


\subsection{\IFRU{Работа с битовыми полями в структуре}{Bit fields in structure}}

\subsubsection{\IFRU{Пример CPUID}{CPUID example}}

\IFRU{Язык \CCpp позволяет указывать, сколько именно бит отвести для каждого поля структуры. 
Это удобно если нужно экономить место в памяти. К примеру, для переменной типа \Tbool достаточно одного бита.
Но, это не очень удобно, если нужна скорость.}
{\CCpp language allow to define exact number of bits for each structure fields.
It is very useful if one needs to save memory space. 
For example, one bit is enough for variable of \Tbool type.
But of course, it is not rational if speed is important.}

\newcommand{\FNCPUID}{\footnote{\url{http://en.wikipedia.org/wiki/CPUID}}}

\index{x86!\Instructions!CPUID}
\label{cpuid}
\IFRU{Рассмотрим пример с инструкцией \CPUID\FNCPUID. 
Эта инструкция возвращает информацию о том, какой процессор имеется в наличии и какие возможности он имеет.}
{Let's consider \CPUID\FNCPUID instruction example.
This instruction returning information about current CPU and its features.}

\IFRU{Если перед исполнением инструкции в \EAX будет 1, 
то \CPUID вернет упакованную в \EAX такую информацию о процессоре:}
{If the \EAX is set to 1 before instruction execution, 
\CPUID will return this information packed into the \EAX register:}

\begin{center}
\begin{tabular}{ | l | l | }
\hline
3:0 & Stepping \\
7:4 & Model \\
11:8 & Family \\
13:12 & Processor Type \\
19:16 & Extended Model \\
27:20 & Extended Family \\
\hline
\end{tabular}
\end{center}

\newcommand{\FNGCCAS}{\footnote{\href{http://www.ibiblio.org/gferg/ldp/GCC-Inline-Assembly-HOWTO.html}
{\IFRU{Подробнее о встроенном ассемблере GCC}{More about internal GCC assembler}}}}

\IFRU{MSVC 2010 имеет макрос для \CPUID, а GCC 4.4.1 ~--- нет. 
Поэтому для GCC сделаем эту функцию сами, используя его встроенный ассемблер\FNGCCAS.}
{MSVC 2010 has \CPUID macro, but GCC 4.4.1~---has not.
So let's make this function by yourself for GCC with the help of its built-in assembler\FNGCCAS.}

\lstinputlisting{patterns/15_structs/CPUID.c}

\IFRU{После того как \CPUID заполнит \EAX/\EBX/\ECX/\EDX, у нас они отразятся в массиве \TT{b[]}. 
Затем, мы имеем указатель на структуру \TT{CPUID\_1\_EAX}, и мы указываем его на значение 
\EAX из массива \TT{b[]}.}
{After \CPUID will fill \EAX/\EBX/\ECX/\EDX, these registers will be reflected in the \TT{b[]} array.
Then, we have a pointer to the \TT{CPUID\_1\_EAX} structure and we point it to the value in the \EAX from \TT{b[]} array.}

\IFRU{Иными словами, мы трактуем 32-битный \Tint как структуру.}
{In other words, we treat 32-bit \Tint value as a structure.}

\IFRU{Затем мы читаем из структуры.}{Then we read from the stucture.}

\IFRU{Компилируем в MSVC 2008 с опцией \Ox}{Let's compile it in MSVC 2008 with \Ox option}:

\lstinputlisting[caption=\Optimizing MSVC 2008]{patterns/15_structs/CPUID_msvc_Ox.asm}

\index{x86!\Instructions!SHR}
\IFRU{Инструкция \TT{SHR} сдвигает значение из \EAX на то количество бит, 
которое нужно \IT{пропустить}, то есть, мы игнорируем некоторые биты \IT{справа}.}
{\TT{SHR} instruction shifting value in the \EAX register by number of bits must be
\IT{skipped}, e.g., we ignore a bits \IT{at right}.}

\index{x86!\Instructions!AND}
\IFRU{А инструкция \ANDIns очищает биты \IT{слева} которые нам не нужны, или же, говоря иначе, 
она оставляет по маске только те биты в \EAX, которые нам сейчас нужны.}
{\ANDIns instruction clears bits not needed \IT{at left}, or, in other words, 
leaves only those bits in the \EAX register we need now.}

\IFRU{Попробуем GCC 4.4.1 с опцией \Othree.}{Let's try GCC 4.4.1 with \Othree option.}

\lstinputlisting[caption=\Optimizing GCC 4.4.1]{patterns/15_structs/CPUID_gcc_O3.asm}

\IFRU{Практически, то же самое. Единственное что стоит отметить это то, что GCC решил зачем-то объединить 
вычисление \TT{extended\_model\_id} и \TT{extended\_family\_id} в один блок, 
вместо того чтобы вычислять их перед соответствующим вызовом \printf.}
{Almost the same.
The only thing worth noting is the GCC somehow united calculation of
\TT{extended\_model\_id} and \TT{extended\_family\_id} into one block,
instead of calculating them separately, before corresponding each \printf call.}

\subsubsection{\WorkingWithFloatAsWithStructSubSubSectionName}
\label{sec:floatasstruct}

\IFRU{Как уже раннее указывалось в секции о FPU~(\ref{sec:FPU}), и \Tfloat и \Tdouble содержат в себе знак, 
мантиссу и экспоненту. 
Однако, можем ли мы работать с этими полями напрямую? Попробуем с \Tfloat.}
{As it was already noted in section about FPU~(\ref{sec:FPU}), both \Tfloat and \Tdouble types consisted of sign,
significand (or fraction) and exponent.
But will we able to work with these fields directly? Let's try with \Tfloat.}

\bigskip
% a hack used here! http://tex.stackexchange.com/questions/73524/bytefield-package
\begin{center}
\begin{bytefield}{32}
	\bitheader[endianness=big]{0,22,23,30,31} \\
	\bitbox{1}{S} & 
	\bitbox{8}{\IFRU{экспонента}{exponent}} & 
	\bitbox{23}{\IFRU{мантисса}{mantissa or fraction}}
\end{bytefield}
\end{center}

\begin{center}
( S\EMDASH{}\IFRU{знак}{sign} )
\end{center}

\lstinputlisting{patterns/15_structs/float_en.c}

\IFRU{Структура \TT{float\_as\_struct} занимает в памяти столько же места сколько и \Tfloat, 
то есть 4 байта или 32 бита.}
{\TT{float\_as\_struct} structure occupies as much space is memory as \Tfloat, e.g., 4 bytes or 32 bits.}

\IFRU{Далее мы выставляем во входящем значении отрицательный знак, 
а также прибавляя двойку к экспоненте, мы тем 
самым умножаем всё значение на \TT{$2^2$}, то есть на 4.}
{Now we setting negative sign in input value and also by adding 2 to exponent we thereby multiplicating
the whole number by \TT{$2^2$}, e.g., by 4.}

\IFRU{Компилируем в MSVC 2008 без оптимизации:}{Let's compile in MSVC 2008 without optimization:}

\lstinputlisting[caption=\NonOptimizing MSVC 2008]{patterns/15_structs/float_msvc_\LANG.asm}

\IFRU{Слегка избыточно. В версии скомпилированной с флагом \Ox нет вызовов \TT{memcpy()}, 
там работа происходит сразу с переменной f. Но по неоптимизированной версии будет проще понять.}
{Redundant for a bit.
If it is compiled with \Ox flag there is no \TT{memcpy()} call,
\TT{f} variable is used directly.
But it is easier to understand it all considering unoptimized version.}

\IFRU{А что сделает GCC 4.4.1 с опцией \Othree?}{What GCC 4.4.1 with \Othree will do?}

\lstinputlisting[caption=\Optimizing GCC 4.4.1]{patterns/15_structs/float_gcc_O3_\LANG.asm}

\IFRU{Да, функция \TT{f()} в целом понятна. Однако, что интересно, еще при компиляции, 
не взирая на мешанину с полями структуры, GCC умудрился вычислить результат функции \TT{f(1.234)} и 
сразу подставить его в аргумент для \printf{}!}
{The \TT{f()} function is almost understandable. However, what is interesting, GCC was able to calculate
\TT{f(1.234)} result during compilation stage despite all this hodge-podge with structure fields
and prepared this argument to the \printf{} as precalculated!}



\section{\IFRU{Структуры}{Structures}}

\IFRU{В принципе, структура в \CCpp это, с некоторыми допущениями, просто всегда лежащий рядом, 
и в той же последовательности, набор переменных, не обязательно одного типа.}
{It can be defined that \CCpp structure, with some assumptions, just a set of variables, always stored
in memory together, not necessary of the same type.}

\subsection{\IFRU{Пример SYSTEMTIME}{SYSTEMTIME example}}

\newcommand{\FNSYSTEMTIME}{\footnote{\href{http://msdn.microsoft.com/en-us/library/ms724950(VS.85).aspx}{MSDN: SYSTEMTIME structure}}}

\IFRU{Возьмем, к примеру, структуру SYSTEMTIME\FNSYSTEMTIME{} из win32 описывающую время.}
{Let's take SYSTEMTIME\FNSYSTEMTIME{} win32 structure describing time.}

\IFRU{Она объявлена так:}{That's how it's defined:}

\begin{lstlisting}[caption=WinBase.h]
typedef struct _SYSTEMTIME {
  WORD wYear;
  WORD wMonth;
  WORD wDayOfWeek;
  WORD wDay;
  WORD wHour;
  WORD wMinute;
  WORD wSecond;
  WORD wMilliseconds;
} SYSTEMTIME, *PSYSTEMTIME;
\end{lstlisting}

\IFRU{Пишем на Си функцию для получения текущего системного времени:}
{Let's write a C function to get current time:}

\lstinputlisting{15_structs/systemtime.c}

\IFRU{Что в итоге}{We got} (MSVC 2010):

\lstinputlisting[caption=MSVC 2010]{15_structs/systemtime.asm}

\IFRU{Под структуру в стеке выделено 16 байт ~--- именно столько будет \TT{sizeof(WORD)*8}
(в структуре 8 переменных с типом WORD).}
{16 bytes are allocated for this structure in local stack ~--- that's exactly \TT{sizeof(WORD)*8}
(there are 8 WORD variables in the structure).}

\newcommand{\FNMSDNGST}{\footnote{\href{http://msdn.microsoft.com/en-us/library/ms724390(VS.85).aspx}{MSDN: GetSystemTime function}}}

\IFRU{Обратите внимание на тот факт что структура начинается с поля \TT{wYear}. 
Можно сказать что в качестве аргумента для \TT{GetSystemTime()}\FNMSDNGST передается указатель на структуру 
SYSTEMTIME, но можно также сказать, что передается указатель на поле \TT{wYear}, 
что одно и тоже! 
\TT{GetSystemTime()} пишет текущий год в тот WORD на который указывает переданный указатель, 
затем сдвигается на 2 байта вправо, пишет текущий месяц, итд, итд.}
{Pay attention to the fact the structure beginning with \TT{wYear} field.
It can be said, an pointer to SYSTEMTIME structure is passed to \TT{GetSystemTime()}\FNSYSTEMTIME,
but it's also can be said, pointer to \TT{wYear} field is passed, and that's the same!
\TT{GetSystemTime()} writting current year to the WORD pointer pointing to, then shifting 2 bytes
ahead, then writting current month, etc, etc.}

Тот факт что поля структуры это просто переменные расположенные рядом, 
я могу проиллюстрировать следующим образом.
Глядя на описание структуры \TT{SYSTEMTIME}, мы можем переписать наш простой пример так:

\lstinputlisting{15_structs/systemtime2.c}

Компилятор немного поворчит:

\begin{lstlisting}
systemtime2.c(7) : warning C4133: 'function' : incompatible types - from 'WORD [8]' to 'LPSYSTEMTIME'
\end{lstlisting}

Тем не менее, выдаст такой код:

\lstinputlisting[caption=MSVC 2010]{15_structs/systemtime2.asm}

И это работает так же!

Любопытно что результат на ассемблере неотличим от предыдущего. Таким образом, глядя на этот код, 
никогда нельзя сказать с уверенностью, была ли там объявлена структура, либо просто набор переменных.

Тем не менее, никто в здравом уме делать так не будет. 
Потому что это неудобно. К тому же, иногда, поля в структуре могут меняться, переставляться местами, итд.




\subsection{\IFRU{Выделяем место для структуры через malloc()}{Let's allocate place for structure using malloc()}}

\IFRU{Однако, бывает и так, что проще хранить структуры не в стеке а в куче\footnote{heap}:}
{However, sometimes it's simpler to place structures not in local stack, but in heap:}

\lstinputlisting{15_structs/systemtime_malloc.c}

\IFRU{Скомпилируем на этот раз с оптимизацией (\Ox) чтобы было проще увидеть то, что нам нужно.}
{Let's compile it now with optimization (\Ox) so to easily see what we need.}

\lstinputlisting[caption=\Optimizing MSVC]{15_structs/systemtime_malloc.asm}

\index{\CLanguageElements!malloc()}
\IFRU{Итак, \TT{sizeof(SYSTEMTIME) = 16}, именно столько байт выделяется при помощи \TT{malloc()}. 
Она возвращает указатель на только что выделенный блок памяти в \EAX, который копируется в \ESI. 
Win32 функция \TT{GetSystemTime()} обязуется сохранить состояние \ESI, 
поэтому здесь оно нигде не сохраняется и продолжает использоваться после вызова \TT{GetSystemTime()}.}
{So, \TT{sizeof(SYSTEMTIME) = 16}, that's exact number of bytes to be allocated by \TT{malloc()}.
It return the pointer to freshly allocated memory block in \EAX, which is then moved into \ESI.
\TT{GetSystemTime()} win32 function undertake to save \ESI value, 
and that's why it is not saved here and continue to be used after \TT{GetSystemTime()} call.}

\index{x86!\Instructions!MOVZX}
\IFRU{
Новая инструкция ~--- \MOVZX (\IT{Move with Zero eXtent}). 
Она нужна почти там же где и \MOVSX, 
только всегда очищает остальные биты в $0$. Дело в том что \printf требует 32-битный тип \Tint, 
а в структуре лежит WORD ~--- это 16-битный беззнаковый тип. Поэтому копируя значение из WORD в \Tint, 
нужно очистить биты от 16 до 31, иначе там будет просто случайный мусор, оставшийся от предыдущих действий 
с регистрами.}
{New instruction ~--- \MOVZX (\IT{Move with Zero eXtent}).
It may be used almost in those cases as \MOVSX, but, it clearing other bits to $0$.
That's because \printf require 32-bit \Tint, but we got WORD in structure ~--- that's 16-bit unsigned type.
That's why by copying value from WORD into \Tint{}, bits from 16 to 31 should be cleared, 
because there will be random noise otherwise, leaved from previous operations on registers.}

\IFRU{В этом примере я тоже могу представить структуру как массив WORD-ов}{In this example, I can represent
structure as array of WORD-s}:

\lstinputlisting{15_structs/systemtime_malloc2.c}

\IFRU{Получим такое}{We got}:

\lstinputlisting[caption=\Optimizing MSVC]{15_structs/systemtime_malloc2.asm}

\IFRU{И снова мы получаем идетичный код, неотличимый от предыдущего}{Again, we got a code that cannot be distinguished
from previous}.
\IFRU{Но и снова я должен отметить, что в реальности так лучше не делать}{And again I should note, one shouldn't do
this in practice}.



\subsection{Linux}

\IFRU{В Линуксе, для примера, возьем структуру \TT{tm} из \TT{time.h}:}
{As of Linux, let's take \TT{tm} structure from \TT{time.h} for example:}

\lstinputlisting{15_structs/GCC_tm.c}

\IFRU{Компилируем при помощи}{Let's compile it in} GCC 4.4.1:

\IFRU{\lstinputlisting[caption=GCC 4.4.1]{15_structs/GCC_tm_ru.asm}}{\lstinputlisting{15_structs/GCC_tm_en.asm}}

\IFRU{К сожалению, по какой-то причине, \IDA не сформировала названия локальных переменных в стеке. 
Но так как мы уже опытные реверсеры :-) то можем обойтись и без этого в таком простом примере.}
{Somehow, \IDA didn't created local variables names in local stack.
But since we already experienced reverse engineers :-) we may do it without this information in 
this simple example.}

\IFRU{Обратите внимание на \TT{lea edx, [eax+76Ch]} ~--- эта инструкция прибавляет $0x76C$ к \EAX, 
но не модифицирует флаги. См. также соответствующий раздел об инструкции \LEA{}~\ref{sec:LEA}.}
{Please also pay attention to \TT{lea edx, [eax+76Ch]} ~--- this instruction just adding $0x76C$ to \EAX,
but not modify any flags. See also relevant section about \LEA{}~\ref{sec:LEA}.}

Чтобы проиллюстрировать то что структура это просто набор переменных лежащих в одном месте, переделаем немного
пример, заглянув предварительно в файл time.h:

\begin{lstlisting}[caption=time.h]
struct tm
{
  int	tm_sec;
  int	tm_min;
  int	tm_hour;
  int	tm_mday;
  int	tm_mon;
  int	tm_year;
  int	tm_wday;
  int	tm_yday;
  int	tm_isdst;
};
\end{lstlisting}

\lstinputlisting{15_structs/GCC_tm2.c}

Обратите внимание на то что в \TT{localtime\_r} передается указатель именно на \TT{tm\_sec}, 
т.е., на первый элемент ``структуры''.

В итоге:

\lstinputlisting[caption=GCC 4.7.3]{15_structs/GCC_tm2.asm}

Этот код почти идентичен уже рассмотренному, и нельзя сказать, была ли структура
в оригинальном исходном коде либо набор переменных.

И это работает. Однако, в реальности так лучше не делать. Обычно, компилятор располагает переменные в локальном
стеке в том же порядке, в котором они объявляются в функции. Тем не менее, никакой гарантии нет.

Я выбрал именно этот пример для иллюстрации, потому что члены структуры имеют тип \Tint, а члены структуры
\TT{SYSTEMTIME} ~--- 16-битные \TT{WORD}, и если их объявлять так же, то они будут выровнены по 32-битной границе 
и ничего не выйдет (потому что \TT{GetSystemTime()} заполнит их неверно). Читайте об этом в следующей секции
``\StructurePackingSectionName''.

Так что, структура это просто набор переменных лежащих в одном месте, рядом. Я мог бы сказать что структура
это такой синтаксический сахар, заставляющий компилятор удерживать их в одном месте. Впрочем, я не специалист
по языкам программирования, так что, скорее всего, ошибаюсь с этим термином.



\subsection{\StructurePackingSectionName}

\IFRU{Достаточно немаловажный момент, это упаковка полей в структурах\footnote{См.также: \URLWPDA}.}
{One important thing is fields packing in structures\footnote{See also: \URLWPDA}.}

\IFRU{Возьмем простой пример:}{Let's take a simple example:}

\lstinputlisting{15_structs/15_5.c}

\IFRU{Как видно, мы имеем два поля \Tchar (занимающий один байт) и еще два ~--- \Tint (по 4 байта).}
{As we see, we have two \Tchar fields (each is exactly one byte) and two more ~--- \Tint (each - 4 bytes).}

\IFRU{Компилируется это все в:}{That's all compiling into:}

\lstinputlisting{15_structs/15_5.asm}

\IFRU{Мы видим здесь что адрес каждого поля в структуре выравнивается по 4-байтной границе. 
Так что каждый \Tchar здесь занимает те же 4 байта что и \Tint. Зачем? 
Затем что процессору удобнее обращаться по таким адресам и кешировать данные из памяти.}
{As we can see, each field's address is aligned by 4-bytes border.
That's why each \Tchar using 4 bytes here, like \Tint. Why?
Thus it's easier for CPU to access memory at aligned addresses and to cache data from it.}

\IFRU{Но это не экономично по размеру данных.}{However, it's not very economical in size sense.}

\IFRU{Попробуем скомпилировать тот же исходник с опцией}{Let's try to compile it with option} (\TT{/Zp1}) 
(\IT{/Zp[n] pack structs on n-byte boundary}).

\lstinputlisting[caption=MSVC /Zp1]{15_structs/15_5_msvc_Zp1.asm}

\IFRU{Теперь структура занимает 10 байт и все \Tchar занимают по байту. Что это дает? 
Экономию места. Недостаток ~--- процессор будет обращаться к этим полям не так эффективно 
по скорости как мог бы.}
{Now the structure takes only 10 bytes and each \Tchar value takes 1 byte. What it give to us?
Size economy. And as drawback ~--- CPU will access these fields without maximal performance it can.}

\IFRU{Как нетрудно догадаться, если структура используется много в каких исходниках и объектных файлах, 
все они должны быть откомпилированы с одним и тем же соглашением об упаковке структур.}
{As it can be easily guessed, if the structure is used in many source and object files,
all these should be compiled with the same convention about structures packing.}

\newcommand{\FNURLMSDNZP}{\footnote{\href{http://msdn.microsoft.com/en-us/library/ms253935.aspx}
{MSDN: Working with Packing Structures}}}
\newcommand{\FNURLGCCPC}{\footnote{\href{http://gcc.gnu.org/onlinedocs/gcc/Structure_002dPacking-Pragmas.html}
{Structure-Packing Pragmas}}}

\IFRU{Помимо ключа MSVC \TT{/Zp}, указывающего, по какой границе упаковывать поля структур, есть также 
опция компилятора \TT{\#pragma pack}, её можно указывать прямо в исходнике. 
Это справедливо и для MSVC\FNURLMSDNZP и GCC\FNURLGCCPC{}.}
{Aside from MSVC \TT{/Zp} option which set how to align each structure field, here is also
\TT{\#pragma pack} compiler option, it can be defined right in source code.
It's available in both MSVC\FNURLMSDNZP and GCC\FNURLGCCPC{}.}

\IFRU{Давайте теперь вернемся к \TT{SYSTEMTIME}, которая состоит из 16-битных полей. 
Откуда наш компилятор знает что их надо паковать по однобайтной границе?}
{Let's back to \TT{SYSTEMTIME} structure consisting in 16-bit fields.
How our compiler know to pack them on 1-byte alignment method?}

\IFRU{В файле \TT{WinNT.h} попадается такое:}{\TT{WinNT.h} file has this:}

\begin{lstlisting}[caption=WinNT.h]
#include "pshpack1.h"
\end{lstlisting}

\IFRU{И такое:}{And this:}

\begin{lstlisting}[caption=WinNT.h]
#include "pshpack4.h"                   // 4 byte packing is the default
\end{lstlisting}

\IFRU{Сам файл PshPack1.h выглядит так:}{The file PshPack1.h looks like:}

\begin{lstlisting}[caption=PshPack1.h]
#if ! (defined(lint) || defined(RC_INVOKED))
#if ( _MSC_VER >= 800 && !defined(_M_I86)) || defined(_PUSHPOP_SUPPORTED)
#pragma warning(disable:4103)
#if !(defined( MIDL_PASS )) || defined( __midl )
#pragma pack(push,1)
#else
#pragma pack(1)
#endif
#else
#pragma pack(1)
#endif
#endif /* ! (defined(lint) || defined(RC_INVOKED)) */
\end{lstlisting}

\IFRU{Собственно, так и задается компилятору, как паковать объявленные после \TT{\#pragma pack} структуры.}
{That's how compiler will pack structures defined after \TT{\#pragma pack}.}

\subsection{\IFRU{Вложенные структуры}{Nested structures}}

\IFRU{Теперь, как насчет ситуаций, когда одна структура определяет внутри себя еще одну структуру?}
{Now what about situations when one structure define another structure inside?}

\lstinputlisting{15_structs/15_6.c}

\dots \IFRU{в этом случае, оба поля \TT{inner\_struct} просто будут располагаться между полями a,b и d,e в 
\TT{outer\_struct}.}
{in this case, both \TT{inner\_struct} fields will be placed between a,b and d,e fields of
\TT{outer\_struct}.}

\IFRU{Компилируем}{Let's compile} (MSVC 2010):

\lstinputlisting[caption=MSVC 2010]{15_structs/15_6_msvc.asm}

\IFRU{Очень любопытный момент в том, что глядя на этот код на ассемблере, мы даже не видим, 
что была использована какая-то еще другая структура внутри этой!
Так что, пожалуй, можно сказать, что все вложенные структуры в итоге разворачиваются в одну, \IT{линейную} 
или \IT{одномерную} структуру.}
{One curious point here is that by looking onto this assembly code, we do not even see that
another structure was used inside of it!
Thus, we would say, nested structures are finally unfolds into \IT{linear} or \IT{one-dimensional} structure.}

\IFRU{Конечно, если заменить объявление \TT{struct inner\_struct c;} на \TT{struct inner\_struct *c;} 
(объявляя таким образом указатель), ситауция будет совсем иная.}
{Of course, if to replace \TT{struct inner\_struct c;} declaration to \TT{struct inner\_struct *c;} 
(thus making a pointer here) situation will be significally different.}



\subsection{\IFRU{Работа с битовыми полями в структуре}{Bit fields in structure}}

\subsubsection{\IFRU{Пример CPUID}{CPUID example}}

\IFRU{Язык \CCpp позволяет укзывать, сколько именно бит отвести для каждого поля структуры. 
Это удобно если нужно экономить место в памяти. К примеру, для переменной типа \Tbool достаточно одного бита.
Но, это не очень удобно, если нужна скорость.}
{\CCpp language allow to define exact number of bits for each structure fields.
It's very useful if one needs to save memory space. 
For example, one bit is enough for variable of \Tbool type.
But of course, it's not rational if speed is important.}

\newcommand{\FNCPUID}{\footnote{\url{http://en.wikipedia.org/wiki/CPUID}}}

\index{x86!\Instructions!CPUID}
\IFRU{Рассмотрим пример с инструкцией \CPUID\FNCPUID. 
Эта инструкция возвращает информацию о том, какой процессор имеется в наличии и какие фичи он имеет.}
{Let's consider \CPUID\FNCPUID instruction example.
This instruction returning information about current CPU and its features.}

\IFRU{Если перед исполнением инструкции в \EAX будет 1, 
то \CPUID вернет упакованную в \EAX такую информацию о процессоре:}
{If the \EAX is set to 1 before instruction execution, 
\CPUID will return this information packed into the \EAX register:}

\begin{center}
\begin{tabular}{ | l | l | }
\hline
3:0 & Stepping \\
7:4 & Model \\
11:8 & Family \\
13:12 & Processor Type \\
19:16 & Extended Model \\
27:20 & Extended Family \\
\hline
\end{tabular}
\end{center}

\newcommand{\FNGCCAS}{\footnote{\href{http://www.ibiblio.org/gferg/ldp/GCC-Inline-Assembly-HOWTO.html}
{\IFRU{Подробнее о встроенном ассемблере GCC}{More about internal GCC assembler}}}}

\IFRU{MSVC 2010 имеет макрос для \CPUID, а GCC 4.4.1 ~--- нет. 
Поэтому для GCC сделаем эту функцию сами, используя его встроенный ассемблер\FNGCCAS.}
{MSVC 2010 has \CPUID macro, but GCC 4.4.1 ~--- hasn't.
So let's make this function by yourself for GCC with the help of its built-in assembler\FNGCCAS.}

\lstinputlisting{15_structs/CPUID.c}

\IFRU{После того как \CPUID заполнит \EAX/\EBX/\ECX/\EDX, у нас они отразятся в массиве \TT{b[]}. 
Затем, мы имеем указатель на структуру \TT{CPUID\_1\_EAX}, и мы указываем его на значение 
\EAX из массива \TT{b[]}.}
{After \CPUID will fill \EAX/\EBX/\ECX/\EDX, these registers will be reflected in the \TT{b[]} array.
Then, we have a pointer to the \TT{CPUID\_1\_EAX} structure and we point it to the value in the \EAX from \TT{b[]} array.}

\IFRU{Иными словами, мы трактуем 32-битный \Tint как структуру.}
{In other words, we treat 32-bit \Tint value as a structure.}

\IFRU{Затем мы читаем из структуры.}{Then we read from the stucture.}

\IFRU{Компилируем в MSVC 2008 с опцией \Ox}{Let's compile it in MSVC 2008 with \Ox option}:

\lstinputlisting[caption=\Optimizing MSVC 2008]{15_structs/CPUID_msvc_Ox.asm}

\index{x86!\Instructions!SHR}
\IFRU{Инструкция \TT{SHR} сдвигает значение из \EAX на то количество бит, 
которое нужно \IT{пропустить}, то есть, мы игнорируем некоторые биты \IT{справа}.}
{\TT{SHR} instruction shifting value in the \EAX register by number of bits should be
\IT{skipped}, e.g., we ignore some bits \IT{at right}.}

\index{x86!\Instructions!AND}
\IFRU{А инструкция \AND очищает биты \IT{слева} которые нам не нужны, или же, говоря иначе, 
она оставляет по маске только те биты в \EAX, которые нам сейчас нужны.}
{\AND instruction clears bits not needed \IT{at left}, or, in other words, 
leaves only those bits in the \EAX register we need now.}

\IFRU{Попробуем GCC 4.4.1 с опцией \Othree.}{Let's try GCC 4.4.1 with \Othree option.}

\lstinputlisting[caption=\Optimizing GCC 4.4.1]{15_structs/CPUID_gcc_O3.asm}

\IFRU{Практически, то же самое. Единственное что стоит отметить это то, что GCC решил зачем-то объеденить 
вычисление \TT{extended\_model\_id} и \TT{extended\_family\_id} в один блок, 
вместо того чтобы вычислять их перед соответствующим вызовом \printf.}
{Almost the same. The only thing worth noting is that GCC somehow united calculation of
\TT{extended\_model\_id} and \TT{extended\_family\_id} into one block,
instead of calculating them separately, before corresponding each \printf call.}

\subsubsection{\WorkingWithFloatAsWithStructSubSubSectionName}
\label{sec:floatasstruct}

\IFRU{Как уже раннее указывалось в секции о FPU~\ref{sec:FPU}, и \Tfloat и \Tdouble содержат в себе знак, 
мантиссу и экспоненту. 
Однако, можем ли мы работать с этими полями напрямую? Попробуем с \Tfloat.}
{As it was already noted in section about FPU~\ref{sec:FPU}, both \Tfloat and \Tdouble types consisted of sign,
significand (or fraction) and exponent.
But will we able to work with these fields directly? Let's try with \Tfloat.}

\index{IEEE 754}
\index{float}
\begin{figure}[ht!]
\centering
\includegraphics[scale=0.66]{15_structs/500px-Float_example.png}
\caption{\IFRU{Формат значения float\protect\footnotemark}
{float value format\protect\footnotemark}}
\end{figure}

\footnotetext{\IFRU{иллюстрация взята из}{illustration taken from} wikipedia}

\lstinputlisting{15_structs/float_en.c}

\IFRU{Структура \TT{float\_as\_struct} занимает в памяти столько же места сколько и \Tfloat, 
то есть 4 байта или 32 бита.}
{\TT{float\_as\_struct} structure occupies as much space is memory as \Tfloat, e.g., 4 bytes or 32 bits.}

\IFRU{Далее мы выставляем во входящем значении отрицательный знак, 
а также прибавляя двойку к экспоненте, мы тем 
самым умножаем всё значение на \TT{$2^2$}, то есть на 4.}
{Now we setting negative sign in input value and also by addding 2 to exponent we thereby multiplicating
the whole number by \TT{$2^2$}, e.g., by 4.}

\IFRU{Компилируем в MSVC 2008 без оптимизации:}{Let's compile in MSVC 2008 without optimization:}

\lstinputlisting[caption=\NonOptimizing MSVC 2008]
{\IFRU{15_structs/float_msvc_ru.asm}{15_structs/float_msvc_en.asm}}

\IFRU{Слекга избыточно. В версии скомпилированной с флагом \Ox нет вызовов \TT{memcpy()}, 
там работа происходит сразу с переменной f. Но по неоптимизированной версии будет проще понять.}
{Redundant for a bit. If it compiled with \Ox flag there are no \TT{memcpy()} call,
f variable is used directly. But it's easier to understand it all considering unoptimized version.}

\IFRU{А что сделает GCC 4.4.1 с опцией \TT{-O3}?}{What GCC 4.4.1 with \TT{-O3} will do?}

\lstinputlisting[caption=\Optimizing GCC 4.4.1]
{\IFRU{15_structs/float_gcc_O3_ru.asm}{15_structs/float_gcc_O3_en.asm}}

\IFRU{Да, функция \TT{f()} в целом понятна. Однако, что интересно, еще при компиляции, 
не взирая на мешанину с полями структуры, GCC умудрился вычислить результат функции \TT{f(1.234)} и 
сразу подставить его в аргумент для \printf{}!}
{The \TT{f()} function is almost understandable. However, what is interesting, GCC was able to calculate
\TT{f(1.234)} result during compilation stage despite all this hodge-podge with structure fields
and prepared this argument to the \printf{} as precalculated!}





\input{16_classes/classes}
\subsection{\IFRU{Наследование классов в C++}{Class inheritance in C++}}

\IFRU{О наследованных классах можно сказать что это та же простая структура которую мы уже рассмотрели, 
только расширяемая в наследуемых классах.}
{It can be said about inherited classes that it's simple structure we already considered, but extending 
in inherited classes.}

\IFRU{Возьмем очень простой пример}{Let's take simple example}:

\lstinputlisting{16_classes/classes1_inheritance.cpp}

\IFRU{Исследуя сгенерированный код для функций/методов \TT{dump()}, а также \TT{object::print\_color()},
посмотрим какая будет разметка памяти для структур-объектов (для 32-битного кода).}
{Let's investigate generated code of \TT{dump()} functions/methods and also \TT{object::print\_color()},
let's see memory layout for structures-objects (as of 32-bit code).}

\IFRU{Итак, методы \TT{dump()} разных классов сгенерированные MSVC 2008 с опциями \Ox и \Obzero}
{So, \TT{dump()} methods for several classes, generated by MSVC 2008 with \Ox and \Obzero options}
\footnote{\IFRU{опция \Obzero означает отмену inline expansion, 
ведь вставка компилятором тела функции/метода прямо в код где он вызывается только затруднит наши эксперименты}{
\Obzero options mean inline expansion disabling, because, function inlining right into the code where the function
is called will make our experiment harder}}

\lstinputlisting[caption=\Optimizing MSVC 2008 /Ob0]{16_classes/classes1_1.asm}

\lstinputlisting[caption=\Optimizing MSVC 2008 /Ob0]{16_classes/classes1_2.asm}

\lstinputlisting[caption=\Optimizing MSVC 2008 /Ob0]{16_classes/classes1_3.asm}

\IFRU{Итак, разметка полей получается следующая}{So, here is memory layout}:

\IFRU{(базовый класс \IT{object})}{(base class \IT{object})}

\begin{center}
\begin{tabular}{ | l | l | }
\hline
  \tableheader{} \\
  +0x0 & int color \\
\hline
\end{tabular}
\end{center}

\IFRU{(унаследованные классы)}{(inherited classes)}

\IT{box}:

\begin{center}
\begin{tabular}{ | l | l | }
\hline
  \tableheader{} \\
  +0x0 & int color \\
  +0x4 & int width \\
  +0x8 & int height \\
  +0xC & int depth \\
\hline
\end{tabular}
\end{center}

\IT{sphere}:

\begin{center}
\begin{tabular}{ | l | l | }
\hline
  \tableheader{} \\
  +0x0 & int color \\
  +0x4 & int radius \\
\hline
\end{tabular}
\end{center}

\IFRU{Посмотрим тело \main}{Let's see \main function body}:

\lstinputlisting[caption=\Optimizing MSVC 2008 /Ob0]{16_classes/classes1_4.asm}

\IFRU{Наследованные классы всегда должны добавлять свои поля после полей базового класса для того, чтобы методы
базового класса могли продолжать работать со своими полями.}
{Inherited classes should always add their fields after base classes' fields, so to make possible for base 
class methods to work with their fields.}

\IFRU{Когда метод \TT{object::print\_color()} вызывается, ему в качестве \TT{this} передается указатель и на объект типа \IT{box} 
и на объект типа \IT{sphere}, так как он может легко работать с классами \IT{box} и \IT{sphere}, потому что поле \IT{color} в этих
классах всегда стоит по тому же адресу (по смещению \IT{0x0}).}
{When \TT{object::print\_color()} method is called, a pointers to both \IT{box} object and \IT{sphere} object are passed as \TT{this},
it can work with these objects easily because \IT{color} field in these objects is always at the pinned address (at \IT{+0x0} offset).}

\IFRU{Можно также сказать что методу \TT{object::print\_color()} даже не нужно знать,
с каким классом он работает, до тех пор пока будет соблюдаться условие /IT{закрепления} полей по тем же адресам,
а это условие соблюдается всегда.}
{It can be said, \TT{object::print\_color()} method is agnostic in relation to input object type as long as fields will be \IT{pinned}
at the same addresses, and this condition is always true.}

\IFRU{А если вы создадите класс-наследник класса \IT{box}, например, 
то компилятор будет добавлять новые поля уже за полем \IT{depth}, оставляя уже имеющиеся поля класса \IT{box} по тем же адресам.}
{And if you create inherited class of \IT{box} class, for example, compiler will add new fields after \IT{depth} field,
leaving \IT{box} class fields at the pinned addresses.}

\IFRU{Так, метод \TT{box::dump()} будет нормально работать обращаясь к полям \IT{color}/\IT{width}/\IT{height}/\IT{depth} всегда находящимся по известным адресам.}
{Thus, \TT{box::dump()} method will work fine accessing \IT{color}/\IT{width}/\IT{height}/\IT{depths} fields always pinned on known addresses.}

\IFRU{Код на GCC практически точно такой же, за исключением способа передачи \TT{this} (он, как уже было указано, 
передается в первом аргументе, вместо регистра \ECX).}
{GCC-generated code is almost the same, with the sole exception of \TT{this} pointer passing (as it was described above,
it passing as first argument instead of \ECX registers.}


\input{16_classes/classes_2_encapsulation}
\input{16_classes/classes_3_mutiple}
\input{16_classes/classes_4_virtual}
\section{\IFRU{Объединения (union)}{Unions}}

\subsection{\IFRU{Пример генератора случайных чисел}{Pseudo-random number generator example}}

\IFRU{Если нам нужны случайные значения с плавающей запятой в интервале от 0 до 1, самое простое это взять
\ac{PRNG} вроде Mersenne twister выдающий случайные 32-битные числа в виде DWORD, преобразовать
это число в \Tfloat и затем разделить на \TT{RAND\_MAX} (\TT{0xFFFFFFFF} в данном случае) ~--- 
полученное число будет в интервале от 0 до 1.}
{If we need float random numbers from 0 to 1, the most simplest thing is to use \ac{PRNG} like
Mersenne twister produces random 32-bit values in DWORD form, transform this value to \Tfloat and then
dividing it by \TT{RAND\_MAX} (\TT{0xFFFFFFFF} in our case)~---value we got will be in 0..1 interval.}

\IFRU{Но как известно, операция деления ~--- это медленная операция. 
Сможем ли мы избежать её, как в случае с делением через умножение?}
{But as we know, division operation is slow.
Will it be possible to get rid of it, as in case of division by multiplication?}
~(\ref{sec:divisionbynine})

\index{IEEE 754}
\IFRU{Вспомним состав числа с плавающей запятой: это бит знака, биты мантиссы и биты экспоненты. 
Для получения случайного числа, нам нужно просто заполнить случайными битами все биты мантиссы!}
{Let's recall what float number consisted of: sign bit, significand bits and exponent bits.
We need just to store random bits to all significand bits for getting random float number!}

\IFRU{Экспонента не может быть нулевой (иначе число будет денормализованным), 
так что в эти биты мы запишем \TT{01111111} ~--- 
это будет означать что экспонента равна единице. Далее заполняем мантиссу случайными битами, 
знак оставляем в виде 0 (что значит наше число положительное), и вуаля. 
Генерируемые числа будут в интервале от 1 до 2, так что нам еще нужно будет отнять единицу.}
{Exponent cannot be zero (number will be denormalized in this case), so we will store \TT{01111111} 
to exponent~---this means exponent will be 1. Then fill significand with random bits, set sign bit to
0 (which means positive number) and voilà.
Generated numbers will be in 1 to 2 interval, so we also must subtract 1 from it.}

\newcommand{\URLXOR}{\url{http://xor0110.wordpress.com/2010/09/24/how-to-generate-floating-point-random-numbers-efficiently}}

\IFRU{В моем примере\footnote{идея взята здесь: \URLXOR} 
применяется очень простой линейный конгруэнтный генератор случайных чисел, выдающий 32-битные числа.
Генератор инициализируется текущим временем в стиле UNIX.}
{Very simple linear congruential random numbers generator is used in my 
example\footnote{idea was taken from: \URLXOR}, produces 32-bit numbers. 
The PRNG initializing by current time in UNIX-style.}

\IFRU{Далее, тип \Tfloat представляется в виде \IT{union} ~--- это конструкция \CCpp позволяющая 
интерпретировать часть памти по-разному. В нашем случае, мы можем создать переменную типа \TT{union} 
и затем обращаться к ней как к \Tfloat или как к \IT{uint32\_t}. Можно сказать, что это хак, причем грязный.}
{Then, \Tfloat type represented as \IT{union}~---it is the \CCpp construction enabling us
to interpret piece of memory as differently typed.
In our case, we are able to create a variable
of \TT{union} type and then access to it as it is \Tfloat or as it is \IT{uint32\_t}. 
It can be said, it is just a hack. A dirty one.}

\lstinputlisting{patterns/17_unions/FPU_PRNG.cpp}

\lstinputlisting[caption=MSVC 2010 (\Ox)]{patterns/17_unions/FPU_PRNG_msvc_2010_Ox_\LANG.asm}

\IFRU{А результат GCC будет почти таким же.}{GCC produces very similar code.}



\input{18_pointers_to_functions/pointers_to_functions}
\input{19_SIMD/SIMD}
\subsection{x64}

\index{x86-64}
\RU{Всё то же самое, только используются регистры вместо стека для передачи аргументов функций}%
\EN{The picture here is similar with the difference that the registers, rather than the stack, are used for arguments passing}.

\subsubsection{MSVC}

\lstinputlisting[caption=MSVC 2012 x64]{patterns/04_scanf/1_simple/ex1_MSVC_x64.asm.\LANG}

\ifdefined\IncludeGCC
\subsubsection{GCC}

\lstinputlisting[caption=\Optimizing GCC 4.4.6 x64]{patterns/04_scanf/1_simple/ex1_GCC_x64.s.\LANG}
\fi

\section{C99 restrict}

А вот причина из-за которой программы на FORTRAN, в некоторых случаях, работают быстрее чем на Си.

\begin{lstlisting}
void f1 (int* x, int* y, int* sum, int* product, int* sum_product, int* update_me, size_t s)
{
	for (int i=0; i<s; i++)
	{
		sum[i]=x[i]+y[i];
		product[i]=x[i]*y[i];
		update_me[i]=i*123; // some dummy value
		sum_product[i]=sum[i]+product[i];	
	};
};
\end{lstlisting}

Это очень простой пример, в котором есть одна особенность: указатель на массив \TT{update\_me} может быть указателем 
на массив
\TT{sum}, \TT{product}, или даже \TT{sum\_product} ~--- 
ведь нет ничего криминального в том чтобы аргументам функции быть такими.

Компилятор знает об этом, поэтому генерирует код, где в теле цикла будет 4 основных стадии:
\begin{itemize}
\item вычислить следующий \TT{sum[i]}
\item вычислить следующий \TT{product[i]}
\item вычислить следующий \TT{update\_me[i]}
\item вычислить следующий \TT{sum\_product[i]} ~--- на этой стадии придется снова загружать из 
      памяти подсчитанные \TT{sum[i]} и \TT{product[i]}
\end{itemize}

Возможно ли соптимизировать последнюю стадию? Ведь подсчитанные \TT{sum[i]} и \TT{product[i]} не обязательно 
снова загружать из памяти,
ведь мы их только что подсчитали. Можно, но компилятор не уверен, что на третьей стадии ничего не затерлось! Это называется
``pointer aliasing'', ситуация, когда компилятор не может быть уверен что память на которую указывает какой-то указатель, 
не изменилась.

\IT{restrict} в стандарте Си C99\cite[6.7.3.1]{C99TC3} это обещание, даваемое компилятору программистом, 
что аргументы функции отмеченные этим 
ключевым словом,
всегда будут указывать на разные места в памяти и пересекаться не будут.

Если быть более точным, и описывать это формально, \IT{restrict} показывает, что только данный указатель будет
использоваться для доступа к этому объекту, с которым мы работаем через этот указатель, больше никакой указатель для
этого использоваться не будет. Можно даже сказать, что к всякому объекту, доступ будет осуществляться только через
один единственный указатель, если он отмечен как \IT{restrict}.

Добавим это ключевое слово к каждому аргументу-указателю:

\begin{lstlisting}
void f2 (int* restrict x, int* restrict y, int* restrict sum, int* restrict product, int* restrict sum_product, 
	int* restrict update_me, size_t s)
{
	for (int i=0; i<s; i++)
	{
		sum[i]=x[i]+y[i];
		product[i]=x[i]*y[i];
		update_me[i]=i*123; // some dummy value
		sum_product[i]=sum[i]+product[i];	
	};
};
\end{lstlisting}

Посмотрим результат:

\lstinputlisting[caption=GCC x64: f1()]{21_C99_restrict/f1.asm}

\lstinputlisting[caption=GCC x64: f2()]{21_C99_restrict/f2.asm}

Разница между скомпилированной функцией \TT{f1()} и \TT{f2()} такая: 
в \TT{f1()} \TT{sum[i]} и \TT{product[i]} загружаются снова посреди тела цикла,
а в \TT{f2()} этого нет, используются уже подсчитанные значения, ведь мы ``пообещали'' компилятору, 
что никто и ничто не изменит
значения в \TT{sum[i]} и \TT{product[i]} во время исполнения тела цикла, поэтому он ``уверен'', что так можно делать. 
Очевидно, второй вариант будет работать быстрее.

Но что будет если указатели в аргументах функций все же будут пересекаться? Это останется на совести программиста, 
а результаты вычислений будут неверными.

Вернемся к FORTRAN. Компиляторы с этого ЯП, по умолчанию, все указатели считают таковыми, поэтому, когда в Си не было
возможности указать \IT{restrict}, FORTRAN в этих случаях мог генерировать более быстрый код.

Насколько это практично? Там где функция работает с несколькими большими блоками в памяти. 
Такого очень много в линейной алгебре, например. Очень много линейной алгебры используется на суперкомпьютерах/HPC,
возможно, поэтому, традиционно, там часто используется FORTRAN, до сих пор\cite{Loh:2010:IHP:1810226.1820518}.

Ну а когда итераций цикла не очень много, конечно, тогда прирост скорости не будет ощутимым.


\section{\IFRU{Inline-функции}{Inline functions}}
\index{Inline code}

\IFRU{Inline-код это когда компилятор, вместо того чтобы генерировать инструкцию вызова небольшой функции,
просто вставляет её тело прямо в это место.}
{Inlined code is when compiler, instead of placing call instruction to small or tiny function,
just placing its body right in-place.}

\lstinputlisting[caption=\IFRU{Простой пример}{Simple example}]{22_inline_function/1.c}

\IFRU{... это компилируется вполне предсказуемо, хотя, если включить оптимизации GCC (-O3), мы увидим:}
{... is compiled in very predictable way, however, if to turn on GCC optimization (-O3), we'll see:}

\lstinputlisting[caption=GCC 4.8.1 -O3]{22_inline_function/1.s}

(\IFRU{Здесь деление заменено умножением}{Here division is done by multiplication}\ref{sec:divisionbynine}.)

\IFRU{Да, наша маленькая ф-ция была помещена прямо перед вызовом \printf.}
{Yes, our small function was just placed befor \printf call.}
\IFRU{Почему? Это может быть быстрее чем исполнять код самой ф-ции плюс затраты на вызов и возврат.}
{Why? It may be faster than executing this function's code plus calling/returning overhead.}

\IFRU{В прошлом, такие ф-ции нужно было маркировать ключевым словом ``inline'' в определении ф-ции, хотя,
в наше время, такие ф-ции выбираются компилятором автоматически.}
{In past, such function should be marked with ``inline'' keyword in function's declaration, however,
in modern times, these functions are chosen automatically by compiler.}

\IFRU{Другая очень частая оптимизация это вставка кода строковых ф-ций таких как}
{Another very common automatic optimization is inlining of string functions like}
\IT{strcpy()}, \IT{strcmp()}, \IFRU{итд}{etc}.

\lstinputlisting[caption=\IFRU{Еще один простой пример}{Another simple example}]{22_inline_function/2.c}

\lstinputlisting[caption=GCC 4.8.1 -O3]{22_inline_function/2.s}

\IFRU{Вот пример очень часто попадающегося фрагмента кода strcmp() генерируемого MSVC:}
{Here is an example of very frequently seen piece of strcmp() code generated by MSVC:}

\lstinputlisting[caption=MSVC]{22_inline_function/strcmp.lst}

\IFRU{Я написал небольшой скрипт для \IDA для поиска и сворачивания таких очень часто 
попадающихся inline-функций:}
{I wrote small \IDA script for searching and folding such very frequently seen pieces of inline code:}
\url{https://github.com/yurichev/IDA_scripts}.

