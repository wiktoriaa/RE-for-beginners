\chapter{\IFRU{Паттерны компиляторов}{Compiler's patterns}}

\IFRU
{Когда я учил Си, а затем Си++, я просто писал небольшие фрагменты кода, компилировал и смотрел что 
получилось на ассемблере. Так намного проще было понять. Я делал это такое количество раз, 
что связь между кодом на \CCpp и тем что генерирует компилятор вбилась мне в подсознание достаточно 
глубоко, поэтому я могу глядя на код на ассемблере сразу понимать, в общих чертах, что там было написано 
на Си. Возможно это поможет кому-то еще, попробую описать некоторые примеры.}
{When I first learned C and then C++ I was just writing small pieces of code, compiling it, 
and seeing what 
was produced in assembly language. That was easy for me. I did it many times and the relation 
between \CCpp code and what the compiler produced was imprinted in my mind so deep that 
I can quickly understand what was in C code when I look at produced x86 code. 
Perhaps this technique may be helpful for someone else so I will try to describe some examples here.}

% sections here:

\section{\HelloWorldSectionName}
\label{sec:helloworld}

\IFRU{Начнем с знаменитого примера из книги}
{Let's start with the famous example from the book}
``The C programming Language''\cite{Kernighan:1988:CPL:576122}:

\lstinputlisting{01_helloworld/1_1.c}

\subsection{x86: \IFRU{3 аргумента}{3 arguments}}

\subsubsection{MSVC}

\IFRU{Компилируем при помощи MSVC 2010 Express, и в итоге получим:}
{Let's compile it by MSVC 2010 Express and we got:}

\begin{lstlisting}
$SG3830	DB	'a=%d; b=%d; c=%d', 00H

...

	push	3
	push	2
	push	1
	push	OFFSET $SG3830
	call	_printf
	add	esp, 16					; 00000010H
\end{lstlisting}

\IFRU{Все почти то же, за исключением того, что теперь видно, что аргументы для \printf заталкиваются в стек в обратном порядке: самый первый аргумент заталкивается последним.}
{Almost the same, but now we can see the \printf arguments are pushing into stack in reverse order: and the first argument is pushing in as the last one.}

\IFRU{Кстати, вспомним что переменные типа \Tint в 32-битной системе, как известно, имеет ширину 32 бита, это 4 байта}
{By the way, variables of \Tint type in 32-bit environment has 32-bit width that is 4 bytes}.

\IFRU{Итак, у нас всего 4 аргумента. $4*4 = 16$ ~--- именно 16 байт занимают в стеке указатель на строку плюс еще 3 числа типа \Tint.}
{So, we got here 4 arguments. $4*4 = 16$~---they occupy exactly 16 bytes in the stack: 32-bit pointer to string and 3 number of \Tint type.}

\index{x86!\Instructions!ADD}
\index{x86!\Registers!ESP}
\index{cdecl}
\IFRU{Когда при помощи инструкции \TT{``ADD ESP, X''} корректируется \glslink{stack pointer}{указатель стека} \ESP 
после вызова какой-либо функции, зачастую можно сделать вывод о том, сколько аргументов 
у вызываемой функции было, разделив X на 4.}
{When \gls{stack pointer} (the \ESP register) is corrected by \TT{``ADD ESP, X''}
instruction after a function 
call, often, the number of function arguments could be deduced here: just divide X by 4.}

\IFRU{Конечно, это относится только к cdecl-методу передачи аргументов через стек.}
{Of course, this is related only to \IT{cdecl} calling convention.}

\IFRU{См. также в соответствующем разделе о способах передачи аргументов через стек}
{See also section about calling conventions}~(\ref{sec:callingconventions}).

\IFRU{Иногда бывает так, что подряд идут несколько вызовов разных функций, 
но стек корректируется только один раз, после последнего вызова:}
{It is also possible for compiler to merge several \TT{``ADD ESP, X''} instructions into one, after last call:}

\begin{lstlisting}
push a1
push a2
call ...
...
push a1
call ...
...
push a1
push a2
push a3
call ...
add esp, 24
\end{lstlisting}

\subsubsection{MSVC \AndENRU \olly}
\index{\olly}

\IFRU{Попробуем этот же пример в}{Now let's try to load this example in} \olly.
\IFRU{Это один из наиболее популярных win32-отладчиков user-режима}{It is one of the most 
popular user-land win32 debugger}.
\IFRU{Мы можем компилировать наш пример в}{We can try to compile our example in} MSVC 2012 
\IFRU{с опцией}{with} \TT{/MD} \IFRU{что означает, линковать с библиотекой}{option, meaning, to link 
against} \TT{MSVCR*.DLL},
\IFRU{чтобы импортируемые ф-ции были хорошо видны в отладчике}{so we will able to see imported 
functions clearly in debugger}.

\IFRU{Затем загружаем исполняемый файл в}{Then load executable in} \olly.
\IFRU{Самый первый брякпойнт в}{The very first breakpoint is in} \TT{ntdll.dll}, \IFRU{нажмите}{press} 
F9 (\IFRU{запустить}{run}).
\IFRU{Второй брякпойнт в}{The second breakpoint is in} \ac{CRT}-\IFRU{коде}{code}.
\IFRU{Теперь мы должны найти ф-цию}{Now we should find} \main\EN{ function}.

\IFRU{Найдите этот код скроллируя окно кода до самого верха (MSVC располагает ф-цию \main в самом начале
секции кода)}{Find this code by scrolling the code to the very bottom (MSVC allocates \main function at
the very beginning of the code section)}: 
\figname \ref{fig:printf3_olly_1}.

\IFRU{Кликните на инструкции}{Click on} \TT{PUSH EBP}\IFRU{, нажмите}{ instruction, press} F2 
(\IFRU{установка брякпойнта}{set breakpoint}) \IFRU{и нажмите}{and press} F9 (\IFRU{запустить}{run}).
\IFRU{Нам нужно произвести все эти манипуляции, чтобы пропустить \ac{CRT}-код, потому что нам он пока
не интересен}{We need to do these manupulations in order to skip \ac{CRT}-code, because, we don't really 
interesting in it yet}.

\IFRU{Нажмите}{Press} F8 (\stepover) 6 \IFRU{раз, т.е., пропустить
6 инструкций}{times, i.e., skip 6 instructions}: \figname \ref{fig:printf3_olly_2}.

\IFRU{Теперь}{Now the} \PC \IFRU{указывает на инструкцию}{points to the}
\TT{CALL printf}\EN{ instruction}.
\olly, \IFRU{как и другие отладчики, подсвечивает регистры со значениями, которые изменились}
{like other debuggers, highlights value of registers which were changed}.
\IFRU{Так что, каждый раз, когда мы нажимаем}{So each time you press F8}, \EIP 
\IFRU{изменяется и его значение подсвечивается красным}{is changing and its value looking red}.
\ESP \IFRU{также меняется, потому что значения заталкиваются в стек}{is changing as well, 
because values are pushed into the stack}.

\IFRU{Где находятся эти значения в стеке}{Where are the values in the stack}?
\IFRU{Посмотрите на правое/нижнее окно в отладчике}{Take a look into right/bottom window of debugger}:

\begin{figure}[H]
\centering
\includegraphics[scale=0.66]{patterns/03_printf/olly3_stack.png}
\caption{\olly: \IFRU{стек, после того как значения там сохранены}{stack after values pushed}
(\IFRU{я сделал здесь округлую красную пометку в графическом редакторе}{I made round red mark 
here in graphics editor})}
\end{figure}

\IFRU{Так что здесь видно 3 столбца: адрес в стеке, значение в стеке и еще дополнительный комментарий
от \olly}{So we can see there 3 columns: address in the stack, 
value in the stack and some additional \olly comments}. 
\olly \IFRU{понимает}{understands} \printf\IFRU{-строки}{-like strings}, 
\IFRU{так что он показывает здесь и строку и 3 значения \IT{привязанных} к ней}{so it reports the 
string here and 3 values \IT{attached} to it}.

\IFRU{Нажмите}{Press} F8 (\stepover).

\IFRU{В коносил мы видим вывод}{In the console we'll see the output}:

\begin{figure}[H]
\centering
\includegraphics[scale=0.66]{patterns/03_printf/olly3_console.png}
\caption{\RU{Ф-ция }\printf \IFRU{исполнилась}{function executed}}
\end{figure}

\IFRU{Посмотрим, как изменились регистры и состояние стека}{Let's see how registers and stack state 
are changed}: \figname \ref{fig:printf3_olly_3}.

\RU{Регистр }\EAX \IFRU{теперь содержит}{register now contains} \TT{0xD} (13).
That's correct, \printf returns number of characters printed.
\RU{Значение }\EIP \IFRU{изменилось: действительно, теперь здесь адрес инструкции после}
{value is changed: indeed, now there is address of the instruction after} \TT{CALL printf}.
\RU{Значения регистров }\ECX \AndENRU \EDX \IFRU{также изменились}{values are changed as well}.
\IFRU{Очевидно, внутренности ф-ции \printf используют их для каких-то своих нужд}{Apparently, 
\printf function's hidden machinery used them for its own needs}.

\IFRU{Очень важный момент в том что значение \ESP не изменилось. И состояние стека также!}
{A very important thing is that \ESP value is not changed. And stack state too!}
\IFRU{Мы ясно видим здесь и строку формата и соответствующие ей 3 значения, они все еще здесь}
{We clearly see that format string and corresponding 3 values are still there}.
\IFRU{Действительно, по соглашению вызовов \IT{cdecl}, вызывающая ф-ция не очищает аргументы из стека}
{Indeed, that's \IT{cdecl} calling convention, calling function doesn't clear arguments in stack}.
\IFRU{Это должна делать вызывающая ф-ция}{It's caller's duty to do so}.

\IFRU{Нажмите}{Press} F8 \IFRU{снова, чтобы исполнилась инструкция}{again to execute} 
\TT{ADD ESP, 10}\EN{ instruction}: \figname \ref{fig:printf3_olly_4}.

\ESP \IFRU{изменился, но значения все еще в стеке}{is changed, but values are still in the stack}!
\IFRU{Конечно, никому не нужно заполнять эти значения нулями или что-то в этом роде}{Yes, 
of course, no one needs to fill these values by zero or something like that}.
\IFRU{Потому что всё что выше указателя стека}{Because, everything above stack pointer} (\SP) 
\IFRU{это}{is} \IT{\IFRU{шум}{noise}} \OrENRU \IT{\IFRU{мусор}{garbage}}, \IFRU{это всё не имеет
особой ценности}{it has no value at all}.
\IFRU{Было бы очень затратно по времени очищать ненужные элементы стека, к тому же, никому это и не 
нужно}{It would be time consuming to clear unused stack entries, besides, no one really needs to}.

\begin{figure}[H]
\centering
\includegraphics[scale=0.66]{patterns/03_printf/olly3_1.png}
\caption{\olly: \IFRU{самое начало ф-ции}{the very start of the} \main\EN{ function}}
\label{fig:printf3_olly_1}
\end{figure}

\begin{figure}[H]
\centering
\includegraphics[scale=0.66]{patterns/03_printf/olly3_2.png}
\caption{\olly: \IFRU{перед исполнением}{before} \printf\EN{ execution}}
\label{fig:printf3_olly_2}
\end{figure}

\begin{figure}[H]
\centering
\includegraphics[scale=0.66]{patterns/03_printf/olly3_3.png}
\caption{\olly: \IFRU{после исполнения}{after} \printf\EN{ execution}}
\label{fig:printf3_olly_3}
\end{figure}

\begin{figure}[H]
\centering
\includegraphics[scale=0.66]{patterns/03_printf/olly3_4.png}
\caption{\olly: \IFRU{после исполнения инструкции}{after} \TT{ADD ESP, 10}\EN{ instruction execution}}
\label{fig:printf3_olly_4}
\end{figure}

\subsubsection{GCC}

\IFRU{Скомпилируем то же самое в Linux при помощи GCC 4.4.1 и посмотрим в \IDA что вышло:}
{Now let's compile the same in Linux by GCC 4.4.1 and take a look in \IDA what we got:}

\begin{lstlisting}
main            proc near

var_10          = dword ptr -10h
var_C           = dword ptr -0Ch
var_8           = dword ptr -8
var_4           = dword ptr -4

                push    ebp
                mov     ebp, esp
                and     esp, 0FFFFFFF0h
                sub     esp, 10h
                mov     eax, offset aADBDCD ; "a=%d; b=%d; c=%d"
                mov     [esp+10h+var_4], 3
                mov     [esp+10h+var_8], 2
                mov     [esp+10h+var_C], 1
                mov     [esp+10h+var_10], eax
                call    _printf
                mov     eax, 0
                leave
                retn
main            endp
\end{lstlisting}

\IFRU{Можно сказать, что этот короткий код, созданный GCC, отличается от кода MSVC только способом помещения 
значений в стек.
Здесь GCC снова работает со стеком напрямую без \PUSH/\POP.}
{It can be said, the difference between code by MSVC and GCC is only in method of placing arguments on the stack.
Here GCC working directly with stack without \PUSH/\POP.}


\section{ARM}

\subsection{\NonOptimizingXcode + \ARMMode}

\lstinputlisting[caption=\NonOptimizingXcode + \ARMMode]{patterns/10_strlen/xcode_ARM_O0_en.asm}

\IFRU{Неоптимизирующий LLVM генерирует слишком много кода, зато на этом примере можно посмотреть, 
как функции работают с локальными переменными в стеке.}
{Non-optimizing LLVM generates too much code, however, here we can see how function works with 
local variables in the stack.}
\IFRU{В нашей функции только локальных переменных две, это два указателя}
{There are only two local variables in our function},
\IT{eos} \AndENRU \IT{str}.

\IFRU{В этом листинге}{In this listing}, \IFRU{сгенерированном при помощи}{generated by} \IDA, 
\IFRU{я переименовал}{I renamed} \IT{var\_8} \AndENRU \IT{var\_4} \IFRU{в}{into} \IT{eos} 
\AndENRU \IT{str} \IFRU{вручную}{manually}.

\IFRU{Итак, первые несколько инструкций просто сохраняют входное значение в переменных}{So, 
first instructions are just saves input value in} \IT{str} \AndENRU \IT{eos}.

\IFRU{Начиная с метки}{Loop body is beginning at} \IT{loc\_2CB8}\IFRU{, начинается тело цикла}{ label}.

\IFRU{Первые три инструкции в теле цикла}{First three instruction in loop body} (\TT{LDR}, \ADD, \TT{STR}) 
\IFRU{загружают значение}{loads} \IT{eos} \IFRU{в}{value into} \Reg{0}, 
\IFRU{затем происходит инкремент значения и оно сохраняется назад в локальной переменной \IT{eos} расположенной 
в стеке.}{then value is \glslink{increment}{incremented} and it is saved back into \IT{eos} local variable located in the stack.}

\index{ARM!\Instructions!LDRSB}
\IFRU{Следующая инструкция}{The next} \TT{``LDRSB R0, [R0]''} (\IT{Load Register Signed Byte}) 
\IFRU{загружает байт из памяти по адресу \Reg{0}, расширяет его до 32-бит считая его знаковым (signed) 
и сохраняет в \Reg{0}}{instruction loading byte from memory at \Reg{0} address and sign-extends it to 32-bit}.
\index{x86!\Instructions!MOVSX}
\IFRU{Это немного похоже на инструкцию}{This is similar to} \MOVSX \IFRU{в}{instruction in} x86.
\IFRU{Компилятор считает этот байт знаковым (signed), потому что тип \Tchar по стандарту Си ~--- знаковый.}
{The compiler treating this byte as signed since \Tchar type in C standard is signed.}
\IFRU{Об это я уже немного писал}{I already wrote about it}~(\ref{MOVSX}) \IFRU{в этой же секции, 
но посвященной x86}{in this section, but related to x86}.

\index{x86!8086}
\index{8080}
\index{ARM}
\IFRU{Следует также заметить, что, в ARM нет возможности использовать 8-битную или 16-битную часть 
регистра, как это возможно в x86.}
{It is should be noted, it is impossible in ARM to use 8-bit part or 16-bit part 
of 32-bit register separately of the whole register,
as it is in x86.}
\IFRU{Вероятно, это связано с тем что за x86 тянется длинный шлейф совместимости со своими предками, 
такими как
16-битный 8086 и даже 8-битный 8080, а ARM разрабатывался с чистого листа как 32-битный RISC-процессор.}
{Apparently, it is because x86 has a huge history of compatibility with its ancestors like 16-bit 8086 
and even 8-bit 8080,
but ARM was developed from scratch as 32-bit RISC-processor.}
\IFRU{Следовательно, чтобы работать с отдельными байтами на ARM, так или иначе, придется использовать 
32-битные регистры.}
{Consequently, in order to process separate bytes in ARM, one have to use 32-bit registers anyway.}

\IFRU{Итак}{So}, \TT{LDRSB} \IFRU{загружает символ из строки в \Reg{0}, по одному}
{loads symbol from string into \Reg{0}, one by one}.
\IFRU{Следующие инструкции}{Next} \CMP \AndENRU \ac{BEQ} \IFRU{проверяют, является ли этот символ $0$.}
{instructions checks, if loaded symbol is $0$.}
\IFRU{Если не $0$, то происходит переход на начало тела цикла.}{If not $0$, control passing to loop body
begin.}
\IFRU{А если $0$, выходим из цикла.}{And if $0$, loop is finishing.}

\IFRU{В конце функции вычисляется разница между}{At the end of function, a difference between} 
\IT{eos} \AndENRU \IT{str}\IFRU{, вычитается еще единица и вычисленное 
значение возвращается через \Reg{0}.}{ is calculated, 1 is also subtracting, and resulting value is returned
via \Reg{0}.}

N.B. \IFRU{В этой функции не сохранялись регистры}{Registers was not saved in this function}.
\index{ARM!\Registers!scratch registers}
\IFRU{Это потому что, по стандарту, регистры \Reg{0}-\Reg{3} называются также ``scratch registers'',
они предназначены для передачи аргументов, 
их значения не нужно восстанавливать при выходе из функции, потому что они больше не нужны в вызывающей функции.
Таким образом, их можно использовать как захочется}
{That's because by ARM calling convention, \Reg{0}-\Reg{3} registers are ``scratch registers'', 
they are intended for arguments passing,
its values may not be restored upon function exit since calling function will not use them anymore.
Consequently, they may be used for anything we want.}
\IFRU{А так как никакие больше регистры не используются, то и сохранять нечего.}
{Other registers are not used here, so that is why we have nothing to save on the stack.}
\IFRU{Поэтому, управление можно вернуть назад вызывающей функции 
простым переходом (\TT{BX}), по адресу в регистре \LR.}
{Thus, control may be returned back to calling function by simple jump (\TT{BX}),
to address in the \LR register.}

%\subsection{\NonOptimizingXcode + режим thumb}
%Практически, точно такой же код.

\subsection{\OptimizingXcode + \ThumbMode}

\lstinputlisting[caption=\OptimizingXcode + \ThumbMode]{patterns/10_strlen/xcode_thumb_O3.asm}

\IFRU{Оптимизирующий LLVM решил, что под переменные \IT{eos} и \IT{str} выделять место в стеке не обязательно}
{As optimizing LLVM concludes, space on the stack for \IT{eos} and \IT{str} may not be allocated},
\IFRU{и эти переменные можно хранить прямо в регистрах.}
{and these variables may always be stored right in registers.}
\IFRU{Перед началом тела цикла}{Before loop body beginning}, \IT{str} \IFRU{будет находиться в}{will always be in} 
\Reg{0}, \IFRU{а}{and} \IT{eos}\EMDASH\InENRU \Reg{1}.

\index{ARM!\Instructions!LDRB.W}
\index{ARM!\IFRU{Режимы адресации}{Adressing modes}}
\RU{Инструкция }\TT{``LDRB.W R2, [R1],\#1''} \IFRU{загружает в \Reg{2} байт из памяти по адресу \Reg{1}, 
расширяя его как знаковый (signed), до 32-битного
значения, но не только это.}
{instruction loads byte from memory at the address \Reg{1} into \Reg{2}, sign-extending it to 32-bit value, but not
only that.}
\TT{\#1} \IFRU{в конце инструкции называется}{at the instruction's end calling} ``Post-indexed addressing'', 
\IFRU{это значит, что после загрузки байта, к \Reg{1} добавится единица.}{this means, $1$ is to be added
to the \Reg{1} after byte load.}
\IFRU{Это очень удобно для работы с массивами.}
{That's convenient when accessing arrays.}

\index{PDP-11}
\index{\CLanguageElements!\PostIncrement}
\index{\CLanguageElements!\PostDecrement}
\index{\CLanguageElements!\PreIncrement}
\index{\CLanguageElements!\PreDecrement}
\IFRU{Такого режима адресации в x86 нет, но он есть в некоторых других процессорах, даже на PDP-11.}
{There is no such addressing mode in x86, but it is present in some other processors, even on PDP-11.}
\IFRU{Существует байка, что режимы пре-инкремента, пост-инкремента, 
пре-декремента и пост-декремента адреса в PDP-11}
{There is a legend the pre-increment, post-increment, pre-decrement and post-decrement modes in PDP-11},
\IFRU{были ``виновны'' в появлении таких конструкций языка Си (который разрабатывался на PDP-11) как}
{were ``guilty'' in appearance such C language (which developed on PDP-11) constructs as}
*ptr++, *++ptr, *ptr-{}-, *-{}-ptr. 
\IFRU{Кстати, это является труднозапоминаемой особенностью в Си.}
{By the way, this is one of hard to memorize C feature.}
\IFRU{Дела обстоят так:}{This is how it is:}

\begin{center}
\begin{tabular}{ | l | l | l | l | }
\hline
\headercolor{} \IFRU{термин в Си}{C term} & 
\headercolor{} \IFRU{термин в ARM}{ARM term} & 
\headercolor{} \IFRU{выражение Си}{C statement} & 
\headercolor{} \IFRU{как это работает}{how it works} \\
\hline
\PostIncrement & 
post-indexed addressing & 
\TT{*ptr++} & 
\IFRU{использовать значение \TT{*ptr}}{use \TT{*ptr} value}, \\
& & & \IFRU{затем инкремент указателя \TT{ptr}}{then \gls{increment} \TT{ptr} pointer} \\
\hline
\PostDecrement & 
post-indexed addressing & 
\TT{*ptr-{}-} & 
\IFRU{использовать значение \TT{*ptr}}{use \TT{*ptr} value}, \\
& & & \IFRU{затем \glslink{decrement}{декремент} указателя \TT{ptr}}{then \gls{decrement} \TT{ptr} pointer} \\
\hline
\PreIncrement & 
pre-indexed addressing & 
\TT{*++ptr} & 
\IFRU{инкремент указателя \TT{ptr}}{\gls{increment} \TT{ptr} pointer}, \\
& & & \IFRU{затем использовать значение \TT{*ptr}}{then use \TT{*ptr} value} \\
\hline
\PreDecrement & 
post-indexed addressing & 
\TT{*-{}-ptr} & 
\IFRU{\glslink{decrement}{декремент} указателя \TT{ptr}}{\gls{decrement} \TT{ptr} pointer}, \\
& & & \IFRU{затем использовать значение \TT{*ptr}}{then use \TT{*ptr} value} \\
\hline
\end{tabular}
\end{center}

\IFRU{Деннис Ритчи (один из создателей ЯП Си) указывал, что, это, вероятно, придумал Кен Томпсон 
(еще один создатель Си),
потому что подобная возможность процессора имелась еще в PDP-7}
{Dennis Ritchie (one of C language creators) mentioned that it is, probably, was invented by Ken Thompson
(another C creator) because this processor feature was present in PDP-7}
\cite{Ritchie:1986}\cite{Ritchie:1993:DCL:155360.155580}.
\IFRU{Таким образом, компиляторы с ЯП Си на тот процессор, где это есть, могут использовать это.}
{Thus, C language compilers may use it, if it is present in target processor.}

\IFRU{Далее в теле цикла можно увидеть \CMP и \ac{BNE}, они продолжают работу цикла до тех пор, 
пока не будет встречен $0$.}
{Then one may spot \CMP and \ac{BNE} in loop body, these instructions continue operation until
$0$ will be met in string.}

\index{ARM!\Instructions!MVNS}
\index{x86!\Instructions!NOT}
\RU{После конца цикла }\TT{MVNS}\footnote{MoVe Not} 
\IFRU{(инвертирование всех бит, аналог \NOT на x86)}
{(inverting all bits, \NOT in x86 analogue)}
\IFRU{и \ADD вычисляют}{instructions and \ADD computes} $eos - str - 1$.
\IFRU{На самом деле, эти две инструкции вычисляют}
{In fact, these two instructions computes}
$R0 = ~str + eos$, 
\IFRU{что эквивалентно тому, что было в исходном коде, а почему это так, я уже описывал чуть раньше, здесь}
{which is effectively equivalent to what was in source code, and why it is so, I already described here}
~(\ref{strlen_NOT_ADD}).

\IFRU{Вероятно, LLVM, как и GCC, посчитал что такой код будет короче, или быстрее.}
{Apparently, LLVM, just like GCC, concludes this code will be shorter, or faster.}

%\subsection{\OptimizingXcode + \ARMMode}
%Практически, точно такой же код.

\subsection{\OptimizingKeil{} + \ARMMode}

\lstinputlisting[caption=\OptimizingKeil + \ARMMode]{patterns/10_strlen/Keil_ARM_O3.asm}

\index{ARM!\Instructions!SUBEQ}
\IFRU{Практически то же самое что мы уже видели, за тем исключением что выражение}
{Almost the same what we saw before, with the exception the}
$str - eos - 1$ 
\IFRU{может быть вычислено не в самом конце функции, а прямо в теле цикла.}
{expression may be computed not at the function's end, but right in loop body.}
\RU{Суффикс }\TT{-EQ}\IFRU{, как мы помним, означает что инструкция будет выполнена только
если операнды в исполненной перед этим инструкции \CMP были равны.}
{suffix, as we may recall, means the instruction will be executed only if operands in executed before
\CMP were equal to each other.}
\IFRU{Таким образом}{Thus}, \IFRU{если в \Reg{0} будет $0$}{if $0$ will be in the \Reg{0} register},
\IFRU{обе инструкции}{both} \TT{SUBEQ} \IFRU{исполнятся и результат останется в \Reg{0}.}
{instructions are to be executed and result is leaving in the \Reg{0} register.}



\section{\Stack}
\label{sec:stack}
\index{\Stack}

\IFRU{Стек в компьютерных науках ~--- это одна из наиболее фундаментальных вещей}
{Stack ~--- is one of the most fundamental things in computer science}
\footnote{\url{http://en.wikipedia.org/wiki/Call_stack}}.

\IFRU{Технически, это просто блок памяти в памяти процесса + регистр \ESP или \RSP в x86, либо \SP в ARM, который указывает где-то в пределах этого блока.}
{Technically it is just a memory block in process memory + the an \ESP or the \RSP register in x86 or the \SP register in ARM as a pointer within the block.}

\index{ARM!\Instructions!PUSH}
\index{ARM!\Instructions!POP}
\index{x86!\Instructions!PUSH}
\index{x86!\Instructions!POP}
\IFRU{Часто используемые инструкции для работы со стеком это \PUSH и \POP (в x86 и thumb-режиме ARM). 
\PUSH уменьшает \ESP/\RSP/\SP на $4$ в 32-битном режиме (или на $8$ в 64-битном),
затем записывает по адресу на который указывает \ESP/\RSP/\SP содержимое своего единственного операнда.}
{The most frequently used stack access instructions are \PUSH and \POP (in both x86 and ARM thumb-mode). 
\PUSH subtracts $4$ in 32-bit mode (or $8$ in 64-bit mode) from \ESP/\RSP/\SP and then writes the contents of its sole operand to the memory address pointed to by \ESP/\RSP/\SP.} 

\IFRU{\POP это обратная операция ~--- сначала достает из \glslink{stack pointer}{указателя стека} значение и помещает его в операнд 
(который очень часто является регистром) и затем увеличивает указатель стека на $4$ (или $8$).}
{\POP is the reverse operation: get the data from memory pointed to by \SP, 
put it in the operand (often a register) and then add $4$ (or $8$) to the \gls{stack pointer}.}

\IFRU{В самом начале, \glslink{stack pointer}{регистр-указатель} указывает на конец стека.}
{After stack allocation the \gls{stack pointer} points to the end of stack.}
\IFRU{\PUSH уменьшает \glslink{stack pointer}{регистр-указатель}, а \POP ~--- увеличивает.}
{\PUSH increases the \gls{stack pointer} and \POP decreases it.}
\IFRU{Конец стека находится в начале блока памяти выделенного под стек. Это странно, но это так.}
{The end of the stack is actually at the beginning of the memory allocated for the stack block. 
It seems strange, but it is so.}

\IFRU{В процессоре ARM, тем не менее, есть поддержка стеков растущих как в сторону уменьшения, так и в
сторону увеличения}
{Nevertheless ARM has not only instructions supporting ascending stacks but also descending stacks}. \\
\\
\index{ARM!\Instructions!STMFD}
\index{ARM!\Instructions!LDMFD}
\index{ARM!\Instructions!STMED}
\index{ARM!\Instructions!LDMED}
\index{ARM!\Instructions!STMFA}
\index{ARM!\Instructions!LDMFA}
\index{ARM!\Instructions!STMEA}
\index{ARM!\Instructions!LDMEA}
\IFRU{Например, инструкции}{For example the} 
STMFD\footnote{\STMFDdesc}/LDMFD\footnote{\LDMFDDESC}, 
STMED\footnote{\STMEDdesc}/LDMED\footnote{\LDMEDdesc} 
\IFRU{предназначены для descending-стека, т.е., уменьшающегося}{instructions are intended to work with 
a descending stack}.
\IFRU{Инструкции}{The}
STMFA\footnote{\STMFAdesc}/LMDFA\footnote{\LDMFAdesc}, 
STMEA\footnote{\STMEAdesc}/LDMEA\footnote{\LDMEAdesc} 
\IFRU{предназначены для ascending-стека, т.е., увеличивающегося}{instructions are intended to work with 
an ascending stack}.

\subsection{\IFRU{Для чего используется стек?}{What is the stack used for?}}

\subsubsection{\IFRU{Сохранение адреса куда должно вернуться управление после вызова функции}
{Save the return address where a function should return control after execution}}

\paragraph{x86}

\index{x86!\Instructions!CALL}
\IFRU{При вызове другой функции через \CALL, сначала в стек записывается адрес указывающий на место аккурат после 
инструкции \CALL, затем делается безусловный переход (почти как \TT{JMP}) на адрес указанный в операнде.} 
{While calling another function with a \CALL instruction the address of the point exactly after the \CALL instruction is saved 
to the stack and then an unconditional jump to the address in the CALL operand is executed.} 

\index{x86!\Instructions!PUSH}
\index{x86!\Instructions!JMP}
\IFRU{\CALL это аналог пары инструкций \TT{PUSH address\_after\_call / JMP}.}
{The \CALL instruction is equivalent to a \TT{PUSH address\_after\_call / JMP operand} instruction pair}.

\index{x86!\Instructions!RET}
\index{x86!\Instructions!POP}
\IFRU{\RET вытаскивает из стека значение и передает управление по этому адресу ~--- 
это аналог пары инструкций \TT{POP tmp / JMP tmp}.}
{\RET fetches a value from the stack and jumps to it ~--- it is equivalent to a \TT{POP tmp / JMP tmp} instruction pair.}

\index{\Stack!\IFRU{Переполнение стека}{Stack overflow}}
\index{\Recursion}
\IFRU{Крайне легко устроить переполнение стека запустив бесконечную рекурсию:}
{Overflow the stack is simple. Just run eternal recursion:}

\begin{lstlisting}
void f()
{
	f();
};
\end{lstlisting}

\IFRU{MSVC 2008 предупреждает о проблеме:}{MSVC 2008 reports the problem:}

\begin{lstlisting}
c:\tmp6>cl ss.cpp /Fass.asm
Microsoft (R) 32-bit C/C++ Optimizing Compiler Version 15.00.21022.08 for 80x86
Copyright (C) Microsoft Corporation.  All rights reserved.

ss.cpp
c:\tmp6\ss.cpp(4) : warning C4717: 'f' : recursive on all control paths, function will cause runtime stack overflow
\end{lstlisting}

\dots \IFRU{но тем не менее создает нужный код}{but generates the right code anyway}:

\begin{lstlisting}
?f@@YAXXZ PROC						; f
; File c:\tmp6\ss.cpp
; Line 2
	push	ebp
	mov	ebp, esp
; Line 3
	call	?f@@YAXXZ				; f
; Line 4
	pop	ebp
	ret	0
?f@@YAXXZ ENDP						; f
\end{lstlisting}

\dots \IFRU
{причем, если включить оптимизацию (\Ox), то будет даже интереснее, без переполнения стека, 
но работать будет \IT{корректно}\footnote{здесь ирония}:}
{Also if we turn on optimization (\Ox option) the optimized code will not overflow the stack 
but will work \IT{correctly}\footnote{irony here}:}

\begin{lstlisting}
?f@@YAXXZ PROC						; f
; File c:\tmp6\ss.cpp
; Line 2
$LL3@f:
; Line 3
	jmp	SHORT $LL3@f
?f@@YAXXZ ENDP						; f
\end{lstlisting}

\IFRU{GCC 4.4.1 генерирует точно такой же код в обоих случаях, хотя и не предупреждает о проблеме.}
{GCC 4.4.1 generating the same code in both cases, although not warning about problem.}

\paragraph{ARM}

\index{ARM!\Registers!Link Register}
\IFRU{Программы для ARM также используют стек для сохранения \ac{RA}, куда нужно вернуться, но несколько иначе}{ARM
programs also use the stack for saving return addresses, but differently}.
\IFRU{Как уже упоминалось в секции}{As it was mentioned in} ``\HelloWorldSectionName''~\ref{sec:hw_ARM}, 
\IFRU{\ac{RA} записывается в регистр}{the \ac{RA} is saved to the} \LR (\IT{link register}).
\IFRU{Но если есть необходимость вызывать какую-то другую функцию, и использовать регистр \LR еще
раз, его значение желательно сохранить}
{However, if one needs to call another function and use the \LR register
one more time its value should be saved}.
\index{Function prologue}
\IFRU{Обычно, это происходит в прологе функции, часто мы видим там инструкцию вроде}
{Usually it is saved in the function prologue. Often, we see instructions like}
\index{ARM!\Instructions!PUSH}
\index{ARM!\Instructions!POP}
\TT{``PUSH {R4-R7,LR}''} \IFRU{, а в эпилоге}{along with this instruction in epilogue} \TT{``POP {R4-R7,PC}''} ~--- 
\IFRU{так сохраняются регистры, которые будут использоваться в текущей функции, в том числе}
{thus register values
to be used in the function are saved in the stack, including} \LR.

\index{ARM!Leaf function}
\IFRU{Тем не менее, если некая функция не вызывает никаких более функций, в терминологии ARM она называется}
{Nevertheless, if a function never calls any other function, in ARM terminology it is called}
\IT{leaf function}\footnote{\url{http://infocenter.arm.com/help/index.jsp?topic=/com.arm.doc.faqs/ka13785.html}}. 
\IFRU{Как следствие, ``leaf''-функция не использует регистр \LR}
{As a consequence ``leaf'' functions do not use the \LR register}.
\IFRU{А если эта функция небольшая, использует мало регистров, она может не использовать стек вообще}
{And if this function is small and it uses a small number of registers it may not use stack at all}.
\IFRU{Таким образом, в ARM возможен вызов небольших ``leaf'' функций не используя стек}
{Thus, it is possible to call ``leaf'' functions without using stack}.
\IFRU{Это может быть быстрее чем в x86, ведь внешняя память для стека не используется}
{This can be faster than on x86 because external RAM is not used for the stack}
\footnote{\IFRU{Когда-то очень давно, на PDP-11 и VAX, на инструкцию CALL (вызов других функций) могло тратиться
вплоть до 50\% времени, возможно из-за работы с памятью, 
поэтому считалось что много небольших функций это анти-паттерн}
{Some time ago, on PDP-11 and VAX, CALL instruction (calling other functions) was expensive, up to 50\%
of execution time might be spent on it, so it was common sense that big number of small function is anti-pattern}\cite[Chapter 4, Part II]{Raymond:2003:AUP:829549}.}.
\IFRU{Либо, это может быть полезным для тех ситуаций, когда память для стека еще не выделена либо недоступна}
{It can be useful for such situations when memory for the stack is not yet allocated or not available}.

\subsection{\IFRU{Передача параметров для функции}{Function arguments passing}}

\begin{lstlisting}
push arg3
push arg2
push arg1
call f
add esp, 4*3
\end{lstlisting}

\IFRU{Вызываемая функция получает свои параметры также через указатель стека.}
{Callee{\footnote{Function being called}} function get its arguments via stack ponter.}

\IFRU{См.также в соответствующем разделе о способах передачи аргументов через стек}
{See also section about calling conventions}~\ref{sec:callingconventions}.

\IFRU{Важно отметить, что, в общем, никто не заставляет программистов передавать параметры именно через стек,
это не является требованием к исполняемому коду.}
{It is important to note that no one oblige programmers to pass arguments through stack, it is not prerequisite.}

\IFRU{Вы можете делать это совершенно иначе, не используя стек.}
{One could implement any other method not using stack.}

\IFRU{К примеру, можно выделять в куче\footnote{heap в англоязычной литературе} место для аргументов, 
заполнять их и передавать в функцию указатель на это место через \EAX. И это вполне будет работать}
{For example, it is possible to allocate a place for arguments in heap, fill it and pass to a function 
via pointer to this pack in \EAX register. And this will work}
\footnote{\IFRU{Например, в книге Дональда Кнута ``Искусство программирования'', в разделе 1.4.1 
посвященном подпрограммам\cite[раздел 1.4.1]{Knuth:1998:ACP:521463}, 
мы можем прочитать о возможности располагать параметры для вызываемой подпрограммы после инструкции \JMP
передающей управление подпрограмме. Кнут описывает что это было особенно удобно для компьютеров System/360.}
{For example, in ``The Art of Computer Programming'' book by Donald Knuth, 
in section 1.4.1 dedicated to subroutines\cite[section 1.4.1]{Knuth:1998:ACP:521463},
we can read about one way to supply arguments to subroutine is simply to list them after the \JMP instruction
passing control to subroutine. Knuth writes that this method was particularly convenient on System/360.}}.

\IFRU{Однако, так традиционно сложилось, что в x86 и ARM передача аргументов происходит именно через стек.}
{However, it is convenient tradition in x86 and ARM to use stack for this.}

\subsubsection{\IFRU{Хранение локальных переменных}{Local variable storage}}

\IFRU{Функция может выделить для себя некоторое место в стеке для локальных переменных просто отодвинув 
\glslink{stack pointer}{указатель стека} глубже к концу стека.}
{A function could allocate a space in the stack for its local variables just by shifting 
the \gls{stack pointer} towards stack bottom.}

\IFRU{Это снова не является необходимым требованием. Вы можете хранить локальные переменные где угодно. 
Но по традиции всё сложилось так.}
{It is also not a requirement. You could store local variables wherever you like. 
But traditionally it is so.}

\subsection{x86: \IFRU{Функция alloca()}{alloca() function}}
\label{alloca}
\index{\CStandardLibrary!alloca()}
\IFRU{Интересен случай с функцией \TT{alloca()}}
{It is worth noting \TT{alloca()} function.}\footnote{
\IFRU
{В MSVC, реализацию функции можно посмотреть в файлах}
{As of MSVC, function implementation can be found in} 
  \TT{alloca16.asm} 
  \IFRU{и}{and} 
  \TT{chkstk.asm} 
  \IFRU{в}{in} 
  \TT{C:\textbackslash{}Program Files (x86)\textbackslash{}Microsoft Visual Studio 10.0\textbackslash{}VC\textbackslash{}crt\textbackslash{}src\textbackslash{}intel}}. 

\IFRU{Эта функция работает как \TT{malloc()}, но выделяет память прямо в стеке.} 
{This function works like \TT{malloc()} but allocates memory just in stack.}

\IFRU{Память освобождать через \TT{free()} не нужно, так как эпилог функции~\ref{sec:prologepilog} 
вернет \ESP назад в изначальное состояние и выделенная память просто аyнулируется.}
{Allocated memory chunk is not needed to be freed via \TT{free()} function call since 
function epilogue~\ref{sec:prologepilog} shall return value of the \ESP back to initial state and 
allocated memory will be just annuled.} 

\IFRU{Интересна реализация функции \TT{alloca()}.}
{It is worth noting how \TT{alloca()} implemented.}

\IFRU{Эта функция, если упрощенно, просто сдвигает \ESP вглубь стека 
на столько байт сколько вам нужно и возвращает \ESP в качестве указателя на выделенный блок.}
{This function, if to simplify, just shifting \ESP deeply to stack bottom so much bytes you 
need and set \ESP as a pointer to that \IT{allocated} block.}
\IFRU{Попробуем:}{Let's try:}

\lstinputlisting{02_stack/2_1.c}

\IFRU{(Функция \TT{\_snprintf()} работает так же как и \printf, только вместо выдачи результата в 
stdout (т.е., на терминал или в консоль),
записывает его в буфер \TT{buf}. \puts выдает содержимое буфера \TT{buf} в stdout. Конечно, можно было бы
заменить оба этих вызова на один \printf, но мне нужно проиллюстрировать использование небольшого буфера.)}
{(\TT{\_snprintf()} function works just like \printf, but instead dumping result into stdout (e.g., to terminal or 
console), write it to the \TT{buf} buffer. \puts copies \TT{buf} contents to stdout. Of course, these two
function calls might be replaced by one \printf call, but I would like to illustrate small buffer usage.)}

\subsubsection{MSVC}

\IFRU{Компилируем}{Let's compile} (MSVC 2010):

\lstinputlisting[caption=MSVC 2010]{02_stack/2_2_msvc.asm}

\index{Compiler intrinsic}
\IFRU {Единственный параметр в \TT{alloca()} передается через \EAX, а не как обычно через стек}
{The sole \TT{alloca()} argument passed via \EAX (instead of pushing into stack)}
\footnote{\IFRU{Это потому что alloca() это не сколько функция, сколько т.е. compiler intrinsic}{It's because
alloca() is rather compiler intrinsic than usual function}}.
\IFRU{После вызова \TT{alloca()}, \ESP теперь указывает на блок в 600 байт который 
мы можем использовать под \TT{buf}.}
{After \TT{alloca()} call, \ESP is now pointing to the block of 600 bytes and we can 
use it as memory for \TT{buf} array.}

\subsubsection{GCC + \IntelSyntax}

\IFRU{А GCC 4.4.1 обходится без вызова других функций:}
{GCC 4.4.1 can do the same without calling external functions:}

\lstinputlisting[caption=GCC 4.7.3]{\IFRU{02_stack/2_1_gcc_intel_O3_ru.asm}{02_stack/2_1_gcc_intel_O3_en.asm}}

\subsubsection{GCC + \ATTSyntax}

\IFRU{Посмотрим на тот же код, только в синтаксисе AT\&T}{Let's see the same code, but in AT\&T syntax}:

\lstinputlisting[caption=GCC 4.7.3]{02_stack/2_1_gcc_ATT_O3.s}

\index{\ATTSyntax}
\IFRU{Всё то же самое что и в прошлом листинге.}{The same code as in previos listing.}

\IFRU{Обратите внимание что, например}{Please note that, for example}, \TT{movl \$3, 20(\%esp)} 
\IFRU{это аналог}{is analogous to} \TT{mov DWORD PTR [esp+20], 3} \IFRU{в Intel-синтаксисе}{in Intel-syntax} ~--- 
\IFRU{при адресации памяти в виде}{when addressing memory in form} \IT{\IFRU{регистр+смещение}{register+offset}}, 
\IFRU{это записывается в AT\&T синтаксисе как}{it's written in AT\&T syntax as} 
\TT{\IFRU{смещение}{offset}(\%\IFRU{регистр}{register})}.


\subsection{(Windows) SEH}
\index{Windows!Structured Exception Handling}

\IFRU{В стеке хранятся записи SEH (\IT{Structured Exception Handling}) для функции (если имеются)}
{SEH (\IT{Structured Exception Handling}) records are also stored in stack (if needed).}
\footnote{
\IFRU{О SEH: классическая статья Мэтта Питрека}{Classic Matt Pietrek article about SEH}: 
\url{http://www.microsoft.com/msj/0197/Exception/Exception.aspx}}.

\subsection{\RU{Защита от переполнений буфера}\EN{Buffer overflow protection}\PTBR{Proteção contra estouro de buffer}}

\RU{Здесь больше об этом}\EN{More about it here}\PTBR{Mais sobre aqui}~(\myref{subsec:bufferoverflow}).



\section{\PrintfSeveralArgumentsSectionName}

\IFRU{Попробуем теперь немного расширить пример \IT{\HelloWorldSectionName}~\ref{sec:helloworld}, 
написав в теле функции \main:}
{Now let's extend \IT{\HelloWorldSectionName}~\ref{sec:helloworld} example, replacing \printf in 
the \main function body by this:}

\begin{lstlisting}
printf("a=%d; b=%d; c=%d", 1, 2, 3);
\end{lstlisting}

\subsection{x86}

\IFRU{Компилируем при помощи MSVC 2010 Express, и в итоге получим:}
{Let's compile it by MSVC 2010 and we got:}

\begin{lstlisting}
$SG3830	DB	'a=%d; b=%d; c=%d', 00H

...

	push	3
	push	2
	push	1
	push	OFFSET $SG3830
	call	_printf
	add	esp, 16					; 00000010H
\end{lstlisting}

\IFRU{Все почти то же, за исключением того, что теперь видно, что аргументы для \printf заталкиваются в стек в обратном порядке: самый первый аргумент заталкивается последним.}
{Almost the same, but now we can see the \printf arguments are pushing into stack in reverse order: and the first argument is pushing in as the last one.}

\IFRU{Кстати, вспомним что переменные типа \Tint в 32-битной системе, как известно, имеет ширину 32 бита, это 4 байта}
{By the way, variables of \Tint type in 32-bit environment has 32-bit width that is 4 bytes}.

\IFRU{Итак, у нас всего 4 аргумента. $4*4 = 16$ ~--- именно 16 байт занимают в стеке указатель на строку плюс еще 3 числа типа \Tint.}
{So, we got here 4 arguments. $4*4 = 16$ ~--- they occupy exactly 16 bytes in the stack: 32-bit pointer to string and 3 number of \Tint type.}

\index{x86!\Instructions!ADD}
\index{x86!\Registers!ESP}
\index{cdecl}
\IFRU{Когда при помощи инструкции \TT{``ADD ESP, X''} корректируется указатель стека \ESP 
после вызова какой-либо функции, зачастую можно сделать вывод о том, сколько аргументов 
у вызываемой функции было, разделив X на 4.}
{When stack pointer (the \ESP register) is corrected by \TT{``ADD ESP, X''}
instruction after a function 
call, often, the number of function arguments could be deduced here: just divide X by 4.}

\IFRU{Конечно, это относится только к cdecl-методу передачи аргументов через стек.}
{Of course, this is related only to \IT{cdecl} calling convention.}

\IFRU{См.также в соответствующем разделе о способах передачи аргументов через стек}
{See also section about calling conventions}~(\ref{sec:callingconventions}).

\IFRU{Иногда бывает так, что подряд идут несколько вызовов разных функций, 
но стек корректируется только один раз, после последнего вызова:}
{It is also possible for compiler to merge several \TT{``ADD ESP, X''} instructions into one, after last call:}

\begin{lstlisting}
push a1
push a2
call ...
...
push a1
call ...
...
push a1
push a2
push a3
call ...
add esp, 24
\end{lstlisting}

\IFRU{Скомпилируем то же самое в Linux при помощи GCC 4.4.1 и посмотрим в \IDA что вышло:}
{Now let's compile the same in Linux by GCC 4.4.1 and take a look in \IDA what we got:}

\begin{lstlisting}
main            proc near

var_10          = dword ptr -10h
var_C           = dword ptr -0Ch
var_8           = dword ptr -8
var_4           = dword ptr -4

                push    ebp
                mov     ebp, esp
                and     esp, 0FFFFFFF0h
                sub     esp, 10h
                mov     eax, offset aADBDCD ; "a=%d; b=%d; c=%d"
                mov     [esp+10h+var_4], 3
                mov     [esp+10h+var_8], 2
                mov     [esp+10h+var_C], 1
                mov     [esp+10h+var_10], eax
                call    _printf
                mov     eax, 0
                leave
                retn
main            endp
\end{lstlisting}

\IFRU{Можно сказать, что этот короткий код созданный GCC отличается от кода MSVC только способом помещения 
значений в стек.
Здесь GCC снова работает со стеком напрямую без \PUSH/\POP.}
{It can be said, the difference between code by MSVC and GCC is only in method of placing arguments on the stack.
Here GCC working directly with stack without \PUSH/\POP.}

\subsection{ARM: \IFRU{3 аргумента в \printf}{3 \printf arguments}}

\IFRU{В ARM традиционно принята такая схема передачи аргументов в функцию: 4 первых аргумента через регистры R0-R3,
а остальные ~--- через стек}{Traditionally, ARM arguments passing scheme (calling convention) is as follows:
the 4 first arguments are passed in R0-R3 registers and remaining arguments ~--- via stack}.
\IFRU{Это немного похоже на то как аргументы передаются в}{This resembling arguments passing scheme in} 
fastcall~\ref{fastcall} \IFRU{или}{or} win64~\ref{sec:callingconventions_win64}.

\subsubsection{\NonOptimizingKeil: \ARMMode}

\begin{lstlisting}
.text:00000014             printf_main1
.text:00000014 10 40 2D E9                 STMFD   SP!, {R4,LR}
.text:00000018 03 30 A0 E3                 MOV     R3, #3
.text:0000001C 02 20 A0 E3                 MOV     R2, #2
.text:00000020 01 10 A0 E3                 MOV     R1, #1
.text:00000024 1D 0E 8F E2                 ADR     R0, aADBDCD     ; "a=%d; b=%d; c=%d\n"
.text:00000028 0D 19 00 EB                 BL      __2printf
.text:0000002C 10 80 BD E8                 LDMFD   SP!, {R4,PC}
\end{lstlisting}

\IFRU{Итак, первые 4 аргумента передаются через регистры R0-R3, по порядку: указатель на формат-строку для \printf
в R0, затем $1$ в R1, $2$ в R2 и $3$ в R3}{So, the first 4 arguments are passing via R0-R3 arguments in this order:
pointer to \printf format string in R0, then $1$ in R1, $2$ in R2 and $3$ in R3}.

\IFRU{Пока что, здесь нет ничего необычного}{Nothing unusual so far}.

\subsubsection{\OptimizingKeil: \ARMMode}
\label{ARM_B_to_printf}

\begin{lstlisting}
.text:00000014                             EXPORT printf_main1
.text:00000014             printf_main1
.text:00000014 03 30 A0 E3                 MOV     R3, #3
.text:00000018 02 20 A0 E3                 MOV     R2, #2
.text:0000001C 01 10 A0 E3                 MOV     R1, #1
.text:00000020 1E 0E 8F E2                 ADR     R0, aADBDCD     ; "a=%d; b=%d; c=%d\n"
.text:00000024 CB 18 00 EA                 B       __2printf
\end{lstlisting}

\IFRU{Это соптимизированная версия (\Othree) для режима ARM, и здесь мы видим последнюю инструкцию: 
\TT{B} вместо привычной нам \TT{BL}}{This is optimized (\Othree) version for ARM mode and here we see \TT{B} as
the last instruction instead of familiar \TT{BL}}.
\IFRU{Отличия между этой соптимзированной версией и предыдущей, скомпилированной без оптимизации, еще и в том, 
что здесь нет пролога и эпилога функции (инструкций, сохранающих состояние регистров \TT{R4} и \LR)}
{Another difference between this optimized version and previous one, compiled without optimization, is also in the
fact that there are no function prologue and epilogue (instructions saving \TT{R4} and \LR registers values)}.
\IFRU{Инструкция \TT{B} просто переходит на другой адрес, без манипуляций с регистром \LR, то есть,
это аналог \JMP в x86}{\TT{B} instruction just jumping to another address, without any \LR register manipulation, 
that is, it's \JMP analogue in x86}.
\IFRU{Почему это работает нормально? Потому что этот код эквивалентен предыдущему.}{Why it works fine? Because
this code is in fact equivalent to the previous.}
\IFRU{Основных причин две: 1) стек не модифицируется, как и указатель стека \SP; 2) вызов функции \printf последний, 
после него ничего не происходит}{There are two main reasons: 1) stack is not modified, as well as \SP stack pointer; 
2) \printf call is the last one, there are nothing going on after it}.
\IFRU{Функция \printf, отработав, просто вернет управление по адресу, записанному в \LR}{After finishing, \printf
function will just return control to the address stored in \LR}. 
\IFRU{Но в \LR находится адрес места, откуда была вызвана наша функция}{But the address of the place from where our function
was called is now in \LR}!
\IFRU{А следовательно, управление из \printf вернется сразу туда}{And consequently, control from \printf will returned to that
place}.
\IFRU{Следовательно, нет нужды сохранять \LR, потому что нет нужны модифицировать \LR}{As a consequent, we don't need to save
\LR, because we don't need to modify \LR}.
\IFRU{А нет нужды модифицировать \LR, потому что нет иных вызовов функций, кроме \printf, к тому же, после этого вызова не нужно ничего здесь больше делать}{And we don't need to modify \LR because there are no other functions calls except \printf, furthermore,
after this call we are not planning to do anything}!
\IFRU{Поэтому такая оптимизация возможна}{That's why this optimization is possible}.

\IFRU{Еще один похожий пример описан в секции}{Another similar example was described in} ``\SwitchCaseDefaultSectionName'' 
\IFRU{, здесь}{section, here}~\ref{jump_to_last_printf}.

\subsubsection{\OptimizingKeil: \ThumbMode}

\begin{lstlisting}
.text:0000000C             printf_main1
.text:0000000C 10 B5                       PUSH    {R4,LR}
.text:0000000E 03 23                       MOVS    R3, #3
.text:00000010 02 22                       MOVS    R2, #2
.text:00000012 01 21                       MOVS    R1, #1
.text:00000014 A4 A0                       ADR     R0, aADBDCD     ; "a=%d; b=%d; c=%d\n"
.text:00000016 06 F0 EB F8                 BL      __2printf
.text:0000001A 10 BD                       POP     {R4,PC}
\end{lstlisting}

\IFRU{Здесь нет особых отличий от неоптимизированного варианта для режима ARM}{There are no significant difference from 
non-optimized code for ARM mode}.



\subsection{ARM: \IFRU{8 аргументов в \printf}{8 \printf arguments}}

Для того, чтобы посмотреть, как остальные аргументы будут передаваться через стек, изменим пример еще раз, 
увеличив количество передаваемых аргументов до 9 (строка формата \printf и еще 8 переменных типа \Tint):

\begin{lstlisting}
void printf_main2()
{
	printf("a=%d; b=%d; c=%d; d=%d; e=%d; f=%d; g=%d; h=%d\n", 1, 2, 3, 4, 5, 6, 7, 8);
};
\end{lstlisting}

\subsubsection{\OptimizingKeil: \ARMMode}

\begin{lstlisting}
.text:00000028             printf_main2
.text:00000028
.text:00000028             var_18          = -0x18
.text:00000028             var_14          = -0x14
.text:00000028             var_4           = -4
.text:00000028
.text:00000028 04 E0 2D E5                 STR     LR, [SP,#var_4]!
.text:0000002C 14 D0 4D E2                 SUB     SP, SP, #0x14
.text:00000030 08 30 A0 E3                 MOV     R3, #8
.text:00000034 07 20 A0 E3                 MOV     R2, #7
.text:00000038 06 10 A0 E3                 MOV     R1, #6
.text:0000003C 05 00 A0 E3                 MOV     R0, #5
.text:00000040 04 C0 8D E2                 ADD     R12, SP, #0x18+var_14
.text:00000044 0F 00 8C E8                 STMIA   R12, {R0-R3}
.text:00000048 04 00 A0 E3                 MOV     R0, #4
.text:0000004C 00 00 8D E5                 STR     R0, [SP,#0x18+var_18]
.text:00000050 03 30 A0 E3                 MOV     R3, #3
.text:00000054 02 20 A0 E3                 MOV     R2, #2
.text:00000058 01 10 A0 E3                 MOV     R1, #1
.text:0000005C 6E 0F 8F E2                 ADR     R0, aADBDCDDDEDFDGD ; "a=%d; b=%d; c=%d; d=%d; e=%d; f=%d; g=%"...
.text:00000060 BC 18 00 EB                 BL      __2printf
.text:00000064 14 D0 8D E2                 ADD     SP, SP, #0x14
.text:00000068 04 F0 9D E4                 LDR     PC, [SP+4+var_4],#4
\end{lstlisting}

Этот код можно условно разделить на несколько частей:

\begin{itemize}
\item Пролог функции:

Самая первая инструкция \TT{``STR LR, [SP,\#var\_4]!''} сохраняет в стеке \LR, ведь, 
нам придется использовать его для вызова \printf.

Вторая инструкция \TT{``SUB SP, SP, \#0x14''} уменьшает указатель стека \SP, но на самом деле, эта процедура нужна для выделения в локальном стеке места размером 0x14 (20) байт. Действительно, нам нужно передать 5 32-битных значений через стек в \printf, каждое значение занимает 4 байта, а $5*4=20$ --- как раз. Остальные 4 32-битных значения будут переданы через регистры.

\item Передача 5, 6, 7 и 8 через стек:

Затем значения 5, 6, 7 и 8 записываются в регистры R0, R1, R2 и R3 соответственно. Затем инструкция \TT{``ADD R12, SP, \#0x18+var\_14''} записывает в регистр R12 место в стеке, куда будут помещены эти 4 значения. 
\IT{var\_14} это макрос ассемблера, равный $-0x14$, такие макросы создает \IDA, чтобы удобнее было показывать, как код обращается к стеку. Переменные \IT{var\_?}, создаваемые в \IDA это локальные переменные в стеке. 
Так что, в \TT{R12} будет записано $SP+4$. 
Следующая инструкция \TT{``STMIA R12, {R0-R3}''} записывает содержимое регистров R0-R3 по адресу в памяти, на который указывает R12. 
Инструкция \TT{STMIA} означает \IT{Store Multiple Increment After}. 
\IT{Increment After} означает что R12 будет увеличиваться на 4 после записи каждого значения регистра.

\item Передача 4 через стек:
4 записывается в R0, затем, это значение, при помощи инструкции \TT{``STR R0, [SP,\#0x18+var\_18]''} попадает
в стек. \IT{var\_18} равен $-0x18$, смещение будет 0, так что, значение из регистра R0 (4) запишется туда, куда
указывает \SP.

\item Передача 1, 2 и 3 через регистры:

Значения для первых трех чисел (a, b, c) (1, 2, 3 соответственно) передаются в регистрах R1, R2 и R3 перед самим
вызововм \printf, а остальные 5 значений передаются через стек, и вот как.

\item Вызов \printf:

\item Эпилог функции:

Инструкция \TT{``ADD SP, SP, \#0x14''} возвращает \SP на прежнее место, аннулируя таким образом, всё что было
записано в стеке. Конечно, то что было записано туда, там пока и остается, но всё это будет многократно 
перезаписано следующими функциями.

Инструкция \TT{``LDR PC, [SP+4+var\_4],\#4''} загружает в \PC сохраненное значение \LR из стека, таким образом,
обеспечивая выход из функции.

\end{itemize}

\subsubsection{\OptimizingKeil: \ThumbMode}

\begin{lstlisting}
.text:0000001C             printf_main2
.text:0000001C
.text:0000001C             var_18          = -0x18
.text:0000001C             var_14          = -0x14
.text:0000001C             var_8           = -8
.text:0000001C
.text:0000001C 00 B5                       PUSH    {LR}
.text:0000001E 08 23                       MOVS    R3, #8
.text:00000020 85 B0                       SUB     SP, SP, #0x14
.text:00000022 04 93                       STR     R3, [SP,#0x18+var_8]
.text:00000024 07 22                       MOVS    R2, #7
.text:00000026 06 21                       MOVS    R1, #6
.text:00000028 05 20                       MOVS    R0, #5
.text:0000002A 01 AB                       ADD     R3, SP, #0x18+var_14
.text:0000002C 07 C3                       STMIA   R3!, {R0-R2}
.text:0000002E 04 20                       MOVS    R0, #4
.text:00000030 00 90                       STR     R0, [SP,#0x18+var_18]
.text:00000032 03 23                       MOVS    R3, #3
.text:00000034 02 22                       MOVS    R2, #2
.text:00000036 01 21                       MOVS    R1, #1
.text:00000038 A0 A0                       ADR     R0, aADBDCDDDEDFDGD ; "a=%d; b=%d; c=%d; d=%d; e=%d; f=%d; g=%"...
.text:0000003A 06 F0 D9 F8                 BL      __2printf
.text:0000003E
.text:0000003E             loc_3E                                  ; CODE XREF: example13_f+16
.text:0000003E 05 B0                       ADD     SP, SP, #0x14
.text:00000040 00 BD                       POP     {PC}
\end{lstlisting}

Это почти то же самое что и в предыдущем примере, только код для thumb и значения укладываются в 
стек немного иначе: в начале 8 за первый раз, затем 5, 6, 7 за второй раз и 4 за третий раз.

\subsubsection{\OptimizingXcode: \ARMMode}

\begin{lstlisting}
__text:0000290C             _printf_main2
__text:0000290C
__text:0000290C             var_1C          = -0x1C
__text:0000290C             var_C           = -0xC
__text:0000290C
__text:0000290C 80 40 2D E9                 STMFD           SP!, {R7,LR}
__text:00002910 0D 70 A0 E1                 MOV             R7, SP
__text:00002914 14 D0 4D E2                 SUB             SP, SP, #0x14
__text:00002918 70 05 01 E3                 MOV             R0, #0x1570
__text:0000291C 07 C0 A0 E3                 MOV             R12, #7
__text:00002920 00 00 40 E3                 MOVT            R0, #0
__text:00002924 04 20 A0 E3                 MOV             R2, #4
__text:00002928 00 00 8F E0                 ADD             R0, PC, R0
__text:0000292C 06 30 A0 E3                 MOV             R3, #6
__text:00002930 05 10 A0 E3                 MOV             R1, #5
__text:00002934 00 20 8D E5                 STR             R2, [SP,#0x1C+var_1C]
__text:00002938 0A 10 8D E9                 STMFA           SP, {R1,R3,R12}
__text:0000293C 08 90 A0 E3                 MOV             R9, #8
__text:00002940 01 10 A0 E3                 MOV             R1, #1
__text:00002944 02 20 A0 E3                 MOV             R2, #2
__text:00002948 03 30 A0 E3                 MOV             R3, #3
__text:0000294C 10 90 8D E5                 STR             R9, [SP,#0x1C+var_C]
__text:00002950 A4 05 00 EB                 BL              _printf
__text:00002954 07 D0 A0 E1                 MOV             SP, R7
__text:00002958 80 80 BD E8                 LDMFD           SP!, {R7,PC}
\end{lstlisting}

Почти то же самое что мы уже видели, за исключением того что \TT{STMFA} (Store Multiple Full Ascending) это синоним
инструкции \TT{STMIB} (Store Multiple Increment Before). 
Эта инструкция увеличивает \SP и только затем записывает в память очередной регистр, но не наоборот.

Второе что бросается в глаза, это то что инструкции как будто бы расположены случайно. Например, регистр R0
подготавливается в трех местах, по адресам 0x2918, 0x2920, 0x2928, когда это можно было бы сделать в одном месте.
Однако, у оптимизирующего компилятора могут быть свои доводы о том, как лучше составлять инструкции друг с другом
для лучшей эффективности исполнения.
Процессор обычно старается исполнять одновременно идущие друг за другом инструкции.
К примеру, инструкции \TT{``MOVT R0, \#0''} и \TT{``ADD R0, PC, R0''} не могут быть исполнены одновременно,
потому что обе инструкции модифицируют R0. 
А вот инструкции \TT{``MOVT R0, \#0''} и \TT{``MOV R2, \#4''} легко можно исполнить одновременно, 
потому что их действия никак не конфликтуют друг с другом. 
Вероятно, компилятор старается генерировать код именно таким образом, конечно, там где это возможно.
 
\subsubsection{\OptimizingXcode: \ThumbMode}

\begin{lstlisting}
__text:00002BA0                   _printf_main2
__text:00002BA0
__text:00002BA0                   var_1C          = -0x1C
__text:00002BA0                   var_18          = -0x18
__text:00002BA0                   var_C           = -0xC
__text:00002BA0
__text:00002BA0 80 B5                             PUSH            {R7,LR}
__text:00002BA2 6F 46                             MOV             R7, SP
__text:00002BA4 85 B0                             SUB             SP, SP, #0x14
__text:00002BA6 41 F2 D8 20                       MOVW            R0, #0x12D8
__text:00002BAA 4F F0 07 0C                       MOV.W           R12, #7
__text:00002BAE C0 F2 00 00                       MOVT.W          R0, #0
__text:00002BB2 04 22                             MOVS            R2, #4
__text:00002BB4 78 44                             ADD             R0, PC  ; char *
__text:00002BB6 06 23                             MOVS            R3, #6
__text:00002BB8 05 21                             MOVS            R1, #5
__text:00002BBA 0D F1 04 0E                       ADD.W           LR, SP, #0x1C+var_18
__text:00002BBE 00 92                             STR             R2, [SP,#0x1C+var_1C]
__text:00002BC0 4F F0 08 09                       MOV.W           R9, #8
__text:00002BC4 8E E8 0A 10                       STMIA.W         LR, {R1,R3,R12}
__text:00002BC8 01 21                             MOVS            R1, #1
__text:00002BCA 02 22                             MOVS            R2, #2
__text:00002BCC 03 23                             MOVS            R3, #3
__text:00002BCE CD F8 10 90                       STR.W           R9, [SP,#0x1C+var_C]
__text:00002BD2 01 F0 0A EA                       BLX             _printf
__text:00002BD6 05 B0                             ADD             SP, SP, #0x14
__text:00002BD8 80 BD                             POP             {R7,PC}
\end{lstlisting}

Почти то же самое что и в  предыдущем примере, лишь за тем исключением что здесь используются thumb-инструкции.



\subsection{\IFRU{Кстати}{By the way}}

\IFRU{Кстати, разница между способом передачи параметров принятая в x86 и ARM, неплохо иллюстрирует тот важный момент, что процессору, в общем, все равно как будут 
передаваться параметры функций. Можно создать гипотетический компилятор, который будет передавать их при 
помощи указателя на структуру с параметрами, не пользуясь стеком вообще.}
{By the way, this difference between passing arguments in x86 and ARM is a good illustration the CPU is not aware of how arguments is passed to functions. 
It is also possible to create hypothetical compiler which is able to pass arguments 
via a special structure not using stack at all.}


\section{scanf()}
\index{\CStandardLibrary!scanf()}
\label{label_scanf}

\IFRU{Теперь попробуем использовать scanf().}{Now let's use scanf().}

\begin{lstlisting}
int main() 
{
	int x;
	printf ("Enter X:\n");

	scanf ("%d", &x);

	printf ("You entered %d...\n", x);

	return 0;
};
\end{lstlisting}

\IFRU
{Да, согласен, использовать \scanf в наши времена для того чтобы спросить у пользователя что-то: 
не самая хорошая идея.
Но я хотел проиллюстрировать передачу указателя на \Tint.}
{OK, I agree, it is not clever to use \scanf today. But I wanted to illustrate passing pointer to \Tint.}

\subsection{\IFRU{Об указателях}{About pointers}}
\index{\CLanguageElements!\Pointers}

\IFRU{Это одна из фундаментальных вещей в компьютерных науках.}{It is one of the most fundamental things in computer
science.}
\IFRU{Часто, большой массив, структуру или объект, передавать в другую функцию никак не выгодно, 
а передать её адрес куда проще.}
{Often, large array, structure or object, it is too costly to pass to another function, 
while passing its address is much easier.}
\IFRU{К тому же, если вызываемая функция должна изменить что-то в этом большом массиве или структуре,
то возвращать её полностью это так же абсурдно.}
{More than that: if calling function should modify something in the large array or structure,
to return it as a whole is absurdical as well.}
\IFRU{Так что самое простое что можно сделать, это передать в функцию адрес массива или структуры,
и пусть она что-то там изменит.}
{So the most simple thing to do is to pass an address of array or structure to function,
and let it change what should be changed.}

\IFRU{Указатель в}{In} \CCpp \IFRU{это просто адрес какого-либо места в памяти.}
{it is just an address of some point in memory.}

\index{x86-64}
\IFRU{В x86 адрес представляется в виде 32-битного числа (т.е., занимает 4 байта), а в x86-64 как 64-битное число 
(занимает 8 байт).}
{In x86, address is represented as 32-bit number (i.e., occupying 4 bytes), while in x86-64 it is 64-bit number
(occupying 8 bytes).}
\IFRU{Кстати, отсюда негодование некоторых людей связанное с переходом на x86-64 ~--- на этой архитектуре все указатели
будут занимать места в 2 раза больше.}
{By the way, that is the reson of some people's indignation related to switching to x86-64 ~--- all pointers
on x64-architecture will require twice as more space.}

\index{\CStandardLibrary!memcpy()}
\IFRU{При некотором упорстве, можно работать только с бестиповыми указателями (\TT{void*})}{With some effort,
it is possible to work only with untyped pointers}, \IFRU{например}{for example}, 
\IFRU{стандартная функция}{standard function} \TT{memcpy()},
\IFRU{копирующая блок из одного места памяти в другое}{copying a block from one place in memory to another}, 
\IFRU{принимает на вход 2 указателя типа}{takes 2 pointers of} \TT{void*}\IFRU{}{ type on input}, 
\IFRU{потому что, нельзя
зараннее предугадать, какого типа блок вы собираетесь копировать, да в общем это и не важно, важно только знать размер
блока.}
{since it is not possible to predict block type you would like to copy, and it is not even important to know, 
only block size is important.}

\IFRU{Также, указатели широко используются когда функции нужно вернуть более одного значения}
{Also pointers are widely used when function needs to return more than one value}
(\IFRU{мы еще вернемся к этому в будущем}{we will back to this in future}~\ref{label_pointers}).
\IT{scanf()} \IFRU{это как раз такой случай}{is just that case}. 
\IFRU{Помимо того, что этой функции нужно показать, сколько значений
было прочитано успешно, ей еще и нужно вернуть сами значения.}
{In addition to the function's need to show how many values were read successfully, 
it also should return all these values.}

\IFRU{Тип указателя в}{In} \CCpp \IFRU{нужен для проверки типов на стадии компиляции.}
{pointer type is needed only for type checking on compiling stage.}
\IFRU{Внутри, в скомпилированном коде, никакой информации о типах указателей нет.}
{Internally, in compiled code, there is no information about pointers types.}

\subsection{x86: \IFRU{3 аргумента}{3 arguments}}

\subsubsection{MSVC}

\IFRU{Компилируем при помощи MSVC 2010 Express, и в итоге получим:}
{Let's compile it by MSVC 2010 Express and we got:}

\begin{lstlisting}
$SG3830	DB	'a=%d; b=%d; c=%d', 00H

...

	push	3
	push	2
	push	1
	push	OFFSET $SG3830
	call	_printf
	add	esp, 16					; 00000010H
\end{lstlisting}

\IFRU{Все почти то же, за исключением того, что теперь видно, что аргументы для \printf заталкиваются в стек в обратном порядке: самый первый аргумент заталкивается последним.}
{Almost the same, but now we can see the \printf arguments are pushing into stack in reverse order: and the first argument is pushing in as the last one.}

\IFRU{Кстати, вспомним что переменные типа \Tint в 32-битной системе, как известно, имеет ширину 32 бита, это 4 байта}
{By the way, variables of \Tint type in 32-bit environment has 32-bit width that is 4 bytes}.

\IFRU{Итак, у нас всего 4 аргумента. $4*4 = 16$ ~--- именно 16 байт занимают в стеке указатель на строку плюс еще 3 числа типа \Tint.}
{So, we got here 4 arguments. $4*4 = 16$~---they occupy exactly 16 bytes in the stack: 32-bit pointer to string and 3 number of \Tint type.}

\index{x86!\Instructions!ADD}
\index{x86!\Registers!ESP}
\index{cdecl}
\IFRU{Когда при помощи инструкции \TT{``ADD ESP, X''} корректируется \glslink{stack pointer}{указатель стека} \ESP 
после вызова какой-либо функции, зачастую можно сделать вывод о том, сколько аргументов 
у вызываемой функции было, разделив X на 4.}
{When \gls{stack pointer} (the \ESP register) is corrected by \TT{``ADD ESP, X''}
instruction after a function 
call, often, the number of function arguments could be deduced here: just divide X by 4.}

\IFRU{Конечно, это относится только к cdecl-методу передачи аргументов через стек.}
{Of course, this is related only to \IT{cdecl} calling convention.}

\IFRU{См. также в соответствующем разделе о способах передачи аргументов через стек}
{See also section about calling conventions}~(\ref{sec:callingconventions}).

\IFRU{Иногда бывает так, что подряд идут несколько вызовов разных функций, 
но стек корректируется только один раз, после последнего вызова:}
{It is also possible for compiler to merge several \TT{``ADD ESP, X''} instructions into one, after last call:}

\begin{lstlisting}
push a1
push a2
call ...
...
push a1
call ...
...
push a1
push a2
push a3
call ...
add esp, 24
\end{lstlisting}

\subsubsection{MSVC \AndENRU \olly}
\index{\olly}

\IFRU{Попробуем этот же пример в}{Now let's try to load this example in} \olly.
\IFRU{Это один из наиболее популярных win32-отладчиков user-режима}{It is one of the most 
popular user-land win32 debugger}.
\IFRU{Мы можем компилировать наш пример в}{We can try to compile our example in} MSVC 2012 
\IFRU{с опцией}{with} \TT{/MD} \IFRU{что означает, линковать с библиотекой}{option, meaning, to link 
against} \TT{MSVCR*.DLL},
\IFRU{чтобы импортируемые ф-ции были хорошо видны в отладчике}{so we will able to see imported 
functions clearly in debugger}.

\IFRU{Затем загружаем исполняемый файл в}{Then load executable in} \olly.
\IFRU{Самый первый брякпойнт в}{The very first breakpoint is in} \TT{ntdll.dll}, \IFRU{нажмите}{press} 
F9 (\IFRU{запустить}{run}).
\IFRU{Второй брякпойнт в}{The second breakpoint is in} \ac{CRT}-\IFRU{коде}{code}.
\IFRU{Теперь мы должны найти ф-цию}{Now we should find} \main\EN{ function}.

\IFRU{Найдите этот код скроллируя окно кода до самого верха (MSVC располагает ф-цию \main в самом начале
секции кода)}{Find this code by scrolling the code to the very bottom (MSVC allocates \main function at
the very beginning of the code section)}: 
\figname \ref{fig:printf3_olly_1}.

\IFRU{Кликните на инструкции}{Click on} \TT{PUSH EBP}\IFRU{, нажмите}{ instruction, press} F2 
(\IFRU{установка брякпойнта}{set breakpoint}) \IFRU{и нажмите}{and press} F9 (\IFRU{запустить}{run}).
\IFRU{Нам нужно произвести все эти манипуляции, чтобы пропустить \ac{CRT}-код, потому что нам он пока
не интересен}{We need to do these manupulations in order to skip \ac{CRT}-code, because, we don't really 
interesting in it yet}.

\IFRU{Нажмите}{Press} F8 (\stepover) 6 \IFRU{раз, т.е., пропустить
6 инструкций}{times, i.e., skip 6 instructions}: \figname \ref{fig:printf3_olly_2}.

\IFRU{Теперь}{Now the} \PC \IFRU{указывает на инструкцию}{points to the}
\TT{CALL printf}\EN{ instruction}.
\olly, \IFRU{как и другие отладчики, подсвечивает регистры со значениями, которые изменились}
{like other debuggers, highlights value of registers which were changed}.
\IFRU{Так что, каждый раз, когда мы нажимаем}{So each time you press F8}, \EIP 
\IFRU{изменяется и его значение подсвечивается красным}{is changing and its value looking red}.
\ESP \IFRU{также меняется, потому что значения заталкиваются в стек}{is changing as well, 
because values are pushed into the stack}.

\IFRU{Где находятся эти значения в стеке}{Where are the values in the stack}?
\IFRU{Посмотрите на правое/нижнее окно в отладчике}{Take a look into right/bottom window of debugger}:

\begin{figure}[H]
\centering
\includegraphics[scale=0.66]{patterns/03_printf/olly3_stack.png}
\caption{\olly: \IFRU{стек, после того как значения там сохранены}{stack after values pushed}
(\IFRU{я сделал здесь округлую красную пометку в графическом редакторе}{I made round red mark 
here in graphics editor})}
\end{figure}

\IFRU{Так что здесь видно 3 столбца: адрес в стеке, значение в стеке и еще дополнительный комментарий
от \olly}{So we can see there 3 columns: address in the stack, 
value in the stack and some additional \olly comments}. 
\olly \IFRU{понимает}{understands} \printf\IFRU{-строки}{-like strings}, 
\IFRU{так что он показывает здесь и строку и 3 значения \IT{привязанных} к ней}{so it reports the 
string here and 3 values \IT{attached} to it}.

\IFRU{Нажмите}{Press} F8 (\stepover).

\IFRU{В коносил мы видим вывод}{In the console we'll see the output}:

\begin{figure}[H]
\centering
\includegraphics[scale=0.66]{patterns/03_printf/olly3_console.png}
\caption{\RU{Ф-ция }\printf \IFRU{исполнилась}{function executed}}
\end{figure}

\IFRU{Посмотрим, как изменились регистры и состояние стека}{Let's see how registers and stack state 
are changed}: \figname \ref{fig:printf3_olly_3}.

\RU{Регистр }\EAX \IFRU{теперь содержит}{register now contains} \TT{0xD} (13).
That's correct, \printf returns number of characters printed.
\RU{Значение }\EIP \IFRU{изменилось: действительно, теперь здесь адрес инструкции после}
{value is changed: indeed, now there is address of the instruction after} \TT{CALL printf}.
\RU{Значения регистров }\ECX \AndENRU \EDX \IFRU{также изменились}{values are changed as well}.
\IFRU{Очевидно, внутренности ф-ции \printf используют их для каких-то своих нужд}{Apparently, 
\printf function's hidden machinery used them for its own needs}.

\IFRU{Очень важный момент в том что значение \ESP не изменилось. И состояние стека также!}
{A very important thing is that \ESP value is not changed. And stack state too!}
\IFRU{Мы ясно видим здесь и строку формата и соответствующие ей 3 значения, они все еще здесь}
{We clearly see that format string and corresponding 3 values are still there}.
\IFRU{Действительно, по соглашению вызовов \IT{cdecl}, вызывающая ф-ция не очищает аргументы из стека}
{Indeed, that's \IT{cdecl} calling convention, calling function doesn't clear arguments in stack}.
\IFRU{Это должна делать вызывающая ф-ция}{It's caller's duty to do so}.

\IFRU{Нажмите}{Press} F8 \IFRU{снова, чтобы исполнилась инструкция}{again to execute} 
\TT{ADD ESP, 10}\EN{ instruction}: \figname \ref{fig:printf3_olly_4}.

\ESP \IFRU{изменился, но значения все еще в стеке}{is changed, but values are still in the stack}!
\IFRU{Конечно, никому не нужно заполнять эти значения нулями или что-то в этом роде}{Yes, 
of course, no one needs to fill these values by zero or something like that}.
\IFRU{Потому что всё что выше указателя стека}{Because, everything above stack pointer} (\SP) 
\IFRU{это}{is} \IT{\IFRU{шум}{noise}} \OrENRU \IT{\IFRU{мусор}{garbage}}, \IFRU{это всё не имеет
особой ценности}{it has no value at all}.
\IFRU{Было бы очень затратно по времени очищать ненужные элементы стека, к тому же, никому это и не 
нужно}{It would be time consuming to clear unused stack entries, besides, no one really needs to}.

\begin{figure}[H]
\centering
\includegraphics[scale=0.66]{patterns/03_printf/olly3_1.png}
\caption{\olly: \IFRU{самое начало ф-ции}{the very start of the} \main\EN{ function}}
\label{fig:printf3_olly_1}
\end{figure}

\begin{figure}[H]
\centering
\includegraphics[scale=0.66]{patterns/03_printf/olly3_2.png}
\caption{\olly: \IFRU{перед исполнением}{before} \printf\EN{ execution}}
\label{fig:printf3_olly_2}
\end{figure}

\begin{figure}[H]
\centering
\includegraphics[scale=0.66]{patterns/03_printf/olly3_3.png}
\caption{\olly: \IFRU{после исполнения}{after} \printf\EN{ execution}}
\label{fig:printf3_olly_3}
\end{figure}

\begin{figure}[H]
\centering
\includegraphics[scale=0.66]{patterns/03_printf/olly3_4.png}
\caption{\olly: \IFRU{после исполнения инструкции}{after} \TT{ADD ESP, 10}\EN{ instruction execution}}
\label{fig:printf3_olly_4}
\end{figure}

\subsubsection{GCC}

\IFRU{Скомпилируем то же самое в Linux при помощи GCC 4.4.1 и посмотрим в \IDA что вышло:}
{Now let's compile the same in Linux by GCC 4.4.1 and take a look in \IDA what we got:}

\begin{lstlisting}
main            proc near

var_10          = dword ptr -10h
var_C           = dword ptr -0Ch
var_8           = dword ptr -8
var_4           = dword ptr -4

                push    ebp
                mov     ebp, esp
                and     esp, 0FFFFFFF0h
                sub     esp, 10h
                mov     eax, offset aADBDCD ; "a=%d; b=%d; c=%d"
                mov     [esp+10h+var_4], 3
                mov     [esp+10h+var_8], 2
                mov     [esp+10h+var_C], 1
                mov     [esp+10h+var_10], eax
                call    _printf
                mov     eax, 0
                leave
                retn
main            endp
\end{lstlisting}

\IFRU{Можно сказать, что этот короткий код, созданный GCC, отличается от кода MSVC только способом помещения 
значений в стек.
Здесь GCC снова работает со стеком напрямую без \PUSH/\POP.}
{It can be said, the difference between code by MSVC and GCC is only in method of placing arguments on the stack.
Here GCC working directly with stack without \PUSH/\POP.}

\section{ARM}

\subsection{\NonOptimizingXcode + \ARMMode}

\lstinputlisting[caption=\NonOptimizingXcode + \ARMMode]{patterns/10_strlen/xcode_ARM_O0_en.asm}

\IFRU{Неоптимизирующий LLVM генерирует слишком много кода, зато на этом примере можно посмотреть, 
как функции работают с локальными переменными в стеке.}
{Non-optimizing LLVM generates too much code, however, here we can see how function works with 
local variables in the stack.}
\IFRU{В нашей функции только локальных переменных две, это два указателя}
{There are only two local variables in our function},
\IT{eos} \AndENRU \IT{str}.

\IFRU{В этом листинге}{In this listing}, \IFRU{сгенерированном при помощи}{generated by} \IDA, 
\IFRU{я переименовал}{I renamed} \IT{var\_8} \AndENRU \IT{var\_4} \IFRU{в}{into} \IT{eos} 
\AndENRU \IT{str} \IFRU{вручную}{manually}.

\IFRU{Итак, первые несколько инструкций просто сохраняют входное значение в переменных}{So, 
first instructions are just saves input value in} \IT{str} \AndENRU \IT{eos}.

\IFRU{Начиная с метки}{Loop body is beginning at} \IT{loc\_2CB8}\IFRU{, начинается тело цикла}{ label}.

\IFRU{Первые три инструкции в теле цикла}{First three instruction in loop body} (\TT{LDR}, \ADD, \TT{STR}) 
\IFRU{загружают значение}{loads} \IT{eos} \IFRU{в}{value into} \Reg{0}, 
\IFRU{затем происходит инкремент значения и оно сохраняется назад в локальной переменной \IT{eos} расположенной 
в стеке.}{then value is \glslink{increment}{incremented} and it is saved back into \IT{eos} local variable located in the stack.}

\index{ARM!\Instructions!LDRSB}
\IFRU{Следующая инструкция}{The next} \TT{``LDRSB R0, [R0]''} (\IT{Load Register Signed Byte}) 
\IFRU{загружает байт из памяти по адресу \Reg{0}, расширяет его до 32-бит считая его знаковым (signed) 
и сохраняет в \Reg{0}}{instruction loading byte from memory at \Reg{0} address and sign-extends it to 32-bit}.
\index{x86!\Instructions!MOVSX}
\IFRU{Это немного похоже на инструкцию}{This is similar to} \MOVSX \IFRU{в}{instruction in} x86.
\IFRU{Компилятор считает этот байт знаковым (signed), потому что тип \Tchar по стандарту Си ~--- знаковый.}
{The compiler treating this byte as signed since \Tchar type in C standard is signed.}
\IFRU{Об это я уже немного писал}{I already wrote about it}~(\ref{MOVSX}) \IFRU{в этой же секции, 
но посвященной x86}{in this section, but related to x86}.

\index{x86!8086}
\index{8080}
\index{ARM}
\IFRU{Следует также заметить, что, в ARM нет возможности использовать 8-битную или 16-битную часть 
регистра, как это возможно в x86.}
{It is should be noted, it is impossible in ARM to use 8-bit part or 16-bit part 
of 32-bit register separately of the whole register,
as it is in x86.}
\IFRU{Вероятно, это связано с тем что за x86 тянется длинный шлейф совместимости со своими предками, 
такими как
16-битный 8086 и даже 8-битный 8080, а ARM разрабатывался с чистого листа как 32-битный RISC-процессор.}
{Apparently, it is because x86 has a huge history of compatibility with its ancestors like 16-bit 8086 
and even 8-bit 8080,
but ARM was developed from scratch as 32-bit RISC-processor.}
\IFRU{Следовательно, чтобы работать с отдельными байтами на ARM, так или иначе, придется использовать 
32-битные регистры.}
{Consequently, in order to process separate bytes in ARM, one have to use 32-bit registers anyway.}

\IFRU{Итак}{So}, \TT{LDRSB} \IFRU{загружает символ из строки в \Reg{0}, по одному}
{loads symbol from string into \Reg{0}, one by one}.
\IFRU{Следующие инструкции}{Next} \CMP \AndENRU \ac{BEQ} \IFRU{проверяют, является ли этот символ $0$.}
{instructions checks, if loaded symbol is $0$.}
\IFRU{Если не $0$, то происходит переход на начало тела цикла.}{If not $0$, control passing to loop body
begin.}
\IFRU{А если $0$, выходим из цикла.}{And if $0$, loop is finishing.}

\IFRU{В конце функции вычисляется разница между}{At the end of function, a difference between} 
\IT{eos} \AndENRU \IT{str}\IFRU{, вычитается еще единица и вычисленное 
значение возвращается через \Reg{0}.}{ is calculated, 1 is also subtracting, and resulting value is returned
via \Reg{0}.}

N.B. \IFRU{В этой функции не сохранялись регистры}{Registers was not saved in this function}.
\index{ARM!\Registers!scratch registers}
\IFRU{Это потому что, по стандарту, регистры \Reg{0}-\Reg{3} называются также ``scratch registers'',
они предназначены для передачи аргументов, 
их значения не нужно восстанавливать при выходе из функции, потому что они больше не нужны в вызывающей функции.
Таким образом, их можно использовать как захочется}
{That's because by ARM calling convention, \Reg{0}-\Reg{3} registers are ``scratch registers'', 
they are intended for arguments passing,
its values may not be restored upon function exit since calling function will not use them anymore.
Consequently, they may be used for anything we want.}
\IFRU{А так как никакие больше регистры не используются, то и сохранять нечего.}
{Other registers are not used here, so that is why we have nothing to save on the stack.}
\IFRU{Поэтому, управление можно вернуть назад вызывающей функции 
простым переходом (\TT{BX}), по адресу в регистре \LR.}
{Thus, control may be returned back to calling function by simple jump (\TT{BX}),
to address in the \LR register.}

%\subsection{\NonOptimizingXcode + режим thumb}
%Практически, точно такой же код.

\subsection{\OptimizingXcode + \ThumbMode}

\lstinputlisting[caption=\OptimizingXcode + \ThumbMode]{patterns/10_strlen/xcode_thumb_O3.asm}

\IFRU{Оптимизирующий LLVM решил, что под переменные \IT{eos} и \IT{str} выделять место в стеке не обязательно}
{As optimizing LLVM concludes, space on the stack for \IT{eos} and \IT{str} may not be allocated},
\IFRU{и эти переменные можно хранить прямо в регистрах.}
{and these variables may always be stored right in registers.}
\IFRU{Перед началом тела цикла}{Before loop body beginning}, \IT{str} \IFRU{будет находиться в}{will always be in} 
\Reg{0}, \IFRU{а}{and} \IT{eos}\EMDASH\InENRU \Reg{1}.

\index{ARM!\Instructions!LDRB.W}
\index{ARM!\IFRU{Режимы адресации}{Adressing modes}}
\RU{Инструкция }\TT{``LDRB.W R2, [R1],\#1''} \IFRU{загружает в \Reg{2} байт из памяти по адресу \Reg{1}, 
расширяя его как знаковый (signed), до 32-битного
значения, но не только это.}
{instruction loads byte from memory at the address \Reg{1} into \Reg{2}, sign-extending it to 32-bit value, but not
only that.}
\TT{\#1} \IFRU{в конце инструкции называется}{at the instruction's end calling} ``Post-indexed addressing'', 
\IFRU{это значит, что после загрузки байта, к \Reg{1} добавится единица.}{this means, $1$ is to be added
to the \Reg{1} after byte load.}
\IFRU{Это очень удобно для работы с массивами.}
{That's convenient when accessing arrays.}

\index{PDP-11}
\index{\CLanguageElements!\PostIncrement}
\index{\CLanguageElements!\PostDecrement}
\index{\CLanguageElements!\PreIncrement}
\index{\CLanguageElements!\PreDecrement}
\IFRU{Такого режима адресации в x86 нет, но он есть в некоторых других процессорах, даже на PDP-11.}
{There is no such addressing mode in x86, but it is present in some other processors, even on PDP-11.}
\IFRU{Существует байка, что режимы пре-инкремента, пост-инкремента, 
пре-декремента и пост-декремента адреса в PDP-11}
{There is a legend the pre-increment, post-increment, pre-decrement and post-decrement modes in PDP-11},
\IFRU{были ``виновны'' в появлении таких конструкций языка Си (который разрабатывался на PDP-11) как}
{were ``guilty'' in appearance such C language (which developed on PDP-11) constructs as}
*ptr++, *++ptr, *ptr-{}-, *-{}-ptr. 
\IFRU{Кстати, это является труднозапоминаемой особенностью в Си.}
{By the way, this is one of hard to memorize C feature.}
\IFRU{Дела обстоят так:}{This is how it is:}

\begin{center}
\begin{tabular}{ | l | l | l | l | }
\hline
\headercolor{} \IFRU{термин в Си}{C term} & 
\headercolor{} \IFRU{термин в ARM}{ARM term} & 
\headercolor{} \IFRU{выражение Си}{C statement} & 
\headercolor{} \IFRU{как это работает}{how it works} \\
\hline
\PostIncrement & 
post-indexed addressing & 
\TT{*ptr++} & 
\IFRU{использовать значение \TT{*ptr}}{use \TT{*ptr} value}, \\
& & & \IFRU{затем инкремент указателя \TT{ptr}}{then \gls{increment} \TT{ptr} pointer} \\
\hline
\PostDecrement & 
post-indexed addressing & 
\TT{*ptr-{}-} & 
\IFRU{использовать значение \TT{*ptr}}{use \TT{*ptr} value}, \\
& & & \IFRU{затем \glslink{decrement}{декремент} указателя \TT{ptr}}{then \gls{decrement} \TT{ptr} pointer} \\
\hline
\PreIncrement & 
pre-indexed addressing & 
\TT{*++ptr} & 
\IFRU{инкремент указателя \TT{ptr}}{\gls{increment} \TT{ptr} pointer}, \\
& & & \IFRU{затем использовать значение \TT{*ptr}}{then use \TT{*ptr} value} \\
\hline
\PreDecrement & 
post-indexed addressing & 
\TT{*-{}-ptr} & 
\IFRU{\glslink{decrement}{декремент} указателя \TT{ptr}}{\gls{decrement} \TT{ptr} pointer}, \\
& & & \IFRU{затем использовать значение \TT{*ptr}}{then use \TT{*ptr} value} \\
\hline
\end{tabular}
\end{center}

\IFRU{Деннис Ритчи (один из создателей ЯП Си) указывал, что, это, вероятно, придумал Кен Томпсон 
(еще один создатель Си),
потому что подобная возможность процессора имелась еще в PDP-7}
{Dennis Ritchie (one of C language creators) mentioned that it is, probably, was invented by Ken Thompson
(another C creator) because this processor feature was present in PDP-7}
\cite{Ritchie:1986}\cite{Ritchie:1993:DCL:155360.155580}.
\IFRU{Таким образом, компиляторы с ЯП Си на тот процессор, где это есть, могут использовать это.}
{Thus, C language compilers may use it, if it is present in target processor.}

\IFRU{Далее в теле цикла можно увидеть \CMP и \ac{BNE}, они продолжают работу цикла до тех пор, 
пока не будет встречен $0$.}
{Then one may spot \CMP and \ac{BNE} in loop body, these instructions continue operation until
$0$ will be met in string.}

\index{ARM!\Instructions!MVNS}
\index{x86!\Instructions!NOT}
\RU{После конца цикла }\TT{MVNS}\footnote{MoVe Not} 
\IFRU{(инвертирование всех бит, аналог \NOT на x86)}
{(inverting all bits, \NOT in x86 analogue)}
\IFRU{и \ADD вычисляют}{instructions and \ADD computes} $eos - str - 1$.
\IFRU{На самом деле, эти две инструкции вычисляют}
{In fact, these two instructions computes}
$R0 = ~str + eos$, 
\IFRU{что эквивалентно тому, что было в исходном коде, а почему это так, я уже описывал чуть раньше, здесь}
{which is effectively equivalent to what was in source code, and why it is so, I already described here}
~(\ref{strlen_NOT_ADD}).

\IFRU{Вероятно, LLVM, как и GCC, посчитал что такой код будет короче, или быстрее.}
{Apparently, LLVM, just like GCC, concludes this code will be shorter, or faster.}

%\subsection{\OptimizingXcode + \ARMMode}
%Практически, точно такой же код.

\subsection{\OptimizingKeil{} + \ARMMode}

\lstinputlisting[caption=\OptimizingKeil + \ARMMode]{patterns/10_strlen/Keil_ARM_O3.asm}

\index{ARM!\Instructions!SUBEQ}
\IFRU{Практически то же самое что мы уже видели, за тем исключением что выражение}
{Almost the same what we saw before, with the exception the}
$str - eos - 1$ 
\IFRU{может быть вычислено не в самом конце функции, а прямо в теле цикла.}
{expression may be computed not at the function's end, but right in loop body.}
\RU{Суффикс }\TT{-EQ}\IFRU{, как мы помним, означает что инструкция будет выполнена только
если операнды в исполненной перед этим инструкции \CMP были равны.}
{suffix, as we may recall, means the instruction will be executed only if operands in executed before
\CMP were equal to each other.}
\IFRU{Таким образом}{Thus}, \IFRU{если в \Reg{0} будет $0$}{if $0$ will be in the \Reg{0} register},
\IFRU{обе инструкции}{both} \TT{SUBEQ} \IFRU{исполнятся и результат останется в \Reg{0}.}
{instructions are to be executed and result is leaving in the \Reg{0} register.}


\subsection{\IFRU{Глобальные переменные}{Global variables}}
\index{\IFRU{Глобальные переменные}{Global variables}}
\subsubsection{x86}

\IFRU
{А что если переменная \TT{x} из предыдущего примера будет глобальной переменной а не локальной? 
Тогда к ней смогут обращаться из любого другого места, а не только из тела функции. 
Это снова не очень хорошая практика программирования, но ради примера мы можем себе это позволить.}
{What if \TT{x} variable from previous example will not be local but global variable? 
Then it will be accessible from any point, not only from function body. 
It is not very good programming practice, but for the sake of experiment we could do this.}

\lstinputlisting{04_scanf/4_2_msvc.asm}

\IFRU
{Ничего особенного, в целом. Теперь \TT{x} объявлена в сегменте \TT{\_DATA}. 
Память для нее в стеке более не выделяется. Все обращения к ней происходит не через стек, а уже напрямую. 
Её значение неопределено. 
Это означает, что память под нее будет выделена, но ни компилятор, ни \ac{ОС} не будет заботиться о том, 
что там будет лежать на момент старта функции \main.
В качестве домашнего задания, попробуйте объявить большой неопределенный массив и посмотреть 
что там будет лежать после загрузки.}
{Now \TT{x} variable is defined in the \TT{\_DATA} segment. 
Memory in local stack is not allocated anymore. 
All accesses to it are not via stack but directly to process memory. 
Its value is not defined. 
This means that memory will be allocated by \ac{OS}, but not compiler, 
neither \ac{OS} will not take care about its initial value at the moment of 
the \main function start.
As experiment, try to declare large array and see what will it contain after 
program loading.}

\IFRU{Попробуем изменить объявление этой переменной:}
{Now let's assign value to variable explicitly:}

\begin{lstlisting}
int x=10; // default value
\end{lstlisting}

\IFRU{Выйдет в итоге:}{We got:}

\begin{lstlisting}
_DATA	SEGMENT
_x	DD	0aH

...
\end{lstlisting}

\IFRU{Здесь уже по месту этой переменной записано \TT{0xA} с типом DD (dword = 32 бита).}
{Here we see value \TT{0xA} of DWORD type (DD meaning DWORD = 32 bit).}

\IFRU{Если вы откроете скомпилированный .exe-файл в \IDA, то увидите что \IT{x} 
находится аккурат в начале сегмента \TT{\_DATA}, после этой переменной будут текстовые строки.}
{If you will open compiled .exe in \IDA, you will see the \IT{x} variable placed at the beginning of 
the \TT{\_DATA} segment, and after you'll see text strings.}

\IFRU{А вот если вы откроете в \IDA, .exe скомплированный в прошлом примере, 
где значение \IT{x} неопределено, то в IDA вы увидите:}
{If you will open compiled .exe in \IDA from previous example where \IT{x} value is not defined, 
you'll see something like this:}

\begin{lstlisting}
.data:0040FA80 _x              dd ?                    ; DATA XREF: _main+10
.data:0040FA80                                         ; _main+22
.data:0040FA84 dword_40FA84    dd ?                    ; DATA XREF: _memset+1E
.data:0040FA84                                         ; unknown_libname_1+28
.data:0040FA88 dword_40FA88    dd ?                    ; DATA XREF: ___sbh_find_block+5
.data:0040FA88                                         ; ___sbh_free_block+2BC
.data:0040FA8C ; LPVOID lpMem
.data:0040FA8C lpMem           dd ?                    ; DATA XREF: ___sbh_find_block+B
.data:0040FA8C                                         ; ___sbh_free_block+2CA
.data:0040FA90 dword_40FA90    dd ?                    ; DATA XREF: _V6_HeapAlloc+13
.data:0040FA90                                         ; __calloc_impl+72
.data:0040FA94 dword_40FA94    dd ?                    ; DATA XREF: ___sbh_free_block+2FE
\end{lstlisting}

\IFRU{\TT{\_x} обозначен как \TT{?}, наряду с другими переменными не требующими инициализции. 
Это означает, что при загрузке .exe в память, место под все это выделено будет. 
Но в самом .exe ничего этого нет. Неинициализированные переменные не занимают места в исполняемых файлах. Удобно для больших массивов, например.}
{\TT{\_x} marked as \TT{?} among another variables not required to be initialized. 
This means that after loading .exe to memory, a space for all these variables will be 
allocated and some random garbage will be here. 
But in an .exe file these not initialized variables are not occupy anything. 
It is suitable for large arrays, for example.}

\index{ELF}
\IFRU{В Linux все также почти. За исключением того что если значение \TT{x} не определено, 
то эта переменная будет находится в сегменте \TT{\_bss}. В ELF\footnote{Формат исполняемых файлов, использующийся в Linux и некоторых других *NIX} этот сегмент имеет такие аттрибуты:}
{It is almost the same in Linux, except segment names and properties: 
not initialized variables are located in the \TT{\_bss} segment. 
In ELF\footnote{Executable file format widely used in *NIX system including Linux} 
file format this segment has such attributes:}

\begin{lstlisting}
; Segment type: Uninitialized
; Segment permissions: Read/Write
\end{lstlisting}

\IFRU{Ну а если сделать присвоение этой переменной значения $10$, то она будет находится 
в сегменте \TT{\_data},
это сегмент с такими аттрибутами:}
{If to assign some value to variable, e.g. $10$, it will be placed in the \TT{\_data} segment, 
this is segment with such attributes:}

\begin{lstlisting}
; Segment type: Pure data
; Segment permissions: Read/Write
\end{lstlisting}

\subsubsection{ARM: \OptimizingKeil + \ThumbMode}

\begin{lstlisting}
.text:00000000 ; Segment type: Pure code
.text:00000000                 AREA .text, CODE
...
.text:00000000 main
.text:00000000                 PUSH    {R4,LR}
.text:00000002                 ADR     R0, aEnterX     ; "Enter X:\n"
.text:00000004                 BL      __2printf
.text:00000008                 LDR     R1, =x
.text:0000000A                 ADR     R0, aD          ; "%d"
.text:0000000C                 BL      __0scanf
.text:00000010                 LDR     R0, =x
.text:00000012                 LDR     R1, [R0]
.text:00000014                 ADR     R0, aYouEnteredD___ ; "You entered %d...\n"
.text:00000016                 BL      __2printf
.text:0000001A                 MOVS    R0, #0
.text:0000001C                 POP     {R4,PC}
...
.text:00000020 aEnterX         DCB "Enter X:",0xA,0    ; DATA XREF: main+2
.text:0000002A                 DCB    0
.text:0000002B                 DCB    0
.text:0000002C off_2C          DCD x                   ; DATA XREF: main+8
.text:0000002C                                         ; main+10
.text:00000030 aD              DCB "%d",0              ; DATA XREF: main+A
.text:00000033                 DCB    0
.text:00000034 aYouEnteredD___ DCB "You entered %d...",0xA,0 ; DATA XREF: main+14
.text:00000047                 DCB 0
.text:00000047 ; .text         ends
.text:00000047
...
.data:00000048 ; Segment type: Pure data
.data:00000048                 AREA .data, DATA
.data:00000048                 ; ORG 0x48
.data:00000048                 EXPORT x
.data:00000048 x               DCD 0xA                 ; DATA XREF: main+8
.data:00000048                                         ; main+10
.data:00000048 ; .data         ends
\end{lstlisting}

Итак, переменная \TT{x} теперь глобальная, и она расположена, почему-то, в другом сегменте данных (\IT{.data}). 
Можно спросить, почему текстовые строки расположены в сегменте кода (\IT{.text}) а \TT{x} нельзя было разместить
тут же? Потому что эта переменная, и как следует из определения, она может меняться. Сегмент кода нередко может 
быть расположен в ПЗУ микроконтроллера (не забывайте, мы сейчас имеем дело с embedded-микроэлектроникой),
а изменяемые переменные --- в ОЗУ.
Нередко, ОЗУ дороже чем ПЗУ, так что хранить в нем неизменяемые данные, когда в наличии есть ПЗУ, не экономно.

Далее, мы видим, в сегменте кода, хранится указатель на переменную \TT{x} (\TT{off\_2C}) и вообще, все операции 
с ним, происходят через этот указатель.
Это связано с тем что переменная \TT{x} может быть расположена где-то довольно далеко от данного участка кода
и её адрес нужно сохранить в переменной рядом с кодом.
Инструкция \TT{LDR} в thumb-режиме может адресовать только переменные в пределах вплоть до 1020 байт от места
где она находится. Эта же инструкция в ARM-режиме --- переменные в пределах $\pm{}4095$, таким образом, 
адрес глобальной переменной \TT{x} нужно иметь где-то рядом, ведь нет никакой гарантии, что саму переменную
получится хранить где-то рядом, она может быть даже в другом чипе памяти!

Еще одна вещь: если переменную объявить как \IT{const}, то компилятор Keil разместит её в сегменте \TT{.constdata}.
Должно быть, впоследствии, линкер и этот сегмент сможет разместить в ПЗУ.




\subsection{\IFRU{Проверка результата scanf()}{scanf() result checking}}

\subsubsection{x86}

\IFRU {Как я уже упоминал, использовать \scanf в наше время это слегка старомодно. 
Но если уж жизнь заставила этим заниматься, нужно хотя бы проверять, сработал ли \scanf 
правильно или пользователь ввел вместо числа что-то другое, что \scanf не смог трактовать как число.}
{As I noticed before, it is slightly old-fashioned to use \scanf today. 
But if we have to, we need at least check if \scanf finished correctly without error.}

\lstinputlisting{04_scanf/retval_check.c}

\IFRU{По стандарту}{By standard}, \scanf\footnote{\href{http://msdn.microsoft.com/en-us/library/9y6s16x1(VS.71).aspx}{MSDN: scanf, wscanf}} 
\IFRU{возвращает количество успешно полученных значений.}{function returns number of fields it successfully read.}

\IFRU{В нашем случае, если все успешно и пользователь ввел таки некое число, \scanf вернет 1. 
А если нет, то 0 или EOF.} 
{In our case, if everything went fine and user entered a number, 
\scanf will return 1 or 0 or EOF in case of error.}

\IFRU{Я добавил код проверяющий результат \scanf и в случае ошибки, он сообщает пользователю что-то другое.}
{I added C code for \scanf result checking and printing error message in case of error.}

\IFRU{Вот, что выходит на ассемблере}{What we got in assembly language} (MSVC 2010):

\lstinputlisting{04_scanf/retval_check_MSVC.asm}

\index{x86!\Registers!EAX}
\IFRU{Для того чтобы вызывающая функция имела доступ к результату вызываемой функции, 
вызываемая функция (в нашем случае \scanf) оставляет это значение в регистре \EAX.}
{Caller function (\main) must have access to the result of callee function (\scanf), 
so callee leaves this value in the \EAX register.}

\index{x86!\Instructions!CMP}
\IFRU{Мы проверяем его инструкцией \TT{CMP EAX, 1} (\IT{CoMPare}), то есть, 
сравниваем значение в \EAX с 1.}
{After, we check it with the help of instruction \TT{CMP EAX, 1} (\IT{CoMPare}),
in other words, we compare value in the \EAX register with $1$.} 

\index{x86!\Instructions!JNE}
\IFRU{Следующий за инструкцией \CMP: условный переход \JNE. 
Это означает \IT{Jump if Not Equal}, то есть, условный переход \IT{если не равно}.}
{\JNE conditional jump follows \CMP instruction. \JNE means \IT{Jump if Not Equal}.}

\IFRU{Итак, если \EAX не равен 1, то \JNE заставит перейти процессор 
по адресу указанном в операнде \JNE, у нас это \TT{\$LN2@main}.}
{So, if value in the \EAX register not equals to $1$, then the processor will pass execution to the 
address mentioned in operand of \JNE, in our case it is \TT{\$LN2@main}.}
\IFRU
{Передав управление по этому адресу, процессор как раз начнет исполнять вызов \printf с 
аргументом \TT{``What you entered? Huh?''}.}
{Passing control to this address, microprocesor will execute function \printf 
with argument \TT{``What you entered? Huh?''}.}
\IFRU
{Но если все нормально, перехода не случится, и исполнится другой \printf с двумя аргументами: 
\TT{'You entered \%d...'} и значением переменной \TT{x}.}
{But if everything is fine, conditional jump will not be taken, and another \printf call 
will be executed, with two arguments: \TT{'You entered \%d...'} and value of variable \TT{x}. }

\index{x86!\Instructions!XOR}
\index{\CLanguageElements!return}
\IFRU {А для того чтобы после этого вызова не исполнился сразу второй вызов \printf, 
после него имеется инструкция \JMP, безусловный переход, он отправит процессор на место аккурат 
после второго \printf и перед инструкцией \TT{XOR EAX, EAX}, которая собственно \TT{return 0}.}
{Since second subsequent \printf not needed to be executed, there is \JMP after (unconditional jump),
it will pass control to the point after second \printf and before \TT{XOR EAX, EAX} instruction, 
which implement \TT{return 0}.}

\index{x86!\Registers!\Flags}
\IFRU{Итак, можно сказать, что в подавляющих случаях сравнение какой либо переменной с чем-то другим 
происходит при помощи пары инструкций \CMP и \Jcc, где \IT{cc} это \IT{condition code}.}
{So, it can be said that comparing a value with another is \IT{usually} implemented
by \CMP/\Jcc instructions pair, where \IT{cc} is \IT{condition code}.}
\IFRU{\CMP сравнивает два значения и выставляет 
флаги процессора\footnote{См.также о флагах x86-процессора: \url{http://en.wikipedia.org/wiki/FLAGS_register_(computing)}.}.}
{\CMP comparing two values and set 
processor flags\footnote{About x86 flags, see also: \url{http://en.wikipedia.org/wiki/FLAGS_register_(computing)}.}.}
\IFRU
{\Jcc проверяет нужные ему флаги и выполняет переход по указанному адресу (или не выполняет).}
{\Jcc check flags needed to be checked and pass control to mentioned address (or not pass).}

\index{x86!\Instructions!CMP}
\index{x86!\Instructions!SUB}
\label{CMPandSUB}
\IFRU{Но на самом деле, как это не парадоксально поначалу звучит, \CMP это почти то же самое что и 
инструкция \SUB, которая отнимает числа одно от другого.}
{But in fact, this could be perceived paradoxical, but \CMP instruction is in fact \SUB (subtract).}
\IFRU{Все арифметические инструкции также выставляют флаги в соответствии с результатом, не только \CMP.}
{All arithmetic instructions set processor flags too, not only \CMP.}
\IFRU{Если мы сравним 1 и 1, от единицы отнимется единица, получится $0$, и выставится флаг 
\ZF (\IT{zero flag}), означающий что последний полученный результат был $0$.}
{If we compare 1 and 1, $1-1$ will be $0$ in result, \ZF flag will be set (meaning the last result was $0$).}
\IFRU{Ни при каких других значениях \EAX, флаг \ZF выставлен не будет, кроме тех, когда операнды равны друг другу.}
{There is no any other circumstance when it is possible except when operands are equal.}
\index{x86!\Instructions!JNE}
\index{x86!\Registers!ZF}
\IFRU{Инструкция \JNE проверяет только флаг \ZF, и совершает переход только если флаг не поднят. 
Фактически, \JNE это синоним инструкции \JNZ (\IT{Jump if Not Zero}).}
{\JNE checks only \ZF flag and jumping only if it is not set. 
\JNE is in fact a synonym of \JNZ (\IT{Jump if Not Zero}) instruction.}
\IFRU{Ассемблер транслирует обе инструкции в один и тот же опкод.}
{Assembler translating both \JNE and \JNZ instructions into one single opcode.}
\IFRU
{Таким образом, можно \CMP заменить на \SUB и все будет работать также, но разница в том что \SUB 
все-таки испортит значение в первом операнде. \CMP это \IT{SUB без сохранения результата}.}
{So, \CMP instruction can be replaced to \SUB instruction and almost everything will be fine,
but the difference is in 
the \SUB alter the value of the first operand.
\CMP is \IT{``SUB without saving result''}.}

\IFRU
{Код созданный при помощи GCC 4.4.1 в Linux практически такой же, если не считать мелких отличий, 
которые мы уже рассмотрели раннее.}
{Code generated by GCC 4.4.1 in Linux is almost the same, except differences we already considered.}

\subsubsection{ARM: \OptimizingKeil + \ThumbMode}

\lstinputlisting{04_scanf/checking_retval_ARM_Keil_thumb_O3.asm}

\IFRU{Новые инструкции здесь для нас: \CMP и \TT{BEQ}.}
{New instructions here are \CMP and \TT{BEQ}.}

\CMP \IFRU{аналогична той что в x86, она отнимает один аргумент от второго и сохраняет флаги.}
{is similar to the x86 instruction, it subtracts one argument from another and save flags.}
% TODO: в мануале ARM $op1 + NOT(op2) + 1$ вместо вычитания

\TT{BEQ} (\IT{Branch Equal}) \IFRU{совершает переход по другому адресу, 
если операнды при сравнении были равны, 
либо если результат последнего вычисления был ноль, либо если флаг Z равен $1$.}
{is jumping to another address if operands while comparing were equal to each other, or,
if result of last computation was zero, or if Z flag is $1$.}
\IFRU{То же что и \JZ в}{Same thing as \JZ in} x86.

\IFRU{Всё остальное просто: исполнение разветвляется на две ветки, затем они сходятся там, 
где в \Rzero записывается $0$ как возвращаемое из функции значение и происходит выход из функции.}
{Everything else is simple: execution flow is forking into two branches, then the branches are 
converging at the place
where $0$ is written into \Rzero, as a value returned from the function, and then function finishing.}




\section{\IFRU{Передача параметров через стек}{Passing arguments via stack}}
\index{\Stack}

\IFRU{Как мы уже успели заметить, вызывающая функция передает аргументы для вызываемой через стек. 
А как вызываемая функция имеет к ним доступ?}
{Now we figured out the caller function passing arguments to the callee via stack. 
But how callee\footnote{function being called} access them?}

\lstinputlisting{05_passing_arguments/ex.c}

\subsection{x86: \IFRU{3 аргумента}{3 arguments}}

\subsubsection{MSVC}

\IFRU{Компилируем при помощи MSVC 2010 Express, и в итоге получим:}
{Let's compile it by MSVC 2010 Express and we got:}

\begin{lstlisting}
$SG3830	DB	'a=%d; b=%d; c=%d', 00H

...

	push	3
	push	2
	push	1
	push	OFFSET $SG3830
	call	_printf
	add	esp, 16					; 00000010H
\end{lstlisting}

\IFRU{Все почти то же, за исключением того, что теперь видно, что аргументы для \printf заталкиваются в стек в обратном порядке: самый первый аргумент заталкивается последним.}
{Almost the same, but now we can see the \printf arguments are pushing into stack in reverse order: and the first argument is pushing in as the last one.}

\IFRU{Кстати, вспомним что переменные типа \Tint в 32-битной системе, как известно, имеет ширину 32 бита, это 4 байта}
{By the way, variables of \Tint type in 32-bit environment has 32-bit width that is 4 bytes}.

\IFRU{Итак, у нас всего 4 аргумента. $4*4 = 16$ ~--- именно 16 байт занимают в стеке указатель на строку плюс еще 3 числа типа \Tint.}
{So, we got here 4 arguments. $4*4 = 16$~---they occupy exactly 16 bytes in the stack: 32-bit pointer to string and 3 number of \Tint type.}

\index{x86!\Instructions!ADD}
\index{x86!\Registers!ESP}
\index{cdecl}
\IFRU{Когда при помощи инструкции \TT{``ADD ESP, X''} корректируется \glslink{stack pointer}{указатель стека} \ESP 
после вызова какой-либо функции, зачастую можно сделать вывод о том, сколько аргументов 
у вызываемой функции было, разделив X на 4.}
{When \gls{stack pointer} (the \ESP register) is corrected by \TT{``ADD ESP, X''}
instruction after a function 
call, often, the number of function arguments could be deduced here: just divide X by 4.}

\IFRU{Конечно, это относится только к cdecl-методу передачи аргументов через стек.}
{Of course, this is related only to \IT{cdecl} calling convention.}

\IFRU{См. также в соответствующем разделе о способах передачи аргументов через стек}
{See also section about calling conventions}~(\ref{sec:callingconventions}).

\IFRU{Иногда бывает так, что подряд идут несколько вызовов разных функций, 
но стек корректируется только один раз, после последнего вызова:}
{It is also possible for compiler to merge several \TT{``ADD ESP, X''} instructions into one, after last call:}

\begin{lstlisting}
push a1
push a2
call ...
...
push a1
call ...
...
push a1
push a2
push a3
call ...
add esp, 24
\end{lstlisting}

\subsubsection{MSVC \AndENRU \olly}
\index{\olly}

\IFRU{Попробуем этот же пример в}{Now let's try to load this example in} \olly.
\IFRU{Это один из наиболее популярных win32-отладчиков user-режима}{It is one of the most 
popular user-land win32 debugger}.
\IFRU{Мы можем компилировать наш пример в}{We can try to compile our example in} MSVC 2012 
\IFRU{с опцией}{with} \TT{/MD} \IFRU{что означает, линковать с библиотекой}{option, meaning, to link 
against} \TT{MSVCR*.DLL},
\IFRU{чтобы импортируемые ф-ции были хорошо видны в отладчике}{so we will able to see imported 
functions clearly in debugger}.

\IFRU{Затем загружаем исполняемый файл в}{Then load executable in} \olly.
\IFRU{Самый первый брякпойнт в}{The very first breakpoint is in} \TT{ntdll.dll}, \IFRU{нажмите}{press} 
F9 (\IFRU{запустить}{run}).
\IFRU{Второй брякпойнт в}{The second breakpoint is in} \ac{CRT}-\IFRU{коде}{code}.
\IFRU{Теперь мы должны найти ф-цию}{Now we should find} \main\EN{ function}.

\IFRU{Найдите этот код скроллируя окно кода до самого верха (MSVC располагает ф-цию \main в самом начале
секции кода)}{Find this code by scrolling the code to the very bottom (MSVC allocates \main function at
the very beginning of the code section)}: 
\figname \ref{fig:printf3_olly_1}.

\IFRU{Кликните на инструкции}{Click on} \TT{PUSH EBP}\IFRU{, нажмите}{ instruction, press} F2 
(\IFRU{установка брякпойнта}{set breakpoint}) \IFRU{и нажмите}{and press} F9 (\IFRU{запустить}{run}).
\IFRU{Нам нужно произвести все эти манипуляции, чтобы пропустить \ac{CRT}-код, потому что нам он пока
не интересен}{We need to do these manupulations in order to skip \ac{CRT}-code, because, we don't really 
interesting in it yet}.

\IFRU{Нажмите}{Press} F8 (\stepover) 6 \IFRU{раз, т.е., пропустить
6 инструкций}{times, i.e., skip 6 instructions}: \figname \ref{fig:printf3_olly_2}.

\IFRU{Теперь}{Now the} \PC \IFRU{указывает на инструкцию}{points to the}
\TT{CALL printf}\EN{ instruction}.
\olly, \IFRU{как и другие отладчики, подсвечивает регистры со значениями, которые изменились}
{like other debuggers, highlights value of registers which were changed}.
\IFRU{Так что, каждый раз, когда мы нажимаем}{So each time you press F8}, \EIP 
\IFRU{изменяется и его значение подсвечивается красным}{is changing and its value looking red}.
\ESP \IFRU{также меняется, потому что значения заталкиваются в стек}{is changing as well, 
because values are pushed into the stack}.

\IFRU{Где находятся эти значения в стеке}{Where are the values in the stack}?
\IFRU{Посмотрите на правое/нижнее окно в отладчике}{Take a look into right/bottom window of debugger}:

\begin{figure}[H]
\centering
\includegraphics[scale=0.66]{patterns/03_printf/olly3_stack.png}
\caption{\olly: \IFRU{стек, после того как значения там сохранены}{stack after values pushed}
(\IFRU{я сделал здесь округлую красную пометку в графическом редакторе}{I made round red mark 
here in graphics editor})}
\end{figure}

\IFRU{Так что здесь видно 3 столбца: адрес в стеке, значение в стеке и еще дополнительный комментарий
от \olly}{So we can see there 3 columns: address in the stack, 
value in the stack and some additional \olly comments}. 
\olly \IFRU{понимает}{understands} \printf\IFRU{-строки}{-like strings}, 
\IFRU{так что он показывает здесь и строку и 3 значения \IT{привязанных} к ней}{so it reports the 
string here and 3 values \IT{attached} to it}.

\IFRU{Нажмите}{Press} F8 (\stepover).

\IFRU{В коносил мы видим вывод}{In the console we'll see the output}:

\begin{figure}[H]
\centering
\includegraphics[scale=0.66]{patterns/03_printf/olly3_console.png}
\caption{\RU{Ф-ция }\printf \IFRU{исполнилась}{function executed}}
\end{figure}

\IFRU{Посмотрим, как изменились регистры и состояние стека}{Let's see how registers and stack state 
are changed}: \figname \ref{fig:printf3_olly_3}.

\RU{Регистр }\EAX \IFRU{теперь содержит}{register now contains} \TT{0xD} (13).
That's correct, \printf returns number of characters printed.
\RU{Значение }\EIP \IFRU{изменилось: действительно, теперь здесь адрес инструкции после}
{value is changed: indeed, now there is address of the instruction after} \TT{CALL printf}.
\RU{Значения регистров }\ECX \AndENRU \EDX \IFRU{также изменились}{values are changed as well}.
\IFRU{Очевидно, внутренности ф-ции \printf используют их для каких-то своих нужд}{Apparently, 
\printf function's hidden machinery used them for its own needs}.

\IFRU{Очень важный момент в том что значение \ESP не изменилось. И состояние стека также!}
{A very important thing is that \ESP value is not changed. And stack state too!}
\IFRU{Мы ясно видим здесь и строку формата и соответствующие ей 3 значения, они все еще здесь}
{We clearly see that format string and corresponding 3 values are still there}.
\IFRU{Действительно, по соглашению вызовов \IT{cdecl}, вызывающая ф-ция не очищает аргументы из стека}
{Indeed, that's \IT{cdecl} calling convention, calling function doesn't clear arguments in stack}.
\IFRU{Это должна делать вызывающая ф-ция}{It's caller's duty to do so}.

\IFRU{Нажмите}{Press} F8 \IFRU{снова, чтобы исполнилась инструкция}{again to execute} 
\TT{ADD ESP, 10}\EN{ instruction}: \figname \ref{fig:printf3_olly_4}.

\ESP \IFRU{изменился, но значения все еще в стеке}{is changed, but values are still in the stack}!
\IFRU{Конечно, никому не нужно заполнять эти значения нулями или что-то в этом роде}{Yes, 
of course, no one needs to fill these values by zero or something like that}.
\IFRU{Потому что всё что выше указателя стека}{Because, everything above stack pointer} (\SP) 
\IFRU{это}{is} \IT{\IFRU{шум}{noise}} \OrENRU \IT{\IFRU{мусор}{garbage}}, \IFRU{это всё не имеет
особой ценности}{it has no value at all}.
\IFRU{Было бы очень затратно по времени очищать ненужные элементы стека, к тому же, никому это и не 
нужно}{It would be time consuming to clear unused stack entries, besides, no one really needs to}.

\begin{figure}[H]
\centering
\includegraphics[scale=0.66]{patterns/03_printf/olly3_1.png}
\caption{\olly: \IFRU{самое начало ф-ции}{the very start of the} \main\EN{ function}}
\label{fig:printf3_olly_1}
\end{figure}

\begin{figure}[H]
\centering
\includegraphics[scale=0.66]{patterns/03_printf/olly3_2.png}
\caption{\olly: \IFRU{перед исполнением}{before} \printf\EN{ execution}}
\label{fig:printf3_olly_2}
\end{figure}

\begin{figure}[H]
\centering
\includegraphics[scale=0.66]{patterns/03_printf/olly3_3.png}
\caption{\olly: \IFRU{после исполнения}{after} \printf\EN{ execution}}
\label{fig:printf3_olly_3}
\end{figure}

\begin{figure}[H]
\centering
\includegraphics[scale=0.66]{patterns/03_printf/olly3_4.png}
\caption{\olly: \IFRU{после исполнения инструкции}{after} \TT{ADD ESP, 10}\EN{ instruction execution}}
\label{fig:printf3_olly_4}
\end{figure}

\subsubsection{GCC}

\IFRU{Скомпилируем то же самое в Linux при помощи GCC 4.4.1 и посмотрим в \IDA что вышло:}
{Now let's compile the same in Linux by GCC 4.4.1 and take a look in \IDA what we got:}

\begin{lstlisting}
main            proc near

var_10          = dword ptr -10h
var_C           = dword ptr -0Ch
var_8           = dword ptr -8
var_4           = dword ptr -4

                push    ebp
                mov     ebp, esp
                and     esp, 0FFFFFFF0h
                sub     esp, 10h
                mov     eax, offset aADBDCD ; "a=%d; b=%d; c=%d"
                mov     [esp+10h+var_4], 3
                mov     [esp+10h+var_8], 2
                mov     [esp+10h+var_C], 1
                mov     [esp+10h+var_10], eax
                call    _printf
                mov     eax, 0
                leave
                retn
main            endp
\end{lstlisting}

\IFRU{Можно сказать, что этот короткий код, созданный GCC, отличается от кода MSVC только способом помещения 
значений в стек.
Здесь GCC снова работает со стеком напрямую без \PUSH/\POP.}
{It can be said, the difference between code by MSVC and GCC is only in method of placing arguments on the stack.
Here GCC working directly with stack without \PUSH/\POP.}


\subsection{ARM}

\subsubsection{\NonOptimizingKeil + \ARMMode}

\begin{lstlisting}
.text:000000A4 00 30 A0 E1                 MOV     R3, R0
.text:000000A8 93 21 20 E0                 MLA     R0, R3, R1, R2
.text:000000AC 1E FF 2F E1                 BX      LR
...
.text:000000B0             main
.text:000000B0 10 40 2D E9                 STMFD   SP!, {R4,LR}
.text:000000B4 03 20 A0 E3                 MOV     R2, #3
.text:000000B8 02 10 A0 E3                 MOV     R1, #2
.text:000000BC 01 00 A0 E3                 MOV     R0, #1
.text:000000C0 F7 FF FF EB                 BL      f
.text:000000C4 00 40 A0 E1                 MOV     R4, R0
.text:000000C8 04 10 A0 E1                 MOV     R1, R4
.text:000000CC 5A 0F 8F E2                 ADR     R0, aD_0        ; "%d\n"
.text:000000D0 E3 18 00 EB                 BL      __2printf
.text:000000D4 00 00 A0 E3                 MOV     R0, #0
.text:000000D8 10 80 BD E8                 LDMFD   SP!, {R4,PC}
\end{lstlisting}

\IFRU{В функции \main просто вызываются две функции, в первую (\TT{f}) передается три значения.}
{In \main function, two other functions are simply called, and three values are passed to the 
first one (\TT{f}).}

\IFRU{Как я уже упоминал, первые 4 значения, в ARM обычно передаются в первых 4-х регистрах}
{As I mentioned before, in ARM, first 4 values are usually passed in first 4 registers} (\Rzero-\Rthree).

\IFRU{Функция }{}\TT{f}\IFRU{, как видно, использует три первых регистра (\Rzero-\Rtwo) как аргументы.}
{function, as it seems, use first 3 registers (\Rzero-\Rtwo) as arguments.}

\IFRU{Инструкция }{}\TT{MLA} (\IT{Multiply Accumulate}) \IFRU{перемножает два первых операнда (\Rthree и \Rone), 
прибавляет к произведению
третий операнд (\Rtwo) и помещает результат в нулевой операнд (\Rzero), через который, по стандарту, 
возвращаются значения функций.}
{instruction multiplicates two first operands (\Rthree and \Rone), adds third operand (\Rtwo) to product and places
result into zeroth operand (\Rzero), via which, by standard, values are returned from functions.}

\IFRU{Умножение и сложение одновременно}{Multiplication and addition at once}\footnote{\WPMAO} 
(\IT{Fused multiply–add}) \IFRU{это много где применяемая операция, кстати, аналогичной
инструкции в x86 нет}{is very useful operation, by the way, there are no such instruction in x86}, 
\IFRU{если не считать новых FMA-инструкций}{if not to count new FMA-instruction}\footnote{\url{https://en.wikipedia.org/wiki/FMA_instruction_set}} \IFRU{в}{in} SIMD.

\IFRU{Самая первая инструкция}{The very first} \TT{MOV R3, R0}, \IFRU{по видимому, избыточна (можно было бы обойтись только одной инструкцией \TT{MLA})}
{instruction, as it seems, redundant (single \TT{MLA} instruction could be used here instead)}, 
\IFRU{компилятор не оптимизировал её, ведь, это компиляция без оптимизации}{compiler wasn't optimized it,
because, this is non-optimizing compilation}.

\IFRU{Инструкция \TT{BX} возвращает управление по адресу записанному в \LR и, если нужно, 
переключает режимы процессора с thumb на ARM или наоборот.}
{\TT{BX} instruction returns control to the address stored in \LR and, if need, switches processor mode from
thumb to ARM or vice versa.}
\IFRU{Это может быть необходимым потому, что, как мы видим, 
функции \TT{f} неизвестно, из какого кода она будет вызываться, из ARM или thumb.}
{This can be necessary because, as we can see, \TT{f} function is not aware, from which code it may be
called, from ARM or thumb.}
\IFRU{Поэтому, если она будет вызываться из кода thumb, \TT{BX} не только вернет
управление в вызывающую функцию, но также переключит процессор в режим thumb.}
{This, if it will be called from thumb code, \TT{BX} will not only return control to the calling function,
but also will switch processor mode to thumb mode.}
\IFRU{Либо не переключит, если функция вызывалась из кода для режима ARM.}
{Or not switch, if the function was called from ARM code.}

\subsubsection{\OptimizingKeil + \ARMMode}

\begin{lstlisting}
.text:00000098             f
.text:00000098 91 20 20 E0                 MLA     R0, R1, R0, R2
.text:0000009C 1E FF 2F E1                 BX      LR
\end{lstlisting}

\IFRU{А вот и функция \TT{f} скомпилированная компилятором Keil в режиме полной оптимизации}
{And here is \TT{f} function compiled by Keil compiler in full optimization mode} (\Othree).
\IFRU{Инструкция \MOV была соптимизирована и теперь \TT{MLA} использует все входящие регистры 
и помещает результат в \Rzero, как раз, где вызываемая функция будет его читать и использовать.}
{\MOV instruction was optimized and now \TT{MLA} uses all input registers and place result into \Rzero, 
exactly where calling function will read it and use.}

\subsubsection{\OptimizingKeil + \ThumbMode}

\begin{lstlisting}
.text:0000005E 48 43                       MULS    R0, R1
.text:00000060 80 18                       ADDS    R0, R0, R2
.text:00000062 70 47                       BX      LR
\end{lstlisting}

\IFRU{В режиме thumb, инструкция \TT{MLA} недоступна, так что компилятору пришлось сгенерировать код, делающий
обе операции по отдельности.}
{\TT{MLA} instruction is not available in thumb mode, so, compiler generates the code doing these two operations
separately.}
\IFRU{Первая инструкция \TT{MULS} умножает \Rzero на \Rone оставляя результат в \Rone.}
{First \TT{MULS} instruction multiply \Rzero by \Rone leaving result in \Rone.}
\IFRU{Вторая (\TT{ADDS}) складывает результат и \Rtwo, оставляя результат в \Rzero.}
{Second (\TT{ADDS}) instruction adds result and \Rtwo leaving result in \Rzero.}



\section{\IFRU{И еще немного о возвращаемых результатах}{One more word about results returning.}}

\newcommand{\MSDNURL}{\href{http://msdn.microsoft.com/en-us/library/7572ztz4.aspx}{MSDN: Return Values (C++)}}

\index{x86!\Registers!EAX}
\index{ARM!\Registers!R0}
\IFRU{Резльутат выполнения функции в x86 обычно возвращается\footnote{См.также: \MSDNURL} через регистр \EAX, 
а если результат имеет тип байт или символ (\IT{char}), 
то в самой младшей части \EAX ~--- \AL. Если функция возвращает число с плавающей запятой, 
то регистр FPU \STZERO будет использован.
В ARM обычно результат возвращается в регистре R0.}
{As of x86, function execution result is usually returned\footnote{See also: \MSDNURL} in 
the \EAX register. 
If it is byte type or character (\IT{char}) ~--- then in the lowest register \EAX part ~--- \AL. 
If function returns \Tfloat number, the FPU register 
\STZERO is to be used instead.
In ARM, result is usually returned in the \Rzero register.} \\
\\

\IFRU{Кстати, что будет если возвращаемое значение в ф-ции \main объявлять не как \IT{int} а как \IT{void}?}
{By the way, what if returning value of the \main function will be declared not as \IT{int} but as \IT{void}?}

\IFRU{Т.н. startup-код вызывает \main примерно так:}
{So called startup-code is calling \main roughly as:}

\begin{lstlisting}
push envp
push argv
push argc
call main
push eax
call exit
\end{lstlisting}

\IFRU{Т.е., иными словами:}{In other words:}

\begin{lstlisting}
exit(main(argc,argv,envp));
\end{lstlisting}

\IFRU{Если вы объявите \main как \IT{void}, и ничего не будете возвращать явно (при помощи выражения \IT{return}), 
то в единственный аргумент exit() попадет
то, что лежало в регистре \EAX на момент выхода из \main.}
{If you declare \main as \IT{void} and nothing will be returned explicitely (by \IT{return} statement),
then something random, that was stored in the \EAX register at the moment of the \main finish, will come into
the sole exit() function argument.}
\IFRU{Там, скорее всего, будет какие-то случайное число, оставшееся от работы вашей ф-ции.}
{Most likely, there will be some random value, leaved from your function execution.}
\IFRU{Так что, код завершения программы будет псевдослучайным.}
{So, exit code of program will be pseudorandom.} \\
\\

\index{\CLanguageElements!return}
\IFRU{Вернемся к тому факту, что возвращемое значение остается в регистре \EAX}
{Let's back to the fact that returning value is leaved in the \EAX register}.
\IFRU{Вот почему старые компиляторы Си не способны создавать функции возвращающие нечто большее нежели 
помещается 
в один регистр (обычно, тип \Tint), а когда нужно, приходится возвращать через указатели, указываемые 
в аргументах.}
{That is why old C compilers cannot create functions capable of returning something not fitting in one 
register (usually type \Tint) but if one needs it, one should return information via pointers passed 
in function arguments.}
\IFRU{Хотя, позже и стало возможным, вернуть, скажем, целую структуру, но этот метод до сих пор не 
очень популярен. 
Если функция должна вернуть структуру, вызывающая функция должна сама, скрыто и прозрачно для программиста, 
выделить место и передать указатель на него в качестве первого аргумента. Это почти то же самое 
что и сделать это вручную, но компилятор прячет это.

Небольшой пример:}
{Now it is possible, to return, let's say, whole structure, but its still not very popular. 
If function should return a large structure, caller must allocate it and pass pointer to it via first argument, 
hiddenly and transparently for programmer. 
That is almost the same as to pass pointer in first argument manually, but compiler hide this.

Small example:}

\lstinputlisting{06_return_results/6_1.c}

\dots \IFRU{получим}{what we got} (MSVC 2010 \Ox):

\lstinputlisting{06_return_results/6_1.asm}

\IFRU{Имя внутреннего макроса для передачи указателя на структуру здесь это \TT{\$T3853}.}
{Macro name for internal variable passing pointer to structure is \TT{\$T3853} here.}


\section{\IFRU{Указатели}{Pointers}}
\index{\CLanguageElements!\Pointers}
\label{label_pointers}

\IFRU{Указатели также часто используются для возврата значений из функции (вспомните случай
со scanf()~(\ref{label_scanf})).}
{Pointers are often used to return values from function (recall scanf() case~(\ref{label_scanf})).}
\IFRU{Например, когда функции нужно вернуть сразу два значения:}
{For example, when function should return two values:}

\begin{lstlisting}
void f1 (int x, int y, int *sum, int *product)
{
	*sum=x+y;
	*product=x*y;
};

void main()
{
	int sum, product;

	f1(123, 456, &sum, &product);
	printf ("sum=%d, product=%d\n", sum, product);
};
\end{lstlisting}

\IFRU{Это компилируется в:}
{This compiling into:}

\begin{lstlisting}[caption=\Optimizing MSVC 2010]
CONST	SEGMENT
$SG3863	DB	'sum=%d, product=%d', 0aH, 00H
$SG3864	DB	'sum=%d, product=%d', 0aH, 00H
CONST	ENDS
_TEXT	SEGMENT
_x$ = 8							; size = 4
_y$ = 12						; size = 4
_sum$ = 16						; size = 4
_product$ = 20						; size = 4
f1 PROC					; f1
	mov	ecx, DWORD PTR _y$[esp-4]
	mov	eax, DWORD PTR _x$[esp-4]
	lea	edx, DWORD PTR [eax+ecx]
	imul	eax, ecx
	mov	ecx, DWORD PTR _product$[esp-4]
	push	esi
	mov	esi, DWORD PTR _sum$[esp]
	mov	DWORD PTR [esi], edx
	mov	DWORD PTR [ecx], eax
	pop	esi
	ret	0
f1 ENDP					; f1

_product$ = -8						; size = 4
_sum$ = -4						; size = 4
_main	PROC
	sub	esp, 8
	lea	eax, DWORD PTR _product$[esp+8]
	push	eax
	lea	ecx, DWORD PTR _sum$[esp+12]
	push	ecx
	push	456					; 000001c8H
	push	123					; 0000007bH
	call	f1			; f1
	mov	edx, DWORD PTR _product$[esp+24]
	mov	eax, DWORD PTR _sum$[esp+24]
	push	edx
	push	eax
	push	OFFSET $SG3863
	call	_printf

...
\end{lstlisting}

\subsection{C++ references}
\index{C++!References}

\IFRU{References в Си++ это тоже указатели, но их называют \IT{безопасными} (safe), потому что работая с ними, 
труднее сделать
ошибку}
{In C++, references are pointers as well, but they are called \IT{safe}, because it is harder to make a mistake while
working with them}\cite[8.3.2]{CPP11}.
\IFRU{Например, reference всегда должен указывать объект того же типа и не может быть NULL}
{For example, reference must always be pointing to the object of corresponding type and cannot be NULL}
\cite[8.6]{ParashiftCPPFAQ}.
\IFRU{Более того, reference нельзя менять, нельзя его заставить указывать на другой объект (reseat)}
{Even more than that, reference cannot be changed, it is not possible to point it to another object (reseat)}
\cite[8.5]{ParashiftCPPFAQ}.

\IFRU{Если мы попробуем изменить наш пример чтобы он использовал reference вместо указателей:}
{If we will try to change are example to use references instead of pointers:}

\begin{lstlisting}
void f2 (int x, int y, int & sum, int & product)
{
	sum=x+y;
	product=x*y;
};
\end{lstlisting}

\IFRU{То выяснится что скомпилированный код абсолютно такой же:}
{Then we'll figure out the compiled code is just the same:}

\begin{lstlisting}[caption=\Optimizing MSVC 2010]
_x$ = 8							; size = 4
_y$ = 12						; size = 4
_sum$ = 16						; size = 4
_product$ = 20						; size = 4
?f2@@YAXHHAAH0@Z PROC					; f2
	mov	ecx, DWORD PTR _y$[esp-4]
	mov	eax, DWORD PTR _x$[esp-4]
	lea	edx, DWORD PTR [eax+ecx]
	imul eax, ecx
	mov ecx, DWORD PTR _product$[esp-4]
	push esi
	mov	esi, DWORD PTR _sum$[esp]
	mov	DWORD PTR [esi], edx
	mov	DWORD PTR [ecx], eax
	pop	esi
	ret	0
?f2@@YAXHHAAH0@Z ENDP					; f2
\end{lstlisting}

(\IFRU{Почему у С++ функций такие странные имена, будет описано позже}{A reason why C++ functions has such strange
names, will be described later}~(\ref{namemangling}).)


\chapter{\RU{Файл сохранения состояния в игре Millenium}\EN{Millenium game save file}}
\label{Millenium_DOS_game}
\index{MS-DOS}

\RU{Игра}\EN{The} \q{Millenium Return to Earth} \RU{под DOS довольно древняя (1991), позволяющая
добывать ресурсы, строить корабли, снаряжать их на другие планеты,\etc{}.}
\EN{is an ancient DOS game (1991), that allows you to mine resources, build ships,
equip them on other planets, and so on}\footnote{\RU{Её можно скачать бесплатно}\EN{It can be downloaded for free}
\href{http://go.yurichev.com/17316}{\RU{здесь}\EN{here}}}.

\RU{Как и многие другие игры, она позволяет сохранять состояние игры в файл.}
\EN{Like many other games, it allows you to save all game state into a file.}

\RU{Посмотрим, сможем ли мы найти что-нибудь в нем}\EN{Let's see if we can find something in it}.

\clearpage
\RU{В игре есть шахта}\EN{So there is a mine in the game}.
\RU{Шахты на некоторых планетах работают быстрее, на некоторых других --- медленнее}\EN{Mines at some planets 
work faster, or slower on others}. 
\RU{Набор ресурсов также разный}\EN{The set of resources is also different}.

\RU{Здесь видно, какие ресурсы добыты в этот момент}\EN{Here we can see what resources are mined at the time}: 

\begin{figure}[H]
\centering
\includegraphics[scale=\FigScale]{ff/millenium/1.png}
\caption{\RU{Шахта: первое состояние}\EN{Mine: state 1}}
\label{fig:mill_1}
\end{figure}

\RU{Сохраним состояние игры}\EN{Let's save a game state}.
\RU{Это файл размером}\EN{This is a file of size} 9538 \RU{байт}\EN{bytes}.

\RU{Подождем несколько \q{дней} здесь в игре и теперь в шахте добыто больше ресурсов}%
\EN{Let's wait some \q{days} here in the game, and now we've got more resources from the mine}:

\begin{figure}[H]
\centering
\includegraphics[scale=\FigScale]{ff/millenium/2.png}
\caption{\RU{Шахта: второе состояние}\EN{Mine: state 2}}
\label{fig:mill_2}
\end{figure}

\RU{Снова сохраним состояние игры}\EN{Let's sav game state again}.

\RU{Теперь просто попробуем сравнить оба файла побайтово используя простую утилиту FC под DOS/Windows:}
\EN{Now let's try to just do binary comparison of the save files using the simple DOS/Windows FC utility:}

\lstinputlisting{ff/millenium/fc_result.txt}

\RU{Вывод здесь неполный, там было больше отличий, но мы обрежем результат до самого интересного.}%
\EN{The output is incomplete here, there are more differences, but we will cut result to show the most interesting.}

\RU{В первой версии у нас было 14 единиц водорода (hydrogen) и 102 --- кислорода (oxygen).}
\EN{In the first state, we have 14 \q{units} of hydrogen and 102 \q{units} of oxygen.}
\RU{Во второй версии у нас 22 и 155 единиц соответственно.}
\EN{We have 22 and 155 \q{units} respectively in the second state.}
\RU{Если эти значения сохраняются в файл, мы должны увидеть разницу}\EN{If these values are saved into 
the save file, we would see this in the difference}.
\RU{И она действительно есть}\EN{And indeed we do}. 
\RU{Там}\EN{There is} 0x0E (14) \RU{на позиции}\EN{at position} 0xBDA \RU{и это значение}\EN{and this value is} 
0x16 (22) \RU{в новой версии файла}\EN{in the new version of the file}.
\RU{Это, наверное, водород}\EN{This is probably hydrogen}.
\RU{Там также}\EN{There is} 0x66 (102) \RU{на позиции}\EN{at position} 0xBDC \RU{в старой версии и}\EN{in the old 
version and} 0x9B (155) \RU{в новой версии файла}\EN{in the new version of the file}. 
\RU{Это, наверное, кислород}\EN{This seems to be the oxygen}.

\RU{Обе версии файла доступны на сайте, для тех кто хочет их изучить (или поэкспериментировать)}%
\EN{Both files are available on the website for those who wants to inspect them (or experiment) more}: 
\href{http://go.yurichev.com/17212}{beginners.re}.

\clearpage
\RU{Новую версию файла откроем в Hiew и отметим значения, связанные с ресурсами, добытыми на шахте в игре}%
\EN{Here is the new version of file opened in Hiew, we marked the values related to the resources mined in the game}: 

\begin{figure}[H]
\centering
\includegraphics[scale=\FigScale]{ff/millenium/hiew3.png}
\caption{Hiew: \RU{первое состояние}\EN{state 1}}
\label{fig:mill_hiew3}
\end{figure}

\RU{Проверим каждое, и это они}\EN{Let's check each, and these are}.
\RU{Это явно 16-битные значения: не удивительно для 16-битной программы под DOS, где \Tint имел длину в 16 бит.}
\EN{These are clearly 16-bit values: not a strange thing for 16-bit DOS software where the \Tint type has 16-bit width.}

\clearpage
\RU{Проверим наши предположения}\EN{Let's check our assumptions}.
\RU{Запишем 1234 (0x4D2) на первой позиции (это должен быть водород)}%
\EN{We will write the 1234 (0x4D2) value at the first position (this must be hydrogen)}:

\begin{figure}[H]
\centering
\includegraphics[scale=\FigScale]{ff/millenium/hiew4.png}
\caption{Hiew: \RU{запишем там}\EN{let's write 1234} (0x4D2)\EN{ there}}
\label{fig:mill_hiew4}
\end{figure}

\RU{Затем загрузим измененный файл в игру и посмотрим на статистику в шахте}%
\EN{Then we will load the changed file in the game and took a look at mine statistics}:

\begin{figure}[H]
\centering
\includegraphics[scale=\FigScale]{ff/millenium/5.png}
\caption{\RU{Проверим значение водорода}\EN{Let's check for hydrogen value}}
\label{fig:mill_5}
\end{figure}

\RU{Так что да, это оно}\EN{So yes, this is it}.

\clearpage
\RU{Попробуем пройти игру как можно быстрее, установим максимальные значения везде}\EN{Now let's try to 
finish the game as soon as possible, set the maximal values everywhere}:

\begin{figure}[H]
\centering
\includegraphics[scale=\FigScale]{ff/millenium/hiew7.png}
\caption{Hiew: \RU{установим максимальные значения}\EN{let's set maximal values}}
\label{fig:mill_hiew7}
\end{figure}

0xFFFF \RU{это}\EN{is} 65535, \RU{так что да, у нас много ресурсов теперь}\EN{so yes, we now have a 
lot of resources}:

\begin{figure}[H]
\centering
\includegraphics[scale=\FigScale]{ff/millenium/6.png}
\caption{\RU{Все ресурсы теперь действительно}\EN{All resources are} 65535 (0xFFFF)\EN{ indeed}}
\label{fig:mill_6}
\end{figure}

\clearpage
\RU{Пропустим еще несколько \q{дней} в игре и видим что-то неладное}\EN{Let's skip some \q{days} in the game and oops}! 
\RU{Некоторых ресурсов стало меньше}\EN{We have a lower amount of some resources}:

\begin{figure}[H]
\centering
\includegraphics[scale=\FigScale]{ff/millenium/8.png}
\caption{\RU{Переполнение переменных ресурсов}\EN{Resource variables overflow}}
\label{fig:mill_8}
\end{figure}

\RU{Это просто переполнение}\EN{That's just overflow}. 
\RU{Разработчик игры вероятно никогда не думал, что значения ресурсов будут такими большими,
так что, здесь, наверное, нет проверок на переполнение, но шахта в игре \q{работает}, ресурсы добавляются,
отсюда и переполнение.}
\EN{The game's developer probably didn't think about such high amounts of resources,
so there are probably no overflow checks, but the mine is \q{working} in the game, resources are added,
hence the overflows.}
\RU{Вероятно, не нужно было жадничать}\EN{Apparently, it was a bad idea to be that greedy}.

\RU{Здесь наверняка еще какие-то значения в этом файле}\EN{There are probably a lot of more values 
saved in this file}.

\RU{Так что это очень простой способ читинга в играх}\EN{So this is very simple method of cheating in games}.
\RU{Файл с таблицей очков также можно легко модифицировать}\EN{High score files often can be easily 
patched like that}.

\EN{More about files and memory snapshots comparing}\RU{Еще насчет сравнения файлов и снимков памяти}: 
\myref{snapshots_comparing}.

\chapter{\RU{Файл сохранения состояния в игре Millenium}\EN{Millenium game save file}}
\label{Millenium_DOS_game}
\index{MS-DOS}

\RU{Игра}\EN{The} \q{Millenium Return to Earth} \RU{под DOS довольно древняя (1991), позволяющая
добывать ресурсы, строить корабли, снаряжать их на другие планеты,\etc{}.}
\EN{is an ancient DOS game (1991), that allows you to mine resources, build ships,
equip them on other planets, and so on}\footnote{\RU{Её можно скачать бесплатно}\EN{It can be downloaded for free}
\href{http://go.yurichev.com/17316}{\RU{здесь}\EN{here}}}.

\RU{Как и многие другие игры, она позволяет сохранять состояние игры в файл.}
\EN{Like many other games, it allows you to save all game state into a file.}

\RU{Посмотрим, сможем ли мы найти что-нибудь в нем}\EN{Let's see if we can find something in it}.

\clearpage
\RU{В игре есть шахта}\EN{So there is a mine in the game}.
\RU{Шахты на некоторых планетах работают быстрее, на некоторых других --- медленнее}\EN{Mines at some planets 
work faster, or slower on others}. 
\RU{Набор ресурсов также разный}\EN{The set of resources is also different}.

\RU{Здесь видно, какие ресурсы добыты в этот момент}\EN{Here we can see what resources are mined at the time}: 

\begin{figure}[H]
\centering
\includegraphics[scale=\FigScale]{ff/millenium/1.png}
\caption{\RU{Шахта: первое состояние}\EN{Mine: state 1}}
\label{fig:mill_1}
\end{figure}

\RU{Сохраним состояние игры}\EN{Let's save a game state}.
\RU{Это файл размером}\EN{This is a file of size} 9538 \RU{байт}\EN{bytes}.

\RU{Подождем несколько \q{дней} здесь в игре и теперь в шахте добыто больше ресурсов}%
\EN{Let's wait some \q{days} here in the game, and now we've got more resources from the mine}:

\begin{figure}[H]
\centering
\includegraphics[scale=\FigScale]{ff/millenium/2.png}
\caption{\RU{Шахта: второе состояние}\EN{Mine: state 2}}
\label{fig:mill_2}
\end{figure}

\RU{Снова сохраним состояние игры}\EN{Let's sav game state again}.

\RU{Теперь просто попробуем сравнить оба файла побайтово используя простую утилиту FC под DOS/Windows:}
\EN{Now let's try to just do binary comparison of the save files using the simple DOS/Windows FC utility:}

\lstinputlisting{ff/millenium/fc_result.txt}

\RU{Вывод здесь неполный, там было больше отличий, но мы обрежем результат до самого интересного.}%
\EN{The output is incomplete here, there are more differences, but we will cut result to show the most interesting.}

\RU{В первой версии у нас было 14 единиц водорода (hydrogen) и 102 --- кислорода (oxygen).}
\EN{In the first state, we have 14 \q{units} of hydrogen and 102 \q{units} of oxygen.}
\RU{Во второй версии у нас 22 и 155 единиц соответственно.}
\EN{We have 22 and 155 \q{units} respectively in the second state.}
\RU{Если эти значения сохраняются в файл, мы должны увидеть разницу}\EN{If these values are saved into 
the save file, we would see this in the difference}.
\RU{И она действительно есть}\EN{And indeed we do}. 
\RU{Там}\EN{There is} 0x0E (14) \RU{на позиции}\EN{at position} 0xBDA \RU{и это значение}\EN{and this value is} 
0x16 (22) \RU{в новой версии файла}\EN{in the new version of the file}.
\RU{Это, наверное, водород}\EN{This is probably hydrogen}.
\RU{Там также}\EN{There is} 0x66 (102) \RU{на позиции}\EN{at position} 0xBDC \RU{в старой версии и}\EN{in the old 
version and} 0x9B (155) \RU{в новой версии файла}\EN{in the new version of the file}. 
\RU{Это, наверное, кислород}\EN{This seems to be the oxygen}.

\RU{Обе версии файла доступны на сайте, для тех кто хочет их изучить (или поэкспериментировать)}%
\EN{Both files are available on the website for those who wants to inspect them (or experiment) more}: 
\href{http://go.yurichev.com/17212}{beginners.re}.

\clearpage
\RU{Новую версию файла откроем в Hiew и отметим значения, связанные с ресурсами, добытыми на шахте в игре}%
\EN{Here is the new version of file opened in Hiew, we marked the values related to the resources mined in the game}: 

\begin{figure}[H]
\centering
\includegraphics[scale=\FigScale]{ff/millenium/hiew3.png}
\caption{Hiew: \RU{первое состояние}\EN{state 1}}
\label{fig:mill_hiew3}
\end{figure}

\RU{Проверим каждое, и это они}\EN{Let's check each, and these are}.
\RU{Это явно 16-битные значения: не удивительно для 16-битной программы под DOS, где \Tint имел длину в 16 бит.}
\EN{These are clearly 16-bit values: not a strange thing for 16-bit DOS software where the \Tint type has 16-bit width.}

\clearpage
\RU{Проверим наши предположения}\EN{Let's check our assumptions}.
\RU{Запишем 1234 (0x4D2) на первой позиции (это должен быть водород)}%
\EN{We will write the 1234 (0x4D2) value at the first position (this must be hydrogen)}:

\begin{figure}[H]
\centering
\includegraphics[scale=\FigScale]{ff/millenium/hiew4.png}
\caption{Hiew: \RU{запишем там}\EN{let's write 1234} (0x4D2)\EN{ there}}
\label{fig:mill_hiew4}
\end{figure}

\RU{Затем загрузим измененный файл в игру и посмотрим на статистику в шахте}%
\EN{Then we will load the changed file in the game and took a look at mine statistics}:

\begin{figure}[H]
\centering
\includegraphics[scale=\FigScale]{ff/millenium/5.png}
\caption{\RU{Проверим значение водорода}\EN{Let's check for hydrogen value}}
\label{fig:mill_5}
\end{figure}

\RU{Так что да, это оно}\EN{So yes, this is it}.

\clearpage
\RU{Попробуем пройти игру как можно быстрее, установим максимальные значения везде}\EN{Now let's try to 
finish the game as soon as possible, set the maximal values everywhere}:

\begin{figure}[H]
\centering
\includegraphics[scale=\FigScale]{ff/millenium/hiew7.png}
\caption{Hiew: \RU{установим максимальные значения}\EN{let's set maximal values}}
\label{fig:mill_hiew7}
\end{figure}

0xFFFF \RU{это}\EN{is} 65535, \RU{так что да, у нас много ресурсов теперь}\EN{so yes, we now have a 
lot of resources}:

\begin{figure}[H]
\centering
\includegraphics[scale=\FigScale]{ff/millenium/6.png}
\caption{\RU{Все ресурсы теперь действительно}\EN{All resources are} 65535 (0xFFFF)\EN{ indeed}}
\label{fig:mill_6}
\end{figure}

\clearpage
\RU{Пропустим еще несколько \q{дней} в игре и видим что-то неладное}\EN{Let's skip some \q{days} in the game and oops}! 
\RU{Некоторых ресурсов стало меньше}\EN{We have a lower amount of some resources}:

\begin{figure}[H]
\centering
\includegraphics[scale=\FigScale]{ff/millenium/8.png}
\caption{\RU{Переполнение переменных ресурсов}\EN{Resource variables overflow}}
\label{fig:mill_8}
\end{figure}

\RU{Это просто переполнение}\EN{That's just overflow}. 
\RU{Разработчик игры вероятно никогда не думал, что значения ресурсов будут такими большими,
так что, здесь, наверное, нет проверок на переполнение, но шахта в игре \q{работает}, ресурсы добавляются,
отсюда и переполнение.}
\EN{The game's developer probably didn't think about such high amounts of resources,
so there are probably no overflow checks, but the mine is \q{working} in the game, resources are added,
hence the overflows.}
\RU{Вероятно, не нужно было жадничать}\EN{Apparently, it was a bad idea to be that greedy}.

\RU{Здесь наверняка еще какие-то значения в этом файле}\EN{There are probably a lot of more values 
saved in this file}.

\RU{Так что это очень простой способ читинга в играх}\EN{So this is very simple method of cheating in games}.
\RU{Файл с таблицей очков также можно легко модифицировать}\EN{High score files often can be easily 
patched like that}.

\EN{More about files and memory snapshots comparing}\RU{Еще насчет сравнения файлов и снимков памяти}: 
\myref{snapshots_comparing}.

\chapter{\RU{Файл сохранения состояния в игре Millenium}\EN{Millenium game save file}}
\label{Millenium_DOS_game}
\index{MS-DOS}

\RU{Игра}\EN{The} \q{Millenium Return to Earth} \RU{под DOS довольно древняя (1991), позволяющая
добывать ресурсы, строить корабли, снаряжать их на другие планеты,\etc{}.}
\EN{is an ancient DOS game (1991), that allows you to mine resources, build ships,
equip them on other planets, and so on}\footnote{\RU{Её можно скачать бесплатно}\EN{It can be downloaded for free}
\href{http://go.yurichev.com/17316}{\RU{здесь}\EN{here}}}.

\RU{Как и многие другие игры, она позволяет сохранять состояние игры в файл.}
\EN{Like many other games, it allows you to save all game state into a file.}

\RU{Посмотрим, сможем ли мы найти что-нибудь в нем}\EN{Let's see if we can find something in it}.

\clearpage
\RU{В игре есть шахта}\EN{So there is a mine in the game}.
\RU{Шахты на некоторых планетах работают быстрее, на некоторых других --- медленнее}\EN{Mines at some planets 
work faster, or slower on others}. 
\RU{Набор ресурсов также разный}\EN{The set of resources is also different}.

\RU{Здесь видно, какие ресурсы добыты в этот момент}\EN{Here we can see what resources are mined at the time}: 

\begin{figure}[H]
\centering
\includegraphics[scale=\FigScale]{ff/millenium/1.png}
\caption{\RU{Шахта: первое состояние}\EN{Mine: state 1}}
\label{fig:mill_1}
\end{figure}

\RU{Сохраним состояние игры}\EN{Let's save a game state}.
\RU{Это файл размером}\EN{This is a file of size} 9538 \RU{байт}\EN{bytes}.

\RU{Подождем несколько \q{дней} здесь в игре и теперь в шахте добыто больше ресурсов}%
\EN{Let's wait some \q{days} here in the game, and now we've got more resources from the mine}:

\begin{figure}[H]
\centering
\includegraphics[scale=\FigScale]{ff/millenium/2.png}
\caption{\RU{Шахта: второе состояние}\EN{Mine: state 2}}
\label{fig:mill_2}
\end{figure}

\RU{Снова сохраним состояние игры}\EN{Let's sav game state again}.

\RU{Теперь просто попробуем сравнить оба файла побайтово используя простую утилиту FC под DOS/Windows:}
\EN{Now let's try to just do binary comparison of the save files using the simple DOS/Windows FC utility:}

\lstinputlisting{ff/millenium/fc_result.txt}

\RU{Вывод здесь неполный, там было больше отличий, но мы обрежем результат до самого интересного.}%
\EN{The output is incomplete here, there are more differences, but we will cut result to show the most interesting.}

\RU{В первой версии у нас было 14 единиц водорода (hydrogen) и 102 --- кислорода (oxygen).}
\EN{In the first state, we have 14 \q{units} of hydrogen and 102 \q{units} of oxygen.}
\RU{Во второй версии у нас 22 и 155 единиц соответственно.}
\EN{We have 22 and 155 \q{units} respectively in the second state.}
\RU{Если эти значения сохраняются в файл, мы должны увидеть разницу}\EN{If these values are saved into 
the save file, we would see this in the difference}.
\RU{И она действительно есть}\EN{And indeed we do}. 
\RU{Там}\EN{There is} 0x0E (14) \RU{на позиции}\EN{at position} 0xBDA \RU{и это значение}\EN{and this value is} 
0x16 (22) \RU{в новой версии файла}\EN{in the new version of the file}.
\RU{Это, наверное, водород}\EN{This is probably hydrogen}.
\RU{Там также}\EN{There is} 0x66 (102) \RU{на позиции}\EN{at position} 0xBDC \RU{в старой версии и}\EN{in the old 
version and} 0x9B (155) \RU{в новой версии файла}\EN{in the new version of the file}. 
\RU{Это, наверное, кислород}\EN{This seems to be the oxygen}.

\RU{Обе версии файла доступны на сайте, для тех кто хочет их изучить (или поэкспериментировать)}%
\EN{Both files are available on the website for those who wants to inspect them (or experiment) more}: 
\href{http://go.yurichev.com/17212}{beginners.re}.

\clearpage
\RU{Новую версию файла откроем в Hiew и отметим значения, связанные с ресурсами, добытыми на шахте в игре}%
\EN{Here is the new version of file opened in Hiew, we marked the values related to the resources mined in the game}: 

\begin{figure}[H]
\centering
\includegraphics[scale=\FigScale]{ff/millenium/hiew3.png}
\caption{Hiew: \RU{первое состояние}\EN{state 1}}
\label{fig:mill_hiew3}
\end{figure}

\RU{Проверим каждое, и это они}\EN{Let's check each, and these are}.
\RU{Это явно 16-битные значения: не удивительно для 16-битной программы под DOS, где \Tint имел длину в 16 бит.}
\EN{These are clearly 16-bit values: not a strange thing for 16-bit DOS software where the \Tint type has 16-bit width.}

\clearpage
\RU{Проверим наши предположения}\EN{Let's check our assumptions}.
\RU{Запишем 1234 (0x4D2) на первой позиции (это должен быть водород)}%
\EN{We will write the 1234 (0x4D2) value at the first position (this must be hydrogen)}:

\begin{figure}[H]
\centering
\includegraphics[scale=\FigScale]{ff/millenium/hiew4.png}
\caption{Hiew: \RU{запишем там}\EN{let's write 1234} (0x4D2)\EN{ there}}
\label{fig:mill_hiew4}
\end{figure}

\RU{Затем загрузим измененный файл в игру и посмотрим на статистику в шахте}%
\EN{Then we will load the changed file in the game and took a look at mine statistics}:

\begin{figure}[H]
\centering
\includegraphics[scale=\FigScale]{ff/millenium/5.png}
\caption{\RU{Проверим значение водорода}\EN{Let's check for hydrogen value}}
\label{fig:mill_5}
\end{figure}

\RU{Так что да, это оно}\EN{So yes, this is it}.

\clearpage
\RU{Попробуем пройти игру как можно быстрее, установим максимальные значения везде}\EN{Now let's try to 
finish the game as soon as possible, set the maximal values everywhere}:

\begin{figure}[H]
\centering
\includegraphics[scale=\FigScale]{ff/millenium/hiew7.png}
\caption{Hiew: \RU{установим максимальные значения}\EN{let's set maximal values}}
\label{fig:mill_hiew7}
\end{figure}

0xFFFF \RU{это}\EN{is} 65535, \RU{так что да, у нас много ресурсов теперь}\EN{so yes, we now have a 
lot of resources}:

\begin{figure}[H]
\centering
\includegraphics[scale=\FigScale]{ff/millenium/6.png}
\caption{\RU{Все ресурсы теперь действительно}\EN{All resources are} 65535 (0xFFFF)\EN{ indeed}}
\label{fig:mill_6}
\end{figure}

\clearpage
\RU{Пропустим еще несколько \q{дней} в игре и видим что-то неладное}\EN{Let's skip some \q{days} in the game and oops}! 
\RU{Некоторых ресурсов стало меньше}\EN{We have a lower amount of some resources}:

\begin{figure}[H]
\centering
\includegraphics[scale=\FigScale]{ff/millenium/8.png}
\caption{\RU{Переполнение переменных ресурсов}\EN{Resource variables overflow}}
\label{fig:mill_8}
\end{figure}

\RU{Это просто переполнение}\EN{That's just overflow}. 
\RU{Разработчик игры вероятно никогда не думал, что значения ресурсов будут такими большими,
так что, здесь, наверное, нет проверок на переполнение, но шахта в игре \q{работает}, ресурсы добавляются,
отсюда и переполнение.}
\EN{The game's developer probably didn't think about such high amounts of resources,
so there are probably no overflow checks, but the mine is \q{working} in the game, resources are added,
hence the overflows.}
\RU{Вероятно, не нужно было жадничать}\EN{Apparently, it was a bad idea to be that greedy}.

\RU{Здесь наверняка еще какие-то значения в этом файле}\EN{There are probably a lot of more values 
saved in this file}.

\RU{Так что это очень простой способ читинга в играх}\EN{So this is very simple method of cheating in games}.
\RU{Файл с таблицей очков также можно легко модифицировать}\EN{High score files often can be easily 
patched like that}.

\EN{More about files and memory snapshots comparing}\RU{Еще насчет сравнения файлов и снимков памяти}: 
\myref{snapshots_comparing}.

\chapter{\RU{Файл сохранения состояния в игре Millenium}\EN{Millenium game save file}}
\label{Millenium_DOS_game}
\index{MS-DOS}

\RU{Игра}\EN{The} \q{Millenium Return to Earth} \RU{под DOS довольно древняя (1991), позволяющая
добывать ресурсы, строить корабли, снаряжать их на другие планеты,\etc{}.}
\EN{is an ancient DOS game (1991), that allows you to mine resources, build ships,
equip them on other planets, and so on}\footnote{\RU{Её можно скачать бесплатно}\EN{It can be downloaded for free}
\href{http://go.yurichev.com/17316}{\RU{здесь}\EN{here}}}.

\RU{Как и многие другие игры, она позволяет сохранять состояние игры в файл.}
\EN{Like many other games, it allows you to save all game state into a file.}

\RU{Посмотрим, сможем ли мы найти что-нибудь в нем}\EN{Let's see if we can find something in it}.

\clearpage
\RU{В игре есть шахта}\EN{So there is a mine in the game}.
\RU{Шахты на некоторых планетах работают быстрее, на некоторых других --- медленнее}\EN{Mines at some planets 
work faster, or slower on others}. 
\RU{Набор ресурсов также разный}\EN{The set of resources is also different}.

\RU{Здесь видно, какие ресурсы добыты в этот момент}\EN{Here we can see what resources are mined at the time}: 

\begin{figure}[H]
\centering
\includegraphics[scale=\FigScale]{ff/millenium/1.png}
\caption{\RU{Шахта: первое состояние}\EN{Mine: state 1}}
\label{fig:mill_1}
\end{figure}

\RU{Сохраним состояние игры}\EN{Let's save a game state}.
\RU{Это файл размером}\EN{This is a file of size} 9538 \RU{байт}\EN{bytes}.

\RU{Подождем несколько \q{дней} здесь в игре и теперь в шахте добыто больше ресурсов}%
\EN{Let's wait some \q{days} here in the game, and now we've got more resources from the mine}:

\begin{figure}[H]
\centering
\includegraphics[scale=\FigScale]{ff/millenium/2.png}
\caption{\RU{Шахта: второе состояние}\EN{Mine: state 2}}
\label{fig:mill_2}
\end{figure}

\RU{Снова сохраним состояние игры}\EN{Let's sav game state again}.

\RU{Теперь просто попробуем сравнить оба файла побайтово используя простую утилиту FC под DOS/Windows:}
\EN{Now let's try to just do binary comparison of the save files using the simple DOS/Windows FC utility:}

\lstinputlisting{ff/millenium/fc_result.txt}

\RU{Вывод здесь неполный, там было больше отличий, но мы обрежем результат до самого интересного.}%
\EN{The output is incomplete here, there are more differences, but we will cut result to show the most interesting.}

\RU{В первой версии у нас было 14 единиц водорода (hydrogen) и 102 --- кислорода (oxygen).}
\EN{In the first state, we have 14 \q{units} of hydrogen and 102 \q{units} of oxygen.}
\RU{Во второй версии у нас 22 и 155 единиц соответственно.}
\EN{We have 22 and 155 \q{units} respectively in the second state.}
\RU{Если эти значения сохраняются в файл, мы должны увидеть разницу}\EN{If these values are saved into 
the save file, we would see this in the difference}.
\RU{И она действительно есть}\EN{And indeed we do}. 
\RU{Там}\EN{There is} 0x0E (14) \RU{на позиции}\EN{at position} 0xBDA \RU{и это значение}\EN{and this value is} 
0x16 (22) \RU{в новой версии файла}\EN{in the new version of the file}.
\RU{Это, наверное, водород}\EN{This is probably hydrogen}.
\RU{Там также}\EN{There is} 0x66 (102) \RU{на позиции}\EN{at position} 0xBDC \RU{в старой версии и}\EN{in the old 
version and} 0x9B (155) \RU{в новой версии файла}\EN{in the new version of the file}. 
\RU{Это, наверное, кислород}\EN{This seems to be the oxygen}.

\RU{Обе версии файла доступны на сайте, для тех кто хочет их изучить (или поэкспериментировать)}%
\EN{Both files are available on the website for those who wants to inspect them (or experiment) more}: 
\href{http://go.yurichev.com/17212}{beginners.re}.

\clearpage
\RU{Новую версию файла откроем в Hiew и отметим значения, связанные с ресурсами, добытыми на шахте в игре}%
\EN{Here is the new version of file opened in Hiew, we marked the values related to the resources mined in the game}: 

\begin{figure}[H]
\centering
\includegraphics[scale=\FigScale]{ff/millenium/hiew3.png}
\caption{Hiew: \RU{первое состояние}\EN{state 1}}
\label{fig:mill_hiew3}
\end{figure}

\RU{Проверим каждое, и это они}\EN{Let's check each, and these are}.
\RU{Это явно 16-битные значения: не удивительно для 16-битной программы под DOS, где \Tint имел длину в 16 бит.}
\EN{These are clearly 16-bit values: not a strange thing for 16-bit DOS software where the \Tint type has 16-bit width.}

\clearpage
\RU{Проверим наши предположения}\EN{Let's check our assumptions}.
\RU{Запишем 1234 (0x4D2) на первой позиции (это должен быть водород)}%
\EN{We will write the 1234 (0x4D2) value at the first position (this must be hydrogen)}:

\begin{figure}[H]
\centering
\includegraphics[scale=\FigScale]{ff/millenium/hiew4.png}
\caption{Hiew: \RU{запишем там}\EN{let's write 1234} (0x4D2)\EN{ there}}
\label{fig:mill_hiew4}
\end{figure}

\RU{Затем загрузим измененный файл в игру и посмотрим на статистику в шахте}%
\EN{Then we will load the changed file in the game and took a look at mine statistics}:

\begin{figure}[H]
\centering
\includegraphics[scale=\FigScale]{ff/millenium/5.png}
\caption{\RU{Проверим значение водорода}\EN{Let's check for hydrogen value}}
\label{fig:mill_5}
\end{figure}

\RU{Так что да, это оно}\EN{So yes, this is it}.

\clearpage
\RU{Попробуем пройти игру как можно быстрее, установим максимальные значения везде}\EN{Now let's try to 
finish the game as soon as possible, set the maximal values everywhere}:

\begin{figure}[H]
\centering
\includegraphics[scale=\FigScale]{ff/millenium/hiew7.png}
\caption{Hiew: \RU{установим максимальные значения}\EN{let's set maximal values}}
\label{fig:mill_hiew7}
\end{figure}

0xFFFF \RU{это}\EN{is} 65535, \RU{так что да, у нас много ресурсов теперь}\EN{so yes, we now have a 
lot of resources}:

\begin{figure}[H]
\centering
\includegraphics[scale=\FigScale]{ff/millenium/6.png}
\caption{\RU{Все ресурсы теперь действительно}\EN{All resources are} 65535 (0xFFFF)\EN{ indeed}}
\label{fig:mill_6}
\end{figure}

\clearpage
\RU{Пропустим еще несколько \q{дней} в игре и видим что-то неладное}\EN{Let's skip some \q{days} in the game and oops}! 
\RU{Некоторых ресурсов стало меньше}\EN{We have a lower amount of some resources}:

\begin{figure}[H]
\centering
\includegraphics[scale=\FigScale]{ff/millenium/8.png}
\caption{\RU{Переполнение переменных ресурсов}\EN{Resource variables overflow}}
\label{fig:mill_8}
\end{figure}

\RU{Это просто переполнение}\EN{That's just overflow}. 
\RU{Разработчик игры вероятно никогда не думал, что значения ресурсов будут такими большими,
так что, здесь, наверное, нет проверок на переполнение, но шахта в игре \q{работает}, ресурсы добавляются,
отсюда и переполнение.}
\EN{The game's developer probably didn't think about such high amounts of resources,
so there are probably no overflow checks, but the mine is \q{working} in the game, resources are added,
hence the overflows.}
\RU{Вероятно, не нужно было жадничать}\EN{Apparently, it was a bad idea to be that greedy}.

\RU{Здесь наверняка еще какие-то значения в этом файле}\EN{There are probably a lot of more values 
saved in this file}.

\RU{Так что это очень простой способ читинга в играх}\EN{So this is very simple method of cheating in games}.
\RU{Файл с таблицей очков также можно легко модифицировать}\EN{High score files often can be easily 
patched like that}.

\EN{More about files and memory snapshots comparing}\RU{Еще насчет сравнения файлов и снимков памяти}: 
\myref{snapshots_comparing}.

\chapter{\RU{Файл сохранения состояния в игре Millenium}\EN{Millenium game save file}}
\label{Millenium_DOS_game}
\index{MS-DOS}

\RU{Игра}\EN{The} \q{Millenium Return to Earth} \RU{под DOS довольно древняя (1991), позволяющая
добывать ресурсы, строить корабли, снаряжать их на другие планеты,\etc{}.}
\EN{is an ancient DOS game (1991), that allows you to mine resources, build ships,
equip them on other planets, and so on}\footnote{\RU{Её можно скачать бесплатно}\EN{It can be downloaded for free}
\href{http://go.yurichev.com/17316}{\RU{здесь}\EN{here}}}.

\RU{Как и многие другие игры, она позволяет сохранять состояние игры в файл.}
\EN{Like many other games, it allows you to save all game state into a file.}

\RU{Посмотрим, сможем ли мы найти что-нибудь в нем}\EN{Let's see if we can find something in it}.

\clearpage
\RU{В игре есть шахта}\EN{So there is a mine in the game}.
\RU{Шахты на некоторых планетах работают быстрее, на некоторых других --- медленнее}\EN{Mines at some planets 
work faster, or slower on others}. 
\RU{Набор ресурсов также разный}\EN{The set of resources is also different}.

\RU{Здесь видно, какие ресурсы добыты в этот момент}\EN{Here we can see what resources are mined at the time}: 

\begin{figure}[H]
\centering
\includegraphics[scale=\FigScale]{ff/millenium/1.png}
\caption{\RU{Шахта: первое состояние}\EN{Mine: state 1}}
\label{fig:mill_1}
\end{figure}

\RU{Сохраним состояние игры}\EN{Let's save a game state}.
\RU{Это файл размером}\EN{This is a file of size} 9538 \RU{байт}\EN{bytes}.

\RU{Подождем несколько \q{дней} здесь в игре и теперь в шахте добыто больше ресурсов}%
\EN{Let's wait some \q{days} here in the game, and now we've got more resources from the mine}:

\begin{figure}[H]
\centering
\includegraphics[scale=\FigScale]{ff/millenium/2.png}
\caption{\RU{Шахта: второе состояние}\EN{Mine: state 2}}
\label{fig:mill_2}
\end{figure}

\RU{Снова сохраним состояние игры}\EN{Let's sav game state again}.

\RU{Теперь просто попробуем сравнить оба файла побайтово используя простую утилиту FC под DOS/Windows:}
\EN{Now let's try to just do binary comparison of the save files using the simple DOS/Windows FC utility:}

\lstinputlisting{ff/millenium/fc_result.txt}

\RU{Вывод здесь неполный, там было больше отличий, но мы обрежем результат до самого интересного.}%
\EN{The output is incomplete here, there are more differences, but we will cut result to show the most interesting.}

\RU{В первой версии у нас было 14 единиц водорода (hydrogen) и 102 --- кислорода (oxygen).}
\EN{In the first state, we have 14 \q{units} of hydrogen and 102 \q{units} of oxygen.}
\RU{Во второй версии у нас 22 и 155 единиц соответственно.}
\EN{We have 22 and 155 \q{units} respectively in the second state.}
\RU{Если эти значения сохраняются в файл, мы должны увидеть разницу}\EN{If these values are saved into 
the save file, we would see this in the difference}.
\RU{И она действительно есть}\EN{And indeed we do}. 
\RU{Там}\EN{There is} 0x0E (14) \RU{на позиции}\EN{at position} 0xBDA \RU{и это значение}\EN{and this value is} 
0x16 (22) \RU{в новой версии файла}\EN{in the new version of the file}.
\RU{Это, наверное, водород}\EN{This is probably hydrogen}.
\RU{Там также}\EN{There is} 0x66 (102) \RU{на позиции}\EN{at position} 0xBDC \RU{в старой версии и}\EN{in the old 
version and} 0x9B (155) \RU{в новой версии файла}\EN{in the new version of the file}. 
\RU{Это, наверное, кислород}\EN{This seems to be the oxygen}.

\RU{Обе версии файла доступны на сайте, для тех кто хочет их изучить (или поэкспериментировать)}%
\EN{Both files are available on the website for those who wants to inspect them (or experiment) more}: 
\href{http://go.yurichev.com/17212}{beginners.re}.

\clearpage
\RU{Новую версию файла откроем в Hiew и отметим значения, связанные с ресурсами, добытыми на шахте в игре}%
\EN{Here is the new version of file opened in Hiew, we marked the values related to the resources mined in the game}: 

\begin{figure}[H]
\centering
\includegraphics[scale=\FigScale]{ff/millenium/hiew3.png}
\caption{Hiew: \RU{первое состояние}\EN{state 1}}
\label{fig:mill_hiew3}
\end{figure}

\RU{Проверим каждое, и это они}\EN{Let's check each, and these are}.
\RU{Это явно 16-битные значения: не удивительно для 16-битной программы под DOS, где \Tint имел длину в 16 бит.}
\EN{These are clearly 16-bit values: not a strange thing for 16-bit DOS software where the \Tint type has 16-bit width.}

\clearpage
\RU{Проверим наши предположения}\EN{Let's check our assumptions}.
\RU{Запишем 1234 (0x4D2) на первой позиции (это должен быть водород)}%
\EN{We will write the 1234 (0x4D2) value at the first position (this must be hydrogen)}:

\begin{figure}[H]
\centering
\includegraphics[scale=\FigScale]{ff/millenium/hiew4.png}
\caption{Hiew: \RU{запишем там}\EN{let's write 1234} (0x4D2)\EN{ there}}
\label{fig:mill_hiew4}
\end{figure}

\RU{Затем загрузим измененный файл в игру и посмотрим на статистику в шахте}%
\EN{Then we will load the changed file in the game and took a look at mine statistics}:

\begin{figure}[H]
\centering
\includegraphics[scale=\FigScale]{ff/millenium/5.png}
\caption{\RU{Проверим значение водорода}\EN{Let's check for hydrogen value}}
\label{fig:mill_5}
\end{figure}

\RU{Так что да, это оно}\EN{So yes, this is it}.

\clearpage
\RU{Попробуем пройти игру как можно быстрее, установим максимальные значения везде}\EN{Now let's try to 
finish the game as soon as possible, set the maximal values everywhere}:

\begin{figure}[H]
\centering
\includegraphics[scale=\FigScale]{ff/millenium/hiew7.png}
\caption{Hiew: \RU{установим максимальные значения}\EN{let's set maximal values}}
\label{fig:mill_hiew7}
\end{figure}

0xFFFF \RU{это}\EN{is} 65535, \RU{так что да, у нас много ресурсов теперь}\EN{so yes, we now have a 
lot of resources}:

\begin{figure}[H]
\centering
\includegraphics[scale=\FigScale]{ff/millenium/6.png}
\caption{\RU{Все ресурсы теперь действительно}\EN{All resources are} 65535 (0xFFFF)\EN{ indeed}}
\label{fig:mill_6}
\end{figure}

\clearpage
\RU{Пропустим еще несколько \q{дней} в игре и видим что-то неладное}\EN{Let's skip some \q{days} in the game and oops}! 
\RU{Некоторых ресурсов стало меньше}\EN{We have a lower amount of some resources}:

\begin{figure}[H]
\centering
\includegraphics[scale=\FigScale]{ff/millenium/8.png}
\caption{\RU{Переполнение переменных ресурсов}\EN{Resource variables overflow}}
\label{fig:mill_8}
\end{figure}

\RU{Это просто переполнение}\EN{That's just overflow}. 
\RU{Разработчик игры вероятно никогда не думал, что значения ресурсов будут такими большими,
так что, здесь, наверное, нет проверок на переполнение, но шахта в игре \q{работает}, ресурсы добавляются,
отсюда и переполнение.}
\EN{The game's developer probably didn't think about such high amounts of resources,
so there are probably no overflow checks, but the mine is \q{working} in the game, resources are added,
hence the overflows.}
\RU{Вероятно, не нужно было жадничать}\EN{Apparently, it was a bad idea to be that greedy}.

\RU{Здесь наверняка еще какие-то значения в этом файле}\EN{There are probably a lot of more values 
saved in this file}.

\RU{Так что это очень простой способ читинга в играх}\EN{So this is very simple method of cheating in games}.
\RU{Файл с таблицей очков также можно легко модифицировать}\EN{High score files often can be easily 
patched like that}.

\EN{More about files and memory snapshots comparing}\RU{Еще насчет сравнения файлов и снимков памяти}: 
\myref{snapshots_comparing}.

\chapter{\RU{Файл сохранения состояния в игре Millenium}\EN{Millenium game save file}}
\label{Millenium_DOS_game}
\index{MS-DOS}

\RU{Игра}\EN{The} \q{Millenium Return to Earth} \RU{под DOS довольно древняя (1991), позволяющая
добывать ресурсы, строить корабли, снаряжать их на другие планеты,\etc{}.}
\EN{is an ancient DOS game (1991), that allows you to mine resources, build ships,
equip them on other planets, and so on}\footnote{\RU{Её можно скачать бесплатно}\EN{It can be downloaded for free}
\href{http://go.yurichev.com/17316}{\RU{здесь}\EN{here}}}.

\RU{Как и многие другие игры, она позволяет сохранять состояние игры в файл.}
\EN{Like many other games, it allows you to save all game state into a file.}

\RU{Посмотрим, сможем ли мы найти что-нибудь в нем}\EN{Let's see if we can find something in it}.

\clearpage
\RU{В игре есть шахта}\EN{So there is a mine in the game}.
\RU{Шахты на некоторых планетах работают быстрее, на некоторых других --- медленнее}\EN{Mines at some planets 
work faster, or slower on others}. 
\RU{Набор ресурсов также разный}\EN{The set of resources is also different}.

\RU{Здесь видно, какие ресурсы добыты в этот момент}\EN{Here we can see what resources are mined at the time}: 

\begin{figure}[H]
\centering
\includegraphics[scale=\FigScale]{ff/millenium/1.png}
\caption{\RU{Шахта: первое состояние}\EN{Mine: state 1}}
\label{fig:mill_1}
\end{figure}

\RU{Сохраним состояние игры}\EN{Let's save a game state}.
\RU{Это файл размером}\EN{This is a file of size} 9538 \RU{байт}\EN{bytes}.

\RU{Подождем несколько \q{дней} здесь в игре и теперь в шахте добыто больше ресурсов}%
\EN{Let's wait some \q{days} here in the game, and now we've got more resources from the mine}:

\begin{figure}[H]
\centering
\includegraphics[scale=\FigScale]{ff/millenium/2.png}
\caption{\RU{Шахта: второе состояние}\EN{Mine: state 2}}
\label{fig:mill_2}
\end{figure}

\RU{Снова сохраним состояние игры}\EN{Let's sav game state again}.

\RU{Теперь просто попробуем сравнить оба файла побайтово используя простую утилиту FC под DOS/Windows:}
\EN{Now let's try to just do binary comparison of the save files using the simple DOS/Windows FC utility:}

\lstinputlisting{ff/millenium/fc_result.txt}

\RU{Вывод здесь неполный, там было больше отличий, но мы обрежем результат до самого интересного.}%
\EN{The output is incomplete here, there are more differences, but we will cut result to show the most interesting.}

\RU{В первой версии у нас было 14 единиц водорода (hydrogen) и 102 --- кислорода (oxygen).}
\EN{In the first state, we have 14 \q{units} of hydrogen and 102 \q{units} of oxygen.}
\RU{Во второй версии у нас 22 и 155 единиц соответственно.}
\EN{We have 22 and 155 \q{units} respectively in the second state.}
\RU{Если эти значения сохраняются в файл, мы должны увидеть разницу}\EN{If these values are saved into 
the save file, we would see this in the difference}.
\RU{И она действительно есть}\EN{And indeed we do}. 
\RU{Там}\EN{There is} 0x0E (14) \RU{на позиции}\EN{at position} 0xBDA \RU{и это значение}\EN{and this value is} 
0x16 (22) \RU{в новой версии файла}\EN{in the new version of the file}.
\RU{Это, наверное, водород}\EN{This is probably hydrogen}.
\RU{Там также}\EN{There is} 0x66 (102) \RU{на позиции}\EN{at position} 0xBDC \RU{в старой версии и}\EN{in the old 
version and} 0x9B (155) \RU{в новой версии файла}\EN{in the new version of the file}. 
\RU{Это, наверное, кислород}\EN{This seems to be the oxygen}.

\RU{Обе версии файла доступны на сайте, для тех кто хочет их изучить (или поэкспериментировать)}%
\EN{Both files are available on the website for those who wants to inspect them (or experiment) more}: 
\href{http://go.yurichev.com/17212}{beginners.re}.

\clearpage
\RU{Новую версию файла откроем в Hiew и отметим значения, связанные с ресурсами, добытыми на шахте в игре}%
\EN{Here is the new version of file opened in Hiew, we marked the values related to the resources mined in the game}: 

\begin{figure}[H]
\centering
\includegraphics[scale=\FigScale]{ff/millenium/hiew3.png}
\caption{Hiew: \RU{первое состояние}\EN{state 1}}
\label{fig:mill_hiew3}
\end{figure}

\RU{Проверим каждое, и это они}\EN{Let's check each, and these are}.
\RU{Это явно 16-битные значения: не удивительно для 16-битной программы под DOS, где \Tint имел длину в 16 бит.}
\EN{These are clearly 16-bit values: not a strange thing for 16-bit DOS software where the \Tint type has 16-bit width.}

\clearpage
\RU{Проверим наши предположения}\EN{Let's check our assumptions}.
\RU{Запишем 1234 (0x4D2) на первой позиции (это должен быть водород)}%
\EN{We will write the 1234 (0x4D2) value at the first position (this must be hydrogen)}:

\begin{figure}[H]
\centering
\includegraphics[scale=\FigScale]{ff/millenium/hiew4.png}
\caption{Hiew: \RU{запишем там}\EN{let's write 1234} (0x4D2)\EN{ there}}
\label{fig:mill_hiew4}
\end{figure}

\RU{Затем загрузим измененный файл в игру и посмотрим на статистику в шахте}%
\EN{Then we will load the changed file in the game and took a look at mine statistics}:

\begin{figure}[H]
\centering
\includegraphics[scale=\FigScale]{ff/millenium/5.png}
\caption{\RU{Проверим значение водорода}\EN{Let's check for hydrogen value}}
\label{fig:mill_5}
\end{figure}

\RU{Так что да, это оно}\EN{So yes, this is it}.

\clearpage
\RU{Попробуем пройти игру как можно быстрее, установим максимальные значения везде}\EN{Now let's try to 
finish the game as soon as possible, set the maximal values everywhere}:

\begin{figure}[H]
\centering
\includegraphics[scale=\FigScale]{ff/millenium/hiew7.png}
\caption{Hiew: \RU{установим максимальные значения}\EN{let's set maximal values}}
\label{fig:mill_hiew7}
\end{figure}

0xFFFF \RU{это}\EN{is} 65535, \RU{так что да, у нас много ресурсов теперь}\EN{so yes, we now have a 
lot of resources}:

\begin{figure}[H]
\centering
\includegraphics[scale=\FigScale]{ff/millenium/6.png}
\caption{\RU{Все ресурсы теперь действительно}\EN{All resources are} 65535 (0xFFFF)\EN{ indeed}}
\label{fig:mill_6}
\end{figure}

\clearpage
\RU{Пропустим еще несколько \q{дней} в игре и видим что-то неладное}\EN{Let's skip some \q{days} in the game and oops}! 
\RU{Некоторых ресурсов стало меньше}\EN{We have a lower amount of some resources}:

\begin{figure}[H]
\centering
\includegraphics[scale=\FigScale]{ff/millenium/8.png}
\caption{\RU{Переполнение переменных ресурсов}\EN{Resource variables overflow}}
\label{fig:mill_8}
\end{figure}

\RU{Это просто переполнение}\EN{That's just overflow}. 
\RU{Разработчик игры вероятно никогда не думал, что значения ресурсов будут такими большими,
так что, здесь, наверное, нет проверок на переполнение, но шахта в игре \q{работает}, ресурсы добавляются,
отсюда и переполнение.}
\EN{The game's developer probably didn't think about such high amounts of resources,
so there are probably no overflow checks, but the mine is \q{working} in the game, resources are added,
hence the overflows.}
\RU{Вероятно, не нужно было жадничать}\EN{Apparently, it was a bad idea to be that greedy}.

\RU{Здесь наверняка еще какие-то значения в этом файле}\EN{There are probably a lot of more values 
saved in this file}.

\RU{Так что это очень простой способ читинга в играх}\EN{So this is very simple method of cheating in games}.
\RU{Файл с таблицей очков также можно легко модифицировать}\EN{High score files often can be easily 
patched like that}.

\EN{More about files and memory snapshots comparing}\RU{Еще насчет сравнения файлов и снимков памяти}: 
\myref{snapshots_comparing}.

\section{\IFRU{Массивы}{Arrays}}
\label{arrays}

\IFRU{Массив это просто набор переменных в памяти, 
обязательно лежащих рядом, и обязательно одного типа
\footnote{\ac{AKA} ``гомогенный контейнер''}..}
{Array is just a set of variables in memory, 
always lying next to each other, always has same type
\footnote{\ac{AKA} ``homogeneous container''}.}

\subsection{\IFRU{Простой пример}{Simple example}}

\label{arrays_simple}
\lstinputlisting{13_arrays/simple.c}

\subsubsection{x86}

\IFRU{Компилируем}{Let's compile}:

\lstinputlisting[caption=MSVC]{13_arrays/simple_msvc.asm}

\index{x86!\Instructions!SHL}
\IFRU{Однако, ничего особенного, просто два цикла, один заполняет цикл, второй печатает его содержимое. 
Команда \TT{shl ecx, 1} используется для умножения \ECX на 2, об этом ниже~\ref{SHR}.}
{Nothing very special, just two loops: first is filling loop and second is printing loop.
\TT{shl ecx, 1} instruction is used for value multiplication by 2 in the \ECX, more about below~\ref{SHR}.}

\IFRU{Под массив выделено в стеке 80 байт, это 20 элементов по 4 байта.}
{80 bytes are allocated on the stack for array, that's 20 elements of 4 bytes.}

\IFRU{То что делает GCC 4.4.1:}{Here is what GCC 4.4.1 does:}

\lstinputlisting[caption=GCC 4.4.1]{13_arrays/simple_gcc.asm}

\IFRU{Кстати, переменная \IT{a} в нашем примере имеет тип \IT{int*} (то есть, указатель на \Tint{}) ~--- вы можете попробовать передать в другую функцию указатель на массив, но точнее было бы сказать что передается указатель на первый элемент массива (а адреса остальных элементов массива можно вычислить очевидным образом).}{By the way, \IT{a} variable has \IT{int*} type (that is pointer to \Tint{}) ~--- you can try to pass a pointer to array to another function, but it much correctly to say that pointer to the first array elemnt is passed (addresses of another element's places are calculated in obvious way).}
\IFRU{Если индексировать этот указатель как \IT{a[idx]}, \IT{idx} просто прибавляется к указателю и возвращается элемент, расположенный там, куда ссылается вычисленный указатель.}{If to index this pointer as \IT{a[idx]}, \IT{idx} just to be added to the pointer and the element placed there (to which calculated pointer is pointing) returned.}

\IFRU{Вот любопытный пример: строка символов вроде \IT{``string''} это массив из символов, и она имеет тип \IT{const char*}.}{An interesting example: string of characters like \IT{``string''} is array of characters and it has \IT{const char*} type.}\IFRU{К этому указателю также можно применять индекc.}{Index can be applied to this pointer.}
\IFRU{И поэтому можно написать даже так:  \TT{``string''[i]} ~--- это совершенно легальное выражение в \CCpp!}{And that's why it is possible to write like \TT{``string''[i]} ~--- this is correct \CCpp expression!}


\subsubsection{ARM + \NonOptimizingKeil + \ARMMode}

\lstinputlisting{13_arrays/simple_Keil_ARM_O0_en.asm}

\IFRU{Тип \Tint требует 32 бита для хранения, или 4 байта}{\Tint type require 32 bits for storage,
or 4 bytes}, 
\IFRU{так что для хранения 20 переменных типа \Tint, нужно $80$ (\TT{0x50}) байт}
{so for storage of 20 \Tint variables, $80$ (\TT{0x50}) bytes are needed},
\IFRU{поэтому инструкция}{so that's why} \TT{``SUB SP, SP, \#0x50''} 
\IFRU{в эпилоге функции выделяет в локальном стеке под массив
именно столько места.}
{instruction in function epilogue allocates exactly this ammount of space in local stack.}

\IFRU{И в первом и во втором цикле, итератор цикла $i$ будет постоянно находится в регистре \Rfour.}
{In both first and second loops, $i$ loop iterator will be placed in \Rfour register.}

\IFRU{Число, которое нужно записать в массив}{A number to be written into array},
\IFRU{вычисляется так $i*2$, и это эквивалентно сдвигу на 1 бит влево}
{is calculating as $i*2$ which is equivalent to shifting left by one bit},
\index{ARM!Optional operators!LSL}
\IFRU{инструкция}{so} \TT{``MOV R0, R4,LSL\#1''} \IFRU{делает это}{instruction do this}.

\index{ARM!\Instructions!STR}
\TT{``STR R0, [SP,R4,LSL\#2]''} \IFRU{записывает содержимое \Rzero в массив}{writes \Rzero contents into array}.
\IFRU{Указатель на элемент массива вычисляется так: \SP указывает на начало массива, \Rfour это $i$.}
{Here is how a pointer to array element is to be calculated: \SP pointing to array begin, \Rfour is $i$.}
\IFRU{Так что сдвигаем $i$ на 2 бита влево, что эквивалентно умножению на $4$ 
(ведь каждый элемент массива занимает 4 байта) и прибавляем это к адресу начала массива.}
{So shift $i$ left by 2 bits, that's equivalent to multiplication by $4$
(because each array element has size of 4 bytes) and add it to address of array begin.}

\index{ARM!\Instructions!LDR}
\IFRU{Во втором цикле используется обратная инструкция}{The second loop has inverse} 
\TT{``LDR R2, [SP,R4,LSL\#2]''}, 
\IFRU{она загружает из массива нужное значение, и указатель на него вычисляется точно так же.}
{instruction, it loads from array value we need, and the pointer to it is calculated likewise.}

\subsubsection{ARM + \OptimizingKeil + \ThumbMode}

\lstinputlisting{13_arrays/simple_Keil_thumb_O3_en.asm}

\IFRU{Код для thumb очень похожий.}{Thumb code is very similar.}
\index{ARM!\Instructions!LSLS}
\IFRU{В thumb имеются отдельные инструкции для битовых сдвигов (как \TT{LSLS}), 
вычисляющие и число для записи в массив и адрес каждого элемента массива.}
{Thumb mode has special instructions for bit shifting (like \TT{LSLS}),
calculating value to be written into array and address of each element in array as well.}

\IFRU{Компилятор почему-то выделил в локальном стеке немного больше места, 
однако последние 4 байта не используются.}
{Compiler allocates slightly more space in local stack, however, last 4 bytes are not used.}




\subsection{\IFRU{Переполнение буфера}{Buffer overflow}}
\label{subsec:bufferoverflow}

\IFRU{Итак, индексация массива это просто \IT{массив\lbrack{}индекс\rbrack}.  % TODO как-то плохо отображаются []
Если вы присмотритесь к коду, в цикле печати значений массива через \printf вы 
не увидите проверок индекса, \IT{меньше ли он двадцати?} 
А что будет если он будет больше двадцати? 
Эта одна из особенностей \CCpp, за которую их, собственно, и ругают.}
{So, array indexing is just \IT{array\lbrack{}index\rbrack}.
If you study generated code closely, you'll probably note missing index bounds checking,
which could check index, \IT{if it is less than 20}.
What if index will be greater than 20?
That's the one \CCpp feature it is often blamed for.}

\IFRU{Вот код который и компилируется и работает:}
{Here is a code successfully compiling and working:}

\begin{lstlisting}
#include <stdio.h>

int main() 
{
	int a[20];
	int i;

	for (i=0; i<20; i++)
		a[i]=i*2;

	printf ("a[100]=%d\n", a[100]);

	return 0;
};
\end{lstlisting}

\IFRU{Вот в это}{Compilation results} (MSVC 2010):

\lstinputlisting{13_arrays/BO2_msvc.asm}

\IFRU{У меня оно при запуске выдало вот это:}{I'm running it, and I got:}

\begin{lstlisting}
a[100]=760826203
\end{lstlisting}

\IFRU{Это просто \IT{что-то}, что волею случая лежало в стеке рядом с массивом, 
через 400 байт от его первого элемента.}
{It is just \IT{something}, occasionally lying in the stack near to array, 400 bytes from its first element.}

\IFRU{Действительно, а как могло бы быть иначе? Компилятор мог бы встроить какой-то код, 
каждый раз проверяющий индекс на соответствие пределам массива, как в языках программирования 
более высокого уровня\footnote{Java, Python, итд}, что делало бы запускаемый код медленнее.}
{Indeed, how it could be done differently?
Compiler may generate some additional code for checking index value to be always
in array's bound (like in higher-level programming languages\footnote{Java, Python, etc})
but this makes running code slower.}

\IFRU{Итак, мы прочитали какое-то число из стека явно \IT{нелегально}, а что если мы запишем?}
{OK, we read some values from the stack \IT{illegally} but what if we could write something to it?}

\IFRU{Вот что мы пишем:}{Here is what we will write:}

\begin{lstlisting}
#include <stdio.h>

int main() 
{
	int a[20];
	int i;

	for (i=0; i<30; i++)
		a[i]=i;

	return 0;
};
\end{lstlisting}

\IFRU{И вот что имеем на ассемблере:}{And what we've got:}

\lstinputlisting{\IFRU{13_arrays/BO_ru.asm}{13_arrays/BO_en.asm}}

\IFRU{Запускаете скомпилированную программу, и она падает. Немудрено. Но давайте теперь узнаем, где именно.}
{Run compiled program and its crashing. No wonder. Let's see, where exactly it is crashing.}

\IFRU{Отладчик я уже давно не использую, так как надоело для всяких мелких задач вроде подсмотреть состояние 
регистров, запускать что-то, двигать мышью, итд. 
Поэтому я написал очень минималистическую утилиту для себя, \tracer, коей обхожусь.}
{I'm not using debugger anymore since I tired to run it each time, move mouse, etc, when I need just to
spot a register's state at the specific point.
That's why I wrote very minimalistic tool for myself, \tracer, which is enough for my tasks.}

\IFRU{Помимо всего прочего, я могу использовать мою утилиту просто чтобы посмотреть 
где и какое исключение произошло. 
Итак, пробую:}
{I can also use it just to see, where \gls{debuggee} is crashed.
So let's see:}

\begin{lstlisting}
generic tracer 0.4 (WIN32), http://conus.info/gt

New process: C:\PRJ\...\1.exe, PID=7988
EXCEPTION_ACCESS_VIOLATION: 0x15 (<symbol (0x15) is in unknown module>), ExceptionInformation[0]=8
EAX=0x00000000 EBX=0x7EFDE000 ECX=0x0000001D EDX=0x0000001D
ESI=0x00000000 EDI=0x00000000 EBP=0x00000014 ESP=0x0018FF48
EIP=0x00000015
FLAGS=PF ZF IF RF
PID=7988|Process exit, return code -1073740791
\end{lstlisting}

\IFRU{Итак, следите внимательно за регистрами.}
{Now please keep your eyes on registers.}

\IFRU{Исключение произошло по адресу 0x15. Это явно нелегальный адрес для кода ~--- по крайней мере, win32-кода! 
Мы там как-то очутились, причем, сами того не хотели. Интересен также тот факт что в \EBP хранится 0x14, 
а в \ECX и \EDX ~--- 0x1D.}
{Exception occurred at address 0x15. It is not legal address for code~---at least for win32 code!
We trapped there somehow against our will.
It is also interesting fact the \EBP register contain 0x14,
\ECX and \EDX{}~---0x1D.}

\IFRU{И еще немного изучим разметку стека.}{Let's study stack layout more.}

\IFRU{После того как управление передалось в \main, в стек было сохранено значение \EBP. 
Затем, для массива + переменной \IT{i} было выделено $84$ байта. Это \TT{(20+1)*sizeof(int)}. 
\ESP сейчас указывает на переменную \TT{\_i} в локальном стеке и при исполнении следующего \TT{PUSH что-либо}, 
\IT{что-либо} появится рядом с \TT{\_i}.}
{After control flow was passed into \TT{\main}, the value in the \EBP register was saved on the stack.
Then, $84$ bytes was allocated for array and \IT{i} variable.
That's \TT{(20+1)*sizeof(int)}.
The \ESP pointing now to the \TT{\_i} variable in the local stack and after execution of next \TT{PUSH something},
\IT{something} will be appeared next to \TT{\_i}.}

\IFRU{Вот так выглядит разметка стека пока управление находится внутри}
{That's stack layout while control is inside} \main:

\begin{center}
\begin{tabular}{ | l | l | }
\hline
  \TT{ESP}    & \IFRU{4 байта для \IT{i}}{4 bytes for \IT{i}} \\
  \TT{ESP+4}  & \IFRU{80 байт для массива \TT{a[20]}}{80 bytes for \TT{a[20]} array} \\
  \TT{ESP+84} & \IFRU{сохраненное значение \EBP}{saved \EBP value} \\
  \TT{ESP+88} & \IFRU{адрес возврата}{returning address} \\
\hline
\end{tabular}
\end{center}

\IFRU{Команда \TT{a[19]=чего\_нибудь} записывает последний \Tint в пределах массива (пока что в пределах!)}
{Instruction \TT{a[19]=something} writes last \Tint in array bounds (in bounds so far!)}

\IFRU{Команда \TT{a[20]=чего\_нибудь} записывает \IT{чего\_нибудь} на место где сохранено значение \EBP.}
{Instruction \TT{a[20]=something} writes \IT{something} to the place where value from the \EBP is saved.}

\IFRU{Обратите внимание на состояние регистров на момент падения процесса. В нашем случае, 
в 20-й элемент записалось значение 20. 
И вот все дело в том, что заканчиваясь, эпилог функции восстанавливал значение \EBP. 
(20 в десятичной системе это как раз 0x14 в шестнадцетиричной). 
Далее выполнилась инструкция \RET, которая на самом деле эквивалентна \TT{POP EIP}.}
{Please take a look at registers state at the crash moment. In our case,
number 20 was written to 20th element. 
By the function ending, function epilogue restores original \EBP value.
(20 in decimal system is 0x14 in hexadecimal).
Then, \RET instruction was executed, which is effectively equivalent to \TT{POP EIP} instruction.}

\IFRU{Инструкция \RET вытащила из стека адрес возврата (это адрес где-то внутри \ac{CRT}), 
которая вызвала \main), 
а там было записано 21 в десятичной системе, то есть 0x15 в шестнадцетиричной. 
И вот процессор оказался по адресу 0x15, но исполняемого кода там нет, так что случилось исключение.}
{\RET instruction taking returning adddress from the stack (that is the address inside of \ac{CRT}),
which was called \main),
and 21 was stored there (0x15 in hexadecimal).
The CPU trapped at the address 0x15,
but there is no executable code, so exception was raised.}

\index{\IFRU{Переполнение буфера}{Buffer overflow}}
\IFRU{Добро пожаловать! Это называется}
{Welcome! It is called} \IT{buffer overflow}\footnote{\url{http://en.wikipedia.org/wiki/Stack_buffer_overflow}}.

\IFRU{Замените массив \Tint на строку (массив \Tchar), нарочно создайте слишком длинную строку, 
просуньте её в ту программу, 
в ту функцию, которая не проверяя длину строки скопирует её в слишком короткий буфер, 
и вы сможете указать программе, по какому именно адресу перейти. 
Не все так просто в реальности, конечно, но началось все с этого
\footnote{Классическая статья об этом: \cite{Phrack4914}}.}
{Replace \Tint array by string (\Tchar array), create a long string deliberately,
pass it to the program, to the function which is not checking string length and copies it to short buffer,
and you'll able to point to a program an address to which it must jump.
Not that simple in reality, but that is how it was emerged
\footnote{Classic article about it: \cite{Phrack4914}.}}

\IFRU{Попробуем то же самое в GCC 4.4.1. У нас выходит такое:}{Let's try the same code in GCC 4.4.1. We got:}

\lstinputlisting{13_arrays/BO2_gcc.asm}

\IFRU{Запуск этого в Linux выдаст:}{Running this in Linux will produce:} \TT{Segmentation fault}.

\index{GDB}
\IFRU{Если запустить полученное в отладчике GDB, получим:}
{If we run this in GDB debugger, we getting this:}

\begin{lstlisting}
(gdb) r
Starting program: /home/dennis/RE/1 

Program received signal SIGSEGV, Segmentation fault.
0x00000016 in ?? ()
(gdb) info registers
eax            0x0	0
ecx            0xd2f96388	-755407992
edx            0x1d	29
ebx            0x26eff4	2551796
esp            0xbffff4b0	0xbffff4b0
ebp            0x15	0x15
esi            0x0	0
edi            0x0	0
eip            0x16	0x16
eflags         0x10202	[ IF RF ]
cs             0x73	115
ss             0x7b	123
ds             0x7b	123
es             0x7b	123
fs             0x0	0
gs             0x33	51
(gdb) 
\end{lstlisting}

\IFRU{Значения регистров немного другие чем в примере win32, это потому что разметка стека чуть другая.}
{Register values are slightly different then in win32 example
since stack layout is slightly different too.}

\subsection{\IFRU{Защита от переполнения буфера}{Buffer overflow protection methods}}

\newcommand{\URLWPB}{\href{http://en.wikipedia.org/wiki/Buffer_overflow_protection}
{Wikipedia: \IFRU{описания защит, которые компилятор может вставлять в код}
{compiler-side buffer overflow protection methods}}}

\IFRU{В наше время пытаются бороться с этой напастью, не взирая на халатность программистов на \CCpp. 
В MSVC есть опции вроде\footnote{\URLWPB}:}
{There are several methods to protect against it, regardless of \CCpp programmers' negligence.
MSVC has options like\footnote{\URLWPB}:}

\begin{lstlisting}
 /RTCs Stack Frame runtime checking
 /GZ Enable stack checks (/RTCs)
\end{lstlisting}

\index{x86!\Instructions!RET}
\index{Function prologue}
\IFRU{Один из методов, это в прологе функции вставлять в область локальных переменных 
некоторое случайное значение 
и в эпилоге функции, перед выходом, это число проверять. 
И если проверка не прошла, то не выполнять инструкцию \RET а остановиться (или зависнуть). 
Процесс зависнет, но это лучше чем удаленная атака на ваш хост.}
{One of the methods is to write random value among local variables to stack at function prologue 
and to check it in function epilogue before function exiting.
And if value is not the same, do not execute last instruction \RET, but halt (or hang).
Process will hang, but that's much better then remote attack to your host.}
    
\newcommand{\CANARYURL}
{
    \IFRU
    {
        \href{http://miningwiki.ru/wiki/\%D0\%9A\%D0\%B0\%D0\%BD\%D0\%B0\%D1\%80\%D0\%B5\%D0\%B9\%D0\%BA\%D0\%B0_\%D0\%B2_\%D1\%88\%D0\%B0\%D1\%85\%D1\%82\%D0\%B5}{Шахтерская энциклопедия: Канарейка в шахте}
    }
    {
        \href{http://en.wikipedia.org/wiki/Domestic\_Canary\#Miner.27s\_canary}{Wikipedia: Miner's canary}
    }
}

\index{Canary}
\IFRU{Это случайное значение иногда называют ``канарейкой''\footnote{``canary'' в англоязычной литературе}, 
по аналогии с шахтной канарейкой\footnote{\CANARYURL},
их использовали шахтеры в свое время, чтобы определять, есть ли в шахте опасный газ.
}
{This random value is called ``canary'' sometimes, it is related to miner's canary\footnote{\CANARYURL},
they were used by miners in these days, in order to detect poisonous gases quickly.}
\IFRU{Канарейки очень к нему чувствительны и либо проявляли сильное беспокойство, либо гибли от газа.}
{Canaries are very sensetive to mine gases, they become very agitated in case of danger, or even dead.}

\IFRU{Если скомпилировать наш простейший пример работы с массивом}
{If to compile our very simple array example}~\ref{arrays_simple} \In \ac{MSVC} 
\IFRU{с опцией RTC1 или RTCs}{with RTC1 and RTCs option}, \IFRU{в конце функции будет вызов 
функции}{you will see call to} \TT{@\_RTC\_CheckStackVars@8}\IFRU{, проверяющей корректность ``канарейки''.}
{ function at the function end, checking ``canary'' correctness.}

\IFRU{Посмотрим, как дела обстоят в GCC}{Let's see how GCC handles this}. 
\IFRU{Возьмем пример из секции про}{Let's take} \TT{alloca()}~\ref{alloca}\IFRU{}{ example}:

\lstinputlisting{02_stack/2_1.c}

\IFRU{По умолчанию, без дополнительных ключей, GCC 4.7.3 вставит в код проверку ``канарейки''}
{By default, without any additional options, GCC 4.7.3 will insert ``canary'' check into code}:

\lstinputlisting[caption=GCC 4.7.3]{13_arrays/gcc_canary.asm}

\IFRU{Случайное значение находится в}{Random value is located in} \TT{gs:20}. 
\IFRU{Оно записывается в стек, затем, в конце функции, значение в стеке
сравнивается с корректной ``канарейкой'' в}{It is to be written on the stack and then, at the function end,
value in the stack is compared with correct ``canary'' in} \TT{gs:20}. 
\IFRU{Если значения не равны, будет вызвана функция}{If values are not equal to each other, } 
\TT{\_\_stack\_chk\_fail}\IFRU{ и в консоли мы увидим что-то вроде такого}{ function will be called and we will see
something like that in console} (Ubuntu 13.04 x86):

\begin{lstlisting}
*** buffer overflow detected ***: ./2_1 terminated
======= Backtrace: =========
/lib/i386-linux-gnu/libc.so.6(__fortify_fail+0x63)[0xb7699bc3]
/lib/i386-linux-gnu/libc.so.6(+0x10593a)[0xb769893a]
/lib/i386-linux-gnu/libc.so.6(+0x105008)[0xb7698008]
/lib/i386-linux-gnu/libc.so.6(_IO_default_xsputn+0x8c)[0xb7606e5c]
/lib/i386-linux-gnu/libc.so.6(_IO_vfprintf+0x165)[0xb75d7a45]
/lib/i386-linux-gnu/libc.so.6(__vsprintf_chk+0xc9)[0xb76980d9]
/lib/i386-linux-gnu/libc.so.6(__sprintf_chk+0x2f)[0xb7697fef]
./2_1[0x8048404]
/lib/i386-linux-gnu/libc.so.6(__libc_start_main+0xf5)[0xb75ac935]
======= Memory map: ========
08048000-08049000 r-xp 00000000 08:01 2097586    /home/dennis/2_1
08049000-0804a000 r--p 00000000 08:01 2097586    /home/dennis/2_1
0804a000-0804b000 rw-p 00001000 08:01 2097586    /home/dennis/2_1
094d1000-094f2000 rw-p 00000000 00:00 0          [heap]
b7560000-b757b000 r-xp 00000000 08:01 1048602    /lib/i386-linux-gnu/libgcc_s.so.1
b757b000-b757c000 r--p 0001a000 08:01 1048602    /lib/i386-linux-gnu/libgcc_s.so.1
b757c000-b757d000 rw-p 0001b000 08:01 1048602    /lib/i386-linux-gnu/libgcc_s.so.1
b7592000-b7593000 rw-p 00000000 00:00 0
b7593000-b7740000 r-xp 00000000 08:01 1050781    /lib/i386-linux-gnu/libc-2.17.so
b7740000-b7742000 r--p 001ad000 08:01 1050781    /lib/i386-linux-gnu/libc-2.17.so
b7742000-b7743000 rw-p 001af000 08:01 1050781    /lib/i386-linux-gnu/libc-2.17.so
b7743000-b7746000 rw-p 00000000 00:00 0
b775a000-b775d000 rw-p 00000000 00:00 0
b775d000-b775e000 r-xp 00000000 00:00 0          [vdso]
b775e000-b777e000 r-xp 00000000 08:01 1050794    /lib/i386-linux-gnu/ld-2.17.so
b777e000-b777f000 r--p 0001f000 08:01 1050794    /lib/i386-linux-gnu/ld-2.17.so
b777f000-b7780000 rw-p 00020000 08:01 1050794    /lib/i386-linux-gnu/ld-2.17.so
bff35000-bff56000 rw-p 00000000 00:00 0          [stack]
Aborted (core dumped)
\end{lstlisting}

gs ~--- \IFRU{это так называемый сегментный регистр, эти регистры широко использовались во времена MS-DOS 
и DOS-экстендеров.}{is so called segment register, these registers were used widely in MS-DOS and DOS-extenders
times.}
\IFRU{Сейчас их функция немного изменилась.}{Today, its function is different.}
\index{TLS}
\IFRU{Если говорить коротко, в Linux \TT{gs} всегда указывает на \ac{TLS}\ref{TLS} ~--- там находится различная 
информация, специфичная для выполняющегося треда}
{If to say briefly, the \TT{gs} register in Linux is always pointing to the \ac{TLS}\ref{TLS} ~--- various information specific
to thread is stored there}
(\IFRU{кстати, в win32 эту же роль играет сегментный регистр \TT{fs},
он всегда указывает на}{by the way, in win32 environment,
the \TT{fs} register plays the same role, it pointing to}
Thread Information Block
\footnote{\url{https://en.wikipedia.org/wiki/Win32_Thread_Information_Block}}). 

\IFRU{Больше информации можно почерпнуть из исходных кодов Linux (по крайней мере, в версии 3.11): 
в файле}{More information can be found in Linux source codes (at least in 3.11 version), in}
\IT{arch/x86/include/asm/stackprotector.h}\IFRU{ в комментариях описывается эта переменная}
{ file this variable is described in comments}.

\subsubsection{\OptimizingXcode + \ThumbTwoMode}

\IFRU{Возвращаясь к нашему простому примеру}{Let's back to our simple array example}~(\ref{arrays_simple}),
\IFRU{можно посмотреть, как LLVM добавит проверку ``канарейки''}
{again, now we can see how LLVM will check ``canary'' correctness}:

\lstinputlisting{patterns/13_arrays/simple_Xcode_thumb_O3_en.asm}

\index{Unrolled loop}
\IFRU{Во-первых, как видно, LLVM ``развернул'' цикл и все значения записываются в массив по одному, 
уже вычисленные, 
потому что LLVM посчитал что так будет быстрее.}
{First of all, as we see, LLVM made loop ``unrolled'' and all values are written into array one-by-one,
already calculated since LLVM concluded it will be faster.}
\IFRU{Кстати, инструкции режима ARM позволяют сделать это еще быстрее и это может быть вашим 
домашним заданием.}{By the way, ARM mode instructions may help to do this even faster, 
and finding this way could be your homework.}

\IFRU{В конце функции мы видим сравнение ``канареек'' ~--- той что лежит в локальном стеке и корректной, 
на которую ссылается регистр \TT{R8}.}
{At the function end wee see ``canaries'' comparison~---that laying in local stack and correct one,
to which the \TT{R8} register pointing.}
\index{ARM!\Instructions!IT}
\IFRU{Если они равны, срабатывает блок из четырех инструкций при помощи \TT{``ITTTT EQ''}, это запись
$0$ в \Rzero, эпилог функции и выход из нее.}
{If they are equal to each other, 4-instruction block is triggered by \TT{``ITTTT EQ''}, it is
writing $0$ into \Rzero, function epilogue and exit.}
\IFRU{Если ``канарейки'' не равны, блок не срабатывает и происходит
переход на функцию}{If ``canaries'' are not equal, block will not be triggered,
and jump to} \TT{\_\_\_stack\_chk\_fail}\IFRU{, которая, вероятно, остановит работу программы.}
{ function will be occurred, which, as I suppose, will halt execution.}
% TODO illustrate this!



\subsection{\IFRU{Еще немного о массивах}{One more word about arrays}}

\IFRU{Теперь понятно, почему нельзя написать в исходном коде на \CCpp что-то вроде
\footnote{Впрочем, по стандарту C99 это возможно\cite[6.7.5.2]{C99TC3}: 
GCC может это сделать выделяя место под массив динамически в стеке (как alloca()~\ref{alloca})}}
{Now we understand, why it's not possible to write something like that in \CCpp code
\footnote{However, it's possible in C99 standard\cite[6.7.5.2]{C99TC3}: 
GCC is actually do this by allocating array dynammically in stack (like alloca()~\ref{alloca})}}:

\begin{lstlisting}
void f(int size)
{
    int a[size];
...
};
\end{lstlisting}

\IFRU{Все просто потому, чтобы выделять место под массив в локальном стеке или же сегменте данных 
(если массив глобальный), компилятору нужно знать его размер, чего он, на стадии компиляции, 
разумеется знать не может.}
{That's just because compiler should know exact array size to allocate place for it in local stack layout or
in data segment (in case of global variable) on compiling stage.}

\index{\CLanguageElements!C99!variable length arrays}
\index{\CStandardLibrary!alloca()}
\IFRU{Если вам нужен массив произвольной длины, то выделите столько, сколько нужно, через \TT{malloc()}, 
затем обращайтесь к выделенному блоку байт как к массиву того типа, который вам нужен.
Либо используйте возможность стандарта C99\cite[6.7.5.2]{C99TC3}, 
но внутри это будет похоже на alloca()~\ref{alloca}}
{If you need array of arbitrary size, allocate it by \TT{malloc()}, then access allocated memory block
as array of variables of type you need.
Or use C99 standard feature\cite[6.7.5.2]{C99TC3}, 
but it will be looks like alloca()~\ref{alloca} internally.}


\subsection{\IFRU{Многомерные массивы}{Multidimensional arrays}}

\IFRU{Внутри, многомерный массив выглядит так же как и линейный.}
{Internally, multidimensional array is essentially the same thing as linear array.}

Вернее, можно сказать, он и есть линейный, ведь память компьютера линейная, это одномерный массив.
Но для удобства, этот одномерный массив легко представить как многомерный.

К примеру, элементы массива $a[3][4]$ будут так расположены в одномерном массиве из 12-и ячеек:

\begin{center}
\begin{tabular}{ | l | l | l | l | }
\hline                        
0 & 1 & 2 & 3 \\
\hline  
4 & 5 & 6 & 7 \\
\hline  
8 & 9 & 10 & 11 \\
\hline  
\end{tabular}
\end{center}

То есть, чтобы адресовать нужный элемент, в начале умножаем первый индекс на 4 (ширину матрицы), 
затем прибавляем второй индекс. Это называется \IT{row-major order}, и такой способ представления массивов
и матриц используется по крайней мере в \CCpp, Python. Термин \IT{row-major order} означает по-русски
примерно следующее: ``в начале записываем элементы первой строки, затем второй \dots и элементы последней 
строки в самом конце''.

Другой способ представления называется \IT{column-major order} (индексы массива используются в 
обратном порядке) и это
используется по крайней мере в FORTRAN, MATLAB, R. Термин \IT{column-major order} означает по-русски
следующее: ``в начале записываем элементы первого столбца, затем второго \dots и элементы последнего столбца
в самом конце''.

То же самое и для многомерных массивов.

\IFRU{Попробуем}{Let's see}:

\lstinputlisting{13_arrays/multi.c}

\subsubsection{x86}

\IFRU{В итоге}{We got} (MSVC 2010):

\lstinputlisting{13_arrays/multi_msvc.asm}

\IFRU{В принципе, ничего удивительного. В \TT{insert()} для вычисления адреса нужного элемента массива, 
три входных аргумента перемножаются по формуле $address=600 \cdot 4 \cdot x + 30 \cdot 4 \cdot y + 4z$, 
чтобы представить массив трехмерным.
Не забывайте также что тип \Tint 32-битный (4 байта), поэтому все коэффициенты нужно умножить на 4.}
{Nothing special. For index calculation, three input arguments are multiplying 
by formula $address=600 \cdot 4 \cdot x + 30 \cdot 4 \cdot y + 4z$ to represent array as multidimensional.
Do not forget that \Tint type is 32-bit (4 bytes), so all coefficients should be multiplied by 4.}

GCC 4.4.1:

\lstinputlisting{13_arrays/multi_gcc.asm}

Компилятор GCC решил всё сделать немного иначе. 
Для вычисления одной из операций ($30y$), GCC создал код, где нет самой операции умножения. 
Происходит это так: $(y+y) \ll 4 - (y+y) = (2y) \ll 4 - 2y = 2 \cdot 16 \cdot y - 2y = 32y - 2y = 30y$. 
Таким образом, для вычисления $30y$
используется только операция сложения, операция битового сдвига и операция вычитания. Это работает быстрее.

\subsubsection{ARM + \NonOptimizingXcode + режим thumb}

\lstinputlisting{13_arrays/multi_Xcode_thumb_O0_en.asm}

Неоптимизирующий LLVM сохраняет все переменные в локальном стеке, хотя это и не обязательно. 
Адрес элемента массива вычисляется по уже рассмотренной формуле.

\subsubsection{ARM + \OptimizingXcode + режим thumb}

\lstinputlisting{13_arrays/multi_Xcode_thumb_O3_en.asm}

Тут исползуются уже описанные трюки для замены умножения на операции сдвига, сложения и вычитания.

Также мы видим новую для себя инструкцию \TT{RSB} (\IT{Reverse Subtract}). 
Она работает так же как и \SUB, только меняет операнды
местами. Зачем? \SUB, \TT{RSB}, это те инструкции, к второму операнду которых можно применить коэффициент сдвига, как мы видим
и здесь (\TT{LSL\#4}). Но этот коэффициент можно применить только ко второму операнду. 
Для коммутативных операций, таких
как сложение или умножение, операнды можно менять местами и там всё хорошо. Но вычитание ~--- операция некоммутативная,
так что, для этих случаев существует инструкция \TT{RSB}.

Инструкция \TT{``LDR.W R9, [R9]''} работает как \LEA~\ref{sec:LEA} в x86, и здесь она ничего не делает, она избыточна. 
Вероятно, компилятор несоптимизировал её.



\section{\IFRU{Битовые поля}{Bit fields}}
\label{sec:bitfields}

\IFRU{Немало функций задают различные флаги в аргументах при помощи битовых 
полей\footnote{bit fields в анлоязычной литературе}.}
{A lot of functions defining input flags in arguments using bit fields.}
\index{\CLanguageElements!C99!bool}
\IFRU{Наверное, вместо этого, можно было бы использовать набор переменных типа \IT{bool}, но это было бы 
не очень экономно.}
{Of course, it could be substituted by \IT{bool}-typed variables set, but it's not frugally.}

\subsection{\IFRU{Проверка какого-либо бита}{Specific bit checking}}

\subsubsection{x86}

\IFRU{Например в Win32 API:}{Win32 API example:}

\begin{lstlisting}
	HANDLE fh;

	fh=CreateFile ("file", GENERIC_WRITE | GENERIC_READ, FILE_SHARE_READ, NULL, OPEN_ALWAYS, FILE_ATTRIBUTE_NORMAL, NULL);
\end{lstlisting}

\IFRU{Получаем}{We got} (MSVC 2010):

\begin{lstlisting}[caption=MSVC 2010]
	push	0
	push	128					; 00000080H
	push	4
	push	0
	push	1
	push	-1073741824				; c0000000H
	push	OFFSET $SG78813
	call	DWORD PTR __imp__CreateFileA@28
	mov	DWORD PTR _fh$[ebp], eax
\end{lstlisting}

\IFRU{Заглянем в файл}{Let's take a look into} WinNT.h:

\begin{lstlisting}[caption=WinNT.h]
#define GENERIC_READ                     (0x80000000L)
#define GENERIC_WRITE                    (0x40000000L)
#define GENERIC_EXECUTE                  (0x20000000L)
#define GENERIC_ALL                      (0x10000000L)
\end{lstlisting}

\IFRU{Все ясно}{Everything is clear}, 
\TT{GENERIC\_READ | GENERIC\_WRITE = 0x80000000 | 0x40000000 = 0xC0000000}, 
\IFRU{и это значение используется как второй аргумент для}
{and that's value is used as second argument for} \TT{CreateFile()}\footnote{\href{http://msdn.microsoft.com/en-us/library/aa363858(VS.85).aspx}{MSDN: CreateFile function}} function.

\IFRU{Как \TT{CreateFile()} будет проверять флаги?}{How \TT{CreateFile()} will check flags?}

\index{Windows!KERNEL32.DLL}
\IFRU{Заглянем в KERNEL32.DLL от Windows XP SP3 x86 и найдем в функции \TT{CreateFileW()} в том числе и 
такой фрагмент кода:}
{Let's take a look into KERNEL32.DLL in Windows XP SP3 x86 and we'll find
this fragment of code in the function \TT{CreateFileW}:}

\begin{lstlisting}[caption=KERNEL32.DLL (Windows XP SP3 x86)]
.text:7C83D429                 test    byte ptr [ebp+dwDesiredAccess+3], 40h
.text:7C83D42D                 mov     [ebp+var_8], 1
.text:7C83D434                 jz      short loc_7C83D417
.text:7C83D436                 jmp     loc_7C810817
\end{lstlisting}

\index{x86!\Instructions!TEST}
\IFRU{Здесь мы видим инструкцию \TEST, впрочем, она берет не весь второй аргумент функции, 
но только его самый старший байт (\TT{ebp+dwDesiredAccess+3}) и проверяет его на флаг 0x40 
(имеется ввиду флаг \TT{GENERIC\_WRITE}).}
{Here we see \TEST instruction, it takes, however, not the whole second argument,
but only most significant byte (\TT{ebp+dwDesiredAccess+3}) and checks it for 0x40 flag
(meaning \TT{GENERIC\_WRITE} flag here)}

\index{x86!\Instructions!AND}
\IFRU{\TEST это то же что и \AND, только без сохранения результата 
(вспомните что \CMP это то же что и \SUB, только без сохранения результатов}
{\TEST is merely the same instruction as \AND, but without result saving 
(recall the fact \CMP instruction is merely the same as \SUB, but without result saving}~\ref{CMPandSUB}).

\IFRU{Логика данного фрагмента кода примерно такая:}{This fragment of code logic is as follows:}

\begin{lstlisting}
if ((dwDesiredAccess&0x40000000) == 0) goto loc_7C83D417
\end{lstlisting}

\index{x86!\Instructions!AND}
\index{x86!\Registers!ZF}
\IFRU{Если после операции \AND останется этот бит, то флаг \ZF не будет поднят и условный переход 
\JZ не сработает. 
Переход возможен только если в переменной \TT{dwDesiredAccess} отсутствует бит \TT{0x40000000} ~--- 
тогда результат \AND будет $0$, флаг \ZF будет поднят и переход сработает.}
{If \AND instruction leaving this bit, \ZF flag will be cleared and \JZ conditional jump will not 
be triggered.
Conditional jump will be triggered only if \TT{0x40000000} bit is absent in \TT{dwDesiredAccess} variable ~--- 
then \AND result will be $0$, \ZF flag will be set and conditional jump is to be triggered.}

\IFRU{Попробуем GCC 4.4.1 и Linux:}{Let's try GCC 4.4.1 and Linux:}

\begin{lstlisting}
#include <stdio.h>
#include <fcntl.h>

void main()
{
	int handle;

	handle=open ("file", O_RDWR | O_CREAT);
};
\end{lstlisting}

\IFRU{Получим}{We got}:

\lstinputlisting{14_bitfields/check.asm}[caption=GCC 4.4.1]

\index{Linux!libc.so.6}
\index{syscalls}
\IFRU{Заглянем в реализацию функции \TT{open()} в библиотеке \TT{libc.so.6}, но обнаружим что там 
только вызов сисколла:}
{Let's take a look into \TT{open()} function in \TT{libc.so.6} library, but there is only syscall calling:}

\begin{lstlisting}[caption=open() (libc.so.6)]
.text:000BE69B                 mov     edx, [esp+4+mode] ; mode
.text:000BE69F                 mov     ecx, [esp+4+flags] ; flags
.text:000BE6A3                 mov     ebx, [esp+4+filename] ; filename
.text:000BE6A7                 mov     eax, 5
.text:000BE6AC                 int     80h             ; LINUX - sys_open
\end{lstlisting}

\IFRU{Значит, битовые поля флагов \TT{open()} вероятно проверяются где-то в ядре Linux.}
{So, \TT{open()} bit fields apparently checked somewhere in Linux kernel.}

\IFRU{Разумеется, и стандартные библиотеки Linux и ядро Linux можно получить в виде исходников, 
но нам интересно попробовать разобраться без них.}
{Of course, it is easily to download both Glibc and Linux kernel source code, 
but we are interesting to understand the matter without it.}

\IFRU{Итак, при вызове сисколла \TT{sys\_open}, управление в конечном итоге передается в \TT{do\_sys\_open} в ядре Linux 2.6. 
Оттуда ~--- в \TT{do\_filp\_open()} (эта функция находится в исходниках ядра в файле \TT{fs/namei.c}).}
{So, as of Linux 2.6, when \TT{sys\_open} syscall is called, control eventually passed into \TT{do\_sys\_open} kernel function.
From there ~--- to \TT{do\_filp\_open()} function (this function located in kernel source tree in the file \TT{fs/namei.c}).}

\newcommand{\URLREGPARM}{\url{http://ohse.de/uwe/articles/gcc-attributes.html\#func-regparm}}

\index{fastcall}
\IFRU{Важное отступление. Помимо передачи параметров функции через стек, существует также возможность передавать 
некоторые из них через регистры. Это называется в том числе fastcall~\ref{fastcall}. 
Это работает немного быстрее, так как процессору не нужно обращаться к стеку лежащему в памяти для чтения 
аргументов. 
В GCC есть опция \IT{regparm}\footnote{\URLREGPARM}, 
и с её помощью можно задать, сколько аргументов можно передать через регистры.}
{Important note. Aside from usual passing arguments via stack, there are also method to pass some of them
via registers. This is also called fastcall~\ref{fastcall}.
This works faster, because CPU not needed to access a stack in memory to read argument values.
GCC has option \IT{regparm}\footnote{\URLREGPARM},
and it's possible to set a number of arguments which might be passed via registers.}

\newcommand{\URLKERNELNEWB}{\url{http://kernelnewbies.org/Linux_2_6_20\#head-042c62f290834eb1fe0a1942bbf5bb9a4accbc8f}}
\newcommand{\CALLINGHFILE}{arch\textbackslash{}x86\textbackslash{}include\textbackslash{}asm\textbackslash{}calling.h}

\IFRU{Ядро Linux 2.6 собирается с опцией \TT{-mregparm=3}~\footnote{\URLKERNELNEWB}
\footnote{См. также файл \TT{\CALLINGHFILE} в исходниках ядра}.}
{Linux 2.6 kernel compiled with \TT{-mregparm=3} option~\footnote{\URLKERNELNEWB}
\footnote{See also \TT{\CALLINGHFILE} file in kernel tree}.}

\IFRU{И для нас это означает, что первые три аргумента функции будут передаваться через регистры \EAX, 
\EDX и \ECX, 
а остальные через стек. Разумеется, если аргументов у функции меньше трех, то будет задействована 
только часть регистров.}
{What it means to us, the first 3 arguments will be passed via \EAX, \EDX and \ECX registers, 
the other ones via stack. Of course, if arguments number is less than 3, only part of registers 
will be used.}

\IFRU{Итак, качаем ядро 2.6.31, собираем его в Ubuntu: \TT{make vmlinux}, открываем в \IDA, 
находим функцию \TT{do\_filp\_open()}. В начале мы увидим подобное (комментарии мои):}
{So, let's download Linux Kernel 2.6.31, compile it in Ubuntu: \TT{make vmlinux}, open it in \IDA, 
find the \TT{do\_filp\_open()} function. At the beginning, we will see (comments are mine):}

\lstinputlisting{\IFRU{14_bitfields/check2_ru.asm}{14_bitfields/check2_en.asm}}[caption=do\_filp\_open() (linux kernel 2.6.31)]

\IFRU{GCC сохраняет значения первых трех аргументов в локальном стеке. Иначе, если эти три регистра 
не трогать вообще, то функции компилятора, распределяющей переменные по регистрам (так называемый 
\IT{register allocator}), 
будет очень тесно.}
{GCC saves first 3 arguments values in local stack. 
Otherwise, if compiler would not touch these registers, 
it would be too tight environment for compiler's register allocator}.

\IFRU{Далее находим примерно такой фрагмент кода}{Let's find this fragment of code}:

\lstinputlisting{14_bitfields/check3.asm}[caption=do\_filp\_open() (linux kernel 2.6.31)]

\IFRU{\TT{0x40} ~--- это то чему равен макрос \TT{O\_CREAT}. 
\TT{open\_flag} проверяется на наличие бита \TT{0x40} и если бит равен $1$, то выполняется следующие 
за \JNZ инструкции.}
{\TT{0x40} ~--- is what \TT{O\_CREAT} macro equals to.
\TT{open\_flag} checked for \TT{0x40} bit presence, and if this bit is $1$, 
next \JNZ instruction is triggered.}



\input{14_bitfields/check_ARM}



\subsection{\IFRU{Установка/сброс отдельного бита}{Specific bit setting/clearing}}

\IFRU{Например:}{For example:}

\lstinputlisting{14_bitfields/set_reset.c}

\subsubsection{x86}

\IFRU{Имеем в итоге}{We got} (MSVC 2010):

\lstinputlisting[caption=MSVC 2010]{patterns/14_bitfields/set_reset_msvc.asm}

\index{x86!\Instructions!OR}
\IFRU{Инструкция \OR здесь добавляет в переменную еще один бит, игнорируя остальные.}
{\OR instruction adds one more bit to value while ignoring the rest ones.}

\index{x86!\Instructions!AND}
\IFRU{А \ANDIns сбрасывает некий бит. Можно также сказать, что \ANDIns здесь копирует все биты, кроме одного. 
Действительно, во втором операнде \ANDIns выставлены в единицу те биты, которые нужно сохранить, 
кроме одного, копировать который мы не хотим (и который $0$ в битовой маске).
Так проще понять и запомнить.}
{\ANDIns resetting one bit. It can be said, \ANDIns just copies all bits except one.
Indeed, in the second \ANDIns operand only those bits are set, which are needed to be saved,
except one bit we would not like to copy (which is $0$ in bitmask).
It is easier way to memorize the logic.}

\IFRU{Если скомпилировать в MSVC с оптимизацией (\Ox), то код будет еще короче:}
{If we compile it in MSVC with optimization turned on (\Ox), the code will be even shorter:}

\lstinputlisting[caption=\Optimizing MSVC]{patterns/14_bitfields/set_reset_msvc_Ox.asm}

\IFRU{Попробуем GCC 4.4.1 без оптимизации:}{Let's try GCC 4.4.1 without optimization:}

\lstinputlisting[caption=\NonOptimizing GCC]{patterns/14_bitfields/set_reset_gcc.asm}

\IFRU{Также избыточный код, хотя короче, чем у MSVC без оптимизации.}
{There is a redundant code present,
however, it is shorter then MSVC version without optimization.}

\IFRU{Попробуем теперь GCC с оптимизацией}{Now let's try GCC with optimization turned on} \Othree:

\lstinputlisting[caption=\Optimizing GCC]{patterns/14_bitfields/set_reset_gcc_O3.asm}

\IFRU{Уже короче. Важно отметить что через регистр \AH, компилятор работает с частью регистра \EAX, 
эта его часть от 8-го до 15-го бита включительно.}
{That's shorter.
It is worth noting the compiler works with the \EAX register part via the \AH 
register~---that is the \EAX register part from 8th to 15th bits inclusive.}

\RegTableOne{RAX}{EAX}{AX}{AH}{AL}

\index{x86!8086}
\index{x86!80386}
N.B. \IFRU{В 16-битном процессоре 8086 аккумулятор имел название \AX 
и состоял из двух 8-битных половин ~--- \AL (младшая часть) и \AH (старшая). 
В 80386 регистры были расширены до 32-бит, 
аккумулятор стал называться \EAX, но в целях совместимости, к его \IT{более старым} частям все еще можно 
обращаться как к \AX/\AH/\AL.}
{16-bit CPU 8086 accumulator was named \AX and consisted of two 8-bit 
halves~---\AL (lower byte) and \AH (higher byte).
In 80386 almost all registers were extended to 32-bit, accumulator was named \EAX, 
but for the sake of compatibility,
its \IT{older parts} may be still accessed as \AX/\AH/\AL registers.}

\IFRU{Из-за того, что все x86 процессоры ~--- наследники 16-битного 8086, эти \IT{старые} 16-битные опкоды короче 
нежели более новые 32-битные. 
Поэтому, инструкция \TT{``or ah, 40h''} занимает только 3 байта. 
Было бы логичнее сгенерировать здесь \TT{``or eax, 04000h''}, но это уже 5 байт, или даже 6 
(если регистр в первом операнде не \EAX).}
{Since all x86 CPUs are 16-bit 8086 CPU successors, these \IT{older} 16-bit opcodes are shorter 
than newer 32-bit opcodes.
That's why \TT{``or ah, 40h''} instruction occupying only 3 bytes.
It would be more logical way to emit here \TT{``or eax, 04000h''}
but that is 5 bytes, or even 6
(in case if register in first operand is not \EAX).}

\IFRU{Если мы скомпилируем этот же пример не только с включенной оптимизацией \Othree, 
но еще и с опцией \TT{regparm=3}, о которой я писал немного выше, то получится еще короче:}
{It would be even shorter if to turn on \Othree optimization flag and also set \TT{regparm=3}.}

\lstinputlisting[caption=\Optimizing GCC]{patterns/14_bitfields/set_reset_gcc_O3_regparm3.asm}

\index{Inline code}
\IFRU{Действительно ~--- первый аргумент уже загружен в \EAX, и прямо здесь можно начинать с ним работать. 
Интересно, что и пролог функции (\TT{``push ebp / mov ebp,esp''}) и эпилог (\TT{``pop ebp''}) 
функции можно смело выкинуть
за ненадобностью, 
но возможно GCC еще не так хорош для подобных оптимизаций по размеру кода. 
Впрочем, в реальной жизни, подобные короткие функции лучше всего автоматически делать в виде 
\IT{inline-функций} (\ref{inline_code}).}
{Indeed~---first argument is already loaded into \EAX, so it is possible to work with it in-place.
It is worth noting that both function prologue (\TT{``push ebp / mov ebp,esp''}) and epilogue (\TT{``pop ebp''})
can easily be omitted
here, but GCC probably is not good enough for such code size optimizations.
However, such short functions are better to be \IT{inlined functions} (\ref{inline_code}).}



\subsubsection{ARM + \OptimizingKeil + \ARMMode}

\begin{lstlisting}[caption=\OptimizingKeil + \ARMMode]
02 0C C0 E3          BIC     R0, R0, #0x200
01 09 80 E3          ORR     R0, R0, #0x4000
1E FF 2F E1          BX      LR
\end{lstlisting}

\TT{BIC} это ``логическое и'', аналог \AND в x86. \TT{ORR} это ``логическое или'', аналог \OR в x86.

Пока всё понятно.

\subsubsection{ARM + \OptimizingKeil + \ThumbMode}

\begin{lstlisting}[caption=\OptimizingKeil + \ThumbMode]
01 21 89 03          MOVS    R1, 0x4000
08 43                ORRS    R0, R1
49 11                ASRS    R1, R1, #5   ; generate 0x200 and place to R1
88 43                BICS    R0, R1
70 47                BX      LR
\end{lstlisting}

Вероятно, Keil решил что код в режиме thumb, получающий $0x200$ из $0x4000$, будет компактнее нежели код, 
записывающий $0x200$ в какой-нибудь регистр. 

Поэтому, при помощи инструкции \TT{ASRS} (арифметический сдвиг вправо),
это значение вычисляется как $0x4000 \gg 5$.

\label{anomaly:LLVM}
\index{\CompilerAnomaly}
\subsubsection{ARM + \OptimizingXcode + \ARMMode}

\begin{lstlisting}[caption=\OptimizingXcode + \ARMMode]
42 0C C0 E3          BIC             R0, R0, #0x4200
01 09 80 E3          ORR             R0, R0, #0x4000
1E FF 2F E1          BX              LR
\end{lstlisting}

Код, который был сгенерирован LLVM, на самом деле выглядел бы так:

\begin{lstlisting}
    REMOVE_BIT (rt, 0x4200);
    SET_BIT (rt, 0x4000);
\end{lstlisting}

И он делает то же самое что нам нужно. Но почему $0x4200$? 
\IFRU{Возможно, это артефакт оптимизатора LLVM.}
{Perhaps, that's LLVM optimizer's artifact.}
Можно сказать, ошибка оптимизатора компилятора, но создаваемый код все же работает верно.

Об аномалиях компиляторов, подробнее читайте здесь~\ref{anomaly:Intel}.

Для режима Thumb, \OptimizingXcode генерирует точно такой же код.





\subsection{\ShiftsSectionName}

\IFRU{Битовые сдвиги в \CCpp реализованы при помощи операторов $\ll$ и $\gg$.}
{Bit shifts in \CCpp are implemented via $\ll$ and $\gg$ operators.}

\IFRU{Вот этот несложный пример иллюстрирует функцию, считающую количество бит-единиц во входной переменной:}
{Here is a simple example of function, calculating number of $1$ bits in input variable:}

\lstinputlisting{14_bitfields/shifts.c}

\IFRU{В этом цикле, счетчик итераций \IT{i} считает от $0$ до $31$, а $1 \ll i$ будет от $1$ до $0x80000000$. 
Описывая это словами, можно сказать 
\IT{сдвинуть единицу на $n$ бит влево}.
Т.е., в некотором смысле, выражение $1 \ll i$ последовательно выдаст все возможные позиции бит в 32-битном числе. 
Кстати, освободившийся бит справа всегда обнуляется. Макрос \TT{IS\_SET} проверяет наличие этого бита в \TT{a}.}
{In this loop, iteration count value \IT{i} counting from $0$ to $31$, $1 \ll i$ statement will be counting 
from $1$ to $0x80000000$.
Describing this operation in natural language, we would say \IT{shift $1$ by n bits left}.
In other words, $1 \ll i$ statement will consequently produce all possible bit positions in 32-bit number.
By the way, freed bit at right is always cleared. \TT{IS\_SET} macro is checking bit presence in \TT{a}.}

\begin{figure}[ht!]
\centering
\includegraphics[scale=0.66]{14_bitfields/200px-Rotate_left_logically.png}
\caption{\IFRU{Как работает инструкция \SHL\protect\footnotemark}
{How \SHL instruction works\protect\footnotemark}}
\end{figure}

\footnotetext{\IFRU{иллюстрация взята из}{illustration taken from} wikipedia}

\IFRU{Макрос \TT{IS\_SET} на самом деле это операция логического И (\IT{AND}) 
и она возвращает ноль если бита там нет, 
либо эту же битовую маску, если бит там есть. 
В \CCpp, конструкция \TT{if()} срабатывает, если выражение внутри её не ноль, пусть хоть $123456$, 
поэтому все будет работать.}
{The \TT{IS\_SET} macro is in fact logical and operation (\IT{AND}) 
and it returns zero if specific bit is absent there,
or bit mask, if the bit is present.
\IT{if()} operator triggered in \CCpp if expression in it isn't zero, it might be even $123456$, that's why
it always working correctly.}

\subsubsection{x86}

\IFRU{Компилируем}{Let's compile} (MSVC 2010):

\lstinputlisting[caption=MSVC 2010]{patterns/14_bitfields/shifts_MSVC_\IFRU{ru}{en}.asm}

\IFRU{Вот так работает SHL (\IT{SHift Left})}
{That's how SHL (\IT{SHift Left}) working}.

\IFRU{Скомпилируем то же и в}{Let's compile it in} GCC 4.4.1:

\lstinputlisting[caption=GCC 4.4.1]{patterns/14_bitfields/shifts_gcc.asm}

\IFRU{Инструкции сдвига также активно применяются при делении или умножении 
на числа-степени двойки ($1$, $2$, $4$, $8$, и т.д.).}
{Shift instructions are often used in division and multiplications by power of two numbers 
($1$, $2$, $4$, $8$, etc).}

\IFRU{Например}{For example}:

\begin{lstlisting}
unsigned int f(unsigned int a)
{
	return a/4;
};
\end{lstlisting}

\IFRU{Имеем в итоге}{We got} (MSVC 2010):

\begin{lstlisting}[caption=MSVC 2010]
_a$ = 8							; size = 4
_f	PROC
	mov	eax, DWORD PTR _a$[esp-4]
	shr	eax, 2
	ret	0
_f	ENDP
\end{lstlisting}

\label{SHR}
\index{x86!\Instructions!SHR}
\IFRU{Инструкция \SHR (\IT{SHift Right}) в данном примере сдвигает число на 2 бита вправо. 
При этом, освободившиеся два бита слева (т.е., самые 
старшие разряды), выставляются в нули. А самые младшие 2 бита выкидываются. 
Фактически, эти два выкинутых бита ~--- остаток от деления.}
{\SHR (\IT{SHift Right}) instruction in this example is shifting a number by 2 bits right.
Two freed bits at left (e.g., two most significant bits) are set to zero.
Two least significant bits are dropped.
In fact, these two dropped bits~---division operation remainder.}

\index{x86!\Instructions!SHR}
\IFRU{Инструкция \SHR работает так же, как и \SHL, только в другую сторону.}
{\SHR instruction works just like as \SHL but in other direction.}

\input{shift_right}

\label{division_by_shifting}
\IFRU{Для того, чтобы это проще понять, представьте себе десятичную систему счисления и число $23$. 
$23$ можно разделить на $10$ просто откинув последний разряд ($3$ ~--- это остаток от деления). 
После этой операции останется $2$ как \glslink{quotient}{частное}.}
{It can be easily understood if to imagine decimal numeral system and number $23$.
$23$ can be easily divided by $10$ just by dropping last digit ($3$~---is division remainder). 
$2$ is leaving after operation as a \gls{quotient}.}

\IFRU{Так и с умножением. Умножить на $4$ это просто сдвинуть число на 2 бита влево, 
вставив 2 нулевых бита справа (как два самых младших бита). 
Это как умножить $3$ на $100$ ~--- нужно просто дописать два нуля справа.}
{The same story about multiplication.
Multiplication by $4$ is just shifting the number to the left by 2 bits,
while inserting 2 zero bits at right (as the last two bits).
It is just like to multiply $3$ by $100$~---we need just to add two zeroes at the right.}




\subsubsection{ARM + \OptimizingXcode + \ARMMode}

\begin{lstlisting}[caption=\OptimizingXcode + \ARMMode]
                MOV             R1, R0
                MOV             R0, #0
                MOV             R2, #1
                MOV             R3, R0
loc_2E54
                TST             R1, R2,LSL R3 ; set flags according to R1 & (R2<<R3)
                ADD             R3, R3, #1    ; R3++
                ADDNE           R0, R0, #1    ; if ZF flag is cleared by TST, R0++
                CMP             R3, #32
                BNE             loc_2E54
                BX              LR
\end{lstlisting}

\TT{TST} это то же что и \TEST в x86.

Как я уже указывал~\ref{shifts_in_ARM_mode}, в режиме ARM нет отдельной инструкции для сдвигов, 
но модификаторами 
LSL (\IT{Logical Shift Left}), 
LSR (\IT{Logical Shift Right}), 
ASR (\IT{Arithmetic Shift Right}), 
ROR (\IT{Rotate Right}) и 
RRX (\IT{Rotate Right with Extend}) можно дополнять некоторые инструкции, такие как \MOV, \TT{TST},
\CMP, \ADD, \SUB, \TT{RSB}\footnote{Эти инструкции также называются ``data processing Instructions''}.

Таким образом, инструкция \TT{``TST R1, R2,LSL R3''} здесь работает как $R1 \land (R2 \ll R3)$.

\subsubsection{ARM + \OptimizingXcode + \ThumbTwoMode}

Почти такое же, только там применяется пара инструкций \TT{LSL.W}/\TT{TST} вместо одной \TT{TST},
ведь в режиме thumb нельзя добавлять указывать модификатор \TT{LSL} прямо в \TT{TST}.

\begin{lstlisting}
                MOV             R1, R0
                MOVS            R0, #0
                MOV.W           R9, #1
                MOVS            R3, #0
loc_2F7A
                LSL.W           R2, R9, R3
                TST             R2, R1
                ADD.W           R3, R3, #1
                IT NE
                ADDNE           R0, #1
                CMP             R3, #32
                BNE             loc_2F7A
                BX              LR
\end{lstlisting}





\section{\IFRU{Пример вычисления CRC32}{CRC32 calculation example}}
\index{CRC32}
\label{sec:CRC32}

\newcommand{\URLCRCSRC}{\url{http://burtleburtle.net/bob/c/crc.c}}

\IFRU{Это распространенный табличный способ вычисления хеша алгоритмом 
CRC32\footnote{Исходник взят тут: \URLCRCSRC}.}
{This is very popular table-based CRC32 hash calculation 
technique\footnote{Source code was taken here: \URLCRCSRC}.}

\lstinputlisting{patterns/14_bitfields/CRC.c}

\index{\CLanguageElements!for}
\IFRU{Нас интересует функция \TT{crc()}. 
Кстати, обратите внимание на два инициализатора в выражении \TT{for()}: \TT{hash=len, i=0}. 
Стандарт \CCpp, конечно, допускает это. А в итоговом коде, вместо одной операции инициализации цикла, будет две.}
{We are interesting in the \TT{crc()} function only.
By the way, pay attention to two loop initializers in the \TT{for()} statement: \TT{hash=len, i=0}.
\CCpp standard allows this, of course.
Emitted code will contain two operations in loop initialization part
instead of usual one.}

\IFRU{Компилируем в MSVC с оптимизацией (\Ox). 
Для краткости, я приведу только функцию \TT{crc()}, с некоторыми комментариями.}
{Let's compile it in MSVC with optimization (\Ox).
For the sake of brevity, only \TT{crc()} function is listed here, with my comments.}

\lstinputlisting{patterns/14_bitfields/CRC_2_\LANG.asm}

\IFRU{Попробуем то же самое в GCC 4.4.1 с опцией \Othree:}
{Let's try the same in GCC 4.4.1 with \Othree option:}

\lstinputlisting{patterns/14_bitfields/CRC_gcc_O3_\LANG.asm}

\index{x86!\Instructions!NOP}
\index{x86!\Instructions!LEA}
\IFRU{GCC немного выровнял начало тела цикла по 8-байтной границе, для этого добавил 
\NOP и \TT{lea esi, [esi+0]} (что тоже \IT{холостая операция}). 
Подробнее об этом смотрите в разделе о npad~(\ref{sec:npad}).}
{GCC aligned loop start on a 8-byte boundary by adding \NOP and \TT{lea esi, [esi+0]}
(that is the \IT{idle operation} too).
Read more about it in npad section~(\ref{sec:npad}).}




\section{\IFRU{Структуры}{Structures}}

\IFRU{В принципе, структура в \CCpp это, с некоторыми допущениями, просто всегда лежащий рядом, 
и в той же последовательности, набор переменных, не обязательно одного типа
\footnote{\ac{AKA} ``гетерогенный контейнер''}.}
{It can be defined that \CCpp structure, with some assumptions, just a set of variables, always stored
in memory together, not necessary of the same type
\footnote{\ac{AKA} ``heterogeneous container''}.}

\subsection{\IFRU{Пример SYSTEMTIME}{SYSTEMTIME example}}

\newcommand{\FNSYSTEMTIME}{\footnote{\href{http://msdn.microsoft.com/en-us/library/ms724950(VS.85).aspx}{MSDN: SYSTEMTIME structure}}}

\IFRU{Возьмем, к примеру, структуру SYSTEMTIME\FNSYSTEMTIME{} из win32 описывающую время.}
{Let's take SYSTEMTIME\FNSYSTEMTIME{} win32 structure describing time.}

\IFRU{Она объявлена так:}{That's how it's defined:}

\begin{lstlisting}[caption=WinBase.h]
typedef struct _SYSTEMTIME {
  WORD wYear;
  WORD wMonth;
  WORD wDayOfWeek;
  WORD wDay;
  WORD wHour;
  WORD wMinute;
  WORD wSecond;
  WORD wMilliseconds;
} SYSTEMTIME, *PSYSTEMTIME;
\end{lstlisting}

\IFRU{Пишем на Си функцию для получения текущего системного времени:}
{Let's write a C function to get current time:}

\lstinputlisting{15_structs/systemtime.c}

\IFRU{Что в итоге}{We got} (MSVC 2010):

\lstinputlisting[caption=MSVC 2010]{15_structs/systemtime.asm}

\IFRU{Под структуру в стеке выделено 16 байт ~--- именно столько будет \TT{sizeof(WORD)*8}
(в структуре 8 переменных с типом WORD).}
{16 bytes are allocated for this structure in local stack ~--- that's exactly \TT{sizeof(WORD)*8}
(there are 8 WORD variables in the structure).}

\newcommand{\FNMSDNGST}{\footnote{\href{http://msdn.microsoft.com/en-us/library/ms724390(VS.85).aspx}{MSDN: GetSystemTime function}}}

\IFRU{Обратите внимание на тот факт что структура начинается с поля \TT{wYear}. 
Можно сказать что в качестве аргумента для \TT{GetSystemTime()}\FNMSDNGST передается указатель на структуру 
SYSTEMTIME, но можно также сказать, что передается указатель на поле \TT{wYear}, 
что одно и тоже! 
\TT{GetSystemTime()} пишет текущий год в тот WORD на который указывает переданный указатель, 
затем сдвигается на 2 байта вправо, пишет текущий месяц, итд, итд.}
{Pay attention to the fact the structure beginning with \TT{wYear} field.
It can be said, an pointer to SYSTEMTIME structure is passed to \TT{GetSystemTime()}\FNSYSTEMTIME,
but it's also can be said, pointer to \TT{wYear} field is passed, and that's the same!
\TT{GetSystemTime()} writting current year to the WORD pointer pointing to, then shifting 2 bytes
ahead, then writting current month, etc, etc.}

Тот факт что поля структуры это просто переменные расположенные рядом, 
я могу проиллюстрировать следующим образом.
Глядя на описание структуры \TT{SYSTEMTIME}, мы можем переписать наш простой пример так:

\lstinputlisting{15_structs/systemtime2.c}

Компилятор немного поворчит:

\begin{lstlisting}
systemtime2.c(7) : warning C4133: 'function' : incompatible types - from 'WORD [8]' to 'LPSYSTEMTIME'
\end{lstlisting}

Тем не менее, выдаст такой код:

\lstinputlisting[caption=MSVC 2010]{15_structs/systemtime2.asm}

И это работает так же!

Любопытно что результат на ассемблере неотличим от предыдущего. Таким образом, глядя на этот код, 
никогда нельзя сказать с уверенностью, была ли там объявлена структура, либо просто набор переменных.

Тем не менее, никто в здравом уме делать так не будет. 
Потому что это неудобно. К тому же, иногда, поля в структуре могут меняться, переставляться местами, итд.




\subsection{\IFRU{Выделяем место для структуры через malloc()}{Let's allocate place for structure using malloc()}}

\IFRU{Однако, бывает и так, что проще хранить структуры не в стеке а в куче\footnote{heap}:}
{However, sometimes it's simpler to place structures not in local stack, but in heap:}

\lstinputlisting{15_structs/systemtime_malloc.c}

\IFRU{Скомпилируем на этот раз с оптимизацией (\Ox) чтобы было проще увидеть то, что нам нужно.}
{Let's compile it now with optimization (\Ox) so to easily see what we need.}

\lstinputlisting[caption=\Optimizing MSVC]{15_structs/systemtime_malloc.asm}

\index{\CLanguageElements!malloc()}
\IFRU{Итак, \TT{sizeof(SYSTEMTIME) = 16}, именно столько байт выделяется при помощи \TT{malloc()}. 
Она возвращает указатель на только что выделенный блок памяти в \EAX, который копируется в \ESI. 
Win32 функция \TT{GetSystemTime()} обязуется сохранить состояние \ESI, 
поэтому здесь оно нигде не сохраняется и продолжает использоваться после вызова \TT{GetSystemTime()}.}
{So, \TT{sizeof(SYSTEMTIME) = 16}, that's exact number of bytes to be allocated by \TT{malloc()}.
It return the pointer to freshly allocated memory block in \EAX, which is then moved into \ESI.
\TT{GetSystemTime()} win32 function undertake to save \ESI value, 
and that's why it is not saved here and continue to be used after \TT{GetSystemTime()} call.}

\index{x86!\Instructions!MOVZX}
\IFRU{
Новая инструкция ~--- \MOVZX (\IT{Move with Zero eXtent}). 
Она нужна почти там же где и \MOVSX, 
только всегда очищает остальные биты в $0$. Дело в том что \printf требует 32-битный тип \Tint, 
а в структуре лежит WORD ~--- это 16-битный беззнаковый тип. Поэтому копируя значение из WORD в \Tint, 
нужно очистить биты от 16 до 31, иначе там будет просто случайный мусор, оставшийся от предыдущих действий 
с регистрами.}
{New instruction ~--- \MOVZX (\IT{Move with Zero eXtent}).
It may be used almost in those cases as \MOVSX, but, it clearing other bits to $0$.
That's because \printf require 32-bit \Tint, but we got WORD in structure ~--- that's 16-bit unsigned type.
That's why by copying value from WORD into \Tint{}, bits from 16 to 31 should be cleared, 
because there will be random noise otherwise, leaved from previous operations on registers.}

\IFRU{В этом примере я тоже могу представить структуру как массив WORD-ов}{In this example, I can represent
structure as array of WORD-s}:

\lstinputlisting{15_structs/systemtime_malloc2.c}

\IFRU{Получим такое}{We got}:

\lstinputlisting[caption=\Optimizing MSVC]{15_structs/systemtime_malloc2.asm}

\IFRU{И снова мы получаем идетичный код, неотличимый от предыдущего}{Again, we got a code that cannot be distinguished
from previous}.
\IFRU{Но и снова я должен отметить, что в реальности так лучше не делать}{And again I should note, one shouldn't do
this in practice}.



\subsection{struct tm}

\subsubsection{Linux}

\IFRU{В Линуксе, для примера, возьем структуру \TT{tm} из \TT{time.h}:}
{As of Linux, let's take \TT{tm} structure from \TT{time.h} for example:}

\lstinputlisting{15_structs/GCC_tm.c}

\IFRU{Компилируем при помощи}{Let's compile it in} GCC 4.4.1:

\IFRU{\lstinputlisting[caption=GCC 4.4.1]{15_structs/GCC_tm_ru.asm}}{\lstinputlisting{15_structs/GCC_tm_en.asm}}

\IFRU{К сожалению, по какой-то причине, \IDA не сформировала названия локальных переменных в стеке. 
Но так как мы уже опытные реверсеры :-) то можем обойтись и без этого в таком простом примере.}
{Somehow, \IDA didn't created local variables names in local stack.
But since we already experienced reverse engineers :-) we may do it without this information in 
this simple example.}

\IFRU{Обратите внимание на \TT{lea edx, [eax+76Ch]} ~--- эта инструкция прибавляет $0x76C$ к \EAX, 
но не модифицирует флаги. См. также соответствующий раздел об инструкции \LEA{}~\ref{sec:LEA}.}
{Please also pay attention to \TT{lea edx, [eax+76Ch]} ~--- this instruction just adding $0x76C$ to \EAX,
but not modify any flags. See also relevant section about \LEA{}~\ref{sec:LEA}.}

Чтобы проиллюстрировать то что структура это просто набор переменных лежащих в одном месте, переделаем немного
пример, заглянув предварительно в файл time.h:

\begin{lstlisting}[caption=time.h]
struct tm
{
  int	tm_sec;
  int	tm_min;
  int	tm_hour;
  int	tm_mday;
  int	tm_mon;
  int	tm_year;
  int	tm_wday;
  int	tm_yday;
  int	tm_isdst;
};
\end{lstlisting}

\lstinputlisting{15_structs/GCC_tm2.c}

Обратите внимание на то что в \TT{localtime\_r} передается указатель именно на \TT{tm\_sec}, 
т.е., на первый элемент ``структуры''.

В итоге, и этот компилятор поворчит:

\begin{lstlisting}[caption=GCC 4.7.3]
GCC_tm2.c: In function 'main':
GCC_tm2.c:11:5: warning: passing argument 2 of 'localtime_r' from incompatible pointer type [enabled by default]
In file included from GCC_tm2.c:2:0:
/usr/include/time.h:59:12: note: expected 'struct tm *' but argument is of type 'int *'
\end{lstlisting}

Тем не менее, сгенерирует такоу:

\lstinputlisting[caption=GCC 4.7.3]{15_structs/GCC_tm2.asm}

Этот код почти идентичен уже рассмотренному, и нельзя сказать, была ли структура
в оригинальном исходном коде либо набор переменных.

И это работает. Однако, в реальности так лучше не делать. Обычно, компилятор располагает переменные в локальном
стеке в том же порядке, в котором они объявляются в функции. Тем не менее, никакой гарантии нет.

Кстати, какой-нибудь другой компилятор может предупредить, что переменные \TT{tm\_year}, \TT{tm\_mon}, \TT{tm\_mday},
\TT{tm\_hour}, \TT{tm\_min}, но не \TT{tm\_sec}, используются без инициализации. 
Действительно, ведь компилятор не знает
что они будут заполнены при вызове функции \TT{localtime\_r()}.

Я выбрал именно этот пример для иллюстрации, потому что члены структуры имеют тип \Tint, а члены структуры
\TT{SYSTEMTIME} ~--- 16-битные \TT{WORD}, и если их объявлять так же, то они будут выровнены по 32-битной границе 
и ничего не выйдет (потому что \TT{GetSystemTime()} заполнит их неверно). Читайте об этом в следующей секции
``\StructurePackingSectionName''.

\index{\SyntacticSugar}
Так что, структура это просто набор переменных лежащих в одном месте, рядом. Я мог бы сказать что структура
это такой синтаксический сахар, заставляющий компилятор удерживать их в одном месте. Впрочем, я не специалист
по языкам программирования, так что, скорее всего, ошибаюсь с этим термином.
Кстати, когда-то, в очень ранних версиях Си (перед 1972) структур не 
было вовсе\cite{Ritchie:1993:DCL:155360.155580}.

\subsubsection{ARM + \OptimizingKeil + \ThumbMode}

Этот же пример:

\lstinputlisting[caption=\OptimizingKeil + \ThumbMode]{15_structs/tm_ARM_keil_thumb.asm}

\subsubsection{ARM + \OptimizingXcode + \ThumbTwoMode}

\IDA ``узнала'' структуру tm (потому что \IDA ``знает'' типы аргументов библиотечных функций, 
таких как \TT{localtime\_r()}), поэтому показала здесь обращения к элементам структуры.

\lstinputlisting[caption=\OptimizingXcode + \ThumbTwoMode]{15_structs/tm_ARM_xcode_thumb.asm}



\subsection{\StructurePackingSectionName}

\IFRU{Достаточно немаловажный момент, это упаковка полей в структурах\footnote{См.также: \URLWPDA}.}
{One important thing is fields packing in structures\footnote{See also: \URLWPDA}.}

\IFRU{Возьмем простой пример:}{Let's take a simple example:}

\lstinputlisting{15_structs/15_5.c}

\IFRU{Как видно, мы имеем два поля \Tchar (занимающий один байт) и еще два ~--- \Tint (по 4 байта).}
{As we see, we have two \Tchar fields (each is exactly one byte) and two more ~--- \Tint (each - 4 bytes).}

\IFRU{Компилируется это все в:}{That's all compiling into:}

\lstinputlisting{15_structs/15_5.asm}

\IFRU{Мы видим здесь что адрес каждого поля в структуре выравнивается по 4-байтной границе. 
Так что каждый \Tchar здесь занимает те же 4 байта что и \Tint. Зачем? 
Затем что процессору удобнее обращаться по таким адресам и кешировать данные из памяти.}
{As we can see, each field's address is aligned by 4-bytes border.
That's why each \Tchar using 4 bytes here, like \Tint. Why?
Thus it's easier for CPU to access memory at aligned addresses and to cache data from it.}

\IFRU{Но это не экономично по размеру данных.}{However, it's not very economical in size sense.}

\IFRU{Попробуем скомпилировать тот же исходник с опцией}{Let's try to compile it with option} (\TT{/Zp1}) 
(\IT{/Zp[n] pack structs on n-byte boundary}).

\lstinputlisting[caption=MSVC /Zp1]{15_structs/15_5_msvc_Zp1.asm}

\IFRU{Теперь структура занимает 10 байт и все \Tchar занимают по байту. Что это дает? 
Экономию места. Недостаток ~--- процессор будет обращаться к этим полям не так эффективно 
по скорости как мог бы.}
{Now the structure takes only 10 bytes and each \Tchar value takes 1 byte. What it give to us?
Size economy. And as drawback ~--- CPU will access these fields without maximal performance it can.}

\IFRU{Как нетрудно догадаться, если структура используется много в каких исходниках и объектных файлах, 
все они должны быть откомпилированы с одним и тем же соглашением об упаковке структур.}
{As it can be easily guessed, if the structure is used in many source and object files,
all these should be compiled with the same convention about structures packing.}

\newcommand{\FNURLMSDNZP}{\footnote{\href{http://msdn.microsoft.com/en-us/library/ms253935.aspx}
{MSDN: Working with Packing Structures}}}
\newcommand{\FNURLGCCPC}{\footnote{\href{http://gcc.gnu.org/onlinedocs/gcc/Structure_002dPacking-Pragmas.html}
{Structure-Packing Pragmas}}}

\IFRU{Помимо ключа MSVC \TT{/Zp}, указывающего, по какой границе упаковывать поля структур, есть также 
опция компилятора \TT{\#pragma pack}, её можно указывать прямо в исходнике. 
Это справедливо и для MSVC\FNURLMSDNZP и GCC\FNURLGCCPC{}.}
{Aside from MSVC \TT{/Zp} option which set how to align each structure field, here is also
\TT{\#pragma pack} compiler option, it can be defined right in source code.
It's available in both MSVC\FNURLMSDNZP and GCC\FNURLGCCPC{}.}

\IFRU{Давайте теперь вернемся к \TT{SYSTEMTIME}, которая состоит из 16-битных полей. 
Откуда наш компилятор знает что их надо паковать по однобайтной границе?}
{Let's back to \TT{SYSTEMTIME} structure consisting in 16-bit fields.
How our compiler know to pack them on 1-byte alignment method?}

\IFRU{В файле \TT{WinNT.h} попадается такое:}{\TT{WinNT.h} file has this:}

\begin{lstlisting}[caption=WinNT.h]
#include "pshpack1.h"
\end{lstlisting}

\IFRU{И такое:}{And this:}

\begin{lstlisting}[caption=WinNT.h]
#include "pshpack4.h"                   // 4 byte packing is the default
\end{lstlisting}

\IFRU{Сам файл PshPack1.h выглядит так:}{The file PshPack1.h looks like:}

\begin{lstlisting}[caption=PshPack1.h]
#if ! (defined(lint) || defined(RC_INVOKED))
#if ( _MSC_VER >= 800 && !defined(_M_I86)) || defined(_PUSHPOP_SUPPORTED)
#pragma warning(disable:4103)
#if !(defined( MIDL_PASS )) || defined( __midl )
#pragma pack(push,1)
#else
#pragma pack(1)
#endif
#else
#pragma pack(1)
#endif
#endif /* ! (defined(lint) || defined(RC_INVOKED)) */
\end{lstlisting}

\IFRU{Собственно, так и задается компилятору, как паковать объявленные после \TT{\#pragma pack} структуры.}
{That's how compiler will pack structures defined after \TT{\#pragma pack}.}

\subsection{\IFRU{Вложенные структуры}{Nested structures}}

\IFRU{Теперь, как насчет ситуаций, когда одна структура определяет внутри себя еще одну структуру?}
{Now what about situations when one structure define another structure inside?}

\lstinputlisting{15_structs/15_6.c}

\dots \IFRU{в этом случае, оба поля \TT{inner\_struct} просто будут располагаться между полями a,b и d,e в 
\TT{outer\_struct}.}
{in this case, both \TT{inner\_struct} fields will be placed between a,b and d,e fields of
\TT{outer\_struct}.}

\IFRU{Компилируем}{Let's compile} (MSVC 2010):

\lstinputlisting[caption=MSVC 2010]{15_structs/15_6_msvc.asm}

\IFRU{Очень любопытный момент в том, что глядя на этот код на ассемблере, мы даже не видим, 
что была использована какая-то еще другая структура внутри этой!
Так что, пожалуй, можно сказать, что все вложенные структуры в итоге разворачиваются в одну, \IT{линейную} 
или \IT{одномерную} структуру.}
{One curious point here is that by looking onto this assembly code, we do not even see that
another structure was used inside of it!
Thus, we would say, nested structures are finally unfolds into \IT{linear} or \IT{one-dimensional} structure.}

\IFRU{Конечно, если заменить объявление \TT{struct inner\_struct c;} на \TT{struct inner\_struct *c;} 
(объявляя таким образом указатель), ситауция будет совсем иная.}
{Of course, if to replace \TT{struct inner\_struct c;} declaration to \TT{struct inner\_struct *c;} 
(thus making a pointer here) situation will be significally different.}



\subsection{\IFRU{Вложенные структуры}{Nested structures}}

\IFRU{Теперь, как насчет ситуаций, когда одна структура определяет внутри себя еще одну структуру?}
{Now what about situations when one structure defines another structure inside?}

\lstinputlisting{patterns/15_structs/nested.c}

\dots \IFRU{в этом случае, оба поля \TT{inner\_struct} просто будут располагаться между полями a,b и d,e в 
\TT{outer\_struct}.}
{in this case, both \TT{inner\_struct} fields will be placed between a,b and d,e fields of
\TT{outer\_struct}.}

\IFRU{Компилируем}{Let's compile} (MSVC 2010):

\lstinputlisting[caption=MSVC 2010]{patterns/15_structs/nested_msvc.asm}

\IFRU{Очень любопытный момент в том, что глядя на этот код на ассемблере, мы даже не видим, 
что была использована какая-то еще другая структура внутри этой!
Так что, пожалуй, можно сказать, что все вложенные структуры в итоге разворачиваются в одну, \IT{линейную} 
или \IT{одномерную} структуру.}
{One curious point here is that by looking onto this assembly code, we do not even see that
another structure was used inside of it!
Thus, we would say, nested structures are finally unfolds into \IT{linear} or \IT{one-dimensional} structure.}

\IFRU{Конечно, если заменить объявление \TT{struct inner\_struct c;} на \TT{struct inner\_struct *c;} 
(объявляя таким образом указатель), ситауция будет совсем иная.}
{Of course, if to replace \TT{struct inner\_struct c;} declaration to \TT{struct inner\_struct *c;} 
(thus making a pointer here) situation will be quite different.}



\subsection{\IFRU{Работа с битовыми полями в структуре}{Bit fields in structure}}

\subsubsection{\IFRU{Пример CPUID}{CPUID example}}

\IFRU{Язык \CCpp позволяет укзывать, сколько именно бит отвести для каждого поля структуры. 
Это удобно если нужно экономить место в памяти. К примеру, для переменной типа \Tbool достаточно одного бита.
Но, это не очень удобно, если нужна скорость.}
{\CCpp language allow to define exact number of bits for each structure fields.
It's very useful if one needs to save memory space. 
For example, one bit is enough for variable of \Tbool type.
But of course, it's not rational if speed is important.}

\newcommand{\FNCPUID}{\footnote{\url{http://en.wikipedia.org/wiki/CPUID}}}

\index{x86!\Instructions!CPUID}
\IFRU{Рассмотрим пример с инструкцией \CPUID\FNCPUID. 
Эта инструкция возвращает информацию о том, какой процессор имеется в наличии и какие фичи он имеет.}
{Let's consider \CPUID\FNCPUID instruction example.
This instruction returning information about current CPU and its features.}

\IFRU{Если перед исполнением инструкции в \EAX будет 1, 
то \CPUID вернет упакованную в \EAX такую информацию о процессоре:}
{If the \EAX is set to 1 before instruction execution, 
\CPUID will return this information packed into the \EAX register:}

\begin{center}
\begin{tabular}{ | l | l | }
\hline
3:0 & Stepping \\
7:4 & Model \\
11:8 & Family \\
13:12 & Processor Type \\
19:16 & Extended Model \\
27:20 & Extended Family \\
\hline
\end{tabular}
\end{center}

\newcommand{\FNGCCAS}{\footnote{\href{http://www.ibiblio.org/gferg/ldp/GCC-Inline-Assembly-HOWTO.html}
{\IFRU{Подробнее о встроенном ассемблере GCC}{More about internal GCC assembler}}}}

\IFRU{MSVC 2010 имеет макрос для \CPUID, а GCC 4.4.1 ~--- нет. 
Поэтому для GCC сделаем эту функцию сами, используя его встроенный ассемблер\FNGCCAS.}
{MSVC 2010 has \CPUID macro, but GCC 4.4.1 ~--- hasn't.
So let's make this function by yourself for GCC with the help of its built-in assembler\FNGCCAS.}

\lstinputlisting{15_structs/CPUID.c}

\IFRU{После того как \CPUID заполнит \EAX/\EBX/\ECX/\EDX, у нас они отразятся в массиве \TT{b[]}. 
Затем, мы имеем указатель на структуру \TT{CPUID\_1\_EAX}, и мы указываем его на значение 
\EAX из массива \TT{b[]}.}
{After \CPUID will fill \EAX/\EBX/\ECX/\EDX, these registers will be reflected in the \TT{b[]} array.
Then, we have a pointer to the \TT{CPUID\_1\_EAX} structure and we point it to the value in the \EAX from \TT{b[]} array.}

\IFRU{Иными словами, мы трактуем 32-битный \Tint как структуру.}
{In other words, we treat 32-bit \Tint value as a structure.}

\IFRU{Затем мы читаем из структуры.}{Then we read from the stucture.}

\IFRU{Компилируем в MSVC 2008 с опцией \Ox}{Let's compile it in MSVC 2008 with \Ox option}:

\lstinputlisting[caption=\Optimizing MSVC 2008]{15_structs/CPUID_msvc_Ox.asm}

\index{x86!\Instructions!SHR}
\IFRU{Инструкция \TT{SHR} сдвигает значение из \EAX на то количество бит, 
которое нужно \IT{пропустить}, то есть, мы игнорируем некоторые биты \IT{справа}.}
{\TT{SHR} instruction shifting value in the \EAX register by number of bits should be
\IT{skipped}, e.g., we ignore some bits \IT{at right}.}

\index{x86!\Instructions!AND}
\IFRU{А инструкция \AND очищает биты \IT{слева} которые нам не нужны, или же, говоря иначе, 
она оставляет по маске только те биты в \EAX, которые нам сейчас нужны.}
{\AND instruction clears bits not needed \IT{at left}, or, in other words, 
leaves only those bits in the \EAX register we need now.}

\IFRU{Попробуем GCC 4.4.1 с опцией \Othree.}{Let's try GCC 4.4.1 with \Othree option.}

\lstinputlisting[caption=\Optimizing GCC 4.4.1]{15_structs/CPUID_gcc_O3.asm}

\IFRU{Практически, то же самое. Единственное что стоит отметить это то, что GCC решил зачем-то объеденить 
вычисление \TT{extended\_model\_id} и \TT{extended\_family\_id} в один блок, 
вместо того чтобы вычислять их перед соответствующим вызовом \printf.}
{Almost the same. The only thing worth noting is that GCC somehow united calculation of
\TT{extended\_model\_id} and \TT{extended\_family\_id} into one block,
instead of calculating them separately, before corresponding each \printf call.}

\subsubsection{\WorkingWithFloatAsWithStructSubSubSectionName}
\label{sec:floatasstruct}

\IFRU{Как уже раннее указывалось в секции о FPU~\ref{sec:FPU}, и \Tfloat и \Tdouble содержат в себе знак, 
мантиссу и экспоненту. 
Однако, можем ли мы работать с этими полями напрямую? Попробуем с \Tfloat.}
{As it was already noted in section about FPU~\ref{sec:FPU}, both \Tfloat and \Tdouble types consisted of sign,
significand (or fraction) and exponent.
But will we able to work with these fields directly? Let's try with \Tfloat.}

\index{IEEE 754}
\index{float}
\begin{figure}[ht!]
\centering
\includegraphics[scale=0.66]{15_structs/500px-Float_example.png}
\caption{\IFRU{Формат значения float\protect\footnotemark}
{float value format\protect\footnotemark}}
\end{figure}

\footnotetext{\IFRU{иллюстрация взята из}{illustration taken from} wikipedia}

\lstinputlisting{15_structs/float_en.c}

\IFRU{Структура \TT{float\_as\_struct} занимает в памяти столько же места сколько и \Tfloat, 
то есть 4 байта или 32 бита.}
{\TT{float\_as\_struct} structure occupies as much space is memory as \Tfloat, e.g., 4 bytes or 32 bits.}

\IFRU{Далее мы выставляем во входящем значении отрицательный знак, 
а также прибавляя двойку к экспоненте, мы тем 
самым умножаем всё значение на \TT{$2^2$}, то есть на 4.}
{Now we setting negative sign in input value and also by addding 2 to exponent we thereby multiplicating
the whole number by \TT{$2^2$}, e.g., by 4.}

\IFRU{Компилируем в MSVC 2008 без оптимизации:}{Let's compile in MSVC 2008 without optimization:}

\lstinputlisting[caption=\NonOptimizing MSVC 2008]
{\IFRU{15_structs/float_msvc_ru.asm}{15_structs/float_msvc_en.asm}}

\IFRU{Слекга избыточно. В версии скомпилированной с флагом \Ox нет вызовов \TT{memcpy()}, 
там работа происходит сразу с переменной f. Но по неоптимизированной версии будет проще понять.}
{Redundant for a bit. If it compiled with \Ox flag there are no \TT{memcpy()} call,
f variable is used directly. But it's easier to understand it all considering unoptimized version.}

\IFRU{А что сделает GCC 4.4.1 с опцией \TT{-O3}?}{What GCC 4.4.1 with \TT{-O3} will do?}

\lstinputlisting[caption=\Optimizing GCC 4.4.1]
{\IFRU{15_structs/float_gcc_O3_ru.asm}{15_structs/float_gcc_O3_en.asm}}

\IFRU{Да, функция \TT{f()} в целом понятна. Однако, что интересно, еще при компиляции, 
не взирая на мешанину с полями структуры, GCC умудрился вычислить результат функции \TT{f(1.234)} и 
сразу подставить его в аргумент для \printf{}!}
{The \TT{f()} function is almost understandable. However, what is interesting, GCC was able to calculate
\TT{f(1.234)} result during compilation stage despite all this hodge-podge with structure fields
and prepared this argument to the \printf{} as precalculated!}





\chapter{\RU{Файл сохранения состояния в игре Millenium}\EN{Millenium game save file}}
\label{Millenium_DOS_game}
\index{MS-DOS}

\RU{Игра}\EN{The} \q{Millenium Return to Earth} \RU{под DOS довольно древняя (1991), позволяющая
добывать ресурсы, строить корабли, снаряжать их на другие планеты,\etc{}.}
\EN{is an ancient DOS game (1991), that allows you to mine resources, build ships,
equip them on other planets, and so on}\footnote{\RU{Её можно скачать бесплатно}\EN{It can be downloaded for free}
\href{http://go.yurichev.com/17316}{\RU{здесь}\EN{here}}}.

\RU{Как и многие другие игры, она позволяет сохранять состояние игры в файл.}
\EN{Like many other games, it allows you to save all game state into a file.}

\RU{Посмотрим, сможем ли мы найти что-нибудь в нем}\EN{Let's see if we can find something in it}.

\clearpage
\RU{В игре есть шахта}\EN{So there is a mine in the game}.
\RU{Шахты на некоторых планетах работают быстрее, на некоторых других --- медленнее}\EN{Mines at some planets 
work faster, or slower on others}. 
\RU{Набор ресурсов также разный}\EN{The set of resources is also different}.

\RU{Здесь видно, какие ресурсы добыты в этот момент}\EN{Here we can see what resources are mined at the time}: 

\begin{figure}[H]
\centering
\includegraphics[scale=\FigScale]{ff/millenium/1.png}
\caption{\RU{Шахта: первое состояние}\EN{Mine: state 1}}
\label{fig:mill_1}
\end{figure}

\RU{Сохраним состояние игры}\EN{Let's save a game state}.
\RU{Это файл размером}\EN{This is a file of size} 9538 \RU{байт}\EN{bytes}.

\RU{Подождем несколько \q{дней} здесь в игре и теперь в шахте добыто больше ресурсов}%
\EN{Let's wait some \q{days} here in the game, and now we've got more resources from the mine}:

\begin{figure}[H]
\centering
\includegraphics[scale=\FigScale]{ff/millenium/2.png}
\caption{\RU{Шахта: второе состояние}\EN{Mine: state 2}}
\label{fig:mill_2}
\end{figure}

\RU{Снова сохраним состояние игры}\EN{Let's sav game state again}.

\RU{Теперь просто попробуем сравнить оба файла побайтово используя простую утилиту FC под DOS/Windows:}
\EN{Now let's try to just do binary comparison of the save files using the simple DOS/Windows FC utility:}

\lstinputlisting{ff/millenium/fc_result.txt}

\RU{Вывод здесь неполный, там было больше отличий, но мы обрежем результат до самого интересного.}%
\EN{The output is incomplete here, there are more differences, but we will cut result to show the most interesting.}

\RU{В первой версии у нас было 14 единиц водорода (hydrogen) и 102 --- кислорода (oxygen).}
\EN{In the first state, we have 14 \q{units} of hydrogen and 102 \q{units} of oxygen.}
\RU{Во второй версии у нас 22 и 155 единиц соответственно.}
\EN{We have 22 and 155 \q{units} respectively in the second state.}
\RU{Если эти значения сохраняются в файл, мы должны увидеть разницу}\EN{If these values are saved into 
the save file, we would see this in the difference}.
\RU{И она действительно есть}\EN{And indeed we do}. 
\RU{Там}\EN{There is} 0x0E (14) \RU{на позиции}\EN{at position} 0xBDA \RU{и это значение}\EN{and this value is} 
0x16 (22) \RU{в новой версии файла}\EN{in the new version of the file}.
\RU{Это, наверное, водород}\EN{This is probably hydrogen}.
\RU{Там также}\EN{There is} 0x66 (102) \RU{на позиции}\EN{at position} 0xBDC \RU{в старой версии и}\EN{in the old 
version and} 0x9B (155) \RU{в новой версии файла}\EN{in the new version of the file}. 
\RU{Это, наверное, кислород}\EN{This seems to be the oxygen}.

\RU{Обе версии файла доступны на сайте, для тех кто хочет их изучить (или поэкспериментировать)}%
\EN{Both files are available on the website for those who wants to inspect them (or experiment) more}: 
\href{http://go.yurichev.com/17212}{beginners.re}.

\clearpage
\RU{Новую версию файла откроем в Hiew и отметим значения, связанные с ресурсами, добытыми на шахте в игре}%
\EN{Here is the new version of file opened in Hiew, we marked the values related to the resources mined in the game}: 

\begin{figure}[H]
\centering
\includegraphics[scale=\FigScale]{ff/millenium/hiew3.png}
\caption{Hiew: \RU{первое состояние}\EN{state 1}}
\label{fig:mill_hiew3}
\end{figure}

\RU{Проверим каждое, и это они}\EN{Let's check each, and these are}.
\RU{Это явно 16-битные значения: не удивительно для 16-битной программы под DOS, где \Tint имел длину в 16 бит.}
\EN{These are clearly 16-bit values: not a strange thing for 16-bit DOS software where the \Tint type has 16-bit width.}

\clearpage
\RU{Проверим наши предположения}\EN{Let's check our assumptions}.
\RU{Запишем 1234 (0x4D2) на первой позиции (это должен быть водород)}%
\EN{We will write the 1234 (0x4D2) value at the first position (this must be hydrogen)}:

\begin{figure}[H]
\centering
\includegraphics[scale=\FigScale]{ff/millenium/hiew4.png}
\caption{Hiew: \RU{запишем там}\EN{let's write 1234} (0x4D2)\EN{ there}}
\label{fig:mill_hiew4}
\end{figure}

\RU{Затем загрузим измененный файл в игру и посмотрим на статистику в шахте}%
\EN{Then we will load the changed file in the game and took a look at mine statistics}:

\begin{figure}[H]
\centering
\includegraphics[scale=\FigScale]{ff/millenium/5.png}
\caption{\RU{Проверим значение водорода}\EN{Let's check for hydrogen value}}
\label{fig:mill_5}
\end{figure}

\RU{Так что да, это оно}\EN{So yes, this is it}.

\clearpage
\RU{Попробуем пройти игру как можно быстрее, установим максимальные значения везде}\EN{Now let's try to 
finish the game as soon as possible, set the maximal values everywhere}:

\begin{figure}[H]
\centering
\includegraphics[scale=\FigScale]{ff/millenium/hiew7.png}
\caption{Hiew: \RU{установим максимальные значения}\EN{let's set maximal values}}
\label{fig:mill_hiew7}
\end{figure}

0xFFFF \RU{это}\EN{is} 65535, \RU{так что да, у нас много ресурсов теперь}\EN{so yes, we now have a 
lot of resources}:

\begin{figure}[H]
\centering
\includegraphics[scale=\FigScale]{ff/millenium/6.png}
\caption{\RU{Все ресурсы теперь действительно}\EN{All resources are} 65535 (0xFFFF)\EN{ indeed}}
\label{fig:mill_6}
\end{figure}

\clearpage
\RU{Пропустим еще несколько \q{дней} в игре и видим что-то неладное}\EN{Let's skip some \q{days} in the game and oops}! 
\RU{Некоторых ресурсов стало меньше}\EN{We have a lower amount of some resources}:

\begin{figure}[H]
\centering
\includegraphics[scale=\FigScale]{ff/millenium/8.png}
\caption{\RU{Переполнение переменных ресурсов}\EN{Resource variables overflow}}
\label{fig:mill_8}
\end{figure}

\RU{Это просто переполнение}\EN{That's just overflow}. 
\RU{Разработчик игры вероятно никогда не думал, что значения ресурсов будут такими большими,
так что, здесь, наверное, нет проверок на переполнение, но шахта в игре \q{работает}, ресурсы добавляются,
отсюда и переполнение.}
\EN{The game's developer probably didn't think about such high amounts of resources,
so there are probably no overflow checks, but the mine is \q{working} in the game, resources are added,
hence the overflows.}
\RU{Вероятно, не нужно было жадничать}\EN{Apparently, it was a bad idea to be that greedy}.

\RU{Здесь наверняка еще какие-то значения в этом файле}\EN{There are probably a lot of more values 
saved in this file}.

\RU{Так что это очень простой способ читинга в играх}\EN{So this is very simple method of cheating in games}.
\RU{Файл с таблицей очков также можно легко модифицировать}\EN{High score files often can be easily 
patched like that}.

\EN{More about files and memory snapshots comparing}\RU{Еще насчет сравнения файлов и снимков памяти}: 
\myref{snapshots_comparing}.

\chapter{\RU{Файл сохранения состояния в игре Millenium}\EN{Millenium game save file}}
\label{Millenium_DOS_game}
\index{MS-DOS}

\RU{Игра}\EN{The} \q{Millenium Return to Earth} \RU{под DOS довольно древняя (1991), позволяющая
добывать ресурсы, строить корабли, снаряжать их на другие планеты,\etc{}.}
\EN{is an ancient DOS game (1991), that allows you to mine resources, build ships,
equip them on other planets, and so on}\footnote{\RU{Её можно скачать бесплатно}\EN{It can be downloaded for free}
\href{http://go.yurichev.com/17316}{\RU{здесь}\EN{here}}}.

\RU{Как и многие другие игры, она позволяет сохранять состояние игры в файл.}
\EN{Like many other games, it allows you to save all game state into a file.}

\RU{Посмотрим, сможем ли мы найти что-нибудь в нем}\EN{Let's see if we can find something in it}.

\clearpage
\RU{В игре есть шахта}\EN{So there is a mine in the game}.
\RU{Шахты на некоторых планетах работают быстрее, на некоторых других --- медленнее}\EN{Mines at some planets 
work faster, or slower on others}. 
\RU{Набор ресурсов также разный}\EN{The set of resources is also different}.

\RU{Здесь видно, какие ресурсы добыты в этот момент}\EN{Here we can see what resources are mined at the time}: 

\begin{figure}[H]
\centering
\includegraphics[scale=\FigScale]{ff/millenium/1.png}
\caption{\RU{Шахта: первое состояние}\EN{Mine: state 1}}
\label{fig:mill_1}
\end{figure}

\RU{Сохраним состояние игры}\EN{Let's save a game state}.
\RU{Это файл размером}\EN{This is a file of size} 9538 \RU{байт}\EN{bytes}.

\RU{Подождем несколько \q{дней} здесь в игре и теперь в шахте добыто больше ресурсов}%
\EN{Let's wait some \q{days} here in the game, and now we've got more resources from the mine}:

\begin{figure}[H]
\centering
\includegraphics[scale=\FigScale]{ff/millenium/2.png}
\caption{\RU{Шахта: второе состояние}\EN{Mine: state 2}}
\label{fig:mill_2}
\end{figure}

\RU{Снова сохраним состояние игры}\EN{Let's sav game state again}.

\RU{Теперь просто попробуем сравнить оба файла побайтово используя простую утилиту FC под DOS/Windows:}
\EN{Now let's try to just do binary comparison of the save files using the simple DOS/Windows FC utility:}

\lstinputlisting{ff/millenium/fc_result.txt}

\RU{Вывод здесь неполный, там было больше отличий, но мы обрежем результат до самого интересного.}%
\EN{The output is incomplete here, there are more differences, but we will cut result to show the most interesting.}

\RU{В первой версии у нас было 14 единиц водорода (hydrogen) и 102 --- кислорода (oxygen).}
\EN{In the first state, we have 14 \q{units} of hydrogen and 102 \q{units} of oxygen.}
\RU{Во второй версии у нас 22 и 155 единиц соответственно.}
\EN{We have 22 and 155 \q{units} respectively in the second state.}
\RU{Если эти значения сохраняются в файл, мы должны увидеть разницу}\EN{If these values are saved into 
the save file, we would see this in the difference}.
\RU{И она действительно есть}\EN{And indeed we do}. 
\RU{Там}\EN{There is} 0x0E (14) \RU{на позиции}\EN{at position} 0xBDA \RU{и это значение}\EN{and this value is} 
0x16 (22) \RU{в новой версии файла}\EN{in the new version of the file}.
\RU{Это, наверное, водород}\EN{This is probably hydrogen}.
\RU{Там также}\EN{There is} 0x66 (102) \RU{на позиции}\EN{at position} 0xBDC \RU{в старой версии и}\EN{in the old 
version and} 0x9B (155) \RU{в новой версии файла}\EN{in the new version of the file}. 
\RU{Это, наверное, кислород}\EN{This seems to be the oxygen}.

\RU{Обе версии файла доступны на сайте, для тех кто хочет их изучить (или поэкспериментировать)}%
\EN{Both files are available on the website for those who wants to inspect them (or experiment) more}: 
\href{http://go.yurichev.com/17212}{beginners.re}.

\clearpage
\RU{Новую версию файла откроем в Hiew и отметим значения, связанные с ресурсами, добытыми на шахте в игре}%
\EN{Here is the new version of file opened in Hiew, we marked the values related to the resources mined in the game}: 

\begin{figure}[H]
\centering
\includegraphics[scale=\FigScale]{ff/millenium/hiew3.png}
\caption{Hiew: \RU{первое состояние}\EN{state 1}}
\label{fig:mill_hiew3}
\end{figure}

\RU{Проверим каждое, и это они}\EN{Let's check each, and these are}.
\RU{Это явно 16-битные значения: не удивительно для 16-битной программы под DOS, где \Tint имел длину в 16 бит.}
\EN{These are clearly 16-bit values: not a strange thing for 16-bit DOS software where the \Tint type has 16-bit width.}

\clearpage
\RU{Проверим наши предположения}\EN{Let's check our assumptions}.
\RU{Запишем 1234 (0x4D2) на первой позиции (это должен быть водород)}%
\EN{We will write the 1234 (0x4D2) value at the first position (this must be hydrogen)}:

\begin{figure}[H]
\centering
\includegraphics[scale=\FigScale]{ff/millenium/hiew4.png}
\caption{Hiew: \RU{запишем там}\EN{let's write 1234} (0x4D2)\EN{ there}}
\label{fig:mill_hiew4}
\end{figure}

\RU{Затем загрузим измененный файл в игру и посмотрим на статистику в шахте}%
\EN{Then we will load the changed file in the game and took a look at mine statistics}:

\begin{figure}[H]
\centering
\includegraphics[scale=\FigScale]{ff/millenium/5.png}
\caption{\RU{Проверим значение водорода}\EN{Let's check for hydrogen value}}
\label{fig:mill_5}
\end{figure}

\RU{Так что да, это оно}\EN{So yes, this is it}.

\clearpage
\RU{Попробуем пройти игру как можно быстрее, установим максимальные значения везде}\EN{Now let's try to 
finish the game as soon as possible, set the maximal values everywhere}:

\begin{figure}[H]
\centering
\includegraphics[scale=\FigScale]{ff/millenium/hiew7.png}
\caption{Hiew: \RU{установим максимальные значения}\EN{let's set maximal values}}
\label{fig:mill_hiew7}
\end{figure}

0xFFFF \RU{это}\EN{is} 65535, \RU{так что да, у нас много ресурсов теперь}\EN{so yes, we now have a 
lot of resources}:

\begin{figure}[H]
\centering
\includegraphics[scale=\FigScale]{ff/millenium/6.png}
\caption{\RU{Все ресурсы теперь действительно}\EN{All resources are} 65535 (0xFFFF)\EN{ indeed}}
\label{fig:mill_6}
\end{figure}

\clearpage
\RU{Пропустим еще несколько \q{дней} в игре и видим что-то неладное}\EN{Let's skip some \q{days} in the game and oops}! 
\RU{Некоторых ресурсов стало меньше}\EN{We have a lower amount of some resources}:

\begin{figure}[H]
\centering
\includegraphics[scale=\FigScale]{ff/millenium/8.png}
\caption{\RU{Переполнение переменных ресурсов}\EN{Resource variables overflow}}
\label{fig:mill_8}
\end{figure}

\RU{Это просто переполнение}\EN{That's just overflow}. 
\RU{Разработчик игры вероятно никогда не думал, что значения ресурсов будут такими большими,
так что, здесь, наверное, нет проверок на переполнение, но шахта в игре \q{работает}, ресурсы добавляются,
отсюда и переполнение.}
\EN{The game's developer probably didn't think about such high amounts of resources,
so there are probably no overflow checks, but the mine is \q{working} in the game, resources are added,
hence the overflows.}
\RU{Вероятно, не нужно было жадничать}\EN{Apparently, it was a bad idea to be that greedy}.

\RU{Здесь наверняка еще какие-то значения в этом файле}\EN{There are probably a lot of more values 
saved in this file}.

\RU{Так что это очень простой способ читинга в играх}\EN{So this is very simple method of cheating in games}.
\RU{Файл с таблицей очков также можно легко модифицировать}\EN{High score files often can be easily 
patched like that}.

\EN{More about files and memory snapshots comparing}\RU{Еще насчет сравнения файлов и снимков памяти}: 
\myref{snapshots_comparing}.

\chapter{\RU{Файл сохранения состояния в игре Millenium}\EN{Millenium game save file}}
\label{Millenium_DOS_game}
\index{MS-DOS}

\RU{Игра}\EN{The} \q{Millenium Return to Earth} \RU{под DOS довольно древняя (1991), позволяющая
добывать ресурсы, строить корабли, снаряжать их на другие планеты,\etc{}.}
\EN{is an ancient DOS game (1991), that allows you to mine resources, build ships,
equip them on other planets, and so on}\footnote{\RU{Её можно скачать бесплатно}\EN{It can be downloaded for free}
\href{http://go.yurichev.com/17316}{\RU{здесь}\EN{here}}}.

\RU{Как и многие другие игры, она позволяет сохранять состояние игры в файл.}
\EN{Like many other games, it allows you to save all game state into a file.}

\RU{Посмотрим, сможем ли мы найти что-нибудь в нем}\EN{Let's see if we can find something in it}.

\clearpage
\RU{В игре есть шахта}\EN{So there is a mine in the game}.
\RU{Шахты на некоторых планетах работают быстрее, на некоторых других --- медленнее}\EN{Mines at some planets 
work faster, or slower on others}. 
\RU{Набор ресурсов также разный}\EN{The set of resources is also different}.

\RU{Здесь видно, какие ресурсы добыты в этот момент}\EN{Here we can see what resources are mined at the time}: 

\begin{figure}[H]
\centering
\includegraphics[scale=\FigScale]{ff/millenium/1.png}
\caption{\RU{Шахта: первое состояние}\EN{Mine: state 1}}
\label{fig:mill_1}
\end{figure}

\RU{Сохраним состояние игры}\EN{Let's save a game state}.
\RU{Это файл размером}\EN{This is a file of size} 9538 \RU{байт}\EN{bytes}.

\RU{Подождем несколько \q{дней} здесь в игре и теперь в шахте добыто больше ресурсов}%
\EN{Let's wait some \q{days} here in the game, and now we've got more resources from the mine}:

\begin{figure}[H]
\centering
\includegraphics[scale=\FigScale]{ff/millenium/2.png}
\caption{\RU{Шахта: второе состояние}\EN{Mine: state 2}}
\label{fig:mill_2}
\end{figure}

\RU{Снова сохраним состояние игры}\EN{Let's sav game state again}.

\RU{Теперь просто попробуем сравнить оба файла побайтово используя простую утилиту FC под DOS/Windows:}
\EN{Now let's try to just do binary comparison of the save files using the simple DOS/Windows FC utility:}

\lstinputlisting{ff/millenium/fc_result.txt}

\RU{Вывод здесь неполный, там было больше отличий, но мы обрежем результат до самого интересного.}%
\EN{The output is incomplete here, there are more differences, but we will cut result to show the most interesting.}

\RU{В первой версии у нас было 14 единиц водорода (hydrogen) и 102 --- кислорода (oxygen).}
\EN{In the first state, we have 14 \q{units} of hydrogen and 102 \q{units} of oxygen.}
\RU{Во второй версии у нас 22 и 155 единиц соответственно.}
\EN{We have 22 and 155 \q{units} respectively in the second state.}
\RU{Если эти значения сохраняются в файл, мы должны увидеть разницу}\EN{If these values are saved into 
the save file, we would see this in the difference}.
\RU{И она действительно есть}\EN{And indeed we do}. 
\RU{Там}\EN{There is} 0x0E (14) \RU{на позиции}\EN{at position} 0xBDA \RU{и это значение}\EN{and this value is} 
0x16 (22) \RU{в новой версии файла}\EN{in the new version of the file}.
\RU{Это, наверное, водород}\EN{This is probably hydrogen}.
\RU{Там также}\EN{There is} 0x66 (102) \RU{на позиции}\EN{at position} 0xBDC \RU{в старой версии и}\EN{in the old 
version and} 0x9B (155) \RU{в новой версии файла}\EN{in the new version of the file}. 
\RU{Это, наверное, кислород}\EN{This seems to be the oxygen}.

\RU{Обе версии файла доступны на сайте, для тех кто хочет их изучить (или поэкспериментировать)}%
\EN{Both files are available on the website for those who wants to inspect them (or experiment) more}: 
\href{http://go.yurichev.com/17212}{beginners.re}.

\clearpage
\RU{Новую версию файла откроем в Hiew и отметим значения, связанные с ресурсами, добытыми на шахте в игре}%
\EN{Here is the new version of file opened in Hiew, we marked the values related to the resources mined in the game}: 

\begin{figure}[H]
\centering
\includegraphics[scale=\FigScale]{ff/millenium/hiew3.png}
\caption{Hiew: \RU{первое состояние}\EN{state 1}}
\label{fig:mill_hiew3}
\end{figure}

\RU{Проверим каждое, и это они}\EN{Let's check each, and these are}.
\RU{Это явно 16-битные значения: не удивительно для 16-битной программы под DOS, где \Tint имел длину в 16 бит.}
\EN{These are clearly 16-bit values: not a strange thing for 16-bit DOS software where the \Tint type has 16-bit width.}

\clearpage
\RU{Проверим наши предположения}\EN{Let's check our assumptions}.
\RU{Запишем 1234 (0x4D2) на первой позиции (это должен быть водород)}%
\EN{We will write the 1234 (0x4D2) value at the first position (this must be hydrogen)}:

\begin{figure}[H]
\centering
\includegraphics[scale=\FigScale]{ff/millenium/hiew4.png}
\caption{Hiew: \RU{запишем там}\EN{let's write 1234} (0x4D2)\EN{ there}}
\label{fig:mill_hiew4}
\end{figure}

\RU{Затем загрузим измененный файл в игру и посмотрим на статистику в шахте}%
\EN{Then we will load the changed file in the game and took a look at mine statistics}:

\begin{figure}[H]
\centering
\includegraphics[scale=\FigScale]{ff/millenium/5.png}
\caption{\RU{Проверим значение водорода}\EN{Let's check for hydrogen value}}
\label{fig:mill_5}
\end{figure}

\RU{Так что да, это оно}\EN{So yes, this is it}.

\clearpage
\RU{Попробуем пройти игру как можно быстрее, установим максимальные значения везде}\EN{Now let's try to 
finish the game as soon as possible, set the maximal values everywhere}:

\begin{figure}[H]
\centering
\includegraphics[scale=\FigScale]{ff/millenium/hiew7.png}
\caption{Hiew: \RU{установим максимальные значения}\EN{let's set maximal values}}
\label{fig:mill_hiew7}
\end{figure}

0xFFFF \RU{это}\EN{is} 65535, \RU{так что да, у нас много ресурсов теперь}\EN{so yes, we now have a 
lot of resources}:

\begin{figure}[H]
\centering
\includegraphics[scale=\FigScale]{ff/millenium/6.png}
\caption{\RU{Все ресурсы теперь действительно}\EN{All resources are} 65535 (0xFFFF)\EN{ indeed}}
\label{fig:mill_6}
\end{figure}

\clearpage
\RU{Пропустим еще несколько \q{дней} в игре и видим что-то неладное}\EN{Let's skip some \q{days} in the game and oops}! 
\RU{Некоторых ресурсов стало меньше}\EN{We have a lower amount of some resources}:

\begin{figure}[H]
\centering
\includegraphics[scale=\FigScale]{ff/millenium/8.png}
\caption{\RU{Переполнение переменных ресурсов}\EN{Resource variables overflow}}
\label{fig:mill_8}
\end{figure}

\RU{Это просто переполнение}\EN{That's just overflow}. 
\RU{Разработчик игры вероятно никогда не думал, что значения ресурсов будут такими большими,
так что, здесь, наверное, нет проверок на переполнение, но шахта в игре \q{работает}, ресурсы добавляются,
отсюда и переполнение.}
\EN{The game's developer probably didn't think about such high amounts of resources,
so there are probably no overflow checks, but the mine is \q{working} in the game, resources are added,
hence the overflows.}
\RU{Вероятно, не нужно было жадничать}\EN{Apparently, it was a bad idea to be that greedy}.

\RU{Здесь наверняка еще какие-то значения в этом файле}\EN{There are probably a lot of more values 
saved in this file}.

\RU{Так что это очень простой способ читинга в играх}\EN{So this is very simple method of cheating in games}.
\RU{Файл с таблицей очков также можно легко модифицировать}\EN{High score files often can be easily 
patched like that}.

\EN{More about files and memory snapshots comparing}\RU{Еще насчет сравнения файлов и снимков памяти}: 
\myref{snapshots_comparing}.

\chapter{\RU{Файл сохранения состояния в игре Millenium}\EN{Millenium game save file}}
\label{Millenium_DOS_game}
\index{MS-DOS}

\RU{Игра}\EN{The} \q{Millenium Return to Earth} \RU{под DOS довольно древняя (1991), позволяющая
добывать ресурсы, строить корабли, снаряжать их на другие планеты,\etc{}.}
\EN{is an ancient DOS game (1991), that allows you to mine resources, build ships,
equip them on other planets, and so on}\footnote{\RU{Её можно скачать бесплатно}\EN{It can be downloaded for free}
\href{http://go.yurichev.com/17316}{\RU{здесь}\EN{here}}}.

\RU{Как и многие другие игры, она позволяет сохранять состояние игры в файл.}
\EN{Like many other games, it allows you to save all game state into a file.}

\RU{Посмотрим, сможем ли мы найти что-нибудь в нем}\EN{Let's see if we can find something in it}.

\clearpage
\RU{В игре есть шахта}\EN{So there is a mine in the game}.
\RU{Шахты на некоторых планетах работают быстрее, на некоторых других --- медленнее}\EN{Mines at some planets 
work faster, or slower on others}. 
\RU{Набор ресурсов также разный}\EN{The set of resources is also different}.

\RU{Здесь видно, какие ресурсы добыты в этот момент}\EN{Here we can see what resources are mined at the time}: 

\begin{figure}[H]
\centering
\includegraphics[scale=\FigScale]{ff/millenium/1.png}
\caption{\RU{Шахта: первое состояние}\EN{Mine: state 1}}
\label{fig:mill_1}
\end{figure}

\RU{Сохраним состояние игры}\EN{Let's save a game state}.
\RU{Это файл размером}\EN{This is a file of size} 9538 \RU{байт}\EN{bytes}.

\RU{Подождем несколько \q{дней} здесь в игре и теперь в шахте добыто больше ресурсов}%
\EN{Let's wait some \q{days} here in the game, and now we've got more resources from the mine}:

\begin{figure}[H]
\centering
\includegraphics[scale=\FigScale]{ff/millenium/2.png}
\caption{\RU{Шахта: второе состояние}\EN{Mine: state 2}}
\label{fig:mill_2}
\end{figure}

\RU{Снова сохраним состояние игры}\EN{Let's sav game state again}.

\RU{Теперь просто попробуем сравнить оба файла побайтово используя простую утилиту FC под DOS/Windows:}
\EN{Now let's try to just do binary comparison of the save files using the simple DOS/Windows FC utility:}

\lstinputlisting{ff/millenium/fc_result.txt}

\RU{Вывод здесь неполный, там было больше отличий, но мы обрежем результат до самого интересного.}%
\EN{The output is incomplete here, there are more differences, but we will cut result to show the most interesting.}

\RU{В первой версии у нас было 14 единиц водорода (hydrogen) и 102 --- кислорода (oxygen).}
\EN{In the first state, we have 14 \q{units} of hydrogen and 102 \q{units} of oxygen.}
\RU{Во второй версии у нас 22 и 155 единиц соответственно.}
\EN{We have 22 and 155 \q{units} respectively in the second state.}
\RU{Если эти значения сохраняются в файл, мы должны увидеть разницу}\EN{If these values are saved into 
the save file, we would see this in the difference}.
\RU{И она действительно есть}\EN{And indeed we do}. 
\RU{Там}\EN{There is} 0x0E (14) \RU{на позиции}\EN{at position} 0xBDA \RU{и это значение}\EN{and this value is} 
0x16 (22) \RU{в новой версии файла}\EN{in the new version of the file}.
\RU{Это, наверное, водород}\EN{This is probably hydrogen}.
\RU{Там также}\EN{There is} 0x66 (102) \RU{на позиции}\EN{at position} 0xBDC \RU{в старой версии и}\EN{in the old 
version and} 0x9B (155) \RU{в новой версии файла}\EN{in the new version of the file}. 
\RU{Это, наверное, кислород}\EN{This seems to be the oxygen}.

\RU{Обе версии файла доступны на сайте, для тех кто хочет их изучить (или поэкспериментировать)}%
\EN{Both files are available on the website for those who wants to inspect them (or experiment) more}: 
\href{http://go.yurichev.com/17212}{beginners.re}.

\clearpage
\RU{Новую версию файла откроем в Hiew и отметим значения, связанные с ресурсами, добытыми на шахте в игре}%
\EN{Here is the new version of file opened in Hiew, we marked the values related to the resources mined in the game}: 

\begin{figure}[H]
\centering
\includegraphics[scale=\FigScale]{ff/millenium/hiew3.png}
\caption{Hiew: \RU{первое состояние}\EN{state 1}}
\label{fig:mill_hiew3}
\end{figure}

\RU{Проверим каждое, и это они}\EN{Let's check each, and these are}.
\RU{Это явно 16-битные значения: не удивительно для 16-битной программы под DOS, где \Tint имел длину в 16 бит.}
\EN{These are clearly 16-bit values: not a strange thing for 16-bit DOS software where the \Tint type has 16-bit width.}

\clearpage
\RU{Проверим наши предположения}\EN{Let's check our assumptions}.
\RU{Запишем 1234 (0x4D2) на первой позиции (это должен быть водород)}%
\EN{We will write the 1234 (0x4D2) value at the first position (this must be hydrogen)}:

\begin{figure}[H]
\centering
\includegraphics[scale=\FigScale]{ff/millenium/hiew4.png}
\caption{Hiew: \RU{запишем там}\EN{let's write 1234} (0x4D2)\EN{ there}}
\label{fig:mill_hiew4}
\end{figure}

\RU{Затем загрузим измененный файл в игру и посмотрим на статистику в шахте}%
\EN{Then we will load the changed file in the game and took a look at mine statistics}:

\begin{figure}[H]
\centering
\includegraphics[scale=\FigScale]{ff/millenium/5.png}
\caption{\RU{Проверим значение водорода}\EN{Let's check for hydrogen value}}
\label{fig:mill_5}
\end{figure}

\RU{Так что да, это оно}\EN{So yes, this is it}.

\clearpage
\RU{Попробуем пройти игру как можно быстрее, установим максимальные значения везде}\EN{Now let's try to 
finish the game as soon as possible, set the maximal values everywhere}:

\begin{figure}[H]
\centering
\includegraphics[scale=\FigScale]{ff/millenium/hiew7.png}
\caption{Hiew: \RU{установим максимальные значения}\EN{let's set maximal values}}
\label{fig:mill_hiew7}
\end{figure}

0xFFFF \RU{это}\EN{is} 65535, \RU{так что да, у нас много ресурсов теперь}\EN{so yes, we now have a 
lot of resources}:

\begin{figure}[H]
\centering
\includegraphics[scale=\FigScale]{ff/millenium/6.png}
\caption{\RU{Все ресурсы теперь действительно}\EN{All resources are} 65535 (0xFFFF)\EN{ indeed}}
\label{fig:mill_6}
\end{figure}

\clearpage
\RU{Пропустим еще несколько \q{дней} в игре и видим что-то неладное}\EN{Let's skip some \q{days} in the game and oops}! 
\RU{Некоторых ресурсов стало меньше}\EN{We have a lower amount of some resources}:

\begin{figure}[H]
\centering
\includegraphics[scale=\FigScale]{ff/millenium/8.png}
\caption{\RU{Переполнение переменных ресурсов}\EN{Resource variables overflow}}
\label{fig:mill_8}
\end{figure}

\RU{Это просто переполнение}\EN{That's just overflow}. 
\RU{Разработчик игры вероятно никогда не думал, что значения ресурсов будут такими большими,
так что, здесь, наверное, нет проверок на переполнение, но шахта в игре \q{работает}, ресурсы добавляются,
отсюда и переполнение.}
\EN{The game's developer probably didn't think about such high amounts of resources,
so there are probably no overflow checks, but the mine is \q{working} in the game, resources are added,
hence the overflows.}
\RU{Вероятно, не нужно было жадничать}\EN{Apparently, it was a bad idea to be that greedy}.

\RU{Здесь наверняка еще какие-то значения в этом файле}\EN{There are probably a lot of more values 
saved in this file}.

\RU{Так что это очень простой способ читинга в играх}\EN{So this is very simple method of cheating in games}.
\RU{Файл с таблицей очков также можно легко модифицировать}\EN{High score files often can be easily 
patched like that}.

\EN{More about files and memory snapshots comparing}\RU{Еще насчет сравнения файлов и снимков памяти}: 
\myref{snapshots_comparing}.

\section{C99 restrict}
\index{\CLanguageElements!C99!restrict}
\index{\CLanguageElements!restrict}
\index{FORTRAN}

\IFRU{А вот причина из-за которой программы на FORTRAN, в некоторых случаях, работают быстрее чем на Си.}
{Here is a reason why FORTRAN programs, in some cases, works faster than \CCpp ones.}

\begin{lstlisting}
void f1 (int* x, int* y, int* sum, int* product, int* sum_product, int* update_me, size_t s)
{
	for (int i=0; i<s; i++)
	{
		sum[i]=x[i]+y[i];
		product[i]=x[i]*y[i];
		update_me[i]=i*123; // some dummy value
		sum_product[i]=sum[i]+product[i];	
	};
};
\end{lstlisting}

\IFRU{Это очень простой пример, в котором есть одна особенность}
{That's very simple example with one specific
thing in it}: 
\IFRU{указатель на массив}{pointer to} \TT{update\_me} \IFRU{может быть указателем на массив}{array could be
a pointer to}
\TT{sum}\IFRU{}{ array}, \TT{product}\IFRU{}{ array}, \IFRU{или даже}{or even} \TT{sum\_product}\IFRU{}{ array} 
~--- \IFRU{ведь нет ничего криминального в том чтобы аргументам функции быть такими, верно?}
{since it is not a crime in it, right?}

\IFRU{Компилятор знает об этом, поэтому генерирует код, где в теле цикла будет 4 основных стадии:}
{Compiler is fully aware about it, so it generates a code with four stages in loop body:}
\begin{itemize}
\item \IFRU{вычислить следующий}{calculate next} \TT{sum[i]}
\item \IFRU{вычислить следующий}{calculate next} \TT{product[i]}
\item \IFRU{вычислить следующий}{calculate next} \TT{update\_me[i]}
\item \IFRU{вычислить следующий}{calculate next} \TT{sum\_product[i]} ~--- 
\IFRU{на этой стадии придется снова загружать из памяти подсчитанные}
{on this stage, we need to load from memory already calculated} \TT{sum[i]} \AndENRU \TT{product[i]}
\end{itemize}

\IFRU{Возможно ли соптимизировать последнюю стадию?}{Is it possible to optimize the last stage?}
\IFRU{Ведь подсчитанные}{Since already calculated} \TT{sum[i]} \AndENRU \TT{product[i]} 
\IFRU{не обязательно снова загружать из памяти, ведь мы их только что подсчитали.}
{are not necessary to load from memory again, becuse we already calculated them.}
\IFRU{Можно, но компилятор не уверен, что на третьей стадии ничего не затерлось!}
{Yes, but compiler is not sure that nothing was overwritten on 3rd stage!}
\IFRU{Это называется}{This is called}
``pointer aliasing'', \IFRU{ситуация, когда компилятор не может быть уверен что память на которую указывает 
какой-то указатель, не изменилась.}
{a situation, when compiler cannot be sure that a memory to which a pointer is pointing, was not changed.}

\IT{restrict} \IFRU{в стандарте Си C99}{in C99 standard}\cite[6.7.3/1]{C99TC3} 
\IFRU{это обещание, даваемое компилятору программистом, что аргументы функции отмеченные этим ключевым словом,
всегда будут указывать на разные места в памяти и пересекаться не будут.}
{is a promise, given by programmer to compiler the function arguments marked by this keyword will always
be pointing to different memory locations and never be crossed.}

\IFRU{Если быть более точным, и описывать это формально, \IT{restrict} показывает, что только данный указатель будет
использоваться для доступа к этому объекту, с которым мы работаем через этот указатель, больше никакой указатель для
этого использоваться не будет.}
{If to be more precise and describe this formally, \IT{restrict} shows that only this pointer is to be used
to access an object, with which we are working via this pointer, and no other pointer will be used for it.}
\IFRU{Можно даже сказать, что к всякому объекту, доступ будет осуществляться только через
один единственный указатель, если он отмечен как}
{It can be even said the object will be accessed
only via one single pointer, if it is marked as} \IT{restrict}.

\IFRU{Добавим это ключевое слово к каждому аргументу-указателю}{Let's add this keyword to each argument-pointer}:

\begin{lstlisting}
void f2 (int* restrict x, int* restrict y, int* restrict sum, int* restrict product, int* restrict sum_product, 
	int* restrict update_me, size_t s)
{
	for (int i=0; i<s; i++)
	{
		sum[i]=x[i]+y[i];
		product[i]=x[i]*y[i];
		update_me[i]=i*123; // some dummy value
		sum_product[i]=sum[i]+product[i];	
	};
};
\end{lstlisting}

\IFRU{Посмотрим результаты}{Let's see results}:

\lstinputlisting[caption=GCC x64: f1()]{21_C99_restrict/f1.asm}

\lstinputlisting[caption=GCC x64: f2()]{21_C99_restrict/f2.asm}

\IFRU{Разница между скомпилированной функцией \TT{f1()} и \TT{f2()} такая}
{The difference between compiled \TT{f1()} and \TT{f2()} function is as follows}:
\InENRU \TT{f1()}, \TT{sum[i]} \AndENRU \TT{product[i]} \IFRU{загружаются снова посреди тела цикла}
{are reloaded in the middle of loop},
\IFRU{а в}{and in} \TT{f2()} \IFRU{этого нет, используются уже подсчитанные значения}
{there are no such thing,
already calculated values are used}, 
\IFRU{ведь мы ``пообещали'' компилятору}{since we ``promised'' to compiler}, 
\IFRU{что никто и ничто не изменит значения в}
{that no one and nothing will change values in} \TT{sum[i]} 
\AndENRU \TT{product[i]} \IFRU{во время исполнения тела цикла}{while execution of loop body}, 
\IFRU{поэтому он ``уверен'', что значения из памяти можно не загружать снова}
{so it is ``sure'' the value from memory may not be loaded again}.
\IFRU{Очевидно, второй вариант будет работать быстрее.}{Obviously, second example will work faster.}

\IFRU{Но что будет если указатели в аргументах функций все же будут пересекаться?}
{But what if pointers in function arguments will be crossed somehow?}
\IFRU{Это останется на совести программиста, а результаты вычислений будут неверными.}
{This will be on programmer's conscience, but results will be incorrect.}

\IFRU{Вернемся к}{Let's back to} FORTRAN. 
\IFRU{Компиляторы с этого ЯП, по умолчанию, все указатели считают таковыми}
{Compilers from this programming language treats all pointers as such}, 
\IFRU{поэтому, когда в Си не было возможности указать}
{so when it was not possible to set} \IT{restrict}, 
FORTRAN \IFRU{в этих случаях мог генерировать более быстрый код}{in these cases may generate faster code}.

\IFRU{Насколько это практично}{How practical is it}? 
\IFRU{Там где функция работает с несколькими большими блоками в памяти.}
{In the cases when function works with several big blocks in memory.}
\IFRU{Такого очень много в линейной алгебре, например.}
{There are a lot of such in linear algebra, for example.}
\IFRU{Очень много линейной алгебры используется на суперкомпьютерах/HPC,
возможно, поэтому, традиционно, там часто используется FORTRAN, до сих пор}
{A lot of linear algebra used on supercomputers/HPC, probably, that is why, traditionally, FORTRAN is still
used there}\cite{Loh:2010:IHP:1810226.1820518}.

\IFRU{Ну а когда итераций цикла не очень много, конечно, тогда прирост скорости не будет ощутимым.}
{But when a number of iterations is not very big,
certainly, speed boost will not be significant.}


\chapter{\RU{Файл сохранения состояния в игре Millenium}\EN{Millenium game save file}}
\label{Millenium_DOS_game}
\index{MS-DOS}

\RU{Игра}\EN{The} \q{Millenium Return to Earth} \RU{под DOS довольно древняя (1991), позволяющая
добывать ресурсы, строить корабли, снаряжать их на другие планеты,\etc{}.}
\EN{is an ancient DOS game (1991), that allows you to mine resources, build ships,
equip them on other planets, and so on}\footnote{\RU{Её можно скачать бесплатно}\EN{It can be downloaded for free}
\href{http://go.yurichev.com/17316}{\RU{здесь}\EN{here}}}.

\RU{Как и многие другие игры, она позволяет сохранять состояние игры в файл.}
\EN{Like many other games, it allows you to save all game state into a file.}

\RU{Посмотрим, сможем ли мы найти что-нибудь в нем}\EN{Let's see if we can find something in it}.

\clearpage
\RU{В игре есть шахта}\EN{So there is a mine in the game}.
\RU{Шахты на некоторых планетах работают быстрее, на некоторых других --- медленнее}\EN{Mines at some planets 
work faster, or slower on others}. 
\RU{Набор ресурсов также разный}\EN{The set of resources is also different}.

\RU{Здесь видно, какие ресурсы добыты в этот момент}\EN{Here we can see what resources are mined at the time}: 

\begin{figure}[H]
\centering
\includegraphics[scale=\FigScale]{ff/millenium/1.png}
\caption{\RU{Шахта: первое состояние}\EN{Mine: state 1}}
\label{fig:mill_1}
\end{figure}

\RU{Сохраним состояние игры}\EN{Let's save a game state}.
\RU{Это файл размером}\EN{This is a file of size} 9538 \RU{байт}\EN{bytes}.

\RU{Подождем несколько \q{дней} здесь в игре и теперь в шахте добыто больше ресурсов}%
\EN{Let's wait some \q{days} here in the game, and now we've got more resources from the mine}:

\begin{figure}[H]
\centering
\includegraphics[scale=\FigScale]{ff/millenium/2.png}
\caption{\RU{Шахта: второе состояние}\EN{Mine: state 2}}
\label{fig:mill_2}
\end{figure}

\RU{Снова сохраним состояние игры}\EN{Let's sav game state again}.

\RU{Теперь просто попробуем сравнить оба файла побайтово используя простую утилиту FC под DOS/Windows:}
\EN{Now let's try to just do binary comparison of the save files using the simple DOS/Windows FC utility:}

\lstinputlisting{ff/millenium/fc_result.txt}

\RU{Вывод здесь неполный, там было больше отличий, но мы обрежем результат до самого интересного.}%
\EN{The output is incomplete here, there are more differences, but we will cut result to show the most interesting.}

\RU{В первой версии у нас было 14 единиц водорода (hydrogen) и 102 --- кислорода (oxygen).}
\EN{In the first state, we have 14 \q{units} of hydrogen and 102 \q{units} of oxygen.}
\RU{Во второй версии у нас 22 и 155 единиц соответственно.}
\EN{We have 22 and 155 \q{units} respectively in the second state.}
\RU{Если эти значения сохраняются в файл, мы должны увидеть разницу}\EN{If these values are saved into 
the save file, we would see this in the difference}.
\RU{И она действительно есть}\EN{And indeed we do}. 
\RU{Там}\EN{There is} 0x0E (14) \RU{на позиции}\EN{at position} 0xBDA \RU{и это значение}\EN{and this value is} 
0x16 (22) \RU{в новой версии файла}\EN{in the new version of the file}.
\RU{Это, наверное, водород}\EN{This is probably hydrogen}.
\RU{Там также}\EN{There is} 0x66 (102) \RU{на позиции}\EN{at position} 0xBDC \RU{в старой версии и}\EN{in the old 
version and} 0x9B (155) \RU{в новой версии файла}\EN{in the new version of the file}. 
\RU{Это, наверное, кислород}\EN{This seems to be the oxygen}.

\RU{Обе версии файла доступны на сайте, для тех кто хочет их изучить (или поэкспериментировать)}%
\EN{Both files are available on the website for those who wants to inspect them (or experiment) more}: 
\href{http://go.yurichev.com/17212}{beginners.re}.

\clearpage
\RU{Новую версию файла откроем в Hiew и отметим значения, связанные с ресурсами, добытыми на шахте в игре}%
\EN{Here is the new version of file opened in Hiew, we marked the values related to the resources mined in the game}: 

\begin{figure}[H]
\centering
\includegraphics[scale=\FigScale]{ff/millenium/hiew3.png}
\caption{Hiew: \RU{первое состояние}\EN{state 1}}
\label{fig:mill_hiew3}
\end{figure}

\RU{Проверим каждое, и это они}\EN{Let's check each, and these are}.
\RU{Это явно 16-битные значения: не удивительно для 16-битной программы под DOS, где \Tint имел длину в 16 бит.}
\EN{These are clearly 16-bit values: not a strange thing for 16-bit DOS software where the \Tint type has 16-bit width.}

\clearpage
\RU{Проверим наши предположения}\EN{Let's check our assumptions}.
\RU{Запишем 1234 (0x4D2) на первой позиции (это должен быть водород)}%
\EN{We will write the 1234 (0x4D2) value at the first position (this must be hydrogen)}:

\begin{figure}[H]
\centering
\includegraphics[scale=\FigScale]{ff/millenium/hiew4.png}
\caption{Hiew: \RU{запишем там}\EN{let's write 1234} (0x4D2)\EN{ there}}
\label{fig:mill_hiew4}
\end{figure}

\RU{Затем загрузим измененный файл в игру и посмотрим на статистику в шахте}%
\EN{Then we will load the changed file in the game and took a look at mine statistics}:

\begin{figure}[H]
\centering
\includegraphics[scale=\FigScale]{ff/millenium/5.png}
\caption{\RU{Проверим значение водорода}\EN{Let's check for hydrogen value}}
\label{fig:mill_5}
\end{figure}

\RU{Так что да, это оно}\EN{So yes, this is it}.

\clearpage
\RU{Попробуем пройти игру как можно быстрее, установим максимальные значения везде}\EN{Now let's try to 
finish the game as soon as possible, set the maximal values everywhere}:

\begin{figure}[H]
\centering
\includegraphics[scale=\FigScale]{ff/millenium/hiew7.png}
\caption{Hiew: \RU{установим максимальные значения}\EN{let's set maximal values}}
\label{fig:mill_hiew7}
\end{figure}

0xFFFF \RU{это}\EN{is} 65535, \RU{так что да, у нас много ресурсов теперь}\EN{so yes, we now have a 
lot of resources}:

\begin{figure}[H]
\centering
\includegraphics[scale=\FigScale]{ff/millenium/6.png}
\caption{\RU{Все ресурсы теперь действительно}\EN{All resources are} 65535 (0xFFFF)\EN{ indeed}}
\label{fig:mill_6}
\end{figure}

\clearpage
\RU{Пропустим еще несколько \q{дней} в игре и видим что-то неладное}\EN{Let's skip some \q{days} in the game and oops}! 
\RU{Некоторых ресурсов стало меньше}\EN{We have a lower amount of some resources}:

\begin{figure}[H]
\centering
\includegraphics[scale=\FigScale]{ff/millenium/8.png}
\caption{\RU{Переполнение переменных ресурсов}\EN{Resource variables overflow}}
\label{fig:mill_8}
\end{figure}

\RU{Это просто переполнение}\EN{That's just overflow}. 
\RU{Разработчик игры вероятно никогда не думал, что значения ресурсов будут такими большими,
так что, здесь, наверное, нет проверок на переполнение, но шахта в игре \q{работает}, ресурсы добавляются,
отсюда и переполнение.}
\EN{The game's developer probably didn't think about such high amounts of resources,
so there are probably no overflow checks, but the mine is \q{working} in the game, resources are added,
hence the overflows.}
\RU{Вероятно, не нужно было жадничать}\EN{Apparently, it was a bad idea to be that greedy}.

\RU{Здесь наверняка еще какие-то значения в этом файле}\EN{There are probably a lot of more values 
saved in this file}.

\RU{Так что это очень простой способ читинга в играх}\EN{So this is very simple method of cheating in games}.
\RU{Файл с таблицей очков также можно легко модифицировать}\EN{High score files often can be easily 
patched like that}.

\EN{More about files and memory snapshots comparing}\RU{Еще насчет сравнения файлов и снимков памяти}: 
\myref{snapshots_comparing}.

