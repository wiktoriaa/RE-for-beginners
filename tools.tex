\part{\IFRU{Инструменты}{Tools}}

\chapter{\IFRU{Дизассемблер}{Disassembler}}

\section{IDA}

\label{IDA}
\IFRU{Старая бесплатная версия доступна для скачивания}{Older freeware version is available for downloading}
\footnote{\url{http://www.hex-rays.com/idapro/idadownfreeware.htm}}.

\IFRU{Краткий справочник хот-кеев}{Short hot-keys cheatsheet}:

\begin{center}
\begin{tabular}{ | l | l | }
\cellcolor{blue!25} \IFRU{клавиша}{key} & \cellcolor{blue!25} \IFRU{значение}{meaning} \\
Space 	& \IFRU{переключать между листингом и просмотром кода в виде графа}{switch listing and graph view} \\
C 	& \IFRU{конвертировать в код}{convert to code} \\
D 	& \IFRU{конвертировать в данные}{convert to data} \\
A 	& \IFRU{конвертировать в строку}{convert to string} \\
* 	& \IFRU{конвертировать в массив}{convert to array} \\
U 	& \IFRU{сделать неопределенным}{undefine} \\
O 	& \IFRU{сделать смещение из операнда}{make offset of operand} \\
H 	& \IFRU{сделать десятичное число}{make decimal number} \\
R 	& \IFRU{сделать символ}{make char} \\
B 	& \IFRU{сделать двоичное число}{make binary number} \\
Q 	& \IFRU{сделать шестнадцатеричное число}{make hexadecimal number} \\
N 	& \IFRU{переменовать идентификатор}{rename identificator} \\
? 	& \IFRU{калькулятор}{calculator} \\
G 	& \IFRU{переход на адрес}{jump to address} \\
: 	& \IFRU{добавить комментарий}{add comment} \\
Ctrl-X 	& \IFRU{показать ссылки на текущую ф-цию, метку, переменную (в т.ч., в стеке)}
		{show refernces to the current function, label, variable (incl. in local stack)} \\
X 	& \IFRU{показать ссылки на ф-цию, метку, переменную, итд}
		{show references to the function, label, variable, etc} \\
Alt-I 	& \IFRU{искать константу}{search for constant} \\
Ctrl-I 	& \IFRU{искать следующее вхождение константы}{search for the next occurrence of constant} \\
Alt-B 	& \IFRU{искать последовательность байт}{search for byte sequence} \\
Ctrl-B 	& \IFRU{искать следующее вхождение последовательности байт}
		{search for the next occurrence of byte sequence} \\
Alt-T 	& \IFRU{искать текст (включая инструкции, итд)}{search for text (including instructions, etc)} \\
Ctrl-T 	& \IFRU{искать следующее вхождение текста}{search for the next occurrence of text} \\
Alt-P 	& \IFRU{редактировать текущую функцию}{edit current function} \\
Enter 	& \IFRU{перейти к ф-ции, переменной, итд}{jump to function, variable, etc} \\
Esc 	& \IFRU{вернуться назад}{get back} \\
Num -   & \IFRU{свернуть ф-цию или отмеченную область}{fold function or selected area} \\
Num + 	& \IFRU{снова показать ф-цию или область}{unhide function or area}\\
\end{tabular}
\end{center}

\IFRU{Сворачивание ф-ции или области может быть удобно чтобы прятать те части ф-ции,
чья функция вам стала уже ясна}
{Function/area folding may be useful for hiding function parts when you realize what they do}.
\IFRU{это используется в моем скрипте\footnote{\url{\YurichevIDAIDCScripts}}}{this is used in my}
\IFRU{для сворачивания некоторых очень часто используемых фрагментов inline-кода}
{script\footnote{\url{\YurichevIDAIDCScripts}} for hiding some often used patterns of inline code}.

\chapter{\IFRU{Отладчик}{Debugger}}

\index{tracer}
\label{tracer}
\IFRU{Я использую}{I use} \IT{tracer}\footnote{\url{http://yurichev.com/tracer-\LANG.html}}
\IFRU{вместо отладчика}{instead of debugger}.

\IFRU{Со временем я отказался использовать отладчик, потому что все что мне нужно от него: это иногда подсмотреть 
какие-либо аргументы какой-либо функции во время исполнения или состояние регистров в определенном месте. 
Каждый раз загружать отладчик для этого это слишком, поэтому я написал очень простую утилиту \IT{tracer}. 
Она консольная, запускается из командной строки, позволяет перехватывать исполнение функций, 
ставить брякпойнты на произвольные места, смотреть состояние регистров, модифицировать их, и так далее.}
{I stopped to use debugger eventually, since all I need from it is to spot a function's arguments while
execution, or registers' state at some point.
To load debugger each time is too much, so I wrote a small utility \IT{tracer}.
It has console-interface, working from command-line, enable us to intercept function execution,
set breakpoints at arbitrary places, spot registers' state, modify it, etc.}

\IFRU{Но для учебы, очень полезно трассировать код руками в отладчике, наблюдать как меняются значения регистров 
(например, как минимум классический SoftICE, OllyDbg, WinDbg подсвечивают измененные регистры), 
флагов, данные, менять их самому, смотреть реакцию, и т.д.}
{However, as for learning purposes, it is highly advisable to trace code in debugger manually, watch how register's state
changing (e.g. classic SoftICE, OllyDbg, WinDbg highlighting changed registers), flags, data, change them
manually, watch reaction, etc.}

\chapter{\IFRU{Трассировка системных вызовов}{System calls tracing}}

\label{strace}
\index{strace}
\index{dtruss}
\subsection{strace / dtruss}

\index{syscalls}
\IFRU{Позволяет показать, какие системные вызовы (syscalls(\ref{syscalls})) прямо сейчас вызывает процесс}
{Will show which system calls (syscalls(\ref{syscalls})) are called by process right now}.
\IFRU{Например}{For example}:

\begin{lstlisting}
# strace df -h

...

access("/etc/ld.so.nohwcap", F_OK)      = -1 ENOENT (No such file or directory)
open("/lib/i386-linux-gnu/libc.so.6", O_RDONLY|O_CLOEXEC) = 3
read(3, "\177ELF\1\1\1\0\0\0\0\0\0\0\0\0\3\0\3\0\1\0\0\0\220\232\1\0004\0\0\0"..., 512) = 512
fstat64(3, {st_mode=S_IFREG|0755, st_size=1770984, ...}) = 0
mmap2(NULL, 1780508, PROT_READ|PROT_EXEC, MAP_PRIVATE|MAP_DENYWRITE, 3, 0) = 0xb75b3000
\end{lstlisting}

\index{MacOSX}
\IFRU{В MacOSX для этого же имеется dtruss}{MacOSX has dtruss for the same aim}.

\index{cygwin}
\IFRU{В Cygwin также есть strace, впрочем, если я верно понял, 
он показывает результаты только для .exe-файлов скомпилированных для среды самого cygwin}
{The Cygwin also has strace, but if I understood correctly, it works only for .exe-files
compiled for cygwin environment itself}.

\chapter{\IFRU{Прочие инструменты}{Other tools}}

\begin{itemize}
\item
Microsoft Visual Studio Express\footnote{\url{http://www.microsoft.com/express/Downloads/}}:
\IFRU{Усеченная бесплатная версия Visual Studio, пригодная для простых экспериментов}
{Stripped-down free Visual Studio version, convenient for simple experiments}.

\item
\label{Hiew}
Hiew\footnote{\url{http://www.hiew.ru/}} \IFRU{для мелкой модификации кода в исполняемых файлах}
{for small modifications of code in binary files}.

\item
\index{binary grep}
binary grep: \IFRU{небольшая утилита для поиска констант (либо просто последовательности байт)
в большом  кол-ве файлов, включая неисполняемые: \BGREPURL.}
{the small utility for constants searching (or just any byte sequence) in a big pile of files, 
including non-executable: \BGREPURL.}
\end{itemize}

