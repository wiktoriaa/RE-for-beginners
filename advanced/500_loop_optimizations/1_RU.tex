\subsection{Странная оптимизация циклов}

Это самая простая (из всех возможных) реализация memcpy():

\begin{lstlisting}[style=customc]
void memcpy (unsigned char* dst, unsigned char* src, size_t cnt)
{
	size_t i;
	for (i=0; i<cnt; i++)
		dst[i]=src[i];
};
\end{lstlisting}

Как минимум MVC 6.0 из конца 90-х вплоть до MSVC 2013 может выдавать вот такой странный код (этот листинг создан MSVC 2013
x86):

\lstinputlisting[style=customasmx86]{advanced/500_loop_optimizations/1_1_RU.lst}

Это всё странно, потому что как люди работают с двумя указателями? Они сохраняют два адреса в двух регистрах или двух
ячейках памяти.
Компилятор MSVC в данном случае сохраняет два указателя как один указатель (\IT{скользящий dst} в \EAX)
и разницу между указателями \IT{src} и \IT{dst} (она остается неизменной во время исполнения цикла, в \ESI).
Когда нужно загрузить байт из \IT{src}, он загружается на \IT{diff + скользящий dst} и сохраняет байт просто на
\IT{скользящем dst}.

Должно быть это какой-то трюк для оптимизации. Но я переписал эту ф-цию так:

\lstinputlisting[style=customasmx86]{advanced/500_loop_optimizations/1_2.lst}

\dots и она работает также быстро как и \IT{соптимизированная} версия на моем Intel Xeon E31220 @ 3.10GHz.
Может быть, эта оптимизация предназначалась для более старых x86-процессоров 90-х, т.к., этот трюк использует
как минимум древний MS VC 6.0?

Есть идеи?

\myindex{Hex-Rays}
Hex-Rays 2.2 не распознает такие шаблонные фрагменты кода (будем надеятся, это временно?):

\begin{lstlisting}[style=customc]
void __cdecl f1(char *dst, char *src, size_t size)
{
  size_t counter; // edx@1
  char *sliding_dst; // eax@2
  char tmp; // cl@3

  counter = size;
  if ( size )
  {
    sliding_dst = dst;
    do
    {
      tmp = (sliding_dst++)[src - dst];         // разница (src-dst) вычисляется один раз, перед телом цикла
      *(sliding_dst - 1) = tmp;
      --counter;
    }
    while ( counter );
  }
}
\end{lstlisting}

Тем не менее, этот трюк часто используется в MSVC (и не только в самодельных ф-циях \IT{memcpy()}, но также и во многих
циклах, работающих с двумя или более массивами), так что для реверс-инжиниров стоит помнить об этом.

% <!-- As of why writting occurred after <b>dst</b> incrementing? -->

