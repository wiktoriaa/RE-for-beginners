\section{\Example{} \#2}
\label{win16_messagebox}

\begin{lstlisting}
#include <windows.h>

int PASCAL WinMain( HINSTANCE hInstance,
                    HINSTANCE hPrevInstance,
                    LPSTR lpCmdLine,
                    int nCmdShow )
{
	MessageBox (NULL, "hello, world", "caption", MB_YESNOCANCEL);
	return 0;
};
\end{lstlisting}

\begin{lstlisting}
WinMain         proc near
                push    bp
                mov     bp, sp
                xor     ax, ax          ; NULL
                push    ax
                push    ds
                mov     ax, offset aHelloWorld ; 0x18. "hello, world"
                push    ax
                push    ds
                mov     ax, offset aCaption ; 0x10. "caption"
                push    ax
                mov     ax, 3           ; MB_YESNOCANCEL
                push    ax
                call    MESSAGEBOX
                xor     ax, ax          ; return 0
                pop     bp
                retn    0Ah
WinMain         endp

dseg02:0010 aCaption        db 'caption',0
dseg02:0018 aHelloWorld     db 'hello, world',0
\end{lstlisting}

\index{x86!\Instructions!RET}
\RU{Пара важных моментов: соглашение о передаче аргументов здесь \TT{PASCAL}: оно указывает что самый
первый аргумент должен передаваться первым}
\EN{Couple important things here: \TT{PASCAL} calling convention dictates passing the first argument first} 
(\TT{MB\_YESNOCANCEL}), \RU{а самый последний аргумент\EMDASH{}последним}\EN{and the last argument\EMDASH{}last} (NULL).
\RU{Это соглашение также указывает вызываемой ф-ции восстановить}
\EN{This convention also tells \gls{callee} to restore} \gls{stack pointer}:
\RU{поэтому инструкция}\EN{hence} \TT{RETN} \RU{имеет аргумент}\EN{instruction has} \TT{0Ah} 
\RU{означая что указатель нужно сдвинуть вперед на $10$ байт во время возврата из ф-ции}
\EN{argument, meaning pointer should be shifted above by $10$ bytes upon function exit}.
\RU{Это как}\EN{It is like} stdcall (\ref{sec:stdcall}), \EN{but arguments are passed in 
``natural'' order}\RU{только аргументы передаются в ``естественном'' порядке}.

\RU{Указатели передаются парами: сначала сегмент данных, потом указатель внутри сегмента}
\EN{Pointers are passed by pairs: a segment of data is first passed, then the pointer inside of segment}.
\RU{В этом примере только один сегмент, так что \TT{DS} всегда указывает на сегмент данных в исполняемом
файле}\EN{Here is only one segment in this example, so \TT{DS} is always pointing to data segment of executable}.

