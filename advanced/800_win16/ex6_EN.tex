\subsection{\Example{} \#6}

\lstinputlisting[style=customc]{\CURPATH/ex6.c}

\lstinputlisting[style=customasmx86]{\CURPATH/ex6.lst}

\myindex{\CStandardLibrary!time()}
\myindex{\CStandardLibrary!localtime()}

UNIX time is a 32-bit value, so it is returned in the \TT{DX:AX} register pair and stored in two local 16-bit variables.
Then a pointer to the pair is passed to the
\TT{localtime()} function.
The \TT{localtime()} function has a \TT{struct tm} 
allocated somewhere in the guts of the C library, so only a pointer to it is returned. 

By the way, this also implies that the function cannot be called again until its results are used.

For the \TT{time()} and \TT{localtime()} 
functions, a Watcom calling convention is used here:
the first four arguments are passed in the \TT{AX}, \TT{DX}, \TT{BX} and \TT{CX}, 
registers, and the rest arguments are via the stack.

The functions using this convention are also marked by underscore at the end of their name.

\TT{sprintf()} does not use the \TT{PASCAL} calling convention, nor the Watcom one,\\ % varioref bug
so the arguments are passed in the normal \IT{cdecl} way (\myref{cdecl}).

\subsubsection{Global variables}

This is the same example, but now these variables are global:

\lstinputlisting[style=customc]{\CURPATH/ex6_global.c}

\lstinputlisting[style=customasmx86]{\CURPATH/ex6_global.lst}

\TT{t} is not to be used, but the compiler emitted the code which stores the value.

Because it is not sure, maybe that value will eventually be used in some other module.
