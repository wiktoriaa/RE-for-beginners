\chapter{Negative array indices}
\label{negative_array_indices}

It's possible to address the space \IT{before} an array by supplying a negative index, e.g., $array[-1]$.

\myindex{FORTRAN}
It's very hard to say why one should use it, there is probably only one known practical application
of this technique.
% TODO: разобраться:
% another use (which seems to be used more often) can be found in "Transaction processing" by Jim Gray, p. 755
% basically, it's used in structures that describe pages, and the entries are indexed from the beginning,
% and the directory entries - from the end with negative indices.
\CCpp array elements indices start at 0, but some \ac{PL}s have a first index at 1 
(at least FORTRAN).

Programmers may still have this habit, so using this little trick, it's possible to address the first element
in \CCpp using index 1:

\lstinputlisting{\CURPATH/neg_array.c}

\lstinputlisting[caption=\NonOptimizing MSVC 2010,label=neg_array_c,numbers=left]{\CURPATH/neg_array_EN.asm}

So we have \TT{array[]} of ten elements, filled with $0 \ldots 9$ bytes.

Then we have the \TT{fakearray[]} pointer, which points one byte before \TT{array[]}.

\TT{fakearray[1]} points exactly to \TT{array[0]}.
But we are still curious, what is there before \TT{array[]}?
We have added \TT{random\_value} before \TT{array[]} and set 
it to \TT{0x11223344}.
The non-optimizing compiler allocated the variables in the order they were declared, so yes, the 32-bit \TT{random\_value}
is right before the array.

We ran it, and:

\begin{lstlisting}
first element 0
second element 1
last element 9
array[-1]=11, array[-2]=22, array[-3]=33, array[-4]=44
\end{lstlisting}

Here is the stack fragment we will copypaste from \olly's stack window (with comments added by the author):

\lstinputlisting[caption=\NonOptimizing MSVC 2010]{\CURPATH/stack_EN.txt}

The pointer to the \TT{fakearray[]} (\TT{0x001DFBD3}) is indeed 
the address of \TT{array[]} in the stack (\TT{0x001DFBD4}), 
but minus 1 byte.

It's still very hackish and dubious trick. Doubtfully anyone should use it in production code,
but as a demonstration, it fits perfectly here.


