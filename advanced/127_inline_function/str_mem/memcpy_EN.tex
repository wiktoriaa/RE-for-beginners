\subsubsection{memcpy()}
\myindex{\CStandardLibrary!memcpy()}

\myparagraph{Short blocks}
\label{copying_short_blocks}

\myindex{x86!\Instructions!MOV}

The routine to copy short blocks is often implemented as a sequence of \MOV instructions.

\lstinputlisting[caption=memcpy() example,style=customc]{\CURPATH/str_mem/memcpy_7.c}

\lstinputlisting[caption=\Optimizing MSVC 2010,style=customasmx86]{\CURPATH/str_mem/memcpy_7_MSVC_2010_Ox.asm}

\lstinputlisting[caption=\Optimizing GCC 4.8.1,style=customasmx86]{\CURPATH/str_mem/memcpy_7_GCC_O3.s}


That's usually done as follows: 4-byte blocks are copied first, then a 16-bit word (if needed), 
then the last byte (if needed).

Structures are also copied using
\MOV: \myref{short_struct_copying_using_MOV}.

\myparagraph{Long blocks}

The compilers behave differently in this case.

\lstinputlisting[caption=memcpy() example,style=customc]{\CURPATH/str_mem/memcpy.c}

\myindex{x86!\Instructions!MOVSD}

For copying 128 bytes, MSVC uses a single \TT{MOVSD} instruction (because 128 
divides evenly by 4):

\lstinputlisting[caption=\Optimizing MSVC 2010,style=customasmx86]{\CURPATH/str_mem/memcpy_128_MSVC_2010_Ox.asm}


When copying 123 bytes, 30 32-byte words are copied first using \TT{MOVSD}
(that's 120 bytes),
then 2 bytes are copied using \TT{MOVSW}, 
then one more byte using \TT{MOVSB}.

\lstinputlisting[caption=\Optimizing MSVC 2010,style=customasmx86]{\CURPATH/str_mem/memcpy_123_MSVC_2010_Ox.asm}


GCC uses one big universal functions, that works for any block size:

\lstinputlisting[caption=\Optimizing GCC 4.8.1,style=customasmx86]{\CURPATH/str_mem/memcpy_GCC.s}


Universal memory copy functions usually work as follows:
calculate how many 32-bit words can be copied, then copy them using \TT{MOVSD}, then copy
the remaining bytes.

\myindex{SIMD}

More advanced and complex copy functions use \ac{SIMD} instructions and also take the memory alignment
in consideration.

As an example of SIMD strlen() function: \myref{SIMD_strlen}.

