\section{MIPS}

By some reason, optimizing GCC 4.4.5 generate just division instruction:

\begin{lstlisting}[caption=\Optimizing GCC 4.4.5 (IDA)]
f:
                li      $v0, 9
                bnez    $v0, loc_10
                div     $a0, $v0 ; branch delay slot
                break   0x1C00   ; "break 7" in assembly output and objdump

loc_10:
                mflo    $v0
                jr      $ra
                or      $at, $zero ; branch delay slot, NOP
\end{lstlisting}

\index{MIPS!\Instructions!BREAK}
Oh, we see here new instruction: BREAK. It just raises exception.
In this case, exception is raised if divisor is zero (it's not possible to divide by zero in conventional
math).
But GCC probably not very well did optimization job and did not see that \$V0 is never zero.
But there are still check.
So if \$V0 is zero somehow, BREAK will be executed, signalling to \ac{OS} about exception.
\index{MIPS!\Instructions!MFLO}
Otherwise, MFLO executes, which takes result of division from LO register and places it in \$V0.

\index{MIPS!\Registers!LO}
\index{MIPS!\Registers!HI}
By the way, as we may know, MUL instruction leaves high 32-bit of result in HI register and low 32-bit
in LO register.
DIV leaves result in LO register, and remainder in HI register.

\index{MIPS!\Instructions!MFHI}
If to alter statement to ``a \% 9'', MFHI instruction will be used instead of MFLO at the place.
