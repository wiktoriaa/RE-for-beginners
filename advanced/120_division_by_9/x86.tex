\section{x86}

\dots \RU{компилируется вполне предсказуемо:}\EN{is compiled in a very predictable way:}

\lstinputlisting[caption=MSVC]{\CURPATH/11_1_msvc.asm.\LANG}

\index{x86!\Instructions!IDIV}
\RU{\IDIV делит 64-битное число хранящееся в паре регистров \TT{EDX:EAX} на значение в \ECX. 
В результате, \EAX будет содержать \glslink{quotient}{частное}, а \EDX\EMDASH{}остаток от деления. 
Результат возвращается из функции через \EAX, так что после операции деления, 
это значение не перекладывается больше никуда, 
оно уже там, где надо.}
\EN{\IDIV divides the 64-bit number stored in the \TT{EDX:EAX} register pair by the value in the \ECX.
As a result, \EAX will contain the \gls{quotient}, and \EDX\EMDASH{}the remainder.
The result is returned from the \ttf function in the \EAX register, 
so the value is not moved after the division 
operation, it is in right place already.}
\RU{Из-за того, что \IDIV требует пару регистров \TT{EDX:EAX}, то перед этим инструкция \TT{CDQ} 
расширяет \EAX до 64-битного значения учитывая знак, так же, как это делает \MOVSX.}
\EN{Since \IDIV uses the value in the \TT{EDX:EAX} register pair, the \TT{CDQ} instruction (before \IDIV) extends 
the value in \EAX to a 64-bit value taking its sign into account, just as \MOVSX does.}
\RU{Со включенной оптимизацией (\Ox) получается:}
\EN{If we turn optimization on (\Ox), we get:}

\lstinputlisting[caption=\Optimizing MSVC]{\CURPATH/11_1_msvc_Ox.asm}

\RU{Это\EMDASH{}деление через умножение. Умножение конечно быстрее работает. 
Поэтому можно используя этот трюк
\footnote{Читайте подробнее о делении через умножение в \cite[10-3]{Warren:2002:HD:515297}} 
создать код эквивалентный тому что мы хотим и работающий быстрее.}
\EN{This is division by multiplication. Multiplication operations work much faster. 
And it is possible to use this trick
\footnote{Read more about division by multiplication in \cite[10-3]{Warren:2002:HD:515297}} 
to produce code which is effectively equivalent and faster.}

\RU{В оптимизации компиляторов, это также называется}\EN{This is also called} 
\q{strength reduction}\EN{ in compiler optimizations}.

\RU{GCC 4.4.1 даже без включенной оптимизации генерирует примерно такой же код, 
как и MSVC с оптимизацией:}
\EN{GCC 4.4.1 generates almost the same code even without additional optimization flags, 
just like MSVC with optimization turned on:}

\lstinputlisting[caption=\NonOptimizing GCC 4.4.1]{\CURPATH/11_2_gcc.asm}
