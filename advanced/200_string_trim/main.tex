\chapter{\RU{Обрезка строк}\EN{Strings trimming}}
\newcommand{\CRLF}{\ac{CR}/\ac{LF}}

\RU{Весьма востребованная операция со строками\EMDASH{}это удаление некоторых символов в начале и/или конце
строки.}
\EN{A very common string processing task is to remove some characters at the start and/or at the end.}

\RU{В этом примере, мы будем работать с функцией, удаляющей все символы перевода строки 
(\CRLF{}) в конце входной строки:}
\EN{In this example, we are going to work with a function which removes all newline characters 
(\CRLF{}) from the end of the input string:}

\lstinputlisting{\CURPATH/strtrim.c.\LANG}

\RU{Входной аргумент всегда возвращается на выходе, это удобно, когда вам нужно объединять
функции обработки строк в цепочки, как это сделано здесь в функции \main.}
\EN{The input argument is always returned on exit, this is convenient when you need to chain 
string processing functions, like it was done here in the \main function.}

\RU{Вторая часть for() (\TT{str\_len>0 \&\& (c=s[str\_len-1])}) называется в \CCpp \q{short-circuit} 
(короткое замыкание) и это очень удобно: \cite[1.3.8]{CBook}.}
\EN{The second part of for() (\TT{str\_len>0 \&\& (c=s[str\_len-1])}) is the so called \q{short-circuit} 
in \CCpp and is very convenient \cite[1.3.8]{CBook}.}
\RU{Компиляторы \CCpp гарантируют последовательное вычисление слева направо.}
\EN{The \CCpp compilers guarantee an evaluation sequence from left to right.}
\RU{Так что если первое условие не истинно после вычисления, второе никогда не будет
вычисляться.}
\EN{So if the first clause is false after evaluation, the second one is never to be evaluated.}

% subsections
\subsection{x64}

\index{x86-64}
\RU{Всё то же самое, только используются регистры вместо стека для передачи аргументов функций}%
\EN{The picture here is similar with the difference that the registers, rather than the stack, are used for arguments passing}.

\subsubsection{MSVC}

\lstinputlisting[caption=MSVC 2012 x64]{patterns/04_scanf/1_simple/ex1_MSVC_x64.asm.\LANG}

\ifdefined\IncludeGCC
\subsubsection{GCC}

\lstinputlisting[caption=\Optimizing GCC 4.4.6 x64]{patterns/04_scanf/1_simple/ex1_GCC_x64.s.\LANG}
\fi

\ifdefined\IncludeARM
\subsection{ARM64}

\subsubsection{GCC}

\RU{Компилируем пример в}\EN{Let's compile the example using} GCC 4.8.1 \InENRU ARM64:

\lstinputlisting[numbers=left,label=hw_ARM64_GCC,caption=\NonOptimizing GCC 4.8.1 + objdump]
{patterns/01_helloworld/ARM/hw.lst}

\RU{В ARM64 нет режима thumb и thumb-2, только ARM, так что тут только 32-битные инструкции.}
\EN{There are no thumb and thumb-2 modes in ARM64, only ARM, so there are 32-bit instructions only.}
\RU{Регистров тут в 2 раза больше}\EN{Registers count is doubled}: \ref{ARM64_GPRs}.
\RU{64-битные регистры теперь имеют префикс}\EN{64-bit registers has} 
\TT{X-}\EN{ prefixes, while its 32-bit parts}\RU{, а их 32-битные части}\EMDASH{}\TT{W-}.

\RU{Инструкция }\TT{STP}\EN{ instruction} (\IT{Store Pair}) 
\RU{сохраняет в стеке сразу два регистра}\EN{saves two registers in stack simultaneously}: \RegX{29} \InENRU \RegX{30}.
\RU{Конечно, эта инструкция может сохранять эту пару где угодно в памяти, но здесь указан регистр \ac{SP}, так что,
пара сохраняется именно в стеке.}
\EN{Of course, this instruction is able to save this pair at random place of memory, 
but \ac{SP} register is specified here, so the pair is saved in stack.}
\RU{Регистры в ARM64 64-битные, каждый это 8 байт, так что для хранения двух регистров нужно именно 16 байт.}
\EN{ARM64 registers are 64-bit ones, each contain 8 bytes, so one need 16 bytes for saving two registers.}

\RU{Восклицательный знак после операнда означает, что в начале от \ac{SP} будет отнято 16, и только затем
значения из пары регистров будут записаны в стек.}
\EN{Exclamation mark after operand mean that 16 will be subtracted from \ac{SP} first, and only then
values from registers pair will be written into the stack.}
\RU{Это называется}\EN{This is also called} \IT{pre-index}.
\RU{Больше о разнице между}\EN{About difference between} \IT{post-index} \AndENRU \IT{pre-index}, 
\RU{описано здесь}\EN{read here}: \ref{ARM_postindex_vs_preindex}.

\RU{Таким образом, в терминах более знакомого всем процессора x86, первая инструкция это просто аналог 
пары инструкций}
\EN{Hence, in terms of more familiar x86, the first instruction is just analogous to pair of}
\TT{PUSH X29} \AndENRU \TT{PUSH X30}.
\RegX{29} \EN{is used as \ac{FP} in ARM64}\RU{в ARM64 используется как \ac{FP}}, \EN{and}\RU{а} \RegX{30} 
\EN{as}\RU{как} \ac{LR}, \RU{поэтому они сохраняются в прологе ф-ции и
восстанавливаются в эпилоге}\EN{so that's why they are saved in function prologue and restored in function
epilogue}.

\EN{The second instruction saves}\RU{Вторая инструкция записывает} \ac{SP} \InENRU \RegX{29} (\OrENRU \ac{FP}).
\RU{Это нужно для установки стекового фрейма ф-ции}\EN{This is needed for function stack frame setup}.

\RU{Инструкции }\TT{ADRP} \AndENRU \ADD \EN{instructions are needed for forming address of the 
string}\RU{нужны для формирования адреса строки} ``Hello!'' \EN{in the \RegX{0} register}\RU{в регистре \RegX{0}}, 
\RU{ведь первый аргумент ф-ции передается через этот регистр}\EN{because first function argument is passed
in this register}.
\RU{Но в ARM нет инструкций, при помощи которых можно записать в регистр длинное число}\EN{But there are
no instructions in ARM helping to write large number into register} 
(\RU{потому что сама длина инструкции ограничена 4-ю байтами, больше об этом здесь}\EN{because instruction
length is limited by 4 bytes, read more about it here}: \ref{ARM_big_constants_loading}).
\RU{Так что нужно использовать несколько инструкций}\EN{So several instructions should be used}.
\RU{Первая инструкция}\EN{The first instruction} (\TT{ADRP}) \EN{writes address of 4Kb page where string is
located into \RegX{0}}\RU{записывает в \RegX{0} адрес 4-килобайтной страницы где находится строка}, 
\EN{and the second one}\RU{а вторая} (\ADD) \RU{просто прибавляет к этому адресу остаток}\EN{just adds
reminder to the address}.
\EN{Read more about}\RU{Читайте больше об этом}: \ref{ARM64_relocs}.

\TT{0x400000 + 0x648 = 0x400648}, \EN{and we see our ``Hello!'' C-string in the \TT{.rodata} data segment at this
address}\RU{и мы видим что в секции данных \TT{.rodata} по этому адресу как раз находится наша
Си-строка ``Hello!''}.

\RU{Затем, при помощи инструкции \TT{BL} вызывается \puts, это уже рассматривалось раннее: \ref{puts}.}
\EN{\puts is called then using \TT{BL} instruction, this was already discussed before: \ref{puts}.}

\RU{Инструкция }\MOV \EN{instruction writes $0$ into}\RU{записывает $0$ в} \RegW{0}. 
\RegW{0} \RU{это младшие 32 бита регистра}\EN{is low 32 bits of} \RegX{0}\EN{ register}:

\begin{center}
\begin{tabular}{ | l | l | }
\hline
\RU{Старшие 32 бита}\EN{High 32-bit part} & \RU{младщие 32 бита}\EN{low 32-bit part} \\
\hline
\multicolumn{2}{ | c | }{X0} \\
\hline
\multicolumn{1}{ | c | }{} & \multicolumn{1}{ c | }{W0} \\
\hline
\end{tabular}
\end{center}


\RU{А результат ф-ции возвращается через \RegX{0}, и \main возвращает $0$, 
так что вот так готовится возвращаемый результат.}
\EN{Function result is returning via \RegX{0} and \main returning $0$, so that's how returning
result is prepared.}
\RU{Почему именно 32-битная часть}\EN{But why 32-bit part}?
\RU{Потому в ARM64, как и в x86-64, тип \Tint оставили 32-битным, для лучшей совместимости.}
\EN{Because \Tint in ARM64, just like in x86-64, is still 32-bit, for better compatibility.}
\RU{Следовательно, раз уж ф-ция возвращает 32-битный \Tint, то нужно заполнить только 32 младших бита 
регистра \RegX{0}.}
\EN{So if function returning 32-bit \Tint, only 32 lowest bits of \RegX{0} register should be filled.}

\RU{Для того, чтобы удостовериться в этом, я немного отредактировал свой пример и перекомпилировал его.}
\EN{In order to get sure about it, I changed by example slightly and recompiled it.}
\RU{Теперь}\EN{Now} \main \RU{возвращает 64-битное значение}\EN{returns 64-bit value}:

\begin{lstlisting}[caption=\main \RU{возвращающая значение типа}\EN{returning a value of} \TT{uint64\_t}\EN{ type}]
#include <stdio.h>
#include <stdint.h>

uint64_t main()
{
        printf ("Hello!\n");
        return 0;
};
\end{lstlisting}

\RU{Результат точно такой же, только \MOV в той строке теперь выглядит так:}
\EN{Result is very same, but that's how \MOV at that line is now looks like:}

\begin{lstlisting}[caption=\NonOptimizing GCC 4.8.1 + objdump]
  4005a4:       d2800000        mov     x0, #0x0                        // #0
\end{lstlisting}

\RU{Далее, при помощи инструкции \TT{LDP} (\IT{Load Pair}), восстанавливаются регистры \RegX{29} и \RegX{30}.}
\EN{\TT{LDP} (\IT{Load Pair}) then restores \RegX{29} and \RegX{30} registers.}
\RU{Восклицательного знака после инструкции нет: это означает, что в начале значения достаются из стека,
и только потом \ac{SP} увеличивается на 16.}
\EN{There are no exclamation mark after instruction: this mean, the value is first loaded from the stack,
only then \ac{SP} value is increased by 16.}
\RU{Это называется}\EN{This is called} \IT{post-index}.

\RU{В ARM64 есть новая инструкция}\EN{New instruction appeared in ARM64}: \RET. 
\RU{Она работает так же как и}\EN{It works just as} \TT{BX LR}, \RU{но там добавлен специальный бит,
подсказывающий процессору, что это именно выход из ф-ции, а не просто переход, чтобы процессор
мог более оптимально исполнять эту инструкцию}\EN{but a special \IT{hint} bit is added, showing to the \ac{CPU}
that this is return from the function, not just another branch instruction, so it can execute it more optimally}.

\RU{Из-за простоты этой ф-ции, оптимизирующий GCC генерирует точно такой же код.}
\EN{Due to simplicity of the function, optimizing GCC generates the very same code.}

\subsubsection{Xcode}

\OptimizingXcodeV \RU{делает почти такой же код}\EN{doing the same}.
\RU{Вот что показывает}\EN{Here is what} otool\footnote{\RU{аналог}\EN{analogous to} objdump} 
\RU{с ключом}\EN{with the} \TT{-tv}\EN{ key shows}:

\begin{lstlisting}[caption=\OptimizingXcodeV + otool]
0000000000000000		stp	fp, lr, [sp, #-16]!
0000000000000004		add	fp, sp, 0
0000000000000008		adrp	x0, 0 ; 0x0
000000000000000c		add	x0, x0, 0
0000000000000010		bl	0x10
0000000000000014		movz	w0, #0
0000000000000018		ldp	fp, lr, [sp], #16
000000000000001c		ret	lr
\end{lstlisting}

\RU{Код такой же}\EN{The code is the same}.
\TT{MOVZ} \RU{в данном случае это синоним}\EN{here is synonymous to} \MOV.
\RU{Вот что немного отличается: otool показывает регистр}\EN{Here is difference: otool shows} 
\RegX{29}\EN{ register} \RU{как}\EN{as} \ac{FP}, \RU{а}\EN{and} \RegX{30} \RU{как}\EN{as} \ac{LR}.
\RU{Это действительно синонимы для этих регистров.}
\EN{These are indeed nicknames for these registers.}

\RU{\RET otool показывает как \TT{RET LR}, это немного избыточный результат дизассемблера.}
\EN{otool also shows \RET as \TT{RET LR}, this is somewhat redundant disassembler result.}

\section{ARM}

\subsection{\NonOptimizingXcode + \ARMMode}

\lstinputlisting[caption=\NonOptimizingXcode + \ARMMode]{patterns/10_strlen/xcode_ARM_O0_en.asm}

\IFRU{Неоптимизирующий LLVM генерирует слишком много кода, зато на этом примере можно посмотреть, 
как функции работают с локальными переменными в стеке.}
{Non-optimizing LLVM generates too much code, however, here we can see how function works with 
local variables in the stack.}
\IFRU{В нашей функции только локальных переменных две, это два указателя}
{There are only two local variables in our function},
\IT{eos} \AndENRU \IT{str}.

\IFRU{В этом листинге}{In this listing}, \IFRU{сгенерированном при помощи}{generated by} \IDA, 
\IFRU{я переименовал}{I renamed} \IT{var\_8} \AndENRU \IT{var\_4} \IFRU{в}{into} \IT{eos} 
\AndENRU \IT{str} \IFRU{вручную}{manually}.

\IFRU{Итак, первые несколько инструкций просто сохраняют входное значение в переменных}{So, 
first instructions are just saves input value in} \IT{str} \AndENRU \IT{eos}.

\IFRU{Начиная с метки}{Loop body is beginning at} \IT{loc\_2CB8}\IFRU{, начинается тело цикла}{ label}.

\IFRU{Первые три инструкции в теле цикла}{First three instruction in loop body} (\TT{LDR}, \ADD, \TT{STR}) 
\IFRU{загружают значение}{loads} \IT{eos} \IFRU{в}{value into} \Reg{0}, 
\IFRU{затем происходит инкремент значения и оно сохраняется назад в локальной переменной \IT{eos} расположенной 
в стеке.}{then value is \glslink{increment}{incremented} and it is saved back into \IT{eos} local variable located in the stack.}

\index{ARM!\Instructions!LDRSB}
\IFRU{Следующая инструкция}{The next} \TT{``LDRSB R0, [R0]''} (\IT{Load Register Signed Byte}) 
\IFRU{загружает байт из памяти по адресу \Reg{0}, расширяет его до 32-бит считая его знаковым (signed) 
и сохраняет в \Reg{0}}{instruction loading byte from memory at \Reg{0} address and sign-extends it to 32-bit}.
\index{x86!\Instructions!MOVSX}
\IFRU{Это немного похоже на инструкцию}{This is similar to} \MOVSX \IFRU{в}{instruction in} x86.
\IFRU{Компилятор считает этот байт знаковым (signed), потому что тип \Tchar по стандарту Си ~--- знаковый.}
{The compiler treating this byte as signed since \Tchar type in C standard is signed.}
\IFRU{Об это я уже немного писал}{I already wrote about it}~(\ref{MOVSX}) \IFRU{в этой же секции, 
но посвященной x86}{in this section, but related to x86}.

\index{x86!8086}
\index{8080}
\index{ARM}
\IFRU{Следует также заметить, что, в ARM нет возможности использовать 8-битную или 16-битную часть 
регистра, как это возможно в x86.}
{It is should be noted, it is impossible in ARM to use 8-bit part or 16-bit part 
of 32-bit register separately of the whole register,
as it is in x86.}
\IFRU{Вероятно, это связано с тем что за x86 тянется длинный шлейф совместимости со своими предками, 
такими как
16-битный 8086 и даже 8-битный 8080, а ARM разрабатывался с чистого листа как 32-битный RISC-процессор.}
{Apparently, it is because x86 has a huge history of compatibility with its ancestors like 16-bit 8086 
and even 8-bit 8080,
but ARM was developed from scratch as 32-bit RISC-processor.}
\IFRU{Следовательно, чтобы работать с отдельными байтами на ARM, так или иначе, придется использовать 
32-битные регистры.}
{Consequently, in order to process separate bytes in ARM, one have to use 32-bit registers anyway.}

\IFRU{Итак}{So}, \TT{LDRSB} \IFRU{загружает символ из строки в \Reg{0}, по одному}
{loads symbol from string into \Reg{0}, one by one}.
\IFRU{Следующие инструкции}{Next} \CMP \AndENRU \ac{BEQ} \IFRU{проверяют, является ли этот символ $0$.}
{instructions checks, if loaded symbol is $0$.}
\IFRU{Если не $0$, то происходит переход на начало тела цикла.}{If not $0$, control passing to loop body
begin.}
\IFRU{А если $0$, выходим из цикла.}{And if $0$, loop is finishing.}

\IFRU{В конце функции вычисляется разница между}{At the end of function, a difference between} 
\IT{eos} \AndENRU \IT{str}\IFRU{, вычитается еще единица и вычисленное 
значение возвращается через \Reg{0}.}{ is calculated, 1 is also subtracting, and resulting value is returned
via \Reg{0}.}

N.B. \IFRU{В этой функции не сохранялись регистры}{Registers was not saved in this function}.
\index{ARM!\Registers!scratch registers}
\IFRU{Это потому что, по стандарту, регистры \Reg{0}-\Reg{3} называются также ``scratch registers'',
они предназначены для передачи аргументов, 
их значения не нужно восстанавливать при выходе из функции, потому что они больше не нужны в вызывающей функции.
Таким образом, их можно использовать как захочется}
{That's because by ARM calling convention, \Reg{0}-\Reg{3} registers are ``scratch registers'', 
they are intended for arguments passing,
its values may not be restored upon function exit since calling function will not use them anymore.
Consequently, they may be used for anything we want.}
\IFRU{А так как никакие больше регистры не используются, то и сохранять нечего.}
{Other registers are not used here, so that is why we have nothing to save on the stack.}
\IFRU{Поэтому, управление можно вернуть назад вызывающей функции 
простым переходом (\TT{BX}), по адресу в регистре \LR.}
{Thus, control may be returned back to calling function by simple jump (\TT{BX}),
to address in the \LR register.}

%\subsection{\NonOptimizingXcode + режим thumb}
%Практически, точно такой же код.

\subsection{\OptimizingXcode + \ThumbMode}

\lstinputlisting[caption=\OptimizingXcode + \ThumbMode]{patterns/10_strlen/xcode_thumb_O3.asm}

\IFRU{Оптимизирующий LLVM решил, что под переменные \IT{eos} и \IT{str} выделять место в стеке не обязательно}
{As optimizing LLVM concludes, space on the stack for \IT{eos} and \IT{str} may not be allocated},
\IFRU{и эти переменные можно хранить прямо в регистрах.}
{and these variables may always be stored right in registers.}
\IFRU{Перед началом тела цикла}{Before loop body beginning}, \IT{str} \IFRU{будет находиться в}{will always be in} 
\Reg{0}, \IFRU{а}{and} \IT{eos}\EMDASH\InENRU \Reg{1}.

\index{ARM!\Instructions!LDRB.W}
\index{ARM!\IFRU{Режимы адресации}{Adressing modes}}
\RU{Инструкция }\TT{``LDRB.W R2, [R1],\#1''} \IFRU{загружает в \Reg{2} байт из памяти по адресу \Reg{1}, 
расширяя его как знаковый (signed), до 32-битного
значения, но не только это.}
{instruction loads byte from memory at the address \Reg{1} into \Reg{2}, sign-extending it to 32-bit value, but not
only that.}
\TT{\#1} \IFRU{в конце инструкции называется}{at the instruction's end calling} ``Post-indexed addressing'', 
\IFRU{это значит, что после загрузки байта, к \Reg{1} добавится единица.}{this means, $1$ is to be added
to the \Reg{1} after byte load.}
\IFRU{Это очень удобно для работы с массивами.}
{That's convenient when accessing arrays.}

\index{PDP-11}
\index{\CLanguageElements!\PostIncrement}
\index{\CLanguageElements!\PostDecrement}
\index{\CLanguageElements!\PreIncrement}
\index{\CLanguageElements!\PreDecrement}
\IFRU{Такого режима адресации в x86 нет, но он есть в некоторых других процессорах, даже на PDP-11.}
{There is no such addressing mode in x86, but it is present in some other processors, even on PDP-11.}
\IFRU{Существует байка, что режимы пре-инкремента, пост-инкремента, 
пре-декремента и пост-декремента адреса в PDP-11}
{There is a legend the pre-increment, post-increment, pre-decrement and post-decrement modes in PDP-11},
\IFRU{были ``виновны'' в появлении таких конструкций языка Си (который разрабатывался на PDP-11) как}
{were ``guilty'' in appearance such C language (which developed on PDP-11) constructs as}
*ptr++, *++ptr, *ptr-{}-, *-{}-ptr. 
\IFRU{Кстати, это является труднозапоминаемой особенностью в Си.}
{By the way, this is one of hard to memorize C feature.}
\IFRU{Дела обстоят так:}{This is how it is:}

\begin{center}
\begin{tabular}{ | l | l | l | l | }
\hline
\headercolor{} \IFRU{термин в Си}{C term} & 
\headercolor{} \IFRU{термин в ARM}{ARM term} & 
\headercolor{} \IFRU{выражение Си}{C statement} & 
\headercolor{} \IFRU{как это работает}{how it works} \\
\hline
\PostIncrement & 
post-indexed addressing & 
\TT{*ptr++} & 
\IFRU{использовать значение \TT{*ptr}}{use \TT{*ptr} value}, \\
& & & \IFRU{затем инкремент указателя \TT{ptr}}{then \gls{increment} \TT{ptr} pointer} \\
\hline
\PostDecrement & 
post-indexed addressing & 
\TT{*ptr-{}-} & 
\IFRU{использовать значение \TT{*ptr}}{use \TT{*ptr} value}, \\
& & & \IFRU{затем \glslink{decrement}{декремент} указателя \TT{ptr}}{then \gls{decrement} \TT{ptr} pointer} \\
\hline
\PreIncrement & 
pre-indexed addressing & 
\TT{*++ptr} & 
\IFRU{инкремент указателя \TT{ptr}}{\gls{increment} \TT{ptr} pointer}, \\
& & & \IFRU{затем использовать значение \TT{*ptr}}{then use \TT{*ptr} value} \\
\hline
\PreDecrement & 
post-indexed addressing & 
\TT{*-{}-ptr} & 
\IFRU{\glslink{decrement}{декремент} указателя \TT{ptr}}{\gls{decrement} \TT{ptr} pointer}, \\
& & & \IFRU{затем использовать значение \TT{*ptr}}{then use \TT{*ptr} value} \\
\hline
\end{tabular}
\end{center}

\IFRU{Деннис Ритчи (один из создателей ЯП Си) указывал, что, это, вероятно, придумал Кен Томпсон 
(еще один создатель Си),
потому что подобная возможность процессора имелась еще в PDP-7}
{Dennis Ritchie (one of C language creators) mentioned that it is, probably, was invented by Ken Thompson
(another C creator) because this processor feature was present in PDP-7}
\cite{Ritchie:1986}\cite{Ritchie:1993:DCL:155360.155580}.
\IFRU{Таким образом, компиляторы с ЯП Си на тот процессор, где это есть, могут использовать это.}
{Thus, C language compilers may use it, if it is present in target processor.}

\IFRU{Далее в теле цикла можно увидеть \CMP и \ac{BNE}, они продолжают работу цикла до тех пор, 
пока не будет встречен $0$.}
{Then one may spot \CMP and \ac{BNE} in loop body, these instructions continue operation until
$0$ will be met in string.}

\index{ARM!\Instructions!MVNS}
\index{x86!\Instructions!NOT}
\RU{После конца цикла }\TT{MVNS}\footnote{MoVe Not} 
\IFRU{(инвертирование всех бит, аналог \NOT на x86)}
{(inverting all bits, \NOT in x86 analogue)}
\IFRU{и \ADD вычисляют}{instructions and \ADD computes} $eos - str - 1$.
\IFRU{На самом деле, эти две инструкции вычисляют}
{In fact, these two instructions computes}
$R0 = ~str + eos$, 
\IFRU{что эквивалентно тому, что было в исходном коде, а почему это так, я уже описывал чуть раньше, здесь}
{which is effectively equivalent to what was in source code, and why it is so, I already described here}
~(\ref{strlen_NOT_ADD}).

\IFRU{Вероятно, LLVM, как и GCC, посчитал что такой код будет короче, или быстрее.}
{Apparently, LLVM, just like GCC, concludes this code will be shorter, or faster.}

%\subsection{\OptimizingXcode + \ARMMode}
%Практически, точно такой же код.

\subsection{\OptimizingKeil{} + \ARMMode}

\lstinputlisting[caption=\OptimizingKeil + \ARMMode]{patterns/10_strlen/Keil_ARM_O3.asm}

\index{ARM!\Instructions!SUBEQ}
\IFRU{Практически то же самое что мы уже видели, за тем исключением что выражение}
{Almost the same what we saw before, with the exception the}
$str - eos - 1$ 
\IFRU{может быть вычислено не в самом конце функции, а прямо в теле цикла.}
{expression may be computed not at the function's end, but right in loop body.}
\RU{Суффикс }\TT{-EQ}\IFRU{, как мы помним, означает что инструкция будет выполнена только
если операнды в исполненной перед этим инструкции \CMP были равны.}
{suffix, as we may recall, means the instruction will be executed only if operands in executed before
\CMP were equal to each other.}
\IFRU{Таким образом}{Thus}, \IFRU{если в \Reg{0} будет $0$}{if $0$ will be in the \Reg{0} register},
\IFRU{обе инструкции}{both} \TT{SUBEQ} \IFRU{исполнятся и результат останется в \Reg{0}.}
{instructions are to be executed and result is leaving in the \Reg{0} register.}


\fi
\ifdefined\IncludeMIPS
\subsection{MIPS}

\lstinputlisting[caption=\Optimizing GCC 4.4.5 (IDA)]{patterns/14_bitfields/2_set_reset/MIPS_O3_IDA.lst}

\index{MIPS!\Instructions!ORI}
\RU{ORI это, конечно, операция ``ИЛИ'', ``I'' в имени инструкции означает что значение встроено в машинный код.}
\EN{ORI, of course, OR operation. ``I'' in instruction name mean that value is embedded into machine code.}

\index{MIPS!\Instructions!AND}
\RU{И напротив, есть AND. Здесь нет возможности использовать ANDI, потому что невозможно встроить число 
0xFFFFFDFF в одну инструкцию, так что компилятору приходится в начале загружать значение 0xFFFFFDFF в регистр \$V0,
а затем генерировать AND, которая возьмет все значения из регистров.}
\EN{On contrast, here is AND. There was no way to use ANDI because it's not possible to embed 0xFFFFFDFF number
into one single instruction, so compiler ought to load 0xFFFFFDFF value into \$V0 register first and then generate
AND which take all the values from registers.}

\fi

