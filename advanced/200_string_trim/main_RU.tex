\mysection{Обрезка строк}
\newcommand{\CRLF}{\ac{CR}/\ac{LF}}

Весьма востребованная операция со строками --- это удаление некоторых символов в начале и/или конце
строки.

В этом примере, мы будем работать с функцией, удаляющей все символы перевода строки 
(\CRLF{}) в конце входной строки:

\lstinputlisting[style=customc]{\CURPATH/strtrim_RU.c}

Входной аргумент всегда возвращается на выходе, это удобно, когда вам нужно объединять
функции обработки строк в цепочки, как это сделано здесь в функции \main.

\myindex{\CLanguageElements!Short-circuit}
Вторая часть for() (\TT{str\_len>0 \&\& (c=s[str\_len-1])}) называется в \CCpp \q{short-circuit} 
(короткое замыкание) и это очень удобно: \InSqBrackets{\CNotes 1.3.8}.

Компиляторы \CCpp гарантируют последовательное вычисление слева направо.

Так что если первое условие не истинно после вычисления, второе никогда не будет
вычисляться.

% subsections
\subsubsection{x64: 8 аргументов}

\myindex{x86-64}
\label{example_printf8_x64}
Для того чтобы посмотреть, как остальные аргументы будут передаваться через стек, 
изменим пример ещё раз, 
увеличив количество передаваемых аргументов до 9 
(строка формата \printf и 8 переменных типа \Tint):

\lstinputlisting[style=customc]{patterns/03_printf/2.c}

\myparagraph{MSVC}

Как уже было сказано ранее, первые 4 аргумента в Win64 передаются в регистрах \RCX, \RDX, \Reg{8}, \Reg{9}, а остальные~--- через стек.
Здесь мы это и видим.
Впрочем, инструкция \PUSH не используется, вместо неё при помощи \MOV значения сразу записываются в стек.

\lstinputlisting[caption=MSVC 2012 x64,style=customasmx86]{patterns/03_printf/x86/2_MSVC_x64_RU.asm}

Наблюдательный читатель может спросить, почему для значений типа \Tint отводится 8 байт, ведь нужно только 4?
Да, это нужно запомнить: для значений всех типов более коротких чем 64-бита, отводится 8 байт.
Это сделано для удобства: так всегда легко рассчитать адрес того или иного аргумента.
К тому же, все они расположены по выровненным адресам в памяти.
В 32-битных средах точно также: для всех типов резервируется 4 байта в стеке.

% also for local variables?

\myparagraph{GCC}

В *NIX-системах для x86-64 ситуация похожая, вот только первые 6 аргументов передаются через
\RDI, \RSI, \RDX, \RCX, \Reg{8}, \Reg{9}.
Остальные~--- через стек.
GCC генерирует код, записывающий указатель на строку в \EDI вместо \RDI~--- 
это мы уже рассмотрели чуть раньше: \myref{hw_EDI_instead_of_RDI}.

Почему перед вызовом \printf очищается регистр \EAX мы уже рассмотрели ранее \myref{SysVABI_input_EAX}.

\lstinputlisting[caption=\Optimizing GCC 4.4.6 x64,style=customasmx86]{patterns/03_printf/x86/2_GCC_x64_RU.s}

\myparagraph{GCC + GDB}
\myindex{GDB}

Попробуем этот пример в \ac{GDB}.

\begin{lstlisting}
$ gcc -g 2.c -o 2
\end{lstlisting}

\begin{lstlisting}
$ gdb 2
GNU gdb (GDB) 7.6.1-ubuntu
...
Reading symbols from /home/dennis/polygon/2...done.
\end{lstlisting}

\begin{lstlisting}[caption=ставим точку останова на \printf{,} запускаем]
(gdb) b printf
Breakpoint 1 at 0x400410
(gdb) run
Starting program: /home/dennis/polygon/2 

Breakpoint 1, __printf (format=0x400628 "a=%d; b=%d; c=%d; d=%d; e=%d; f=%d; g=%d; h=%d\n") at printf.c:29
29	printf.c: No such file or directory.
\end{lstlisting}

В регистрах \RSI/\RDX/\RCX/\Reg{8}/\Reg{9} 
всё предсказуемо.
А \RIP содержит адрес самой первой инструкции функции \printf{}.

\begin{lstlisting}
(gdb) info registers
rax            0x0	0
rbx            0x0	0
rcx            0x3	3
rdx            0x2	2
rsi            0x1	1
rdi            0x400628	4195880
rbp            0x7fffffffdf60	0x7fffffffdf60
rsp            0x7fffffffdf38	0x7fffffffdf38
r8             0x4	4
r9             0x5	5
r10            0x7fffffffdce0	140737488346336
r11            0x7ffff7a65f60	140737348263776
r12            0x400440	4195392
r13            0x7fffffffe040	140737488347200
r14            0x0	0
r15            0x0	0
rip            0x7ffff7a65f60	0x7ffff7a65f60 <__printf>
...
\end{lstlisting}

\begin{lstlisting}[caption=смотрим на строку формата]
(gdb) x/s $rdi
0x400628:	"a=%d; b=%d; c=%d; d=%d; e=%d; f=%d; g=%d; h=%d\n"
\end{lstlisting}

Дампим стек на этот раз с командой x/g --- \IT{g} означает \IT{giant words}, т.е. 64-битные слова.

\begin{lstlisting}
(gdb) x/10g $rsp
0x7fffffffdf38:	0x0000000000400576	0x0000000000000006
0x7fffffffdf48:	0x0000000000000007	0x00007fff00000008
0x7fffffffdf58:	0x0000000000000000	0x0000000000000000
0x7fffffffdf68:	0x00007ffff7a33de5	0x0000000000000000
0x7fffffffdf78:	0x00007fffffffe048	0x0000000100000000
\end{lstlisting}

Самый первый элемент стека, как и в прошлый раз, это \ac{RA}.
Через стек также передаются 3 значения: 6, 7, 8.
Видно, что 8 передается с неочищенной старшей 32-битной частью: \GTT{0x00007fff00000008}.
Это нормально, ведь передаются числа типа \Tint, а они 32-битные.
Так что в старшей части регистра или памяти стека остался \q{случайный мусор}.

\ac{GDB} показывает всю функцию \main, если попытаться посмотреть, куда вернется управление после исполнения \printf{}.

\begin{lstlisting}[style=customasmx86]
(gdb) set disassembly-flavor intel
(gdb) disas 0x0000000000400576
Dump of assembler code for function main:
   0x000000000040052d <+0>:	push   rbp
   0x000000000040052e <+1>:	mov    rbp,rsp
   0x0000000000400531 <+4>:	sub    rsp,0x20
   0x0000000000400535 <+8>:	mov    DWORD PTR [rsp+0x10],0x8
   0x000000000040053d <+16>:	mov    DWORD PTR [rsp+0x8],0x7
   0x0000000000400545 <+24>:	mov    DWORD PTR [rsp],0x6
   0x000000000040054c <+31>:	mov    r9d,0x5
   0x0000000000400552 <+37>:	mov    r8d,0x4
   0x0000000000400558 <+43>:	mov    ecx,0x3
   0x000000000040055d <+48>:	mov    edx,0x2
   0x0000000000400562 <+53>:	mov    esi,0x1
   0x0000000000400567 <+58>:	mov    edi,0x400628
   0x000000000040056c <+63>:	mov    eax,0x0
   0x0000000000400571 <+68>:	call   0x400410 <printf@plt>
   0x0000000000400576 <+73>:	mov    eax,0x0
   0x000000000040057b <+78>:	leave  
   0x000000000040057c <+79>:	ret    
End of assembler dump.
\end{lstlisting}

Заканчиваем исполнение \printf, исполняем инструкцию обнуляющую \EAX, удостоверяемся что в регистре \EAX именно ноль.
\RIP указывает сейчас на инструкцию \INS{LEAVE}, т.е. предпоследнюю в функции \main{}.

\begin{lstlisting}
(gdb) finish
Run till exit from #0  __printf (format=0x400628 "a=%d; b=%d; c=%d; d=%d; e=%d; f=%d; g=%d; h=%d\n") at printf.c:29
a=1; b=2; c=3; d=4; e=5; f=6; g=7; h=8
main () at 2.c:6
6		return 0;
Value returned is $1 = 39
(gdb) next
7	};
(gdb) info registers
rax            0x0	0
rbx            0x0	0
rcx            0x26	38
rdx            0x7ffff7dd59f0	140737351866864
rsi            0x7fffffd9	2147483609
rdi            0x0	0
rbp            0x7fffffffdf60	0x7fffffffdf60
rsp            0x7fffffffdf40	0x7fffffffdf40
r8             0x7ffff7dd26a0	140737351853728
r9             0x7ffff7a60134	140737348239668
r10            0x7fffffffd5b0	140737488344496
r11            0x7ffff7a95900	140737348458752
r12            0x400440	4195392
r13            0x7fffffffe040	140737488347200
r14            0x0	0
r15            0x0	0
rip            0x40057b	0x40057b <main+78>
...
\end{lstlisting}

\myparagraph{ARM64}

\mysubparagraph{\Optimizing GCC (Linaro) 4.9}

\lstinputlisting[style=customasmARM]{patterns/10_strings/1_strlen/ARM/ARM64_GCC_O3_RU.lst}

Алгоритм такой же как и в \myref{strlen_MSVC_Ox}: 
найти нулевой байт, затем вычислить разницу между указателями, затем отнять 1 от результата.
Комментарии добавлены автором книги.

Стоит добавить, что наш пример имеет ошибку: \TT{my\_strlen()}
возвращает 32-битный \Tint, тогда как должна возвращать \TT{size\_t} или иной 64-битный тип.

Причина в том, что теоретически, \TT{strlen()} можно вызывать для огромных блоков в памяти,
превышающих 4GB, так что она должна иметь возможность вернуть 64-битное значение на 64-битной платформе.

Так что из-за моей ошибки, последняя инструкция \SUB работает над 32-битной частью регистра, тогда
как предпоследняя \SUB работает с полными 64-битными частями (она вычисляет разницу между указателями).

Это моя ошибка, но лучше оставить это как есть, как пример кода, который возможен в таком случае.

\mysubparagraph{\NonOptimizing GCC (Linaro) 4.9}

\lstinputlisting[style=customasmARM]{patterns/10_strings/1_strlen/ARM/ARM64_GCC_O0_RU.lst}

Более многословно.
Переменные часто сохраняются в память и загружаются назад (локальный стек).
Здесь та же ошибка: операция декремента происходит над 32-битной частью регистра.


\subsubsection{ARM}

\myparagraph{\OptimizingKeilVI (\ThumbMode)}

\lstinputlisting[caption=\OptimizingKeilVI (\ThumbMode),style=customasmARM]{patterns/15_structs/4_packing/packing_Keil_thumb.asm}

Как мы помним, здесь передается не указатель на структуру, а сама структура, а так как в ARM первые 4 аргумента
функции передаются через регистры, то поля структуры передаются через \TT{R0-R3}.

\myindex{ARM!\Instructions!LDRB}
\myindex{x86!\Instructions!MOVSX}
Инструкция \TT{LDRB} загружает один байт из памяти и расширяет до 32-бит учитывая знак.

Это то же что и инструкция \MOVSX в x86.
Она здесь применяется для загрузки полей $a$ и $c$ из структуры.

\myindex{Function epilogue}
Еще что бросается в глаза, так это то что вместо эпилога функции, переход на эпилог другой функции!

Действительно, то была совсем другая, не относящаяся к этой, функция, однако, она имела точно такой же эпилог 
(видимо, тоже хранила в стеке 5 локальных переменных ($5*4=0x14$)).
К тому же, она находится рядом (обратите внимание на адреса).

Действительно, нет никакой разницы, какой эпилог исполнять, если он работает так же, как нам нужно.

Keil решил использовать часть другой функции, вероятно, из-за экономии.

Эпилог занимает 4 байта, а переход ~--- только 2.

\myparagraph{ARM + \OptimizingXcodeIV (\ThumbTwoMode)}

\lstinputlisting[caption=\OptimizingXcodeIV (\ThumbTwoMode),style=customasmARM]{patterns/15_structs/4_packing/packing_Xcode_thumb.asm}

\myindex{ARM!\Instructions!SXTB}
\myindex{x86!\Instructions!MOVSX}
\TT{SXTB} (\IT{Signed Extend Byte}) это также аналог \MOVSX в x86.
Всё остальное ~--- так же.


\subsubsection{MIPS}

\myindex{MIPS!\Registers!FCCR}

В сопроцессоре MIPS есть бит результата, который устанавливается в FPU и проверяется в CPU.

Ранние MIPS имели только один бит (с названием FCC0), а у поздних их 8 (с названием FCC7-FCC0).
Этот бит (или биты) находятся в регистре с названием FCCR.

\lstinputlisting[caption=\Optimizing GCC 4.4.5 (IDA),style=customasmMIPS]{patterns/12_FPU/3_comparison/MIPS_O3_IDA_RU.lst}

\myindex{MIPS!\Instructions!C.LT.D}
\INS{C.LT.D} сравнивает два значения. 
\GTT{LT} это условие \q{Less Than} (меньше чем).
\GTT{D} означает переменные типа \Tdouble.

В зависимости от результата сравнения, бит FCC0 устанавливается или очищается.

\myindex{MIPS!\Instructions!BC1T}
\myindex{MIPS!\Instructions!BC1F}
\INS{BC1T} проверяет бит FCC0 и делает переход, если бит выставлен.
\GTT{T} означает, что переход произойдет если бит выставлен (\q{True}).
Имеется также инструкция \INS{BC1F} которая сработает, если бит сброшен (\q{False}).

В зависимости от перехода один из аргументов функции помещается в регистр \$F0.



