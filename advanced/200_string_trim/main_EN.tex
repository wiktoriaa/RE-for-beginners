\mysection{Strings trimming}
\newcommand{\CRLF}{\ac{CR}/\ac{LF}}

A very common string processing task is to remove some characters at the start and/or at the end.

In this example, we are going to work with a function which removes all newline characters 
(\CRLF{}) from the end of the input string:

\lstinputlisting[style=customc]{\CURPATH/strtrim_EN.c}

The input argument is always returned on exit, this is convenient when you want to chain 
string processing functions, like it has done here in the \main function.

\myindex{\CLanguageElements!Short-circuit}
The second part of for() (\TT{str\_len>0 \&\& (c=s[str\_len-1])}) is the so called \q{short-circuit} 
in \CCpp and is very convenient \InSqBrackets{\CNotes 1.3.8}.

The \CCpp compilers guarantee an evaluation sequence from left to right.

So if the first clause is false after evaluation, the second one is never to be evaluated.

% subsections
\subsection{x64}

\myindex{x86-64}
The picture here is similar with the difference that the registers, rather than the stack, are used for arguments passing.

\subsubsection{MSVC}

\lstinputlisting[caption=MSVC 2012 x64]{patterns/04_scanf/1_simple/ex1_MSVC_x64.asm.\LANG}

\ifdefined\IncludeGCC
\subsubsection{GCC}

\lstinputlisting[caption=\Optimizing GCC 4.4.6 x64]{patterns/04_scanf/1_simple/ex1_GCC_x64.s.\LANG}
\fi


\subsubsection{ARM64}

\myparagraph{\Optimizing GCC (Linaro) 4.9}

\myindex{Fused multiply–add}
\myindex{ARM!\Instructions!MADD}
Everything here is simple.
\TT{MADD} is just an instruction doing fused multiply/add (similar to the \TT{MLA} we already saw).
All 3 arguments are passed in the 32-bit parts of X-registers.
Indeed, the argument types are 32-bit \IT{int}'s.
The result is returned in \TT{W0}.

\lstinputlisting[caption=\Optimizing GCC (Linaro) 4.9,style=customasmARM]{patterns/05_passing_arguments/ARM/ARM64_O3_EN.s}

Let's also extend all data types to 64-bit \TT{uint64\_t} and test:

\lstinputlisting[style=customc]{patterns/05_passing_arguments/ex64.c}

\begin{lstlisting}[style=customasmARM]
f:
	madd	x0, x0, x1, x2
	ret
main:
	mov	x1, 13396
	adrp	x0, .LC8
	stp	x29, x30, [sp, -16]!
	movk	x1, 0x27d0, lsl 16
	add	x0, x0, :lo12:.LC8
	movk	x1, 0x122, lsl 32
	add	x29, sp, 0
	movk	x1, 0x58be, lsl 48
	bl	printf
	mov	w0, 0
	ldp	x29, x30, [sp], 16
	ret

.LC8:
	.string	"%lld\n"
\end{lstlisting}

The \ttf{} function is the same, only the whole 64-bit X-registers are now used.
Long 64-bit values are loaded into the registers by parts, this is also described here: \myref{ARM_big_constants_loading}.

\myparagraph{\NonOptimizing GCC (Linaro) 4.9}

The non-optimizing compiler is more redundant:

\begin{lstlisting}[style=customasmARM]
f:
	sub	sp, sp, #16
	str	w0, [sp,12]
	str	w1, [sp,8]
	str	w2, [sp,4]
	ldr	w1, [sp,12]
	ldr	w0, [sp,8]
	mul	w1, w1, w0
	ldr	w0, [sp,4]
	add	w0, w1, w0
	add	sp, sp, 16
	ret
\end{lstlisting}

The code saves its input arguments in the local stack, 
in case someone (or something) in this function needs using the \TT{W0...W2} 
registers. This prevents overwriting the original
function arguments, which may be needed again in the future.

This is called \IT{Register Save Area.} (\ARMPCS).
The callee, however, is not obliged to save them.
This is somewhat similar to \q{Shadow Space}: \myref{shadow_space}.

Why did the optimizing GCC 4.9 drop this argument saving code?
Because it did some additional optimizing work and concluded
that the function arguments will not be needed in the future 
and also that the registers \TT{W0...W2} will not be used.

\myindex{ARM!\Instructions!MUL}
\myindex{ARM!\Instructions!ADD}

We also see a \TT{MUL}/\TT{ADD} instruction pair instead of single a \TT{MADD}.

\subsubsection{ARM}

\myparagraph{\OptimizingKeilVI (\ThumbMode)}

\lstinputlisting[style=customasmARM]{patterns/04_scanf/1_simple/ARM_IDA.lst}

\myindex{\CLanguageElements!\Pointers}

In order for \scanf to be able to read item it needs a parameter---pointer to an \Tint.
\Tint is 32-bit, so we need 4 bytes to store it somewhere in memory, and it fits exactly in a 32-bit register.
\myindex{IDA!var\_?}
A place for the local variable \GTT{x} is allocated in the stack and \IDA
has named it \IT{var\_8}. It is not necessary, however, to allocate a such since \ac{SP} (\gls{stack pointer}) is already pointing to that space and it can be used directly.

So, \ac{SP}'s value is copied to the \Reg{1} register and, together with the format-string, passed to \scanf.
\myindex{ARM!\Instructions!LDR}
Later, with the help of the \INS{LDR} instruction, this value is moved from the stack to the \Reg{1} register in order to be passed to \printf.

\myparagraph{ARM64}

\lstinputlisting[caption=\NonOptimizing GCC 4.9.1 ARM64,numbers=left,style=customasmARM]{patterns/04_scanf/1_simple/ARM64_GCC491_O0_EN.s}

There is 32 bytes are allocated for stack frame, which is bigger than it needed. Perhaps some memory aligning issue?
The most interesting part is finding space for the $x$ variable in the stack frame (line 22).
Why 28? Somehow, compiler decided to place this variable at the end of stack frame instead of beginning.
The address is passed to \scanf, which just stores the user input value in the memory at that address.
This is 32-bit value of type \Tint.
The value is fetched at line 27 and then passed to \printf.


\subsubsection{MIPS}

\lstinputlisting[caption=\Optimizing GCC 4.4.5 (IDA),style=customasmMIPS]{patterns/12_FPU/2_passing_floats/MIPS_O3_IDA_EN.lst}

And again, we see here \INS{LUI} loading a 32-bit part of a \Tdouble number into \$V0.
And again, it's hard to comprehend why.

\myindex{MIPS!\Instructions!MFC1}

The new instruction for us here is \INS{MFC1} (\q{Move From Coprocessor 1}).
The FPU is coprocessor number 1, hence \q{1} in the instruction name.
This instruction transfers values from the coprocessor's registers to the registers of the CPU (\ac{GPR}).
So at the end the result of \TT{pow()} is moved to registers \$A3 and \$A2, 
and \printf takes a 64-bit double value from this register pair.



