\section{Больше об указателях}
\myindex{\CLanguageElements!\Pointers}
\label{label_pointers}

\epigraph{The way C handles pointers, for example, was a brilliant innovation;
it solved a lot of problems that we had before in data structuring and
made the programs look good afterwards.}{Дональд Кнут, интервью (1993)}

Для тех, кому все еще трудно понимать указатели в \CCpp{}, вот еще примеры.
Некоторые из них крайне странные и служат только демонстрационным целям:
использовать подобное в production-коде можно только если вы действительно понимаете, что вы делаете.

\subsection{Работа с адресами вместо указателей}

Указатель это просто адрес в памяти. Но почему мы пишем \TT{char* string} вместо чего-нибудь вроде \TT{address string}?
Переменная-указатель дополнена типом переменной, на которую указатель указывает.
Тогда у компилятора будет возможность находить потенциальные ошибки типизации во время компиляции.

Если быть педантом, типизация данных в языках программирования существует для предотвращения ошибок и самодокументации.
Вполне возможно использовать только два типа данных вроде \IT{int} (или \IT{int64\_t}) и байт --- это те единственные типы, которые
доступны для программистов на ассемблере.
Но написать что-то больше и практичное на ассемблере, при этом без ошибок, это трудная задача.
Любая мелкая опечатка может привести к труднонаходимой ошибке.

Информации о типах нет в скомпилированном коде (и это одна из основных проблем для декомпиляторов),
и я могу это продемонстрировать.

Вот как напишет обычный программист на \CCpp{}:

\begin{lstlisting}[style=customc]
#include <stdio.h>
#include <stdint.h>

void print_string (char *s)
{
	printf ("(address: 0x%llx)\n", s);
	printf ("%s\n", s);
};

int main()
{
	char *s="Hello, world!";

	print_string (s);
};
\end{lstlisting}

А вот что могу напистаь я:

\begin{lstlisting}[style=customc]
#include <stdio.h>
#include <stdint.h>

void print_string (uint64_t address)
{
	printf ("(address: 0x%llx)\n", address);
	puts ((char*)address);
};

int main()
{
	char *s="Hello, world!";

	print_string ((uint64_t)s);
};
\end{lstlisting}

Я использую \IT{uint64\_t} потому что я запускаю этот пример на Linux x64. \IT{int} сгодится для 32-битных \ac{OS}.
В начале, указатель на символ (самый первый в строке с приветствием) приводится к \IT{uint64\_t}, затем он передается далее.
Ф-ция \TT{print\_string()} приводит тип переданного значения из \IT{uint64\_t} в указатель на символ.

Но вот что интересно, это то что GCC 4.8.4 генерирует идентичный результат на ассемблере для обоих версий:

\begin{lstlisting}
gcc 1.c -S -masm=intel -O3 -fno-inline
\end{lstlisting}

\begin{lstlisting}[style=customasmx86]
.LC0:
	.string	"(address: 0x%llx)\n"
print_string:
	push	rbx
	mov	rdx, rdi
	mov	rbx, rdi
	mov	esi, OFFSET FLAT:.LC0
	mov	edi, 1
	xor	eax, eax
	call	__printf_chk
	mov	rdi, rbx
	pop	rbx
	jmp	puts
.LC1:
	.string	"Hello, world!"
main:
	sub	rsp, 8
	mov	edi, OFFSET FLAT:.LC1
	call	print_string
	add	rsp, 8
	ret
\end{lstlisting}

(Я убрал незначительные директивы GCC.)

Я также пробовал утилиту UNIX \IT{diff} и не нашел разницы вообще.

Продолжим и дальше издеваться над традициями программирования в \CCpp{}.
Кто-то может написать так:

\begin{lstlisting}[style=customc]
#include <stdio.h>
#include <stdint.h>

uint8_t load_byte_at_address (uint8_t* address)
{
	return *address;
	//this is also possible: return address[0]; 
};

void print_string (char *s)
{
	char* current_address=s;
	while (1)
	{
		char current_char=load_byte_at_address(current_address);
		if (current_char==0)
			break;
		printf ("%c", current_char);
		current_address++;
	};
};

int main()
{
	char *s="Hello, world!";

	print_string (s);
};
\end{lstlisting}

И это может быьт переписано так:

\begin{lstlisting}[style=customc]
#include <stdio.h>
#include <stdint.h>

uint8_t load_byte_at_address (uint64_t address)
{
	return *(uint8_t*)address;
	//this is also possible: return address[0]; 
};

void print_string (uint64_t address)
{
	uint64_t current_address=address;
	while (1)
	{
		char current_char=load_byte_at_address(current_address);
		if (current_char==0)
			break;
		printf ("%c", current_char);
		current_address++;
	};
};

int main()
{
	char *s="Hello, world!";

	print_string ((uint64_t)s);
};
\end{lstlisting}

И тот и другой исходный код преобразуется в одинаковый результат на ассемблере:

\begin{lstlisting}
gcc 1.c -S -masm=intel -O3 -fno-inline
\end{lstlisting}

\begin{lstlisting}[style=customasmx86]
load_byte_at_address:
	movzx	eax, BYTE PTR [rdi]
	ret
print_string:
.LFB15:
	push	rbx
	mov	rbx, rdi
	jmp	.L4
.L7:
	movsx	edi, al
	add	rbx, 1
	call	putchar
.L4:
	mov	rdi, rbx
	call	load_byte_at_address
	test	al, al
	jne	.L7
	pop	rbx
	ret
.LC0:
	.string	"Hello, world!"
main:
	sub	rsp, 8
	mov	edi, OFFSET FLAT:.LC0
	call	print_string
	add	rsp, 8
	ret
\end{lstlisting}

(Здесь я также убрал незначительные директивы GCC.)

Разницы нет: указатели в \CCpp{}, в сущности, адреса, но несут в себе также информацию о типе, чтобы предотвратить ошибки
во время компиляции.
Типы не проверяются во время исполнения, иначе это был бы огромный (и ненужный) прирост времени исполнения.


\subsection{Передача значений как указателей; тэггированные объединения}

Вот как можно передавать обычные значения как указатели:

\begin{lstlisting}[label=unsigned_multiply_C,style=customc]
#include <stdio.h>
#include <stdint.h>

uint64_t multiply1 (uint64_t a, uint64_t b)
{
	return a*b;
};

uint64_t* multiply2 (uint64_t *a, uint64_t *b)
{
	return (uint64_t*)((uint64_t)a*(uint64_t)b);
};

int main()
{
	printf ("%d\n", multiply1(123, 456));
	printf ("%d\n", (uint64_t)multiply2((uint64_t*)123, (uint64_t*)456));
};
\end{lstlisting}

Это работает нормально и GCC 4.8.4 компилирует обе ф-ции multiply1() и multiply2() полностью идентично!

\begin{lstlisting}[label=unsigned_multiply_lst,style=customasmx86]
multiply1:
	mov	rax, rdi
	imul	rax, rsi
	ret

multiply2:
	mov	rax, rdi
	imul	rax, rsi
	ret
\end{lstlisting}

Пока вы не разыменовываете указатель (\IT{dereference}) (иными словами, если вы не пытаетесь прочитать данные
по адресу в указателе), всё будет работать нормально.
Указатель это переменная, которая может содержать что угодно, как и обычная переменная.

\myindex{x86!\Instructions!MUL}
\myindex{x86!\Instructions!IMUL}
Здесь используется инструкция для знакового умножения (\IMUL) вместо беззнакового (\MUL), об этом читайте больше здесь:
\ref{IMUL_over_MUL}.

\myindex{Тэггированные указатели}
Кстати, это широко известный хак, называющийся \IT{tagged pointers}.
Если коротко, если все ваши указатели указывают на блоки в памяти размером, скажем, 16 байт (или они всегда
выровнены по 16-байтной границе), 4 младших бита указателя буут всегда нулевыми, и это пространство может быть
как-то использовано.
\myindex{LISP}
Это очень популярно в компиляторах и интерпретаторах LISP.
Они хранят тип ячейки/объекта в неиспользующихся битах, и так можно сэкономить немного памяти.
И более того --- имея только указатель, можно сразу выяснить тип ячейки/объекта, без дополнительного обращения к памяти.
Читайте об этом больше: \InSqBrackets{\CNotes 1.3}.

% TODO Example of tagged ptrs here


\subsection{Издевательство над указателями в ядре Windows}

Секция ресурсов в исполняемых файлах типа PE в Windows это секция, содержащая картинки, иконки, строки, итд.
Ранние версии Windows позволяли иметь к ним доступ только при помощи идентификаторов, но потом в Microsoft добавили
также и способ адресовать ресурсы при помощи строк.

Так что потом стало возможным передать идентификатор или строку в ф-цию
\href{https://msdn.microsoft.com/en-us/library/windows/desktop/ms648042%28v=vs.85%29.aspx}{FindResource()}.
Которая декларирована вот так:

\myindex{win32!FindResource()}

\begin{lstlisting}[style=customc]
HRSRC WINAPI FindResource(
  _In_opt_ HMODULE hModule,
  _In_     LPCTSTR lpName,
  _In_     LPCTSTR lpType
);
\end{lstlisting}

\IT{lpName} и \IT{lpType} имеют тип \IT{char*} или \IT{wchar*}, и когда кто-то всё еще хочет передать идентификатор,
нужно использовать макрос
\href{https://msdn.microsoft.com/en-us/library/windows/desktop/ms648029%28v=vs.85%29.aspx}{MAKEINTRESOURCE}, вот так:

\myindex{win32!MAKEINTRESOURCE()}

\begin{lstlisting}[style=customc]
result = FindResource(..., MAKEINTRESOURCE(1234), ...);
\end{lstlisting}

Очень интересно то, что всё что делает MAKEINTRESOURCE это приводит целочисленное к указателю.
В MSVC 2013, в файле\\
\IT{Microsoft SDKs\textbackslash{}Windows\textbackslash{}v7.1A\textbackslash{}Include\textbackslash{}Ks.h},
мы можем найти это:

\begin{lstlisting}[style=customc]
...

#if (!defined( MAKEINTRESOURCE )) 
#define MAKEINTRESOURCE( res ) ((ULONG_PTR) (USHORT) res)
#endif

...
\end{lstlisting}

Звучит безумно. Заглянем внутрь древнего, когда-то утекшего, исходного кода Windows NT4.
В \IT{private/windows/base/client/module.c} мы можем найти исходный код \IT{FindResource()}:

\begin{lstlisting}[style=customc]
HRSRC
FindResourceA(
    HMODULE hModule,
    LPCSTR lpName,
    LPCSTR lpType
    )

...

{
    NTSTATUS Status;
    ULONG IdPath[ 3 ];
    PVOID p;

    IdPath[ 0 ] = 0;
    IdPath[ 1 ] = 0;
    try {
        if ((IdPath[ 0 ] = BaseDllMapResourceIdA( lpType )) == -1) {
            Status = STATUS_INVALID_PARAMETER;
            }
        else
        if ((IdPath[ 1 ] = BaseDllMapResourceIdA( lpName )) == -1) {
            Status = STATUS_INVALID_PARAMETER;
...
\end{lstlisting}

Посмотрим в \IT{BaseDllMapResourceIdA()} в том же исходном файле:

\begin{lstlisting}[style=customc]
ULONG
BaseDllMapResourceIdA(
    LPCSTR lpId
    )
{
    NTSTATUS Status;
    ULONG Id;
    UNICODE_STRING UnicodeString;
    ANSI_STRING AnsiString;
    PWSTR s;

    try {
        if ((ULONG)lpId & LDR_RESOURCE_ID_NAME_MASK) {
            if (*lpId == '#') {
                Status = RtlCharToInteger( lpId+1, 10, &Id );
                if (!NT_SUCCESS( Status ) || Id & LDR_RESOURCE_ID_NAME_MASK) {
                    if (NT_SUCCESS( Status )) {
                        Status = STATUS_INVALID_PARAMETER;
                        }
                    BaseSetLastNTError( Status );
                    Id = (ULONG)-1;
                    }
                }
            else {
                RtlInitAnsiString( &AnsiString, lpId );
                Status = RtlAnsiStringToUnicodeString( &UnicodeString,
                                                       &AnsiString,
                                                       TRUE
                                                     );
                if (!NT_SUCCESS( Status )){
                    BaseSetLastNTError( Status );
                    Id = (ULONG)-1;
                    }
                else {
                    s = UnicodeString.Buffer;
                    while (*s != UNICODE_NULL) {
                        *s = RtlUpcaseUnicodeChar( *s );
                        s++;
                        }

                    Id = (ULONG)UnicodeString.Buffer;
                    }
                }
            }
        else {
            Id = (ULONG)lpId;
            }
        }
    except (EXCEPTION_EXECUTE_HANDLER) {
        BaseSetLastNTError( GetExceptionCode() );
        Id =  (ULONG)-1;
        }
    return Id;
}
\end{lstlisting}

К \IT{lpId} применяется операция ``И'' с \IT{LDR\_RESOURCE\_ID\_NAME\_MASK}. Маску можно найти в \IT{public/sdk/inc/ntldr.h}:

\begin{lstlisting}[style=customc]
...

#define LDR_RESOURCE_ID_NAME_MASK 0xFFFF0000

...
\end{lstlisting}

Так что к \IT{lpId} применяется операция ``И'' c \IT{0xFFFF0000}, и если присутствуют какие-либо биты за младшими 16 битами,
исполняется первая часто ф-ции и (\IT{lpId} принимается за адрес строки).
Иначе --- вторая часть ф-ции (\IT{lpId} принимается за 16-битное значение).

Этот же код можно найти и в Windows 7, в файле kernel32.dll:

\begin{lstlisting}[style=customasmx86]
....

.text:0000000078D24510 ; __int64 __fastcall BaseDllMapResourceIdA(PCSZ SourceString)
.text:0000000078D24510 BaseDllMapResourceIdA proc near         ; CODE XREF: FindResourceExA+34
.text:0000000078D24510                                         ; FindResourceExA+4B
.text:0000000078D24510
.text:0000000078D24510 var_38          = qword ptr -38h
.text:0000000078D24510 var_30          = qword ptr -30h
.text:0000000078D24510 var_28          = _UNICODE_STRING ptr -28h
.text:0000000078D24510 DestinationString= _STRING ptr -18h
.text:0000000078D24510 arg_8           = dword ptr  10h
.text:0000000078D24510
.text:0000000078D24510 ; FUNCTION CHUNK AT .text:0000000078D42FB4 SIZE 000000D5 BYTES
.text:0000000078D24510
.text:0000000078D24510                 push    rbx
.text:0000000078D24512                 sub     rsp, 50h
.text:0000000078D24516                 cmp     rcx, 10000h
.text:0000000078D2451D                 jnb     loc_78D42FB4
.text:0000000078D24523                 mov     [rsp+58h+var_38], rcx
.text:0000000078D24528                 jmp     short $+2
.text:0000000078D2452A ; ---------------------------------------------------------------------------
.text:0000000078D2452A
.text:0000000078D2452A loc_78D2452A:                           ; CODE XREF: BaseDllMapResourceIdA+18
.text:0000000078D2452A                                         ; BaseDllMapResourceIdA+1EAD0
.text:0000000078D2452A                 jmp     short $+2
.text:0000000078D2452C ; ---------------------------------------------------------------------------
.text:0000000078D2452C
.text:0000000078D2452C loc_78D2452C:                           ; CODE XREF: BaseDllMapResourceIdA:loc_78D2452A
.text:0000000078D2452C                                         ; BaseDllMapResourceIdA+1EB74
.text:0000000078D2452C                 mov     rax, rcx
.text:0000000078D2452F                 add     rsp, 50h
.text:0000000078D24533                 pop     rbx
.text:0000000078D24534                 retn
.text:0000000078D24534 ; ---------------------------------------------------------------------------
.text:0000000078D24535                 align 20h
.text:0000000078D24535 BaseDllMapResourceIdA endp

....

.text:0000000078D42FB4 loc_78D42FB4:                           ; CODE XREF: BaseDllMapResourceIdA+D
.text:0000000078D42FB4                 cmp     byte ptr [rcx], '#'
.text:0000000078D42FB7                 jnz     short loc_78D43005
.text:0000000078D42FB9                 inc     rcx
.text:0000000078D42FBC                 lea     r8, [rsp+58h+arg_8]
.text:0000000078D42FC1                 mov     edx, 0Ah
.text:0000000078D42FC6                 call    cs:__imp_RtlCharToInteger
.text:0000000078D42FCC                 mov     ecx, [rsp+58h+arg_8]
.text:0000000078D42FD0                 mov     [rsp+58h+var_38], rcx
.text:0000000078D42FD5                 test    eax, eax
.text:0000000078D42FD7                 js      short loc_78D42FE6
.text:0000000078D42FD9                 test    rcx, 0FFFFFFFFFFFF0000h
.text:0000000078D42FE0                 jz      loc_78D2452A

....

\end{lstlisting}

Если значение больше чем 0x10000, происходит переход в то место, где обрабатывается строка.
Иначе, входное значение \IT{lpId} возвращается как есть.
Маска \IT{0xFFFF0000} здесь больше не используется, т.к., это все же 64-битный код, но всё-таки,
маска \IT{0xFFFFFFFFFFFF0000} могла бы здесь использоваться.

Внимательный читатель может спросить, что если адрес входной строки будет ниже 0x10000?
Этот код полагается на тот факт, что в Windows нет ничего по адресам ниже 0x10000, по крайней мере, в Win32.

Raymond Chen \href{https://blogs.msdn.microsoft.com/oldnewthing/20130925-00/?p=3123}{пишет} об этом:

\begin{framed}
\begin{quotation}
How does MAKE­INT­RESOURCE work? It just stashes the integer in the bottom 16 bits of a pointer, leaving the upper bits zero. This relies on the convention that the first 64KB of address space is never mapped to valid memory, a convention that is enforced starting in Windows 7.
\end{quotation}
\end{framed}

Коротко говоря, это грязный хак, и наверное не стоит его использовать, если только нет большой необходимости.
Вероятно, аргумент ф-ции \IT{FindResource()} в прошлом имел тип \IT{SHORT}, а потом в Microsoft добавили возможность
передавать здесь и строки, но старый код также нужно было поддерживать.

Вот мой короткий очищенный пример:

\begin{lstlisting}[style=customc]
#include <stdio.h>
#include <stdint.h>

void f(char* a)
{
	if (((uint64_t)a)>0x10000)
		printf ("Pointer to string has been passed: %s\n", a);
	else
		printf ("16-bit value has been passed: %d\n", (uint64_t)a);
};

int main()
{
	f("Hello!"); // pass string
	f((char*)1234); // pass 16-bit value
};
\end{lstlisting}

Работает!

\subsubsection{Издевательство над указателями в ядре Linux}

Как было упомянуто среди \href{https://news.ycombinator.com/item?id=11823647}{комментариев на Hacker News},
в ядре Linux также есть что-то подобное.

Например, эта ф-ция может возвращать и код ошибки и указатель:

\begin{lstlisting}[style=customc]
struct kernfs_node *kernfs_create_link(struct kernfs_node *parent,
				       const char *name,
				       struct kernfs_node *target)
{
	struct kernfs_node *kn;
	int error;

	kn = kernfs_new_node(parent, name, S_IFLNK|S_IRWXUGO, KERNFS_LINK);
	if (!kn)
		return ERR_PTR(-ENOMEM);

	if (kernfs_ns_enabled(parent))
		kn->ns = target->ns;
	kn->symlink.target_kn = target;
	kernfs_get(target);	/* ref owned by symlink */

	error = kernfs_add_one(kn);
	if (!error)
		return kn;

	kernfs_put(kn);
	return ERR_PTR(error);
}
\end{lstlisting}

( \url{https://github.com/torvalds/linux/blob/fceef393a538134f03b778c5d2519e670269342f/fs/kernfs/symlink.c#L25} )

\IT{ERR\_PTR} это макрос, приводящий целочисленное к указателю:

\begin{lstlisting}[style=customc]
static inline void * __must_check ERR_PTR(long error)
{
	return (void *) error;
}
\end{lstlisting}

( \url{https://github.com/torvalds/linux/blob/61d0b5a4b2777dcf5daef245e212b3c1fa8091ca/tools/virtio/linux/err.h} )

Этот же заголовочный файл имеет также макрос, который можно использовать, чтобы отличить код ошибки от указателя:

\begin{lstlisting}[style=customc]
#define IS_ERR_VALUE(x) unlikely((x) >= (unsigned long)-MAX_ERRNO)
\end{lstlisting}

Это означает, коды ошибок это ``указатели'' очень близкие к -1, и, будем надеяться, в памяти ядра ничего не находится
по адресам вроде 0xFFFFFFFFFFFFFFFF, 0xFFFFFFFFFFFFFFFE, 0xFFFFFFFFFFFFFFFD, итд.

Намного более популярный способ это возвращать \IT{NULL} в случае ошибки и передавать код ошибки через дополнительный
аргумент.
Авторы ядры Linux так не делают, но все кто пользуется этими ф-циями, должны помнить, что возвращаемый указатель
должен быть вначале проверен при помощи \IT{IS\_ERR\_VALUE} перед разыменовыванием.

Например:

\begin{lstlisting}[style=customc]
	fman->cam_offset = fman_muram_alloc(fman->muram, fman->cam_size);
	if (IS_ERR_VALUE(fman->cam_offset)) {
		dev_err(fman->dev, "%s: MURAM alloc for DMA CAM failed\n",
			__func__);
		return -ENOMEM;
	}
\end{lstlisting}

( \url{https://github.com/torvalds/linux/blob/aa00edc1287a693eadc7bc67a3d73555d969b35d/drivers/net/ethernet/freescale/fman/fman.c#L826} )

\subsubsection{Издевательство над указателями в пользовательской среде UNIX}

\myindex{UNIX!mmap()}
Ф-ция mmap() возвращает -1 в случае ошибки (или \TT{MAP\_FAILED}, что равно -1).
Некоторые люди говорят, что в некоторых случаях, mmap() может подключить память по нулевому адресу, так что использовать
0 или NULL как код ошибки нельзя.


\subsection{Нулевые указатели}

\subsubsection{Ошибка ``Null pointer assignment'' во времена MS-DOS}

\myindex{MS-DOS}
Некоторые люди постарше могут помнить очень странную ошибку эпохи MS-DOS: ``Null pointer assignment''.
Что она означает?

В *NIX и Windows нельзя записывать в память по нулевому адресу, но это было возможно в MS-DOS, из-за отсутствия
защиты памяти как таковой.

\myindex{Turbo C++}
\myindex{Borland C++}
Так что я могу найти древний Turbo C++ 3.0 (позже он был переименован в C++) из начала 1990-х и попытаться
скомпилировать это:

\begin{lstlisting}[style=customc]
#include <stdio.h>

int main()
{
	int *ptr=NULL;
	*ptr=1234;
	printf ("Now let's read at NULL\n");
	printf ("%d\n", *ptr);
};
\end{lstlisting}

Трудно поверить, но это работает, но с ошибкой при выходе:

\begin{lstlisting}[caption=Древний Turbo C++ 3.0]
C:\TC30\BIN\1
Now let's read at NULL
1234
Null pointer assignment

C:\TC30\BIN>_
\end{lstlisting}

Посмотрим внутри исходного кода \ac{CRT} компилятора Borland C++ 3.1, файл \IT{c0.asm}:

\begin{lstlisting}[style=customasmx86]
;       _checknull()    check for null pointer zapping copyright message

...

;       Check for null pointers before exit

__checknull     PROC    DIST
                PUBLIC  __checknull

IF      LDATA  EQ  false
  IFNDEF  __TINY__
                push    si
                push    di
                mov     es, cs:DGROUP@@
                xor     ax, ax
                mov     si, ax
                mov     cx, lgth_CopyRight
ComputeChecksum label   near
                add     al, es:[si]
                adc     ah, 0
                inc     si
                loop    ComputeChecksum
                sub     ax, CheckSum
                jz      @@SumOK
                mov     cx, lgth_NullCheck
                mov     dx, offset DGROUP: NullCheck
                call    ErrorDisplay
@@SumOK:        pop     di
                pop     si
  ENDIF
ENDIF

_DATA           SEGMENT

;       Magic symbol used by the debug info to locate the data segment
                public DATASEG@
DATASEG@        label   byte

;       The CopyRight string must NOT be moved or changed without
;       changing the null pointer check logic

CopyRight       db      4 dup(0)
                db      'Borland C++ - Copyright 1991 Borland Intl.',0
lgth_CopyRight  equ     $ - CopyRight

IF      LDATA  EQ  false
IFNDEF  __TINY__
CheckSum        equ     00D5Ch
NullCheck       db      'Null pointer assignment', 13, 10
lgth_NullCheck  equ     $ - NullCheck
ENDIF
ENDIF

...

\end{lstlisting}

Модель памяти в MS-DOS крайне странная (\ref{8086_memory_model}), и, вероятно, её и не нужно изучать, если только вы не фанат ретрокомпьютинга
или ретрогейминга.
Одну только вещь можно держать в памяти, это то, что сегмент памяти (включая сегмент данных) в MS-DOS это место где
хранится код или данные, но в отличие от ``серьезных'' \ac{OS}, он начинается с нулевого адреса.

И в Borland C++ \ac{CRT}, сегмент данных начинается с 4-х нулевых байт и строки копирайта
``Borland C++ - Copyright 1991 Borland Intl.''.
Целостность 4-х нулевых байт и текстовой строки проверяется в конце, и если что-то нарушено, выводится сообщение об ошибке.

Но зачем? Запись по нулевому указателю это распространенная ошибка в \CCpp, и если вы делаете это в *NIX или Windows,
ваше приложение упадет.
В MS-DOS нет защиты памяти, так что это приходится проверять в \ac{CRT} во время выхода, пост-фактум.
Если вы видите это сообщение, значит ваша программа в каком-то месте что-то записала по нулевому адресу.

Наша программа это сделала. И вот почему число 1234 было прочитано корректно: потому что оно было записано на месте
первых 4-х байт.
Контрольная сумма во время выхода неверна (потому что наше число там осталось), так что сообщение было выведено.

Прав ли я?
Я переписал программу для проверки моих предположений:

\begin{lstlisting}[style=customc]
#include <stdio.h>

int main()
{
	int *ptr=NULL;
	*ptr=1234;
	printf ("Now let's read at NULL\n");
	printf ("%d\n", *ptr);
	*ptr=0; // psst, cover our tracks!
};
\end{lstlisting}

Программа исполняется без ошибки во время выхода.

Хотя и метод предупреждать о записи по нулевому указателю имел смысл в MS-DOS,
вероятно, это всё может использоваться и сегодня, на маломощных \ac{MCU} без защиты памяти и/или \ac{MMU}.

\subsubsection{Почему кому-то может понадобиться писать по нулевому адресу?}

Но почему трезвомыслящему программисту может понадобиться записывать что-то по нулевому адресу?
Это может быть сделано случайно, например, указатель должен быть инициализирован и указывать на только что выделенный
блок в памяти, а затем должен быть передан в какую-то ф-цию, возвращающую данные через указатель.

\begin{lstlisting}[style=customc]
int *ptr=NULL;

... мы забыли выделить память и инициализировать ptr

strcpy (ptr, buf); // strcpy() завершает работу молча, потому что в MS-DOS нет защиты памяти
\end{lstlisting}

И даже хуже:

\begin{lstlisting}[style=customc]
int *ptr=malloc(1000);

... мы забыли проверить, действительно ли память была выделена: это же MS-DOS и у тогдащних компьютеров было мало памяти,
... и нехватка памяти была обычной ситуацией.
... если malloc() вернул NULL, тогда ptr будет тоже NULL.

strcpy (ptr, buf); // strcpy() завершает работу молча, потому что в MS-DOS нет защиты памяти
\end{lstlisting}

\subsubsection{NULL в \CCpp}

NULL в C/C++ это просто макрос, который часто определяют так:

\begin{lstlisting}[style=customc]
#define NULL  ((void*)0)
\end{lstlisting}
( \href{https://github.com/wzhy90/linaro_toolchains/blob/8ff8ae680bac04558d10cc9626e12c4c2f6c1348/arm-cortex_a15-linux-gnueabihf/libc/usr/include/libio.h#L70}{libio.h file} )

\IT{void*} это тип данных, отражающий тот факт, что это указатель, но на значение неизвестного типа (\IT{void}).

NULL обычно используется чтобы показать отсутствие объекта.
Например, у вас есть односвязный список, и каждый узел имеет значение (или указатель на значение) и указатель вроде \IT{next}.
Чтобы показать, что следующего узла нет, в поле \IT{next} записывается 0.
Другие решения просто хуже.
Вероятно, вы можете использовать какую-то крайне экзотическую среду, где можно выделить память по нулевому адресу.
Как вы будете показывать отсутствие следующего узла?
Какой-нибудь \IT{magic number}? Может быть -1? Или дополнительным битом?

В Википедии мы можем найти это:

\begin{framed}
\begin{quotation}
In fact, quite contrary to the zero page's original preferential use, some modern operating systems such as FreeBSD, Linux and Microsoft Windows[2] actually make the zero page inaccessible to trap uses of NULL pointers. 
\end{quotation}
\end{framed}
( \url{https://en.wikipedia.org/wiki/Zero_page} )

\subsubsection{Нулевой указатель на ф-цию}

Можно вызывать ф-ции по их адресу.
Например, я компилирую это при помощи MSVC 2010 и запускаю в Windows 7:

\begin{lstlisting}[style=customc]
#include <windows.h>
#include <stdio.h>

int main()
{
	printf ("0x%x\n", &MessageBoxA);
};
\end{lstlisting}

Результат \IT{0x7578feae}, и он не меняется и после того, как я запустил это несколько раз, потому что
user32.dll (где находится ф-ция MessageBoxA) всегда загружается по одному и тому же адресу.
И потому что \ac{ASLR} не включено (тогда результат был бы всё время разным).

Вызовем ф-цию \IT{MessageBoxA()} по адресу:

\begin{lstlisting}[style=customc]
#include <windows.h>
#include <stdio.h>

typedef int (*msgboxtype)(HWND hWnd, LPCTSTR lpText, LPCTSTR lpCaption,  UINT uType);

int main()
{
	msgboxtype msgboxaddr=0x7578feae;

	// заставить загрузиться DLL в память процесса,
	// т.к., наш код не использует никакую ф-цию из user32.dll, 
	// и DLL не импортируется
	LoadLibrary ("user32.dll");

	msgboxaddr(NULL, "Hello, world!", "hello", MB_OK);
};
\end{lstlisting}

Странно выглядит, но работает в Windows 7 x86.

Это часто используется в шеллкода, потому что оттуда трудно вызывать ф-ции из DLL по их именам.
А \ac{ASLR} это контрмера.

И вот теперь что по-настоящему странно выглядит, некоторые программисты на Си для встраиваемых (embedded) систем, могут быть
знакомы с таким кодом:

\begin{lstlisting}[style=customc]
int reset()
{
	void (*foo)(void) = 0;
	foo();
};
\end{lstlisting}

Кому понадобится вызывать ф-цию по адресу 0?
Это портабельный способ перейти на нулевой адрес.
Множество маломощных микроконтроллеров не имеют защиты памяти или \ac{MMU}, и после сброса, они просто начинают
исполнять код по нулевому адресу, где может быть записан инициализирующий код.
Так что переход по нулевому адресу это способ сброса.
Можно использовать и inline-ассемблер, но если это неудобно, тогда можно использовать этот портабельный метод.

Это даже корректно компилируется при помощи GCC 4.8.4 на Linux x64:

\begin{lstlisting}[style=customasmx86]
reset:
        sub     rsp, 8
        xor     eax, eax
        call    rax
        add     rsp, 8
        ret
\end{lstlisting}

То обстоятельство, что указатель стека сдвинут, это не проблема: инициализирующий код в микроконтроллерах обычно
полностью игнорирует состояние регистров и памяти и загружает всё ``с чистого листа''.

И конечно, этот код упадет в *NIX или Windows, из-за защиты памяти, и даже если бы её не было, по нулевому адресу
нет никакого кода.

В GCC даже есть нестандартное расширение, позволяющее перейти по определенному адресу, вместо того чтобы вызывать ф-цию:
\url{http://gcc.gnu.org/onlinedocs/gcc/Labels-as-Values.html}.


\subsection{Массив как аргумент функции}

Кто-то может спросить, какая разница между объявлением аргумента ф-ции как массива и как указателя?

Как видно, разницы вообще нет:

\begin{lstlisting}[style=customc]
void write_something1(int a[16])
{
	a[5]=0;
};

void write_something2(int *a)
{
	a[5]=0;
};

int f()
{
	int a[16];
	write_something1(a);
	write_something2(a);
};
\end{lstlisting}

Оптимизирующий GCC 4.8.4:

\begin{lstlisting}[style=customasmx86]
write_something1:
        mov     DWORD PTR [rdi+20], 0
        ret

write_something2:
        mov     DWORD PTR [rdi+20], 0
        ret
\end{lstlisting}

Но вы можете объявлять массив вместо указателя для самодокументации, если размер массива известен зараннее и определен.
И может быть, какой-нибудь инструмент для статического анализа выявит возможное переполнение буфера.
Или такие инструменты есть уже сегодня?

Некоторые люди, включая Линуса Торвальдса, критикуют эту возможность \CCpp{}: \url{https://lkml.org/lkml/2015/9/3/428}.

В стандарте C99 имеется также ключевое слово \IT{static} \InSqBrackets{\CNineNineStd{} 6.7.5.3}:

\begin{framed}
\begin{quotation}
If the keyword static also appears  within the [ and ] of the array type derivation, then for each call to the function, the value of the corresponding actual argument shall provide access to the first element of an array with at least as many elements as specified by the size expression.
\end{quotation}
\end{framed}


\subsection{Указатель на функцию}

Имя ф-ции в \CCpp{} без скобок, как ``printf'' это указатель на ф-цию типа \IT{void (*)()}.
Попробуем прочитать содержимое ф-ции и пропатчить его:

\lstinputlisting[style=customc]{advanced/450_more_ptrs/6.c}

При запуске видно что первые 3 байта ф-ции это \TT{55 89 e5}.
Действительно, это опкоды инструкций \INS{PUSH EBP} и \INS{MOV EBP, ESP} (это опкоды x86).
Но потом процесс падает, потому что секция \IT{text} доступна только для чтения.

Мы можем перекомпилировать наш пример и сделать так, чтобы секция \IT{text} была доступна для записи
\footnote{\url{http://stackoverflow.com/questions/27581279/make-text-segment-writable-elf}}:

\begin{lstlisting}
gcc --static -g -Wl,--omagic -o example example.c
\end{lstlisting}

Это работает!

\begin{lstlisting}
we are in print_something()
first 3 bytes: 55 89 e5...
going to call patched print_something():
it must exit at this point
\end{lstlisting}



\subsection{Указатель как идентификатор объекта}

В ассемблере и Си нет возможностей \ac{OOP}, но там вполне можно писать код в стиле \ac{OOP}
(просто относитесь к структуре, как к объекту).

Интересно что, иногда, указатель на объект (или его адрес) называется идентификатором (в смысле сокрытия данных/инкапсуляции).

\myindex{win32!LoadLibrary()}
\myindex{win32!GetProcAddress()}
Например, LoadLibrary(), судя по \ac{MSDN}, возвращает ``handle'' модуля
\footnote{\url{https://msdn.microsoft.com/ru-ru/library/windows/desktop/ms684175(v=vs.85).aspx}}.
Затем вы передаете этот ``handle'' в другую ф-цию вроде GetProcAddress().
Но на самом деле, LoadLibrary() возвращает указатель на DLL-файл загруженный (\IT{mapped}) в памяти
\footnote{\url{https://blogs.msdn.microsoft.com/oldnewthing/20041025-00/?p=37483}}.
Вы можете прочитать два байта по адресу возвращенному LoadLibrary(), и это будет ``MZ'' (первые два байта любого файла
типа .EXE/.DLL в Windows).

\myindex{win32!HMODULE}
\myindex{win32!HINSTANCE}
Очевидно, Microsoft ``скрывает'' этот факт для обеспечения лучшей совместимости в будущем.
Также, типы данных HMODULE и HINSTANCE имели другой смысл в 16-битной Windows.

Возможно, это причина, почему \printf имеет модификатор ``\%p'', который используется для вывода указателей (32-битные
целочисленные на 32-битных архитектурах, 64-битные на 64-битных, итд) в шестнадцатеричной форме.
Адрес структуры сохраненный в отладочном протоколе может помочь в поисках такого же в том же протоколе.

\myindex{SQLite}
Вот например из исходного кода SQLite:

\begin{lstlisting}

...

struct Pager {
  sqlite3_vfs *pVfs;          /* OS functions to use for IO */
  u8 exclusiveMode;           /* Boolean. True if locking_mode==EXCLUSIVE */
  u8 journalMode;             /* One of the PAGER_JOURNALMODE_* values */
  u8 useJournal;              /* Use a rollback journal on this file */
  u8 noSync;                  /* Do not sync the journal if true */

....

static int pagerLockDb(Pager *pPager, int eLock){
  int rc = SQLITE_OK;

  assert( eLock==SHARED_LOCK || eLock==RESERVED_LOCK || eLock==EXCLUSIVE_LOCK );
  if( pPager->eLock<eLock || pPager->eLock==UNKNOWN_LOCK ){
    rc = sqlite3OsLock(pPager->fd, eLock);
    if( rc==SQLITE_OK && (pPager->eLock!=UNKNOWN_LOCK||eLock==EXCLUSIVE_LOCK) ){
      pPager->eLock = (u8)eLock;
      IOTRACE(("LOCK %p %d\n", pPager, eLock))
    }
  }
  return rc;
}

...

  PAGER_INCR(sqlite3_pager_readdb_count);
  PAGER_INCR(pPager->nRead);
  IOTRACE(("PGIN %p %d\n", pPager, pgno));
  PAGERTRACE(("FETCH %d page %d hash(%08x)\n",
               PAGERID(pPager), pgno, pager_pagehash(pPg)));

...

\end{lstlisting}

