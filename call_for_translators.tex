\vspace*{\fill}

\Huge%
	\RU{Нужны переводчики!}%
	\EN{Call for translators!}%
	\ESph{}%
	\PTBRph{}%
	\DEph{}%
	\PLph{}%
	\ITAph{}
\normalsize

\bigskip
\bigskip
\bigskip

\EN{You may want to help me with translation this work into languages other than English and Russian.}%
\RU{Возможно, вы захотите мне помочь с переводом этой работы на другие языки, кроме английского и русского.}

\EN{For those who are not afraid of TeX: \href{https://github.com/dennis714/RE-for-beginners/blob/master/Translation.md}{read here}.}%
\RU{Для тех, кто не боится TeX: \href{https://github.com/dennis714/RE-for-beginners/blob/master/Translation.md}{читайте здесь}.}
\EN{For those who afraid, you may just open PDF file in OpenOffice and gradually rewrite each sentence.}%
\RU{Для тех, кто боится, вы можете просто открыть PDF-файл в OpenOffice и постепенно переписывать каждое предложение.}
\EN{I'll copypaste your work back to my LaTeX source code.}%
\RU{Я затем скопирую вашу работу назад в исходный код на LaTeX.}

\EN{It's a tedious and boring work, so you probably may want to start with shortened \href{http://beginners.re/\#lite}{LITE version}.}%
\RU{Это рутинная и скучная работа, так что вы можете начать с сокращенной \href{http://beginners.re/\#lite}{LITE-версии}.}
\EN{There is even a better way: to my own experience, you can gain your motivation by translating short pieces of my book and posting them to your blog(s).}%
\RU{Есть даже еще лучше способ: по моему опыту, мотивировать себя можно переводя короткие части текста из моей книги и выкладывая их в своем блоге.}
\EN{I can publish URLs to these your posts here and also in my twitter (\href{http://twitter.com/yurichev}{@yurichev}).}%
\RU{Я могу публиковать URL-ы на ваши посты здесь, а также в моем twitter (\href{http://twitter.com/yurichev}{@yurichev}).}

\EN{Speed isn't important, because this is open-source project, after all.}%
\RU{Скорость не важна, потому что это опен-сорсный проект все-таки.}
\EN{Your name will be mentioned as project contributor.}%
\RU{Ваше имя будет указано в числе участников проекта.}

\EN{Korean, Chinese and Persian languages are reserved by publishers.}%
\RU{Корейский, китайский и персидский языки зарезервированы издателями.}
\EN{As of March 2016, there are Brazilian Portuguese and German language teams working, drop me email, so I will connect you to them.}%
\RU{По состоянию на март 2016, есть две команды (бразильский португальский и немецкий), напишите мне, и я соеденю вас с ними.}
\EN{All other attempts to translate pieces of these texts to other languages are highly welcomed!}%
\RU{Все остальные попытки перевести части этих текстов на другие языки очень приветствуются!}

\EN{English and Russian versions I do by myself, but my English is still that horrible, so I'm very grateful for any notes about grammar, etc.}%
\RU{Английскую и русскую версии я делаю сам, но английский у меня все еще ужасный, так что я буду очень признателен за коррективы, итд.}
\EN{Even my Russian is also flawed, so I'm grateful for notes about Russian text as well!}%
\RU{Даже мой русский несовершенный, так что я благодарен за коррективы и русского текста!}

\EN{So do not hesitate to contact me: \TT{\EMAIL}.}%
\RU{Не стесняйтесь писать мне: \TT{\EMAIL}.}

\vspace*{\fill}
\vfill
