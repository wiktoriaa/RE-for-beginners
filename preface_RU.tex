\section*{Предисловие}

У термина \q{\gls{reverse engineering}} несколько популярных значений:
1) исследование скомпилированных
программ; 2) сканирование трехмерной модели для последующего копирования;
3) восстановление структуры СУБД. Настоящий сборник заметок
связан с первым значением.

\subsection*{Желательные знания перед началом чтения}

Очень желательно базовое знание \ac{PL} Си.
Рекомендуемые материалы: \myref{CCppBooks}.

\subsection*{Упражнения и задачи}

\dots 
все перемещены на отдельный сайт: \url{http://challenges.re}.

\subsection*{Об авторе}
\begin{tabularx}{\textwidth}{ l X }

\raisebox{-\totalheight}{
\includegraphics[scale=0.60]{Dennis_Yurichev.jpg}
}

&
Денис Юричев~--- опытный reverse engineer и программист.
С ним можно контактировать по емейлу: \textbf{\EMAIL{}} или по Skype: \textbf{dennis.yurichev}.

% FIXME: no link. \tablefootnote doesn't work
\end{tabularx}

% subsections:
\subsection*{%
	\RU{Отзывы о книге}%
	\EN{Praise for}%
	\ES{Elogios para}%
	\PTBRph{}%
	\DEph{}\PLph{}%
	\ITAph{}
	\IT{\TITLE}%
}

\begin{itemize}
% expanded URLs to make it more robust for printouts. In electronic editions people will click anyway, so tracking will keep working
\item \q{It's very well done .. and for free .. amazing.}\footnote{\href{http://go.yurichev.com/17095}{twitter.com/daniel\_bilar/status/436578617221742593}} Daniel Bilar, Siege Technologies, LLC.

\item \q{... excellent and free}\footnote{\href{http://go.yurichev.com/17096}{twitter.com/petefinnigan/status/400551705797869568}} Pete Finnigan,%
	\RU{гуру по безопасности}%
	\ES{gur\'u de seguridad en}%
	\PTBRph{}%
	\DEph{}\PLph{}%
	\ITAph{}
\oracle
	\EN{security guru}.

\item \q{... book is interesting, great job!} Michael Sikorski,
	\RU{автор книги}%
	\EN{author of}%
	\ES{autor de}%
	\PTBRph{}%
	\DEph{}\PLph{}%
	\ITAph{}
\IT{Practical Malware Analysis: The Hands-On Guide to Dissecting Malicious Software}.

\item \q{... my compliments for the very nice tutorial!} Herbert Bos,
	\RU{профессор университета}%
	\EN{full professor at the}%
	\ES{catedr\'atico de tiempo completo en la}%
	\PTBRph{}%
	\DEph{}\PLph{}%
	\ITAph{}
Vrije Universiteit Amsterdam,
	\RU{соавтор}%
	\EN{co-author of}%
	\ES{coautor de}%
	\PTBRph{}%
	\DEph{}\PLph{}%
	\ITAph{}
\IT{Modern Operating Systems (4th Edition)}.

\item \q{... It is amazing and unbelievable.} Luis Rocha, CISSP / ISSAP, Technical Manager, Network \& Information Security at Verizon Business.

\item \q{Thanks for the great work and your book.} Joris van de Vis,
	\RU{специалист по}%
	\ES{especialista en}%
	\PTBRph{}%
	\DEph{}\PLph{}%
	\ITAph{}
SAP Netweaver \& Security
	\EN{specialist}.

\item \q{... reasonable intro to some of the techniques.}\footnote{\href{http://go.yurichev.com/17099}{reddit}} Mike Stay,
	\RU{преподаватель в}%
	\EN{teacher at the}%
	\ES{profesor en el}%
	\PTBRph{}%
	\DEph{}\PLph{}%
	\ITAph{}
Federal Law Enforcement Training Center, Georgia, US.

\item \q{I love this book! I have several students reading it at the moment, plan to use it in graduate course.}\footnote{\href{http://go.yurichev.com/17097}{twitter.com/sergeybratus/status/505590326560833536}}
	\RU{Сергей Братусь}%
	\EN{Sergey Bratus}%
	\ES{Sergey Bratus}%
	\PTBRph{}%
	\DEph{}\PLph{}%
	\ITAph{},
Research Assistant Professor
	\RU{в отделе Computer Science в}%
	\EN{at the Computer Science Department at}%
	\ES{en el Departamento de Ciencias de la Computaci\'on en}%
	\PTBRph{}%
	\DEph{}\PLph{}%
	\ITAph{}
Dartmouth College

\item \q{Dennis @Yurichev has published an impressive (and free!) book on reverse engineering}\footnote{\href{http://go.yurichev.com/17098}{twitter.com/TanelPoder/status/524668104065159169}} Tanel Poder,
	\RU{эксперт по настройке производительности Oracle RDBMS}%
	\EN{Oracle RDBMS performance tuning expert}%
	\ES{experto en afinaci\'on de rendimiento de Oracle RDBMS}%
	\PTBRph{}%
	\DEph{}\PLph{}
	\ITAph{}.

\item \q{This book is some kind of Wikipedia to beginners...} Archer, Chinese Translator, IT Security Researcher.

\RU{\item \q{Прочел Вашу книгу~--- отличная работа, рекомендую на своих курсах студентам
в качестве учебного пособия}. Николай Ильин, преподаватель в ФТИ НТУУ \q{КПИ} и DefCon-UA}
\end{itemize}

\ifdefined\RUSSIAN
\newcommand{\PeopleMistakesInaccuracies}{Станислав \q{Beaver} Бобрицкий, Александр Лысенко, Shell Rocket, Zhu Ruijin, Changmin Heo, Александр \q{Solar Designer} Песляк, Vitor Vidal, Марк Уилсон.}
\else
\newcommand{\PeopleMistakesInaccuracies}{Stanislav \q{Beaver} Bobrytskyy, Alexander Lysenko, Shell Rocket, Zhu Ruijin, Changmin Heo, Alexander \q{Solar Designer} Peslyak, Vitor Vidal, Mark Wilson.}
\fi

\EN{\subsection*{Thanks}

For patiently answering all my questions: \HERMIT, Slava \q{Avid} Kazakov.

For sending me notes about mistakes and inaccuracies: \PeopleMistakesInaccuracies{}.

For helping me in other ways:
Andrew Zubinski,
Arnaud Patard (rtp on \#debian-arm IRC),
noshadow on \#gcc IRC,
Aliaksandr Autayeu,
Mohsen Mostafa Jokar.

For translating the book into Simplified Chinese:
Antiy Labs (\href{http://antiy.cn}{antiy.cn}), Archer.

For translating the book into Korean: Byungho Min.

For translating the book into Dutch: Cedric Sambre (AKA Midas).

For translating the book into Spanish: \PeopleSpanishTranslators{}.

For translating the book into Portuguese: Thales Stevan de A. Gois.

For translating the book into Italian: \PeopleItalianTranslators{}.

For translating the book into French: \PeopleFrenchTranslators{}.

For translating the book into German: \PeopleGermanTranslators{}.

For proofreading:
Alexander \q{Lstar} Chernenkiy,
Vladimir Botov,
Andrei Brazhuk,
Mark ``Logxen'' Cooper, Yuan Jochen Kang, Mal Malakov, Lewis Porter, Jarle Thorsen, Hong Xie.

Vasil Kolev\footnote{\url{https://vasil.ludost.net/}} did a great amount of work in proofreading and correcting many mistakes.

For illustrations and cover art: Andy Nechaevsky.

Thanks also to all the folks on github.com who have contributed notes and corrections\FNGithubContributors{}.

Many \LaTeX\ packages were used: I would like to thank the authors as well.

\subsubsection*{Donors}

Those who supported me during the time when I wrote significant part of the book:

\subsubsection*{\RU{Жертвователи}\EN{Donors}}

10 * \RU{аноним}\EN{anonymous}, 2 * \RU{Олег Выговский}\EN{Oleg Vygovsky}, Daniel Bilar, James Truscott,
Luis Rocha, Joris van de Vis, Richard S Shultz, Jang Minchang, Shade Atlas, Yao Xiao,
Pawel Szczur, Justin Simms, Shawn the R0ck, Ki Chan Ahn, Triop AB, Ange Albertini,
\RU{Сергей Лукьянов}\EN{Sergey Lukianov}, Ludvig Gislason, Gérard Labadie, Sergey Volchkov.


Thanks a lot to every donor!
}
\ES{\subsection*{Agradecimientos}

Por contestar pacientemente a todas mis preguntas: \HERMIT, Slava \q{Avid} Kazakov.

Por enviarme notas acerca de errores e inexactitudes: \PeopleMistakesInaccuracies{}.

Por ayudarme de otras formas:
Andrew Zubinski,
Arnaud Patard (rtp en \#debian-arm IRC),
noshadow en \#gcc IRC,
Aliaksandr Autayeu,
Mohsen Mostafa Jokar.

Por traducir el libro a Chino Simplificado:
Antiy Labs (\href{http://antiy.cn}{antiy.cn}), Archer.

Por traducir el libro a Coreano: Byungho Min.

\ESph{}: Cedric Sambre (AKA Midas).

\ESph{}: \PeopleSpanishTranslators{}.

\ESph{}: Thales Stevan de A. Gois.

\ESph{}: \PeopleItalianTranslators{}.

\ESph{}: \PeopleFrenchTranslators{}.

\DEph{}: \PeopleGermanTranslators{}.

\ES{Por correcci\'on de pruebas}%
Alexander \q{Lstar} Chernenkiy,
Vladimir Botov,
Andrei Brazhuk,
Mark ``Logxen'' Cooper, Yuan Jochen Kang, Mal Malakov, Lewis Porter, Jarle Thorsen, Hong Xie.

Vasil Kolev\footnote{\url{https://vasil.ludost.net/}} realiz\'o una gran cantidad de trabajo en correcci\'on de pruebas y correcci\'on de muchos errores.

Por las ilustraciones y el arte de la portada: Andy Nechaevsky.

Gracias a toda la gente en github.com que ha contribuido con notas y correcciones\FNGithubContributors{}.

Muchos paquetes de \LaTeX\ fueron utiliados: quiero agradecer tambi\'en a sus autores.

\subsubsection*{Donadores}

Aquellos que me apoyaron durante el tiempo que escrib\'i una parte significativa del libro:

\subsubsection*{\RU{Жертвователи}\EN{Donors}}

10 * \RU{аноним}\EN{anonymous}, 2 * \RU{Олег Выговский}\EN{Oleg Vygovsky}, Daniel Bilar, James Truscott,
Luis Rocha, Joris van de Vis, Richard S Shultz, Jang Minchang, Shade Atlas, Yao Xiao,
Pawel Szczur, Justin Simms, Shawn the R0ck, Ki Chan Ahn, Triop AB, Ange Albertini,
\RU{Сергей Лукьянов}\EN{Sergey Lukianov}, Ludvig Gislason, Gérard Labadie, Sergey Volchkov.


!`Gracias a cada donante!

}
\NL{\subsection*{Dankwoord}

Voor al mijn vragen geduldig te beantwoorden: \HERMIT, Slava \q{Avid} Kazakov.

Om me nota\'s over fouten en onnauwkeurigheden toe te sturen: \PeopleMistakesInaccuracies{}.

Om me te helpen op andere manieren:
Andrew Zubinski,
Arnaud Patard (rtp op \#debian-arm IRC),
noshadow op \#gcc IRC,
Aliaksandr Autayeu, Mohsen Mostafa Jokar.

Om het boek te vertalen naar het Vereenvoudigd Chinees:
Antiy Labs (\href{http://antiy.cn}{antiy.cn}), Archer.

Om dit boek te vertalen in het Koreaans: Byungho Min.

\NLph{}: Cedric Sambre (AKA Midas).

\NLph{}: \PeopleSpanishTranslators{}.

\NLph{}: Thales Stevan de A. Gois.

\NLph{}: \PeopleItalianTranslators{}.

\NLph{}: \PeopleFrenchTranslators{}.

\NLph{}: \PeopleGermanTranslators{}.

Voor proofreading:
Alexander \q{Lstar} Chernenkiy,
Vladimir Botov,
Andrei Brazhuk,
Mark ``Logxen'' Cooper, Yuan Jochen Kang, Mal Malakov, Lewis Porter, Jarle Thorsen, Hong Xie.

Vasil Kolev\footnote{\url{https://vasil.ludost.net/}}, voor het vele werk in proofreading en het verbeteren van vele fouten.

Voor de illustraties en cover art: Andy Nechaevsky.

Dank aan al de mensen op github.com die hebben nota\'s en correcties hebben bijgedragen\FNGithubContributors{}.

Veel \LaTeX\ packages zijn gebruikt. Ik zou de auteurs hiervan ook graag bedanken.

\subsubsection*{Donaties}

Zij die me gesteund hebben tijdens het schrijven van een groot deel van dit boek:

\subsubsection*{\RU{Жертвователи}\EN{Donors}}

10 * \RU{аноним}\EN{anonymous}, 2 * \RU{Олег Выговский}\EN{Oleg Vygovsky}, Daniel Bilar, James Truscott,
Luis Rocha, Joris van de Vis, Richard S Shultz, Jang Minchang, Shade Atlas, Yao Xiao,
Pawel Szczur, Justin Simms, Shawn the R0ck, Ki Chan Ahn, Triop AB, Ange Albertini,
\RU{Сергей Лукьянов}\EN{Sergey Lukianov}, Ludvig Gislason, Gérard Labadie, Sergey Volchkov.


Veel dank aan elke donor!
}
\RU{\subsection*{Благодарности}

Тем, кто много помогал мне отвечая на массу вопросов: \HERMIT, Слава \q{Avid} Казаков.

Тем, кто присылал замечания об ошибках и неточностях: \PeopleMistakesInaccuracies{}.

Просто помогали разными способами:
Андрей Зубинский,
Arnaud Patard (rtp на \#debian-arm IRC),
noshadow на \#gcc IRC,
Александр Автаев,
Mohsen Mostafa Jokar.

Переводчикам на китайский язык:
Antiy Labs (\href{http://antiy.cn}{antiy.cn}), Archer.

Переводчику на корейский язык: Byungho Min.

Переводчику на голландский язык: Cedric Sambre (AKA Midas).

Переводчикам на испанский язык: \PeopleSpanishTranslators{}.

Переводчикам на португальский язык: Thales Stevan de A. Gois.

Переводчикам на итальянский язык: \PeopleItalianTranslators{}.

Переводчикам на французский язык: \PeopleFrenchTranslators{}.

Переводчикам на немецкий язык: \PeopleGermanTranslators{}.

Корректорам:
Александр \q{Lstar} Черненький,
Владимир Ботов,
Андрей Бражук,
Марк ``Logxen'' Купер, Yuan Jochen Kang, Mal Malakov, Lewis Porter, Jarle Thorsen, Hong Xie.

Васил Колев\footnote{\url{https://vasil.ludost.net/}} сделал очень много исправлений и указал на многие ошибки.

За иллюстрации и обложку: Андрей Нечаевский.

И ещё всем тем на github.com кто присылал замечания и исправления\FNGithubContributors{}.

Было использовано множество пакетов \LaTeX. Их авторов я также хотел бы поблагодарить.

\subsubsection*{Жертвователи}

Те, кто поддерживал меня во время написании этой книги:

\subsubsection*{\RU{Жертвователи}\EN{Donors}}

10 * \RU{аноним}\EN{anonymous}, 2 * \RU{Олег Выговский}\EN{Oleg Vygovsky}, Daniel Bilar, James Truscott,
Luis Rocha, Joris van de Vis, Richard S Shultz, Jang Minchang, Shade Atlas, Yao Xiao,
Pawel Szczur, Justin Simms, Shawn the R0ck, Ki Chan Ahn, Triop AB, Ange Albertini,
\RU{Сергей Лукьянов}\EN{Sergey Lukianov}, Ludvig Gislason, Gérard Labadie, Sergey Volchkov.


Огромное спасибо каждому!

}


\subsection*{mini-ЧаВО}

\par Q: Что необходимо знать перед чтением книги?
\par A: Желательно иметь базовое понимание Си/Си++.

\par Q: Возможно ли купить русскую/английскую бумажную книгу?
\par A: К сожалению нет, пока ни один издатель не заинтересовался в издании русской или английской версии.
А пока вы можете распечатать/переплести её в вашем любимом копи-шопе или копи-центре.

\par Q: Существует ли версия epub/mobi?
\par A: Книга очень сильно завязана на специфические для TeX/LaTeX хаки, поэтому преобразование в HTML (epub/mobi это набор HTML)
легким не будет.

\par Q: Зачем в наше время нужно изучать язык ассемблера?
\par A: Если вы не разработчик \ac{OS}, вам наверное не нужно писать на ассемблере: современные компиляторы (2010-ые) оптимизируют код намного лучше человека
\footnote{Очень хороший текст на эту тему: \InSqBrackets{\AgnerFog}}.

К тому же, современные \ac{CPU} это крайне сложные устройства и знание ассемблера вряд ли
поможет узнать их внутренности.

Но все-таки остается по крайней мере две области, где знание ассемблера может хорошо помочь:
1) исследование malware (\IT{зловредов}) с целью анализа; 2) лучшее понимание
вашего скомпилированного кода в процессе отладки.
Таким образом, эта книга предназначена для тех, кто хочет скорее понимать ассемблер,
нежели писать на нем, и вот почему здесь масса примеров, связанных с результатами
работы компиляторов.

\par Q: Я кликнул на ссылку внутри PDF-документа, как теперь вернуться назад?
\par A: В Adobe Acrobat Reader нажмите сочетание Alt+LeftArrow. В Evince кликните на ``<''.

\par Q: Могу ли я распечатать эту книгу? Использовать её для обучения?
\par A: Конечно, поэтому книга и лицензирована под лицензией Creative Commons (CC BY-SA 4.0).

\par Q: Почему эта книга бесплатная? Вы проделали большую работу. Это подозрительно, как и многие другие бесплатные вещи.
\par A: По моему опыту, авторы технической литературы делают это, в основном ради само-рекламы. Такой работой заработать приличные деньги невозможно.

\par Q: Как можно найти работу reverse engineer-а?
\par A: На reddit, посвященному RE\FNURLREDDIT, время от времени бывают hiring thread (\RedditHiringThread{}).
Посмотрите там.

В смежном субреддите \q{netsec} имеется похожий тред: \NetsecHiringThread{}.

\par Q: Куда пойти учиться в Украине?
\par A: \href{http://go.yurichev.com/17336}{НТУУ \q{КПИ}: \q{Аналіз програмного коду та бінарних вразливостей}};
\href{http://go.yurichev.com/17337}{факультативы}.

\par Q: У меня есть вопрос...
\par A: Напишите мне его емейлом (\EMAIL).


\subsection*{О переводе на корейский язык}

В январе 2015, издательство Acorn в Южной Корее сделало много работы в переводе 
и издании моей книги (по состоянию на август 2014) на корейский язык.
Она теперь доступна на \href{http://go.yurichev.com/17343}{их сайте}.

\iffalse
\begin{figure}[H]
\centering
\includegraphics[scale=0.3]{acorn_cover.jpg}
\end{figure}
\fi

Переводил Byungho Min (\href{http://go.yurichev.com/17344}{twitter/tais9}).
Обложку нарисовал мой хороший знакомый художник Андрей Нечаевский
\href{http://go.yurichev.com/17023}{facebook/andydinka}.
Они также имеют права на издание книги на корейском языке.
Так что если вы хотите иметь \IT{настоящую} книгу на полке на корейском языке и
хотите поддержать мою работу, вы можете купить её.

\subsection*{О переводе на персидский язык (фарси)}

В 2016 году книга была переведена Mohsen Mostafa Jokar (который также известен иранскому сообществу по переводу руководства Radare\footnote{\url{http://rada.re/get/radare2book-persian.pdf}}).
Книга доступна на сайте издательства\footnote{\url{http://goo.gl/2Tzx0H}} (Pendare Pars).

Первые 40 страниц: \url{https://beginners.re/farsi.pdf}.

Регистрация книги в Национальной Библиотеке Ирана: \url{http://opac.nlai.ir/opac-prod/bibliographic/4473995}.

\subsection*{О переводе на китайский язык}

В апреле 2017, перевод на китайский был закончен китайским издательством PTPress. Они также имеют права на издание книги на китайском языке.

Она доступна для заказа здесь: \url{http://www.epubit.com.cn/book/details/4174}. Что-то вроде рецензии и история о переводе: \url{http://www.cptoday.cn/news/detail/3155}.

Основным переводчиком был Archer, перед которым я теперь в долгу.
Он был крайне дотошным (в хорошем смысле) и сообщил о большинстве известных ошибок и баг, что крайне важно для литературы вроде этой книги.
Я буду рекомендовать его услуги всем остальным авторам!

Ребята из \href{http://www.antiy.net/}{Antiy Labs} также помогли с переводом. \href{http://www.epubit.com.cn/book/onlinechapter/51413}{Здесь предисловие} написанное ими.

