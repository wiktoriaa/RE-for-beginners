\section{Циклы}

Когда ваша программа работает с некоторым файлом, или буфером некоторой длины,
внутри кода где-то должен быть цикл с дешифровкой/обработкой.

Вот реальный пример выхода инструмента \tracer.
Был код, загружающий некоторый зашифрованный файл размером 258 байт.
Я могу запустить его с целью подсчета исполнения каждой инструкции (в наше время \ac{DBI} послужила бы куда лучше).
И я могу быстро найти место в коде, которое было исполнено 259/258 раз:

\lstinputlisting{digging_into_code/crypto_loop.txt}

Как потом оказалось, это цикл дешифрования.

