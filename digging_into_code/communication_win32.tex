\chapter{\RU{Связь с внешним миром}\EN{Communication with the outer world} (win32)}

\RU{Обращения к файлам и реестру}\EN{Files and registry access}: 
\RU{для самого простого анализа может помочь утилита}\EN{for the very basic
analysis, } Process Monitor\footnote{\url{http://technet.microsoft.com/en-us/sysinternals/bb896645.aspx}}
\RU{от}\EN{utility from} SysInternals\EN{ may help}. 

\RU{Для анализа обращения программы к сети, может помочь}\EN{For the basic analysis of
network accesses,} Wireshark\footnote{\url{http://www.wireshark.org/}}\EN{ may help}.

\RU{Затем всё-таки придётся смотреть внутрь}\EN{But then you will need to look inside anyway}. \\
\\
\RU{Первое на что нужно обратить внимание, это какие функции из \ac{API} \ac{OS} 
и какие функции стандартных библиотек используются.}
\EN{First what to look on is which functions from \ac{OS} \ac{API} and standard libraries are used.}

\RU{Если программа поделена на главный исполняемый файл и группу DLL-файлов, 
то имена функций в этих DLL, бывает так, могут помочь.}
\EN{If the program is divided into main executable file and a group of DLL-files, sometimes,
these function's names may be helpful.}

\RU{Если нас интересует, что именно приводит к вызову \TT{MessageBox()} с определенным текстом, 
то первое что можно попробовать сделать: найти в сегменте данных этот текст, найти ссылки на него, и найти, 
откуда может передаться управление к интересующему нас вызову \TT{MessageBox()}.}
\EN{If we are interesting, what exactly may lead to the \TT{MessageBox()} call with specific text,
first what we can try to do: find this text in data segment, find references to it and find the points
from which a control may be passed to the \TT{MessageBox()} call we're interesting in.}

\index{\CStandardLibrary!rand()}
\RU{Если речь идет о компьютерной игре, и нам интересно какие события в ней более-менее случайны, 
мы можем найти функцию \rand или её заменитель (как алгоритм Mersenne twister), и посмотреть, 
из каких мест эта функция вызывается и что самое главное: как используется результат этой функции.}
\EN{If we are talking about a video game and we're interesting, which events are more or less random in it,
we may try to find \rand function or its replacement (like Mersenne twister algorithm) and find a places
from which this function called and most important: how the results are used.}

\RU{Но если это не игра, а \rand используется, то также весьма любопытно, зачем. 
Бывают неожиданные случаи вроде использования \rand в алгоритме для сжатия данных (для имитации шифрования):}
\EN{But if it is not a game, but \rand is used, it is also interesting, why.
There are cases of unexpected \rand usage in data compression algorithm (for encryption imitation):}
\url{http://blog.yurichev.com/node/44}.

\section{\RU{Часто используемые ф-ции}\EN{Often used functions in} Windows API}

\RU{Это ф-ции которые можно увидеть в числе импортируемых}\EN{These functions may be among imported}.
\RU{Но также нельзя забывать, что далеко не все они были использованы в коде написанном автором}
\EN{It is worth to note that not every function might be used by the code written by author}.
\RU{Немалая часть может вызываться из библиотечных ф-ций и}\EN{A lot of functions might
be called from library functions and} \ac{CRT}\RU{-кода}\EN{ code}.

\begin{itemize}

\item
\RU{Работа с реестром}\EN{Registry access} (advapi32.dll): 
RegEnumKeyEx\footnote{\url{http://msdn.microsoft.com/en-us/library/windows/desktop/ms724862(v=vs.85).aspx}}
\footnote{
	\RU{Может иметь суффикс -A для ASCII-версии и -W для Unicode-версии}
	\EN{May have -A suffix for ASCII-version and -W for Unicode-version}
	\label{note1}},
RegEnumValue\footnote{\url{http://msdn.microsoft.com/en-us/library/windows/desktop/ms724865(v=vs.85).aspx}}
\footnoteref{note1},
RegGetValue\footnote{\url{http://msdn.microsoft.com/en-us/library/windows/desktop/ms724868(v=vs.85).aspx}}
\footnoteref{note1},
RegOpenKeyEx\footnote{\url{http://msdn.microsoft.com/en-us/library/windows/desktop/ms724897(v=vs.85).aspx}}
\footnoteref{note1},
RegQueryValueEx\footnote{\url{http://msdn.microsoft.com/en-us/library/windows/desktop/ms724911(v=vs.85).aspx}}
\footnoteref{note1}.

\item
\RU{Работа с текстовыми .ini-файлами}\EN{Access to text .ini-files} (kernel32.dll): 
GetPrivateProfileString
\footnote{\url{http://msdn.microsoft.com/en-us/library/windows/desktop/ms724353(v=vs.85).aspx}}
\footnoteref{note1}.

\item
\RU{Диалоговые окна}\EN{Dialog boxes} (user32.dll): 
MessageBox
\footnote{\url{http://msdn.microsoft.com/en-us/library/ms645505(VS.85).aspx}}
\footnoteref{note1}, 
MessageBoxEx
\footnote{\url{http://msdn.microsoft.com/en-us/library/ms645507(v=vs.85).aspx}}
\footnoteref{note1},
SetDlgItemText
\footnote{\url{http://msdn.microsoft.com/en-us/library/ms645521(v=vs.85).aspx}}
\footnoteref{note1},
GetDlgItemText
\footnote{\url{http://msdn.microsoft.com/en-us/library/ms645489(v=vs.85).aspx}}
\footnoteref{note1}.

\item
\RU{Работа с ресурсами}\EN{Resources access}(\ref{PEresources}): (user32.dll): LoadMenu
\footnote{\url{http://msdn.microsoft.com/en-us/library/ms647990(v=vs.85).aspx}}
\footnoteref{note1}.
\item
\RU{Работа с TCP/IP-сетью}\EN{TCP/IP-network} (ws2\_32.dll):
WSARecv
\footnote{\url{http://msdn.microsoft.com/en-us/library/windows/desktop/ms741688(v=vs.85).aspx}},
WSASend
\footnote{\url{http://msdn.microsoft.com/en-us/library/windows/desktop/ms742203(v=vs.85).aspx}}.

\item
\RU{Работа с файлами}\EN{File access} (kernel32.dll):
CreateFile
\footnote{\url{http://msdn.microsoft.com/en-us/library/windows/desktop/aa363858(v=vs.85).aspx}}
\footnoteref{note1},
ReadFile
\footnote{\url{http://msdn.microsoft.com/en-us/library/windows/desktop/aa365467(v=vs.85).aspx}},
ReadFileEx
\footnote{\url{http://msdn.microsoft.com/en-us/library/windows/desktop/aa365468(v=vs.85).aspx}},
WriteFile
\footnote{\url{http://msdn.microsoft.com/en-us/library/windows/desktop/aa365747(v=vs.85).aspx}},
WriteFileEx
\footnote{\url{http://msdn.microsoft.com/en-us/library/windows/desktop/aa365748(v=vs.85).aspx}}.

\item
\RU{Высокоуровневая работа с}\EN{High-level access to the} Internet
(wininet.dll):
WinHttpOpen
\footnote{\url{http://msdn.microsoft.com/en-us/library/windows/desktop/aa384098(v=vs.85).aspx}}.

\item
\RU{Проверка цифровой подписи исполняемого файла}
\EN{Check digital signature of a executable file} (wintrust.dll):
WinVerifyTrust
\footnote{\url{http://msdn.microsoft.com/library/windows/desktop/aa388208.aspx}}.

\item
\RU{Стандартная библиотека MSVC (при случае динамического связывания)}
\EN{Standard MSVC library (in case of dynamic linking)} (msvcr*.dll):
assert, itoa, ltoa, open, printf, read, strcmp, atol, atoi, fopen, fread, fwrite, memcmp, rand,
strlen, strstr, strchr.

\end{itemize}

\section{tracer: \RU{Перехват всех ф-ций в отдельном модуле}
\EN{Intercepting all functions in specific module}}
\index{tracer}

\RU{В \tracer есть INT3-брякпойнты, хотя и срабатывающие только один раз, но зато их можно установить на все
сразу ф-ции в некоей DLL.}
\EN{There is INT3-breakpoints in \tracer, triggering only once, however, they can be set to all functions
in specific DLL.}

\begin{lstlisting}
--one-time-INT3-bp:somedll.dll!.*
\end{lstlisting}

\RU{Либо, поставим INT3-прерывание на все функции, имена которых начинаются с префикса \TT{xml}:}
\EN{Or, let's set INT3-breakpoints to all functions with \TT{xml} prefix in name:}

\begin{lstlisting}
--one-time-INT3-bp:somedll.dll!xml.*
\end{lstlisting}

\RU{В качестве обратной стороны медали, такие прерывания срабатывают только один раз.}
\EN{On the other side of coin, such breakpoints are triggered only once.}

Tracer \RU{покажет вызов какой-либо функции, если он случится, но только один раз.}
\EN{will show calling of a function, if it happens, but only once.}
\RU{Еще один недостаток --- увидеть аргументы функции также нельзя.}
\EN{Another drawback~---it is impossible to see function's arguments.}

\RU{Тем не менее, эта возможность очень удобна для тех ситуаций, 
когда вы знаете что некая программа использует некую DLL,
но не знаете какие именно функции в этой DLL.}
\EN{Nevertheless, this feature is very useful when you know the program uses a DLL,
but do not know which functions in it.}
\RU{И функций много.}\EN{And there are a lot of functions.} \\
\\
\index{cygwin}
\RU{Например, попробуем узнать, что использует cygwin-утилита uptime}
\EN{For example, let's see, what uptime cygwin-utility uses}:

\begin{lstlisting}
tracer -l:uptime.exe --one-time-INT3-bp:cygwin1.dll!.*
\end{lstlisting}

\RU{Так мы можем увидеть все ф-ции из библиотеки cygwin1.dll, которые были вызваны хотя бы один раз, и откуда}
\EN{Thus we may see all cygwin1.dll library functions which were called at least once, and where from}:

\lstinputlisting{digging_into_code/uptime_cygwin.txt}

