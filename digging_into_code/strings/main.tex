\section{\IFRU{Текстовые строки}{Text strings}}

\label{C_strings}
\IFRU{Обычные строки в Си заканчиваются нулем}{Usual C-strings are zero-terminated} 
(\ac{ASCIIZ}-\IFRU{строки}{strings}).

\IFRU{Причина, почему формат строки в Си именно такой (оканчивающийся нулем) вероятно историческая}
{The reason why C string format is as it is (zero-terminating) is apparently hisorical}.
\IFRU{В}{In} \cite{Ritchie79} \IFRU{мы можем прочитать}{we can read}:

\begin{framed}
\begin{quotation}
A minor difference was that the unit of I/O was the word, not the byte, because the PDP-7 was a word-addressed
machine. In practice this meant merely that all programs dealing with character streams ignored null
characters, because null was used to pad a file to an even number of characters.
\end{quotation}
\end{framed}

\index{Hiew}
\IFRU{Строки выглядят в Hiew или FAR Manager точно так же}
{In Hiew or FAR Manager these strings looks like as it is}:

\begin{lstlisting}
int main()
{
	printf ("Hello, world!\n");
};
\end{lstlisting}

\begin{figure}[H]
\centering
\includegraphics[scale=0.66]{digging_into_code/strings/C-string.png}
\caption{Hiew}
\end{figure}

\index{Pascal}
\index{Borland Delphi}
\IFRU{Когда кодируются строки в Pascal и Delphi, сама строка предваряется 8-битным или 32-битным значением,
в котором закодирована длина строки}{The string is preceeded by 8-bit or 32-bit string length value}.

\IFRU{Например}{For example}:

\begin{lstlisting}[caption=Delphi]
CODE:00518AC8                 dd 19h
CODE:00518ACC aLoading___Plea db 'Loading... , please wait.',0

...

CODE:00518AFC                 dd 10h
CODE:00518B00 aPreparingRun__ db 'Preparing run...',0
\end{lstlisting}

\subsection{Unicode}

\index{Unicode}
\IFRU{Нередко уникодом называют все способы передачи символов, когда символ занимает 2 байта или 16 бит}
{Often, what is called by Unicode is a methods of strings encoding when each character occupies 2 bytes or 16 bits}.
\IFRU{Это распространенная терминологическая ошибка}{This is common terminological mistake}.
\IFRU{Уникод --- это стандарт, присваивающий номер каждому символу многих письменностей мира, но не описывающий
способ кодирования}{Unicode is a standard assigning a number to each character of many writing systems of the 
world, but not describing encoding method}.

\index{UTF-8}
\index{UTF-16LE}
\IFRU{Наиболее популярные способы кодирования}{Most popular encoding methods are}: 
UTF-8 (\IFRU{наиболее часто используется в Интернете и *NIX-системах}{often used in Internet and *NIX systems})
\AndENRU UTF-16LE (\IFRU{используется в}{used in} Windows).

\subsubsection{UTF-8}

\index{UTF-8}
UTF-8 \IFRU{это один из очень удачных способов кодирования символов}{is one of the most successful methods of
character encoding}.
\IFRU{Все символы латинницы кодируются так же, как и в ASCII-кодировке, а символы выходящие за пределы
ASCII-7-таблицы, кодируются несколькими байтами}{All Latin symbols are encoded just like in an ASCII-encoding,
and symbols beyond ASCII-table are encoded by several bytes}.
\IFRU{$0$ кодируется, как и прежде, нулевыми байтом, так что все стандартные
ф-ции Си продолжают работать с UTF-8-строками так же как и с обычными строками}{$0$ is encoded as it was
before, so all standard C string functions works with UTF-8-strings just like any other string}.

\IFRU{Посмотрим как символы из разных языков кодируются в UTF-8 и как это выглядит в FAR, в кодировке 437}
{Let's see how symbols in various languages are encoded in UTF-8 and how it looks like in FAR in 437 codepage}
\footnote{\IFRU{Я взял пример и переводы на разные языки здесь}{I've got example and translations from there}: 
\url{http://www.columbia.edu/~fdc/utf8/}}:

\begin{figure}[H]
\centering
\includegraphics[scale=0.66]{digging_into_code/strings/multilang_sampler.png}
\end{figure}

\begin{figure}[H]
\centering
\includegraphics[scale=\FigScale]{digging_into_code/strings/multilang_sampler_UTF8.png}
\caption{FAR: UTF-8}
\end{figure}

\IFRU{Видно что строка на английском языке выглядит точно так же, как и в ASCII-кодировке}
{As it seems, English language string looks like as it is in ASCII-encoding}.
\IFRU{В венгерском языке используются латинница плюс латинские буквы с диакритическими знаками}
{Hungarian language uses Latin symbols plus symbols with diacritic marks}.
\IFRU{Здесь видно что эти буквы кодируются несколькими байтами, я подчеркнул их красным}
{These symbols are encoded by several bytes, I underscored them by red}.
\IFRU{То же самое с исландским и польским языками}{The same story with Icelandic and Polish languages}.
\IFRU{В самом начале я также применил символ валюты ``Евро'', который кодируется тремя байтами}
{I also used ``Euro'' currency symbol at the begin, which is encoded by 3 bytes}.
\IFRU{Остальные системы письма здесь никак не связаны с латинницей}
{All the rest writing systems here have no connection with Latin}.
\IFRU{По крайней мере о русском, арабском, иврите и хинди мы можем сказать что здесь видны повторяющиеся
байты, что не удивительно, ведь, обычно буквы из одной и той же системы письменности расположены в одной
или нескольких таблицах уникода, поэтому часто их коды начинаются с одних и тех же цифр}
{At least about Russian, Arabic, Hebrew and Hindi we could see recurring bytes, and that is not surprise:
all symbols from the writing system is usually located in the same Unicode table, so their code begins with
the same numbers}.

\IFRU{В самом начале, перед строкой ``How much?'', видны три байта, которые на самом деле \ac{BOM}}
{At the very beginning, before ``How much?'' string we see 3 bytes, which is \ac{BOM} in fact}.
\ac{BOM} \IFRU{показывает, какой способ кодирования будет сейчас использоваться}{defines encoding system to be
used now}.

\subsubsection{UTF-16LE}

\index{UTF-16LE}
\index{Win32}
\IFRU{Многие ф-ции win32 в Windows имееют суффикс}{Many win32 functions in Windows has a suffix} \TT{-A} 
\AndENRU \TT{-W}.
\IFRU{Первые ф-ции работают с обычными строками, вторые с UTF-16LE-строками}{The first functions works
with usual strings, the next with UTF-16LE-strings} (\IT{wide}).
\IFRU{Во втором случае, каждый символ обычно хранится в 16-битной переменной типа \IT{short}}
{As in the second case, each symbol is usually stored in 16-bit value of \IT{short} type}.

\IFRU{Cтроки с латинскими буквами выглядят в Hiew или FAR как перемежающиеся с нулевыми байтами}
{Latin symbols in UTF-16 strings looks in Hiew or FAR as interleaved with zero byte}:

\begin{lstlisting}
int wmain()
{
	wprintf (L"Hello, world!\n");
};
\end{lstlisting}

\begin{figure}[H]
\centering
\includegraphics[scale=0.66]{digging_into_code/strings/UTF16-string.png}
\caption{Hiew}
\end{figure}

\IFRU{Подобное можно часто увидеть в системных файлах \gls{Windows NT}}{We may often see this in \gls{Windows NT} 
system files}:

\begin{figure}[H]
\centering
\includegraphics[scale=0.66]{digging_into_code/strings/ntoskrnl_UTF16.png}
\caption{Hiew}
\end{figure}

\index{IDA}
\IFRU{В \IDA, уникодом называется именно строки с символами занимающими 2 байта}{String with characters
occupying exactly 2 bytes are called by ``Unicode'' in \IDA}:

\begin{lstlisting}
.data:0040E000 aHelloWorld:
.data:0040E000                 unicode 0, <Hello, world!>
.data:0040E000                 dw 0Ah, 0
\end{lstlisting}

\IFRU{Вот как может выглядеть строка на русском языке}{Here is how Russian language 
string}\RU{ (``И снова здравствуйте!'')} \IFRU{закодированная в UTF-16LE}{encoded in UTF-16LE may looks like}:

\begin{figure}[H]
\centering
\includegraphics[scale=0.66]{digging_into_code/strings/russian_UTF16.png}
\caption{Hiew: UTF-16LE}
\end{figure}

\IFRU{То что бросается в глаза --- это то что символы перемежаются ромбиками (который имеет код 4)}
{What we can easily spot---is that symbols are interleaved by diamond character (which has code of 4)}.
\IFRU{Действительно, в уникоде кирилличные символы находятся в четвертом блоке}{Indeed, Cyrillic symbols
are located in the fourth Unicode plane}
\footnote{\url{https://en.wikipedia.org/wiki/Cyrillic_(Unicode_block)}}.
\IFRU{Таким образом, все кирилличные символы в UTF-16LE находятся в диапазоне}{Hence, all
Cyrillic symbols in UTF-16LE are located in} \TT{0x400-0x4FF}\EN{ range}.

\IFRU{Вернемся к примеру, где одна и та же строка написана разными языками}{Let's back to the example with the
string written in multiple languages}.
\IFRU{Здесь посмотрим в кодировке UTF-16LE}{Here we can see it in UTF-16LE encoding}.

\begin{figure}[H]
\centering
\includegraphics[scale=\FigScale]{digging_into_code/strings/multilang_sampler_UTF16.png}
\caption{FAR: UTF-16LE}
\end{figure}

\IFRU{Здесь мы также видим \ac{BOM} в самом начале}{Here we can also see \ac{BOM} in the very beginning}.
\IFRU{Все латинские буквы перемежаются с нулевыми байтом}{All Latin characters are interleaved with zero byte}.
\IFRU{Некоторые буквы с диакритическими знаками (венгерский и исландский языки) также подчеркнуты красным}
{I also underscored by red some characters with diacritic marks (Hungarian and Icelandic languages)}.

% TODO: strings *NIX utility. procmonitor also shows strings...

