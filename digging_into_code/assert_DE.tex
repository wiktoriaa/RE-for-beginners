\section{Calls to assert()}
\myindex{\CStandardLibrary!assert()}

Manchmal ist die presänz des \TT{assert()} macro's ebenfalls nützlich:
allgemein erlaubt dieses Makro rückschlüsse auf source code Dateinamen,
Zeilen nummern und die Bedinung für das Macro im Code.

% Sometimes the presence of the \TT{assert()} macro is useful too: 
% commonly this macro leaves source file name, line number and condition in the code.

Die nützlichste Informastionen ist enthalten in der Bedingung von assert, wir können Variablennamen oder Namen
von Struct Feldern ableiten. Ein weiteres nützliches Stück Information sind die Datei namen---Wir können versuchen
abzuleiten von welcher Art der Code ist. 
Es ist ebenfalls möglich bekannte open-source librarie-namen von den Datei namen abzu leiten.

% The most useful information is contained in the assert's condition, we can deduce variable names or structure field
% names from it. Another useful piece of information are the file names---we can try to deduce what type of
% code is there.
% Also it is possible to recognize well-known open-source libraries by the file names.

\lstinputlisting[caption=Example of informative assert() calls,style=customasmx86]{digging_into_code/assert_examples.lst}

Es ist empfehlens wert beides die Konditionen und die Datei namen in \q{google} zu suchen, was uns zu einer open-source library führen kann.
Zum Beispiel, wenn wir \q{sp->lzw\_nbits <= BITS\_MAX} in \q{google} suchen, ist es absehbar das wir als ergebnis code aus der 
Open-Source library für die LZW Kompression bekommen. % <-- umbedingt noch mal überarbeiten.

% It is advisable to \q{google} both the conditions and file names, which can lead us to an open-source library.
% For example, if we \q{google} \q{sp->lzw\_nbits <= BITS\_MAX}, this predictably 
% gives us some open-source code that's related to the LZW compression.
