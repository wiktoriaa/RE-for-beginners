% TODO move section...

\subsection{Some binary file patterns}

All examples here were prepared on the Windows with active code page 437
\footnote{\url{https://en.wikipedia.org/wiki/Code_page_437}} in console.
Binary files internally may look visually different if another code page is set.

\clearpage
\subsubsection{Arrays}

Sometimes, we can clearly spot an array of 16/32/64-bit values visually, in hex editor.

Here is an example of array of 16-bit values.
We see that the first byte in pair is 7 or 8, and the second looks random:

\begin{figure}[H]
\centering
\myincludegraphics{digging_into_code/binary/16bit_array.png}
\caption{FAR: array of 16-bit values}
\end{figure}

I used a file containing 12-channel signal digitized using 16-bit \ac{ADC}.

\clearpage
\myindex{MIPS}
\par And here is an example of very typical MIPS code.

As we may recall, every MIPS (and also ARM in ARM mode or ARM64) instruction has size of 32 bits (or 4 bytes), 
so such code is array of 32-bit values.

By looking at this screenshot, we may see some kind of pattern.

Vertical red lines are added for clarity:

\begin{figure}[H]
\centering
\myincludegraphics{digging_into_code/binary/typical_MIPS_code.png}
\caption{Hiew: very typical MIPS code}
\end{figure}

Another example of such pattern here is book: 
\myref{Oracle_SYM_files_example}.

\clearpage
\subsubsection{Sparse files}

This is sparse file with data scattered amidst almost empty file.
Each space character here is in fact zero byte (which is looks like space).
This is a file to program FPGA (Altera Stratix GX device).
Of course, files like these can be compressed easily, but formats like this one are very popular in scientific and engineering software where efficient access is important while compactness is not.

\begin{figure}[H]
\centering
\myincludegraphics{digging_into_code/binary/sparse_FPGA.png}
\caption{FAR: Sparse file}
\end{figure}

\clearpage
\subsubsection{Compressed file}

% FIXME \ref{} ->
This file is just some compressed archive.
It has relatively high entropy and visually looks just chaotic.
This is how compressed and/or encrypted files looks like.

\begin{figure}[H]
\centering
\myincludegraphics{digging_into_code/binary/compressed.png}
\caption{FAR: Compressed file}
\end{figure}

\clearpage
\subsubsection{\ac{CDFS}}

\ac{OS} installations are usually distributed as ISO files which are copies of CD/DVD discs.
Filesystem used is named \ac{CDFS}, here is you see file names mixed with some additional data.
This can be file sizes, pointers to another directories, file attributes, etc.
This is how typical filesystems may look internally.

\begin{figure}[H]
\centering
\myincludegraphics{digging_into_code/binary/cdfs.png}
\caption{FAR: ISO file: Ubuntu 15 installation \ac{CD}}
\end{figure}

\clearpage
\subsubsection{32-bit x86 executable code}

This is how 32-bit x86 executable code looks like.
It has not very high entropy, because some bytes occurred more often than others.

\begin{figure}[H]
\centering
\myincludegraphics{digging_into_code/binary/x86_32.png}
\caption{FAR: Executable 32-bit x86 code}
\end{figure}

% TODO: Read more about x86 statistics: \ref{}. % FIXME blog post about decryption...

\clearpage
\subsubsection{BMP graphics files}

% TODO: bitmap, bit, group of bits...

BMP files are not compressed, so each byte (or group of bytes) describes each pixel.
I've found this picture somewhere inside my installed Windows 8.1:

\begin{figure}[H]
\centering
\myincludegraphicsSmall{digging_into_code/binary/bmp.png}
\caption{Example picture}
\end{figure}

You see that this picture has some pixels which unlikely can be compressed very good (around center), 
but there are long one-color lines at top and bottom.
Indeed, lines like these also looks as lines during viewing the file:

\begin{figure}[H]
\centering
\myincludegraphics{digging_into_code/binary/bmp_FAR.png}
\caption{BMP file fragment}
\end{figure}

