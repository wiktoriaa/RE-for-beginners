% TODO move section...

\subsection{Some binary file patterns} % <-- Find hier ne gute übersetzung

Alle Beispiele hier wurden vorbereitet mit Windows mit aktiver Code Page 437
\footnote{\url{https://en.wikipedia.org/wiki/Code_page_437}} in der Konsole.
Binär Datein sehen intern etwas anders aus wenn eine andere Code page gesetzt ist.

% All examples here were prepared on the Windows with active code page 437
% \footnote{\url{https://en.wikipedia.org/wiki/Code_page_437}} in console.
% Binary files internally may look visually different if another code page is set.

\clearpage
\subsubsection{Arrays}

Manchmal kann man klar ein Array von 16/32/64-Bit Werten mit blosem Auge im hex editor erkennen.
% Sometimes, we can clearly spot an array of 16/32/64-bit values visually, in hex editor.

Hier ist ein Beispiel eines 16-Bit Wertes.
Wir sehen das das erse Byte ein paar aus 7 oder 8 ist und das zweite sieht
zufällig aus:
% Here is an example of array of 16-bit values.
% We see that the first byte in pair is 7 or 8, and the second looks random:

\begin{figure}[H]
\centering
\myincludegraphics{digging_into_code/binary/16bit_array.png}
\caption{FAR: array of 16-bit values}
\end{figure}

Ich habe eine Datei benutzt die ein 12 Kannal Signal digitalisiert benutzt mit 16-Bit \ac{ADC}. % <-- Besseres? 
% I used a file containing 12-channel signal digitized using 16-bit \ac{ADC}.

\clearpage
\myindex{MIPS}
\par Und hier ist ein Beispiel von einem Typischen MIPS code.
% \par And here is an example of very typical MIPS code.

Wie wir uns vielleicht erinnern, jede MIPS ( also auch ARM in ARM mode oder ARM64 ) Instruktion hat eine größe von 32 Bits (oder 4 Bytes),
also ist solcher Code ein Array von 32-Bit Werten. 

% As we may recall, every MIPS (and also ARM in ARM mode or ARM64) instruction has size of 32 bits (or 4 bytes), 
% so such code is array of 32-bit values.

Wenn man den Screenshot anschaut, sehen wir eine Art Muster.
% By looking at this screenshot, we may see some kind of pattern.

Vertikale und rote Linien wurden zur besseren lesbarkeit eingefügt:
% 5Vertical red lines are added for clarity:

\begin{figure}[H]
\centering
\myincludegraphics{digging_into_code/binary/typical_MIPS_code.png}
% \caption{Hiew: very typical MIPS code}
\caption{Hiew: sehr typischer MIPS code}
\end{figure}

Ein weiteres Beispiel eines solchen Musters ist Buch:
% Another example of such pattern here is book: 
\myref{Oracle_SYM_files_example}.

\clearpage
\subsubsection{Sparse files}

Diese dürftige Datei mit zerstreuten Daten inmitten einer fast leeren Datei.
Jedes Space Zeichen hier ist in der tat ein Zero Byte (das wie ein space aussieht). % <-- findet man sicher was besseres
Das ist eine Datei mit der ein FPGA Programmiert wird (Ein Altera Stratix GX Gerät).
Sicher können Dateien wie diese einfach Kompremiert werden, aber diese Formate sind in 
der Wissenschaft und im Ingeneurs wesen so wie in der Softwareentwicklung sehr verbreitet.
Wo es oft um effizienten Zugriff geht und weniger um die Komprimierung der Daten.

% This is sparse file with data scattered amidst almost empty file.
% Each space character here is in fact zero byte (which is looks like space).
% This is a file to program FPGA (Altera Stratix GX device).
% Of course, files like these can be compressed easily, but formats like this one are very popular in scientific and engineering software where efficient access is important while compactness is not.

\begin{figure}[H]
\centering
\myincludegraphics{digging_into_code/binary/sparse_FPGA.png}
\caption{FAR: Sparse file}
\end{figure}

\clearpage
\subsubsection{Compressed file}

% FIXME \ref{} ->
Diese Datei ist einfach ein komprimiertes Archiv. 
Es hat eine relativ hohe entropie und visuell betrachtet sieht es 
eher Chaotisch aus. So sehen komprimierte oder verschlüsselte Datein aus.

% FIXME \ref{} ->
% This file is just some compressed archive.
% It has relatively high entropy and visually looks just chaotic.
% This is how compressed and/or encrypted files looks like.

\begin{figure}[H]
\centering
\myincludegraphics{digging_into_code/binary/compressed.png}
\caption{FAR: Komprimierte Datei}
\end{figure}

\clearpage
\subsubsection{\ac{CDFS}}

\ac{OS} installationen werden üblicherweise als ISO Datei bereit gestellt, die kopien von CD/DVD Disks sind. 
Das Dateisystem das benutzt wird heißt \ac{cdfs}, hier sieht man wie Dateinamen mit zusätzlichen Daten vermischt sind.
Das können Datei größen, Pointer auf andere Verzeichnise, Datei attribute und anderes sein. 
So sehen Dateisysteme typischerweise auch von innen aus.

% \ac{OS} installations are usually distributed as ISO files which are copies of CD/DVD discs.
% Filesystem used is named \ac{CDFS}, here is you see file names mixed with some additional data.
% This can be file sizes, pointers to another directories, file attributes, etc.
% This is how typical filesystems may look internally.

\begin{figure}[H]
\centering
\myincludegraphics{digging_into_code/binary/cdfs.png}
\caption{FAR: ISO file: Ubuntu 15 installation \ac{CD}}
\end{figure}

\clearpage
\subsubsection{32-bit x86 executable code}

So sieht 32-Bit x86 ausführbarer Code aus. 
Der Code hat nicht wirklich viel entropy, weil manche Bytes öfters vorkommen als andere.

% This is how 32-bit x86 executable code looks like.
% It has not very high entropy, because some bytes occurred more often than others.

\begin{figure}[H]
\centering
\myincludegraphics{digging_into_code/binary/x86_32.png}
\caption{FAR: Executable 32-bit x86 code}
\end{figure}

% TODO: Read more about x86 statistics: \ref{}. % FIXME blog post about decryption...

\clearpage
\subsubsection{BMP graphics files}

% TODO: bitmap, bit, group of bits...

BMP Datein sind nicht komprimiert, also ist jedes Byte ( oder Gruppen von Bytes ) beschrieben als
ein pixel. Diese Bild habe ich irgendwo in meiner Windows 8.1 installation gefunden: 

% BMP files are not compressed, so each byte (or group of bytes) describes each pixel.
% I've found this picture somewhere inside my installed Windows 8.1:

\begin{figure}[H]
\centering
\myincludegraphicsSmall{digging_into_code/binary/bmp.png}
\caption{Example picture}
\end{figure}

Man kann sehen das dieses Bild Pixel hat, die nicht wirklich gut komprimiert werden könne (um das Zentrum herrum),
aber es sind lange ein-farben linien am anfang und am ende der Datei. Tatzächlich linien wie diese sehen wie linien aus
wenn man sich die Datei anschaut:

% You see that this picture has some pixels which unlikely can be compressed very good (around center), 
% but there are long one-color lines at top and bottom.
% Indeed, lines like these also looks as lines during viewing the file:

\begin{figure}[H]
\centering
\myincludegraphics{digging_into_code/binary/bmp_FAR.png}
\caption{BMP file fragment}
\end{figure}

