\chapter{\RU{Поиск в коде того что нужно}
\EN{Finding important/interesting stuff in the code}
\DE{Finden von wichtigen / interessanten Stellen im Code}}

\RU{Современное ПО, в общем-то, минимализмом не отличается.}
\EN{Minimalism it is not a prominent feature of modern software.}

\myindex{\Cpp!STL}
\RU{Но не потому, что программисты слишком много пишут, 
а потому что к исполняемым файлам обыкновенно прикомпилируют все подряд библиотеки. 
Если бы все вспомогательные библиотеки всегда выносили во внешние DLL, мир был бы иным.
(Еще одна причина для Си++ --- \ac{STL} и прочие библиотеки шаблонов.)}
\EN{But not because the programmers are writing a lot, but because a lot of libraries are commonly linked statically
to executable files.
If all external libraries were shifted into an external DLL files, the world would be different.
(Another reason for C++ are the \ac{STL} and other template libraries.)}

\newcommand{\FOOTNOTEBOOST}{\footnote{\url{http://go.yurichev.com/17036}}}
\newcommand{\FOOTNOTELIBPNG}{\footnote{\url{http://go.yurichev.com/17037}}}

\RU{Таким образом, очень полезно сразу понимать, какая функция из стандартной библиотеки или 
более-менее известной (как Boost\FOOTNOTEBOOST, libpng\FOOTNOTELIBPNG), 
а какая --- имеет отношение к тому что мы пытаемся найти в коде.}
\EN{Thus, it is very important to determine the origin of a function, if it is from standard library or 
well-known library (like Boost\FOOTNOTEBOOST, libpng\FOOTNOTELIBPNG),
or if it is related to what we are trying to find in the code.}

\RU{Переписывать весь код на \CCpp, чтобы разобраться в нем, безусловно, не имеет никакого смысла.}
\EN{It is just absurd to rewrite all code in \CCpp to find what we're looking for.}

\RU{Одна из важных задач reverse engineer-а это быстрый поиск в коде того что собственно его интересует.}
\EN{One of the primary tasks of a reverse engineer is to find quickly the code he/she needs.}

\myindex{\GrepUsage}
\RU{Дизассемблер \IDA позволяет делать поиск как минимум строк, последовательностей байт, констант.
Можно даже сделать экспорт кода в текстовый файл .lst или .asm и затем натравить на него \TT{grep}, \TT{awk}, итд.}
\EN{The \IDA disassembler allow us to search among text strings, byte sequences and constants.
It is even possible to export the code to .lst or .asm text files and then use \TT{grep}, \TT{awk}, etc.}

\RU{Когда вы пытаетесь понять, что делает тот или иной код, это запросто может быть какая-то 
опенсорсная библиотека вроде libpng. Поэтому, когда находите константы, или текстовые строки, которые 
выглядят явно знакомыми, всегда полезно их \IT{погуглить}.
А если вы найдете искомый опенсорсный проект где это используется, 
то тогда будет достаточно будет просто сравнить вашу функцию с ней. 
Это решит часть проблем.}
\EN{When you try to understand what some code is doing, this easily could be some open-source library like libpng.
So when you see some constants or text strings which look familiar, it is always worth to \IT{google} them.
And if you find the opensource project where they are used, 
then it's enough just to compare the functions.
It may solve some part of the problem.}

\RU{К примеру, если программа использует какие-то XML-файлы, первым шагом может быть
установление, какая именно XML-библиотека для этого используется, ведь часто используется какая-то
стандартная (или очень известная) вместо самодельной.}
\EN{For example, if a program uses XML files, the first step may be determining which
XML library is used for processing, since the standard (or well-known) libraries are usually used
instead of self-made one.}

\myindex{SAP}
\myindex{Windows!PDB}
\RU{К примеру, автор этих строк однажды пытался разобраться как происходит компрессия/декомпрессия сетевых пакетов в SAP 6.0. 
Это очень большая программа, но к ней идет подробный .\gls{PDB}-файл с отладочной информацией, и это очень удобно. 
Он в конце концов пришел к тому что одна из функций декомпрессирующая пакеты называется CsDecomprLZC(). 
Не сильно раздумывая, он решил погуглить и оказалось, что функция с таким же названием имеется в MaxDB
(это опен-сорсный проект SAP) \footnote{Больше об этом в соответствующей секции~(\myref{sec:SAPGUI})}.}
\EN{For example, the author of these lines once tried to understand how the compression/decompression of network packets works in SAP 6.0. 
It is a huge software, but a detailed .\gls{PDB} with debugging information is present, 
and that is convenient.
He finally came to the idea that one of the functions, that was called \IT{CsDecomprLZC}, was doing the decompression of network packets.
Immediately he tried to google its name and he quickly found the function was used in MaxDB
(it is an open-source SAP project) \footnote{More about it in relevant section~(\myref{sec:SAPGUI})}.}

\url{http://www.google.com/search?q=CsDecomprLZC}

\RU{Каково же было мое удивление, когда оказалось, что в MaxDB используется точно такой же алгоритм, 
скорее всего, с таким же исходником.}
\EN{Astoundingly, MaxDB and SAP 6.0 software shared likewise code for the compression/decompression of network packets.}

\EN{\section{Identification of executable files}

\subsection{Microsoft Visual C++}
\label{MSVC_versions}

MSVC versions and DLLs that can be imported:

\small
\begin{center}
\begin{tabular}{ | l | l | l | l | l | }
\hline
\HeaderColor Marketing ver. & 
\HeaderColor Internal ver. & 
\HeaderColor CL.EXE ver. &
\HeaderColor DLLs imported &
\HeaderColor Release date \\
\hline
% 4.0, April 1995
% 97 & 5.0 & February 1997
6		&  6.0	& 12.00	& msvcrt.dll	& June 1998		\\
		&	&	& msvcp60.dll	&			\\
\hline
.NET (2002)	&  7.0	& 13.00	& msvcr70.dll	& February 13, 2002	\\
		&	&	& msvcp70.dll	&			\\
\hline
.NET 2003	&  7.1	& 13.10 & msvcr71.dll	& April 24, 2003	\\
		&	&	& msvcp71.dll	&			\\
\hline
2005		&  8.0	& 14.00 & msvcr80.dll	& November 7, 2005	\\
		&	&	& msvcp80.dll	&			\\
\hline
2008		&  9.0	& 15.00 & msvcr90.dll	& November 19, 2007	\\
		&	&	& msvcp90.dll	&			\\
\hline
2010		& 10.0	& 16.00 & msvcr100.dll	& April 12, 2010 	\\
		&	&	& msvcp100.dll	&			\\
\hline
2012		& 11.0	& 17.00 & msvcr110.dll	& September 12, 2012 	\\
		&	&	& msvcp110.dll	&			\\
\hline
2013		& 12.0	& 18.00 & msvcr120.dll	& October 17, 2013 	\\
		&	&	& msvcp120.dll	&			\\
\hline
\end{tabular}
\end{center}
\normalsize

msvcp*.dll has \Cpp{}-related functions, so if it is imported, 
this is probably a \Cpp program.

\subsubsection{Name mangling}

The names usually start with the \TT{?} symbol.

You can read more about MSVC's \gls{name mangling} here: \myref{namemangling}.

\subsection{GCC}
\myindex{GCC}

Aside from *NIX targets, GCC is also present in the win32 environment, in the form of Cygwin and MinGW.

\subsubsection{Name mangling}

Names usually start with the \TT{\_Z} symbols.

You can read more about GCC's \gls{name mangling} here: \myref{namemangling}.

\subsubsection{Cygwin}
\myindex{Cygwin}

cygwin1.dll is often imported.

\subsubsection{MinGW}
\myindex{MinGW}

msvcrt.dll may be imported.

\subsection{Intel Fortran}
\myindex{Fortran}

libifcoremd.dll, libifportmd.dll and libiomp5md.dll (OpenMP support) may be imported.

libifcoremd.dll has a lot of functions prefixed with \TT{for\_}, which means \IT{Fortran}.

\subsection{Watcom, OpenWatcom}
\myindex{Watcom}
\myindex{OpenWatcom}

\subsubsection{Name mangling}

Names usually start with the \TT{W} symbol.

For example, that is how the method named \q{method} of the class \q{class} that does not have any arguments and returns
\Tvoid is encoded:

\begin{lstlisting}
W?method$_class$n__v
\end{lstlisting}

\subsection{Borland}
\myindex{Borland Delphi}
\myindex{Borland C++Builder}

Here is an example of Borland Delphi's and C++Builder's \gls{name mangling}:

\lstinputlisting{digging_into_code/identification/borland_mangling.txt}

The names always start with the \TT{@} 
symbol, then we have the class name came, method name, and encoded the types of the arguments of the method.

These names can be in the .exe imports, .dll exports, debug data, etc.

Borland Visual Component Libraries (VCL) 
are stored in .bpl files instead of .dll ones, for example, vcl50.dll, rtl60.dll.

Another DLL that might be imported: BORLNDMM.DLL.

\subsubsection{Delphi}

Almost all Delphi executables has the \q{Boolean} text string at the beginning of the code segment, along with other type names.

This is a very typical beginning of the \TT{CODE} 
segment of a Delphi program, this block came right after the win32 PE file header:

\lstinputlisting{digging_into_code/identification/delphi.txt}

The first 4 bytes of the data segment (\TT{DATA}) can be \TT{00 00 00 00}, \TT{32 13 8B C0} or \TT{FF FF FF FF}.%

This information can be useful when dealing with packed/encrypted Delphi executables.

\subsection{Other known DLLs}

\myindex{OpenMP}
\begin{itemize}
\item vcomp*.dll---Microsoft's implementation of OpenMP.
\end{itemize}

}
\RU{\section{Идентификация исполняемых файлов}

\subsection{Microsoft Visual C++}
\label{MSVC_versions}

Версии MSVC и DLL которые могут быть импортированы:

\small
\begin{center}
\begin{tabular}{ | l | l | l | l | l | }
\hline
\HeaderColor Маркетинговая вер. & 
\HeaderColor Внутренняя вер. & 
\HeaderColor Вер. CL.EXE &
\HeaderColor Импорт.DLL &
\HeaderColor Дата выхода \\
\hline
% 4.0, April 1995
% 97 & 5.0 & February 1997
6		&  6.0	& 12.00	& msvcrt.dll	& June 1998		\\
		&	&	& msvcp60.dll	&			\\
\hline
.NET (2002)	&  7.0	& 13.00	& msvcr70.dll	& February 13, 2002	\\
		&	&	& msvcp70.dll	&			\\
\hline
.NET 2003	&  7.1	& 13.10 & msvcr71.dll	& April 24, 2003	\\
		&	&	& msvcp71.dll	&			\\
\hline
2005		&  8.0	& 14.00 & msvcr80.dll	& November 7, 2005	\\
		&	&	& msvcp80.dll	&			\\
\hline
2008		&  9.0	& 15.00 & msvcr90.dll	& November 19, 2007	\\
		&	&	& msvcp90.dll	&			\\
\hline
2010		& 10.0	& 16.00 & msvcr100.dll	& April 12, 2010 	\\
		&	&	& msvcp100.dll	&			\\
\hline
2012		& 11.0	& 17.00 & msvcr110.dll	& September 12, 2012 	\\
		&	&	& msvcp110.dll	&			\\
\hline
2013		& 12.0	& 18.00 & msvcr120.dll	& October 17, 2013 	\\
		&	&	& msvcp120.dll	&			\\
\hline
\end{tabular}
\end{center}
\normalsize

msvcp*.dll содержит функции связанные с \Cpp{}, так что если она импортируется, скорее всего, 
вы имеете дело с программой на \Cpp.

\subsubsection{Name mangling}

Имена обычно начинаются с символа \TT{?}.

О \gls{name mangling} в MSVC читайте также здесь: \myref{namemangling}.

\subsection{GCC}
\myindex{GCC}

Кроме компиляторов под *NIX, GCC имеется так же и для win32-окружения: в виде Cygwin и MinGW.

\subsubsection{Name mangling}

Имена обычно начинаются с символов \TT{\_Z}.

О \gls{name mangling} в GCC читайте также здесь: \myref{namemangling}.

\subsubsection{Cygwin}
\myindex{Cygwin}

cygwin1.dll часто импортируется.

\subsubsection{MinGW}
\myindex{MinGW}

msvcrt.dll может импортироваться.

\subsection{Intel Fortran}
\myindex{Фортран}

libifcoremd.dll, libifportmd.dll и libiomp5md.dll (поддержка OpenMP) могут импортироваться.

В libifcoremd.dll много функций с префиксом \TT{for\_}, что значит \IT{Fortran}.

\subsection{Watcom, OpenWatcom}
\myindex{Watcom}
\myindex{OpenWatcom}

\subsubsection{Name mangling}

Имена обычно начинаются с символа \TT{W}.

Например, так кодируется метод \q{method} класса \q{class} не имеющий аргументов и возвращающий \Tvoid{}:

\begin{lstlisting}
W?method$_class$n__v
\end{lstlisting}

\subsection{Borland}
\myindex{Borland Delphi}
\myindex{Borland C++Builder}

Вот пример \gls{name mangling} в Borland Delphi и C++Builder:

\lstinputlisting{digging_into_code/identification/borland_mangling.txt}

Имена всегда начинаются с символа \TT{@} 
затем следует имя класса, имя метода
и закодированные типы аргументов.

Эти имена могут присутствовать с импортах .exe, экспортах .dll, отладочной информации, итд.

Borland Visual Component Libraries (VCL) находятся в файлах .bpl вместо .dll, например, vcl50.dll, rtl60.dll.

Другие DLL которые могут импортироваться: BORLNDMM.DLL.

\subsubsection{Delphi}

Почти все исполняемые файлы имеют текстовую строку \q{Boolean} 
в самом начале сегмента кода, среди остальных имен типов.

Вот очень характерное для Delphi начало сегмента \TT{CODE}, 
этот блок следует сразу за заголовком win32 PE-файла:

\lstinputlisting{digging_into_code/identification/delphi.txt}

Первые 4 байта сегмента данных (\TT{DATA}) в исполняемых файлах могут быть \TT{00 00 00 00}, \TT{32 13 8B C0} или \TT{FF FF FF FF}.
Эта информация может помочь при работе с запакованными/зашифрованными программами на Delphi.

\subsection{Другие известные DLL}

\myindex{OpenMP}
\begin{itemize}
\item vcomp*.dll --- Реализация OpenMP от Microsoft.
\end{itemize}

}
% binary files might be also here
\EN{\mysection{Communication with the outer world (win32)}

Sometimes it's enough to observe some function's inputs and outputs in order to understand what it does.
That way you can save time.

Files and registry access: 
for the very basic analysis, Process Monitor\footnote{\url{http://go.yurichev.com/17301}}
utility from SysInternals can help.

For the basic analysis of network accesses, Wireshark\footnote{\url{http://go.yurichev.com/17303}} can be useful.

But then you will have to look inside anyway. \\
\\
The first thing to look for is which functions from the \ac{OS}'s \ac{API}s and standard libraries are used.

If the program is divided into a main executable file and a group of DLL files, sometimes the names of the functions in these DLLs can help.

If we are interested in exactly what can lead to a call to \TT{MessageBox()} with specific text, 
we can try to find this text in the data segment, find the references to it and find the points
from which the control may be passed to the \TT{MessageBox()} call we're interested in.

\myindex{\CStandardLibrary!rand()}
If we are talking about a video game and we're interested in which events are more or less random in it,
we may try to find the \rand function or its replacements (like the Mersenne twister algorithm) and find the places
from which those functions are called, and more importantly, how are the results used.
% BUG in varioref: http://tex.stackexchange.com/questions/104261/varioref-vref-or-vpageref-at-page-boundary-may-loop
One example: \ref{chap:color_lines}. 

But if it is not a game, and \rand is still used, it is also interesting to know why.
There are cases of unexpected \rand usage in data compression algorithms (for encryption imitation):
\href{http://go.yurichev.com/17221}{blog.yurichev.com}.

\subsection{Often used functions in the Windows API}

These functions may be among the imported.
It is worth to note that not every function might be used in the code that was written by the programmer.
A lot of functions might be called from library functions and \ac{CRT} code.

Some functions may have the \GTT{-A} suffix for the ASCII version and \GTT{-W} for the Unicode version.

\begin{itemize}

\item
Registry access (advapi32.dll): 
RegEnumKeyEx, RegEnumValue, RegGetValue, RegOpenKeyEx, RegQueryValueEx.

\item
Access to text .ini-files (kernel32.dll): 
GetPrivateProfileString.

\item
Dialog boxes (user32.dll): 
MessageBox, MessageBoxEx, CreateDialog, SetDlgItemText, GetDlgItemText.

\item
Resources access (\myref{PEresources}): (user32.dll): LoadMenu.

\item
TCP/IP networking (ws2\_32.dll):
WSARecv, WSASend.

\item
File access (kernel32.dll):
CreateFile, ReadFile, ReadFileEx, WriteFile, WriteFileEx.

\item
High-level access to the Internet (wininet.dll): WinHttpOpen.

\item
Checking the digital signature of an executable file (wintrust.dll):
WinVerifyTrust.

\item
The standard MSVC library (if it's linked dynamically) (msvcr*.dll):
assert, itoa, ltoa, open, printf, read, strcmp, atol, atoi, fopen, fread, fwrite, memcmp, rand,
strlen, strstr, strchr.

\end{itemize}

\subsection{Extending trial period}

Registry access functions are frequent targets for those who try to crack trial period of some software, which may save
installation date/time into registry.

Another popular target are GetLocalTime() and GetSystemTime() functions:
a trial software, at each startup, must check current date/time somehow anyway.

\subsection{Removing nag dialog box}

A popular way to find out what causing popping nag dialog box is intercepting MessageBox(), 
CreateDialog() and CreateWindow() functions.

\subsection{tracer: Intercepting all functions in specific module}
\myindex{tracer}

\myindex{x86!\Instructions!INT3}
There are INT3 breakpoints in the \tracer, that are triggered only once, however, they can be set for all functions
in a specific DLL.

\begin{lstlisting}
--one-time-INT3-bp:somedll.dll!.*
\end{lstlisting}

Or, let's set INT3 breakpoints on all functions with the \TT{xml} prefix in their name:

\begin{lstlisting}
--one-time-INT3-bp:somedll.dll!xml.*
\end{lstlisting}

On the other side of the coin, such breakpoints are triggered only once.
Tracer will show the call of a function, if it happens, but only once.
Another drawback---it is impossible to see the function's arguments.

Nevertheless, this feature is very useful when you know that the program uses a DLL,
but you do not know which functions are actually used.
And there are a lot of functions. 

\par
\myindex{Cygwin}
For example, let's see, what does the uptime utility from cygwin use:

\begin{lstlisting}
tracer -l:uptime.exe --one-time-INT3-bp:cygwin1.dll!.*
\end{lstlisting}

Thus we may see all that cygwin1.dll library functions that were called at least once, and where from:

\lstinputlisting{digging_into_code/uptime_cygwin.txt}

}
\RU{\mysection{Связь с внешним миром (win32)}

Иногда, чтобы понять, что делает та или иная функция, можно её не разбирать, а просто посмотреть на её входы и выходы.
Так можно сэкономить время.

Обращения к файлам и реестру: 
для самого простого анализа может помочь утилита Process Monitor\footnote{\url{http://go.yurichev.com/17301}}
от SysInternals.

Для анализа обращения программы к сети, может помочь  Wireshark\footnote{\url{http://go.yurichev.com/17303}}.

Затем всё-таки придётся смотреть внутрь. \\
\\
Первое на что нужно обратить внимание, это какие функции из \ac{API} \ac{OS}
и какие функции стандартных библиотек используются.
Если программа поделена на главный исполняемый файл и группу DLL-файлов, то имена функций в этих DLL, бывает так, могут помочь.

Если нас интересует, что именно приводит к вызову \TT{MessageBox()} с определенным текстом, 
то первое, что можно попробовать сделать: найти в сегменте данных этот текст, найти ссылки на него, и найти, 
откуда может передаться управление к интересующему нас вызову \TT{MessageBox()}.

\myindex{\CStandardLibrary!rand()}
Если речь идет о компьютерной игре, и нам интересно какие события в ней более-менее случайны, 
мы можем найти функцию \rand или её заменитель (как алгоритм Mersenne twister), и посмотреть, 
из каких мест эта функция вызывается и что самое главное: как используется результат этой функции.%

% BUG in varioref: http://tex.stackexchange.com/questions/104261/varioref-vref-or-vpageref-at-page-boundary-may-loop
Один пример: \ref{chap:color_lines}. 

Но если это не игра, а \rand используется, то также весьма любопытно, зачем. 
Бывают неожиданные случаи вроде использования \rand в алгоритме для сжатия данных (для имитации шифрования):
\href{http://go.yurichev.com/17221}{blog.yurichev.com}.

\subsection{Часто используемые функции Windows API}

Это функции которые можно увидеть в числе импортируемых.
Но также нельзя забывать, что далеко не все они были использованы в коде написанном автором.
Немалая часть может вызываться из библиотечных функций и \ac{CRT}-кода.
	
Многие ф-ции могут иметь суффикс \GTT{-A} для ASCII-версии и \GTT{-W} для Unicode-версии.

\begin{itemize}

\item
Работа с реестром (advapi32.dll): 
RegEnumKeyEx, RegEnumValue, RegGetValue, RegOpenKeyEx, RegQueryValueEx.

\item
Работа с текстовыми .ini-файлами (kernel32.dll):\\
GetPrivateProfileString.

\item
Диалоговые окна (user32.dll):\\
MessageBox, MessageBoxEx, CreateDialog, SetDlgItemText, GetDlgItemText.

\item
Работа с ресурсами (\myref{PEresources}): (user32.dll):
LoadMenu.

\item
Работа с TCP/IP-сетью (ws2\_32.dll):
WSARecv, WSASend.

\item
Работа с файлами (kernel32.dll):
CreateFile, ReadFile, ReadFileEx, WriteFile, WriteFileEx.

\item
Высокоуровневая работа с Internet
(wininet.dll):
WinHttpOpen.

\item
Проверка цифровой подписи исполняемого файла (wintrust.dll):
WinVerifyTrust.

\item
Стандартная библиотека MSVC (в случае динамического связывания)%
 (msvcr*.dll):
assert, itoa, ltoa, open, printf, read, strcmp, atol, atoi, fopen, fread, fwrite, memcmp, rand,
strlen, strstr, strchr.

\end{itemize}

\subsection{Расширение триального периода}

Ф-ции доступа к реестру это частая цель тех, кто пытается расширить триальный период ПО, которое
может сохранять дату/время инсталляции в реестре.

Другая популярная цель это ф-ции GetLocalTime() и GetSystemTime():
триальное ПО, при каждом запуске, должно как-то проверять текущую дату/время.

% FIXME language!
\subsection{Удаление nag-окна}

Популярный метод поиска того, что заставляет выводить nag-окно это перехват ф-ций MessageBox(),
CreateDialog() и CreateWindow().

\subsection{tracer: Перехват всех функций в отдельном модуле}
\myindex{tracer}

\myindex{x86!\Instructions!INT3}
В \tracer есть поддержка точек останова INT3, хотя и срабатывающие только один раз, но зато их можно установить на все
сразу функции в некоей DLL.

\begin{lstlisting}
--one-time-INT3-bp:somedll.dll!.*
\end{lstlisting}

Либо, поставим INT3-прерывание на все функции, имена которых начинаются с префикса \TT{xml}:

\begin{lstlisting}
--one-time-INT3-bp:somedll.dll!xml.*
\end{lstlisting}

В качестве обратной стороны медали, такие прерывания срабатывают только один раз.
Tracer покажет вызов какой-либо функции, если он случится, но только один раз.
Еще один недостаток --- увидеть аргументы функции также нельзя.

Тем не менее, эта возможность очень удобна для тех ситуаций, 
когда вы знаете что некая программа использует некую DLL,
но не знаете какие именно функции в этой DLL.

И функций много. 

\par
\myindex{Cygwin}
Например, попробуем узнать, что использует cygwin-утилита uptime:

\begin{lstlisting}
tracer -l:uptime.exe --one-time-INT3-bp:cygwin1.dll!.*
\end{lstlisting}

Так мы можем увидеть все функции из библиотеки cygwin1.dll, которые были вызваны хотя бы один раз, и откуда:

\lstinputlisting{digging_into_code/uptime_cygwin.txt}

}
\section{\IFRU{Строки}{Strings}}
\label{sec:digging_strings}

\IFRU{Очень сильно помогают отладочные сообщения, если они имеются. В некотором смысле, отладочные сообщения, 
это отчет о том, что сейчас происходит в программе. Зачастую, это \printf-подобные функции, 
которые пишут куда-нибудь в лог, а бывает так что и не пишут ничего, но вызовы остались, так как эта сборка ~--- 
не отладочная, а release.}
{Debugging messages are often very helpful if present.
In some sense, debugging messages are reporting
about what's going on in program right now. Often these are \printf-like functions,
which writes to log-files, and sometimes, not writing anything but calls are still present 
since this build is not a debug build but release one.}
\index{\oracle}
\IFRU{Если в отладочных сообщениях дампятся значения некоторых локальных или глобальных переменных, 
это тоже может помочь, как минимум, узнать их имена. 
Например, в \oracle одна из таких функций: \TT{ksdwrt()}.}
{If local or global variables are dumped in debugging messages, it might be helpful as well 
since it is possible to get variable names at least.
For example, one of such functions in \oracle is \TT{ksdwrt()}.}

\newcommand{\CONUSONE}{http://blog.yurichev.com/node/32}
\newcommand{\CONUSTWO}{http://blog.yurichev.com/node/43}

\IFRU{Осмысленные текстовые строки вообще очень сильно могут помочь. 
Дизассемблер \IDA может сразу указать, из какой функции и из какого её места используется эта строка. 
Встречаются и \href{\CONUSONE}{смешные случаи}.}
{Meaningful text strings are often helpful.
\IDA disassembler may show from which function and from which point this specific string is used.
Funny cases \href{\CONUSONE}{sometimes happen}.}

\IFRU{Сообщения об ошибках также могут помочь найти то что нужно. 
В \oracle сигнализация об ошибках проходит при помощи вызова некоторой группы функций. 
\href{\CONUSTWO}{Тут еще немного об этом}.}
{Error messages may help us as well.
In \oracle, errors are reporting using group of functions.
\href{\CONUSTWO}{More about it}.}

\index{Error messages}
\IFRU{Можно довольно быстро найти, какие функции сообщают о каких ошибках, и при каких условиях.}
{It is possible to find very quickly, which functions reporting about errors and in which conditions.}
\IFRU{Это, кстати, одна из причин, почему в защите софта от копирования, 
бывает так, что сообщение об ошибке заменяется 
невнятным кодом или номером ошибки. Мало кому приятно, если взломщик быстро поймет, 
из за чего именно срабатывает защита от копирования, просто по сообщению об ошибке.}
{By the way, it is often a reason why copy-protection systems has inarticulate cryptic error messages 
or just error numbers. No one happy when software cracker quickly understand why copy-protection
is triggered just by error message.}

% TODO software protection... set ref to section about dongle for SCO UNIX...

\chapter{\RU{Вызовы assert()}\EN{Calls to assert()}}
\index{\CStandardLibrary!assert()}
\RU{Может также помочь наличие \TT{assert()} в коде: обычно этот макрос оставляет название файла-исходника, 
номер строки, и условие.}
\EN{Sometimes \TT{assert()} macro presence is useful too: 
commonly this macro leaves source file name, line number and condition in code.}

\RU{Наиболее полезная информация содержится в assert-условии, по нему можно судить по именам переменных
или именам полей структур. Другая полезная информация ~--- это имена файлов, по их именам можно попытаться
предположить, что там за код. Также, по именам файлов можно опознать какую-либо очень известную опен-сорсную
библиотеку.}
\EN{Most useful information is contained in assert-condition, we can deduce variable names, or structure field
names from it. Another useful piece of information is file names~---we can try to deduce what type of
code is here.
Also by file names it is possible to recognize a well-known open-source libraries.}

\lstinputlisting[caption=\RU{Пример информативных вызовов assert()}
\EN{Example of informative assert() calls}]{digging_into_code/assert_examples.lst}

\RU{Полезно ``гуглить'' и условия и имена файлов, это может вывести вас к опен-сорсной бибилотеке.
Например, если ``погуглить'' ``sp->lzw\_nbits <= BITS\_MAX'', 
это вполне предсказуемо выводит на опенсорсный код, что-то связанное с LZW-компрессией.}
\EN{It is advisable to ``google'' both conditions and file names, that may lead us to open-source library.
For example, if to ``google'' ``sp->lzw\_nbits <= BITS\_MAX'', this predictably 
give us some open-source code, something related to LZW-compression.}

\EN{\section{Constants}

Humans, including programmers, often use round numbers like 10, 100, 1000, 
in real life as well as in the code.

The practicing reverse engineer usually know them well in hexadecimal representation:
0b10=0xA, 0b100=0x64, 0b1000=0x3E8, 0b10000=0x2710.

The constants \TT{0xAAAAAAAA} (0b10101010101010101010101010101010) and \\
\TT{0x55555555} (0b01010101010101010101010101010101)  are also popular---those
are composed of alternating bits.

That may help to distinguish some signal from a signal where all bits are turned on (0b1111 \dots) or off (0b0000 \dots).
For example, the \TT{0x55AA} constant
is used at least in the boot sector, \ac{MBR}, 
and in the \ac{ROM} of IBM-compatible extension cards.

Some algorithms, especially cryptographical ones use distinct constants, which are easy to find
in code using \IDA.

\myindex{MD5}
\newcommand{\URLMD}{http://go.yurichev.com/17111}

For example, the MD5\footnote{\href{\URLMD}{wikipedia}} algorithm initializes its own internal variables like this:

\begin{verbatim}
var int h0 := 0x67452301
var int h1 := 0xEFCDAB89
var int h2 := 0x98BADCFE
var int h3 := 0x10325476
\end{verbatim}

If you find these four constants used in the code in a row, it is highly probable that this function is related to MD5.

\par Another example are the CRC16/CRC32 algorithms, 
whose calculation algorithms often use precomputed tables like this one:

\begin{lstlisting}[caption=linux/lib/crc16.c,style=customc]
/** CRC table for the CRC-16. The poly is 0x8005 (x^16 + x^15 + x^2 + 1) */
u16 const crc16_table[256] = {
	0x0000, 0xC0C1, 0xC181, 0x0140, 0xC301, 0x03C0, 0x0280, 0xC241,
	0xC601, 0x06C0, 0x0780, 0xC741, 0x0500, 0xC5C1, 0xC481, 0x0440,
	0xCC01, 0x0CC0, 0x0D80, 0xCD41, 0x0F00, 0xCFC1, 0xCE81, 0x0E40,
	...
\end{lstlisting}

See also the precomputed table for CRC32: \myref{sec:CRC32}.

In tableless CRC algorithms well-known polynomials are used, for example, 0xEDB88320 for CRC32.

\subsection{Magic numbers}
\label{magic_numbers}

\newcommand{\FNURLMAGIC}{\footnote{\href{http://go.yurichev.com/17112}{wikipedia}}}

A lot of file formats define a standard file header where a \IT{magic number(s)}\FNURLMAGIC{} is used, single one or even several.

\myindex{MS-DOS}

For example, all Win32 and MS-DOS executables start with the two characters \q{MZ}\footnote{\href{http://go.yurichev.com/17113}{wikipedia}}.

\myindex{MIDI}

At the beginning of a MIDI file the \q{MThd} signature must be present. 
If we have a program which uses MIDI files for something,
it's very likely that it must check the file for validity by checking at least the first 4 bytes.

This could be done like this:
(\IT{buf} points to the beginning of the loaded file in memory)

\begin{lstlisting}[style=customasmx86]
cmp [buf], 0x6468544D ; "MThd"
jnz _error_not_a_MIDI_file
\end{lstlisting}

\myindex{\CStandardLibrary!memcmp()}
\myindex{x86!\Instructions!CMPSB}

\dots or by calling a function for comparing memory blocks like \TT{memcmp()} or any other equivalent code
up to a \TT{CMPSB} (\myref{REPE_CMPSx}) instruction.

When you find such point you already can say where the loading of the MIDI file starts,
also, we could see the location
of the buffer with the contents of the MIDI file, what is used from the buffer, and how.

\subsubsection{Dates}

\myindex{UFS2}
\myindex{FreeBSD}
\myindex{HASP}

Often, one may encounter number like \TT{0x19870116}, which is clearly looks like a date (year 1987, 1th month (January), 16th day).
This may be someone's birthday (a programmer, his/her relative, child), or some other important date.
The date may also be written in a reverse order, like \TT{0x16011987}.
American-style dates are also popular, like \TT{0x01161987}.

Well-known example is \TT{0x19540119} (magic number used in UFS2 superblock structure), which is a birthday of Marshall Kirk McKusick, prominent FreeBSD contributor.

\myindex{Stuxnet}
Stuxnet uses the number ``19790509'' (not as 32-bit number, but as string, though), and this led to speculation
that the malware is connected to Israel
\footnote{This is a date of execution of Habib Elghanian, persian jew.}

Also, numbers like those are very popular in amateur-grade cryptography, for example, excerpt from the \IT{secret function} internals from HASP3 dongle
\footnote{\url{https://web.archive.org/web/20160311231616/http://www.woodmann.com/fravia/bayu3.htm}}:

\begin{lstlisting}[style=customc]
void xor_pwd(void) 
{ 
	int i; 
	
	pwd^=0x09071966;
	for(i=0;i<8;i++) 
	{ 
		al_buf[i]= pwd & 7; pwd = pwd >> 3; 
	} 
};

void emulate_func2(unsigned short seed)
{ 
	int i, j; 
	for(i=0;i<8;i++) 
	{ 
		ch[i] = 0; 
		
		for(j=0;j<8;j++)
		{ 
			seed *= 0x1989; 
			seed += 5; 
			ch[i] |= (tab[(seed>>9)&0x3f]) << (7-j); 
		}
	} 
}
\end{lstlisting}

\subsubsection{DHCP}

This applies to network protocols as well.
For example, the DHCP protocol's network packets contains the so-called \IT{magic cookie}: \TT{0x63538263}.
Any code that generates DHCP packets somewhere must embed this constant into the packet.
If we find it in the code we may find where this happens and, not only that.
Any program which can receive DHCP packet must verify the \IT{magic cookie}, comparing it with the constant.

For example, let's take the dhcpcore.dll file from Windows 7 x64 and search for the constant.
And we can find it, twice:
it seems that the constant is used in two functions with descriptive names\\
\TT{DhcpExtractOptionsForValidation()} and \TT{DhcpExtractFullOptions()}:

\begin{lstlisting}[caption=dhcpcore.dll (Windows 7 x64),style=customasmx86]
.rdata:000007FF6483CBE8 dword_7FF6483CBE8 dd 63538263h          ; DATA XREF: DhcpExtractOptionsForValidation+79
.rdata:000007FF6483CBEC dword_7FF6483CBEC dd 63538263h          ; DATA XREF: DhcpExtractFullOptions+97
\end{lstlisting}

And here are the places where these constants are accessed:

\begin{lstlisting}[caption=dhcpcore.dll (Windows 7 x64),style=customasmx86]
.text:000007FF6480875F  mov     eax, [rsi]
.text:000007FF64808761  cmp     eax, cs:dword_7FF6483CBE8
.text:000007FF64808767  jnz     loc_7FF64817179
\end{lstlisting}

And:

\begin{lstlisting}[caption=dhcpcore.dll (Windows 7 x64),style=customasmx86]
.text:000007FF648082C7  mov     eax, [r12]
.text:000007FF648082CB  cmp     eax, cs:dword_7FF6483CBEC
.text:000007FF648082D1  jnz     loc_7FF648173AF
\end{lstlisting}

\subsection{Specific constants}

Sometimes, there is a specific constant for some type of code.
For example, the author once dug into a code, where number 12 was encountered suspiciously often.
Size of many arrays is 12, or multiple of 12 (24, etc).
As it turned out, that code takes 12-channel audio file at input and process it.

And vice versa: for example, if a program works with text field which has length of 120 bytes,
there has to be a constant 120 or 119 somewhere in the code.
If UTF-16 is used, then $2 \cdot 120$.
If a code works with network packets of fixed size, it's good idea to search for this constant in the code as well.

This is also true for amateur cryptography (license keys, etc).
If encrypted block has size of $n$ bytes, you may want to try to find occurences of this number throughout the code.
Also, if you see a piece of code which is been repeated $n$ times in loop during execution,
this may be encryption/decryption routine.

\subsection{Searching for constants}

It is easy in \IDA: Alt-B or Alt-I.
\myindex{binary grep}
And for searching for a constant in a big pile of files, or for searching in non-executable files,
there is a small utility called \IT{binary grep}\footnote{\BGREPURL}.

}
\RU{\section{Константы}

Люди, включая программистов, часто используют круглые числа вроде 10, 100, 1000, в т.ч. и в коде.

Практикующие реверсеры, обычно, хорошо знают их в шестнадцатеричном представлении:
0b10=0xA, 0b100=0x64, 0b1000=0x3E8, 0b10000=0x2710.

Иногда попадаются константы \TT{0xAAAAAAAA} \\
(0b10101010101010101010101010101010) и
\TT{0x55555555} (0b01010101010101010101010101010101) --- это чередующиеся биты.
Это помогает отличить некоторый сигнал от сигнала где все биты включены (0b1111 \dots) или выключены (0b0000 \dots).

Например, константа \TT{0x55AA} используется как минимум в бут-секторе, \ac{MBR}, 
и в \ac{ROM} плат-расширений IBM-компьютеров.

Некоторые алгоритмы, особенно криптографические, используют хорошо различимые константы, 
которые при помощи \IDA легко находить в коде.

\myindex{MD5}
\newcommand{\URLMD}{http://go.yurichev.com/17110}

Например, алгоритм MD5\footnote{\href{\URLMD}{wikipedia}} инициализирует свои внутренние переменные так:

\begin{verbatim}
var int h0 := 0x67452301
var int h1 := 0xEFCDAB89
var int h2 := 0x98BADCFE
var int h3 := 0x10325476
\end{verbatim}

Если в коде найти использование этих четырех констант подряд --- очень высокая вероятность что эта функция имеет отношение к MD5.

\par
Еще такой пример это алгоритмы CRC16/CRC32, часто, алгоритмы вычисления контрольной суммы по CRC 
используют заранее заполненные таблицы, вроде:

\begin{lstlisting}[caption=linux/lib/crc16.c,style=customc]
/** CRC table for the CRC-16. The poly is 0x8005 (x^16 + x^15 + x^2 + 1) */
u16 const crc16_table[256] = {
	0x0000, 0xC0C1, 0xC181, 0x0140, 0xC301, 0x03C0, 0x0280, 0xC241,
	0xC601, 0x06C0, 0x0780, 0xC741, 0x0500, 0xC5C1, 0xC481, 0x0440,
	0xCC01, 0x0CC0, 0x0D80, 0xCD41, 0x0F00, 0xCFC1, 0xCE81, 0x0E40,
	...
\end{lstlisting}

См. также таблицу CRC32: \myref{sec:CRC32}.

В бестабличных алгоритмах CRC используются хорошо известные полиномы, например 0xEDB88320 для CRC32.

\subsection{Магические числа}
\label{magic_numbers}

\newcommand{\FNURLMAGIC}{\footnote{\href{http://go.yurichev.com/17112}{wikipedia}}}

Немало форматов файлов определяет стандартный заголовок файла где используются \IT{магическое число} (magic number)\FNURLMAGIC{}, один или даже несколько.

\myindex{MS-DOS}
Скажем, все исполняемые файлы для Win32 и MS-DOS начинаются с двух символов \q{MZ}\footnote{\href{http://go.yurichev.com/17113}{wikipedia}}.

\myindex{MIDI}
В начале MIDI-файла должно быть \q{MThd}. Если у нас есть использующая для чего-нибудь MIDI-файлы программа,
наверняка она будет проверять MIDI-файлы на правильность хотя бы проверяя первые 4 байта.

Это можно сделать при помощи:
(\IT{buf} указывает на начало загруженного в память файла)

\begin{lstlisting}[style=customasmx86]
cmp [buf], 0x6468544D ; "MThd"
jnz _error_not_a_MIDI_file
\end{lstlisting}

\myindex{\CStandardLibrary!memcmp()}
\myindex{x86!\Instructions!CMPSB}
\dots либо вызвав функцию сравнения блоков памяти \TT{memcmp()} или любой аналогичный код, 
вплоть до инструкции \TT{CMPSB} (\myref{REPE_CMPSx}).

Найдя такое место мы получаем как минимум информацию о том, где начинается загрузка MIDI-файла, во-вторых, 
мы можем увидеть где располагается буфер с содержимым файла, и что еще оттуда берется, и как используется.

\subsubsection{Даты}

\myindex{UFS2}
\myindex{FreeBSD}
\myindex{HASP}

Часто, можно встретить число вроде \TT{0x19870116}, которое явно выглядит как дата (1987-й год, 1-й месяц (январь), 16-й день).
Это может быть чей-то день рождения (программиста, его/её родственника, ребенка), либо какая-то другая важная дата.
Дата может быть записана и в другом порядке, например \TT{0x16011987}.
Даты в американском стиле также популярны, например \TT{0x01161987}.

Известный пример это \TT{0x19540119} (магическое число используемое в структуре суперблока UFS2), это день рождения Маршала Кирка МакКузика, видного разработчика FreeBSD.

\myindex{Stuxnet}
В Stuxnet используется число ``19790509'' (хотя и не как 32-битное число, а как строка), и это привело к догадкам,
что этот зловред связан с Израелем\footnote{Это дата казни персидского еврея Habib Elghanian-а}.

Также, числа вроде таких очень популярны в любительской криптографии, например, это отрывок из внутренностей \IT{секретной функции} донглы HASP3
\footnote{\url{https://web.archive.org/web/20160311231616/http://www.woodmann.com/fravia/bayu3.htm}}:

\begin{lstlisting}[style=customc]
void xor_pwd(void) 
{ 
	int i; 
	
	pwd^=0x09071966;
	for(i=0;i<8;i++) 
	{ 
		al_buf[i]= pwd & 7; pwd = pwd >> 3; 
	} 
};

void emulate_func2(unsigned short seed)
{ 
	int i, j; 
	for(i=0;i<8;i++) 
	{ 
		ch[i] = 0; 
		
		for(j=0;j<8;j++)
		{ 
			seed *= 0x1989; 
			seed += 5; 
			ch[i] |= (tab[(seed>>9)&0x3f]) << (7-j); 
		}
	} 
}
\end{lstlisting}

\subsubsection{DHCP}

Это касается также и сетевых протоколов. 
Например, сетевые пакеты протокола DHCP содержат так называемую \IT{magic cookie}: \TT{0x63538263}. 
Какой-либо код, генерирующий пакеты по протоколу DHCP где-то и как-то должен внедрять в пакет также и эту константу. 
Найдя её в коде мы сможем найти место где происходит это и не только это. 
Любая программа, получающая DHCP-пакеты, должна где-то как-то проверять \IT{magic cookie}, 
сравнивая это поле пакета с константой.

Например, берем файл dhcpcore.dll из Windows 7 x64 и ищем эту константу. 
И находим, два раза: оказывается, эта константа используется в функциях с красноречивыми названиями \\
\TT{DhcpExtractOptionsForValidation()} и \TT{DhcpExtractFullOptions()}:

\begin{lstlisting}[caption=dhcpcore.dll (Windows 7 x64),style=customasmx86]
.rdata:000007FF6483CBE8 dword_7FF6483CBE8 dd 63538263h          ; DATA XREF: DhcpExtractOptionsForValidation+79
.rdata:000007FF6483CBEC dword_7FF6483CBEC dd 63538263h          ; DATA XREF: DhcpExtractFullOptions+97
\end{lstlisting}

А вот те места в функциях где происходит обращение к константам:

\begin{lstlisting}[caption=dhcpcore.dll (Windows 7 x64),style=customasmx86]
.text:000007FF6480875F  mov     eax, [rsi]
.text:000007FF64808761  cmp     eax, cs:dword_7FF6483CBE8
.text:000007FF64808767  jnz     loc_7FF64817179
\end{lstlisting}

И:

\begin{lstlisting}[caption=dhcpcore.dll (Windows 7 x64),style=customasmx86]
.text:000007FF648082C7  mov     eax, [r12]
.text:000007FF648082CB  cmp     eax, cs:dword_7FF6483CBEC
.text:000007FF648082D1  jnz     loc_7FF648173AF
\end{lstlisting}

\subsection{Специфические константы}

Иногда, бывают какие-то специфические константы для некоторого типа кода.
Например, однажды автор сих строк пытался разобраться с кодом, где подозрительно часто встречалось число 12.
Размеры многих массивов также были 12, или кратные 12 (24, итд).
Оказалось, этот код брал на вход 12-канальный аудиофайл и обрабатывал его.

И наоборот: например, если программа работает с текстовым полем длиной 120 байт, значит где-то в коде должна
быть константа 120, или 119.
Если используется UTF-16, то тогда $2 \cdot 120$.
Если код работает с сетевыми пакетами фиксированной длины, то хорошо бы и такую константу поискать в коде.

Это также справедливо для любительской криптографии (ключи с лицензией, итд).
Если зашифрованный блок имеет размер в $n$ байт, вы можете попробовать поискать это число в коде.
Также, если вы видите фрагмент кода, который при исполнении, повторяется $n$ раз в цикле,
это может быть ф-ция шифрования/дешифрования.

\subsection{Поиск констант}

В \IDA это очень просто, Alt-B или Alt-I.

\myindex{binary grep}
А для поиска константы в большом количестве файлов, либо для поиска их в неисполняемых файлах, имеется небольшая утилита
\IT{binary grep}\footnote{\BGREPURL}.

}
\chapter{\RU{Поиск нужных инструкций}\EN{Finding the right instructions}}

\RU{Если программа использует инструкции сопроцессора, и их не очень много, 
то можно попробовать вручную проверить отладчиком какую-то из них.}
\EN{If the program is utilizing FPU instructions and there are very few of them in the code,
one can try to check each one manually with a debugger.}\PTBRph{}\ESph{}\PLph{}\ITAph{}\\
\\
\RU{К примеру, нас может заинтересовать, при помощи чего Microsoft Excel считает 
результаты формул, введенных пользователем. Например, операция деления.}
\EN{For example, we may be interested how Microsoft Excel calculates the formulae entered by user.
For example, the division operation.}

\index{\GrepUsage}
\index{x86!\Instructions!FDIV}
\RU{Если загрузить excel.exe (из Office 2010) версии 14.0.4756.1000 в \IDA, затем сделать полный листинг 
и найти все инструкции \FDIV (но кроме тех, которые в качестве второго операнда используют константы\EMDASH{}они, 
очевидно, не подходят нам):}
\EN{If we load excel.exe (from Office 2010) version 14.0.4756.1000 into \IDA, make a full listing
and to find every \FDIV instruction (except the ones which use constants as a second 
operand\EMDASH{}obviously, they do not suit us):}\PTBRph{}\ESph{}\PLph{}\ITAph{}\\

\begin{lstlisting}
cat EXCEL.lst | grep fdiv | grep -v dbl_ > EXCEL.fdiv
\end{lstlisting}

\RU{\dots то окажется, что их всего 144.}\EN{\dots then we see that there are 144 of them.}\PTBRph{}\ESph{}\PLph{}\ITAph{}\\
\\
\RU{Мы можем вводить в Excel строку вроде \TT{=(1/3)} и проверить все эти инструкции.}
\EN{We can enter a string like \TT{=(1/3)} in Excel and check each instruction.}\PTBRph{}\ESph{}\PLph{}\ITAph{}\\
\\
\index{tracer}
\RU{Проверяя каждую инструкцию в отладчике или \tracer 
(проверять эти инструкции можно по 4 за раз), 
окажется, что нам везет и срабатывает всего лишь 14-я по счету:}
\EN{By checking each instruction in a debugger or \tracer
(one may check 4 instruction at a time),
we get lucky and the sought-for instruction is just the 14th:}

\begin{lstlisting}
.text:3011E919 DC 33                                fdiv    qword ptr [ebx]
\end{lstlisting}

\begin{lstlisting}
PID=13944|TID=28744|(0) 0x2f64e919 (Excel.exe!BASE+0x11e919)
EAX=0x02088006 EBX=0x02088018 ECX=0x00000001 EDX=0x00000001
ESI=0x02088000 EDI=0x00544804 EBP=0x0274FA3C ESP=0x0274F9F8
EIP=0x2F64E919
FLAGS=PF IF
FPU ControlWord=IC RC=NEAR PC=64bits PM UM OM ZM DM IM 
FPU StatusWord=
FPU ST(0): 1.000000
\end{lstlisting}

\RU{В \ST{0} содержится первый аргумент (1), второй содержится в}
\EN{\ST{0} holds the first argument (1) and second one is in} \TT{[EBX]}.\\
\\
\index{x86!\Instructions!FDIV}
\RU{Следующая за \FDIV инструкция (\TT{FSTP}) записывает результат в память:}
\EN{The instruction after \FDIV (\TT{FSTP}) writes the result in memory:}\\

\begin{lstlisting}
.text:3011E91B DD 1E                                fstp    qword ptr [esi]
\end{lstlisting}

\RU{Если поставить breakpoint на ней, то мы можем видеть результат:}
\EN{If we set a breakpoint on it, we can see the result:}

\begin{lstlisting}
PID=32852|TID=36488|(0) 0x2f40e91b (Excel.exe!BASE+0x11e91b)
EAX=0x00598006 EBX=0x00598018 ECX=0x00000001 EDX=0x00000001
ESI=0x00598000 EDI=0x00294804 EBP=0x026CF93C ESP=0x026CF8F8
EIP=0x2F40E91B
FLAGS=PF IF
FPU ControlWord=IC RC=NEAR PC=64bits PM UM OM ZM DM IM 
FPU StatusWord=C1 P 
FPU ST(0): 0.333333
\end{lstlisting}

\RU{А также, в рамках пранка\footnote{practical joke}, модифицировать его на лету:}
\EN{Also as a practical joke, we can modify it on the fly:}\PTBRph{}\ESph{}\PLph{}\ITAph{}\\

\begin{lstlisting}
tracer -l:excel.exe bpx=excel.exe!BASE+0x11E91B,set(st0,666)
\end{lstlisting}

\begin{lstlisting}
PID=36540|TID=24056|(0) 0x2f40e91b (Excel.exe!BASE+0x11e91b)
EAX=0x00680006 EBX=0x00680018 ECX=0x00000001 EDX=0x00000001
ESI=0x00680000 EDI=0x00395404 EBP=0x0290FD9C ESP=0x0290FD58
EIP=0x2F40E91B
FLAGS=PF IF
FPU ControlWord=IC RC=NEAR PC=64bits PM UM OM ZM DM IM 
FPU StatusWord=C1 P 
FPU ST(0): 0.333333
Set ST0 register to 666.000000
\end{lstlisting}

\RU{Excel показывает в этой ячейке 666, что окончательно убеждает нас в том, что мы нашли нужное место.}
\EN{Excel shows 666 in the cell, finally convincing us that we have found the right point.}

\begin{figure}[H]
\centering
\includegraphics[scale=\NormalScale]{digging_into_code/Excel_prank.png}
\caption{\RU{Пранк сработал}\EN{The practical joke worked}}
\end{figure}

\RU{Если попробовать ту же версию Excel, только x64, то окажется что там инструкций \FDIV всего 12, 
причем нужная нам\EMDASH{}третья по счету.}
\EN{If we try the same Excel version, but in x64,
we will find only 12 \FDIV instructions there,
and the one we looking for is the third one.}

\begin{lstlisting}
tracer.exe -l:excel.exe bpx=excel.exe!BASE+0x1B7FCC,set(st0,666)
\end{lstlisting}

\index{x86!\Instructions!DIVSD}
\RU{Видимо, все дело в том, что много операций деления переменных типов \Tfloat и \Tdouble 
компилятор заменил на SSE-инструкции вроде \TT{DIVSD}, 
коих здесь теперь действительно много (\TT{DIVSD} присутствует в количестве 268 инструкций).}
\EN{It seems that a lot of division operations of \Tfloat and \Tdouble types, were replaced by the compiler with SSE instructions
like \TT{DIVSD} (\TT{DIVSD} is present 268 times in total).}

\chapter{\RU{Подозрительные паттерны кода}\EN{Suspicious code patterns}}

\section{\RU{Инструкции XOR}\EN{XOR instructions}}
\index{x86!\Instructions!XOR}

\RU{Инструкции вроде}\EN{Instructions like} \TT{XOR op, op} (\RU{например}\EN{for example}, \TT{XOR EAX, EAX}) 
\RU{обычно используются для обнуления регистра,
однако, если операнды разные, то применяется операция именно}\EN{are usually used for setting the register value
to zero, but if the operands are different, the} \q{\RU{исключающего или}\EN{exclusive or}}\EN{ operation
is executed}.
\RU{Эта операция очень редко применяется в обычном программировании, но применяется очень часто в криптографии,
включая любительскую.}
\EN{This operation is rare in common programming, but widespread in cryptography,
including amateur one.}
\RU{Особенно подозрительно, если второй операнд\EMDASH{}это большое число}\EN{It's especially suspicious if the
second operand is a big number}.
\RU{Это может указывать на шифрование, вычисление контрольной суммы,}
\EN{This may point to encrypting/decrypting, checksum computing,}\etc{}.\\
\\
\ifx\LITE\undefined
\RU{Одно из исключений из этого наблюдения о котором стоит сказать, то, что генерация и проверка значения \q{канарейки}
(\myref{subsec:BO_protection}) часто происходит, используя инструкцию \XOR.}
\EN{One exception to this observation worth noting is the \q{canary} (\myref{subsec:BO_protection}). 
Its generation and checking are often done using the \XOR instruction.} \\
\\
\fi
\index{AWK}
\RU{Этот AWK-скрипт можно использовать для обработки листингов (.lst) созданных \IDA{}}
\EN{This AWK script can be used for processing \IDA{} listing (.lst) files}:

\begin{lstlisting}
gawk -e '$2=="xor" { tmp=substr($3, 0, length($3)-1); if (tmp!=$4) if($4!="esp") if ($4!="ebp") { print $1, $2, tmp, ",", $4 } }' filename.lst
\end{lstlisting}

\ifx\LITE\undefined
\RU{Нельзя также забывать,
что если использовать подобный скрипт, то, возможно, он захватит и неверно дизассемблированный
код}\EN{It is also worth noting that this kind of script can also match incorrectly disassembled code} 
(\myref{sec:incorrectly_disasmed_code}).
\fi

\section{\RU{Вручную написанный код на ассемблере}\EN{Hand-written assembly code}}

\index{Function prologue}
\index{Function epilogue}
\index{x86!\Instructions!LOOP}
\index{x86!\Instructions!RCL}
\RU{Современные компиляторы не генерируют инструкции \TT{LOOP} и \TT{RCL}. 
С другой стороны, эти инструкции хорошо знакомы кодерам, предпочитающим писать прямо на ассемблере. 
\ifx\LITE\undefined
Подобные инструкции отмечены как (M) в списке инструкций в приложении: 
\myref{sec:x86_instructions}.
\fi
Если такие инструкции встретились, можно сказать с какой-то вероятностью, что этот фрагмент кода написан вручную.}
\EN{Modern compilers do not emit the \TT{LOOP} and \TT{RCL} instructions.
On the other hand, these instructions are well-known to coders who like to code directly in assembly language.
If you spot these, it can be said that there is a high probability that this fragment of code was hand-written.
\ifx\LITE\undefined
Such instructions are marked as (M) in the instructions list in this appendix: 
\myref{sec:x86_instructions}.
\fi
}\PTBRph{}\ESph{}\PLph{}\ITAph{}\\
\\
\RU{Также, пролог/эпилог функции обычно не встречается в ассемблерном коде, написанном вручную.}
\EN{Also the function prologue/epilogue are not commonly present in hand-written assembly.}\\
\\
\RU{Как правило, в вручную написанном коде, нет никакого четкого метода передачи аргументов в 
функцию}
\EN{Commonly there is no fixed system for passing arguments to functions in the hand-written
code}.\\
\\
\RU{Пример из ядра}\EN{Example from the} Windows 2003\EN{ kernel} 
(\RU{файл }ntoskrnl.exe\EN{ file}):

\begin{lstlisting}
MultiplyTest proc near               ; CODE XREF: Get386Stepping
             xor     cx, cx
loc_620555:                          ; CODE XREF: MultiplyTest+E
             push    cx
             call    Multiply
             pop     cx
             jb      short locret_620563
             loop    loc_620555
             clc
locret_620563:                       ; CODE XREF: MultiplyTest+C
             retn
MultiplyTest endp

Multiply     proc near               ; CODE XREF: MultiplyTest+5
             mov     ecx, 81h
             mov     eax, 417A000h
             mul     ecx
             cmp     edx, 2
             stc
             jnz     short locret_62057F
             cmp     eax, 0FE7A000h
             stc
             jnz     short locret_62057F
             clc
locret_62057F:                       ; CODE XREF: Multiply+10
                                     ; Multiply+18
             retn
Multiply     endp
\end{lstlisting}

\RU{Действительно, если заглянуть в исходные коды}\EN{Indeed, if we look in the} 
\ac{WRK} v1.2\RU{, данный код можно найти в файле}\EN{ source code, this code
can be found easily in file} 
\IT{WRK-v1.2\textbackslash{}base\textbackslash{}ntos\textbackslash{}ke\textbackslash{}i386\textbackslash{}cpu.asm}.

\chapter{\RU{Использование magic numbers для трассировки}\EN{Using magic numbers while tracing}}

\RU{Нередко бывает нужно узнать, как используется то или иное значение, прочитанное из файла либо взятое из пакета,
принятого по сети. Часто, ручное слежение за нужной переменной это трудный процесс. Один из простых методов (хотя и не
полностью надежный на 100\%) это использование вашей собственной \IT{magic number}.}
\EN{Often, our main goal is to understand how the program uses a value that was either read from file or received via network. 
The manual tracing of a value is often a very labour-intensive task. One of the simplest techniques for this (although not 100\% reliable) 
is to use your own \IT{magic number}.}

\RU{Это чем-то напоминает компьютерную томографию: пациенту перед сканированием вводят в кровь 
рентгеноконтрастный препарат, хорошо отсвечивающий в рентгеновских лучах.
Известно, как кровь нормального человека
расходится, например, по почкам, и если в этой крови будет препарат, то при томографии будет хорошо видно,
достаточно ли хорошо кровь расходится по почкам и нет ли там камней, например, и прочих образований.}
\EN{This resembles X-ray computed tomography is some sense: a radiocontrast agent is injected into the patient's blood,
which is then used to improve the visibility of the patient's internal structure in to the X-rays.
It is well known how the blood of healthy humans
percolates in the kidneys and if the agent is in the blood, it can be easily seen on tomography, how blood is percolating,
and are there any stones or tumors.}

\RU{Мы можем взять 32-битное число вроде \TT{0x0badf00d}, либо чью-то дату рождения вроде \TT{0x11101979} 
и записать это, занимающее 4 байта число, в какое-либо место файла используемого исследуемой нами программой.}
\EN{We can take a 32-bit number like \TT{0x0badf00d}, or someone's birth date like \TT{0x11101979}
and write this 4-byte number to some point in a file used by the program we investigate.}

\index{\GrepUsage}
\index{tracer}
\RU{Затем, при трассировки этой программы, в том числе, при помощи \tracer в режиме 
\IT{code coverage}, а затем при помощи
\IT{grep} или простого поиска по текстовому файлу с результатами трассировки, мы можем легко увидеть, в каких местах кода использовалось 
это значение, и как.}
\EN{Then, while tracing this program with \tracer in \IT{code coverage} mode, with the help of \IT{grep}
or just by searching in the text file (of tracing results), we can easily see where the value was used and how.}

\RU{Пример результата работы \tracer в режиме \IT{cc}, к которому легко применить утилиту \IT{grep}}\EN{Example 
of \IT{grepable} \tracer results in \IT{cc} mode}:

\begin{lstlisting}
0x150bf66 (_kziaia+0x14), e=       1 [MOV EBX, [EBP+8]] [EBP+8]=0xf59c934 
0x150bf69 (_kziaia+0x17), e=       1 [MOV EDX, [69AEB08h]] [69AEB08h]=0 
0x150bf6f (_kziaia+0x1d), e=       1 [FS: MOV EAX, [2Ch]] 
0x150bf75 (_kziaia+0x23), e=       1 [MOV ECX, [EAX+EDX*4]] [EAX+EDX*4]=0xf1ac360 
0x150bf78 (_kziaia+0x26), e=       1 [MOV [EBP-4], ECX] ECX=0xf1ac360 
\end{lstlisting}
% TODO: good example!
\RU{Это справедливо также и для сетевых пакетов.
Важно только, чтобы наш \IT{magic number} был как можно более уникален и не присутствовал в самом коде.}
\EN{This can be used for network packets as well.
It is important for the \IT{magic number} to be unique and not to be present in the program's code.}

\newcommand{\DOSBOXURL}{\href{http://go.yurichev.com/17222}{blog.yurichev.com}}

\index{DosBox}
\index{MS-DOS}
\RU{Помимо \tracer, такой эмулятор MS-DOS как DosBox, в режиме heavydebug, может писать в отчет информацию обо всех
состояниях регистра на каждом шаге исполнения программы\footnote{См. также мой пост в блоге об этой возможности в 
DosBox: \DOSBOXURL{}}, так что этот метод может пригодиться и для исследования программ под DOS.}\EN{Aside of 
the \tracer, DosBox (MS-DOS emulator) in heavydebug mode
is able to write information about all registers' states for each executed instruction of the program to a plain text file\footnote{See also my 
blog post about this DosBox feature: \DOSBOXURL{}}, so this technique may be useful for DOS programs as well.}



\section{\RU{Прочее}\EN{Other things}}

\subsection{\EN{General idea}\RU{Общая идея}}

\RU{Нужно стараться как можно чаще ставить себя на место программиста и задавать себе вопрос, 
как бы вы сделали ту или иную вещь в этом случае и в этой программе.}
\EN{A reverse engineer should try to be in programmer's shoes as often as possible. 
To take his/her viewpoint and ask himself, how would one solve some task the specific case.}

\subsection{\RU{Порядок функций в бинарном коде}\EN{Order of functions in binary code}}

\RU{Все функции расположеные в одном .c или .cpp файле компилируются в соответствующий объектный (.o) файл.
Линкер впоследствии складывает все нужные объектные файлы вместе, не меняя порядок ф-ций в них.
Как следствие, если вы видите в коде две или более идущих подряд ф-ций, то это означает, что и в исходном коде они 
были расположены в одном и том же файле (если только вы не на границе двух объектных файлов, конечно).
Это может означать, что эти ф-ции имеют что-то общее между собой, что они из одного слоя \ac{API}, из одной библиотеки, итд.}%
\EN{All functions located in a single .c or .cpp-file are compiled into corresponding object (.o) file.
Later, linker puts all object files it needs together, not changing order or functions in them.
As a consequence, if you see two or more consecutive functions, it means, that they were placed together
in a single source code file (unless you're on border of two object files, of course.)
This means these functions have something in common, that they are from the same \ac{API} level, from same library, etc.}

\subsection{\EN{Tiny functions}\RU{Крохотные функции}}

\EN{Tiny functions like empty functions (\myref{empty_func})
or function which returns just ``true'' (1) or ``false'' (0) (\myref{ret_val_func}) are very common,
and almost all decent compiler tends put only one such function into resulting executable code even if there was several
similar functions in source code.
So, whenever you see a tiny function consisting just of \TT{mov eax, 1 / ret}
which is referenced (and can be called) from many places,
which are seems unconnected to each other, this may be a result of such optimization.}%
\RU{Крохотные ф-ции, такие как пустые ф-ции (\myref{empty_func})
или ф-ции возвращающие только ``true'' (1) или ``false'' (0) (\myref{ret_val_func}) очень часто встречаются,
и почти все современные компиляторы, как правило, помещают только одну такую ф-цию в исполняемый код,
даже если в исходном их было много одинаковых.
Так что если вы видите ф-цию состояющую только из \TT{mov eax, 1 / ret}, которая может вызываться из разных мест,
которые, судя по всему, друг с другом никак не связаны, это может быть результат подобной оптимизации.}

\subsection{\Cpp}

\ac{RTTI}~(\myref{RTTI})-\RU{информация также может быть полезна для идентификации 
классов в \Cpp}\EN{data may be also useful for \Cpp class identification}.

% sections
\section{\RU{Некоторые паттерны в бинарных файлах}\EN{Some binary file patterns}}

% TODO translate

All examples here were prepared on the Windows with active code page 437 in console. % FIXME URL
Binary files internally may look visually different if another code page is set.

\subsection{Arrays}

\EN{Sometimes, we can clearly spot an array of 16/32/64-bit values visually, in hex editor.}
\RU{Иногда мы можем легко заметить массив 16/32/64-битных значений визуально, в шестнадцатеричном 
редакторе.}

% TODO translate
Вот пример массива 16-битных значений.
Мы видим что каждый первый байт в паре всегда равен 7 или 8, а второй выглядит случайным:

\begin{figure}[H]
\centering
\includegraphics[scale=\NormalScale]{digging_into_code/binary/16bit_array.png}
\caption{Hiew: \EN{массив 16-битных значений}\RU{array of 16-bit values}}
\end{figure}

% TODO translate
Для примера я использовал файл содержащий 12-канальный сигнал оцифрованный при помощи 16-битного АЦП.

\index{MIPS}
\par
\EN{And here is an example of very typical MIPS code.}
\RU{А вот пример очень типичного MIPS-кода.}
\EN{As we may remember, every MIPS (and also ARM in ARM mode or ARM64) instruction has size of 32 bits (or 4 bytes), 
so such code is array of 32-bit values.}
\RU{Как мы наверное помним, каждая инструкция в MIPS (а также в ARM в режиме ARM, или ARM64) имеет 
длину 32 бита (или 4 байта),
так что такой код это массив 32-битных значений.}
\EN{By looking at this screenshot, we may see some kind of pattern.}
\RU{Глядя на этот скриншот, можно увидеть некий узор.}
\EN{Vertical red lines are added for clarity}\RU{Вертикальные красные линии добавлены для ясности}:

\begin{figure}[H]
\centering
\includegraphics[scale=\NormalScale]{digging_into_code/binary/typical_MIPS_code.png}
\caption{Hiew: \EN{very typical MIPS code}\RU{очень типичный код для MIPS}}
\end{figure}

\ifx\LITE\undefined
\RU{Еще пример таких файлов в этой книге}\EN{Another example of such pattern here is book}: 
\myref{Oracle_SYM_files_example}.
\fi

\subsection{Sparse files}

This is sparse file with data scattered at relatively long distances.
Each space here is in fact zero byte (which is looks like space).
This is a file to program FPGA (Altera Stratix GX device).
Of course, files like these can be compressed easily, but formats like this one are very popular in scientific and engineering software where efficient access is important while compactness is not.

\begin{figure}[H]
\centering
\includegraphics[scale=\NormalScale]{digging_into_code/binary/sparse_FPGA.png}
\caption{Hiew: \EN{Sparse file}\RU{...}}
\end{figure}

\subsection{Compressed file}

This file is just some compressed archive.
It has relatively high entropy (\ref{...}) and visually looks just chaotic.
This is how compressed and/or encrypted files looks like.

\begin{figure}[H]
\centering
\includegraphics[scale=\NormalScale]{digging_into_code/binary/compressed.png}
\caption{Hiew: \EN{Compressed file}\RU{...}}
\end{figure}

\subsection{\ac{CDFS}}

\ac{OS} installations are usually distributed as ISO files which are copies of CD/DVD discs.
Filesystem used is named \ac{CDFS}, here is you see file names mixed with some additional data.
This can be file sizes, pointers to another directories, file attributes, etc.
This is how typical filesystems may look internally.

\begin{figure}[H]
\centering
\includegraphics[scale=\NormalScale]{digging_into_code/binary/cdfs.png}
\caption{Hiew: \EN{ISO file: Ubuntu 15 installation CD\RU{...}}
\end{figure}

\subsection{32-bit x86 executable code}

This is how 32-bit x86 executable code looks like.
It has not very high entropy, so some bytes occurred more often than others.

\begin{figure}[H]
\centering
\includegraphics[scale=\NormalScale]{digging_into_code/binary/x86_32.png}
\caption{Hiew: \EN{Executable 32-bit x86 code\RU{...}}
\end{figure}

Read more about x86 statistics: \ref{}. % FIXME blog post about decryption...

\subsection{BMP files}

BMP files are not compressed, so each byte (or group of bytes) describes each pixel.
I've found this picture somewhere inside my Windows 8.1 installation:

\begin{figure}[H]
\centering
\includegraphics[scale=\NormalScale]{digging_into_code/binary/bmp.png}
\caption{\EN{Example picture}\RU{...}}
\end{figure}

You see that this picture has some pixels which probably cannot be compressed very good (around center), but there are long one-color lines at top and bottom.
Indeed, lines like these also looks as lines inside:

\begin{figure}[H]
\centering
\includegraphics[scale=\NormalScale]{digging_into_code/binary/bmp_hiew.png}
\caption{Hiew: \EN{... inside}\RU{...}}
\end{figure}


\section{\RU{Сравнение ``снимков'' памяти}\EN{Memory ``snapshots'' comparing}}

\RU{Метод простого сравнения двух снимков памяти для поиска изменений часто применялся для взлома игр 
на 8-битных компьютерах и взлома файлов с записанными рекордными очками.}
\EN{The technique of straightforward two memory snapshots comparing in order to see changes, was often used to hack
8-bit computer games and hacking ``high score'' files.}

\RU{К примеру, если вы имеете загруженную игру на 8-битном компьютере (где самой памяти не очень много, но игра
занимает еще меньше), и вы знаете что сейчас у вас, условно, 100 пуль, вы можете сделать ``снимок'' всей
памяти и сохранить где-то. Затем просто стреляете куда угодно, у вас станет 99 пуль, сделать второй ``снимок'',
и затем сравнить эти два снимка: где-то наверняка должен быть байт, который в начале был 100, а затем стал 99.}
\EN{For example, if you got a loaded game on 8-bit computer (it is not much memory on these, but game is usually
consumes even less memory) and you know that you have now, let's say, 100 bullets, you can do a ``snapshot''
of all memory and back it up to some place. Then shoot somewhere, bullet count now 99, do second ``snapshot''
and then compare both: somewhere must be a byte which was 100 in the beginning and now it is 99.}
\RU{Если учесть, что игры на тех маломощных домашних компьютерах обычно были написаны на ассемблере и подобные
переменные там были глобальные, то можно с уверенностью сказать, какой адрес в памяти всегда отвечает за количество
пуль. Если поискать в дизассемблированном коде игры все обращения по этому адресу, несложно найти код,
отвечающий за уменьшение пуль и записать туда инструкцию \gls{NOP}
или несколько \gls{NOP}-в, так мы получим игру в которой у игрока всегда будет 100 пуль, например.}
\EN{Considering a fact these 8-bit games were often written in assembly language and such variables were global,
it can be said for sure, which address in memory holding bullets count. If to search all references to the
address in disassembled game code, it is not very hard to find a piece of code \glslink{decrement}{decrementing} bullets count,
write \gls{NOP} instruction there, or couple of \gls{NOP}-s, 
we'll have a game with e.g 100 bullets forever.}
\index{BASIC!POKE}
\RU{А так как игры на тех домашних 8-битных 
компьютерах всегда загружались по одним и тем же адресам, и версий одной игры редко когда было больше одной продолжительное время,
то геймеры-энтузиасты знали, по какому адресу (используя инструкцию языка BASIC \gls{POKE}) что записать после загрузки
игры, чтобы хакнуть её. Это привело к появлению списков ``читов'' состоящих из инструкций \gls{POKE}, публикуемых
в журналах посвященным 8-битным играм. См. также:}\EN{Games on these 8-bit computers was commonly loaded on the same
address, also, there were no much different versions of each game (commonly just one version was popular for a long span of time),
enthusiastic gamers knew, which byte must be written (using BASIC instruction \gls{POKE}) to which address in
order to hack it. This led to ``cheat'' lists containing of \gls{POKE} instructions published in magazines related to
8-bit games. See also:} \url{http://en.wikipedia.org/wiki/PEEK\_and\_POKE}.

\index{MS-DOS}
\RU{Точно так же легко модифицировать файлы с сохраненными рекордами (кто сколько очков набрал), впрочем, это может
сработать не только с 8-битными играми. Нужно заметить, какой у вас сейчас рекорд и где-то сохранить файл
с очками. Затем, когда очков станет другое количество, просто сравнить два файла, можно даже
DOS-утилитой FC\footnote{утилита MS-DOS для сравнения двух файлов побайтово} (файлы рекордов, часто, бинарные).}
\EN{Likewise, it is easy to modify ``high score'' files, this may work not only with 8-bit games. Let's notice 
your score count and back the file up somewhere. When ``high score'' count will be different, just compare two files,
it can be even done with DOS-utility FC\footnote{MS-DOS utility for binary files comparing} (``high score'' files
are often in binary form).}
\RU{Где-то будут отличаться несколько байт, и легко будет увидеть, какие именно отвечают за количество очков. 
Впрочем, разработчики игр осведомлены о таких хитростях и могут защититься от этого.}
\EN{There will be a point where couple of bytes will be different and it will be easy to see which ones are
holding score number.
However, game developers are aware of such tricks and may protect against it.}

\RU{В каком-то смысле похожий пример в этой книге здесь}
\EN{Somewhat similar example here in the book is}: \ref{Millenium_DOS_game}.

% TODO: пример с какой-то простой игрушкой?

\subsection{\RU{Реестр Windows}\EN{Windows registry}}

\RU{А еще можно вспомнить сравнение реестра Windows до инсталляции программы и после}
\EN{It is also possible to compare Windows registry before and after a program installation}.
\RU{Это также популярный метод поиска, какие элементы реестра программа будет использовать}
\EN{It is very popular method of finding, which registry elements a program will use}.
\EN{Probably, this is a reason why ``windows registy cleaner'' shareware is so popluar.}
\RU{Наверное это причина, почему так популярны shareware-программы для очистки реестра в Windows.}


