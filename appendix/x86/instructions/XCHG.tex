\myindex{x86!\Instructions!XCHG}
  \item[XCHG] (M) \RU{обменять местами значения в операндах}\EN{exchange the values in the operands}

\myindex{Borland Delphi}
\RU{Это редкая инструкция: компиляторы её не генерируют, потому что начиная с Pentium, XCHG с адресом в памяти в операнде
исполняется так, как если имеет префикс LOCK (см.\InSqBrackets{\MAbrash глава 19}).
Вероятно, в Intel так сделали для совместимости с синхронизирующими примитивами.
Таким образом, XCHG начиная с Pentium может быть медленной.
С другой стороны, XCHG была очень популярна у программистов на ассемблере.
Так что, если вы видите XCHG в коде, это может быть знаком, что код написан вручную.
Впрочем, по крайней мере компилятор Borland Delphi генерирует эту инструкцию.}
\EN{This instruction is rare: compilers don't generate it, because starting at Pentium, XCHG with address in memory in operand executes as if it has LOCK prefix (\InSqBrackets{\MAbrash chapter 19}).
Perhaps, Intel engineers did so for compatibility with synchronizing primitives.
Hence, XCHG starting at Pentium can be slow.
On the other hand, XCHG was very popular in assembly language programmers.
So if you see XCHG in code, it can be a sign that this piece of code is written manually.
However, at least Borland Delphi compiler generates this instruction.}


