\index{\CStandardLibrary!strlen()}
\index{\CStandardLibrary!memchr()}
\index{x86!\Instructions!SCASB}
\index{x86!\Instructions!SCASW}
\index{x86!\Instructions!SCASD}
\index{x86!\Instructions!SCASQ}
\item[SCASB/SCASW/SCASD/SCASQ] (M) \RU{сравнить}\EN{compare} \RU{байт}\EN{byte}/
16-\RU{битное слово}\EN{bit word}/
32-\RU{битное слово}\EN{bit word}/
64-\RU{битное слово}\EN{bit word} \RU{записанный в}\EN{stored in the}
AX/EAX/RAX \RU{со значением, адрес которого находится
в}\EN{with a variable address of which is in the} DI/EDI/RDI.
\RU{Выставить флаги так же, как это делает \CMP}\EN{Set flags as \CMP does}.

\label{REPNE_SCASx}
\RU{Эта инструкция часто используется с префиксом REPNE: продолжать сканировать буфер до тех
пор, пока не встретится специальное значение, записанное в AX/EAX/RAX}
\EN{This instruction is often used with REPNE prefix: continue to scan a buffer until a special value
stored in AX/EAX/RAX is found}.
\RU{Отсюда ``NE'' в REPNE: продолжать сканирование если сравниваемые значения не равны и остановиться
если равны}
\EN{Hence ``NE'' in REPNE: continue to scan if compared values are not equal and stop when equal}.

\RU{Она часто используется как стандартная ф-ция Си strlen(), для определения длины \ac{ASCIIZ}-строки}
\EN{It is often used as strlen() C standard function, to determine \ac{ASCIIZ} string length}:

\RU{Пример}\EN{Example}:

\lstinputlisting{appendix/x86/instructions/SCASB_ex1_\LANG.asm}

\RU{Если использовать другое значение AX/EAX/RAX, ф-ция будет работать как стандартная ф-ция Си memchr(),
т.е., для поиска определенного байта}
\EN{If to use different AX/EAX/RAX value, the function will act as memchr() standard C function, i.e.,
it will find specific byte}.

