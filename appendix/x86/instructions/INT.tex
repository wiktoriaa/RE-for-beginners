\index{x86!\Instructions!INT}

\index{MS-DOS}
\item[INT] (M): \TT{INT x} \IFRU{аналогична}{is analogous to} \TT{PUSHF; CALL dword ptr [x*4]} 
\IFRU{в 16-битной среде}{in 16-bit environment}.
  \IFRU{Она активно использовалась в MS-DOS, работая как сисколл. Аргументы записывались в регистры
  AX/BX/CX/DX/SI/DI и затем происходил переход на таблицу векторов прерываний (расположенную в самом
  начале адресного пространства)}
  {It was widely used in MS-DOS, functioning as syscalls. Registers AX/BX/CX/DX/SI/DI were filled
  by arguments and jump to the address in the Interrupt Vector Table 
  (located at the address space beginning)
  will be occurred}.
  \IFRU{Она была очень популярна потому что имела короткий опкод (2 байта) и программе использующая
  сервисы MS-DOS не нужно было заморачиваться узнавая адреса всех ф-ций этих сервисов}
  {It was popular because INT has short opcode (2 bytes) and the program which needs
  some MS-DOS services is not bothering by determining service's entry point address}.
\index{x86!\Instructions!IRET}
  \IFRU{Обработчик прерываний возвращал управление назад при помощи инструкции IRET}
  {Interrupt handler return control flow to called using IRET instruction}.

  \IFRU{Самое используемое прерывание в MS-DOS было 0x21, там была основная часть его \ac{API}}
  {Most busy MS-DOS interrupt number was 0x21, serving a huge amount of its \ac{API}}.
  \IFRU{См. также}{See also}: \cite{RalfBrown} 
  \IFRU{самый крупный список всех известных прерываний и вообще там много информации о MS-DOS}
  {for the most comprehensive interrupt lists and other MS-DOS information}.

\index{x86!\Instructions!SYSENTER}
\index{x86!\Instructions!SYSCALL}
  \IFRU{Во времена после MS-DOS, эта инструкция все еще использовалась как сискол, и в Linux
  и в Windows (\ref{syscalls}), но позже была заменена инструкцией SYSENTER или SYSCALL}
  {In post-MS-DOS era, this instruction was still used as syscall both in Linux and 
  Windows (\ref{syscalls}), but later replaced by SYSENTER or SYSCALL instruction}.

\item[INT 3] (M): \IFRU{эта инструкция стоит немного в стороне от}{this instruction is somewhat standing aside of} 
\TT{INT}, \IFRU{она имеет собственный 1-байтный опкод}{it has its own 1-byte opcode} (\TT{0xCC}), 
\IFRU{и активно используется в отладке}{and actively used while debugging}.
\IFRU{Часто, отладчик просто записывает байт}{Often, debuggers just write} \TT{0xCC} 
\IFRU{по адресу в памяти где устанавливается брякпойнт, и когда исключение поднимается, оригинальный байт
будет восстановлен и оригинальная инструкция по этому адресу исполнена заново}{byte at the address of 
breakpoint to be set, and when exception is raised,
original byte will be restored and original instruction at this address will be re-executed}.
\IFRU{В}{As of} \gls{Windows NT}, \IFRU{исключение}{an} \TT{EXCEPTION\_BREAKPOINT} \IFRU{поднимается,
когда \ac{CPU} исполняет эту инструкцию}{exception will be raised when \ac{CPU} executes this instruction}.
\IFRU{Это отладочное событие может быть перехвачено и обработано отладчиком, если он загружен}
{This debugging event may be intercepted and handled by a host debugger, if loaded}.
\IFRU{Если он не загружен, Windows предложит запустить один из зарегистрированных в системе отладчиков}
{If it is not loaded, Windows will offer to run one of the registered in the system debuggers}.
\IFRU{Если}{If} \ac{MSVS} \IFRU{установлена, его отладчик может быть загружен и подключен к процессу}{is installed, 
its debugger may be loaded and connected to the process}.
\IFRU{В целях защиты от}{In order to protect from} \gls{reverse engineering}, \IFRU{множество анти-отладочных методов
проверяют целостность загруженного кода}{a lot of anti-debugging methods are checking integrity of the code loaded}.

\RU{В }\ac{MSVC} \IFRU{есть}{has} \gls{compiler intrinsic} \IFRU{для этой инструкции}{for the instruction}:
\TT{\_\_debugbreak()}\footnote{\url{http://msdn.microsoft.com/en-us/library/f408b4et.aspx}}.

\IFRU{В win32 также имеется ф-ция в}{There are also a win32 function in} kernel32.dll \IFRU{с названием}{named}
\TT{DebugBreak()}\footnote{\url{http://msdn.microsoft.com/en-us/library/windows/desktop/ms679297(v=vs.85).aspx}},
\IFRU{которая также исполняет}{which also executes} \TT{INT 3}.

