\section{GDB}
\index{GDB}
\label{sec:GDB_cheatsheet}

\RU{Некоторые команды, которые были использованы в книге}\EN{Some of commands we used in this book}:

\begin{center}
\begin{tabular}{ | l | l | }
\hline
\cellcolor{blue!25} \RU{опция}\EN{option} & 
\cellcolor{blue!25} \RU{значение}\EN{meaning} \\
\hline
break filename.c:number		& \RU{установить точку останова на номере строки в исходном файле}
					\EN{set a breakpoint on line number in source code} \\
break function			& \RU{установить точку останова на функции}\EN{set a breakpoint on function} \\
break *address			& \RU{установить точку останова на адресе}\EN{set a breakpoint on address} \\
b				& \dittoclosing \\
p variable			& \RU{вывести значение переменной}\EN{print value of variable} \\
run				& \RU{запустить}\EN{run} \\
r				& \dittoclosing \\
cont				& \RU{продолжить исполнение}\EN{continue execution} \\
c				& \dittoclosing \\
bt				& \RU{вывести стек}\EN{print stack} \\
set disassembly-flavor intel	& \RU{установить Intel-синтаксис}\EN{set Intel syntax} \\
disas				& disassemble current function \\
disas function			& \RU{дизассемблировать функцию}\EN{disassemble function} \\
disas function,+50		& disassemble portion \\
disas \$eip,+0x10		& \dittoclosing \\
disas/r				& \EN{disassemble with opcodes}\RU{дизассемблировать с опкодами} \\
info registers			& \RU{вывести все регистры}\EN{print all registers} \\
info float			& \RU{вывести FPU-регистры}\EN{print FPU-registers} \\
info locals			& \RU{вывести локальные переменные (если известны)}\EN{dump local variables (if known)} \\
x/w ...				& \RU{вывести память как 32-битные слова}\EN{dump memory as 32-bit word} \\
x/w \$rdi			& \RU{вывести память как 32-битные слова по адресу, на который указывает \TT{RDI}}\EN{dump memory as 32-bit word at address stored in \TT{RDI}} \\
x/10w ...			& \RU{вывести 10 слов памяти}\EN{dump 10 memory words} \\
x/s ...				& \RU{вывести строку из памяти}\EN{dump memory as string} \\
x/i ...				& \RU{трактовать память как код}\EN{dump memory as code} \\
x/10c ...			& \RU{вывести 10 символов}\EN{dump 10 characters} \\
x/b ...				& \RU{вывести байты}\EN{dump bytes} \\
x/h ...				& \RU{вывести 16-битные полуслова}\EN{dump 16-bit halfwords} \\
x/g ...				& \RU{вывести 64-битные слова}\EN{dump giant (64-bit) words} \\
finish				& \RU{исполнять до конца функции}\EN{execute till the end of function} \\
next				& \RU{следующая инструкция (не заходить в функции)}
					\EN{next instruction (don't dive into functions)} \\
step				& \RU{следующая инструкция (заходить в функции)}
					\EN{next instruction (dive into functions)} \\
set step-mode on		& \RU{не использовать информацию о номерах строк при использовании команды step}
					\EN{do not use line number information while stepping} \\
frame n				& \RU{переключить фрейм стека}\EN{switch stack frame} \\
info break			& \RU{список точек останова}\EN{list of breakpoints} \\
del n				& \RU{удалить точку установа}\EN{delete breakpoint} \\
set args ...			& \RU{установить аргументы командной строки}\EN{set command-line arguments} \\
\hline
\end{tabular}
\end{center}


