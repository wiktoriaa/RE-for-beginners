\section{\RU{Инструкции}\EN{Instructions}}

\RU{В ARM имеется также для некоторых инструкций суффикс \IT{-S}, указывающий, 
что эта инструкция будет модифицировать флаги, а при отсутствии суффикса ~--- не будет.}
\EN{There is \IT{-S} suffix for some instructions in ARM,
indicating the instruction will set the flags according to the result, and without 
it~---the flags will not be touched.}
\index{ARM!\Instructions!ADD}
\index{ARM!\Instructions!ADDS}
\index{ARM!\Instructions!CMP}
\RU{Например, инструкция}\EN{For example} \TT{ADD} \RU{в отличие от}\EN{unlike} \TT{ADDS}
\RU{сложит два числа, но флаги не изменит}
\EN{will add two numbers, but flags will not be touched}.
\RU{Такие инструкции удобно использовать
между \CMP где выставляются флаги и, например, инструкциями перехода, где флаги используются.}
\EN{Such instructions are convenient to use between \CMP where flags are set and, 
e.g. conditional jumps, where flags are used.}

% ADD
% ADDAL
% ADDCC
% ADDS
% ADR
% ADREQ
% ADRGT
% ADRHI
% ADRNE
% ASRS
% B
% BCS
% BEQ
% BGE
% BIC
% BL
% BLE
% BLEQ
% BLGT
% BLHI
% BLS
% BLT
% BLX
% BNE
% BX
% CMP
% IDIV
% IT
% LDMCSFD
% LDMEA
% LDMED
% LDMFA
% LDMFD
% LDMGEFD
% LDR.W
% LDR
% LDRB.W
% LDRB
% LDRSB
% LSL.W
% LSL
% LSLS
% MLA
% MOV
% MOVT.W
% MOVT
% MOVW
% MULS
% MVNS
% ORR
% POP
% PUSH
% RSB
% SMMUL
% STMEA
% STMED
% STMFA
% STMFD
% STMIA
% STMIB
% STR
% SUB
% SUBEQ
% SXTB
% TEST
% TST
% VADD
% VDIV
% VLDR
% VMOV
% VMOVGT
% VMRS
% VMUL
%\index{ARM!Optional operators!ASR
%\index{ARM!Optional operators!LSL
%\index{ARM!Optional operators!LSR
%\index{ARM!Optional operators!ROR
%\index{ARM!Optional operators!RRX

% AArch64
% RET is BR X30 or BR LR but with additional hint to CPU

