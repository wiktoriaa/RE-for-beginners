\section{\RU{По главам}\EN{Per chapter}}

\subsection{\RU{Глава ``\Stack''}\EN{``\Stack'' chapter}}
%02..

\subsubsection{\Exercise \#1}
\label{exercise_solutions_stack_1}

\Exercise: \ref{exercise_stack_1}.

\EN{\NonOptimizing MSVC, these numbers are: saved \EBP value}\RU{Если MSVC без оптимизации, 
то эти числа таковы: сохраненное значение \EBP}, \ac{RA} \AndENRU \TT{argc}.
\RU{В этом легко убедиться, если запускать пример с разным количеством аргументов в командной 
строке}\EN{It's easy to be assured in that by running the example with different number
of arguments in command-line}.

\RU{Если MSVC с оптимизацией, то числа таковы}\EN{\Optimizing MSVC, these numbers are}: 
\ac{RA}, \TT{argc} \RU{и указатель на массив}\EN{and a pointer to} \TT{argv[]}\EN{ array}.

GCC 4.8.x \RU{выделяет в прологе ф-ции \main 16-байтное пространство, поэтому числа на 
выходе совсем другие}\EN{allocates 16-byte space in \main function prologue, 
hence different output numbers}.

\subsection{\RU{Глава ``\Loops''}\EN{``\Loops'' chapter}}
%09...

\subsection{\Exercise \#3}
\label{exercise_solutions_loops_3}

\Exercise: \ref{exercise_loops_3}.

\begin{lstlisting}
#include <stdio.h>

int main()
{
	int i;
	for (i=100; i>0; i--)
		printf ("%d\n", i);
};
\end{lstlisting}

\subsection{\Exercise \#4}
\label{exercise_solutions_loops_4}

\Exercise: \ref{exercise_loops_4}.

\begin{lstlisting}
#include <stdio.h>

int main()
{
	int i;
	for (i=1; i<100; i=i+3)
		printf ("%d\n", i);
};
\end{lstlisting}

\subsection{\RU{Глава ``\SimpleStringsProcessings''}\EN{``\SimpleStringsProcessings'' chapter}}
%10..

\subsubsection{\Exercise \#1}
\label{exercise_solutions_strlen_1}

\Exercise: \ref{exercise_strlen_1}.

\RU{Эта ф-ция подсчитывает пробелы во входящей Си-строке.}
\EN{This is a function counting spaces in the input C-string.}

\begin{lstlisting}
int f(char *s)
{
	int rt=0;
	for (;*s;s++)
	{
		if (*s==' ')
			rt++;
	};
	return rt;
};
\end{lstlisting}

\subsection{\RU{Глава ``\ArithOptimizations''}\EN{``\ArithOptimizations'' chapter}}
%11..

\subsubsection{\Exercise \#1}
\label{exercise_solutions_arith_optimizations_1}

\Exercise: \ref{exercise_arith_optimizations_1}.

\begin{lstlisting}
int f(int a)
{
	return a/661;
};
\end{lstlisting}

\subsubsection{\Exercise \#2}
\label{exercise_solutions_arith_optimizations_2}

\Exercise: \ref{exercise_arith_optimizations_2}.

\begin{lstlisting}
int f(int a)
{
	return a*7;
};
\end{lstlisting}

\subsection{\RU{Глава ``\FPUChapterName''}\EN{``\FPUChapterName'' chapter}}
%12..

\subsubsection{\Exercise \#1}
\label{exercise_solutions_FPU_2}

\Exercise: \ref{exercise_FPU_2}.

\EN{Calculating arithmetic mean for 5 \Tdouble values.}
\RU{Вычисление среднего арифметического для пяти значений типа \Tdouble.}

\begin{lstlisting}
double f(double a1, double a2, double a3, double a4, double a5)
{
	return (a1+a2+a3+a4+a5) / 5;
};
\end{lstlisting}

\subsection{\RU{Глава ``\Arrays''}\EN{``\Arrays'' chapter}}
%13..

\subsubsection{\Exercise \#1}
\label{exercise_solutions_arrays_1}

\Exercise: \ref{exercise_array_1}.

\RU{Ответ: сложение двух матриц размером 100 на 200 элементов типа \Tdouble.}
\EN{Solution: two 100*200 matrices of \Tdouble type addition.}

\RU{Исходник на \CCpp}\EN{\CCpp source code}:

\begin{lstlisting}
#define M    100
#define N    200

void s(double *a, double *b, double *c)
{
  for(int i=0;i<N;i++)
    for(int j=0;j<M;j++)
      *(c+i*M+j)=*(a+i*M+j) + *(b+i*M+j);
};
\end{lstlisting}

\subsubsection{\Exercise \#2}
\label{exercise_solutions_arrays_2}

\Exercise: \ref{exercise_array_2}.

\RU{Ответ: умножение двух матриц размерами 100*200 и 100*300 элементов типа \Tdouble, результат: матрица 100*300.}
\EN{Solution: two matrices (one is 100*200, second is 100*300) of \Tdouble type multiplication, result: 100*300
matrix.}

\RU{Исходник на \CCpp}\EN{\CCpp source code}:

\begin{lstlisting}
#define M     100
#define N     200
#define P     300

void m(double *a, double *b, double *c)
{
  for(int i=0;i<M;i++)
    for(int j=0;j<P;j++)
    {
      *(c+i*M+j)=0;
      for (int k=0;k<N;k++) *(c+i*M+j)+=*(a+i*M+j) * *(b+i*M+j);
    }
};
\end{lstlisting}

\subsubsection{\Exercise \#3}
\label{exercise_solutions_arrays_3}

\Exercise: \ref{exercise_array_3}.

\begin{lstlisting}
double f(double array[50][120], int x, int y)
{
	return array[x][y];
};
\end{lstlisting}

\subsubsection{\Exercise \#4}
\label{exercise_solutions_arrays_4}

\Exercise: \ref{exercise_array_4}.

\begin{lstlisting}
int f(int array[50][60][80], int x, int y, int z)
{
	return array[x][y][z];
};
\end{lstlisting}
