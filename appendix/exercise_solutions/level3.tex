\section{\RU{Уровень}\EN{Level} 3}

\subsection{\Exercise 3.1}

\RU{Подсказка \#1: В этом коде есть одна особенность, по которой можно значительно сузить поиск функции в glibc.}
\EN{Hint \#1: The code has one characteristic thing, if considering it, it may help narrowing search of right function 
among glibc functions}.

\RU{Ответ: особенность ~--- это вызов callback-функции}\EN{Solution: characteristic~---is callback-function
calling}~(\ref{sec:pointerstofunctions}),
\RU{указатель на которую передается в четвертом аргументе}\EN{pointer to which is passed in 4th
argument}. \RU{Это}\EN{It is} \TT{quicksort()}.

\index{\CStandardLibrary!qsort()}
\RU{Исходник на Си}\EN{C source code}: \url{http://yurichev.com/RE-exercise-solutions/3/1/2_1.c}

\subsection{\Exercise 3.2}

\RU{Подсказка: проще всего конечно же искать по значениями в таблицах}
\EN{Hint: easiest way is to find by values in the tables}.

\RU{Исходник на Си с комментариями}\EN{Commented C source code}:\\
\url{http://yurichev.com/RE-exercise-solutions/3/2/gost.c}

\subsection{\Exercise 3.3}

\RU{Исходник на Си с комментариями}\EN{Commented C source code}:\\
\url{http://yurichev.com/RE-exercise-solutions/3/3/entropy.c}

\subsection{\Exercise 3.4}
\RU{Исходник на Си с комментариями, а также расшифрованный файл}
\EN{Commented C source code, and also decrypted file}:
\url{http://yurichev.com/RE-exercise-solutions/3/4/}

\subsection{\Exercise 3.5}

\RU{Подсказка: как видно, строка где указано имя пользователя занимает не весь ключевой файл}
\EN{Hint: as we can see, the string with user name occupies not the whole file}.

\RU{Байты за терминирующим нулем вплоть до смещения \TT{0x7F} игнорируются программой}
\EN{Bytes after terminated zero till offset \TT{0x7F} are ignored by program}.

\RU{Исходник на Си с комментариями}\EN{Commented C source code}:\\
\url{http://yurichev.com/RE-exercise-solutions/3/5/crc16_keyfile_check.c}

\subsection{\Exercise 3.6}

{\RU{Исходник на Си с комментариями}\EN{Commented C source code}}:\\
\url{http://yurichev.com/RE-exercise-solutions/3/6/}

\RU{В качестве еще одного упражнения, теперь вы можете попробовать исправить уязвимости в этом веб-сервере}
\EN{As another exercise, now you may try to fix all vulnerabilities you found in this web-server}.

\subsection{\Exercise 3.8}

{\RU{Исходник на Си с комментариями}\EN{Commented C source code}}:\\
\url{http://yurichev.com/RE-exercise-solutions/3/8/}

