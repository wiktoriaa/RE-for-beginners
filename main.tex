\ifdefined\ebook
\documentclass[a5paper,oneside]{book}
\newcommand{\FigScale}{0.4}
\else
\documentclass[a4paper,oneside]{book}
\newcommand{\FigScale}{0.66} % FIXME
\fi

\usepackage{cmap}
\usepackage{fancyhdr}

\ifdefined\RUSSIAN
\usepackage[english,russian]{babel}
\else
\usepackage[english]{babel}
\fi

\usepackage{inputenc}
\usepackage{listings}
\usepackage{ulem}
\usepackage{url}
\usepackage{graphicx}
\usepackage{makeidx}
\usepackage[backend=biber,style=alphabetic]{biblatex}
%\usepackage{cite}
\usepackage[cm]{fullpage}
\usepackage{color}
\usepackage{fancyvrb}
\usepackage{xspace}
\usepackage{tabularx}
\usepackage{framed}

\usepackage{fontspec}
% fonts
\setmonofont{Droid Sans Mono}
\setmainfont[Ligatures=TeX]{PT Sans}

\usepackage{epigraph}
\usepackage{ccicons}
\usepackage[nottoc]{tocbibind}
\usepackage{longtable}
\usepackage[footnote,printonlyused,withpage]{acronym}
\usepackage[table]{xcolor}% http://ctan.org/pkg/xcolor
\usepackage[]{bookmark,hyperref} % must be last

\usepackage{glossaries}
\usepackage{tikz}
%\usepackage{fixltx2e}
\usepackage{bytefield}

\usepackage{amsmath}
\usepackage{MnSymbol}
\undef\mathdollar 

\usepackage{float}

\usepackage{shorttoc}
\usetikzlibrary{calc,positioning,chains,arrows}
\ifdefined\ebook
\usepackage[margin=0.5in,headheight=11pt]{geometry}
\else
\usepackage[margin=0.5in,headheight=12.5pt]{geometry}
\fi

\ifdefined\RUSSIAN
\renewcommand\lstlistingname{Листинг}
\renewcommand\lstlistlistingname{Листинг}
\fi

%\iffalse
% fancyhdr ********************************************************
\makeatletter
    \let\stdchapter\chapter
    \renewcommand*\chapter{%
    \@ifstar{\starchapter}{\@dblarg\nostarchapter}}
    \newcommand*\starchapter[1]{%
        \stdchapter*{#1}
        \thispagestyle{fancy}
        \markboth{\MakeUppercase{#1}}{}
    }
    \def\nostarchapter[#1]#2{%
        \stdchapter[{#1}]{#2}
        \thispagestyle{fancy}
    }
\makeatother

% taken from http://texblog.org/tag/fancyhdr-font-size/
\newcommand{\changefont}{%
\ifdefined\ebook
    \fontsize{6}{7}\selectfont
\else
    \fontsize{8}{9.5}\selectfont
\fi
}
\fancyhf{}
\fancyhead[LE,RO]{\changefont \slshape \rightmark} %section
\fancyhead[RE,LO]{\changefont \slshape \leftmark} %chapter
\fancyfoot[C]{\changefont \thepage} %footer
% *****************************************************************
%\fi

\newcommand{\footnoteref}[1]{\textsuperscript{\ref{#1}}}

\definecolor{lstbgcolor}{rgb}{0.94,0.94,0.94}
\makeindex

\newcommand*{\TT}[1]{\texttt{#1}}
\newcommand*{\IT}[1]{\textit{#1}}
\newcommand*{\EN}[1]{\iflanguage{english}{#1}{}}
\newcommand*{\RU}[1]{\iflanguage{english}{}{#1}}
\newcommand{\LANG}{\RU{ru}\EN{en}}
\newcommand*{\dittoclosing}{---''---}
\newcommand*{\EMDASH}{\RU{ --- }\EN{---}}
\newcommand*{\AsteriskOne}{${}^{*}$}
\newcommand*{\AsteriskTwo}{${}^{**}$}
\newcommand*{\AsteriskThree}{${}^{***}$}

\newcommand{\ttf}{\TT{f()}\xspace}
\newcommand{\ttfone}{\TT{f1()}\xspace}

% http://tex.stackexchange.com/questions/32160/new-line-after-paragraph
\newcommand{\myparagraph}[1]{\paragraph{#1}\mbox{}\\} 

\newcommand{\figname}{\RU{илл}\EN{fig}.\xspace}
\newcommand{\figref}[1]{\figname{}\ref{#1}\xspace}
\newcommand{\listingname}{\RU{листинг}\EN{listing}.\xspace}
\newcommand{\lstref}[1]{\listingname{}\ref{#1}\xspace}
\newcommand{\bitENRU}{\RU{бит}\EN{bit}\xspace}
\newcommand{\bitsENRU}{\RU{бита}\EN{bits}\xspace}
\newcommand{\Sourcecode}{\RU{Исходный код}\EN{Source code}\xspace}
\newcommand{\Seealso}{\RU{См. также}\EN{See also}\xspace}
\newcommand{\MacOSX}{Mac OS X\xspace}

% FIXME TODO non-overlapping color!
% \newcommand{\headercolor}{\cellcolor{blue!25}}
\newcommand{\headercolor}{}

\newcommand{\tableheader}{\headercolor{} \RU{смещение}\EN{offset} & \headercolor{} \RU{описание}\EN{description}}

\newcommand{\IDA}{\ac{IDA}\xspace}

\newcommand{\tracer}{\protect\gls{tracer}\xspace}

\newcommand{\Tchar}{\IT{char}\xspace} 
\newcommand{\Tint}{\IT{int}\xspace}
\newcommand{\Tbool}{\IT{bool}\xspace}
\newcommand{\Tfloat}{\IT{float}\xspace}
\newcommand{\Tdouble}{\IT{double}\xspace}
\newcommand{\Tvoid}{\IT{void}\xspace}
\newcommand{\ITthis}{\IT{this}\xspace}

\newcommand{\Ox}{\TT{/Ox}\xspace}
\newcommand{\Obzero}{\TT{/Ob0}\xspace}
\newcommand{\Othree}{\TT{-O3}\xspace}

\newcommand{\oracle}{Oracle RDBMS\xspace}

\newcommand{\idevices}{iPod/iPhone/iPad\xspace}
\newcommand{\olly}{OllyDbg\xspace}

% common C functions
\newcommand{\printf}{\TT{printf()}\xspace} 
\newcommand{\puts}{\TT{puts()}\xspace} 
\newcommand{\main}{\TT{main()}\xspace} 
\newcommand{\qsort}{\TT{qsort()}\xspace} 
\newcommand{\strlen}{\TT{strlen()}\xspace} 
\newcommand{\scanf}{\TT{scanf()}\xspace} 
\newcommand{\rand}{\TT{rand()}\xspace} 

% x86 instructions
\newcommand{\ADD}{\TT{ADD}\xspace} 
\newcommand{\ANDIns}{\TT{AND}\xspace} 
\newcommand{\CALL}{\TT{CALL}\xspace} 
\newcommand{\CPUID}{\TT{CPUID}\xspace} 
\newcommand{\CMP}{\TT{CMP}\xspace} 
\newcommand{\DEC}{\TT{DEC}\xspace} 
\newcommand{\FADDP}{\TT{FADDP}\xspace} 
\newcommand{\FCOM}{\TT{FCOM}\xspace}
\newcommand{\FCOMP}{\TT{FCOMP}\xspace}
\newcommand{\FCOMI}{\TT{FCOMI}\xspace}
\newcommand{\FCOMIP}{\TT{FCOMIP}\xspace}
\newcommand{\FUCOM}{\TT{FUCOM}\xspace}
\newcommand{\FUCOMI}{\TT{FUCOMI}\xspace}
\newcommand{\FUCOMIP}{\TT{FUCOMIP}\xspace}
\newcommand{\FUCOMPP}{\TT{FUCOMPP}\xspace}
\newcommand{\FDIVR}{\TT{FDIVR}\xspace} 
\newcommand{\FDIV}{\TT{FDIV}\xspace} 
\newcommand{\FLD}{\TT{FLD}\xspace} 
\newcommand{\FMUL}{\TT{FMUL}\xspace} 
\newcommand{\FSTP}{\TT{FSTP}\xspace} 
\newcommand{\FDIVP}{\TT{FDIVP}\xspace}
\newcommand{\IDIV}{\TT{IDIV}\xspace} 
\newcommand{\IMUL}{\TT{IMUL}\xspace} 
\newcommand{\INC}{\TT{INC}\xspace} 
\newcommand{\JAE}{\TT{JAE}\xspace} 
\newcommand{\JA}{\TT{JA}\xspace} 
\newcommand{\JBE}{\TT{JBE}\xspace} 
\newcommand{\JB}{\TT{JBE}\xspace} 
\newcommand{\JE}{\TT{JE}\xspace} 
\newcommand{\JGE}{\TT{JGE}\xspace} 
\newcommand{\JG}{\TT{JG}\xspace} 
\newcommand{\JLE}{\TT{JLE}\xspace} 
\newcommand{\JL}{\TT{JL}\xspace} 
\newcommand{\JMP}{\TT{JMP}\xspace} 
\newcommand{\JNE}{\TT{JNE}\xspace} 
\newcommand{\JNZ}{\TT{JNZ}\xspace} 
\newcommand{\JNA}{\TT{JNA}\xspace} 
\newcommand{\JNAE}{\TT{JNAE}\xspace} 
\newcommand{\JNB}{\TT{JNB}\xspace} 
\newcommand{\JNBE}{\TT{JNBE}\xspace} 
\newcommand{\JZ}{\TT{JZ}\xspace} 
\newcommand{\JP}{\TT{JP}\xspace} 
\newcommand{\Jcc}{\TT{Jcc}\xspace} 
\newcommand{\SETcc}{\TT{SETcc}\xspace} 
\newcommand{\LEA}{\TT{LEA}\xspace} 
\newcommand{\LOOP}{\TT{LOOP}\xspace}
\newcommand{\MOVSX}{\TT{MOVSX}\xspace} 
\newcommand{\MOVZX}{\TT{MOVZX}\xspace} 
\newcommand{\MOV}{\TT{MOV}\xspace} 
\newcommand{\NOP}{\TT{NOP}\xspace} 
\newcommand{\POP}{\TT{POP}\xspace} 
\newcommand{\PUSH}{\TT{PUSH}\xspace} 
\newcommand{\NOT}{\TT{NOT}\xspace} 
\newcommand{\RET}{\TT{RET}\xspace} 
\newcommand{\RETN}{\TT{RETN}\xspace} 
\newcommand{\SETNZ}{\TT{SETNZ}\xspace} 
\newcommand{\SETBE}{\TT{SETBE}\xspace} 
\newcommand{\SETNBE}{\TT{SETNBE}\xspace} 
\newcommand{\SUB}{\TT{SUB}\xspace} 
\newcommand{\TEST}{\TT{TEST}\xspace} 
\newcommand{\FNSTSW}{\TT{FNSTSW}\xspace}
\newcommand{\SAHF}{\TT{SAHF}\xspace}
\newcommand{\XOR}{\TT{XOR}\xspace} 
\newcommand{\OR}{\TT{OR}\xspace} 
\newcommand{\SHL}{\TT{SHL}\xspace} 
\newcommand{\SHR}{\TT{SHR}\xspace} 
\newcommand{\LEAVE}{\TT{LEAVE}\xspace} 
\newcommand{\MOVDQA}{\TT{MOVDQA}\xspace} 
\newcommand{\MOVDQU}{\TT{MOVDQU}\xspace} 
\newcommand{\PADDD}{\TT{PADDD}\xspace} 
\newcommand{\PCMPEQB}{\TT{PCMPEQB}\xspace} 

% x86 flags

\newcommand{\ZF}{\TT{ZF}\xspace} 
\newcommand{\CF}{\TT{CF}\xspace} 
\newcommand{\PF}{\TT{PF}\xspace} 

% x86 registers

\newcommand{\AL}{\TT{AL}\xspace} 
\newcommand{\AH}{\TT{AH}\xspace} 
\newcommand{\AX}{\TT{AX}\xspace} 
\newcommand{\EAX}{\TT{EAX}\xspace} 
\newcommand{\EBX}{\TT{EBX}\xspace} 
\newcommand{\ECX}{\TT{ECX}\xspace} 
\newcommand{\EDX}{\TT{EDX}\xspace} 
\newcommand{\DL}{\TT{DL}\xspace} 
\newcommand{\ESI}{\TT{ESI}\xspace} 
\newcommand{\EDI}{\TT{EDI}\xspace} 
\newcommand{\EBP}{\TT{EBP}\xspace} 
\newcommand{\ESP}{\TT{ESP}\xspace} 
\newcommand{\RSP}{\TT{RSP}\xspace} 
\newcommand{\EIP}{\TT{EIP}\xspace} 
\newcommand{\RIP}{\TT{RIP}\xspace} 
\newcommand{\RAX}{\TT{RAX}\xspace} 
\newcommand{\RBX}{\TT{RBX}\xspace} 
\newcommand{\RCX}{\TT{RCX}\xspace} 
\newcommand{\RDX}{\TT{RDX}\xspace} 
\newcommand{\RBP}{\TT{RBP}\xspace} 
\newcommand{\RSI}{\TT{RSI}\xspace} 
\newcommand{\RDI}{\TT{RDI}\xspace} 
\newcommand*{\ST}[1]{\TT{ST(#1)}\xspace}
\newcommand*{\XMM}[1]{\TT{XMM#1}\xspace}

% ARM
\newcommand*{\Reg}[1]{\TT{R#1}\xspace}
\newcommand*{\RegX}[1]{\TT{X#1}\xspace}
\newcommand*{\RegW}[1]{\TT{W#1}\xspace}
\newcommand*{\RegD}[1]{\TT{D#1}\xspace}
\newcommand{\ADREQ}{\TT{ADREQ}\xspace}
\newcommand{\ADRNE}{\TT{ADRNE}\xspace}
\newcommand{\BEQ}{\TT{BEQ}\xspace}

% instructions descriptions
\newcommand{\ASRdesc}{\RU{арифметический сдвиг вправо}\EN{arithmetic shift right}}

% x86 registers tables
% TODO: non-overlapping color!
\newcommand{\RegHeader}{
\RU{
 7 \textsuperscript{(номер байта)} &
 6 &
 5 &
 4 &
 3 &
 2 &
 1 &
 0 }
\EN{
 7th \textsuperscript{(byte number)} &
 6th &
 5th &
 4th &
 3rd &
 2nd &
 1st &
 0th}
}

% FIXME навести порядок тут...
\newcommand{\RegTableThree}[5]{
\begin{center}
\begin{tabular}{ | l | l | l | l | l | l | l | l | l |}
\hline
\RegHeader \\
\hline
\multicolumn{8}{ | c | }{#1} \\
\hline
\multicolumn{4}{ | c | }{} & \multicolumn{4}{ c | }{#2} \\
\hline
\multicolumn{6}{ | c | }{} & \multicolumn{2}{ c | }{#3} \\
\hline
\multicolumn{6}{ | c | }{} & #4 & #5 \\
\hline
\end{tabular}
\end{center}
}

\newcommand{\RegTableOne}[5]{\RegTableThree{#1\textsuperscript{x64}}{#2}{#3}{#4}{#5}}

\newcommand{\RegTableTwo}[4]{
\begin{center}
\begin{tabular}{ | l | l | l | l | l | l | l | l | l |}
\hline
\RegHeader \\
\hline
\multicolumn{8}{ | c | }{#1\textsuperscript{x64}} \\
\hline
\multicolumn{4}{ | c | }{} & \multicolumn{4}{ c | }{#2} \\
\hline
\multicolumn{6}{ | c | }{} & \multicolumn{2}{ c | }{#3} \\
\hline
\multicolumn{7}{ | c | }{} & #4\textsuperscript{x64} \\
\hline
\end{tabular}
\end{center}
}

\newcommand{\RegTableFour}[4]{
\begin{center}
\begin{tabular}{ | l | l | l | l | l | l | l | l | l |}
\hline
\RegHeader \\
\hline
\multicolumn{8}{ | c | }{#1} \\
\hline
\multicolumn{4}{ | c | }{} & \multicolumn{4}{ c | }{#2} \\
\hline
\multicolumn{6}{ | c | }{} & \multicolumn{2}{ c | }{#3} \\
\hline
\multicolumn{7}{ | c | }{} & #4 \\
\hline
\end{tabular}
\end{center}
}


\newglossaryentry{tail call}
{
  name=\RU{хвостовая рекурсия}\EN{tail call}\ESph{}\PTBRph{}\DEph{}\PLph{}\ITAph{}\FR{tail call},
  description={\RU{Это когда компилятор или интерпретатор превращает рекурсию 
  (с которой возможно это проделать, т.е. \IT{хвостовую}) в итерацию для эффективности}
  \EN{It is when the compiler (or interpreter) transforms the recursion (with which it is possible: \IT{tail recursion}) 
  into an iteration for efficiency}\ESph{}\PTBRph{}\DEph{}\PLph{}\ITAph{}
  \FR{C'est lorsque le compilateur (ou l'interpréteur) transforme la récursion (ce qui est possible: \IT{tail recursion})
  en une itération pour l'efficacité}: \href{http://go.yurichev.com/17105}{wikipedia}}
}

\newglossaryentry{endianness}
{
  name=endianness,
  description={\RU{Порядок байт}\EN{Byte order}\ESph{}\PTBRph{}\DEph{}\PLph{}\ITAph{}\FR{Ordre des octets}: \myref{sec:endianness}}
}

\newglossaryentry{caller}
{
  name=caller,
  description={\RU{Функция вызывающая другую функцию}\EN{A function calling another}\ESph{}\PTBRph{}\DE{aufrufende Funktion}\PLph{}\ITAph{}
  \FR{Une fonction en appelant une autre}}
}

\newglossaryentry{callee}
{
  name=callee,
  description={\RU{Вызываемая функция}\EN{A function being called by another}\ESph{}\PTBRph{}\DE{aufgerufene Funktion}\PLph{}\ITAph{}
  \FR{Une fonction appelée par une autre}}
}

\newglossaryentry{debuggee}
{
  name=debuggee,
  description={\RU{Отлаживаемая программа}\EN{A program being debugged}\ESph{}\PTBRph{}\DEph{}\PLph{}\ITAph{}
  \FR{Un programme en train d'être débogué}}
}

\newglossaryentry{leaf function}
{
  name=leaf function,
  description={\RU{Функция не вызывающая больше никаких функций}
  \EN{A function which does not call any other function}\ESph{}\PTBRph{}\DEph{}\PLph{}\ITAph{}
  \FR{Une fonction qui n'appelle pas d'autre fonction}}
}

\newglossaryentry{link register}
{
  name=link register,
  description=(RISC) {\RU{Регистр в котором обычно записан адрес возврата.
  Это позволяет вызывать leaf-функции без использования стека, т.е. быстрее}
  \EN{A register where the return address is usually stored.
  This makes it possible to call leaf functions without using the stack, i.e., faster}\ESph{}\PTBRph{}\DEph{}\PLph{}\ITAph{}
  \FR{Un registre oú l'adresse de retour est en général stockée. Ceci permet
  d'appeler une fonction leaf sans utiliser la pile, i.e, plus rapidemment}
  }
}

\newglossaryentry{anti-pattern}
{
  name=anti-pattern,
  description={\RU{Нечто широко известное как плохое решение}
  \EN{Generally considered as bad practice}\ESph{}\PTBRph{}\DEph{}\PLph{}\ITAph{}
  \FR{En général considéré comme une mauvaise pratique}
  }
}

\newglossaryentry{stack pointer}
{
  name=\RU{указатель стека}\EN{stack pointer}\FR{pointeur de pile}\ESph{}\PTBRph{}\DE{Stapel-Zeiger}\PLph{}\ITAph{},
  description={\RU{Регистр указывающий на место в стеке}
  \EN{A register pointing to a place in the stack}\FR{Un registre qui pointe dans la pile}\ESph{}\PTBRph{}\DE{Ein Register das auf eine Stelle im Stack zeigt}\PLph{}\ITAph{}}
}

\newglossaryentry{decrement}
{
  name=\RU{декремент}\EN{decrement}\FR{décrémenter}\ESph{}\PTBRph{}\DEph{}\PLph{}\ITAph{},
  description={\RU{Уменьшение на 1}\EN{Decrease by 1}\ESph{}\PTBRph{}\DEph{}\PLph{}\ITAph{}
  \FR{Décrémenter de 1}
  }
}

\newglossaryentry{increment}
{
  name=\RU{инкремент}\EN{increment}\FR{incrémenter}\ESph{}\PTBRph{}\DEph{}\PLph{}\ITAph{},
  description={\RU{Увеличение на 1}\EN{Increase by 1}\ESph{}\PTBRph{}\DEph{}\PLph{}\ITAph{}
  \FR{Incrémenter de 1}
  }
}

\newglossaryentry{loop unwinding}
{
  name=loop unwinding,
  description={\RU{Это когда вместо организации цикла на $n$ итераций, компилятор генерирует $n$ копий тела
  цикла, для экономии на инструкциях, обеспечивающих сам цикл}
  \EN{It is when a compiler, instead of generating loop code for $n$ iterations, generates just $n$ copies of the
  loop body, in order to get rid of the instructions for loop maintenance}\ESph{}\PTBRph{}\DEph{}\PLph{}\ITAph{}
  \FR{C'est lorsqu'un compilateur, au lieu de générer du code pour une boucle de
  $n$ itérations, génère juste $n$ copies du corps de la boucle, afin de supprimer
  les instructions pour la gestion de la boucle}
  }
}

\newglossaryentry{register allocator}
{
  name=register allocator,
  description={\RU{Функция компилятора распределяющая локальные переменные по регистрам процессора}
  \EN{The part of the compiler that assigns CPU registers to local variables}\ESph{}\PTBRph{}\DEph{}\PLph{}\ITAph{}
  \FR{La partie du compilateur qui assigne des registes du CPU aux variables locales}}
}

\newglossaryentry{quotient}
{
  name=\RU{частное}\EN{quotient}\ESph{}\PTBRph{}\DEph{}\PLph{}\ITAph{}\FR{quotient},
  description={\RU{Результат деления}\EN{Division result}\ESph{}\PTBRph{}\DEph{}\PLph{}\ITAph{}
  \FR{Résultat de la division}}
}

\newglossaryentry{product}
{
  name=\RU{произведение}\EN{product}\ESph{}\PTBRph{}\DE{Produkt}\PLph{}\ITAph{}\FR{produit},
  description={\RU{Результат умножения}\EN{Multiplication result}\ESph{}\PTBRph{}\DE{Ergebnis einer Multiplikation}\PLph{}\ITAph{}
  \FR{Résultat d'une multiplication}}
}

\newglossaryentry{NOP}
{
  name=NOP,
  description={\q{no operation}, \RU{холостая инструкция}\EN{idle instruction}\ESph{}\PTBRph{}\DEph{}\PLph{}\ITAph{}
  \FR{instruction ne faisant rien}}
}

\newglossaryentry{POKE}
{
  name=POKE,
  description={\RU{Инструкция языка BASIC записывающая байт по определенному адресу}
  	\EN{BASIC language instruction for writing a byte at a specific address}\ESph{}\PTBRph{}\DEph{}\PLph{}\ITAph{}
    \FR{instruction du langage BASIC pour écrire un octet a une adresse spécifique}}
}

\newglossaryentry{keygenme}
{
  name=keygenme,
  description={\RU{Программа, имитирующая защиту вымышленной программы, для которой нужно сделать 
  генератор ключей/лицензий}\EN{A program which imitates software protection,
  for which one needs to make a key/license generator}\ESph{}\PTBRph{}\DEph{}\PLph{}\ITAph{}
  \FR{Un programme qui imite la protection des logiciels pour lesquels on a besoin d'un générateur de clef/licence}}
} % TODO clarify: A software which generate key/license value to bypass sotfware protection?

\newglossaryentry{dongle}
{
  name=dongle,
  description={\RU{Небольшое устройство подключаемое к LPT-порту для принтера (в прошлом) или к USB}
  \EN{Dongle is a small piece of hardware connected to LPT printer port (in past) or to USB}\ESph{}\PTBRph{}\DEph{}\PLph{}\ITAph{}
  \FR{Un dongle est un petit périphérique se connectant sur un port d'imprimante LPT (par le passé) ou USB}.
  \RU{Исполняло функции security token-а, имела память и, иногда,
  секретную (крипто-)хеширующую функцию}\EN{Its function was similar to a security token, 
  it has some memory and, sometimes, a secret (crypto-)hashing algorithm}\ESph{}\PTBRph{}\DEph{}\PLph{}\ITAph{}i
  \FR{Sa fonction est similaire au tokens de sécurité, il y a de la mémoire et, parfois, un algorithme secret de (crypto-)hachage.}}
}

\newglossaryentry{thunk function}
{
  name=thunk function,
  description={\RU{Крохотная функция делающая только одно: вызывающая другую функцию}
  \EN{Tiny function with a single role: call another function}\ESph{}\PTBRph{}\DEph{}\PLph{}\ITAph{}
  \FR{Minuscule fonction qui a un seul rôle: appeler une autre fonction}}
}

\newglossaryentry{user mode}
{
  name=user mode,
  description={\RU{Режим CPU с ограниченными возможностями в котором он исполняет прикладное ПО. ср.}
  \EN{A restricted CPU mode in which it all application software code is executed. cf.}\ESph{}\PTBRph{}\DEph{}\PLph{}\ITAph{}
  \FR{Un mode CPU restreint dans lequel tout le code des applications est exécuté. cf.} \gls{kernel mode}}
}

\newglossaryentry{kernel mode}
{
  name=kernel mode,
  description={\RU{Режим CPU с неограниченными возможностями в котором он исполняет ядро OS и драйвера. ср.}
  \EN{A restrictions-free CPU mode in which the OS kernel and drivers execute. cf.}\ESph{}\PTBRph{}\DEph{}\PLph{}\ITAph{}
  \FR{Un mode CPU sans restriction dans lequel le noyaux de l'OS et les drivers sont exécutés. cf.} \gls{user mode}}
}

\newglossaryentry{Windows NT}
{
  name=Windows NT,
  description={Windows NT, 2000, XP, Vista, 7, 8, 10}
}

\newglossaryentry{atomic operation}
{
  name=atomic operation,
  description={
  \q{$\alpha{}\tau{}o\mu{}o\varsigma{}$}
  %\q{atomic}
  \RU{означает \q{неделимый} в греческом языке, так что атомарная операция ---
  это операция которая гарантированно не будет прервана другими тредами}
  \EN{stands for \q{indivisible} in Greek, so an atomic operation is guaranteed not
  to be interrupted by other threads}\ESph{}\PTBRph{}\DEph{}\PLph{}\ITAph{}
  \FR{signifie \q{indivisible} en grec, donc il est garantie qu'une opération atomique ne sera pas interrompue par d'autres threads}
  }
}

% to be proofreaded (begin)
\newglossaryentry{NaN}
{
  name=NaN,
  description={
  	\RU{не число: специальные случаи чисел с плавающей запятой, 
  	обычно сигнализирующие об ошибках}\EN{not a number: 
  	a special cases for floating point numbers, usually signaling about errors}\ESph{}\PTBRph{}\DEph{}\PLph{}\ITAph{}
    \FR{pas un nombre: nu cas particulier pour les nombres flottants, signifiant généralement une erreur}
  }
}

\newglossaryentry{basic block}
{
  name=basic block,
  description={
  	\RU{группа инструкций, не имеющая инструкций переходов,
	а также не имеющая переходов в середину блока извне.
	В \IDA он выглядит как просто список инструкций без строк-разрывов}\EN{a group of 
	instructions that do not have jump/branch instructions, and also don't have
	jumps inside the block from the outside.
	In \IDA it looks just like as a list of instructions without empty lines}\ESph{}\PTBRph{}\DEph{}\PLph{}\ITAph{}
  }
}

\newglossaryentry{NEON}
{
  name=NEON,
  description={\ac{AKA} \q{Advanced SIMD}\EMDASH\ac{SIMD} \RU{от}\EN{from}\ESph{}\PTBRph{}\DEph{}\PLph{}\ITAph{}\FR{de} ARM}
}

\newglossaryentry{reverse engineering}
{
  name=reverse engineering,
  description={\RU{процесс понимания как устроена некая вещь, иногда, с целью клонирования оной}
  \EN{act of understanding how the thing works, sometimes in order to clone it}\ESph{}\PTBRph{}\DEph{}\PLph{}\ITAph{}
  \FR{action d'examiner et de comprendre comment quelque chose fonctionne, parfois dans le but de le reproduire}
  }
}

\newglossaryentry{compiler intrinsic}
{
  name=compiler intrinsic,
  description={\RU{Специфичная для компилятора функция не являющаяся обычной библиотечной функцией.
	Компилятор вместо её вызова генерирует определенный машинный код.
	Нередко, это псевдофункции для определенной инструкции \ac{CPU}. Читайте больше:}
	\EN{A function specific to a compiler which is not an usual library function.
	The compiler generates a specific machine code instead of a call to it.
	Often, it's a pseudofunction for a specific \ac{CPU} instruction. Read more:}\ESph{}\PTBRph{}\DEph{}\PLph{}\ITAph{}
    \FR{Une foncion spécifique à un compilateur, qui n'est pas une fonction usuelle de bibliothèquee.
    Le compilateur génère du code machine spcifique au lieu d'un appel à celui-ci.
    Souvent il s'agit d'une pseudo-fonction pour une instruction \ac{CPU} spécifique. Lire plus:} (\myref{sec:compiler_intrinsic})
  }
}

\newglossaryentry{heap}
{
  name=\RU{heap}\EN{heap}\FR{tas},
  description={\RU{(куча) обычно, большой кусок памяти предоставляемый \ac{OS}, так что прикладное ПО может делить его
  как захочет. malloc()/free() работают с кучей}
  \EN{usually, a big chunk of memory provided by the \ac{OS} so that applications can divide it by themselves as they wish.
  malloc()/free() work with the heap}\FR{Généralement c'est un gros bout de mémoire fournit par l'\ac{OS} et utilisé par
  les applications pour le diviser comme elles le souhaitent. malloc()/free() fonctionnent en utilisant le tas}\ESph{}\PTBRph{}\DEph{}\PLph{}\ITAph{}}
}

\newglossaryentry{name mangling}
{
  name=name mangling,
  description={\RU{применяется как минимум в \Cpp, где компилятору нужно закодировать имя класса,
  метода и типы аргументов в одной
  строке, которая будет внутренним именем функции. читайте также здесь}
  \EN{used at least in \Cpp, where the compiler needs to encode the name of class, method and argument types in one string,
  which will become the internal name of the function. You can read more about it here}\ESph{}\PTBRph{}\DEph{}\PLph{}\ITAph{}
  \FR{utilisé au moins en \Cpp, oú le compilateur doit encoder le nom de la classe, la méthode et le type des arguments dans un chaîne,
  qui devient le nom interne de la fonction. Vous pouvez en lire plus à ce propos ici}: \myref{namemangling}}
}

\newglossaryentry{xoring}
{
  name=xoring,
  description={\RU{нередко применяемое в английском языке, означает применение операции 
  \ac{XOR}}
  \EN{often used in the English language, which implying applying the \ac{XOR} operation}\ESph{}\PTBRph{}\DEph{}\PLph{}\ITAph{}
  \FR{souvent utilisé en angalis, qui signifie appliquer l'opération \ac{XOR}}
  }
}

\newglossaryentry{security cookie}
{
  name=security cookie,
  description={\RU{Случайное значение, разное при каждом исполнении. Читайте больше об этом тут}
  \EN{A random value, different at each execution. You can read more about it here}\ESph{}\PTBRph{}\DEph{}\PLph{}\ITAph{}
  \FR{Une valeur aléatoire, différente à chaque exécution. Vous pouvez en lire plus à ce propos ici}: \myref{subsec:BO_protection}}
}

\newglossaryentry{tracer}
{
  name=tracer,
  description={\RU{Моя простейшая утилита для отладки. Читайте больше об этом тут}
  \EN{My own simple debugging tool. You can read more about it here}\ESph{}\PTBRph{}\DEph{}\PLph{}\ITAph{}
  \FR{Mon propre outil de debugging. Vous pouvez en lire plus à son propos ici}: \myref{tracer}}
}

\newglossaryentry{GiB}
{
  name=GiB,
  description={\RU{Гибибайт: $2^{30}$ или 1024 мебибайт или 1073741824 байт}
  \EN{Gibibyte: $2^{30}$ or 1024 mebibytes or 1073741824 bytes}\ESph{}\PTBRph{}\DEph{}\PLph{}\ITAph{}
  \FR{Gibioctet: $2^{30}$ or 1024 mebioctets ou 1073741824 octets} %TODO: translate byte?
  }
}

\newglossaryentry{CP/M}
{
  name=CP/M,
  description={Control Program for Microcomputers: \RU{очень простая дисковая \ac{OS} использовавшаяся перед}
  \EN{a very basic disk \ac{OS} used before}\ESph{}\PTBRph{}\DEph{}\PLph{}\ITAph{}
  \FR{un \ac{OS} de disque trè basique utilisé avant} MS-DOS}
}

\newglossaryentry{stack frame}
{
  name=stack frame,
  description={\RU{Часть стека, в которой хранится информация, связанная с текущей функцией: локальные переменные,
  аргументы функции, \ac{RA}, итд.}\EN{A part of the stack that contains information specific to the current function:
  local variables, function arguments, \ac{RA}, etc.}\ESph{}\PTBRph{}\DEph{}\PLph{}\ITAph{}
  \FR{Une partie de la pile qui contient des informations spécifiques à la fonction courante:
  variables locales, arguments de la fonciton, \ac{RA}, etc.}
  }
}

\newglossaryentry{jump offset}
{
  name=jump offset,
  description={\RU{Часть опкода JMP или Jcc инструкции, просто прибавляется к адресу следующей инструкции,
  и так вычисляется новый \ac{PC}. Может быть отрицательным}\EN{a part of the JMP or Jcc instruction's opcode, 
  to be added to the address
  of the next instruction, and this is how the new \ac{PC} is calculated. May be negative as well}\ESph{}\PTBRph{}\DEph{}\PLph{}\ITAph{}
  \FR{une partie de l'opcode de l'instruction JMP ou Jcc, qui doit être ajoutée à l'adresse de l'instruction suivante,
  et c'est ainsi que le nouveau \ac{PC} est calculé. Peut-être négatif}
  }
}

\newglossaryentry{integral type}
{
  name=\RU{интегральный тип данных}\EN{integral data type}\ESph{}\PTBRph{}\DEph{}\PLph{}\ITAph{},
  description={\RU{обычные числа, но не вещественные. могут использоваться для передачи булевых типов и перечислений (enumerations)}
  \EN{usual numbers, but not a real ones. may be used for passing variables of boolean data type and enumerations}\ESph{}\PTBRph{}\DEph{}\PLph{}\ITAph{}
  \FR{nombre usuel, mais pas un réel. peut être utilisé pour passer des variables de type booléen et des énumérations}
  }
}

\newglossaryentry{real number}
{
  name=\RU{вещественное число}\EN{real number}\ESph{}\PTBRph{}\DEph{}\PLph{}\ITAph{},
  description={\RU{числа, которые могут иметь точку. в \CCpp это \Tfloat и \Tdouble}
  \EN{numbers which may contain a dot. this is \Tfloat and \Tdouble in \CCpp}\ESph{}\PTBRph{}\DEph{}\PLph{}\ITAph{}
  \FR{nombre qui peut contenir un point. ceci est \Tfloat et \Tdouble en \CCpp}
  }
}

\newglossaryentry{PDB}
{
  name=PDB,
  description={(Win32) \RU{Файл с отладочной информацией, обычно просто имена функций, 
  но иногда имена аргументов функций и локальных переменных}
  \EN{Debugging information file, usually just function names, but sometimes also function
  arguments and local variables names}\ESph{}\PTBRph{}\DEph{}\PLph{}\ITAph{}
  \FR{Fichier contenant des informations de débogage, en général seulement les noms des fonctions,
  mais aussi parfois les arguments des fonctions et le nom des variables locales}
  }
}

\newglossaryentry{NTAPI}
{
  name=NTAPI,
  description={\RU{\ac{API} доступное только в линии Windows NT. 
  Большей частью не документировано Microsoft-ом}\EN{\ac{API} available only in the Windows NT line. 
  Largely not documented by Microsoft}\ESph{}\PTBRph{}\DEph{}\PLph{}\ITAph{}
  \FR{i\ac{API} disponible seulement dans la série de Windows NT. Très peu documentée par Microsoft}}
}

\newglossaryentry{stdout}
{
  name=stdout,
  description={standard output}
}

\newglossaryentry{word}
{
  name=word,
  description={\EN{data type fitting in \ac{GPR}}\RU{(слово) тип данных помещающийся в \ac{GPR}\ESph{}\PTBRph{}\DEph{}\PLph{}\ITAph{}}. 
  \RU{В компьютерах старше персональных, память часто измерялась не в байтах, 
  а в словах}\EN{In the computers older than PCs, 
  the memory size was often measured in words rather than bytes}\ESph{}\PTBRph{}\DEph{}\PLph{}\ITAph{}
  \FR{Dans les ordinateurs plus vieux que les PCs, la taille de la mémoire était
  souvent mesurée en mots plutôt qu'en octet}}
}

\newglossaryentry{arithmetic mean}
{
  name=\RU{среднее арифметическое}\EN{arithmetic mean}\ESph{}\PTBRph{}\DEph{}\PLph{}\ITAph{}\FR{moyenne arithmétique},
  description={\EN{a sum of all values divided by their count}
  \RU{сумма всех значений, разделенная на их количество}\ESph{}\PTBRph{}\DEph{}\PLph{}\ITAph{}
  \FR{la somme de toutes les valeurs, divisé par leur nombre}}
}
\newglossaryentry{padding}
{
  name=padding,
  description=
  \EN{\IT{Padding} in English language means to stuff a pillow with something
  to give it a desired (bigger) form.
  In computer science, padding means to add more bytes to a block so it will have desired size, like $2^n$ bytes.}
  \RU{\IT{Padding} в английском языке означает набивание подушки чем-либо для придания ей желаемой (большей)
  формы. В информатике, \IT{padding} означает добавление к блоку дополнительных байт, чтобы он имел нужный
  размер, например, $2^n$ байт.}
  \FR{\IT{Padding} en anglais signifie rembourrer un oreiller, un matelat, etc. avec quelque chose afin de lui donner la forme désirée.
  En informatique, padding signifie ajouter des octets à un bloc, afin qu'il ait une certaine taille, comme $2^n$ octets.}
}


\newcommand{\URLWPDA}
{\RU
 {
  \href{http://go.yurichev.com/17012}{Wikipedia: Выравнивание данных}
 }
 \EN{
  \href{http://go.yurichev.com/17013}{Wikipedia: Data structure alignment}
 }
}

\newcommand{\OracleTablesName}{oracle tables\xspace}
\newcommand{\oracletables}{\OracleTablesName\footnote{\url{http://go.yurichev.com/17014}}\xspace}

\newcommand{\WPMAO}
{\RU
{
    \href{http://go.yurichev.com/17015}{wikipedia: Умножение-сложение}
}
\EN{
    \href{http://go.yurichev.com/17016}{wikipedia: Multiply–accumulate operation}
}
}

\newcommand{\BGREPURL}{\url{http://go.yurichev.com/17017}}
\newcommand{\FNMSDNROTxURL}{\footnote{\url{http://go.yurichev.com/17018}}}

\newcommand{\YurichevIDAIDCScripts}{http://go.yurichev.com/17019}

% for index
\newcommand{\GrepUsage}{\IFRU{Использование grep}{grep usage}}
\newcommand{\SyntacticSugar}{\IFRU{Синтаксический сахар}{Syntactic Sugar}}
\newcommand{\CompilerAnomaly}{\IFRU{Аномалии компиляторов}{Compiler's anomalies}}
\newcommand{\CLanguageElements}{\IFRU{Элементы языка Си}{C language elements}}
\newcommand{\CStandardLibrary}{\IFRU{Стандартная библиотека Си}{C standard library}}
\newcommand{\Flags}{\IFRU{Флаги}{Flags}}
\newcommand{\Registers}{\IFRU{Регистры}{Registers}}
\newcommand{\Stack}{\IFRU{Стек}{Stack}}
\newcommand{\Recursion}{\IFRU{Рекурсия}{Recursion}}
\newcommand{\RAM}{\IFRU{ОЗУ}{RAM}}
\newcommand{\ROM}{\IFRU{ПЗУ}{ROM}}
\newcommand{\Pointers}{\IFRU{Указатели}{Pointers}}
\newcommand{\BufferOverflow}{\IFRU{Переполнение буфера}{Buffer Overflow}}

\newcommand{\Exercise}{\IFRU{Задача}{Exercise}\xspace}
\newcommand{\Cpp}{\IFRU{Си++}{C++}\xspace}
\newcommand{\CCpp}{\IFRU{Си/Си++}{C/C++}\xspace}
\newcommand{\NonOptimizing}{\IFRU{Неоптимизирующий}{Non-optimizing}\xspace}
\newcommand{\Optimizing}{\IFRU{Оптимизирующий}{Optimizing}\xspace}
\newcommand{\NonOptimizingKeil}{\NonOptimizing Keil\xspace}
\newcommand{\OptimizingKeil}{\Optimizing Keil\xspace}
\newcommand{\NonOptimizingXcode}{\NonOptimizing Xcode (LLVM)\xspace}
\newcommand{\OptimizingXcode}{\Optimizing Xcode (LLVM)\xspace}
\newcommand{\ARMMode}{\IFRU{Режим ARM}{ARM mode}\xspace}
\newcommand{\ThumbMode}{\IFRU{Режим thumb}{thumb mode}\xspace}
\newcommand{\ThumbTwoMode}{\IFRU{Режим thumb-2}{thumb-2 mode}\xspace}
\newcommand{\AndENRU}{\IFRU{и}{and}\xspace}
\newcommand{\OrENRU}{\IFRU{или}{or}\xspace}
\newcommand{\InENRU}{\IFRU{в}{in}\xspace}
\newcommand{\ForENRU}{\IFRU{для}{for}\xspace}

\newcommand{\FNQUOTIENT}{\footnote{\IFRU{результат деления}{result of division}}}
\newcommand{\FNPRODUCT}{\footnote{\IFRU{результат умножения}{result of multiplication}}}
\newcommand{\FNSUM}{\footnote{\IFRU{результат сложения}{result of addition}}}

\newcommand{\DataProcessingInstructionsFootNote}{\IFRU{Эти инструкции также называются}
{These instructions are also called} ``data processing instructions''}

\newcommand{\Instructions}{\IFRU{Инструкции}{Instructions}}

% for .bib files
\newcommand{\AlsoAvailableAs}{\IFRU{Также доступно здесь:}{Also available as}\xspace}

% section names
\newcommand{\ShiftsSectionName}{\IFRU{Сдвиги}{Shifts}}
\newcommand{\SignedNumbersSectionName}{\IFRU{Представление знака в числах}{Signed number representations}}
\newcommand{\HelloWorldSectionName}{Hello, world!}
\newcommand{\SwitchCaseDefaultSectionName}{switch()/case/default}
\newcommand{\PrintfSeveralArgumentsSectionName}{\printf \IFRU{с несколькими аргументами}{with several arguments}}
\newcommand{\DivisionByNineSectionName}{\IFRU{Деление на 9}{Division by 9}}
\newcommand{\WorkingWithFloatAsWithStructSubSubSectionName}{\IFRU
{Работа с типом float как со структурой}{Working with the float type as with a structure}}

\newcommand{\StructurePackingSectionName}{\IFRU{Упаковка полей в структуре}{Fields packing in structure}}

\newcommand{\PICcode}{\IFRU{адресно-независимый код}{position-independent code}}
\newcommand{\CapitalPICcode}{\IFRU{Адресно-независимый код}{Position-independent code}}
\newcommand{\Loops}{\IFRU{Циклы}{Loops}}

% C
\newcommand{\PostIncrement}{\IFRU{Пост-инкремент}{Post-increment}}
\newcommand{\PostDecrement}{\IFRU{Пост-декремент}{Post-decrement}}
\newcommand{\PreIncrement}{\IFRU{Пре-инкремент}{Pre-increment}}
\newcommand{\PreDecrement}{\IFRU{Пре-декремент}{Pre-decrement}}

% other
\newcommand{\IntelSyntax}{\IFRU{Синтаксис Intel}{Intel syntax}}
\newcommand{\ATTSyntax}{\IFRU{Синтаксис AT\&T}{AT\&T syntax}}
\newcommand{\randomNoise}{\IFRU{случайный шум}{random noise}}
\newcommand{\Example}{\IFRU{Пример}{Example}}
\newcommand{\argument}{\IFRU{аргумент}{argument}}
\newcommand{\MarkedInIDAAs}{\IFRU{маркируется в \IDA как}{marked in \IDA as}}
\newcommand{\HERMIT}{\IFRU{Андрей}{Andrey} ``herm1t'' \IFRU{Баранович}{Baranovich}}
\newcommand{\stepover}{\IFRU{сделать шаг не входя в ф-цию}{step over}}
\newcommand{\ShortHotKeyCheatsheet}{\IFRU{Краткий справочник хот-кеев}{Short hot-keys cheatsheet}}



\makeglossaries

\newcommand{\TITLE}{\RU{Reverse Engineering для начинающих}\EN{Reverse Engineering for Beginners}}
\newcommand{\AUTHOR}{\RU{Денис Юричев}\EN{Dennis Yurichev}}
\newcommand{\EMAIL}{dennis(a)yurichev.com}

\hypersetup{
    pdftex,
    colorlinks=true,
    allcolors=blue,
    pdfauthor={\AUTHOR},
    pdftitle={\TITLE}
    }

\lstset{
    backgroundcolor=\color{lstbgcolor},
    basicstyle=\ttfamily\small,
    %basicstyle=\ttfamily,
    breaklines=true,
    %prebreak=\raisebox{0ex}[0ex][0ex]{->},
    %postbreak=\raisebox{0ex}[0ex][0ex]{->},
    prebreak=\raisebox{0ex}[0ex][0ex]{\ensuremath{\rhookswarrow}},
    postbreak=\raisebox{0ex}[0ex][0ex]{\ensuremath{\rcurvearrowse\space}},
    frame=single,
    columns=fullflexible,keepspaces,
    inputencoding=utf8
}

\DeclareMathSizes{12}{30}{16}{12}
\bibliography{C_book_\LANG,books,articles,usenet,misc}

\begin{document}

\pagestyle{fancy}

\VerbatimFootnotes

\frontmatter

\begin{titlepage}
\subsection*{%
	\RU{Отзывы о книге}%
	\EN{Praise for}%
	\ES{Elogios para}%
	\PTBRph{}%
	\DEph{}\PLph{}%
	\ITAph{}
	\IT{\TITLE}%
}

\begin{itemize}
% expanded URLs to make it more robust for printouts. In electronic editions people will click anyway, so tracking will keep working
\item \q{It's very well done .. and for free .. amazing.}\footnote{\href{http://go.yurichev.com/17095}{twitter.com/daniel\_bilar/status/436578617221742593}} Daniel Bilar, Siege Technologies, LLC.

\item \q{... excellent and free}\footnote{\href{http://go.yurichev.com/17096}{twitter.com/petefinnigan/status/400551705797869568}} Pete Finnigan,%
	\RU{гуру по безопасности}%
	\ES{gur\'u de seguridad en}%
	\PTBRph{}%
	\DEph{}\PLph{}%
	\ITAph{}
\oracle
	\EN{security guru}.

\item \q{... book is interesting, great job!} Michael Sikorski,
	\RU{автор книги}%
	\EN{author of}%
	\ES{autor de}%
	\PTBRph{}%
	\DEph{}\PLph{}%
	\ITAph{}
\IT{Practical Malware Analysis: The Hands-On Guide to Dissecting Malicious Software}.

\item \q{... my compliments for the very nice tutorial!} Herbert Bos,
	\RU{профессор университета}%
	\EN{full professor at the}%
	\ES{catedr\'atico de tiempo completo en la}%
	\PTBRph{}%
	\DEph{}\PLph{}%
	\ITAph{}
Vrije Universiteit Amsterdam,
	\RU{соавтор}%
	\EN{co-author of}%
	\ES{coautor de}%
	\PTBRph{}%
	\DEph{}\PLph{}%
	\ITAph{}
\IT{Modern Operating Systems (4th Edition)}.

\item \q{... It is amazing and unbelievable.} Luis Rocha, CISSP / ISSAP, Technical Manager, Network \& Information Security at Verizon Business.

\item \q{Thanks for the great work and your book.} Joris van de Vis,
	\RU{специалист по}%
	\ES{especialista en}%
	\PTBRph{}%
	\DEph{}\PLph{}%
	\ITAph{}
SAP Netweaver \& Security
	\EN{specialist}.

\item \q{... reasonable intro to some of the techniques.}\footnote{\href{http://go.yurichev.com/17099}{reddit}} Mike Stay,
	\RU{преподаватель в}%
	\EN{teacher at the}%
	\ES{profesor en el}%
	\PTBRph{}%
	\DEph{}\PLph{}%
	\ITAph{}
Federal Law Enforcement Training Center, Georgia, US.

\item \q{I love this book! I have several students reading it at the moment, plan to use it in graduate course.}\footnote{\href{http://go.yurichev.com/17097}{twitter.com/sergeybratus/status/505590326560833536}}
	\RU{Сергей Братусь}%
	\EN{Sergey Bratus}%
	\ES{Sergey Bratus}%
	\PTBRph{}%
	\DEph{}\PLph{}%
	\ITAph{},
Research Assistant Professor
	\RU{в отделе Computer Science в}%
	\EN{at the Computer Science Department at}%
	\ES{en el Departamento de Ciencias de la Computaci\'on en}%
	\PTBRph{}%
	\DEph{}\PLph{}%
	\ITAph{}
Dartmouth College

\item \q{Dennis @Yurichev has published an impressive (and free!) book on reverse engineering}\footnote{\href{http://go.yurichev.com/17098}{twitter.com/TanelPoder/status/524668104065159169}} Tanel Poder,
	\RU{эксперт по настройке производительности Oracle RDBMS}%
	\EN{Oracle RDBMS performance tuning expert}%
	\ES{experto en afinaci\'on de rendimiento de Oracle RDBMS}%
	\PTBRph{}%
	\DEph{}\PLph{}
	\ITAph{}.

\item \q{This book is some kind of Wikipedia to beginners...} Archer, Chinese Translator, IT Security Researcher.

\RU{\item \q{Прочел Вашу книгу~--- отличная работа, рекомендую на своих курсах студентам
в качестве учебного пособия}. Николай Ильин, преподаватель в ФТИ НТУУ \q{КПИ} и DefCon-UA}
\end{itemize}

\cleardoublepage

\begin{center}
\vspace*{\fill}

{\Huge\TITLE}

\bigskip
\bigskip

\begin{figure}[H]
\centering
%TODO The cover is inherently vector. Any way to avoid bitmap JPG?
\includegraphics[width=0.7\textwidth]{cover.jpg}
\end{figure}

\bigskip

{\hfill \huge \AUTHOR}

\vspace*{\fill}
\end{center}

\newpage

\begin{center}
\vspace*{\fill}
\LARGE \TITLE

\vspace*{\fill}

\large \AUTHOR

\large \TT{<\EMAIL>}
\vspace*{\fill}
\vfill

\ccbyncnd

\textcopyright 2013-2014, \AUTHOR. 

\RU{Это произведение доступно по лицензии Creative Commons «Attribution-NonCommercial-NoDerivs» 
(«Атрибуция — Некоммерческое использование — Без производных произведений») 3.0 Непортированная. 
Чтобы увидеть копию этой лицензии, посетите}
\EN{This work is licensed under the Creative Commons Attribution-NonCommercial-NoDerivs 3.0 Unported License. 
To view a copy of this license, visit} \url{http://creativecommons.org/licenses/by-nc-nd/3.0/}.

\RU{Версия этого текста}\EN{Text version} ({\large \today}).

\RU{Возможно, более новая версии текста, а также англоязычная версия, также доступна по ссылке}
\EN{There is probably a newer version of this text, and Russian language version also accessible at} 
\url{http://beginners.re}.
\ifdefined\ebook
\RU{Версия формата A4 так же доступна по ссылке}
\EN{A4-format version is also available on the page}.
\else
\RU{Версия для электронных читалок так же доступна по ссылке}
\EN{E-book reader version is also available on the page}.
\fi

\RU{Вы также можете подписаться на мой twitter для получения информации о новых версиях этого текста, 
и т.д: \TT{@yurichev}\footnote{\url{https://twitter.com/yurichev}}, 
или facebook\footnote{\url{https://www.facebook.com/Dennis.Yurichev}}
, либо подписаться на список рассылки}
\EN{You may also subscribe to my twitter, to get information about updates of this text, etc: 
\TT{@yurichev}\footnote{\url{https://twitter.com/yurichev}}, 
or facebook\footnote{\url{https://www.facebook.com/Dennis.Yurichev}}
or to subscribe to mailing list}
\footnote{\url{http://yurichev.com/mailing_lists.html}}.

\RU{Обложка нарисована Андреем Нечаевским}\EN{The cover was made by Andy Nechaevsky}: \url{https://www.facebook.com/andydinka}.

\end{center}
\end{titlepage}

\EN{\begin{titlepage}

\begin{center}
\vspace*{\fill}

{\Huge\TITLE}

\bigskip
\bigskip

\begin{figure}[H]
\centering
%TODO The cover is inherently vector. Any way to avoid bitmap JPG?
\includegraphics[width=0.7\textwidth]{cover.jpg}
\end{figure}

\bigskip

{\hfill \huge \AUTHOR}

\vspace*{\fill}
\end{center}

\end{titlepage}

\newpage

\begin{center}
\vspace*{\fill}
{\LARGE \TITLE}

\vspace*{\fill}

{\large \AUTHOR}

{\large \TT{<\EMAIL>}}
\vspace*{\fill}
\vfill

\ccbysa

\textcopyright 2013-2016, \AUTHOR. 

This work is licensed under the Creative Commons Attribution-ShareAlike 4.0 International (CC BY-SA 4.0) license.
To view a copy of this license, visit \url{https://creativecommons.org/licenses/by-sa/4.0/}.

Text version ({\large \today}).

The latest version (and Russian edition) of this text is accessible at \href{http://go.yurichev.com/17009}{beginners.re}.
\ifdefined\ebook
An A4-format version is also available there.
\else
An e-book reader version is also available there.
\fi

You can also follow me on twitter to get information about updates of this text:
\TT{@yurichev}\footnote{\href{http://go.yurichev.com/17021}{twitter.com/yurichev}},
or Facebook\footnote{\url{https://www.facebook.com/dennis.yurichev.5}},
or subscribe to the mailing list
\footnote{\href{http://go.yurichev.com/17020}{yurichev.com}}.

The cover was made by Andy Nechaevsky: \href{http://go.yurichev.com/17023}{facebook}.

\end{center}
}
\RU{\begin{titlepage}

\begin{center}
\vspace*{\fill}

{\Huge\TITLE}

\bigskip
\bigskip

\begin{figure}[H]
\centering
%TODO The cover is inherently vector. Any way to avoid bitmap JPG?
\includegraphics[width=0.7\textwidth]{cover.jpg}
\end{figure}

\bigskip

{\hfill \huge \AUTHOR}

\vspace*{\fill}
\end{center}

\end{titlepage}

\newpage

\begin{center}
\vspace*{\fill}
{\LARGE \TITLE}

\vspace*{\fill}

{\large \AUTHOR}

{\large \TT{<\EMAIL>}}
\vspace*{\fill}
\vfill

\ccbysa

\textcopyright 2013-2016, \AUTHOR. 

Это произведение доступно по лицензии Creative Commons «Attribution-ShareAlike 4.0 International» (CC BY-SA 4.0).
Чтобы увидеть копию этой лицензии, посетите \url{https://creativecommons.org/licenses/by-sa/4.0/}.

Версия этого текста ({\large \today}).

Самая новая версия текста (а также англоязычная версия) доступна на сайте \href{http://go.yurichev.com/17009}{beginners.re}.
\ifdefined\ebook
Версия формата A4 доступна там же.
\else
Версия для электронных читалок так же доступна на сайте.
\fi

Вы также можете подписаться на мой twitter для получения информации о новых версиях этого текста:
\TT{@yurichev}\footnote{\href{http://go.yurichev.com/17021}{twitter.com/yurichev}},
либо Facebook\footnote{\url{https://www.facebook.com/dennis.yurichev.5}},
либо подписаться на список рассылки
\footnote{\href{http://go.yurichev.com/17020}{yurichev.com}}.

Обложка нарисована Андреем Нечаевским: \href{http://go.yurichev.com/17023}{facebook}.

\end{center}
}
\ES{\begin{titlepage}

\begin{center}
\vspace*{\fill}

{\Huge\TITLE}

\bigskip
\bigskip

\begin{figure}[H]
\centering
%TODO The cover is inherently vector. Any way to avoid bitmap JPG?
\includegraphics[width=0.7\textwidth]{cover.jpg}
\end{figure}

\bigskip

{\hfill \huge \AUTHOR}

\vspace*{\fill}
\end{center}

\end{titlepage}

\newpage

\begin{center}
\vspace*{\fill}
{\LARGE \TITLE}

\vspace*{\fill}

{\large \AUTHOR}

{\large \TT{<\EMAIL>}}
\vspace*{\fill}
\vfill

\ccbysa

\textcopyright 2013-2016, \AUTHOR. 

Esta obra est\'a bajo una Licencia Creative Commons ``Attribution-ShareAlike 4.0 International'' (CC BY-SA 4.0)
Para ver una copia de esta licencia, visita \url{https://creativecommons.org/licenses/by-sa/4.0/}.

Versi\'on del texto ({\large \today}).

La \'ultima versi\'on (as\'i como las versiones en ingl\'es y ruso) de este texto est\'a disponible en
\href{http://go.yurichev.com/17009}{beginners.re}.
\ifdefined\ebook
Una versi\'on en formato A4 tambi\'en est\'a disponible.
\else
Una versi\'on para lector de libros electr\'onicos tambi\'en est\'a disponible.
\fi

Adem\'as puedes seguirme en twitter para obtener informaci\'on sobre actualizaciones de este texto:
\TT{@yurichev}\footnote{\href{http://go.yurichev.com/17021}{twitter.com/yurichev}},
\ESph{} Facebook\footnote{\url{https://www.facebook.com/dennis.yurichev.5}},
o subscribirte a la lista de correo
\footnote{\href{http://go.yurichev.com/17020}{yurichev.com}}.

La portada fue hecha por Andy Nechaevsky: \href{http://go.yurichev.com/17023}{facebook}.

\end{center}
}
\ITA{\begin{titlepage}

\begin{center}
\vspace*{\fill}

{\Huge\TITLE}

\bigskip
\bigskip

\begin{figure}[H]
\centering
%TODO The cover is inherently vector. Any way to avoid bitmap JPG?
\includegraphics[width=0.7\textwidth]{cover.jpg}
\end{figure}

\bigskip

{\hfill \huge \AUTHOR}

\vspace*{\fill}
\end{center}

\end{titlepage}

\newpage

\begin{center}
\vspace*{\fill}
{\LARGE \TITLE}

\vspace*{\fill}

{\large \AUTHOR}

{\large \TT{<\EMAIL>}}
\vspace*{\fill}
\vfill

\ccbysa

\textcopyright 2013-2016, \AUTHOR. 

This work is licensed under the Creative Commons Attribution-ShareAlike 4.0 International (CC BY-SA 4.0) license.
To view a copy of this license, visit \url{https://creativecommons.org/licenses/by-sa/4.0/}.

Text version ({\large \today}).

The latest version (and Russian edition) of this text is accessible at \href{http://go.yurichev.com/17009}{beginners.re}.
\ifdefined\ebook
An A4-format version is also available there.
\else
An e-book reader version is also available there.
\fi

The cover was made by Andy Nechaevsky: \href{http://go.yurichev.com/17023}{facebook}.

\end{center}
}
\FR{\begin{titlepage}

\begin{center}
\vspace*{\fill}

{\Huge\TITLE}

\bigskip
\bigskip

\begin{figure}[H]
\centering
%TODO The cover is inherently vector. Any way to avoid bitmap JPG?
\includegraphics[width=0.7\textwidth]{cover.jpg}
\end{figure}

\bigskip

{\hfill \huge \AUTHOR}

\vspace*{\fill}
\end{center}

\end{titlepage}

\newpage

\begin{center}
\vspace*{\fill}
{\LARGE \TITLE}

\vspace*{\fill}

{\large \AUTHOR}

{\large \TT{<\EMAIL>}}
\vspace*{\fill}
\vfill

\ccbysa

\textcopyright 2013-2016, \AUTHOR. 

Ce travail est sous licence Creative Commons Attribution-ShareAlike 4.0 International (CC BY-SA 4.0).
Pour voir une copie de cette licence, rendez vous sur \url{https://creativecommons.org/licenses/by-sa/4.0/}.

Version du texte ({\large \today}).

La dernière version (et édition en russe) de ce texte est accessible sur \href{http://go.yurichev.com/17009}{beginners.re}.

La couverture a été réalisée par Andy Nechaevsky: \href{http://go.yurichev.com/17023}{facebook}.

\end{center}
}


\ifx\LITE\undefined
\shorttoc{\RU{Краткое оглавление}\EN{Short contents}}{-1} % Only sections
\fi
\tableofcontents
\cleardoublepage

\cleardoublepage
\EN{\section*{Preface}

There are several popular meanings of the term \q{\gls{reverse engineering}}:
1) The reverse engineering of software: researching compiled programs;
2) The scanning of 3D structures and the subsequent digital manipulation required in order to duplicate them;
3) Recreating \ac{DBMS} structure.
This book is about the first meaning.

\subsection*{Topics discussed in-depth}

x86/x64, ARM/ARM64, MIPS, Java/JVM.

\subsection*{Topics touched upon}

\oracle (\myref{oracle}),
Itanium (\myref{itanium}),
copy-protection dongles (\myref{dongles}), 
LD\_PRELOAD (\myref{ld_preload}),
stack overflow,
\ac{ELF},
win32 PE file format (\myref{win32_pe}),
x86-64 (\myref{x86-64}),
critical sections (\myref{critical_sections}),
syscalls (\myref{syscalls}), 
\ac{TLS},
position-independent code (\ac{PIC}) (\myref{sec:PIC}), 
profile-guided optimization (\myref{PGO}),
C++ STL (\myref{cpp_STL}),
OpenMP (\myref{openmp}),
SEH (\myref{sec:SEH}).

\subsection*{Prerequisites}

Basic C \ac{PL} knowledge.
Recommended reading: \myref{CCppBooks}.

\subsection*{Exercises and tasks}

\dots 
are all moved to the separate website: \url{http://challenges.re}.

\subsection*{About the author}
\begin{tabularx}{\textwidth}{ l X }

\raisebox{-\totalheight}{
\includegraphics[scale=0.60]{Dennis_Yurichev.jpg}
}

&
Dennis Yurichev is an experienced reverse engineer and programmer.
He can be contacted by email: \textbf{\EMAIL{}}.

% FIXME: no link. \tablefootnote doesn't work
\end{tabularx}

% subsections:
\subsection*{%
	\RU{Отзывы о книге}%
	\EN{Praise for}%
	\ES{Elogios para}%
	\PTBRph{}%
	\DEph{}\PLph{}%
	\ITAph{}
	\IT{\TITLE}%
}

\begin{itemize}
% expanded URLs to make it more robust for printouts. In electronic editions people will click anyway, so tracking will keep working
\item \q{It's very well done .. and for free .. amazing.}\footnote{\href{http://go.yurichev.com/17095}{twitter.com/daniel\_bilar/status/436578617221742593}} Daniel Bilar, Siege Technologies, LLC.

\item \q{... excellent and free}\footnote{\href{http://go.yurichev.com/17096}{twitter.com/petefinnigan/status/400551705797869568}} Pete Finnigan,%
	\RU{гуру по безопасности}%
	\ES{gur\'u de seguridad en}%
	\PTBRph{}%
	\DEph{}\PLph{}%
	\ITAph{}
\oracle
	\EN{security guru}.

\item \q{... book is interesting, great job!} Michael Sikorski,
	\RU{автор книги}%
	\EN{author of}%
	\ES{autor de}%
	\PTBRph{}%
	\DEph{}\PLph{}%
	\ITAph{}
\IT{Practical Malware Analysis: The Hands-On Guide to Dissecting Malicious Software}.

\item \q{... my compliments for the very nice tutorial!} Herbert Bos,
	\RU{профессор университета}%
	\EN{full professor at the}%
	\ES{catedr\'atico de tiempo completo en la}%
	\PTBRph{}%
	\DEph{}\PLph{}%
	\ITAph{}
Vrije Universiteit Amsterdam,
	\RU{соавтор}%
	\EN{co-author of}%
	\ES{coautor de}%
	\PTBRph{}%
	\DEph{}\PLph{}%
	\ITAph{}
\IT{Modern Operating Systems (4th Edition)}.

\item \q{... It is amazing and unbelievable.} Luis Rocha, CISSP / ISSAP, Technical Manager, Network \& Information Security at Verizon Business.

\item \q{Thanks for the great work and your book.} Joris van de Vis,
	\RU{специалист по}%
	\ES{especialista en}%
	\PTBRph{}%
	\DEph{}\PLph{}%
	\ITAph{}
SAP Netweaver \& Security
	\EN{specialist}.

\item \q{... reasonable intro to some of the techniques.}\footnote{\href{http://go.yurichev.com/17099}{reddit}} Mike Stay,
	\RU{преподаватель в}%
	\EN{teacher at the}%
	\ES{profesor en el}%
	\PTBRph{}%
	\DEph{}\PLph{}%
	\ITAph{}
Federal Law Enforcement Training Center, Georgia, US.

\item \q{I love this book! I have several students reading it at the moment, plan to use it in graduate course.}\footnote{\href{http://go.yurichev.com/17097}{twitter.com/sergeybratus/status/505590326560833536}}
	\RU{Сергей Братусь}%
	\EN{Sergey Bratus}%
	\ES{Sergey Bratus}%
	\PTBRph{}%
	\DEph{}\PLph{}%
	\ITAph{},
Research Assistant Professor
	\RU{в отделе Computer Science в}%
	\EN{at the Computer Science Department at}%
	\ES{en el Departamento de Ciencias de la Computaci\'on en}%
	\PTBRph{}%
	\DEph{}\PLph{}%
	\ITAph{}
Dartmouth College

\item \q{Dennis @Yurichev has published an impressive (and free!) book on reverse engineering}\footnote{\href{http://go.yurichev.com/17098}{twitter.com/TanelPoder/status/524668104065159169}} Tanel Poder,
	\RU{эксперт по настройке производительности Oracle RDBMS}%
	\EN{Oracle RDBMS performance tuning expert}%
	\ES{experto en afinaci\'on de rendimiento de Oracle RDBMS}%
	\PTBRph{}%
	\DEph{}\PLph{}
	\ITAph{}.

\item \q{This book is some kind of Wikipedia to beginners...} Archer, Chinese Translator, IT Security Researcher.

\RU{\item \q{Прочел Вашу книгу~--- отличная работа, рекомендую на своих курсах студентам
в качестве учебного пособия}. Николай Ильин, преподаватель в ФТИ НТУУ \q{КПИ} и DefCon-UA}
\end{itemize}

\ifdefined\RUSSIAN
\newcommand{\PeopleMistakesInaccuracies}{Станислав \q{Beaver} Бобрицкий, Александр Лысенко, Shell Rocket, Zhu Ruijin, Changmin Heo, Александр \q{Solar Designer} Песляк, Vitor Vidal, Марк Уилсон.}
\else
\newcommand{\PeopleMistakesInaccuracies}{Stanislav \q{Beaver} Bobrytskyy, Alexander Lysenko, Shell Rocket, Zhu Ruijin, Changmin Heo, Alexander \q{Solar Designer} Peslyak, Vitor Vidal, Mark Wilson.}
\fi

\EN{\input{thanks_EN}}
\ES{\input{thanks_ES}}
\NL{\input{thanks_NL}}
\RU{\input{thanks_RU}}


\subsection*{mini-FAQ}

\par Q: What are prerequisites for reading this book?
\par A: Basic understanding of C/C++ is desirable.

\par Q: Can I buy Russian/English hardcopy/paper book?
\par A: Unfortunately no, no publisher got interested in publishing Russian or English version so far.
Meanwhile, you can ask your favorite copy shop to print/bind it.

\par Q: Is there epub/mobi version?
\par A: The book is highly dependent on TeX/LaTeX-specific hacks, so converting to HTML (epub/mobi is a set of HTMLs)
will not be easy.

\par Q: Why should one learn assembly language these days?
\par A: Unless you are an \ac{OS} developer, you probably don't need to code in assembly\textemdash{}latest compilers (2010s) are much better at performing optimizations than humans \footnote{A very good text about this topic: \InSqBrackets{\AgnerFog}}.

Also, latest \ac{CPU}s are very complex devices and assembly knowledge doesn't really help one to understand their internals.

That being said, there are at least two areas where a good understanding of assembly can be helpful: 
First and foremost, security/malware research. It is also a good way to gain a better understanding of your compiled code whilst debugging.
This book is therefore intended for those who want to understand assembly language rather 
than to code in it, which is why there are many examples of compiler output contained within.

\par Q: I clicked on a hyperlink inside a PDF-document, how do I go back?
\par A: In Adobe Acrobat Reader click Alt+LeftArrow. In Evince click ``<'' button.

\par Q: May I print this book / use it for teaching?
\par A: Of course! That's why the book is licensed under the Creative Commons license (CC BY-SA 4.0).

\par Q: Why is this book free? You've done great job. This is suspicious, as many other free things.
\par A: In my own experience, authors of technical literature do this mostly for self-advertisement purposes. It's not possible to get any decent money from such work.

\par Q: How does one get a job in reverse engineering?
\par A: There are hiring threads that appear from time to time on reddit, devoted to RE\FNURLREDDIT{}
(\RedditHiringThread{}).
Try looking there.

A somewhat related hiring thread can be found in the \q{netsec} subreddit: \NetsecHiringThread{}.

\par Q: I have a question...
\par A: Send it to me by email (\EMAIL).



\subsection*{About the Korean translation}

In January 2015, the Acorn publishing company (\href{http://www.acornpub.co.kr}{www.acornpub.co.kr}) in South Korea did a huge amount of work in translating and publishing 
my book (as it was in August 2014) into Korean.

It's now available at \href{http://go.yurichev.com/17343}{their website}.

\iffalse
\begin{figure}[H]
\centering
\includegraphics[scale=0.3]{acorn_cover.jpg}
\end{figure}
\fi

The translator is Byungho Min (\href{http://go.yurichev.com/17344}{twitter/tais9}).
The cover art was done by my artistic friend, Andy Nechaevsky:
\href{http://go.yurichev.com/17023}{facebook/andydinka}.
They also hold the copyright to the Korean translation.

So, if you want to have a \IT{real} book on your shelf in Korean and 
want to support my work, it is now available for purchase.

\subsection*{About the Persian/Farsi translation}

In 2016 the book has been translated by Mohsen Mostafa Jokar (who is also known to Iranian community by his translation of Radare manual\footnote{\url{http://rada.re/get/radare2book-persian.pdf}}).
It is available on the publisher’s website\footnote{\url{http://goo.gl/2Tzx0H}} (Pendare Pars).

40 page excerpt: \url{https://beginners.re/farsi.pdf}.

Registration of the book in National Library of Iran: \url{http://opac.nlai.ir/opac-prod/bibliographic/4473995}.

\subsection*{About the Chinese translation}

In April 2017, translation to Chinese has been finished by Chinese PTPress publisher. They are also the Chinese translation copyright holder. 

 It's available for order here: \url{http://www.epubit.com.cn/book/details/4174}. Some kind of review and history behind the translation: \url{http://www.cptoday.cn/news/detail/3155}.

Principal translator is Archer, to whom I owe so much. He was extremely meticulous (in good sense) and reported most of known mistakes and bugs, which is very important to literature like this book.
I'll recommend his services to any other author!

Guys from \href{http://www.antiy.net/}{Antiy Labs} has also helped with translation. \href{http://www.epubit.com.cn/book/onlinechapter/51413}{Here is preface} written by them.


}
\RU{\section*{Предисловие}

У термина \q{\gls{reverse engineering}} несколько популярных значений:
1) исследование скомпилированных
программ; 2) сканирование трехмерной модели для последующего копирования;
3) восстановление структуры СУБД. Настоящий сборник заметок
связан с первым значением.

\subsection*{Желательные знания перед началом чтения}

Очень желательно базовое знание \ac{PL} Си.
Рекомендуемые материалы: \myref{CCppBooks}.

\subsection*{Упражнения и задачи}

\dots 
все перемещены на отдельный сайт: \url{http://challenges.re}.

\subsection*{Об авторе}
\begin{tabularx}{\textwidth}{ l X }

\raisebox{-\totalheight}{
\includegraphics[scale=0.60]{Dennis_Yurichev.jpg}
}

&
Денис Юричев~--- опытный reverse engineer и программист.
С ним можно контактировать по емейлу: \textbf{\EMAIL{}} или по Skype: \textbf{dennis.yurichev}.

% FIXME: no link. \tablefootnote doesn't work
\end{tabularx}

% subsections:
\subsection*{%
	\RU{Отзывы о книге}%
	\EN{Praise for}%
	\ES{Elogios para}%
	\PTBRph{}%
	\DEph{}\PLph{}%
	\ITAph{}
	\IT{\TITLE}%
}

\begin{itemize}
% expanded URLs to make it more robust for printouts. In electronic editions people will click anyway, so tracking will keep working
\item \q{It's very well done .. and for free .. amazing.}\footnote{\href{http://go.yurichev.com/17095}{twitter.com/daniel\_bilar/status/436578617221742593}} Daniel Bilar, Siege Technologies, LLC.

\item \q{... excellent and free}\footnote{\href{http://go.yurichev.com/17096}{twitter.com/petefinnigan/status/400551705797869568}} Pete Finnigan,%
	\RU{гуру по безопасности}%
	\ES{gur\'u de seguridad en}%
	\PTBRph{}%
	\DEph{}\PLph{}%
	\ITAph{}
\oracle
	\EN{security guru}.

\item \q{... book is interesting, great job!} Michael Sikorski,
	\RU{автор книги}%
	\EN{author of}%
	\ES{autor de}%
	\PTBRph{}%
	\DEph{}\PLph{}%
	\ITAph{}
\IT{Practical Malware Analysis: The Hands-On Guide to Dissecting Malicious Software}.

\item \q{... my compliments for the very nice tutorial!} Herbert Bos,
	\RU{профессор университета}%
	\EN{full professor at the}%
	\ES{catedr\'atico de tiempo completo en la}%
	\PTBRph{}%
	\DEph{}\PLph{}%
	\ITAph{}
Vrije Universiteit Amsterdam,
	\RU{соавтор}%
	\EN{co-author of}%
	\ES{coautor de}%
	\PTBRph{}%
	\DEph{}\PLph{}%
	\ITAph{}
\IT{Modern Operating Systems (4th Edition)}.

\item \q{... It is amazing and unbelievable.} Luis Rocha, CISSP / ISSAP, Technical Manager, Network \& Information Security at Verizon Business.

\item \q{Thanks for the great work and your book.} Joris van de Vis,
	\RU{специалист по}%
	\ES{especialista en}%
	\PTBRph{}%
	\DEph{}\PLph{}%
	\ITAph{}
SAP Netweaver \& Security
	\EN{specialist}.

\item \q{... reasonable intro to some of the techniques.}\footnote{\href{http://go.yurichev.com/17099}{reddit}} Mike Stay,
	\RU{преподаватель в}%
	\EN{teacher at the}%
	\ES{profesor en el}%
	\PTBRph{}%
	\DEph{}\PLph{}%
	\ITAph{}
Federal Law Enforcement Training Center, Georgia, US.

\item \q{I love this book! I have several students reading it at the moment, plan to use it in graduate course.}\footnote{\href{http://go.yurichev.com/17097}{twitter.com/sergeybratus/status/505590326560833536}}
	\RU{Сергей Братусь}%
	\EN{Sergey Bratus}%
	\ES{Sergey Bratus}%
	\PTBRph{}%
	\DEph{}\PLph{}%
	\ITAph{},
Research Assistant Professor
	\RU{в отделе Computer Science в}%
	\EN{at the Computer Science Department at}%
	\ES{en el Departamento de Ciencias de la Computaci\'on en}%
	\PTBRph{}%
	\DEph{}\PLph{}%
	\ITAph{}
Dartmouth College

\item \q{Dennis @Yurichev has published an impressive (and free!) book on reverse engineering}\footnote{\href{http://go.yurichev.com/17098}{twitter.com/TanelPoder/status/524668104065159169}} Tanel Poder,
	\RU{эксперт по настройке производительности Oracle RDBMS}%
	\EN{Oracle RDBMS performance tuning expert}%
	\ES{experto en afinaci\'on de rendimiento de Oracle RDBMS}%
	\PTBRph{}%
	\DEph{}\PLph{}
	\ITAph{}.

\item \q{This book is some kind of Wikipedia to beginners...} Archer, Chinese Translator, IT Security Researcher.

\RU{\item \q{Прочел Вашу книгу~--- отличная работа, рекомендую на своих курсах студентам
в качестве учебного пособия}. Николай Ильин, преподаватель в ФТИ НТУУ \q{КПИ} и DefCon-UA}
\end{itemize}

\ifdefined\RUSSIAN
\newcommand{\PeopleMistakesInaccuracies}{Станислав \q{Beaver} Бобрицкий, Александр Лысенко, Shell Rocket, Zhu Ruijin, Changmin Heo, Александр \q{Solar Designer} Песляк, Vitor Vidal, Марк Уилсон.}
\else
\newcommand{\PeopleMistakesInaccuracies}{Stanislav \q{Beaver} Bobrytskyy, Alexander Lysenko, Shell Rocket, Zhu Ruijin, Changmin Heo, Alexander \q{Solar Designer} Peslyak, Vitor Vidal, Mark Wilson.}
\fi

\EN{\input{thanks_EN}}
\ES{\input{thanks_ES}}
\NL{\input{thanks_NL}}
\RU{\input{thanks_RU}}


\subsection*{mini-ЧаВО}

\par Q: Что необходимо знать перед чтением книги?
\par A: Желательно иметь базовое понимание Си/Си++.

\par Q: Возможно ли купить русскую/английскую бумажную книгу?
\par A: К сожалению нет, пока ни один издатель не заинтересовался в издании русской или английской версии.
А пока вы можете распечатать/переплести её в вашем любимом копи-шопе или копи-центре.

\par Q: Существует ли версия epub/mobi?
\par A: Книга очень сильно завязана на специфические для TeX/LaTeX хаки, поэтому преобразование в HTML (epub/mobi это набор HTML)
легким не будет.

\par Q: Зачем в наше время нужно изучать язык ассемблера?
\par A: Если вы не разработчик \ac{OS}, вам наверное не нужно писать на ассемблере: современные компиляторы (2010-ые) оптимизируют код намного лучше человека
\footnote{Очень хороший текст на эту тему: \InSqBrackets{\AgnerFog}}.

К тому же, современные \ac{CPU} это крайне сложные устройства и знание ассемблера вряд ли
поможет узнать их внутренности.

Но все-таки остается по крайней мере две области, где знание ассемблера может хорошо помочь:
1) исследование malware (\IT{зловредов}) с целью анализа; 2) лучшее понимание
вашего скомпилированного кода в процессе отладки.
Таким образом, эта книга предназначена для тех, кто хочет скорее понимать ассемблер,
нежели писать на нем, и вот почему здесь масса примеров, связанных с результатами
работы компиляторов.

\par Q: Я кликнул на ссылку внутри PDF-документа, как теперь вернуться назад?
\par A: В Adobe Acrobat Reader нажмите сочетание Alt+LeftArrow. В Evince кликните на ``<''.

\par Q: Могу ли я распечатать эту книгу? Использовать её для обучения?
\par A: Конечно, поэтому книга и лицензирована под лицензией Creative Commons (CC BY-SA 4.0).

\par Q: Почему эта книга бесплатная? Вы проделали большую работу. Это подозрительно, как и многие другие бесплатные вещи.
\par A: По моему опыту, авторы технической литературы делают это, в основном ради само-рекламы. Такой работой заработать приличные деньги невозможно.

\par Q: Как можно найти работу reverse engineer-а?
\par A: На reddit, посвященному RE\FNURLREDDIT, время от времени бывают hiring thread (\RedditHiringThread{}).
Посмотрите там.

В смежном субреддите \q{netsec} имеется похожий тред: \NetsecHiringThread{}.

\par Q: Куда пойти учиться в Украине?
\par A: \href{http://go.yurichev.com/17336}{НТУУ \q{КПИ}: \q{Аналіз програмного коду та бінарних вразливостей}};
\href{http://go.yurichev.com/17337}{факультативы}.

\par Q: У меня есть вопрос...
\par A: Напишите мне его емейлом (\EMAIL).


\subsection*{О переводе на корейский язык}

В январе 2015, издательство Acorn в Южной Корее сделало много работы в переводе 
и издании моей книги (по состоянию на август 2014) на корейский язык.
Она теперь доступна на \href{http://go.yurichev.com/17343}{их сайте}.

\iffalse
\begin{figure}[H]
\centering
\includegraphics[scale=0.3]{acorn_cover.jpg}
\end{figure}
\fi

Переводил Byungho Min (\href{http://go.yurichev.com/17344}{twitter/tais9}).
Обложку нарисовал мой хороший знакомый художник Андрей Нечаевский
\href{http://go.yurichev.com/17023}{facebook/andydinka}.
Они также имеют права на издание книги на корейском языке.
Так что если вы хотите иметь \IT{настоящую} книгу на полке на корейском языке и
хотите поддержать мою работу, вы можете купить её.

\subsection*{О переводе на персидский язык (фарси)}

В 2016 году книга была переведена Mohsen Mostafa Jokar (который также известен иранскому сообществу по переводу руководства Radare\footnote{\url{http://rada.re/get/radare2book-persian.pdf}}).
Книга доступна на сайте издательства\footnote{\url{http://goo.gl/2Tzx0H}} (Pendare Pars).

Первые 40 страниц: \url{https://beginners.re/farsi.pdf}.

Регистрация книги в Национальной Библиотеке Ирана: \url{http://opac.nlai.ir/opac-prod/bibliographic/4473995}.

\subsection*{О переводе на китайский язык}

В апреле 2017, перевод на китайский был закончен китайским издательством PTPress. Они также имеют права на издание книги на китайском языке.

Она доступна для заказа здесь: \url{http://www.epubit.com.cn/book/details/4174}. Что-то вроде рецензии и история о переводе: \url{http://www.cptoday.cn/news/detail/3155}.

Основным переводчиком был Archer, перед которым я теперь в долгу.
Он был крайне дотошным (в хорошем смысле) и сообщил о большинстве известных ошибок и баг, что крайне важно для литературы вроде этой книги.
Я буду рекомендовать его услуги всем остальным авторам!

Ребята из \href{http://www.antiy.net/}{Antiy Labs} также помогли с переводом. \href{http://www.epubit.com.cn/book/onlinechapter/51413}{Здесь предисловие} написанное ими.

}
\ES{% TODO to be synced with EN version
\section*{Pr\'ologo}

Existen muchos significados populares para el t\'ermino \q{\gls{reverse engineering}}:
1) La ingenier\'ia inversa de software: la investigaci\'on de programas compilados;
2) El escaneo de estructuras 3D y la manipulaci\'on digital subsecuente requerida para duplicarlas;
3) La recreaci\'on de la estructura de un \ac{DBMS}.
Este libro es acerca del primer significado.

\subsection*{T\'opicos discutidos a profundidad}

x86/x64, ARM/ARM64, MIPS, Java/JVM.

\subsection*{T\'opicos tocados}

\oracle (\myref{oracle}),
Itanium (\myref{itanium}),
dongles para protecci\'on de copias (\myref{dongles}), 
LD\_PRELOAD (\myref{ld_preload}),
desbordamiento de pila,
\ac{ELF},
formato de archivo win32 PE
(\myref{win32_pe}),
x86-64 (\myref{x86-64}),
secciones cr\'iticas
(\myref{critical_sections}),
llamadas al sistema
(\myref{syscalls}), 
\ac{TLS},
c\'odigo de posici\'on independiente
(\ac{PIC}) (\myref{sec:PIC}), 
profile-guided optimization (\myref{PGO}),
C++ STL (\myref{cpp_STL}),
OpenMP (\myref{openmp}),
SEH (\myref{sec:SEH}).

\subsection*{Ejercicios y tareas}

\dots 
fueron movidos al sitio web: \url{http://challenges.re}.

\subsection*{Sobre el autor}
\begin{tabularx}{\textwidth}{ l X }

\raisebox{-\totalheight}{
\includegraphics[scale=0.60]{Dennis_Yurichev.jpg}
}

&
Dennis Yurichev es un reverser y programador experimentado.
Puede ser contactado por email: \textbf{\EMAIL{}}.

% FIXME: no link. \tablefootnote doesn't work
\end{tabularx}

% subsections:
\subsection*{%
	\RU{Отзывы о книге}%
	\EN{Praise for}%
	\ES{Elogios para}%
	\PTBRph{}%
	\DEph{}\PLph{}%
	\ITAph{}
	\IT{\TITLE}%
}

\begin{itemize}
% expanded URLs to make it more robust for printouts. In electronic editions people will click anyway, so tracking will keep working
\item \q{It's very well done .. and for free .. amazing.}\footnote{\href{http://go.yurichev.com/17095}{twitter.com/daniel\_bilar/status/436578617221742593}} Daniel Bilar, Siege Technologies, LLC.

\item \q{... excellent and free}\footnote{\href{http://go.yurichev.com/17096}{twitter.com/petefinnigan/status/400551705797869568}} Pete Finnigan,%
	\RU{гуру по безопасности}%
	\ES{gur\'u de seguridad en}%
	\PTBRph{}%
	\DEph{}\PLph{}%
	\ITAph{}
\oracle
	\EN{security guru}.

\item \q{... book is interesting, great job!} Michael Sikorski,
	\RU{автор книги}%
	\EN{author of}%
	\ES{autor de}%
	\PTBRph{}%
	\DEph{}\PLph{}%
	\ITAph{}
\IT{Practical Malware Analysis: The Hands-On Guide to Dissecting Malicious Software}.

\item \q{... my compliments for the very nice tutorial!} Herbert Bos,
	\RU{профессор университета}%
	\EN{full professor at the}%
	\ES{catedr\'atico de tiempo completo en la}%
	\PTBRph{}%
	\DEph{}\PLph{}%
	\ITAph{}
Vrije Universiteit Amsterdam,
	\RU{соавтор}%
	\EN{co-author of}%
	\ES{coautor de}%
	\PTBRph{}%
	\DEph{}\PLph{}%
	\ITAph{}
\IT{Modern Operating Systems (4th Edition)}.

\item \q{... It is amazing and unbelievable.} Luis Rocha, CISSP / ISSAP, Technical Manager, Network \& Information Security at Verizon Business.

\item \q{Thanks for the great work and your book.} Joris van de Vis,
	\RU{специалист по}%
	\ES{especialista en}%
	\PTBRph{}%
	\DEph{}\PLph{}%
	\ITAph{}
SAP Netweaver \& Security
	\EN{specialist}.

\item \q{... reasonable intro to some of the techniques.}\footnote{\href{http://go.yurichev.com/17099}{reddit}} Mike Stay,
	\RU{преподаватель в}%
	\EN{teacher at the}%
	\ES{profesor en el}%
	\PTBRph{}%
	\DEph{}\PLph{}%
	\ITAph{}
Federal Law Enforcement Training Center, Georgia, US.

\item \q{I love this book! I have several students reading it at the moment, plan to use it in graduate course.}\footnote{\href{http://go.yurichev.com/17097}{twitter.com/sergeybratus/status/505590326560833536}}
	\RU{Сергей Братусь}%
	\EN{Sergey Bratus}%
	\ES{Sergey Bratus}%
	\PTBRph{}%
	\DEph{}\PLph{}%
	\ITAph{},
Research Assistant Professor
	\RU{в отделе Computer Science в}%
	\EN{at the Computer Science Department at}%
	\ES{en el Departamento de Ciencias de la Computaci\'on en}%
	\PTBRph{}%
	\DEph{}\PLph{}%
	\ITAph{}
Dartmouth College

\item \q{Dennis @Yurichev has published an impressive (and free!) book on reverse engineering}\footnote{\href{http://go.yurichev.com/17098}{twitter.com/TanelPoder/status/524668104065159169}} Tanel Poder,
	\RU{эксперт по настройке производительности Oracle RDBMS}%
	\EN{Oracle RDBMS performance tuning expert}%
	\ES{experto en afinaci\'on de rendimiento de Oracle RDBMS}%
	\PTBRph{}%
	\DEph{}\PLph{}
	\ITAph{}.

\item \q{This book is some kind of Wikipedia to beginners...} Archer, Chinese Translator, IT Security Researcher.

\RU{\item \q{Прочел Вашу книгу~--- отличная работа, рекомендую на своих курсах студентам
в качестве учебного пособия}. Николай Ильин, преподаватель в ФТИ НТУУ \q{КПИ} и DefCon-UA}
\end{itemize}

\ifdefined\RUSSIAN
\newcommand{\PeopleMistakesInaccuracies}{Станислав \q{Beaver} Бобрицкий, Александр Лысенко, Shell Rocket, Zhu Ruijin, Changmin Heo, Александр \q{Solar Designer} Песляк, Vitor Vidal, Марк Уилсон.}
\else
\newcommand{\PeopleMistakesInaccuracies}{Stanislav \q{Beaver} Bobrytskyy, Alexander Lysenko, Shell Rocket, Zhu Ruijin, Changmin Heo, Alexander \q{Solar Designer} Peslyak, Vitor Vidal, Mark Wilson.}
\fi

\EN{\input{thanks_EN}}
\ES{\input{thanks_ES}}
\NL{\input{thanks_NL}}
\RU{\input{thanks_RU}}


\input{FAQ_ES}

\subsection*{Acerca de la traducci\'on al Coreano}

En enero del 2015, la editorial Acorn (\href{http://www.acornpub.co.kr}{www.acornpub.co.kr}) en Corea del Sur realiz\'o una enorme cantidad de trabajo
traduciendo y publicando mi libro (como era en agosto del 2014) en Coreano.
Ahora se encuentra disponible en
\href{http://go.yurichev.com/17343}{su sitio web}.

\iffalse
\begin{figure}[H]
\centering
\includegraphics[scale=0.3]{acorn_cover.jpg}
\end{figure}
\fi

El traductor es Byungho Min (\href{http://go.yurichev.com/17344}{twitter/tais9}).
El arte de la portada fue hecho por mi art\'istico amigo, Andy Nechaevsky
\href{http://go.yurichev.com/17023}{facebook/andydinka}.
Ellos tambi\'en poseen los derechos de autor de la traducci\'on al coreano.
As\'i que, si quieren tener un libro \IT{real} en coreano en su estante
y quieren apoyar mi trabajo, ya se encuentra disponible a la venta.

%\subsection*{About the Persian/Farsi translation}
%TBT

}
\NL{% TODO to be synced with EN version
\section*{Voorwoord}

Er zijn verschillende populaire betekenissen voor de term \q{\gls{reverse engineering}}:
1) Reverse engineeren van software: gecompileerde programma\'s onderzoeken;
2) Scannen van 3D structuren en de onderliggende digitale bewerkingen om deze te kunnen dupliceren;
3) Het nabootsen van een \ac{DBMS} structuur.
Dit boek gaat over de eerste betekenis.

\subsection*{Oefeningen en opdrachten}

\dots 
zijn allen verplaatst naar de website: \url{http://challenges.re}.

\subsection*{Over de auteur}
\begin{tabularx}{\textwidth}{ l X }

\raisebox{-\totalheight}{
\includegraphics[scale=0.60]{Dennis_Yurichev.jpg}
}

&
Dennis Yurichev is een ervaren reverse engineer en programmeur.
Je kan hem contacteren via email: \textbf{\EMAIL{}}.

% FIXME: no link. \tablefootnote doesn't work
\end{tabularx}

% subsections:
\subsection*{%
	\RU{Отзывы о книге}%
	\EN{Praise for}%
	\ES{Elogios para}%
	\PTBRph{}%
	\DEph{}\PLph{}%
	\ITAph{}
	\IT{\TITLE}%
}

\begin{itemize}
% expanded URLs to make it more robust for printouts. In electronic editions people will click anyway, so tracking will keep working
\item \q{It's very well done .. and for free .. amazing.}\footnote{\href{http://go.yurichev.com/17095}{twitter.com/daniel\_bilar/status/436578617221742593}} Daniel Bilar, Siege Technologies, LLC.

\item \q{... excellent and free}\footnote{\href{http://go.yurichev.com/17096}{twitter.com/petefinnigan/status/400551705797869568}} Pete Finnigan,%
	\RU{гуру по безопасности}%
	\ES{gur\'u de seguridad en}%
	\PTBRph{}%
	\DEph{}\PLph{}%
	\ITAph{}
\oracle
	\EN{security guru}.

\item \q{... book is interesting, great job!} Michael Sikorski,
	\RU{автор книги}%
	\EN{author of}%
	\ES{autor de}%
	\PTBRph{}%
	\DEph{}\PLph{}%
	\ITAph{}
\IT{Practical Malware Analysis: The Hands-On Guide to Dissecting Malicious Software}.

\item \q{... my compliments for the very nice tutorial!} Herbert Bos,
	\RU{профессор университета}%
	\EN{full professor at the}%
	\ES{catedr\'atico de tiempo completo en la}%
	\PTBRph{}%
	\DEph{}\PLph{}%
	\ITAph{}
Vrije Universiteit Amsterdam,
	\RU{соавтор}%
	\EN{co-author of}%
	\ES{coautor de}%
	\PTBRph{}%
	\DEph{}\PLph{}%
	\ITAph{}
\IT{Modern Operating Systems (4th Edition)}.

\item \q{... It is amazing and unbelievable.} Luis Rocha, CISSP / ISSAP, Technical Manager, Network \& Information Security at Verizon Business.

\item \q{Thanks for the great work and your book.} Joris van de Vis,
	\RU{специалист по}%
	\ES{especialista en}%
	\PTBRph{}%
	\DEph{}\PLph{}%
	\ITAph{}
SAP Netweaver \& Security
	\EN{specialist}.

\item \q{... reasonable intro to some of the techniques.}\footnote{\href{http://go.yurichev.com/17099}{reddit}} Mike Stay,
	\RU{преподаватель в}%
	\EN{teacher at the}%
	\ES{profesor en el}%
	\PTBRph{}%
	\DEph{}\PLph{}%
	\ITAph{}
Federal Law Enforcement Training Center, Georgia, US.

\item \q{I love this book! I have several students reading it at the moment, plan to use it in graduate course.}\footnote{\href{http://go.yurichev.com/17097}{twitter.com/sergeybratus/status/505590326560833536}}
	\RU{Сергей Братусь}%
	\EN{Sergey Bratus}%
	\ES{Sergey Bratus}%
	\PTBRph{}%
	\DEph{}\PLph{}%
	\ITAph{},
Research Assistant Professor
	\RU{в отделе Computer Science в}%
	\EN{at the Computer Science Department at}%
	\ES{en el Departamento de Ciencias de la Computaci\'on en}%
	\PTBRph{}%
	\DEph{}\PLph{}%
	\ITAph{}
Dartmouth College

\item \q{Dennis @Yurichev has published an impressive (and free!) book on reverse engineering}\footnote{\href{http://go.yurichev.com/17098}{twitter.com/TanelPoder/status/524668104065159169}} Tanel Poder,
	\RU{эксперт по настройке производительности Oracle RDBMS}%
	\EN{Oracle RDBMS performance tuning expert}%
	\ES{experto en afinaci\'on de rendimiento de Oracle RDBMS}%
	\PTBRph{}%
	\DEph{}\PLph{}
	\ITAph{}.

\item \q{This book is some kind of Wikipedia to beginners...} Archer, Chinese Translator, IT Security Researcher.

\RU{\item \q{Прочел Вашу книгу~--- отличная работа, рекомендую на своих курсах студентам
в качестве учебного пособия}. Николай Ильин, преподаватель в ФТИ НТУУ \q{КПИ} и DefCon-UA}
\end{itemize}

\ifdefined\RUSSIAN
\newcommand{\PeopleMistakesInaccuracies}{Станислав \q{Beaver} Бобрицкий, Александр Лысенко, Shell Rocket, Zhu Ruijin, Changmin Heo, Александр \q{Solar Designer} Песляк, Vitor Vidal, Марк Уилсон.}
\else
\newcommand{\PeopleMistakesInaccuracies}{Stanislav \q{Beaver} Bobrytskyy, Alexander Lysenko, Shell Rocket, Zhu Ruijin, Changmin Heo, Alexander \q{Solar Designer} Peslyak, Vitor Vidal, Mark Wilson.}
\fi

\EN{\input{thanks_EN}}
\ES{\input{thanks_ES}}
\NL{\input{thanks_NL}}
\RU{\input{thanks_RU}}


%\input{FAQ_NL} % to be translated

% {\RU{Целевая аудитория}\EN{Target audience}}

\subsection*{Over de Koreaanse vertaling}

In Januari 2015 heeft de Acorn uitgeverij (\href{http://www.acornpub.co.kr}{www.acornpub.co.kr}) in Zuid Korea een enorme hoeveelheid werk verricht in het vertalen en uitgeven
van mijn boek (zoals het was in augustus 2014) in het Koreaans.

Het is nu beschikbaar op
\href{http://go.yurichev.com/17343}{hun website}

\iffalse
\begin{figure}[H]
\centering
\includegraphics[scale=0.3]{acorn_cover.jpg}
\end{figure}
\fi

De vertaler is Byungho Min (\href{http://go.yurichev.com/17344}{twitter/tais9}).
De cover art is verzorgd door mijn artistieke vriend, Andy Nechaevsky
\href{http://go.yurichev.com/17023}{facebook/andydinka}.
Zij bezitten ook de auteursrechten voor de Koreaanse vertaling.
Dus, als je een \IT{echt} boek op je kast wil in het Koreaans en je
wil mijn werk steunen, is het nu beschikbaar voor verkoop.

%\subsection*{About the Persian/Farsi translation}
%TBT

}
\IT{\input{preface_IT}}



\mainmatter

\ifx\LITE\undefined
\def\IncludeARM{}
\def\IncludeOlly{}
\def\IncludeGDB{}
\def\IncludeExercises{}
\def\IncludeHiew{}
\def\IncludeCPP{}
%\def\IncludeMIPS{}
\fi

% only chapters here!
\ifdefined\SPANISH
\part{Patrones de código}

\epigraph{Todo es relativo}{Autor desconocido}
\fi % SPANISH

\ifdefined\ENGLISH
\part{Code patterns}

\epigraph{Everything is comprehended in comparison}{Author unknown}
% this is popular Russian proverb and is close to "everything is comprehended in comparison", but the source is lost, however, 
% it's traditionally attributed to all sorts of philosophers..
% I don't know exact analgoue in English language, but OK, let it be so.
\fi % ENGLISH

\ifdefined\RUSSIAN
\part{Образцы кода}
\epigraph{Всё познается в сравнении}{Автор неизвестен}
\fi % RUSSIAN

\ifdefined\BRAZILIAN
\part{Padrões de códigos}
\fi % BRAZILIAN

\ifdefined\THAI
\part{รูปแบบของโค้ด}
\fi % THAI

% chapters
\ifdefined\SPANISH
\chapter{\ESph{}}

Cuando el autor de este libro comenzó a aprender C y, más tarde, \Cpp, él solía escribir pequeños trozos de código, compilarlos, 
y luego ver los resultados en lenguaje assembly. Esto lo hizo muy fácil para él entender lo que estaba pasando en el código que había escrito.
\footnote{De hecho, todavia lo hace cuando no puede entender lo que hace una determinada pieza de código.}. 
Él lo hizo tantas veces que la relación entre el código \CCpp y lo que el compilador producido se imprimió profundamente en su mente. 
És fácil imaginar al instante un esbozo de la aparencia y función del código C. 
Quizás esta técnica podría ser útil para otra persona.

%Hay una serie de ejemplos, tanto para x86/x64 y ARM.
%Los que ya están familiarizados con alguna de las arquitecturas, pueden leer superficialmente las páginas siguientes.

En ciertas partes, se han empleado aquí compiladores muy antiguas, con el fin de obtener lo mas corta (o simple) posible snippet.
\fi % SPANISH

\ifdefined\ENGLISH
\chapter{My method}

When the author of this book first started learning C and, later, \Cpp, he used to write small pieces of code, compile them, 
and then look at the assembly language output. This made it very easy for him to understand what was going on in the code that he had written.
\footnote{In fact, he still does it when he can't understand what a particular bit of code does.}. 
He did it so many times that the relationship between the \CCpp code and what the compiler produced was imprinted deeply in his mind. 
It's easy to imagine instantly a rough outline of C code's appearance and function. 
Perhaps this technique could be helpful for others.

%There are a lot of examples for both x86/x64 and ARM.
%Those who already familiar with one of architectures, may freely skim over pages.

Sometimes ancient compilers are used here, in order to get the shortest (or simplest) possible code snippet.
\fi % ENGLISH

\ifdefined\RUSSIAN
\chapter{Мой метод}

Когда автор этой книги учил Си, а затем \Cpp, он просто писал небольшие фрагменты кода, компилировал и смотрел, что 
получилось на ассемблере. Так было намного проще понять%
\footnote{Честно говоря, он и до сих пор так делает, когда не понимает, как работает некий код.}.
Он делал это такое количество раз, что связь между кодом на \CCpp и тем, что генерирует компилятор, вбилась в его подсознание достаточно глубоко.
После этого не трудно, глядя на код на ассемблере, сразу в общих чертах понимать, что там было написано на Си. 
Возможно это поможет кому-то ещё.

%Здесь много примеров и для x86/x64 и для ARM.
%Те, кто уже хорошо знаком с одной из архитектур, могут легко пролистывать страницы.

Иногда здесь используются достаточно древние компиляторы, чтобы получить самый короткий (или простой) фрагмент кода.
\fi % RUSSIAN

\ifdefined\BRAZILIAN
\chapter{\PTBRph{}}

Quando o autor desse livro começou a aprender C e depois C++, ele costumava escrever pequenos pedaços de código, compilar e entã o olhar na sua saída em assembly.
Isso acabou tornando muito fácil para o seu entendimento sobre o que estava acontecendo no código que ele escreveu.
Ele fez isso tantas vezes, que a relação entre o código em \CCpp e o que o compilador produzia ficou gravada em sua mente.
É facil imaginar a aparência e função de um rascunho em C. Algumas vezes essa técnica pode ser útil para outras pessoas.

Ás vezes compiladores antigos serão usados aqui com o objetivo de conseguir o menor (ou mais simples) pedaço de código possível.

\iffalse
% other version...
Quando o autor deste livro começou a aprender C e, mais tarde, \Cpp, ele costumava escrever pequenos pedaços de código, compilá-los, 
e então olhar a saída em linguagem assembly. Isso tornou muito fácil para ele entender o que estava acontecendo no código que ele tinha escrito.
\footnote{Na verdade, ele ainda faz isso quando não consegue entender o que faz um determinado pedaço de código.}. 
Ele fez isso tantas vezes que o relacionamento entre o código \CCpp code e o que o compilador produzia ficou registrado profundamente em sua mente. 
É fácil imaginar de imediato um esboço da aparência e função do código C. 
Talvez essa técnica poderia ser útil para mais alguém.

%Há uma série de exemplos para ambos x86/x64 e ARM.
%Aqueles já familiarizados com alguma das arquiteturas, pode ler superficialmente as próximas páginas.

Em determinadas partes foram usados aqui compiladores muito antigos, para se obter o menor (ou mais simples) snippet possível.
\fi
\fi % BRAZILIAN

\ifdefined\THAI
\chapter{\THph{}}

% google translate: The format of the code

เมื่อครั้งที่ผู้แต่งหนังสือเริ่มต้นเรียนรู้ภาษา C และในเวลาถัดมา C++ เขาเคยเขียนโค้ดสั้น ๆ และทำการคอมไพล์ แล้วสังเกตดูผลลัพธ์ที่ได้จากการคอมไพล์ที่เป็นภาษาแอสแซมบลี ซึ่งช่วยให้เข้าใจได้ง่ายขึ้นว่าโค้ดที่เขียนนั้นมีการทำงานอย่างไร เขาทำแบบนั้นอยู่หลายครั้งจนความสัมพันธ์ระหว่างโค้ดภาษา C/C++ กับผลลัพธ์ที่ได้จากการคอมไพล์ประทับอยู่ในจิตใจ มันเป็นการง่ายมากที่จะนึกถึงถึงรูปร่างและฟังก์ชั่นของโค้ดภาษา C ขึ้นมาทันที บางครั้งเทคนิคนี้อาจจะมีประโยชน์กับคนอื่นก็ได้	
% rough translation (using google) of the text above: "When the author started learning C and C ++ in the next period, he had written a short code and compile. Then see the results of the compilation of the Assembly's language. This allows more easily understand what the code is written work, however. He did that several times the relationship between the code language C / C ++ with the results from the compilation is in the mind. It's very easy to think of the shape and function of the C language code immediately, sometimes this technique could be useful to someone else."
\fi % THAI

\ifdefined\IncludeExercises
\section*{\Exercises}

\ifdefined\RUSSIAN
Когда автор этой книги учил ассемблер, он также часто компилировал короткие функции на Си и затем постепенно 
переписывал их на ассемблер, с целью получить как можно более короткий код.
Наверное, этим не стоит заниматься в наше время на практике (потому что конкурировать с современными
компиляторами в плане эффективности очень трудно), но это очень хороший способ разобраться в ассемблере
лучше.
Так что вы можете взять любой фрагмент кода на ассемблере в этой книге и постараться сделать его короче.
Но не забывайте о тестировании своих результатов.
\fi % RUSSIAN

\ifdefined\ENGLISH
When the author of this book studied assembly language, he also often compiled small C-functions and then rewrote
them gradually to assembly, trying to make their code as short as possible.
This probably is not worth doing in real-world scenarios today, 
because it's hard to compete with modern compilers in terms of efficiency. It is, however, a very good way to gain a better understanding of assembly.
Feel free, therefore, to take any assembly code from this book and try to make it shorter.
However, don't forget to test what you have written.
\fi % ENGLISH

\ifdefined\BRAZILIAN
Quando o autor deste livro estudou a linguagem assembly, ele também frequentemente compilava pequenas funções em C e então as reescrevia gradualmente em assembly, tentando fazer seu código o menor possível.
Provavelmente não vale mais à pena fazer isso em cenários reais atualmente, 
porque é difícil competir com os compiladores modernos em termos de eficiência. É, no entanto, uma forma muito boa de obter um melhor entendimento de assembly.
Sinta-se livre, portanto, para pegar qualquer código assembly deste livro e tentar torná-lo menor.
No entanto, não esqueça de testar o que você tiver escrito.
\fi % BRAZILIAN

\ifdefined\SPANISH
Cuando el autor de este libro estudió la lenguaje assembly, también con frecuencia compilaba pequeñas funciones en C, y reescribia gradualmente en assembly, tratando de hacer el código lo más pequeño posible.
Probablemente no vale la pena hacer esto en escenarios reales actualmente, 
porque es dificil competir con los compiladores modernos en términos de eficiencia. Es, sin embargo, una muy buena manera de obtener una mejor compreensión de la assembly.
Siéntase libre, por lo tanto, para tomar cualquier código de este libro y tratar de hacerlo más pequeño.
Sin embargo, no se olvide de probar lo que has escrito.
\fi % SPANISH

\fi % IncludeExercises

% rewrote to show that debug\release and optimisations levels are orthogonal concepts.
\ifdefined\RUSSIAN
\section*{Уровни оптимизации и отладочная информация}

Исходный код можно компилировать различными компиляторами с различными уровнями оптимизации.
В типичном компиляторе этих уровней около трёх, где нулевой уровень~--- отключить оптимизацию.
Различают также направления оптимизации кода по размеру и по скорости.
Неоптимизирующий компилятор работает быстрее, генерирует более понятный (хотя и более объемный) код.
Оптимизирующий компилятор работает медленнее и старается сгенерировать более быстрый (хотя и не обязательно краткий) код.
Наряду с уровнями и направлениями оптимизации компилятор может включать в конечный файл отладочную информацию,
производя таким образом код, который легче отлаживать.
Одна очень важная черта отладочного кода в том, что он может содержать
связи между каждой строкой в исходном коде и адресом в машинном коде.
Оптимизирующие компиляторы обычно генерируют код, где целые строки из исходного кода
могут быть оптимизированы и не присутствовать в итоговом машинном коде.
Практикующий reverse engineer обычно сталкивается с обоими версиями, потому что некоторые разработчики
включают оптимизацию, некоторые другие --- нет. Вот почему мы постараемся поработать с примерами для обоих версий.
\fi % RUSSIAN

\ifdefined\ENGLISH
\section*{Optimization levels and debug information}

Source code can be compiled by different compilers with various optimization levels.
A typical compiler has about three such levels, where level zero means disable optimization.
Optimization can also be targeted towards code size or code speed.
A non-optimizing compiler is faster and produces more understandable (albeit verbose) code,
whereas an optimizing compiler is slower and tries to produce code that runs faster (but is not necessarily more compact).
In addition to optimization levels and direction, a compiler can include in the resulting file some debug information,
thus producing code for easy debugging.
One of the important features of the ´debug' code is that it might contain links
between each line of the source code and the respective machine code addresses.
Optimizing compilers, on the other hand, tend to produce output where entire lines of source code
can be optimized away and thus not even be present in the resulting machine code.
Reverse engineers can encounter either version, simply because some developers turn on the compiler's optimization flags and others do not. 
Because of this, we'll try to work on examples of both debug and release versions of the code featured in this book, where possible.
\fi % ENGLISH

\ifdefined\BRAZILIAN
\section*{Níveis de otimização e informações de depuração}

O código-fonte pode ser compilado por um número diferente de compiladores com vários níveis de otimização.
Um compilador típico tem por volta de três desses níveis, onde o nível zero representa que a otimização está desabilitada.
A otimização também pode ser relacionada com o tamanho ou velocidade do código.
Um compilador não-otimizado é mais rápido e produz um código de mais fácil compreensão (embora detalhado),
enquanto um compilador com otimização é mais lento e tenta produzir códigos que executam mais rápido (mas não necessariamente mais compactos).
Em adição aos níveis de otimização e  direção, um compilador pode incluir no arquivo de saída algumas informações de depuração, dessa maneira produzindo código para fácil depuração.
Uma das características de um código ``depurado'' é que ele pode conter ligações entre cada linha do código-fonte e o endereço do respectivo código de máquina.
Compiladores otimizadores, por outro lado, tendem a produzir saídas onde linhas completas do código-fonte podem ser otimizadas e apresentadas de uma maneira completamente diferente,
muitas vezes ainda nem estando presente no código de máquina resultate. Com a engenharia reversa podemos obter quaisquer versões,
simplesmente porque alguns desenvolvedores ativam as otimizações do compilador e outros não.
Por causa disso, nós tentaremos trabalhar em ambos exemplos de depuração e versões de lançamento dos códigos demonstrados nesse livro, quando possível.

\iffalse
% another version
\section*{Níveis de otimização e informação de depuração}

O código-fonte pode ser compilado por diferentes compiladores com vários níveis de otimização.
Um compilador típico tem cerca de três destes níveis, onde o nível zero significa desativar a otimização.
A otimização também pode ser direcionada para o tamanho do código ou para a velocidade do código.
Um compilador sem otimização é mais rápido e produz código mais inteligível (embora maior),
enquanto que um compilador com otimização é mais lento e tenta produzir um código que execute mais rápido (mas não é necessariamente mais compacto).
Além dos níveis e direcionamento da otimização, o compilador pode incluir no arquivo resultante algumas informações de depuração, produzindo assim código para fácil depuração.
Uma das características importantes do código de ´debug' é que ele pode conter 
ligações entre cada linha do código-fonte e os respectivos endereços de código de máquina.
Compiladores com otimização, por outro lado, tendem a produzir uma saída onde linhas inteiras de código-fonte podem ser otimizadas a ponto de serem removidas e portanto não estarem presentes no código de máquina resultante.
Engenheiros Reversos podem encontrar ambas as versões, simplesmente porque alguns desenvolvedores ativam as flags de otimização do compilador e outros não ativam. 
Por causa disso, nós tentaremos trabalhar em exemplos de ambas as versões de debug e release do código destacado neste livro, onde possível.
\fi
\fi % BRAZILIAN

\ifdefined\SPANISH
\section*{Níveles de optimización y la información de depuración}
El código fuente puede ser compilado por diferentes compiladores com varios niveles de optimización.
Un compilador típico tiene alredor de tres de esos niveles, donde el nivel cero significa desactivar la optimización.
La optimización también puede dirigirse hacia el tamaño del código o la velocidad de código.
Un compilador sin optimización es más rápido y produce código más inteligible (aunque más grande), 
mientras un compilador con optimización es más lento y trata de producir un código que corre más rápido (pero no necesariamente más compacto).
Además de los niveles y dirección de la otimización, el compilador puede incluir informaciones de depuración en el archivo resultante, produciendo así código para fácil depuración.
Una de los características importantes del código de ´debug' és que puede contener enlaces entre
cada línea del código fuente y las direcciones de código de máquina respectivos.
Compiladores con optimización, por otro lado, tienden a producir una salida donde líneas enteras de código fuente pueden ser optimizados al punto de ser eliminados y por consiguiente no estar presentes en el código de máquina resultante.
Ingenieros Inversos pueden encontrar ambas versiones, simplesmente porque alguns desarrolladores activan los flags de optimización del compilador, y otros no activan. 
Debido a esto, vamos a tratar de trabajar con ejemplos de ambas versiones de debug y release del código resaltado en este libro, cuando sea posible.
\fi % SPANISH



\RU{\chapter{Некоторые базовые понятия}}
\EN{\chapter{Some basics}}

% sections:
\ifdefined\ENGLISH
\section{A short introduction to the CPU}

The \ac{CPU} is the device that executes the machine code a program consists of.

\textbf{A short glossary:}

\begin{description}
\item[Instruction]: A primitive \ac{CPU} command.
The simplest examples include: moving data between registers, working with memory, primitive arithmetic operations.
As a rule, each \ac{CPU} has its own instruction set architecture (\ac{ISA}).

\item[Machine code]: Code that the \ac{CPU} directly processes. 
Each instruction is usually encoded by several bytes.
\item[Assembly language]: Mnemonic code and some extensions like macros that are intended to make a programmer's life easier.
\item[CPU register]: Each \ac{CPU} has a fixed set of general purpose registers (\ac{GPR}).
$\approx 8$ in x86, $\approx 16$ in x86-64, $\approx 16$ in ARM.
The easiest way to understand a register is to think of it as an untyped temporary variable.
Imagine if you were working with a high-level \ac{PL} and could only use eight 32-bit (or 64-bit) variables.
Yet a lot can be done using just these!
\end{description}

% TODO1 add about linker: "компоновщик" и "редактор связей" в русскоязычной лит-ре

% A note on the experiments in this area (like the LISP machines http://en.wikipedia.org/wiki/Lisp_machine
% might be useful
One might wonder why there needs to be a difference between machine code and a \ac{PL}.  The answer lies in the fact that humans and \ac{CPU}s are not alike---%
it is much easier for humans to use a high-level \ac{PL} like \CCpp, Java, Python, etc., but it is easier for a \ac{CPU} to use a much lower level of abstraction.
Perhaps it would be possible to invent a \ac{CPU} that can execute high-level \ac{PL} code, but it would be many times more complex than the \ac{CPU}s we know of today.
In a similar fashion, it is very inconvenient for humans to write in assembly language, due to it being so low-level and difficult to write in without making a huge number of annoying mistakes.
The program that converts the high-level \ac{PL} code into assembly is called a \IT{compiler}.
\footnote{Old-school Russian literature also use term \q{translator}.}.

\ifx\LITE\undefined
\index{ARM!\ARMMode}%
\index{ARM!\ThumbMode}%
\index{ARM!\ThumbTwoMode}%

\subsection{A couple of words about different \ac{ISA}s}
The x86 \ac{ISA} has always been one with variable-length opcodes, so when the 64-bit era came, the x64 extensions did not impact the \ac{ISA} very significantly. In fact, the x86 \ac{ISA} still contains a lot of instructions that first appeared in 16-bit 8086 CPU, yet are still found in the CPUs of today.
ARM is a \ac{RISC} \ac{CPU} designed with constant-length opcode in mind, which had some advantages in the past.
In the very beginning, all ARM instructions were encoded in 4 bytes%
\footnote{
By the way, fixed-length instructions are handy because one can calculate the next (or previous) 
instruction address without effort. This feature will be discussed in the switch() operator~(\myref{sec:SwitchARMLot}) section.
}.
This is now referred to as \q{ARM mode}.
Then they thought it wasn't as frugal as they first imagined.
In fact, most used \ac{CPU} instructions\footnote{These are MOV/PUSH/CALL/Jcc} in real world applications can be encoded using less information.
They therefore added another \ac{ISA}, called Thumb, where each instruction was encoded in just 2 bytes.
This is now referred as \q{Thumb mode}.
However, not \IT{all} ARM instructions can be encoded in just 2 bytes, so the Thumb instruction set is somewhat limited.
It is worth noting that code compiled for ARM mode and Thumb mode may of course coexist within one single program.
The ARM creators thought Thumb could be extended, giving rise to Thumb-2, which appeared in ARMv7.
Thumb-2 still uses 2-byte instructions, but has some new instructions which have the size of 4 bytes.
There is a common misconception that Thumb-2 is a mix of ARM and Thumb. This is incorrect. 
Rather, Thumb-2 was extended to fully support all processor features so it could
compete with ARM mode---a goal that was clearly achieved, as the majority of applications for \idevices are compiled for the Thumb-2 instruction set (admittedly, largely due to the fact that Xcode does this by default).
Later the 64-bit ARM came out. This \ac{ISA} has 4-byte opcodes, and lacked the need of any additional Thumb mode.
However, the 64-bit requirements affected the \ac{ISA}, resulting in us now having three ARM instruction sets: ARM mode, Thumb mode (including Thumb-2) and ARM64.
These \ac{ISA}s intersect partially, but it can be said that they are different \ac{ISA}s, rather than variations of the same one.
Therefore, we would try to add fragments of code in all three ARM \ac{ISA}s in this book.
\index{PowerPC}%
\index{MIPS}%
\index{Alpha AXP}%
There are, by the way, many other \ac{RISC} \ac{ISA}s with fixed length 32-bit opcodes, such as MIPS, PowerPC and Alpha AXP.
\fi % LITE
\fi % ENGLISH

\ifdefined\RUSSIAN
\section{Краткое введение в CPU}

\ac{CPU} это устройство исполняющее все программы.

\textbf{Немного терминологии:}

\begin{description}
\item[Инструкция]: примитивная команда \ac{CPU}.
Простейшие примеры: перемещение между регистрами, работа с памятью, примитивные арифметические операции.
Как правило, каждый \ac{CPU} имеет свой набор инструкций (\ac{ISA}).

\item[Машинный код]: код понимаемый \ac{CPU}. 
Каждая инструкция обычно кодируется несколькими байтами.
\item[Язык ассемблера]: машинный код плюс некоторые расширения, призванные облегчить труд программиста: макросы, имена, \etc.
\item[Регистр CPU]: Каждый \ac{CPU} имеет некоторый фиксированный набор регистров общего назначения (\ac{GPR}).
$\approx 8$ в x86, $\approx 16$ в x86-64, $\approx 16$ в ARM.
Проще всего понимать регистр как временную переменную без типа.
Можно представить, что вы пишете на \ac{PL} высокого уровня и у вас только 8 переменных шириной 32 (или 64) бита.
Можно сделать очень много используя только их!
\end{description}

% TODO1 add about linker: "компоновщик" и "редактор связей" в русскоязычной лит-ре

Откуда взялась разница между машинным кодом и \ac{PL} высокого уровня?  Ответ в том, что люди и \ac{CPU}-ы отличаются друг от друга ---
человеку проще писать на \ac{PL} высокого уровня вроде \CCpp, Java, Python, а \ac{CPU} проще работать с абстракциями куда более низкого уровня.
Возможно, можно было бы придумать \ac{CPU} исполняющий код \ac{PL} высокого уровня, но он был бы значительно сложнее, чем те, что мы имеем сегодня.
И наоборот, человеку очень неудобно писать на ассемблере из-за его низкоуровневости, к тому же, крайне трудно обойтись без мелких ошибок.
Программа, переводящая код из \ac{PL} высокого уровня в ассемблер называется \IT{компилятором}
\footnote{В более старой русскоязычной литературе также часто встречается термин \q{транслятор}.}.

\ifx\LITE\undefined
\index{ARM!\ARMMode}%
\index{ARM!\ThumbMode}%
\index{ARM!\ThumbTwoMode}%

\subsection{Несколько слов о разнице между \ac{ISA}}
x86 всегда был архитектурой с опкодами переменной длины, так что когда пришла 64-битная эра, расширения x64 не очень сильно повлияли на \ac{ISA}.
ARM это \ac{RISC}-процессор разработанный с учетом опкодов одинаковой длины, что было некоторым преимуществом в прошлом.
Так что в самом начале все инструкции ARM кодировались 4-мя байтами%
\footnote{
Кстати, инструкции фиксированного размера удобны тем, что всегда можно легко узнать адрес 
следующей (или предыдущей) инструкции. Эта особенность будет рассмотрена в секции об операторе switch()~(\myref{sec:SwitchARMLot}).
}.
Это то, что сейчас называется \q{режим ARM}.
Потом они подумали, что это не очень экономично.
На самом деле, самые используемые инструкции\footnote{А это MOV/PUSH/CALL/Jcc} процессора на практике могут быть закодированы c использованием меньшего количества информации.
Так что они добавили другую \ac{ISA} с названием Thumb, где каждая инструкция кодируется всего лишь 2-мя байтами.
Теперь это называется \q{режим Thumb}.
Но не все инструкции ARM могут быть закодированы в двух байтах, так что набор инструкций Thumb ограниченный.
Код, скомпилированный для режима ARM и Thumb может сосуществовать в одной программе.
Затем создатели ARM решили, что Thumb можно расширить: так появился Thumb-2 (в ARMv7).
Thumb-2 это всё ещё двухбайтные инструкции, но некоторые новые инструкции имеют длину 4 байта.
Распространено заблуждение, что Thumb-2\EMDASH{}это смесь ARM и Thumb. Это не верно. Режим Thumb-2 был дополнен до
более полной поддержки возможностей процессора и теперь может легко конкурировать с режимом ARM.
Основное количество приложений для \idevices скомпилировано для набора инструкций Thumb-2, потому что Xcode
делает так по умолчанию.
Потом появился 64-битный ARM. Это \ac{ISA} снова с 4-байтными опкодами, без дополнительного режима Thumb.
Но 64-битные требования повлияли на \ac{ISA}, так что теперь у нас 3 набора инструкций ARM: режим ARM, режим Thumb (включая Thumb-2) и ARM64.
Эти наборы инструкций частично пересекаются, но можно сказать, это скорее разные наборы, нежели вариации одного.
Следовательно, в этой книге постараемся добавлять фрагменты кода на всех трех ARM \ac{ISA}.
\index{PowerPC}%
\index{MIPS}%
\index{Alpha AXP}%
Существует ещё много \ac{RISC} \ac{ISA} с опкодами фиксированной 32-битной длины~--- это как минимум MIPS, PowerPC и Alpha AXP.
\fi % LITE
\fi % RUSSIAN

\ifdefined\BRAZILIAN
\section{Uma breve introdução a CPU}

A \ac{CPU} é o dispositivo que executa o código de máquina do qual consiste um programa.

\textbf{Um glossário resumido:}

\begin{description}
\item[Instrução]: Um comando primário da \ac{CPU}. Os exemplos mais simples incluem: mover informação entre os registradores, trabalhar com memória, operações primárias de aritimética.
Como regra, cada \ac{CPU} tem sua própria arquitetura do conjunto de instruções (\ac{ISA}).

\item[Código de máquina]: Código que a \ac{CPU} processa diretamente, cada instrução geralmente é codificada por vários bytes.

\item[Linguagem assembly]: Códigos mnemônicos e algumas extensões como macros que tem por intenção facilitar a vida do programador.

\item[Registrador da CPU]: Cada \ac{CPU} tem um conjunto fixo de registradores de propósito geral (\ac{GPR}).
Aproximadamente 8 na arquitetura x86, 16 na x86-64, 16 na ARM.
A maneira mais fácil de entender um registrador é imaginá-lo como uma variável temporário sem tipo.
Imagine que você está trabalhando em uma linguagem de alto nível e pudesse usar somente oito variáveis de 32-bit (ou 64).
Ainda assim uma gama de coisas podem ser feitas usando somente estes!

\end{description}

Você pode pensar por quê há a necessidade dessa diferença entre código de máquina e linguagens de programação de alto nível.
A resposta está no fato de humanos e CPUs não se parecerem nada --- é muito mais fácil para um humano usar uma linguagem de alto nível como \CCpp, Java, Python, etc., 
mas para a CPU é mais fácil usar um nível muito mais baixo de abstração.
Talvez seja possível inventar uma CPU que pode executar códigos em linguagem de alto nível, mas ela seria muitas vezes mais complexa que as CPUs que conhecemos hoje.
De uma maneira similar, é muito mais inconveniente para humanos escreverem em linguagem assemly,
devido ao fato dela ser tão baixo nível e difícil de se escrever sem cometer um alto número de erros irritantes. O programa que converte linguagem de alto nível em código assembly é chamado de compilador.

\iffalse
% another translation
\section{Uma breve introdução à CPU}

A \ac{CPU} é o dispositivo que executa o código de máquina que consiste num programa.
\textbf{Um pequeno glossário:}
\begin{description}
\item[Instrução]: Um primitivo \ac{CPU} comando.
Os exemplos mais simples incluem: mover dados entre registradores, trabalhar com a memória, operações aritiméticas primitivas.
Como regra geral, cada \ac{CPU} tem seu próprio conjunto de instruções (\ac{ISA}).

\item[\PTBRph{}]: Código que a \ac{CPU} processa diretamente. 
Cada instrução é normalmente codificada em vários bytes.
\item[Linguagem assembly]: Código mnemônico e algumas extensões como macros que têm a finalidade de facilitar a vida do programamdor.
\item[Registradores da CPU]: Cada \ac{CPU} tem um conjunto fixo de registradores de propósito geral (\ac{GPR}).
$\approx 8$ \PTBRph{} x86, $\approx 16$ \PTBRph{} x86-64, $\approx 16$ \PTBRph{} ARM.
A forma mais fácil de entender um registrador é pensar nele como uma variável temporária não tipada.
Imagine que você estivesse trabalhando com uma \ac{PL} de alto nível e pudesse usar apenas oito variáveis de 32-bit (ou de 64-bit).
No entanto, muito ainda pode ser feito usando apenas eles.
\end{description}

Alguém poderia perguntar por que é preciso haver diferença entre código de máquina e uma \ac{PL} de alto nível.  A resposta reside no fato de que humanos e \ac{CPU}s não são iguais---%
É muito mais fácil para os humanos usar uma \ac{PL} de alto nível como \CCpp, Java, Python, etc., mas é muito mais fácil para a \ac{CPU} usar um nível de abstração muito menor.
talvez fosse possível inventar uma \ac{CPU} que pudesse executar código feito em \ac{PL} alto nível, mas seria inúmeras vezes mais complexa do que as \ac{CPU}s que conhecemos hoje.
De forma semelhante, é muito inconveninente para os seres humanos escrever em linguagem assembly, devido ao fato dela ser tão baixo nível e difícil de escrever sem comenter uma enorme quantidade de erros irritantes.
O programa que converte o código de \ac{PL} de alto nível em assembly é chamado \IT{compiler}.
\fi

\ifx\LITE\undefined
\index{ARM!\ARMMode}%
\index{ARM!\ThumbMode}%
\index{ARM!\ThumbTwoMode}%

\subsection{Algumas palavras a respeito de diferentes \ac{ISA}s}
O \ac{ISA} x86 sempre possuiu opcodes de tamanho variável, então com a chegada da era do 64-bit, as extensões x64 não impactaram a \ac{ISA} de forma muito significante. De fato, o \ac{ISA} x86 ainda contém uma série de instrucões que surgiram inicialmente na CPU 8086 16-bit, mas ainda são encontradas nas CPUs de hoje em dia.
ARM é uma \ac{CPU} \ac{RISC} desenvolvido com a idéia de opcodes com tamanho constante, o que trouxe algumas vantagens no passado.
Bem no início, todas as instruções ARM foram codificadas em 4 bytes%
\footnote{A propósito, instruções de tamanho fixo são úteis porque se pode calcular o endereço da próxima instrução (ou da anterior) sem esforço. Esta característica será discutida na seção do operador switch() ~(\myref{sec:SwitchARMLot}).}.
Este é atualmente referenciado como \q{ARM mode}.
Então concluiu-se que não era tão econômico quanto se imaginou a princípio.
Na verdade, as instruções de \ac{CPU} mais utilizadas \footnote{São estas MOV/PUSH/CALL/Jcc} em aplicações do mundo real podem ser codificadas usando menos informação.
Foi adicionado então outro \ac{ISA}, chamado Thumb, onde cada instrução era codificada em apenas 2 bytes.
Este é conhecido como \q{Thumb mode}.
No entanto, nem \IT{all} instruções ARM podem ser codificadas em apenas 2 bytes, então o conjunto de instruções Thumb é de certa forma limitado.
É interessante notar que códigos compilados para os modos ARM e Thumb podem, conforme esperado, coexistir num mesmo programa.
Os criadores do ARM concluíram que o Thumb poderia ser extendido, dando origem ao Thumb-2, que apareceu no ARMv7.
Thumb-2 ainda usa instruções de 2 bytes, mas possui algumas novas instruções com 4 bytes de tamanho.
Há um equívoco comum que Thumb-2 é uma mistura de ARM e Thumb. Isso é incorreto.
Em vez disso, Thumb-2 foi extendido para suportar completamente todos os recursos de processador de forma que ele pudesse competir com o modo ARM---um objetivo que foi claramente alcançado, uma vez que a maioria das aplicações para \idevices são compiladas para o conjunto de instruções do Thumb-2 (admitidamente, principalmente devido ao fato que o Xcode faz isso por padrão).
Posteriormente o ARM 64-bit foi lançado. Este \ac{ISA} tem opcodes de 4 bytes, e descarta a necessidade de qualquer modo Thumb adicional.
No entanto, os requisitos de 64-bit afetaram o \ac{ISA}, resultando em termos atualmente três conjuntos de instruções ARM: ARM mode, Thumb mode (incluindo Thumb-2) e ARM64.
Estes \ac{ISA}s se intersecionam parcialmente, porém podemos dizer que são \ac{ISA}s diferentes, ao invés de variações do mesmo.
Portanto, gostaríamos de tentar adicionar pedaços de código dos três \ac{ISA}s do ARM neste livro.
\index{PowerPC}%
\index{MIPS}%
\index{Alpha AXP}%
Existem, a propósito, muitos outros \ac{RISC} \ac{ISA}s com opcodes de tamanho fixo de 32-bit, como MIPS, PowerPC \PTBRph{} Alpha AXP.
\fi % LITE
\fi % BRAZILIAN

\ifdefined\SPANISH
\section{Una breve introducción a la CPU}

La \ac{CPU} es el dispositivo que ejecuta el código de máquina que constituye un programa.

\textbf{Un breve glosario:}

\begin{description}
\item[Instrucción]: Una primitiva \ac{CPU} comando.
Los ejemplos más simples incluyen: mover datos entre registros, trabajar con la memoria, operaciones aritméticas primitivas.
Como regla general, cada \ac{CPU} tiene su proprio conjunto de instrucciones (\ac{ISA}).

\item[\ESph{}]: Código que la \ac{CPU} procesa directamente. 
Cada instrucción generalmente se codifica por vários bytes.
\item[Lenguaje assembly]: Código mnemónico y algunas extensiones como macros que destinados a hacer la vida del programador más fácil.
\item[Registros de la CPU]: 
Cada \ac{CPU} tiene un conjunto fijo de registros de propósito general (\ac{GPR}).
$\approx 8$ \ESph{} x86, $\approx 16$ \ESph{} x86-64, $\approx 16$ \ESph{} ARM.
La forma más fácil de entender un registro es pensar en ello como una variable temporal sin tipo.
Imagine si estuviera trabajando con una \ac{PL} de alto nivel y sólo podría utilizar ocho variables de 32-bit (o de 64-bit).
Sin embargo mucho se puede hacer usando sólo estos!
\end{description}

Uno podría perguntarse por qué es necessário que haya diferencia entre el código de la máquina y una lenguaje de programación de alto nivel.  La respuesta está en el hecho de que los seres humanos y CPUs no son iguales---%
És mucho más fácil para los humanos utilizar un \ac{PL} de alto nivel como \CCpp, Java, Python, etc., pero és más fácil para una \ac{CPU} utilizar un nivel mucho más bajo de abstración.
Tal vez sería posible inventar una \ac{CPU} que podría ejecutar código de \ac{PL} de alto nivel, pero sería muchas veces más compleja que las \ac{CPU}s que conocemos hoy.
En uma manera similar, es muy incómodo para los seres humanos escribir en lenguaje assembly, debido a que es tan bajo nivel y difícil escribir sin hacer una gran cantidade de errores molestos.
El programa que convierte el código de \ac{PL} de alto nivel en assembly se llama \IT{compiler}.

\ifx\LITE\undefined
\index{ARM!\ARMMode}%
\index{ARM!\ThumbMode}%
\index{ARM!\ThumbTwoMode}%

\subsection{Algunas palabras sobre diferentes \ac{ISA}s}
El \ac{ISA} x86 siempre ha tenido opcodes de tamaño variable, de modo que cuanco llegó la era de 64-bit, las extensiones x64 no impactan el \ac{ISA} de manera muy significativa. De hecho, el \ac{ISA} x86 aún contiene una gran cantidade de instrucciones que primero aparecieron en CPU 8086 16-bit, pero aún se encuentran en las CPUs de hoy.
ARM és una \ac{CPU} \ac{RISC} diseñado con la idea de opcodes con tamaño constante, que tenía algunas ventajas en el pasado.
En el principio, todas las instrucciones ARM fueron codificados en 4 bytes%
\footnote{Dicho sea de paso, las instrucciones de longitud fija son muy útiles porque se puede calcular la dirección de instrucción siguiente (o anterior) sin esfuerzo. Esta característica se discutirá en la sección de el operador switch() ~(\myref{sec:SwitchARMLot}).}.
Esto actualmente se conoce como \q{ARM mode}.
Entonces se llegó a la conclusión que no era tan económico como se imaginó al princípio.
En realidad, la mayoría de las instrucciones de \ac{CPU} utilizados \footnote{Son estos MOV/PUSH/CALL/Jcc} en aplicaciones del mundo real pueden ser codificados utilizando menos información.
Por lo tanto añadieron otra \ac{ISA}, llamado Thumb, donde cada instrucción fue codificada en sólo 2 bytes.
Esto se conoce como \q{Thumb mode}.
No obstante, no todas las instrucciones ARM pueden ser codificadas en apenas 2 bytes, entonces el conjunto de instrucciones Thumb es algo limitada.
Es importante destacar que el código compilado para el modo ARM y para el modo Thumb pueden, por supuesto, coexistir dentro de un solo programa.
Los creadores de ARM concluyeron que se podría extender el Thumb, dando origem al Thumb-2, que apareció en el ARMv7.
Thumb-2 sigue utilizando instrucciones de 2 bytes, pero tiene algunas nuevas instrucciones que tienen el tamaño de 4 bytes.
Hay una idea errónea de que Thumb-2 es una mezcla de ARM y Thumb. Esto es incorrecto. 
Más bien, se extendió Thumb-2 para apoyar plenamente todas las características de processador por lo que podría 
competir con el modo ARM\EMDASH{}un objetivo que se logró con claridad, ya que la mayoria de aplicacciones para \idevices son compmilados para el conjunto de instrucciones del Thumb-2 (la verdade es, en gran parte debido al hecho de que Xcode hace esto por defecto).
Más tarde, el ARM 64-bit salió. Este \ac{ISA} tiene opcodes de 4 bytes, y descarta la necesidade de cualquier modo Thumb adicional.
Pero, los requisitos de 64-bit afectaron la \ac{ISA}, resultando en ahora tenermos tres conjuntos de instrucciones ARM: ARM mode, Thumb mode (incluyendo Thumb-2) y ARM64.
Estos \ac{ISA}s se intersectan parcialmente, pero puede ser más bien decir que son \ac{ISA}s diferentes, en lugar de variaciones de lo mismo.
Por lo tanto, nos gustaría intentar añadir fragmentos de código de los tres \ac{ISA}s del ARM en este libro.
\index{PowerPC}%
\index{MIPS}%
\index{Alpha AXP}%
Hay, por cierto, muchos otros \ac{RISC} \ac{ISA}s con opcodes de tamaño fijo de 32-bit, tales como MIPS, PowerPC \ESph{} Alpha AXP.
\fi % LITE
\fi % SPANISH


% TODO translate
\section{Numeral systems}

Humans accustomed to decimal numeral system probably because almost all ones has 10 fingers.
Nevertheless, number 10 has no significant meaning in science and mathematics.
Natural numeral system in digital electronics is binary: 0 is for absence of current in wire and 1 for presence.
10 in binary is 2 in decimal, 100 in binary is 4 in binary and so on.

How to convert from one system to another?

Positional notation is used almost everywhere, this means, the digit (number placed in single character) has some weight depending on where it is placed.
If 2 is placed at the leftmost place, it's 2.
If it is placed at the place one digit before leftmost, it's 20.

What 1234 stands for?

$10^3 \cdot 3 + 10^2 \cdot 2 + 10^1 \cdot 3 + 1 \cdot 4$ = 1234 or 
$1000 \cdot 3 + 100 \cdot 2 + 10 \cdot 3 + 4 = 1234$

Same story for binary numbers, but base is 2 instead of 10.
What 101011 stands for?

$2^5 \cdot 1 + 2^4 \cdot 0 + 2^3 \cdot 1 + 2^2 \cdot 0 + 2^1 \cdot 1 + 1 \cdot 1 = 43$ or
$32 \cdot 1 + 16 \cdot 0 + 8 \cdot 1 + 4 \cdot 0 + 2 \cdot 1 + 1 = 43$

Positional notation can be opposed to non-positional notation such as Roman numeric system.
Perhaps, humankind switched to positional notation because it's easier to do basic operations (addition, multiplication, etc) on paper by hand.

Indeed, binary numbers can be added, subtracted and so on in the very same as taught in schools, but only 2 digits are available.

Binary numbers are bulkey when represented in source code and dumps, so that is where hexadecimal numeral system can be used.
Hexadecimal system uses 0..9 numbers and also 6 Latin characters: A..F.
Each hexadecimal digit takes 4 bits or 4 binary digits.

\begin{center}
\begin{longtable}{ | l | l | l | }
\hline
\cellcolor{blue!25} hexadecimal & \cellcolor{blue!25} binary & \cellcolor{blue!25} decimal \\
\hline
0	&0000	&0 \\
1	&0001	&1 \\
2	&0010	&2 \\
3	&0011	&3 \\
4	&0100	&4 \\
5	&0101	&5 \\
6	&0110	&6 \\
7	&0111	&7 \\
8	&1000	&8 \\
9	&1001	&9 \\
A	&1010	&10 \\
B	&1011	&11 \\
C	&1100	&12 \\
D	&1101	&13 \\
E	&1110	&14 \\
F	&1111	&15 \\
\hline
\end{longtable}
\end{center}

How to detect, which system is currently used?

Decimal numbers are usually written as is, i.e., 1234. But some assemblers allows to make emphasis on decimal system and this number can be written with "d" suffix: 1234d.

Binary numbers sometimes prepended with "0b" prefix: 0b100110111 (\ac{GCC} has non-standard language extension for this: \url{https://gcc.gnu.org/onlinedocs/gcc/Binary-constants.html}).

Hexadecimal numbers are preprended with "0x" prefix in \CCpp and other \ac{PL}s: 0x1234ABCD.
Or they are has "h" suffix: 1234ABCDh - this is common way of representing them in assemblers and debuggers.
If the number is started with A..F digit, 0 should be added before: 0ABCDEFh.

Should one learn to convert numbers in mind? A table of 1-digit hexadecimal numbers can easily be memorized.
As of larger numbers, probably, it's not worth to torment yourself.

Another numeral system heavily used in past of computer programming is octal: there are 8 digits (0..7) and each is mapped to 3 bits.
It has been superseded by hexadecimal system almost everywhere, but surprisingly, there is *NIX utility used by many people often which takes octal number as argument: \TT{chmod}.

As many *NIX users know, \TT{chmod} argument can be a number of 3 digits. The first one is rights for owner of file, second is rights for group (to which file belongs), third is for everyone else.
And each digit can be represented in binary form:

\begin{center}
\begin{longtable}{ | l | l | l | }
\hline
\cellcolor{blue!25} decimal & \cellcolor{blue!25} binary & \cellcolor{blue!25} meaning \\
\hline
7	&111	&\TT{rwx} \\
6	&110	&\TT{rw-} \\
5	&101	&\TT{r-x} \\
4	&100	&\TT{r--} \\
3	&011	&\TT{-wx} \\
2	&010	&\TT{-w-} \\
1	&001	&\TT{--x} \\
0	&000	&\TT{---} \\
\hline
\end{longtable}
\end{center}

So the each bit is mapped to flag: read/write/execute.

Now the reason why I'm talking about \TT{chmod} here is that the whole number in argument can be represented as octal number.
Let's take for example, 664.
When you run \TT{chmod 664 file}, you set read/write permissions for owner, read permissions for group and again, read permissions for everyone else.
Let's convert 664 octal number to binary, this will be \TT{110100100}, or \TT{110 100 100}.

Now we see that each triplet describe permissions for owner/group/others: first is \TT{rw-}, second is \TT{r--} and third is \TT{r--}.

Octal numeral system was also popular on old computers like PDP-8, because word there could be 12, 24 or 36 bits, and these numbers are divisible by 3, so octal system was natural on that environment.
Nowadays, all popular computers employs word/address size of 16, 32 or 64 bits, and these numbers are divisible by 4, so hexadecimal system is more natural here.



% chapters
\chapter{\RU{Простейшая функция}\EN{The Simplest Function}}

<<<<<<< HEAD
\RU{Наверное, простейшая из возможных функций это та что возвращает некоторую константу.}
\EN{The simplest possible function is arguably one that simply returns a constant value:}
=======
\RU{Наверное, простейшая из возможных функций это та, что возвращает некоторую константу.}
\EN{Probably the simplest possible function is that one which just returns some constant value.}
>>>>>>> importchanges

\RU{Вот, например}

\lstinputlisting[caption=C Code]{patterns/00_ret/1.c}

\RU{Скомпилируем её!}
\EN{Lets compile it!}

\section{x86}

\RU{И вот что делает оптимизирующий GCC}\EN{Here's what both the optimizing GCC and MSVC compilers produce on the x86 platform}:

\lstinputlisting[caption=Optimizing GCC/MSVC ASM output]{patterns/00_ret/1.s}

\RU{Результат работы MSVC точно такой же}

\index{x86!\Instructions!RET}
<<<<<<< HEAD
\RU{Здесь только две инструкции: первая помещает значение 123 в регистр \EAX, который используется
для передачи возвращаемых значений и вторая это \RET, которая возвращает управление в вызывающую ф-цию.}
\EN{In the code above, there are just two instructions: the first places the value 123 into the \EAX register, which is used by convention for storing the return value of a function and the second one is \RET, which returns execution to the \gls{caller}.}
\RU{Вызывающая ф-ция возьмет результат из регистра \EAX.}
\EN{The caller will then take the result from the \EAX register.}
=======
\RU{Здесь только две инструкции. Первая помещает значение 123 в регистр \EAX, который используется
для передачи возвращаемых значений. Вторая это \RET, которая возвращает управление в вызывающую функцию.}
\EN{There are just two instructions: the first places the value 123 into the \EAX register, which is used by convention for storing the return
value and the second one is \RET, which returns execution to the \gls{caller}.}
\RU{Вызывающая функция возьмет результат из регистра \EAX.}
\EN{The caller will take the result from the \EAX register.}
>>>>>>> importchanges

\ifdefined\IncludeARM
\section{ARM}

\RU{А что насчет ARM?}\EN{There are a few differences on the ARM platform:}

\lstinputlisting[caption=\OptimizingKeilVI (\ARMMode) ASM Output]{patterns/00_ret/1_Keil_ARM_O3.s}

\RU{ARM использует регистр \Reg{0} для возврата значений, так что здесь 123 помещается в \Reg{0}.}
\EN{ARM uses the register \Reg{0} for returning the results of functions, so 123 is copied into \Reg{0}.}

<<<<<<< HEAD
\RU{Адрес возврата (\ac{RA}) в ARM не сохраняется в локальном стеке, а в регистре \ac{LR}.
Так что инструкция \TT{BX LR} делает переход по этому адресу, и это то же самое что и вернуть управление
в вызывающую ф-цию.}
%Maybe explain what a link register is, or if it is just a normal register, say so?
\EN{The return address is not saved on the local stack in the ARM \ac{ISA}, but rather in the link register, so the \TT{BX LR} instruction causes execution to jump to that address - effectively returning execution to the \gls{caller}.}
=======
\RU{Адрес возврата (\ac{RA}) в ARM сохраняется не в локальном стеке, а в регистре \ac{LR}.}
\EN{The return address (\ac{RA}) is not saved on the local stack in ARM, but rather in the \ac{LR} register.}
\RU{Так что инструкция \TT{BX LR} делает переход по этому адресу, и это то же самое что и вернуть управление
в вызывающую функцию.}
\EN{So the \TT{BX LR} instruction is jumping to that address, effectively, returning execution to the \gls{caller}.}
>>>>>>> importchanges
\fi

\index{ARM!\Instructions!MOV}
\index{x86!\Instructions!MOV}
\RU{Нужно отметить, что название инструкции \MOV в x86 и ARM сбивает с толку.}
<<<<<<< HEAD
\EN{It worth noting that \MOV is a misleading name for the instruction in both x86 and ARM \ac{ISA}s. }
\RU{На самом деле, данные не \IT{перемещаются}, а скорее \IT{копируются}.}
\EN{The data is not in fact \IT{moved}, but \IT{copied}.}
=======
\EN{It has to be noted that \MOV is a confusing name for the instruction in both x86 and ARM \ac{ISA}s. }
\RU{На самом деле, данные не \IT{перемещаются}, а \IT{копируются}.}
\EN{In fact, data is not \IT{moved}, it's rather \IT{copied}.}
>>>>>>> importchanges

\ifdefined\IncludeMIPS
\section{MIPS}

\label{MIPS_leaf_function_ex1}
\RU{Есть два способа называть регистры в мире MIPS.}
<<<<<<< HEAD
\EN{There are two naming conventions used in the world of MIPS when naming registers:}
\RU{По номеру (от \$0 до \$31) или по псевдоимени (\$V0, \$A0, и т.д.).}
\EN{by number (from \$0 to \$31) or by pseudoname (\$V0, \$A0, etc).}
=======
\EN{There are two ways of registers naming that are used in the MIPS world.}
\RU{По номеру (от \$0 до \$31) или по псевдониму (\$V0, \$A0, \etc{}.).}
\EN{By number (from \$0 to \$31) or by pseudoname (\$V0, \$A0, \etc{}).}
>>>>>>> importchanges
\RU{Вывод на ассемблере в GCC показывает регистры по номерам:}
\EN{The GCC assembly output below lists registers by number:}

\lstinputlisting[caption=\Optimizing GCC (\assemblyOutput)]{patterns/00_ret/MIPS.s}

<<<<<<< HEAD
\dots \RU{а \IDA --- по псевдоименам}\EN{while \IDA does it --- by their pseudonames}:
=======
\dots \RU{а в \IDA~--- по псевдонимам}\EN{while \IDA~--- by pseudoname}:
>>>>>>> importchanges

\lstinputlisting[caption=\Optimizing GCC (IDA)]{patterns/00_ret/MIPS_IDA.lst}

\RU{Так что регистр \$2 (или \$V0) используется для возврата значений.}
\EN{The \$2 (or \$V0) register is used to store the function's return value.}
\index{MIPS!\Pseudoinstructions!LI}
<<<<<<< HEAD
LI \RU{это}\EN{stands for} ``Load Immediate'' \EN{and is the MIPS equivalent to MOV}.

\index{MIPS!\Instructions!J}
\RU{Другая инструкция это инструкция перехода (J или JR), которая возвращает управление в 
\glslink{caller}{вызывающую ф-цию}, переходя по адресу в регистре \$31 (или \$RA).}
\EN{The other instruction is the jump instruction (J or JR) which returns the execution flow to the \gls{caller},
jumping to the address in the \$31 (or \$RA) register.}
\RU{Это аналог регистра \ac{LR} в ARM.}
\EN{This is the register analogous to \ac{LR} in ARM.}

\RU{Но почему инструкция загрузки (LI) и инструкция перехода (J или JR) поменены местами?}
\index{MIPS!Branch delay slot}
\RU{Это артефакт \ac{RISC} и называется он}
\EN{You might be wondering why positions of the the load instruction (LI) and the jump instruction (J or JR) are swapped. This is due to a \ac{RISC} feature called} ``branch delay slot''.
\RU{На самом деле, нам не нужно вникать в эти детали.}
\RU{Нужно просто запомнить: в MIPS инструкция после инструкции перехода исполняется \IT{перед} 
инструкцией перехода.}
\EN{The why this happens is due to quirk in the architecture of some RISC \ac{ISA}s and isn't important for our purposes - we just need to remember that in MIPS, the instruction following a jump or branch instruction
is executed \IT{before} the jump/brunch instruction itself.}
\RU{Таким образом, инструкция перехода всегда поменена местами с той, которая должна быть исполнена перед ней.}
\EN{As a consequence, branch instructions always swap place with the instruction which must be executed beforehand.}
% A footnote/link to http://en.wikipedia.org/wiki/Delay_slot#Branch_delay_slots or
% something similar might be useful for the people more interested in it.

\subsection{\RU{Еще кое-что об именах инструкций и регистров в MIPS}\EN{A note about MIPS instruction/register names}}
=======
LI \RU{это}\EN{is} \q{Load Immediate}.

\index{MIPS!\Instructions!J}
\RU{Другая инструкция это инструкция перехода (J или JR), которая возвращает управление в 
\glslink{caller}{вызывающую функцию}, переходя по адресу в регистре \$31 (или \$RA).}
\EN{The other instruction is jump instruction (J or JR) which returns execution flow to the \gls{caller},
jumping to the address in \$31 (or \$RA) register.}
\RU{Это аналог регистра \ac{LR} в ARM.}
\EN{This is the register analogous to \ac{LR} in ARM.}

\RU{Но почему инструкция загрузки (LI) и инструкция перехода (J или JR) поменялись местами?}
\EN{But why the load instruction (LI) and the jump instruction (J or JR) are swapped?}
\index{MIPS!Branch delay slot}
\RU{Это артефакт \ac{RISC} и называется он}
\EN{This is merely \ac{RISC} feature called} \q{branch delay slot}.
\RU{На самом деле, нам не нужно вникать в эти детали}\EN{We don't need to get into details here}.
\RU{Нужно просто запомнить: в MIPS инструкция после инструкции перехода исполняется \IT{перед} 
инструкцией перехода.}
\EN{We just need to remember that: in MIPS, the instruction following jump or branch instruction
is executed \IT{before} the jump/brunch instruction itslef.}
\RU{Таким образом, инструкция перехода всегда меняется местами с той, которая должна быть исполнена перед ней.}
\EN{Hence, branch instruction always swap place with the instruction, which must be executed beforehand.}
% A footnote/link to http://en.wikipedia.org/wiki/Delay_slot#Branch_delay_slots or
% something similar might be useful for the people more interested in it.

\subsection{\RU{Ещё кое-что об именах инструкций и регистров в MIPS}\EN{Note about MIPS instruction/register names}}
>>>>>>> importchanges

\RU{Имена регистров и инструкций в мире MIPS традиционно пишутся в нижнем регистре.}
\EN{Register and instruction names in the world of MIPS are traditionally written in lowercase.}
\RU{Но я решил использовать верхний регистр, потому что имена инструкций и регистров других \ac{ISA} в этой книге так же в верхнем регистре.}
\EN{However, for the sake of consistency, I've decided to stick to using uppercase letters, as it is the convention followed by all other \ac{ISA}s featured this book.}

\fi

\chapter{\HelloWorldSectionName}
\label{sec:helloworld}

\RU{Продолжим, используя знаменитый пример из книги}
\EN{Let's use the famous example from the book}
\NL{We bekijken het beroemde voorbeeld uit het boek}
``The C programming Language''\cite{Kernighan:1988:CPL:576122}:

\lstinputlisting{patterns/01_helloworld/hw.c}

\section{x86}

\subsection{MSVC}

\RU{Компилируем в}\EN{Let's compile it in}\NL{We compileren het in} MSVC 2010:

\begin{lstlisting}
cl 1.cpp /Fa1.asm
\end{lstlisting}

\RU{(Ключ /Fa означает сгенерировать листинг на ассемблере)}%
\EN{(/Fa option instructs the compiler to generate assembly listing file)}%
\NL{(/Fa optie zorgt ervoor dat de compiler het bestand met assembly listing genereert)}%

\begin{lstlisting}[caption=MSVC 2010]
CONST	SEGMENT
$SG3830	DB	'hello, world', 0AH, 00H
CONST	ENDS
PUBLIC	_main
EXTRN	_printf:PROC
; Function compile flags: /Odtp
_TEXT	SEGMENT
_main	PROC
	push	ebp
	mov	ebp, esp
	push	OFFSET $SG3830
	call	_printf
	add	esp, 4
	xor	eax, eax
	pop	ebp
	ret	0
_main	ENDP
_TEXT	ENDS
\end{lstlisting}

\ifx\LITE\undefined
\RU{MSVC выдает листинги в синтаксисе Intel.}\EN{MSVC produces assembly listings in Intel-syntax.}\NLph{}
\RU{Разница между синтаксисом Intel и AT\&T будет рассмотрена немного позже:}
\EN{The difference between Intel-syntax and AT\&T-syntax will be discussed in} 
\NL{Het verschil tussen Intel-syntax en AT\&T-syntax zal besproken worden in:}\myref{ATT_syntax}.
\fi

\RU{Компилятор сгенерировал файл \TT{1.obj}, который впоследствии будет слинкован линкером в \TT{1.exe}.}%
\EN{The compiler generated the file, \TT{1.obj}, which is to be linked into \TT{1.exe}.}%
\NL{De compiler heeft het bestand, \TT{1.obj} gegenereerd, hetwelk gelinkt wordt tot \TT{1.exe}.}%
\RU{В нашем случае этот файл состоит из двух сегментов: \TT{CONST} (для данных-констант) и \TT{\_TEXT} (для кода).}%
\EN{In our case, the file contains two segments: \TT{CONST} (for data constants) and \TT{\_TEXT} (for code).}%
\NL{In ons geval bevat het bestand twee segmenten: \TT{CONST} (voor data constanten) en \TT{\_TEXT}(voor code).}%

\index{\CLanguageElements!const}
\label{string_is_const_char}
\RU{Строка \TT{hello, world} в \CCpp имеет тип \TT{const char[]} \cite[p176, 7.3.2]{TCPPPL}, 
однако не имеет имени.}%
\EN{The string \TT{hello, world} in \CCpp has type \TT{const char[]} \cite[p176, 7.3.2]{TCPPPL},
but it does not have its own name.}%
\NL{De string \TT{hello, world} in \CCpp is van het type \TT{const char[]} \cite[p176, 7.3.2]{TCPPPL},
maar heeft geen eigen naam.}%
\RU{Но компилятору нужно как-то с ней работать, поэтому он дает ей внутреннее имя \TT{\$SG3830}.}%
\EN{The compiler needs to deal with the string somehow so it defines the internal name \TT{\$SG3830} for it.}%
\NL{De compiler moet een manier hebben om met de string om te kunnen, en definieert er daarom de interne naam \TT{\$SG3830} voor.}%

\RU{Поэтому пример можно было бы переписать вот так}\EN{That is why the example may be rewritten as follows}\NL{Daarom kan het voorbeeld herschreven worden als volgt}:

\lstinputlisting{patterns/01_helloworld/hw_2.c}

\RU{Вернемся к листингу на ассемблере. Как видно, строка заканчивается нулевым байтом~--- это требования стандарта \CCpp для строк.}%
\EN{Let's go back to the assembly listing. As we can see, the string is terminated by a zero byte, which is standard for \CCpp strings.}%
\NL{Laten we terug gaan naar de assembly listing. Zoals je kan zien, wordt de string beeindigd door een nul-byte. Dit is standaard voor \CCpp strings.}%
\RU{Больше о строках в Си}\EN{More about C strings}\NL{Meer over C strings}: \myref{C_strings}.

\RU{В сегменте кода \TT{\_TEXT} находится пока только одна функция}%
\EN{In the code segment, \TT{\_TEXT}, there is only one function so far}%
\NL{In het code segment, \TT{\_TEXT}, is er slechts een functie tot nu toe}: \main.
\RU{Функция \main, как и практически все функции, начинается с пролога и заканчивается эпилогом}%
\EN{The function \main starts with prologue code and ends with epilogue code (like almost any function)}%
\NL{De functie \main begint met een proloog code en eindigt met een epiloog code (zoals bijna elke functie)}%
\footnote{\RU{Об этом смотрите подробнее в разделе о прологе и эпилоге функции}%
\EN{You can read more about it in the section about function prologues and epilogues}%
\NL{Je kan hier meer over lezen in de sectie over functieprologen en epilogen}%
~(\myref{sec:prologepilog}).}.

\index{x86!\Instructions!CALL}
\RU{Далее следует вызов функции \printf}
\EN{After the function prologue we see the call to the \printf function}
\NL{Na de functie proloog zien we de call naar de \printf functie}: \TT{CALL \_printf}. 
\index{x86!\Instructions!PUSH}
\RU{Перед этим вызовом адрес строки (или указатель на неё) с нашим приветствием при помощи инструкции \PUSH помещается в стек.}
\EN{Before the call the string address (or a pointer to it) containing our greeting is placed on the stack with the help of the \PUSH instruction.}
\NL{Voor de call wordt het adres van de string (of een pointer ernaar) die onze begroeting bevat, op de stack geplaatsd met de hulp van de \PUSH instructie.}

\RU{После того, как функция \printf возвращает управление в функцию \main, адрес строки (или указатель на неё) всё ещё лежит в стеке.}%
\EN{When the \printf function returns the control to the \main function, the string address (or a pointer to it) is still on the stack.}%
\NL{Wanneer de \printf functie de controle teruggeeft aan de \main functie, staat het string adres (of de pointer ernaar) nog steeds op de stack.}%
\RU{Так как он больше не нужен, то \glslink{stack pointer}{указатель стека} (регистр \ESP) корректируется.}%
\EN{Since we do not need it anymore, the \gls{stack pointer} (the \ESP register) needs to be corrected.}%
\NL{Aangezien we dit niet meer nodig hebben, moet de \gls{stack pointer} (het \ESP register) gecorrigeerd worden.}%

\index{x86!\Instructions!ADD}
\TT{ADD ESP, 4} \RU{означает прибавить 4 к значению в регистре \ESP.}
\EN{means add 4 to the \ESP register value.}
\NL{betekent dat er 4 wordt opgeteld bij de \ESP registerwaarde.}
\RU{Почему 4? Так как это 32-битный код, для передачи адреса нужно 4 байта. В x64-коде это 8 байт.}
\EN{Why 4? Since this is a 32-bit program, we need exactly 4 bytes for address passing through the stack. If it was x64 code we would need 8 bytes.}
\NL{Waarom 4? Aangezien dit een 32-bit programma is, hebben we exact 4 bytes nodig om een adres door te geven via de stack. als het x64 code was, zouden we 8 bytes nodig gehad hebben.}
\TT{ADD ESP, 4} \RU{эквивалентно \TT{POP регистр}, но без использования какого-либо регистра\footnote{Флаги
процессора, впрочем, модифицируются}.}
\EN{is effectively equivalent to \TT{POP register} but without using any register\footnote{CPU flags, however, are modified}.}
\NL{is een effectief equivalent voor \TT{POP register} maar zonder gebruik van een register\footnote{CPU flags worden echter wel aangepast}.}

\index{Intel C++}
\index{\oracle}
\index{x86!\Instructions!POP}

\RU{Некоторые компиляторы, например, Intel C++ Compiler, в этой же ситуации могут вместо 
\ADD сгенерировать \TT{POP ECX} (подобное можно встретить, например, в коде \oracle{}, им скомпилированном),
что почти то же самое, только портится значение в регистре \ECX.}
\EN{For the same purpose, some compilers (like the Intel C++ Compiler) may emit \TT{POP ECX} 
instead of \ADD (e.g., such a pattern can be observed in the \oracle{} code as it is compiled with the Intel C++ compiler).
This instruction has almost the same effect but the \ECX register contents will be overwritten.}
\NL{Met dezelfde reden zullen sommige compilers (zoals de Intel C++ Compiler) gebruik maken van \TT{POP ECX}
in plaats van \ADD (een dergelijk patroon kan waargenomen worden in de \oracle{} code aangezien deze gecompileerd is met de Intel C++ compiler).
Deze instructie heeft bijna hetzelfde effect, maar de inhoud van het \ECX register zal overschreven worden.}
\RU{Возможно, компилятор применяет \TT{POP ECX}, потому что эта инструкция короче (1 байт у \TT{POP} против 3 у \TT{ADD}).}
\EN{The Intel C++ compiler probably uses \TT{POP ECX} since this instruction's opcode is shorter than 
\TT{ADD ESP, x} (1 byte for \TT{POP} against 3 for \TT{ADD}).}
\NL{De Intel C++ Compiler gebruikt waarschijnlijk \TT{POP ECX} aangezien de opcode van deze instructie
korter is dan \TT{ADD ESP, x} (1 byte voor \TT{POP} tegen 3 voor \TT{ADD}).}

\RU{Вот пример использования \TT{POP} вместо \TT{ADD} из \oracle{}:}
\EN{Here is an example of using \TT{POP} instead of \TT{ADD} from \oracle{}:}
\NL{Hier is een voorbeeld van het gebruik van \TT{POP} in plaats van \TT{ADD} van \oracle{}:}

\begin{lstlisting}[caption=\oracle 10.2 Linux (\RU{файл }app.o\EN{ file}\NL{ bestand})]
.text:0800029A                 push    ebx
.text:0800029B                 call    qksfroChild
.text:080002A0                 pop     ecx
\end{lstlisting}

%\RU{О стеке можно прочитать в соответствующем разделе}
%\EN{Read more about the stack in section}
%\NL{Lees meer over de stack in de sectie} ~(\myref{sec:stack}).
\index{\CLanguageElements!return}
\RU{После вызова \printf в оригинальном коде на \CCpp указано \TT{return 0}~--- вернуть 0 
в качестве результата функции \main.}
\EN{After calling \printf, the original \CCpp code contains the statement \TT{return 0}~---return 0 as the result of the \main function.}
\NL{Na \printf aan te roepen, bevat de originele \CCpp code het statement \TT{return 0}~---return 0 als resultaat van de \main functie.}
\index{x86!\Instructions!XOR}
\RU{В сгенерированном коде это обеспечивается инструкцией \INS{XOR EAX, EAX}.}
\EN{In the generated code this is implemented by the instruction \INS{XOR EAX, EAX}.}
\NL{In de gegenereerde code wordt dit geimplementeerd door de instructie \INS{XOR EAX, EAX}.}
\index{x86!\Instructions!MOV}
\RU{\XOR, как легко догадаться~--- \q{исключающее ИЛИ}}%
\EN{\XOR is in fact just \q{eXclusive OR}}%
\NL{\XOR is feitelijk simpelweg \q{eXclusive OR}}%
\footnote{\href{http://go.yurichev.com/17118}{wikipedia}}
\RU{, но компиляторы часто используют его вместо простого}
\EN{but the compilers often use it instead of}
\NL{maar de compilers gebruiken het vaak in plaats van}
\TT{MOV EAX, 0}\EMDASH{}\RU{снова потому, что опкод короче (2 байта у \TT{XOR} против 5 у \TT{MOV}).}
\EN{again because it is a slightly shorter opcode (2 bytes for \TT{XOR} against 5 for \TT{MOV}).}
\NL{wederom omdat de opcode hiervoor iets korter is (2 bytes voor \TT{XOR} tegenover 5 voor \TT{MOV}).}

\index{x86!\Instructions!SUB}
\RU{Некоторые компиляторы генерируют}\EN{Some compilers emit}\NL{Sommige compilers gebruiken}
\INS{SUB EAX, EAX},
\RU{что значит \IT{отнять значение в} \EAX \IT{от значения в }\EAX,
что в любом случае даст 0 в результате.}
\EN{which means \IT{SUBtract the value in the} \EAX \IT{from the value in} \EAX,
which, in any case, results in zero.}
\NL{wat staat voor \IT{verminder de waarde in} \EAX \IT{met de waarde in} \EAX,
wat in elke situatie resulteert in nul.}

\index{x86!\Instructions!RET}
\RU{Самая последняя инструкция \RET возвращает управление в вызывающую функцию.
Обычно это код \CCpp \ac{CRT}, который, в свою очередь, 
вернёт управление операционной системе.}
\EN{The last instruction \RET returns the control to the \gls{caller}.
Usually, this is \CCpp \ac{CRT} code, which, in turn, returns control to the \ac{OS}.}
\NL{De laatste instructie \RET geeft de controle terug aan de \gls{caller}.
Gewoonlijk is dit \CCpp \ac{CRT} code, die op zijn beurt de controle teruggeeft aan het \ac{OS}.}


\ifdefined\IncludeGCC
\EN{\input{patterns/01_helloworld/GCC_x86_EN}}
\RU{\input{patterns/01_helloworld/GCC_x86_RU}}
\NL{\input{patterns/01_helloworld/GCC_x86_NL}}
\ITA{\input{patterns/01_helloworld/GCC_x86_ITA}}
\DE{\input{patterns/01_helloworld/GCC_x86_DE}}

\fi

\section{x86-64}
\EN{\input{patterns/01_helloworld/MSVC_x64_EN}}
\ITA{\input{patterns/01_helloworld/MSVC_x64_ITA}}
\NL{\input{patterns/01_helloworld/MSVC_x64_NL}}
\RU{\input{patterns/01_helloworld/MSVC_x64_RU}}
\PTBR{\input{patterns/01_helloworld/MSVC_x64_PTBR}}


\ifdefined\IncludeGCC
\EN{\input{patterns/01_helloworld/GCC_x64_EN}}
\RU{\input{patterns/01_helloworld/GCC_x64_RU}}
\NL{\input{patterns/01_helloworld/GCC_x64_NL}}
\ITA{\input{patterns/01_helloworld/GCC_x64_ITA}}


\fi

\ifdefined\IncludeGCC
\section{GCC\EMDASH{}\EN{one more thing}\RU{ещё кое-что}}
\label{use_parts_of_C_strings}

\RU{Тот факт, что \IT{анонимная} Си-строка имеет тип}\EN{The fact that an \IT{anonymous} C-string has} 
\IT{const}\EN{ type} (\myref{string_is_const_char}), 
\RU{и тот факт, что выделенные в сегменте констант Си-строки гаратировано неизменяемые (immutable), 
ведет к интересному следствию}\EN{and
that C-strings allocated in constants segment are guaranteed to be immutable, has an interesting consequence}:
\RU{компилятор может использовать определенную часть строки}\EN{the compiler may use a specific part of the string}.

\RU{Вот простой пример}\EN{Let's try this example}:

\begin{lstlisting}
#include <stdio.h>

int f1()
{
	printf ("world\n");
}

int f2()
{
	printf ("hello world\n");
}

int main()
{
	f1();
	f2();
}
\end{lstlisting}

\RU{Среднестатистический компилятор с \CCpp (включая MSVC) выделит место для двух строк, 
но вот что делает GCC 4.8.1}%
\EN{Common \CCpp{}-compilers (including MSVC) allocate two strings, 
but let's see what GCC 4.8.1 does}:

\begin{lstlisting}[caption=GCC 4.8.1 + \RU{листинг в }IDA\EN{ listing}]
f1              proc near

s               = dword ptr -1Ch

                sub     esp, 1Ch
                mov     [esp+1Ch+s], offset s ; "world\n"
                call    _puts
                add     esp, 1Ch
                retn
f1              endp

f2              proc near

s               = dword ptr -1Ch

                sub     esp, 1Ch
                mov     [esp+1Ch+s], offset aHello ; "hello "
                call    _puts
                add     esp, 1Ch
                retn
f2              endp

aHello          db 'hello '
s               db 'world',0xa,0
\end{lstlisting}

\RU{Действительно, когда мы выводим строку}\EN{Indeed: when we print the \q{hello world} string}, 
\RU{эти два слова расположены в памяти впритык друг к другу и \puts, вызываясь из функции f2(), вообще не знает,
что эти строки разделены}\EN{these two words are positioned in memory adjacently and \puts called from f2() 
function is not aware that this string is divided}. \RU{Они и не разделены на самом деле, они разделены
только \q{виртуально}, в нашем листинге}\EN{In fact, it's not divided; it's divided only \q{virtually}, in this
listing}.

\RU{Когда}\EN{When} \puts \RU{вызывается из f1(), он использует строку}\EN{is called from f1(), it uses the} 
\q{world} \RU{плюс нулевой байт}\EN{string plus a zero byte}. \puts \RU{не знает, что там ещё есть какая-то строка
перед этой}\EN{is not aware that there is something before this string}!

\RU{Этот трюк часто используется (по крайней мере в GCC) и может сэкономить немного памяти.}
\EN{This clever trick is often used by at least GCC and can save some memory.}

\fi
\ifdefined\IncludeARM
\section{ARM}
\label{sec:hw_ARM}

\index{\idevices}
\index{Raspberry Pi}
\index{Xcode}
\index{LLVM}
\index{Keil}
\RU{Для экспериментов с процессором ARM я использовал несколько компиляторов:}
\EN{For my experiments with ARM processors I used several compilers:} 

\begin{itemize}
\item \RU{Популярный в embedded-среде}\EN{Popular in the embedded area} Keil Release 6/2013.

\item Apple Xcode 4.6.3 \EN{IDE} (\RU{с компилятором}\EN{with} LLVM-GCC 4.2 \EN{compiler}
\footnote{\EN{It is indeed so: Apple Xcode 4.6.3 uses open-source GCC as front-end compiler and LLVM 
code generator}\RU{Это действительно так: Apple Xcode 4.6.3 использует опен-сорсный GCC как компилятор
переднего плана и коде-генератор LLVM}}.

%\item GCC 4.8.1 (Linaro) (\RU{для}\EN{for} ARM64).
%
\item GCC 4.9 (Linaro) (\RU{для}\EN{for} ARM64), 
\RU{доступный как исполняемые файлы для win32 на}\EN{available as win32-executables at} 
\url{http://www.linaro.org/projects/armv8/}.

\end{itemize}

\RU{Везде в этой книге, кроме как если указано иное, идет речь о 32-битном ARM.}
\EN{32-bit ARM code is used in all cases in this book, if not mentioned otherwise.}
\RU{Когда речь идет о 64-битном ARM, он называется здесь ARM64.}
\EN{If we talk about 64-bit ARM here, it will be called ARM64.}

% subsections
\input{patterns/01_helloworld/ARM/keil_ARM}
\input{patterns/01_helloworld/ARM/keil_T}
\input{patterns/01_helloworld/ARM/xcode_ARM}
\input{patterns/01_helloworld/ARM/xcode_T2}
\input{patterns/01_helloworld/ARM/ARM64}

\fi
\ifdefined\IncludeMIPS
\section{MIPS}

\subsection{\RU{О \q{глобальном указателе} (\q{global pointer})}\EN{A word about the \q{global pointer}}}
\label{MIPS_GP}

\index{MIPS!\GlobalPointer}
\RU{\q{Глобальный указатель} (\q{global pointer})~--- это важная концепция в MIPS.}
\EN{One important MIPS concept is the \q{global pointer}.}
\RU{Как мы уже возможно знаем, каждая инструкция в MIPS имеет размер 32 бита, поэтому невозможно
закодировать 32-битный адрес внутри одной инструкции. Вместо этого нужно использовать пару инструкций
(как это сделал GCC для загрузки адреса текстовой строки в нашем примере).}
\EN{As we may already know, each MIPS instruction has a size of 32 bits, so it's impossible to embed a 32-bit
address into one instruction: a pair has to be used for this 
(like GCC did in our example for the text string address loading).}

\RU{С другой стороны, используя только одну инструкцию, 
возможно загружать данные по адресам в пределах $register-32768...register+32767$, потому что 16 бит
знакового смещения можно закодировать в одной инструкции).}
\EN{It's possible, however, to load data from the address in the range of $register-32768...register+32767$ using one
single instruction (because 16 bits of signed offset could be encoded in a single instruction).}
\RU{Так мы можем выделить какой-то регистр для этих целей и ещё выделить буфер в 64KiB для самых 
частоиспользуемых данных.}
\EN{So we can allocate some register for this purpose and also allocate a 64KiB area of most used data.}
\RU{Выделенный регистр называется \q{глобальный указатель} (\q{global pointer}) и он указывает на середину
области 64KiB.}
\EN{This allocated register is called a \q{global pointer} and it points to the middle of the 64KiB area.}
\RU{Эта область обычно содержит глобальные переменные и адреса импортированных функций вроде \printf,
потому что разработчики GCC решили, что получение адреса функции должно быть как можно более быстрой операцией,
исполняющейся за одну инструкцию вместо двух.}
\EN{This area usually contains global variables and addresses of imported functions like \printf, 
because the GCC developers decided that getting the address of some function must be as fast as a single instruction
execution instead of two.}
\RU{В ELF-файле эта 64KiB-область находится частично в секции .sbss (\q{small \ac{BSS}}) для неинициализированных
данных и в секции .sdata (\q{small data}) для инициализированных данных.}
\EN{In an ELF file this 64KiB area is located partly in sections .sbss (\q{small \ac{BSS}}) for uninitialized data and 
.sdata (\q{small data}) for initialized data.}

\RU{Это значит что программист может выбирать, к чему нужен как можно более быстрый доступ, и затем расположить
это в секциях .sdata/.sbss.}
\EN{This implies that the programmer may choose what data he/she wants to be accessed fast and place it into 
.sdata/.sbss.}

\RU{Некоторые программисты \q{старой школы} могут вспомнить модель памяти в MS-DOS \myref{8086_memory_model} 
или в менеджерах памяти вроде XMS/EMS, где вся память делилась на блоки по 64KiB.}
\EN{Some old-school programmers may recall the MS-DOS memory model \myref{8086_memory_model} 
or the MS-DOS memory managers like XMS/EMS where all memory was divided in 64KiB blocks.}

\index{PowerPC}
\RU{Эта концепция применяется не только в MIPS. По крайней мере PowerPC также использует эту технику.}
\EN{This concept is not unique to MIPS. At least PowerPC uses this technique as well.}

\subsection{\Optimizing GCC}

\EN{Lets consider the following example, which illustrates the \q{global pointer} concept.}
\RU{Рассмотрим следующий пример, иллюстрирующий концепцию \q{глобального указателя}.}

\lstinputlisting[caption=\Optimizing GCC 4.4.5 (\assemblyOutput),numbers=left]{patterns/01_helloworld/MIPS/hw_O3.s.\LANG}

\RU{Как видно, регистр \$GP в прологе функции выставляется в середину этой области.}
\EN{As we see, the \$GP register is set in the function prologue to point to the middle of this area.}
\RU{Регистр \ac{RA} сохраняется в локальном стеке.}
\EN{The \ac{RA} register is also saved in the local stack.}
\RU{Здесь также используется \puts вместо \printf.}
\EN{\puts is also used here instead of \printf.}
\index{MIPS!\Instructions!LW}
\RU{Адрес функции \puts загружается в \$25 инструкцией LW (\q{Load Word}).}
\EN{The address of the \puts function is loaded into \$25 using LW the instruction (\q{Load Word}).}
\index{MIPS!\Instructions!LUI}
\index{MIPS!\Instructions!ADDIU}
\RU{Затем адрес текстовой строки загружается в \$4 парой инструкций LUI (\q{Load Upper Immediate}) и
ADDIU (\q{Add Immediate Unsigned Word}).}
\EN{Then the address of the text string is loaded to \$4 using LUI (\q{Load Upper Immediate}) and 
ADDIU (\q{Add Immediate Unsigned Word}) instruction pair.}
\RU{LUI устанавливает старшие 16 бит регистра (поэтому в имени инструкции присутствует \q{upper}) и ADDIU
прибавляет младшие 16 бит к адресу.}
\EN{LUI sets the high 16 bits of the register (hence \q{upper} word in instruction name) and ADDIU adds
the lower 16 bits of the address.}
\RU{ADDIU следует за JALR (помните о \IT{branch delay slots}?).}
\EN{ADDIU follows JALR (remember \IT{branch delay slots}?).}
\RU{Регистр \$4 также называется \$A0, который используется для передачи первого аргумента функции}%
\EN{The register \$4 is also called \$A0, which is used for passing the first function argument}%
\footnote{\RU{Таблица регистров в MIPS доступна в приложении}\EN{The MIPS registers table %
is available in appendix} \myref{MIPS_registers_ref}}.

\index{MIPS!\Instructions!JALR}
\RU{JALR (\q{Jump and Link Register}) делает переход по адресу в регистре \$25 (там адрес \puts) 
при этом сохраняя адрес следующей инструкции (LW) в \ac{RA}.}
\EN{JALR (\q{Jump and Link Register}) jumps to the address stored in the \$25 register (address of \puts) 
while saving the address of the next instruction (LW) in \ac{RA}.}
\RU{Это так же как и в ARM}\EN{This is very similar to ARM}.
\RU{И ещё одна важная вещь: адрес сохраняемый в \ac{RA} это адрес не следующей инструкции (потому что
это \IT{delay slot} и исполняется перед инструкцией перехода),
а инструкции после неё (после \IT{delay slot}).}
\EN{Oh, and one important thing is that the address saved in \ac{RA} is not the address of the next instruction (because
it's in a \IT{delay slot} and is executed before the jump instruction),
but the address of the instruction after the next one (after the \IT{delay slot}).}
\RU{Таким образом во время исполнения \TT{JALR} в \ac{RA} записывается $PC + 8$. В нашем случае это адрес
инструкции LW следующей после ADDIU.}
\EN{Hence, $PC + 8$ is written to \ac{RA} during the execution of \TT{JALR}, in our case, this is the address of the LW
instruction next to ADDIU.}

\RU{LW (\q{Load Word}) в строке 20 восстанавливает \ac{RA} из локального стека 
(эта инструкция скорее часть эпилога функции).}
\EN{LW (\q{Load Word}) at line 20 restores \ac{RA} from the local stack 
(this instruction is actually part of the function epilogue).}

\index{MIPS!\Pseudoinstructions!MOVE}
\RU{MOVE в строке 22 копирует значение из регистра \$0 (\$ZERO) в \$2 (\$V0).}
\EN{MOVE at line 22 copies the value from the \$0 (\$ZERO) register to \$2 (\$V0).}
\label{MIPS_zero_register}
\RU{В MIPS есть \IT{константный} регистр, всегда содержащий ноль.}
\EN{MIPS has a \IT{constant} register, which always holds zero.}
\RU{Должно быть, разработчики MIPS решили что 0 это самая востребованная константа в программировании,
так что пусть будет использоваться регистр \$0, всякий раз, когда будет нужен 0.}
\EN{Apparently, the MIPS developers came up with the idea that zero is in fact the busiest constant in the computer programming,
so let's just use the \$0 register every time zero is needed.}
\RU{Другой интересный факт: в MIPS нет инструкции, копирующей значения из регистра в регистр.}
\EN{Another interesting fact is that MIPS lacks an instruction that transfers data between registers.}
\RU{На самом деле}\EN{In fact}, \TT{MOVE DST, SRC} \RU{это}\EN{is} \TT{ADD DST, SRC, \$ZERO} ($DST=SRC+0$), 
\RU{которая делает тоже самое}\EN{which does the same}.
\RU{Очевидно, разработчики MIPS хотели сделать как можно более компактную таблицу опкодов.}
\EN{Apparently, the MIPS developers wanted to have a compact opcode table.}
\RU{Это не значит, что сложение происходит во время каждой инструкции MOVE.}
\EN{This does not mean an actual addition happens at each MOVE instruction.}
\RU{Скорее всего, эти псевдоинструкции оптимизируются в \ac{CPU} и \ac{ALU} никогда не используется.}
\EN{Most likely, the \ac{CPU} optimizes these pseudoinstructions and the \ac{ALU} is never used.}

\index{MIPS!\Instructions!J}
\RU{J в строке 24 делает переход по адресу в \ac{RA}, и это работает как выход из функции.}
\EN{J at line 24 jumps to the address in \ac{RA}, which is effectively performing a return from the function.}
\RU{ADDIU после J на самом деле исполняется перед J (помните о \IT{branch delay slots}?) 
и это часть эпилога функции.}
\EN{ADDIU after J is in fact executed before J (remember \IT{branch delay slots}?) 
and is part of the function epilogue.}

\RU{Вот листинг сгенерированный \IDA. Каждый регистр имеет свой псевдоним:}
\EN{Here is also a listing generated by \IDA. Each register here has its own pseudoname:}

\lstinputlisting[caption=\Optimizing GCC 4.4.5 (\IDA),numbers=left]{patterns/01_helloworld/MIPS/hw_O3_IDA.lst.\LANG}

\RU{Инструкция в строке 15 сохраняет GP в локальном стеке. Эта инструкция мистическим образом отсутствует
в листинге от GCC, может быть из-за ошибки в самом GCC\footnote{Очевидно, функция вывода листингов не так критична
для пользователей GCC, поэтому там вполне могут быть неисправленные ошибки.}.}
\EN{The instruction at line 15 saves the GP value into the local stack, and this instruction is missing mysteriously from the GCC output listing, maybe by a GCC error\footnote{Apparently, functions generating listings 
are not so critical to GCC users, so some unfixed errors may still exist.}.}
\RU{Значение GP должно быть сохранено, потому что всякая функция может работать со своим собственным окном данных
размером 64KiB.}
\EN{The GP value has to be saved indeed, because each function can use its own 64KiB data window.}

\RU{Регистр, содержащий адрес функции \puts называется \$T9, потому что регистры с префиксом T- называются
\q{temporaries} и их содержимое можно не сохранять.}
\EN{The register containing the \puts address is called \$T9, because registers prefixed with T- are called
\q{temporaries} and their contents may not be preserved.}

\subsection{\NonOptimizing GCC}

\NonOptimizing GCC \RU{более многословный}\EN{is more verbose}.

\lstinputlisting[caption=\NonOptimizing GCC 4.4.5 (\assemblyOutput),numbers=left]{patterns/01_helloworld/MIPS/hw_O0.s.\LANG}

\RU{Мы видим, что регистр FP используется как указатель на фрейм стека.}
\EN{We see here that register FP is used as a pointer to the stack frame.}
\RU{Мы также видим 3 \ac{NOP}-а.}\EN{We also see 3 \ac{NOP}s.}
\RU{Второй и третий следуют за инструкциями перехода.}
\EN{The second and third of which follow the branch instructions.}

\RU{Вероятно, компилятор GCC всегда добавляет \ac{NOP}-ы (из-за \IT{branch delay slots})
после инструкций переходов и затем, если включена оптимизация, от них может избавляться.}%
\EN{Perhaps, the GCC compiler always adds \ac{NOP}s (because of \IT{branch delay slots}) after branch
instructions and then, if optimization is turned on, maybe eliminates them.}
\RU{Так что они остались здесь}\EN{So in this case they are left here}.

\RU{Вот также листинг от \IDA:}
\EN{Here is also \IDA listing:}

\lstinputlisting[caption=\NonOptimizing GCC 4.4.5 (\IDA),numbers=left]{patterns/01_helloworld/MIPS/hw_O0_IDA.lst.\LANG}

\index{MIPS!\Pseudoinstructions!LA}
\RU{Интересно что \IDA распознала пару инструкций LUI/ADDIU и собрала их в одну псевдоинструкцию 
LA (\q{Load Address}) в строке 15.}
\EN{Interestingly, \IDA recognized the LUI/ADDIU instructions pair and coalesced them into one 
LA (\q{Load Address}) pseudoinstruction at line 15.}
\RU{Мы также видим, что размер этой псевдоинструкции 8 байт!}
\EN{We may also see that this pseudoinstruction has a size of 8 bytes!}
\RU{Это псевдоинструкция (или \IT{макрос}), потому что это не настоящая инструкция MIPS, а скорее
просто удобное имя для пары инструкций.}
\EN{This is a pseudoinstruction (or \IT{macro}) because it's not a real MIPS instruction, but rather
a handy name for an instruction pair.}

\index{MIPS!\Pseudoinstructions!NOP}
\index{MIPS!\Instructions!OR}
\RU{Ещё кое что: \IDA не распознала \ac{NOP}-инструкции в строках 22, 26 и 41.}
\EN{Another thing is that \IDA doesn't recognize \ac{NOP} instructions, so here they are at lines 22, 26 and 41.}
\RU{Это}\EN{It is} \TT{OR \$AT, \$ZERO}.
\RU{По своей сути это инструкция, применяющая операцию ИЛИ к содержимому регистра \$AT с нулем, что,
конечно же, холостая операция.}
\EN{Essentially, this instruction applies the OR operation to the contents of the \$AT register
with zero, which is, of course, an idle instruction.}
\RU{MIPS, как и многие другие \ac{ISA}, не имеет отдельной \ac{NOP}-инструкции.}
\EN{MIPS, like many other \ac{ISA}s, doesn't have a separate \ac{NOP} instruction.}

\subsection{\RU{Роль стекового фрейма в этом примере}\EN{Role of the stack frame in this example}}

\RU{Адрес текстовой строки передается в регистре.}
\EN{The address of the text string is passed in the register.}
\RU{Так зачем устанавливать локальный стек?}\EN{Why setup a local stack anyway?}
\RU{Причина в том, что значения регистров \ac{RA} и GP должны быть сохранены где-то
(потому что вызывается \printf) и для этого используется локальный стек.}
\EN{The reason for this lies in the fact that the values of registers \ac{RA} and GP have to be saved somewhere 
(because \printf is called), and the local stack is used for this purpose.}
\RU{Если бы это была \gls{leaf function}, тогда можно было бы избавиться от пролога и эпилога функции. Например:}
\EN{If this was a \gls{leaf function}, it would have been possible to get rid of the function prologue and epilogue,
for example:} \myref{MIPS_leaf_function_ex1}.

\subsection{\Optimizing GCC: \RU{загрузим в}\EN{load it into} GDB}

\index{GDB}
\lstinputlisting[caption=\RU{пример сессии в GDB}\EN{sample GDB session}]{patterns/01_helloworld/MIPS/O3_GDB.txt}

\fi

\section{\Conclusion{}}

\ifdefined\RUSSIAN
Основная разница между кодом x86/ARM и x64/ARM64 в том, что указатель на строку теперь 64-битный.
Действительно, ведь для того современные \ac{CPU} и стали 64-битными, потому что подешевела память,
её теперь можно поставить в компьютер намного больше, и чтобы её адресовать, 32-х бит уже
недостаточно.
Поэтому все указатели теперь 64-битные.
\fi

\ifdefined\ENGLISH
The main difference between x86/ARM and x64/ARM64 code is that the pointer to the string is now 64-bits in length.
Indeed, modern \ac{CPU}s are now 64-bit due to both the reduced cost of memory and the greater demand for it by modern applications. 
We can add much more memory to our computers than 32-bit pointers are able to address.
As such, all pointers are now 64-bit.
\fi

\ifdefined\DUTCH
Het grootste verschil tussen x86/ARM en x64/ARM64 code is dat de pointer naar de string nu 64-bits in lengte is.
De meeste moderne \ac{CPU}s zijn tegenwoordig 64-bit wegens zowel de verminderde gebruik van geheugen, als de grote vraag ervoor door moderne applicaties.
We kunnen hierdoor veel meer geheugen aan onze computers toevoegen dan dat 32-bit pointers kunnen aanspreken.
Bijgevolg zijn alle pointers nu 64-bit.
\fi

% sections
\ifdefined\IncludeExercises
\section{\Exercises}

\begin{itemize}
	\item \url{http://challenges.re/48}
	\item \url{http://challenges.re/49}
\end{itemize}


\fi

\chapter{\RU{Пролог и эпилог функций}\EN{Function prologue and epilogue}}
\label{sec:prologepilog}
\index{Function epilogue}
\index{Function prologue}

\RU{Пролог функции это инструкции в самом начале функции. Как правило это что-то вроде такого
фрагмента кода:}
\EN{A function prologue is a sequence of instructions at the start of a function. It often looks something like the following
code fragment:}

\begin{lstlisting}
    push    ebp
    mov     ebp, esp
    sub     esp, X
\end{lstlisting}

\RU{Эти инструкции делают следующее: сохраняют значение регистра \EBP на будущее, выставляют \EBP равным \ESP, 
затем подготавливают место в стеке для хранения локальных переменных.}
\EN{What these instruction do: save the value in the \EBP register,
set the value of the \EBP register to the value of the \ESP and then allocate space on the stack 
for local variables.}

\RU{\EBP сохраняет свое значение на протяжении всей функции, он будет использоваться здесь для доступа 
к локальным переменным и аргументам. Можно было бы использовать и \ESP, но он постоянно меняется и 
это не очень удобно.}
\EN{The value in the \EBP stays the same over the period of the function execution and is to be used for local variables and 
arguments access. 
For the same purpose one can use \ESP, but since it changes over time this approach is not too convenient.}

\RU{Эпилог функции аннулирует выделенное место в стеке, восстанавливает значение \EBP на старое и возвращает 
управление в вызывающую функцию:}
\EN{The function epilogue frees the allocated space in the stack, returns the value in the \EBP register back to its initial state 
and returns the control flow to the \gls{callee}:}

\begin{lstlisting}
    mov    esp, ebp
    pop    ebp
    ret    0
\end{lstlisting}

% what about calling convention?
\RU{Пролог и эпилог функции обычно находятся в дизассемблерах для отделения функций друг от друга.}
\EN{Function prologues and epilogues are usually detected in disassemblers for function delimitation.}

\section{\Recursion}

\index{\Recursion}
\RU{Наличие эпилога и пролога может несколько ухудшить эффективность рекурсии.}
\EN{Epilogues and prologues can negatively affect the recursion performance.}

\EN{More about recursion in this book}\RU{Больше о рекурсии в этой книге}: 
\myref{Recursion_and_tail_call}.

\EN{\section{\Stack}
\label{sec:stack}
\myindex{\Stack}

The stack is one of the most fundamental data structures in computer science
\footnote{\href{http://go.yurichev.com/17119}{wikipedia.org/wiki/Call\_stack}}.
\ac{AKA} \ac{LIFO}.

Technically, it is just a block of memory in process memory along with the \ESP or \RSP register in x86 or x64, or the \ac{SP} register in ARM, as a pointer within that block.

\myindex{ARM!\Instructions!PUSH}
\myindex{ARM!\Instructions!POP}
\myindex{x86!\Instructions!PUSH}
\myindex{x86!\Instructions!POP}
The most frequently used stack access instructions are \PUSH and \POP (in both x86 and ARM Thumb-mode). 
\PUSH subtracts from \ESP/\RSP/\ac{SP} 4 in 32-bit mode (or 8 in 64-bit mode) and then writes the contents of its sole operand to the memory address pointed by \ESP/\RSP/\ac{SP}.

\POP is the reverse operation: retrieve the data from the memory location that \ac{SP} points to, 
load it into the instruction operand (often a register) and then add 4 (or 8) to the \gls{stack pointer}.

After stack allocation, the \gls{stack pointer} points at the bottom of the stack.
\PUSH decreases the \gls{stack pointer} and \POP increases it.
The bottom of the stack is actually at the beginning of the memory allocated for the stack block. It seems strange, but that's the way it is.

ARM supports both descending and ascending stacks.

\myindex{ARM!\Instructions!STMFD}
\myindex{ARM!\Instructions!LDMFD}
\myindex{ARM!\Instructions!STMED}
\myindex{ARM!\Instructions!LDMED}
\myindex{ARM!\Instructions!STMFA}
\myindex{ARM!\Instructions!LDMFA}
\myindex{ARM!\Instructions!STMEA}
\myindex{ARM!\Instructions!LDMEA}

For example the \ac{STMFD}/\ac{LDMFD}, \ac{STMED}/\ac{LDMED} instructions are intended to deal with a descending stack (grows downwards, starting with a high address and progressing to a lower one).
The \ac{STMFA}/\ac{LDMFA}, \ac{STMEA}/\ac{LDMEA} instructions are intended to deal with an ascending stack (grows upwards, starting from a low address and progressing to a higher one).

% It might be worth mentioning that STMED and STMEA write first,
% and then move the pointer,
% and that LDMED and LDMEA move the pointer first, and then read.
% In other words, ARM not only lets the stack grow in a non-standard direction,
% but also in a non-standard order.
% Maybe this can be in the glossary, which would explain why E stands for "empty".

\subsection{Why does the stack grow backwards?}
\label{stack_grow_backwards}

Intuitively, we might think that the stack grows upwards, i.e. towards higher addresses, like any other data structure.

The reason that the stack grows backward is probably historical.
When the computers were big and occupied a whole room, it was easy to divide memory into two parts, one for the \gls{heap} and one for the stack.
Of course, it was unknown how big the \gls{heap} and the stack would be during program execution, so this solution was the simplest possible.

\input{patterns/02_stack/stack_and_heap}

In \RitchieThompsonUNIX we can read:

\begin{framed}
\begin{quotation}
The user-core part of an image is divided into three logical segments. The program text segment begins at location 0 in the virtual address space. During execution, this segment is write-protected and a single copy of it is shared among all processes executing the same program. At the first 8K byte boundary above the program text segment in the virtual address space begins a nonshared, writable data segment, the size of which may be extended by a system call. Starting at the highest address in the virtual address space is a stack segment, which automatically grows downward as the hardware's stack pointer fluctuates.
\end{quotation}
\end{framed}

This reminds us how some students write two lecture notes using only one notebook:
notes for the first lecture are written as usual, 
and notes for the second one are written from the end of notebook, by flipping it.
Notes may meet each other somewhere in between, in case of lack of free space.

% I think if we want to expand on this analogy,
% one might remember that the line number increases as as you go down a page.
% So when you decrease the address when pushing to the stack, visually,
% the stack does grow upwards.
% Of course, the problem is that in most human languages,
% just as with computers,
% we write downwards, so this direction is what makes buffer overflows so messy.

\subsection{What is the stack used for?}

% subsections
\input{patterns/02_stack/01_saving_ret_addr}
\input{patterns/02_stack/02_args_passing}
\EN{\input{patterns/02_stack/03_local_vars_EN}}
\RU{\input{patterns/02_stack/03_local_vars_RU}}
\PTBR{\input{patterns/02_stack/03_local_vars_PTBR}}
\input{patterns/02_stack/04_alloca/main}
\input{patterns/02_stack/05_SEH}
\input{patterns/02_stack/06_BO_protection}

\subsubsection{Automatic deallocation of data in stack}

Perhaps the reason for storing local variables and SEH records in the stack is that they are freed automatically upon function exit,
using just one instruction to correct the stack pointer (it is often \ADD).
Function arguments, as we could say, are also deallocated automatically at the end of function.
In contrast, everything stored in the \IT{heap} must be deallocated explicitly.

% sections
\EN{\input{patterns/02_stack/07_layout_EN}}
\RU{\input{patterns/02_stack/07_layout_RU}}
\PTBR{\input{patterns/02_stack/07_layout_PTBR}}
\input{patterns/02_stack/08_noise/main}
\input{patterns/02_stack/exercises}
}
\FR{\section{\Stack}
\label{sec:stack}
\myindex{\Stack}

La pile est une des structures de données les plus fondamentales en informatique.
\footnote{\href{http://go.yurichev.com/17119}{wikipedia.org/wiki/Call\_stack}}.
\ac{AKA} \ac{LIFO}.

Techniquement, il s'agit d'un bloc de mémoire présent dans l'espace d'adressage
d'un processus et qui est utilisé par le registre \ESP ou \RSP en x86 ou x64,
ou par le registre \ac{SP} en ARM comme un pointeur dans ce bloc mémoire. 

\myindex{ARM!\Instructions!PUSH}
\myindex{ARM!\Instructions!POP}
\myindex{x86!\Instructions!PUSH}
\myindex{x86!\Instructions!POP}
Les instructions d'accès à la pile sont \PUSH et \POP (en x86 ainsi qu'en ARM Thumb-mode).
\PUSH soustrait à \ESP/\RSP/\ac{SP} 4 en mode 32-bit (ou 8 en mode 64-bit) et écrit
ensuite le contenu de l'opérande associée à l'adresse mémoire pointée par \ESP/\RSP/\ac{SP}.

\POP est l'operation inverse: elle récupére la donnée depuis l'adresse mémoire pointée par \ac{SP},
l'écrit dans l'opérande associée (souvent un registre) puis ajoute 4 (ou 8) au \glslink{stack pointer}{pointeur de pile}.

Après une allocation sur la pile, le \glslink{stack pointer}{pointeur de pile} pointe sur le bas de la pile.
\PUSH décrémente le \gls{stack pointer} et \POP l'incrémente.

Le bas de la pile représente en réalité le début de la mémoire allouée pour
 le bloc de pile. Cela semble étrange, mais c'est comme ça.

ARM supporte à la fois les piles ascendantes et descendantes.

\myindex{ARM!\Instructions!STMFD}
\myindex{ARM!\Instructions!LDMFD}
\myindex{ARM!\Instructions!STMED}
\myindex{ARM!\Instructions!LDMED}
\myindex{ARM!\Instructions!STMFA}
\myindex{ARM!\Instructions!LDMFA}
\myindex{ARM!\Instructions!STMEA}
\myindex{ARM!\Instructions!LDMEA}

Par exemple les instructions \ac{STMFD}/\ac{LDMFD}, \ac{STMED}/\ac{LDMED} sont utilisées pour gérer les piles
descendantes (qui grandissent vers le bas en commençant avec une adresse haute et évoluent vers une plus basse).

Les instructions \ac{STMFA}/\ac{LDMFA}, \ac{STMEA}/\ac{LDMEA} sont utilisées pour gérer les piles montantes
(qui grandissent vers les adresses hautes de l'adresse space, en commençant avec une adresse située en bas de l'adresse space)

% It might be worth mentioning that STMED and STMEA write first,
% and then move the pointer,
% and that LDMED and LDMEA move the pointer first, and then read.
% In other words, ARM not only lets the stack grow in a non-standard direction,
% but also in a non-standard order.
% Maybe this can be in the glossary, which would explain why E stands for "empty".

\subsection{Pourquoi la pile grandit en descendant ?}
\label{stack_grow_backwards}

Intuitivement, on pourrait penser que la pile grandit vers le haut, i.e. vers des
adresses plus élevées, comme n'importe qu'elle autre structure de données.

La raison pour laquelle la pile grandit vers le bas est probablement historique.
Dans le passé, les ordinateurs étaient énormes et occupaient des piéces entières,
il était facile de diviser la mémoire en deux parties, une pour le \gls{heap} et
une pour la pile.
Evidemment, on ignorait quelle serait la taille du \gls{heap} et de la pile durant
l'éxécution du progamme, donc cette solution était la plus simple possible.

\input{patterns/02_stack/stack_and_heap}

Dans \RitchieThompsonUNIX on peut lire:

\begin{framed}
\begin{quotation}
The user-core part of an image is divided into three logical segments. The program text segment begins at location 0 in the virtual address space. During execution, this segment is write-protected and a single copy of it is shared among all processes executing the same program. At the first 8K byte boundary above the program text segment in the virtual address space begins a nonshared, writable data segment, the size of which may be extended by a system call. Starting at the highest address in the virtual address space is a pile segment, which automatically grows downward as the hardware's pile pointer fluctuates.
\end{quotation}
\end{framed}

Cela nous rappelle comment certains étudiants prennent des notes pour deux cours différents dans
un seul et même cahier en prenant un cours d'un côté du cahier, et l'autre cours de l'autre côté.
Les notes de cours finissent par se rencontrer à un moment dans le cahier quand il n'y a plus de place.

% I think if we want to expand on this analogy,
% one might remember that the line number increases as as you go down a page.
% So when you decrease the address when pushing to the stack, visually,
% the stack does grow upwards.
% Of course, the problem is that in most human languages,
% just as with computers,
% we write downwards, so this direction is what makes buffer overflows so messy.

\subsection{Quel est le rôle de la pile ?}

% subsections
\input{patterns/02_stack/01_saving_ret_addr_FR}
\input{patterns/02_stack/02_args_passing_FR}
\input{patterns/02_stack/03_local_vars_FR}
\input{patterns/02_stack/04_alloca/main}
\input{patterns/02_stack/05_SEH}
\input{patterns/02_stack/06_BO_protection}

\subsubsection{Désallocation automatique de données dans la pile}

Peut-être que la raison pour laquelle les variables locales et les enregistrements SEH sont stockés dans la
pile est qu'ils sont automatiquement libérés quand la fonction se termine en utilisant simplement une
instruction pour corriger la position du pointeur de pile (souvent \ADD).
Les arguments de fonction sont aussi désalloués automatiquement à la fin de la fonction.
À l'inverse, toutes les données allouées sur le \IT{heap} doivent être désallouées de façon explicite.

% sections
\input{patterns/02_stack/07_layout_FR} % TBT
\input{patterns/02_stack/08_noise/main}
\input{patterns/02_stack/exercises}
}
\RU{\section{\Stack}
\label{sec:stack}
\myindex{\Stack}

Стек в информатике~--- это одна из наиболее фундаментальных структур данных
\footnote{\href{http://go.yurichev.com/17119}{wikipedia.org/wiki/Call\_stack}}.
\ac{AKA} \ac{LIFO}.

Технически это просто блок памяти в памяти процесса + регистр \ESP в x86 или \RSP в x64, либо \ac{SP} в ARM, который указывает где-то в пределах этого блока.

\myindex{ARM!\Instructions!PUSH}
\myindex{ARM!\Instructions!POP}
\myindex{x86!\Instructions!PUSH}
\myindex{x86!\Instructions!POP}
Часто используемые инструкции для работы со стеком~--- это \PUSH и \POP (в x86 и Thumb-режиме ARM). 
\PUSH уменьшает \ESP/\RSP/\ac{SP} на 4 в 32-битном режиме (или на 8 в 64-битном),
затем записывает по адресу, на который указывает \ESP/\RSP/\ac{SP}, содержимое своего единственного операнда.

\POP это обратная операция~--- сначала достает из \glslink{stack pointer}{указателя стека} значение и помещает его в операнд 
(который очень часто является регистром) и затем увеличивает указатель стека на 4 (или 8).

В самом начале \glslink{stack pointer}{регистр-указатель} указывает на конец стека.
Конец стека находится в начале блока памяти, выделенного под стек. Это странно, но это так.
\PUSH уменьшает \glslink{stack pointer}{регистр-указатель}, а \POP~--- увеличивает.

В процессоре ARM, тем не менее, есть поддержка стеков, растущих как в сторону уменьшения, так и в сторону увеличения.

\myindex{ARM!\Instructions!STMFD}
\myindex{ARM!\Instructions!LDMFD}
\myindex{ARM!\Instructions!STMED}
\myindex{ARM!\Instructions!LDMED}
\myindex{ARM!\Instructions!STMFA}
\myindex{ARM!\Instructions!LDMFA}
\myindex{ARM!\Instructions!STMEA}
\myindex{ARM!\Instructions!LDMEA}

Например, инструкции \ac{STMFD}/\ac{LDMFD}, \ac{STMED}/\ac{LDMED} предназначены для descending-стека (растет назад, начиная с высоких адресов в сторону низких).\\
Инструкции \ac{STMFA}/\ac{LDMFA}, \ac{STMEA}/\ac{LDMEA} предназначены для ascending-стека (растет вперед, начиная с низких адресов в сторону высоких).

% It might be worth mentioning that STMED and STMEA write first,
% and then move the pointer,
% and that LDMED and LDMEA move the pointer first, and then read.
% In other words, ARM not only lets the stack grow in a non-standard direction,
% but also in a non-standard order.
% Maybe this can be in the glossary, which would explain why E stands for "empty".

\subsection{Почему стек растет в обратную сторону?}
\label{stack_grow_backwards}

Интуитивно мы можем подумать, что, как и любая другая структура данных, стек мог бы расти вперед, т.е. в сторону увеличения адресов.

Причина, почему стек растет назад, видимо, историческая.
Когда компьютеры были большие и занимали целую комнату, было очень легко разделить сегмент на две части: для \glslink{heap}{кучи} и для стека.
Заранее было неизвестно, насколько большой может быть \glslink{heap}{куча} или стек, так что это решение было самым простым.

\input{patterns/02_stack/stack_and_heap}

В \RitchieThompsonUNIX можно прочитать:

\begin{framed}
\begin{quotation}
The user-core part of an image is divided into three logical segments. The program text segment begins at location 0 in the virtual address space. During execution, this segment is write-protected and a single copy of it is shared among all processes executing the same program. At the first 8K byte boundary above the program text segment in the virtual address space begins a nonshared, writable data segment, the size of which may be extended by a system call. Starting at the highest address in the virtual address space is a stack segment, which automatically grows downward as the hardware's stack pointer fluctuates.
\end{quotation}
\end{framed}

Это немного напоминает как некоторые студенты
пишут два конспекта в одной тетрадке:
первый конспект начинается обычным образом, второй пишется с конца, перевернув тетрадку.
Конспекты могут встретиться где-то посредине, в случае недостатка свободного места.

% I think if we want to expand on this analogy,
% one might remember that the line number increases as as you go down a page.
% So when you decrease the address when pushing to the stack, visually,
% the stack does grow upwards.
% Of course, the problem is that in most human languages,
% just as with computers,
% we write downwards, so this direction is what makes buffer overflows so messy.

\subsection{Для чего используется стек?}

% subsections
\input{patterns/02_stack/01_saving_ret_addr}
\input{patterns/02_stack/02_args_passing}
\EN{\input{patterns/02_stack/03_local_vars_EN}}
\RU{\input{patterns/02_stack/03_local_vars_RU}}
\PTBR{\input{patterns/02_stack/03_local_vars_PTBR}}
\input{patterns/02_stack/04_alloca/main}
\input{patterns/02_stack/05_SEH}
\input{patterns/02_stack/06_BO_protection}

\subsubsection{Автоматическое освобождение данных в стеке}

Возможно, причина хранения локальных переменных и SEH-записей в стеке в том, что после выхода из функции, всё эти данные освобождаются автоматически,
используя только одну инструкцию корректирования указателя стека (часто это \ADD).
Аргументы функций, можно сказать, тоже освобождаются автоматически в конце функции.
А всё что хранится в куче (\IT{heap}) нужно освобождать явно.

% sections
\EN{\input{patterns/02_stack/07_layout_EN}}
\RU{\input{patterns/02_stack/07_layout_RU}}
\PTBR{\input{patterns/02_stack/07_layout_PTBR}}
\input{patterns/02_stack/08_noise/main}
\input{patterns/02_stack/exercises}

}
\PTBR{\mysection{\Stack}
\label{sec:stack}
\myindex{\Stack}

A pilha é uma das estruturas mais fundamentais na ciência da computação.
\footnote{\href{http://go.yurichev.com/17119}{wikipedia.org/wiki/Call\_stack}}.
\ac{AKA} \ac{LIFO}.

Tecnicamente, é só um bloco de memória junto com os registradores \ESP ou \RSP em x86 e x64, ou o \ac{SP} no ARM, como um ponteiro para aquele bloco.

\myindex{ARM!\Instructions!PUSH}
\myindex{ARM!\Instructions!POP}
\myindex{x86!\Instructions!PUSH}
\myindex{x86!\Instructions!POP}
As instruções mais frequente para o acesso da pilha são \PUSH e \POP (em ambos x86 e x64).
\PUSH subtrai de \ESP/\RSP/\ac{SP} 4 no modo 32-bits (ou 8 no modo 64-bits) e então escreve o conteúdo desse operando único para o endereço de memória apontado por \ESP/\RSP/\ac{SP}.

\POP é a operação reversa: recupera a informação da localização de memória que é apontada por \ac{SP}, 
carrega a mesma no operando da instrução (geralmente um registrador) e então adiciona 4 (ou 8) para o ponteiro da pilha.

Depois da alocação da pilha, o ponteiro aponta para o fundo da pilha.
\PUSH decrementa o ponteiro da pilha e \POP incrementa. O fundo da pilha está na verdade no começo do bloco de memória alocado para ela.
Pode parecer estranho, mas é a maneira como é feita.

ARM: \PTBRph{}

\subsection{Por que a pilha ``cresce'' para trás?}
\label{stack_grow_backwards}

Intuitivamente, nós podemos pensar que a pilha cresce para frente, em direção a endereços mais altos, como qualquer outra estrutura de informação.

O motivo da pilha crescer para trás é provavelmente histórico. Quando os computadores era grandes e ocupavam um cômodo todo, era mais fácil dividir a memória em duas partes, uma para a ‘heap’ e outra para a pilha.
Logicamente, era desconhecido o quão grande a heap e a pilha seriam durante a execução do programa, então essa solução era a mais simples possível.

\input{patterns/02_stack/stack_and_heap}

No \RitchieThompsonUNIX nós podemos ler:

\begin{framed}
\begin{quotation}
A parte relacionada ao usuário é dividida em três segmentos lógicos. O segmento de texto do programa começa na localização 0 no espaço virtual de endereçamento.
Durante a execução, esse segmento é protegido para não ser reescrito e uma única cópia dele é compartilhado entre
todos os processos executando o mesmo programa.
Começando no limite de 8Kbytes acima do segmento de texto do programa no espaço de endereçamento virtual começa um segmento de informação gravável,
não compartilhável e de um tamanho que pode ser extendido por uma chamada do sistema.
Começando no endereço mais alto no espaço de endereçamento virtual está a pilha, que automaticamente cresce para trás conforme o ponteiro da pilha do hardware se altera.
\end{quotation}
\end{framed}

Isso pode ser análogo a como um estudante escreve notas de duas matérias diferentes em um caderno só:
as notas para a primeira matéria são escritas como de costume e as notas para a segunda são escritas do final do caderno,
virando o mesmo. As anotações de uma matéria podem encontrar as da outra no meio, no caso de haver falta de espaço.

\subsection{Para que a pilha é usada?}

% subsections
\input{patterns/02_stack/01_saving_ret_addr}
\input{patterns/02_stack/02_args_passing}
\EN{\input{patterns/02_stack/03_local_vars_EN}}
\RU{\input{patterns/02_stack/03_local_vars_RU}}
\PTBR{\input{patterns/02_stack/03_local_vars_PTBR}}
\input{patterns/02_stack/04_alloca/main}
\input{patterns/02_stack/05_SEH}
\input{patterns/02_stack/06_BO_protection}

\subsubsection{\PTBRph{}}

Talvez, o motivo para armazenar variáveis locais e registros SEH na pilha é que eles são desvinculados automaticamente depois do fim da função,
usando somente uma instrução para corrigir o ponteiro da pilha (geralmente é \ADD). Argumentos de funções, como podemos dizer, são
também desalocados automaticamente com o fim da função.
Como contraste, tudo armazenado na memória heap tem de ser desalocado explicitamente.

% sections
\EN{\input{patterns/02_stack/07_layout_EN}}
\RU{\input{patterns/02_stack/07_layout_RU}}
\PTBR{\input{patterns/02_stack/07_layout_PTBR}}
\input{patterns/02_stack/08_noise/main}
\input{patterns/02_stack/exercises}

}
\ITA{\section{\Stack}
\label{sec:stack}
\myindex{\Stack}

Lo stack e' una delle strutture dati piu' importanti in informatica
\footnote{\href{http://go.yurichev.com/17119}{wikipedia.org/wiki/Call\_stack}}.
\ac{AKA} \ac{LIFO}.

Tecnicamente, e' soltanto un blocco di memoria nella memoria di un processo insieme al registro \ESP o \RSP in x86 o x86, o il registro \ac{SP} in ARM, come puntatore all'interno di quel blocco.

\myindex{ARM!\Instructions!PUSH}
\myindex{ARM!\Instructions!POP}
\myindex{x86!\Instructions!PUSH}
\myindex{x86!\Instructions!POP}
Le istruzioni di accesso allo stack piu' usate sono \PUSH e \POP (sia in x86 che in ARM Thumb-mode).
\PUSH sottrae da \ESP/\RSP/\ac{SP} 4 in modalita' 32-bit (oppur 8 in modalita' 64-bit) e scrive successivamente il contenuto del suo unico operando nell'indirizzo di memoria puntato da \ESP/\RSP/\ac{SP}.

\POP e' l'operazione inversa: recupera il dato dalla memoria a cui punta \ac{SP}, lo carica nell'operando dell'istruzione (di solito un registro)
e successivamente aggiunge 4 (o 8) allo \gls{stack pointer}.

A seguito dell'allocazione dello stack, lo \gls{stack pointer} punta alla base (fondo) dello stack.
\PUSH decrementa lo \gls{stack pointer} e \POP lo incrementa.
La base dello stack e' in realta' all'inizio della memoria allocata per il blocco (porzione) dello stack. Sembra strano, ma e' cosi'.

ARM supporta stack decrescenti e crescenti.

\myindex{ARM!\Instructions!STMFD}
\myindex{ARM!\Instructions!LDMFD}
\myindex{ARM!\Instructions!STMED}
\myindex{ARM!\Instructions!LDMED}
\myindex{ARM!\Instructions!STMFA}
\myindex{ARM!\Instructions!LDMFA}
\myindex{ARM!\Instructions!STMEA}
\myindex{ARM!\Instructions!LDMEA}

Ad esempio le istruzioni \ac{STMFD}/\ac{LDMFD}, \ac{STMED}/\ac{LDMED} sono fatte per operare con uno stack decrescente (che cresce verso il basso, inizia con un indirizzo alto e prosegue verso il basso).
Le istruzioni \ac{STMFA}/\ac{LDMFA}, \ac{STMEA}/\ac{LDMEA} sono fatte per operare con uno stack crescente (che cresce verso l'alto, da un indirizzo basso verso uno piu alto).

% It might be worth mentioning that STMED and STMEA write first,
% and then move the pointer,
% and that LDMED and LDMEA move the pointer first, and then read.
% In other words, ARM not only lets the stack grow in a non-standard direction,
% but also in a non-standard order.
% Maybe this can be in the glossary, which would explain why E stands for "empty".

\subsection{Perche' lo stack cresce al contrario?}
\label{stack_grow_backwards}

Intuitivamente potremmo pensare che lo stack cresca verso l'alto, ovvero verso indirizzi piu' alti, come qualunque altra struttura dati.

La ragione per cui lo stack cresce verso il basso e' probabilmente di natura storica.
Quando i computer erano talmente grandi da occupare un'intera stanza, era facile dividere la memoria in due parti, una per lo 
\gls{heap} e l'altra per lo stack.
Ovviamente non era possibile sapere a priori quanto sarebbero stati grandi lo stack e lo \gls{heap} durante l'esecuzione di un programma,
e questa soluzione era la piu' semplice.

\input{patterns/02_stack/stack_and_heap}

In \RitchieThompsonUNIX possiamo leggere:

\begin{framed}
\begin{quotation}
The user-core part of an image is divided into three logical segments.
The program text segment begins at location 0 in the virtual address space.
During execution, this segment is write-protected and a single copy of it is shared among all processes executing the same program.
At the first 8K byte boundary above the program text segment in the virtual address space begins a nonshared, writable data segment, the size of which may be extended by a system call.
Starting at the highest address in the virtual address space is a stack segment, which automatically grows downward as the hardware's stack pointer fluctuates.

Il nucleo utente di una immagine e' diviso in tre segmenti logici.
Il segmento text del programma inizia in posizione 0 nel virtual address space.
Durante l'esecuzione questo segmento viene protetto da scrittura, ed una sua singola copia viene condivisa tra i processi che eseguono lo stesso programma.
Al primo limite di 8K byte dopra il segmento text del programma, nel virtual address space comincia un segmento dati scrivibile, non condiviso, le cui dimensioni possono essere estese da una chiamata di sistema.
A partire dall'indirizzo piu' alto nel virtual address space c'e' lo stack segment, che automaticammente cresce verso il basso al variare dello stack pointer hardware.
\end{quotation}
\end{framed}

Questo ricorda molto come alcuni studenti utilizzino lo stesso quaderno per prendere appunti di due diverse materie:
gli appunti per la prima materia sono scritti normalmente, e quelli della seconda materia sono scritti a partire dalla fine del quaderno, capovolgendolo.
Le note si potrebbero "incontrare" da qualche parte in mezzo al quaderno, nel caso in cui non ci sia abbastanza spazio libero.

% I think if we want to expand on this analogy,
% one might remember that the line number increases as as you go down a page.
% So when you decrease the address when pushing to the stack, visually,
% the stack does grow upwards.
% Of course, the problem is that in most human languages,
% just as with computers,
% we write downwards, so this direction is what makes buffer overflows so messy.

\subsection{Per cosa viene usato lo stack?}

% subsections
\input{patterns/02_stack/01_saving_ret_addr}
\input{patterns/02_stack/02_args_passing}
\EN{\input{patterns/02_stack/03_local_vars_EN}}
\RU{\input{patterns/02_stack/03_local_vars_RU}}
\PTBR{\input{patterns/02_stack/03_local_vars_PTBR}}
\input{patterns/02_stack/04_alloca/main}
\input{patterns/02_stack/05_SEH}
\input{patterns/02_stack/06_BO_protection}

\subsubsection{Deallocazione automatica dei dati nello stack}

Probabilmente la ragione per cui si memorizzano nello stack le variabili locali e i record SEH deriva dal fatto che questi dati vengono "liberati" automaticamente all'uscita dalla funzione,
usando soltanto un'istruzione per correggere lo stack pointer (spesso e' \ADD).
Si puo' dire che anche gli argomenti delle funzioni sono deallocati automaticamente alla fine della funzione.
Invece, qualunque altra cosa memorizzata nello \IT{heap} deve essere deallocata esplicitamente.

% sections
\EN{\input{patterns/02_stack/07_layout_EN}}
\RU{\input{patterns/02_stack/07_layout_RU}}
\PTBR{\input{patterns/02_stack/07_layout_PTBR}}
\input{patterns/02_stack/08_noise/main}
\input{patterns/02_stack/exercises}
}
\DE{\section{\Stack}
\label{sec:stack}
\myindex{\Stack}

Der Stack ist eine der Fundamentalen Datenstrukturen in der Informatik.
\footnote{\href{http://go.yurichev.com/17119}{wikipedia.org/wiki/Call\_Stack}}.
\ac{AKA} \ac{LIFO}.

Technisch betrachtet ist es ein Stapel Speicher innerhalb des Prozessspeichers der zusammen mit den \ESP (x86), \RSP (x64) oder dem \ac{SP} (ARM) Register als ein Zeiger in diesem Speicherblock fungiert.

\myindex{ARM!\Instructions!PUSH}
\myindex{ARM!\Instructions!POP}
\myindex{x86!\Instructions!PUSH}
\myindex{x86!\Instructions!POP}

Die häufigsten Stack-Zugriffsinstruktionen sind die \PUSH und \POP Instruktionen (in beidem x86 und ARM Thumb-Modus). \PUSH subtrahiert vom \ESP/\RSP/\ac{SP} 4 Byte im 32-Bit Modus (oder 8 im 64-Bit Modus) und schreibt dann den Inhalt des Zeigers an die Adresse auf die von \ESP/\RSP/\ac{SP} gezeigt wird.

\POP ist die umgekehrte Operation: Die Daten des Zeigers für die Speicherregion auf die von \ac{SP}
gezeigt wird werden ausgelesen und die Inhalte in den Instruktionsoperanden geschreiben (oft ist das ein Register). Dann werden 4 (beziehungsweise 8 ) Byte zum \gls{stack pointer} addiert.

Nach der Stackallokation, zeigt der \gls{stack pointer} auf den Boden des Stacks.
\PUSH verringert den \gls{stack pointer} und \POP erhöht ihn.
Der Boden des Stacks ist eigentlich der Anfang der Speicherregion die für den Stack reserviert wurde.
Das wirkt zunächst seltsam, aber so funktioniert es.

ARM unterstützt beides, aufsteigende und absteigende Stacks.

\myindex{ARM!\Instructions!STMFD}
\myindex{ARM!\Instructions!LDMFD}
\myindex{ARM!\Instructions!STMED}
\myindex{ARM!\Instructions!LDMED}
\myindex{ARM!\Instructions!STMFA}
\myindex{ARM!\Instructions!LDMFA}
\myindex{ARM!\Instructions!STMEA}
\myindex{ARM!\Instructions!LDMEA}

Zum Beispiel die \ac{STMFD}/\ac{LDMFD} und \ac{STMED}/\ac{LDMED} Instruktionen sind alle dafür gedacht mit einem absteigendem Stack zu arbeiten ( wächst nach unten, fängt mit hohen Adressen an und entwickelt sich zu niedrigeren Adressen). Die \ac{STMFA}/\ac{LDMFA} und \ac{STMEA}/\ac{LDMEA} Instruktionen sind dazu gedacht mit einem aufsteigendem Stack zu arbeiten (wächst nach oben und fängt mit niedrigeren Adressen an und wächst nach oben).

% It might be worth mentioning that STMED and STMEA write first,
% and then move the pointer, and that LDMED and LDMEA move the pointer first, and then read.
% In other words, ARM not only lets the stack grow in a non-standard direction,
% but also in a non-standard order.
% Maybe this can be in the glossary, which would explain why E stands for "empty".

\subsection{Warum wächst der Stack nach unten?}
\label{stack_grow_backwards}

Intuitiv, würden man annehmen das der Stack nach oben wächst z.B Richtung höherer Adressen, so wie bei jeder anderen Datenstruktur.

Der Grund das der Stack rückwärts wächst ist wohl historisch bedingt. Als Computer so groß waren das sie einen ganzen Raum beansprucht haben war es einfach Speicher in zwei Sektionen zu unterteilen, einen Teil für den \gls{heap} und einen Teil für den Stack. Sicher war zu dieser Zeit nicht bekannt wie groß der \gls{heap} und der Stack wachsen würden, während der Programm Laufzeit, also war die Lösung die einfachste mögliche.

\input{patterns/02_stack/stack_and_heap}

In \RitchieThompsonUNIX können wir folgendes lesen:

\begin{framed}
\begin{quotation}
Der user-core eines Programm Images wird in drei logische Segmente unterteilt. Das Programm-Text Segment beginnt bei 0 im virtuellen Adress Speicher. Während der Ausführung wird das Segment als schreibgeschützt markiert und eine einzelne Kopie des Segments wird unter allen Prozessen geteilt die das Programm ausführen. An der ersten 8K grenze über dem Programm Text Segment im Virtuellen Speicher, fängt der ``nonshared'' Bereich an, der nach Bedarf von Syscalls erweitert werden kann. Beginnend bei der höchsten Adresse im Virtuellen Speicher ist das Stack Segment, das Automatisch nach unten wächst während der Hardware Stackpointer sich ändert.
\end{quotation}
\end{framed}

Das erinnert daran wie manche Schüler Notizen zu  zwei Vorträgen in einem Notebook dokumentieren:
Notizen für den ersten Vortrag werden normal notiert, und Notizen zur zum zweiten Vortrag werden 
ans Ende des Notizbuches geschrieben, indem man das Notizbuch umdreht. Die Notizen treffen sich irgendwann
im Notizbuch aufgrund des fehlenden Freien Platzes.

% I think if we want to expand on this analogy,
% one might remember that the line number increases as as you go down a page.
% So when you decrease the address when pushing to the stack, visually,
% the stack does grow upwards.
% Of course, the problem is that in most human languages,
% just as with computers,
% we write downwards, so this direction is what makes buffer overflows so messy.

\subsection{Für was wird der Stack benutzt?}

% subsections
\input{patterns/02_stack/01_saving_ret_addr}
\input{patterns/02_stack/02_args_passing}
\EN{\input{patterns/02_stack/03_local_vars_EN}}
\RU{\input{patterns/02_stack/03_local_vars_RU}}
\DE{\input{patterns/02_stack/03_local_vars_DE}}
\PTBR{\input{patterns/02_stack/03_local_vars_PTBR}}
\input{patterns/02_stack/04_alloca/main}
\input{patterns/02_stack/05_SEH}
\input{patterns/02_stack/06_BO_protection}

\subsubsection{Automatisches deallokieren der Daten auf dem Stack}

Vielleicht ist der Grund warum man lokale Variablen und SEH Einträge auf dem Stack speichert, weil sie beim 
verlassen der Funktion automatisch aufgeräumt werden. Man braucht dabei nur eine Instruktion um die Position
des Stackpointers zu korrigieren (oftmals ist es die \ADD Instruktion). Funktions Argumente, könnte man sagen 
werden auch am Ende der Funktion deallokiert. Im Kontrast dazu, alles was auf dem \IT{heap} gespeichert wird muss
explizit deallokiert werden. 

% sections
\EN{\input{patterns/02_stack/07_layout_EN}}
\RU{\input{patterns/02_stack/07_layout_RU}}
\DE{\input{patterns/02_stack/07_layout_DE}}
\PTBR{\input{patterns/02_stack/07_layout_PTBR}}
\input{patterns/02_stack/08_noise/main}
\input{patterns/02_stack/exercises}
}


\EN{\chapter{\PrintfSeveralArgumentsSectionName}

Now let's extend the \IT{\HelloWorldSectionName}~(\myref{sec:helloworld}) example, replacing \printf in
the \main function body with this:

\lstinputlisting[label=hw_c]{patterns/03_printf/1.c}

% sections
\input{patterns/03_printf/x86/main}
\ifdefined\IncludeARM
\input{patterns/03_printf/ARM/main}
\fi
\ifdefined\IncludeMIPS
\input{patterns/03_printf/MIPS/main}
\fi

\section{\Conclusion{}}

Here is a rough skeleton of the function call:

\lstinputlisting[caption=x86]{patterns/03_printf/skel1.lst.\LANG}

\lstinputlisting[caption=x64 (MSVC)]{patterns/03_printf/skel2.lst.\LANG}

\ifdefined\IncludeGCC
\lstinputlisting[caption=x64 (GCC)]{patterns/03_printf/skel3.lst.\LANG}
\fi

\ifdefined\IncludeARM
\lstinputlisting[caption=ARM]{patterns/03_printf/skel4.lst.\LANG}

\lstinputlisting[caption=ARM64]{patterns/03_printf/skel5.lst.\LANG}
\fi

\ifdefined\IncludeMIPS
\index{MIPS!O32}
\lstinputlisting[caption=MIPS (O32 calling convention)]{patterns/03_printf/skel_MIPS.lst.\LANG}
\fi

\section{By the way}

\index{fastcall}
By the way, this difference between the arguments passing in x86, x64, 
fastcall, ARM and MIPS is a good illustration of the fact that the CPU is oblivious to how the arguments are passed to functions. 
It is also possible to create a hypothetical compiler able to pass arguments 
via a special structure without using stack at all.

\ifdefined\IncludeMIPS
\index{MIPS!O32}
MIPS \$A0 \dots \$A3 registers are labelled this way only for convenience (that is in the O32 calling convention).
Programmers may use any other register (well, maybe except \$ZERO) 
to pass data or use any other calling convention.
\fi

The \ac{CPU} is not aware of calling conventions whatsoever.

We may also recall how newcoming assembly language programmers passing arguments into
other functions:
usually via registers, without any explicit order, or even via global variables.
Of course, it works fine.

}
\RU{\chapter{\PrintfSeveralArgumentsSectionName}

Попробуем теперь немного расширить пример \IT{\HelloWorldSectionName}~(\myref{sec:helloworld}),
написав в теле функции \main:

\lstinputlisting[label=hw_c]{patterns/03_printf/1.c}

% sections
\input{patterns/03_printf/x86/main}
\ifdefined\IncludeARM
\input{patterns/03_printf/ARM/main}
\fi
\ifdefined\IncludeMIPS
\input{patterns/03_printf/MIPS/main}
\fi

\section{\Conclusion{}}

Вот примерный скелет вызова функции:

\lstinputlisting[caption=x86]{patterns/03_printf/skel1.lst.\LANG}

\lstinputlisting[caption=x64 (MSVC)]{patterns/03_printf/skel2.lst.\LANG}

\ifdefined\IncludeGCC
\lstinputlisting[caption=x64 (GCC)]{patterns/03_printf/skel3.lst.\LANG}
\fi

\ifdefined\IncludeARM
\lstinputlisting[caption=ARM]{patterns/03_printf/skel4.lst.\LANG}

\lstinputlisting[caption=ARM64]{patterns/03_printf/skel5.lst.\LANG}
\fi

\ifdefined\IncludeMIPS
\index{MIPS!O32}
\lstinputlisting[caption=MIPS (соглашение о вызовах O32)]{patterns/03_printf/skel_MIPS.lst.\LANG}
\fi

\section{Кстати}

\index{fastcall}
Кстати, разница между способом передачи параметров принятая в x86, x64, fastcall, ARM и MIPS неплохо иллюстрирует тот важный момент, что процессору, в общем, всё равно, как будут 
передаваться параметры функций. Можно создать гипотетический компилятор, который будет передавать их при 
помощи указателя на структуру с параметрами, не пользуясь стеком вообще.

\ifdefined\IncludeMIPS
\index{MIPS!O32}
Регистры \$A0\dots \$A3 в MIPS так названы только для удобства (это соглашение о вызовах O32).
Программисты могут использовать любые другие регистры (может быть, только кроме \$ZERO) для
передачи данных или любое другое соглашение о вызовах.
\fi

\ac{CPU} не знает о соглашениях о вызовах вообще.

Можно также вспомнить, что начинающие программисты на ассемблере передают параметры 
в другие функции обычно через регистры, без всякого явного порядка, или даже через глобальные переменные.
И всё это нормально работает.

}
\PTBR{\chapter{\PrintfSeveralArgumentsSectionName}

Agora vamos extender o nosso exemplo \IT{\HelloWorldSectionName}~(\myref{sec:helloworld}),
trocando \printf no corpo da função \main() por isso:

\lstinputlisting[label=hw_c]{patterns/03_printf/1.c}

% sections
\input{patterns/03_printf/x86/main}
\ifdefined\IncludeARM
\input{patterns/03_printf/ARM/main}
\fi
\ifdefined\IncludeMIPS
\input{patterns/03_printf/MIPS/main}
\fi

\section{\Conclusion{}}

Aqui está uma estrutura bem rústica da chamada da função

\lstinputlisting[caption=x86]{patterns/03_printf/skel1.lst.\LANG}

\lstinputlisting[caption=x64 (MSVC)]{patterns/03_printf/skel2.lst.\LANG}

\ifdefined\IncludeGCC
\lstinputlisting[caption=x64 (GCC)]{patterns/03_printf/skel3.lst.\LANG}
\fi

\ifdefined\IncludeARM
\lstinputlisting[caption=ARM]{patterns/03_printf/skel4.lst.\LANG}

\lstinputlisting[caption=ARM64]{patterns/03_printf/skel5.lst.\LANG}
\fi

\ifdefined\IncludeMIPS
\index{MIPS!O32}
\lstinputlisting[caption=MIPS (\PTBRph{})]{patterns/03_printf/skel_MIPS.lst.\LANG}
\fi

\section{A propósito}

\index{fastcall}
A propósito, a diferença entre os argumentos passados em x86, x64, fastcall, ARM e MIPS  é uma boa demonstração do fato de como a CPU é indiferente sobre como os argumentos são passados para as funções.
Também é possível criar um compilador hipotético capaz de passar argumentos por alguma outra estrutura especial sem usar a pilha de nenhuma maneira.

\ifdefined\IncludeMIPS
\index{MIPS!O32}
\PTBRph{}
\fi

A \ac{CPU} não está ciente de convenções de chamada de funções.

Agora nós podemos também relembrar de dos programadores novatos de assembly passando argumentos para outras funções:
geralmente via registradores, sem nenhuma sequência explícita, ou mesmo por variáveis globais. Logicamente, também funciona.

}
\ITA{\chapter{\PrintfSeveralArgumentsSectionName}

Estendiamo l'esempio \IT{\HelloWorldSectionName}~(\myref{sec:helloworld}) modificando la chiamata a  \printf nella funzione \main:

\lstinputlisting[label=hw_c]{patterns/03_printf/1.c}

% sections
\input{patterns/03_printf/x86/main}
\ifdefined\IncludeARM
\input{patterns/03_printf/ARM/main}
\fi
\ifdefined\IncludeMIPS
\input{patterns/03_printf/MIPS/main}
\fi

\section{\Conclusion{}}

Si riporta di seguito una lista di bozze di chiamate alla call:

\lstinputlisting[caption=x86]{patterns/03_printf/skel1.lst.\LANG}

\lstinputlisting[caption=x64 (MSVC)]{patterns/03_printf/skel2.lst.\LANG}

\ifdefined\IncludeGCC
\lstinputlisting[caption=x64 (GCC)]{patterns/03_printf/skel3.lst.\LANG}
\fi

\ifdefined\IncludeARM
\lstinputlisting[caption=ARM]{patterns/03_printf/skel4.lst.\LANG}

\lstinputlisting[caption=ARM64]{patterns/03_printf/skel5.lst.\LANG}
\fi

\ifdefined\IncludeMIPS
\myindex{MIPS!O32}
\lstinputlisting[caption=MIPS (O32 calling convention)]{patterns/03_printf/skel_MIPS.lst.\LANG}
\fi

\section{A proposito...}

\myindex{fastcall}
Le differenze negli approcci utilizzati per il passaggio di argomenti in x86, x64, 
fastcall, ARM and MIPS e' un'ottima dimostrazione del fatto che la CPU e' inconsapevole di come gli argomenti vengono passati alle funzioni. 
Sarebbe anche possibile creare un compilatore ipotetico in grado di passare gli argomenti attraverso una struttura speciale, senza usare lo stack.

\ifdefined\IncludeMIPS
\myindex{MIPS!O32}
I registri MIPS \$A0 \dots \$A3 sono indicati in questo modo soltanto per convenienza (cioe' nella O32 calling convention).
I programmatori possono usare qualunque altro registro (tranne \$ZERO) per passare i dati, o utilizzare qualunque altra calling convention. 
\fi

La \ac{CPU} non e' assolutamente consapevole delle calling conventions.

Possiamo anche ricordare come i programmatori principianti in assembly passano gli argomenti alle altre funzioni: 
di suolito tramite i registri, senza un ordine esplicitop, o attraverso variabili globali.
Questi approcci sono ovviamente validi e funzionanti.
}
\DE{\section{\PrintfSeveralArgumentsSectionName}

An dieser Stelle wird das \IT{\HelloWorldSectionName}~(\myref{sec:helloworld})-Beispiel ein
wenig erweitert, indem \printf in der \main-Funktion durch folgendes ersetzt wird:

\lstinputlisting[label=hw_c,style=customc]{patterns/03_printf/1.c}

% sections
\input{patterns/03_printf/x86/main}
\input{patterns/03_printf/ARM/main}
\input{patterns/03_printf/MIPS/main}

\subsection{\Conclusion{}}

Hier ist der grobe Aufbau der Aufruffunktion:

\begin{lstlisting}[caption=x86,style=customasmx86]
...
PUSH Drittes Argument
PUSH Zweites Argument
PUSH Erstes Argument
CALL Funktion
; gegebenenfalls den Stackpointer modifizieren
\end{lstlisting}

\begin{lstlisting}[caption=x64 (MSVC),style=customasmx86]
MOV RCX, Erstes Argument
MOV RDX, Zweites Argument
MOV R8, Drittes Argument
MOV R9, Viertes Argument
...
PUSH fünftes, sechstes Argument, usw. (falls notwendig)
CALL Funktion
; gegebenenfalls den Stackpointer modifizieren
\end{lstlisting}

\begin{lstlisting}[caption=x64 (GCC),style=customasmx86]
MOV RDI, Erstes Argument
MOV RSI, Zweites Argument
MOV RDX, Drittes Argument
MOV RCX, Viertes Argument
MOV R8, Fünftes Argument
MOV R9, Sechstes Argument
...
PUSH Siebtes, Achtes Argument, usw. (falls notwendig)
CALL Funktion
; gegebenenfalls den Stackpointer modifizieren
\end{lstlisting}

\begin{lstlisting}[caption=ARM,style=customasmARM]
MOV R0, Erstes Argument
MOV R1, Zweites Argument
MOV R2, Drittes Argument
MOV R3, Viertes Argument
; Fünftes, Sechstes Argument, usw. auf den Stack (falls notwendig)
BL Funktion
; gegebenenfalls den Stackpointer modifizieren
\end{lstlisting}

\begin{lstlisting}[caption=ARM64,style=customasmARM]
MOV X0, Erstes Argument
MOV X1, Zweites Argument
MOV X2, Drittes Argument
MOV X3, Viertes Argument
MOV X4, Fünftes Argument
MOV X5, Sechstes Argument
MOV X6, Siebtes Argument
MOV X7, Achtes Argument
; Neuntes, Zehntes Argument, usw. auf den Stack (falls notwendig)
BL Funktion
; gegebenenfalls den Stackpointer modifizieren
\end{lstlisting}

\myindex{MIPS!O32}
\begin{lstlisting}[caption=MIPS (O32 calling convention),style=customasmMIPS]
LI $4, Erstes argument ; AKA $A0
LI $5, Zweites argument ; AKA $A1
LI $6, Drittes argument ; AKA $A2
LI $7, Viertes argument ; AKA $A3
; pass Fünftes, Sechstes argument, usw. auf den Stack (falls notwendig)
LW temporäres Register, Adresse der Funktion
JALR temporäres Regist
\end{lstlisting}

\subsection{Übrigens\dots{}}

\myindex{fastcall}
Übrigens ist der Unterschied der Art der Argumenten Übergabe in x86, x64, fastcall, ARM und MIPS eine gute
Darstellung der Tatsache, dass die CPU nicht weiß wie die Argumente an die Funktion übergeben werden.
Es ist auch möglich einen hypothetischen Compiler zu erstellen, der die Möglichkeit hat Argumente mittels
einer speziellen Struktur, ohne den Stack an die Funktionen zu übergebe.

\myindex{MIPS!O32}
MIPS \$A0 \dots \$A3-Register sind aus Bequemlichkeitsgründen auf diese Weise beschriftet (O32 Aufrufkonvention).
Programmierer können auch andere Register (vielleicht außer \$ZERO) nutzen um Daten zu übergeben
oder eine andere Aufrufkonvention zu nutzen.

Die \ac{CPU} hatte jedoch keinerlei Kenntnisse über die Aufrufkonvention.

Man sieht hier auch wie Neulinge der Assemblersprache Argumente an andere Funktionen übergeben:
in der Regel per Register ohne explizite Reihenfolge oder globale Variablen.
Natürlich funktioniert das ebenso gut.
}

\section{scanf()}
\myindex{\CStandardLibrary!scanf()}
\label{label_scanf}

\RU{Теперь попробуем использовать scanf().}%
\EN{Now let's use scanf().}%
\PTBR{Agora vamos usar a função scanf().}%
\FR{Maintenant utilisons la fonction scanf().}

% subsections
\EN{\input{patterns/04_scanf/1_simple/main_EN}}
\RU{\input{patterns/04_scanf/1_simple/main_RU}}
\PTBR{\input{patterns/04_scanf/1_simple/main_PTBR}}
\ITA{\input{patterns/04_scanf/1_simple/main_ITA}}
\DE{\input{patterns/04_scanf/1_simple/main_DE}}
\FR{\input{patterns/04_scanf/1_simple/main_FR}}

\EN{\input{patterns/04_scanf/error_EN}}
\DE{\input{patterns/04_scanf/error_DE}}
\FR{\input{patterns/04_scanf/error_FR}}
\PL{\input{patterns/04_scanf/error_PL}}
\JPN{\input{patterns/04_scanf/error_JPN}}

\ifdefined\ENGLISH
\newcommand{\GlobalVarsSectionName}{Global variables}
\section{\GlobalVarsSectionName}
\index{\GlobalVarsSectionName}
\label{scanf_global_variable}

What if the \TT{x} variable from the previous example was not local but a global one? 
Then it would have been accessible from any point, not only from the function body. 
Global variables are considered \gls{anti-pattern}, but for the sake of the experiment, we could do this.
\fi

\ifdefined\RUSSIAN
\newcommand{\GlobalVarsSectionName}{Глобальные переменные}
\section{\GlobalVarsSectionName}
\index{\GlobalVarsSectionName}
\label{scanf_global_variable}

А что если переменная \TT{x} из предыдущего примера будет глобальной переменной, а не локальной? 
Тогда к ней смогут обращаться из любого другого места, а не только из тела функции. 
Глобальные переменные считаются \glslink{anti-pattern}{анти-паттерном},
но ради примера мы можем себе это позволить.
\fi

\ifdefined\BRAZILIAN
\newcommand{\GlobalVarsSectionName}{Variáveis globais}
\section{\GlobalVarsSectionName}
\index{\GlobalVarsSectionName}
\label{scanf_global_variable}

E se a variável \TT{x} do último exemplo não fosse local, mas sim global?
Então ela teria que ser acessível de qualquer ponto, não somente pelo corpo da função.
Variáveis globais são consideradas maus hábitos, mas pelo bem do experimento, nós faremos isso.
\fi

\lstinputlisting{patterns/04_scanf/2_global/ex2.c.\LANG}

\input{patterns/04_scanf/2_global/ex2_global_vars_x86}
\ifdefined\IncludeARM
\input{patterns/04_scanf/2_global/ex2_global_vars_ARM}
\fi
\ifdefined\IncludeMIPS
\input{patterns/04_scanf/2_global/MIPS/main}
\fi

\EN{\input{patterns/04_scanf/3_checking_retval/main_EN}}
\RU{\input{patterns/04_scanf/3_checking_retval/main_RU}}
\PTBR{\input{patterns/04_scanf/3_checking_retval/main_PTBR}}
\ITA{\input{patterns/04_scanf/3_checking_retval/main_ITA}}
\FR{\input{patterns/04_scanf/3_checking_retval/main_FR}}


\subsection{\Exercise}

\begin{itemize}
	\item \url{http://challenges.re/53}
\end{itemize}


\EN{\mysection{Accessing passed arguments}
\myindex{\Stack}

Now we figured out that the \gls{caller} function is passing arguments to the \gls{callee} via the stack. 
But how does the \gls{callee} access them?

\lstinputlisting[label=src:passing_arguments_ex,caption=simple example,style=customc]{patterns/05_passing_arguments/ex.c}

% sections
\input{patterns/05_passing_arguments/x86_EN}
\input{patterns/05_passing_arguments/x64_EN}
\input{patterns/05_passing_arguments/ARM/main.tex}
\input{patterns/05_passing_arguments/MIPS_EN}

}
\RU{\mysection{Доступ к переданным аргументам}
\myindex{\Stack}

Как мы уже успели заметить, вызывающая функция передает аргументы для вызываемой через стек. 
А как вызываемая функция получает к ним доступ?

\lstinputlisting[label=src:passing_arguments_ex,caption=простой пример,style=customc]{patterns/05_passing_arguments/ex.c}

% sections

\input{patterns/05_passing_arguments/x86_RU}
\input{patterns/05_passing_arguments/x64_RU}
\input{patterns/05_passing_arguments/ARM/main.tex}
\input{patterns/05_passing_arguments/MIPS_RU}

}

\chapter{\RU{Еще о возвращаемых результатах}\EN{More about results returning}}

\index{x86!\Registers!EAX}
\index{ARM!\Registers!R0}
\RU{Результат выполнения функции в x86 обычно возвращается}
\EN{As of x86, function execution result is usually returned}
\footnote{\Seealso: 
MSDN: Return Values (C++): \href{http://go.yurichev.com/17258}{MSDN}}
\RU{через регистр \EAX, 
а если результат имеет тип байт или символ (\Tchar), 
то в самой младшей части \EAX ~--- \AL. Если функция возвращает число с плавающей запятой, 
то будет использован регистр FPU \ST{0}.
В ARM обычно результат возвращается в регистре \Reg{0}.}
\EN{in the \EAX register. 
If it is byte type or character (\Tchar)~---then in the lowest register \EAX part~---\AL. 
If a function returns \Tfloat number, the FPU register 
\ST{0} is to be used instead.
In ARM, result is usually returned in the \Reg{0} register.}

\section{\RU{Попытка использовать результат ф-ции возвращающей \Tvoid}
\EN{Attempt to use result of function returning \Tvoid}}

\RU{Кстати, что будет если возвращаемое значение в ф-ции \main объявлять не как \Tint а как \Tvoid?}
\EN{By the way, what if returning value of the \main function will be declared not as \Tint but as \Tvoid?}

\RU{Т.н. startup-код вызывает \main примерно так:}
\EN{so-called startup-code is calling \main roughly as:}

\begin{lstlisting}
push envp
push argv
push argc
call main
push eax
call exit
\end{lstlisting}

\RU{Т.е., иными словами:}\EN{In other words:}

\begin{lstlisting}
exit(main(argc,argv,envp));
\end{lstlisting}

\RU{Если вы объявите \main как \Tvoid, и ничего не будете возвращать явно (при помощи выражения \IT{return}), 
то в единственный аргумент exit() попадет
то, что лежало в регистре \EAX на момент выхода из \main.}
\EN{If you declare \main as \Tvoid and nothing will be returned explicitly (by \IT{return} statement),
then something random, that was stored in the \EAX register at the moment of the \main finish, will come into
the sole exit() function argument.}
\RU{Там, скорее всего, будет какие-то случайное число, оставшееся от работы вашей ф-ции.}
\EN{Most likely, there will be a random value, left from your function execution.}
\RU{Так что, код завершения программы будет псевдослучайным.}
\EN{So, exit code of program will be pseudorandom.} \\

\RU{Я могу это проиллюстрировать}\EN{I can illustrate this fact}. 
\RU{Заметьте что у ф-ции}\EN{Please notice, the} \main \RU{тип возвращаемого значения именно}\EN{function 
has} \Tvoid\EN{ type}:

\begin{lstlisting}
#include <stdio.h>

void main()
{
	printf ("Hello, world!\n");
};
\end{lstlisting}

\RU{Скомпилируем в}\EN{Let's compile it in} Linux.

\index{puts() \RU{вместо}\EN{instead of} printf()}
GCC 4.8.1 \RU{заменила}\EN{replaced} \printf \RU{на}\EN{to} \puts 
(\RU{мы видели это прежде}\EN{we saw this before}: \ref{puts}), 
\RU{но это нормально, потому что}\EN{but that's OK, since} \puts \RU{возвращает количество
выведенных символов, так же как и}\EN{returns number of characters printed, just like} \printf.
\RU{Обратите внимание на то что}\EN{Please notice that} \EAX \RU{не обнуляется перед выходом их}\EN{is not 
zeroed before} \main\EN{ finish}.
\RU{Это значит}\EN{This means}, \EAX \RU{перед выходом из}\EN{value at the} \main 
\RU{будет содержать то, что}\EN{finish will contain what} \puts \RU{оставит там}\EN{left there}.

\begin{lstlisting}[caption=GCC 4.8.1]
.LC0:
	.string	"Hello, world!"
main:
	push	ebp
	mov	ebp, esp
	and	esp, -16
	sub	esp, 16
	mov	DWORD PTR [esp], OFFSET FLAT:.LC0
	call	puts
	leave
	ret
\end{lstlisting}

\index{bash}
\RU{Напишем небольшой скрипт на bash, показывающий статус возврата (``exit status'' или ``exit code'')}
\EN{Let' s write bash script, showing exit status}:

\begin{lstlisting}[caption=tst.sh]
#!/bin/sh
./hello_world
echo $?
\end{lstlisting}

\RU{И запустим}\EN{And run it}:

\begin{lstlisting}
$ tst.sh 
Hello, world!
14
\end{lstlisting}

$14$ \RU{это как раз количество выведенных символов}\EN{is a number of characters printed}.

\section{\RU{Что если не использовать результат ф-ции?}\EN{What if not to use function result?}}

\RU{\printf возвращает количество успешно выведенных символов, но результат работы этой ф-ции 
редко используется на практике.}
\EN{\printf returns count of characters successfully sent to output, but result of this function 
is rarely used in practice.}
\RU{Можно даже явно вызывать ф-ции, чей смысл именно в возвращаемых значениях, но явно не использовать их:}
\EN{It's possible to call functions which essence in returning values, but not to use them explicitely:}

\begin{lstlisting}
int f()
{
    // skip first 3 random values
    rand();
    rand();
    rand();
    // and use 4th
    return rand();
};
\end{lstlisting}

\EN{Result of rand() function will always be leaved in \EAX, in all four cases.}
\RU{Результат работы rand() будет оставаться в \EAX во всех четырех случаях.}
\EN{But in first 3 cases, a value in \EAX will be just thrown away.}
\RU{Но в первых трех случаях, значение лежащее в \EAX, будет просто выброшено.}

\section{\RU{Возврат структуры}\EN{Returning a structure}}

\index{\CLanguageElements!return}
\RU{Вернемся к тому факту, что возвращаемое значение остается в регистре \EAX}
\EN{Let's back to the fact the returning value is left in the \EAX register}.
\RU{Вот почему старые компиляторы Си не способны создавать функции, возвращающие нечто большее нежели 
помещается 
в один регистр (обычно, тип \Tint), а когда нужно, приходится возвращать через указатели, указываемые 
в аргументах.}
\EN{That is why old C compilers cannot create functions capable of returning something not fitting in one 
register (usually type \Tint) but if one needs it, one should return information via pointers passed 
in function arguments.}
\RU{Так что, как правило, если ф-ция должна вернуть несколько значений, она возвращает только одно, 
а остальные --- через указатели.}
\EN{So, usually, if a function should return several values, it returns only one, and 
all the rest---via pointers.}
\RU{Хотя, позже и стало возможным, вернуть, скажем, целую структуру, но этот метод до сих пор не 
очень популярен. 
Если функция должна вернуть структуру, вызывающая функция должна сама, скрыто и прозрачно для программиста, 
выделить место и передать указатель на него в качестве первого аргумента. Это почти то же самое 
что и сделать это вручную, но компилятор прячет это.}
\EN{Now it is possible, to return, let's say, whole structure, but still it is not very popular. 
If function must return a large structure, \gls{caller} must allocate it and pass pointer to it via first argument, 
transparently for programmer. 
That is almost the same as to pass pointer in first argument manually, but compiler hide this.}

\RU{Небольшой пример:}\EN{Small example:}

\lstinputlisting{patterns/06_return_results/6_1.c}

\dots \RU{получим}\EN{what we got} (MSVC 2010 \Ox):

\lstinputlisting{patterns/06_return_results/6_1.asm}

\RU{Имя внутреннего макроса для передачи указателя на структуру здесь это \TT{\$T3853}.}
\EN{Macro name for internal variable passing pointer to structure is \TT{\$T3853} here.}

\index{\CLanguageElements!C99}
\RU{Этот пример можно даже переписать используя расширения C99}\EN{This example can be rewritten using
C99 language extensions}:

\lstinputlisting{patterns/06_return_results/6_1_C99.c}

\lstinputlisting[caption=GCC 4.8.1]{patterns/06_return_results/6_1_C99.asm}

\RU{Как видно, ф-ция просто заполняет поля в структуре, выделенной вызывающей ф-цией. 
Как если бы передавался просто указатель на структуру.
Так что никаких проблем с эффективностью нет.}
\EN{As we may see, the function is just filling fields in the structure, allocated by
caller function. 
As if a pointer to the structure was passed.
So there are no performance drawbacks.}

\ifx\LITE\undefined
\chapter{\RU{Указатели}\EN{Pointers}}
\index{\CLanguageElements!\Pointers}
\label{label_pointers}

\RU{Указатели также часто используются для возврата значений из функции (вспомните случай
со \scanf{}~(\myref{label_scanf})).}
\EN{Pointers are often used to return values from functions (recall \scanf case~(\myref{label_scanf})).}
\RU{Например, когда функции нужно вернуть сразу два значения.}
\EN{For example, when a function needs to return two values.}

\section{\RU{Пример с глобальными переменными}\EN{Global variables example}}

\lstinputlisting{patterns/061_pointers/global.c}

\RU{Это компилируется в}\EN{This compiles to}:

\lstinputlisting[caption=\Optimizing MSVC 2010 (/Ob0)]{patterns/061_pointers/global.asm}

\index{\olly}
\clearpage
\RU{Посмотрим это в}\EN{Let's see this in} \olly:

\begin{figure}[H]
\centering
\includegraphics[scale=\FigScale]{patterns/061_pointers/olly_global1.png}
\caption{\olly: \RU{передаются адреса двух глобальных переменных в}
\EN{global variables addresses are passed to} \ttfone}
\label{fig:pointers_olly_global_1}
\end{figure}

\RU{В начале адреса обоих глобальных переменных передаются в}\EN{First, global
variables' addresses are passed to} \ttfone.
\RU{Можно нажать}\EN{We can click} \q{Follow in dump} 
\RU{на элементе стека и в окне слева 
увидим место в сегменте данных, выделенное для двух переменных.}
\EN{on the stack element, and we can see the place in the data segment allocated 
for the two variables.}
\RU{Эти переменные обнулены, потому что по стандарту неинициализированные данные (\ac{BSS}) 
обнуляются перед началом исполнения: \cite[6.7.8p10]{C99TC3}.}

\clearpage
\EN{These variables are zeroed, because non-initialized data (from \ac{BSS}) is cleared before
the execution begins: \cite[6.7.8p10]{C99TC3}.}
\RU{И они находятся в сегменте данных, о чем можно удостовериться, нажав}
\EN{They reside in the data segment, we can verify this by pressing} Alt-M \RU{и увидев карту
памяти}\EN{and reviewing the memory map}:

\begin{figure}[H]
\centering
\includegraphics[scale=\FigScale]{patterns/061_pointers/olly_global5.png}
\caption{\olly: \RU{карта памяти}\EN{memory map}}
\label{fig:pointers_olly_global_5}
\end{figure}

\clearpage
\RU{Трассируем}\EN{Let's trace} (F7) \RU{до начала исполнения}\EN{to the start of} \ttfone: 

\begin{figure}[H]
\centering
\includegraphics[scale=\FigScale]{patterns/061_pointers/olly_global2.png}
\caption{\olly: \RU{начало работы \ttfone}\EN{\ttfone starts}}
\label{fig:pointers_olly_global_2}
\end{figure}

\RU{В стеке видны значения}\EN{Two values are visible in the stack} 456 (\TT{0x1C8}) \AndENRU 
123 (\TT{0x7B}), \RU{а также адреса двух глобальных переменных}\EN{and also the addresses of the two global variables}.

\clearpage
\RU{Трассируем до конца}\EN{Let's trace until the end of} \ttfone.
\RU{Мы видим в окне слева, как результаты вычисления появились в глобальных переменных}%
\EN{In the left bottom window we see how the results of the calculation appear in the global variables}: 

\begin{figure}[H]
\centering
\includegraphics[scale=\FigScale]{patterns/061_pointers/olly_global3.png}
\caption{\olly: \ttfone \RU{заканчивает работу}\EN{execution completed}}
\label{fig:pointers_olly_global_3}
\end{figure}

\clearpage
\RU{Теперь из глобальных переменных значения загружаются в регистры для передачи в}
\EN{Now the global variables' values are loaded into registers ready for passing to} \printf \EN{(via the stack)}:

\begin{figure}[H]
\centering
\includegraphics[scale=\FigScale]{patterns/061_pointers/olly_global4.png}
\caption{\olly: \RU{адреса глобальных переменных передаются в}
\EN{global variables' addresses are passed into} \printf}
\label{fig:pointers_olly_global_4}
\end{figure}

\section{\RU{Пример с локальными переменными}\EN{Local variables example}}

\RU{Немного переделаем пример}\EN{Let's rework our example slightly}:

\lstinputlisting[caption=\RU{теперь переменные локальные}
\EN{now the \TT{sum} and \TT{product} variables are local}]{patterns/061_pointers/local.c.\LANG}

\RU{Код функции }\ttfone \RU{не изменится}\EN{code will not change}.
\RU{Изменится только \main}\EN{Only the code of \main will do}:

\lstinputlisting[caption=\Optimizing MSVC 2010 (/Ob0)]{patterns/061_pointers/local.asm}

\newcommand{\PtrsAddresses}{\TT{0x2EF854} \AndENRU \TT{0x2EF858}\xspace}

\clearpage
\RU{Снова посмотрим в}\EN{Let's look again with} \olly.
\RU{Адреса локальных переменных в стеке это}\EN{The addresses of the local variables in the stack are} \PtrsAddresses.
\RU{Видно, как они заталкиваются в стек}\EN{We see how these are pushed into the stack}: 

\begin{figure}[H]
\centering
\includegraphics[scale=\FigScale]{patterns/061_pointers/olly_stk1.png}
\caption{\olly: \RU{адреса локальных переменных заталкиваются в стек}\EN{local variables' addresses are
pushed into the stack}}
\label{fig:pointers_olly_stk_1}
\end{figure}

\clearpage
\RU{Начало работы \ttfone}\EN{\ttfone starts}.
\RU{В стеке по адресам}\EN{So far there is only random garbage in the stack at} \PtrsAddresses \RU{пока находится случайный мусор}:

\begin{figure}[H]
\centering
\includegraphics[scale=\FigScale]{patterns/061_pointers/olly_stk2.png}
\caption{\olly: \ttfone \RU{начинает работу}\EN{starting}}
\label{fig:pointers_olly_stk_2}
\end{figure}

\clearpage
\RU{Конец работы \ttfone}\EN{\ttfone completes}:

\begin{figure}[H]
\centering
\includegraphics[scale=\FigScale]{patterns/061_pointers/olly_stk3.png}
\caption{\olly: \ttfone \RU{заканчивает работу}\EN{completes execution}}
\label{fig:pointers_olly_stk_3}
\end{figure}

\RU{В стеке по адресам \PtrsAddresses теперь находятся значения \TT{0xDB18} \AndENRU \TT{0x243}, 
это результаты работы \ttfone.}
\EN{We now find \TT{0xDB18} \AndENRU \TT{0x243} at addresses \PtrsAddresses. These values are
the \ttfone results.}

\section{\Conclusion{}}

\RU{\ttfone может одинаково хорошо возвращать результаты работы в любые места памяти.} 
\EN{\ttfone could return pointers to any place in memory, located anywhere.}
\RU{В этом суть и удобство указателей.}
\EN{This is in essence the usefulness of the pointers.}

\RU{Кстати,}\EN{By the way, \Cpp} \IT{references} \RU{в \Cpp работают точно так же}\EN{work exactly the
same way}. \RU{Читайте больше об этом}\EN{Read more about them}: (\myref{cpp_references}).

\fi
\chapter{GOTO}

\RU{Оператор GOTO считается анти-паттерном}\EN{GOTO operator considered harmful} 
\cite{Dijkstra:1968:LEG:362929.362947}, 
\RU{но тем не менее, его можно использовать в разумных пределах}
\EN{but nevertheless, can be used resonably} \cite{Knuth:1974:SPG:356635.356640}, \cite[1.3.2]{CBook}.

\RU{Вот простейший пример}\EN{Here is a simplest possible example}:

\lstinputlisting{patterns/065_GOTO/goto.c}

\RU{Вот что мы получаем в}\EN{Here is what we've got is} MSVC 2012:

\lstinputlisting[caption=MSVC 2012]{patterns/065_GOTO/MSVC_goto.asm}

\RU{Так что выражение \IT{goto} просто заменяется инструкцией \JMP, которая работает точно также:
безусловный переход в другое место.}
\EN{So the \IT{goto} statement is just replaced by \JMP instruction, which has the very same
effect: unconditional jump to another place.}

\RU{Вызов второго \printf может исполнится только при помощи человеческого вмешательства,
используя отладчик или модифицирование кода.}
\EN{The second \printf call can be executed only with the help of human intervention, 
using debugger or patching.}\\
\\
\ifdefined\IncludeHiew
\clearpage
\RU{Это также может быть простым упражнением на модификацию кода.}
\EN{This also could be a simple patching exercise.}
\RU{Откроем исполняемый файл в}\EN{Let's open resulting executable in} Hiew:

\begin{figure}[H]
\centering
\includegraphics[scale=\NormalScale]{patterns/065_GOTO/hiew1.png}
\caption{Hiew}
\label{fig:goto_hiew1}
\end{figure}

\clearpage
\RU{Поместите курсор по адресу где}\EN{Place cursor to the address of} \JMP (\TT{0x410}), 
\RU{нажмите}\EN{press} F3 (\RU{редактирование}\EN{edit}), \RU{нажмите два нуля, так что
опкод будет}\EN{press two zeroes, so the opcode will be} \TT{EB 00}:

\begin{figure}[H]
\centering
\includegraphics[scale=\NormalScale]{patterns/065_GOTO/hiew2.png}
\caption{Hiew}
\label{fig:goto_hiew2}
\end{figure}

\RU{Второй байт опкода \JMP означает относительное смещение от перехода, 0 означает место
прямо после текущей инструкции.}
\EN{The second byte of \JMP opcode mean relative offset of jump, 0 means the point
right after current instruction.}
\RU{Теперь \JMP не будет пропускать следующий вызов \printf.}
\EN{So now \JMP will not skip second \printf call.}

\RU{Теперь нажмите F9 (запись) и выйдите.}
\EN{Now press F9 (save) and exit.}
\RU{Теперь мы запускаем исполняемый файл и видим это}\EN{Now we run executable and we see 
this}:

\begin{figure}[H]
\centering
\includegraphics[scale=\NormalScale]{patterns/065_GOTO/result.png}
\caption{\RU{Результат}\EN{Result}}
\label{fig:goto_result}
\end{figure}

\RU{Подобного же эффекта можно достичь, если заменить инструкцию \JMP на две инструкции \NOP.}
\EN{The same effect can be achieved if to replace \JMP instruction by 2 \NOP instructions.}
\RU{\NOP имеет опкод \TT{0x90} и длину в 1 байт, так что нужно 2 инструкции для замены.}
\EN{\NOP has \TT{0x90} opcode and length of 1 byte, so we need 2 instructions as replacement.}

\fi

\section{\RU{Мертвый код}\EN{Dead code}}

\RU{Вызов второго \printf также называется ``мертвым кодом'' (``dead code'') 
в терминах компиляторов.}
\EN{The second \printf call is also called ``dead code'' in compiler's term.}
\RU{Это значит что он никогда не будет исполнен}\EN{This mean, the code will never be executed}.
\EN{So when you compile this example with optimization, compiler removing ``dead code'' leaving
no trace of it:}
\RU{Так что если вы компилируете этот пример с оптимизацией, компилятор удаляет ``мертвый
код'' не оставляя следа:}

\lstinputlisting[caption=\Optimizing MSVC 2012]{patterns/065_GOTO/MSVC_goto_Ox.asm}

\RU{Впрочем, строку}\EN{However, compiler forgot to remove the} ``skip me!'' \RU{компилятор 
убрать забыл}\EN{string}.

\ifdefined\IncludeExercises
\section{\Exercise}

\RU{Попробуйте добиться того же самого используя ваш любимый компилятор и отладчик.}
\EN{Try to achieve the same result using your favorite compiler and debugger.}
\fi

\EN{\chapterold{Conditional jumps}
\label{sec:Jcc}
\myindex{\CLanguageElements!if}

% sections
\input{patterns/07_jcc/simple/main}
\input{patterns/07_jcc/abs/main}
\input{patterns/07_jcc/cond_operator/main}
\input{patterns/07_jcc/minmax/main}

\sectionold{\Conclusion{}}

\subsectionold{x86}

Here's the rough skeleton of a conditional jump:

\begin{lstlisting}[caption=x86]
CMP register, register/value
Jcc true ; cc=condition code
false:
... some code to be executed if comparison result is false ...
JMP exit 
true:
... some code to be executed if comparison result is true ...
exit:
\end{lstlisting}

\subsectionold{ARM}

\begin{lstlisting}[caption=ARM]
CMP register, register/value
Bcc true ; cc=condition code
false:
... some code to be executed if comparison result is false ...
JMP exit 
true:
... some code to be executed if comparison result is true ...
exit:
\end{lstlisting}

\subsectionold{MIPS}

\begin{lstlisting}[caption=Check for zero]
BEQZ REG, label
...
\end{lstlisting}

\begin{lstlisting}[caption=Check for less than zero:]
BLTZ REG, label
...
\end{lstlisting}

\begin{lstlisting}[caption=Check for equal values]
BEQ REG1, REG2, label
...
\end{lstlisting}

\begin{lstlisting}[caption=Check for non-equal values]
BNE REG1, REG2, label
...
\end{lstlisting}

\begin{lstlisting}[caption=Check for less than{,} greater than (signed)]
SLT REG1, REG2, REG3
BEQ REG1, label
...
\end{lstlisting}

\begin{lstlisting}[caption=Check for less than{,} greater than (unsigned)]
SLTU REG1, REG2, REG3
BEQ REG1, label
...
\end{lstlisting}

\subsectionold{Branchless}

\myindex{ARM!\Instructions!MOVcc}
\myindex{x86!\Instructions!CMOVcc}
\myindex{ARM!\Instructions!CSEL}
If the body of a condition statement is very short, the conditional move instruction can be used: 
\INS{MOVcc} in ARM (in ARM mode), \INS{CSEL} in ARM64, \INS{CMOVcc} in x86.

\subsubsectionold{ARM}

It's possible to use conditional suffixes in ARM mode for some instructions:

\begin{lstlisting}[caption=ARM (\ARMMode)]
CMP register, register/value
instr1_cc ; some instruction will be executed if condition code is true
instr2_cc ; some other instruction will be executed if other condition code is true
... etc...
\end{lstlisting}

Of course, there is no limit for the number of instructions with conditional code suffixes, 
as long as the CPU flags are not modified by any of them.
% FIXME: list of such instructions or \myref{} to it

\myindex{ARM!\Instructions!IT}

Thumb mode has the \INS{IT} instruction, allowing to add conditional suffixes to the next four instructions.
Read more about it: \myref{ARM_Thumb_IT}.

\begin{lstlisting}[caption=ARM (\ThumbMode)]
CMP register, register/value
ITEEE EQ ; §set these suffixes§: if-then-else-else-else
instr1   ; §instruction will be executed if condition is true§
instr2   ; §instruction will be executed if condition is false§
instr3   ; §instruction will be executed if condition is false§
instr4   ; §instruction will be executed if condition is false§
\end{lstlisting}

\sectionold{\Exercise}

(ARM64) Try rewriting the code in \lstref{cond_ARM64} by removing all 
conditional jump instructions and using the \TT{CSEL} instruction.

}
\RU{\section{Условные переходы}
\label{sec:Jcc}
\myindex{\CLanguageElements!if}

% sections
\input{patterns/07_jcc/simple/main}
\input{patterns/07_jcc/abs/main}
\input{patterns/07_jcc/cond_operator/main}
\input{patterns/07_jcc/minmax/main}

\subsection{\Conclusion{}}

\subsubsection{x86}

Примерный скелет условных переходов:

\begin{lstlisting}[caption=x86,style=customasmx86]
CMP register, register/value
Jcc true ; §cc=код условия§
false:
... код, исполняющийся, если сравнение ложно ...
JMP exit 
true:
... код, исполняющийся, если сравнение истинно ...
exit:
\end{lstlisting}

\subsubsection{ARM}

\begin{lstlisting}[caption=ARM,style=customasmARM]
CMP register, register/value
Bcc true ; §cc=код условия§
false:
... код, исполняющийся, если сравнение ложно ...
JMP exit 
true:
... код, исполняющийся, если сравнение истинно ...
exit:
\end{lstlisting}

\subsubsection{MIPS}

\begin{lstlisting}[caption=Проверка на ноль,style=customasmMIPS]
BEQZ REG, label
...
\end{lstlisting}

\begin{lstlisting}[caption=Меньше ли нуля? (используя псевдоинструкцию),style=customasmMIPS]
BLTZ REG, label
...
\end{lstlisting}

\begin{lstlisting}[caption=Проверка на равенство,style=customasmMIPS]
BEQ REG1, REG2, label
...
\end{lstlisting}

\begin{lstlisting}[caption=Проверка на неравенство,style=customasmMIPS]
BNE REG1, REG2, label
...
\end{lstlisting}

\begin{lstlisting}[caption=Проверка на меньше (знаковое),style=customasmMIPS]
SLT REG1, REG2, REG3
BEQ REG1, label
...
\end{lstlisting}

\begin{lstlisting}[caption=Проверка на меньше (беззнаковое),style=customasmMIPS]
SLTU REG1, REG2, REG3
BEQ REG1, label
...
\end{lstlisting}

\subsubsection{Без инструкций перехода}

\myindex{ARM!\Instructions!MOVcc}
\myindex{x86!\Instructions!CMOVcc}
\myindex{ARM!\Instructions!CSEL}

Если тело условного выражения очень короткое, может быть
использована инструкция условного копирования: \INS{MOVcc} в ARM (в режиме ARM), \INS{CSEL} в ARM64, \INS{CMOVcc} в x86.

\myparagraph{ARM}

В режиме ARM можно использовать условные суффиксы для некоторых инструкций:

\begin{lstlisting}[caption=ARM (\ARMMode),style=customasmARM]
CMP register, register/value
instr1_cc ; инструкция, которая будет исполнена, если условие истинно
instr2_cc ; еще инструкция, которая будет исполнена, если условие истинно
... и т.д....
\end{lstlisting}

Нет никаких ограничений на количество инструкций с условными суффиксами до тех пор,
пока флаги CPU не были модифицированы одной из таких инструкций.

% FIXME: list of such instructions or \myref{} to it

\myindex{ARM!\Instructions!IT}
В режиме Thumb есть инструкция \INS{IT}, позволяющая дополнить следующие 4 инструкции суффиксами, задающими
условие.

Читайте больше об этом: \myref{ARM_Thumb_IT}.

\begin{lstlisting}[caption=ARM (\ThumbMode),style=customasmARM]
CMP register, register/value
ITEEE EQ ; выставить такие суффиксы: if-then-else-else-else
instr1   ; инструкция будет исполнена, если истинно
instr2   ; инструкция будет исполнена, если ложно
instr3   ; инструкция будет исполнена, если ложно
instr4   ; инструкция будет исполнена, если ложно
\end{lstlisting}

\subsection{\Exercise}

(ARM64) Попробуйте переписать код в \lstref{cond_ARM64} 
убрав все инструкции условного перехода, и используйте инструкцию \INS{CSEL}.

}
\ITA{\mysection{Jump condizionali}
\label{sec:Jcc}
\myindex{\CLanguageElements!if}

% sections
\input{patterns/07_jcc/simple/main}
\input{patterns/07_jcc/abs/main}
\input{patterns/07_jcc/cond_operator/main}
\input{patterns/07_jcc/minmax/main}

% Do not translate, this is macro:
\subsection{\Conclusion{}}

\subsubsection{x86}

La forma grezza di un jump condizionale e' la seguente:

\begin{lstlisting}[caption=x86,style=customasmx86]
CMP register, register/value
Jcc true ; cc=condition code
false:
... codice da eseguire se il risultato del confronto e' false ...
JMP exit 
true:
... codice da eseguire se il risultato del confronto e' true ...
exit:
\end{lstlisting}

\subsubsection{ARM}

\begin{lstlisting}[caption=ARM,style=customasmARM]
CMP register, register/value
Bcc true ; cc=condition code
false:
... codice da eseguire se il risultato del confronto e' false ...
JMP exit 
true:
... codice da eseguire se il risultato del confronto e' true ...
exit:
\end{lstlisting}

\subsubsection{MIPS}

\begin{lstlisting}[caption=Check for zero,style=customasmMIPS]
BEQZ REG, label
...
\end{lstlisting}

\begin{lstlisting}[caption=Check for less than zero (using pseudoinstruction),style=customasmMIPS]
BLTZ REG, label
...
\end{lstlisting}

\begin{lstlisting}[caption=Check for equal values,style=customasmMIPS]
BEQ REG1, REG2, label
...
\end{lstlisting}

\begin{lstlisting}[caption=Check for non-equal values,style=customasmMIPS]
BNE REG1, REG2, label
...
\end{lstlisting}

\begin{lstlisting}[caption=Check for less than (signed),style=customasmMIPS]
SLT REG1, REG2, REG3
BEQ REG1, label
...
\end{lstlisting}

\begin{lstlisting}[caption=Check for less than (unsigned),style=customasmMIPS]
SLTU REG1, REG2, REG3
BEQ REG1, label
...
\end{lstlisting}

\subsubsection{Branchless}

\myindex{ARM!\Instructions!MOVcc}
\myindex{x86!\Instructions!CMOVcc}
\myindex{ARM!\Instructions!CSEL}
Se il corpo di uno statement condizionale e' molto piccolo, puo' essere utilizzata l'istruzione "move" condizionale: 
\INS{MOVcc} in ARM (in ARM mode), \INS{CSEL} in ARM64, \INS{CMOVcc} in x86.

\myparagraph{ARM}

In ARM e' possibile usare suffissi condizionali per alcune istruzioni:

\begin{lstlisting}[caption=ARM (\ARMMode),style=customasmARM]
CMP register, register/value
instr1_cc ; istruzione che sara' eseguita se il condition code e' true
instr2_cc ; altra istruzione che sara' eseguita se il condition code e' true
... etc...
\end{lstlisting}

Ovviamente non c'e' limite al numero di istruzioni con il suffisso condizionale, a patto che le flag CPU non siano modificate da nessuna istruzione. 
% FIXME: list of such instructions or \myref{} to it

\myindex{ARM!\Instructions!IT}

La modalita' Thumb ha l'istruzione \INS{IT}, che permette di aggiungere suffissi condizionali alle prossime quattro istruzioni.
Maggiori informazioni qui: \myref{ARM_Thumb_IT}.

\begin{lstlisting}[caption=ARM (\ThumbMode),style=customasmARM]
CMP register, register/value
ITEEE EQ ; set these suffixes: if-then-else-else-else
instr1   ; istruzione da eseguire se la condizione e' true
instr2   ; istruzione da eseguire se la condizione e' false
instr3   ; istruzione da eseguire se la condizione e' false
instr4   ; istruzione da eseguire se la condizione e' false
\end{lstlisting}

% Do not translate, this is macro:
\subsection{\Exercise}

(ARM64) Prova a riscrivere il codice in \lstref{cond_ARM64} rimuovendo tutti i jump condizionali e usando al loro posto l'istruzione \TT{CSEL} instruction.
}

\chapter{\SwitchCaseDefaultSectionName}
\index{\CLanguageElements!switch}

% sections
\section{\RU{Если вариантов мало}\EN{Few number of cases}}

\lstinputlisting{patterns/08_switch/1_few/few.c}

\input{patterns/08_switch/1_few/x86}
\input{patterns/08_switch/1_few/ARM}

\EN{\input{patterns/08_switch/2_lot/main_EN}}
\RU{\input{patterns/08_switch/2_lot/main_RU}}
\DE{\input{patterns/08_switch/2_lot/main_DE}}
\FR{\input{patterns/08_switch/2_lot/main_FR}}


\section{\RU{Когда много \IT{case} в одном блоке}
\EN{When there are several \IT{case} in one block}}

\RU{Вот очень часто используемая конструкция: несколько \IT{case} может быть использовано в одном блоке:}
\EN{Here is also a very often used construction: several \IT{case} statements may be used in single block:}

\lstinputlisting{patterns/08_switch/3_several_cases/several_cases.c}

\RU{Слишком расточительно генерировать каждый блок для каждого случая, поэтому обычно
каждый блок генерируется плюс диспетчер.}
\EN{It's too wasteful to generate each block for each possible case,
so what is usually done, is each block generated plus some kind of dispatcher.}

\subsection{MSVC}

\lstinputlisting[caption=\Optimizing MSVC 2010,numbers=left]{patterns/08_switch/3_several_cases/several_cases_MSVC_2010_Ox.asm}

\RU{Здесь видим две таблицы}\EN{We see two tables here}: 
\RU{первая таблица}\EN{the first table} (\TT{\$LN10@f}) \RU{это таблица индексов}\EN{is index table},
\RU{и вторая таблица}\EN{and the second table} (\TT{\$LN11@f}) \RU{это массив указателей на блоки}\EN{is 
an array of pointers to blocks}.

\RU{В начале, входное значение используется как индекс в таблице индексов}\EN{First, input value 
is used as index in index table} (\LineENRU 13). 

\RU{Вот краткое описание значений в таблице}\EN{Here is short legend for values in the table}: 
0 \RU{это первый блок \IT{case}}\EN{is first \IT{case} block} (\RU{для значений}\EN{for values} 1, 2, 7, 10),
1 \RU{это второй}\EN{is second} (\RU{для значений}\EN{for values} 3, 4, 5),
2 \RU{это третий}\EN{is third} (\RU{для значений}\EN{for values} 8, 9, 21),
3 \RU{это четвертый}\EN{is fourth} (\RU{для значений}\EN{for value} 22),
4 \RU{это для default-блока}\EN{is for default block}.

\RU{Мы получаем индекс для второй таблицы указателей на блоки и переходим туда}\EN{We get there index for 
the second table of block pointers and we we jump there} (\LineENRU 14).

\EN{What is also worth to note that there are no case for input value $0$.}
\RU{Что еще нужно отметить, так это то что здесь нет случая для нулевого входного значения.}
\EN{Hence, we see \DEC instruction at line 10, and the table is beginning at $a=1$. 
Because there are no need to allocate table element for $a=0$.}
\RU{Поэтому мы видим инструкцию \DEC на строке 10 и таблица начинается с $a=1$.
Потому что незачем выделять в таблице элемент для $a=0$.}

\RU{Это очень часто используемый шаблон}\EN{This is very often used pattern}.

\RU{В чем же экономия}\EN{So where economy is}?
\RU{Почему нельзя сделать так, как уже обсуждалось}\EN{Why it's not possible to make it as it was 
already discussed} (\ref{switch_lot_GCC}), \RU{используя только одну таблицу, содержащую указатели на 
блоки}\EN{just with one table, consisting of block pointers}?
\RU{Причина в том что элементы в таблице индексов занимают только по 8-битному байту, поэтому всё это более 
компактно}\EN{The reason is because elements in index table has 8-bit byte type, hence it's all more compact}.

\subsection{GCC}

GCC \RU{делает так, как уже обсуждалось}\EN{do the job like it was already discussed} 
(\ref{switch_lot_GCC}), \RU{используя просто таблицу указателей}\EN{using just one table of pointers}.

\section{Fall-through}

\RU{Ещё одно популярное использование оператора}\EN{Another very popular usage of} \TT{switch()} 
\EN{is the fall-through}\RU{это т.н. \q{fallthrough} (\q{провал})}.
\RU{Вот простой пример}\EN{Here is a small example}:

\lstinputlisting[numbers=left]{patterns/08_switch/4_fallthrough/fallthrough.c}

\RU{Если}\EN{If} $type=1$ (R), $read$ \RU{будет выставлен в}\EN{is to be set to} 1, \RU{если}\EN{if} 
$type=2$ (W), $write$ \RU{будет выставлен в}\EN{is to be set to} 2.
\RU{В случае}\EN{In case of} $type=3$ (RW), \RU{обе}\EN{both} $read$ \AndENRU $write$ \RU{будут 
выставлены в}\EN{is to be set to} 1.

\RU{Фрагмент кода на строке 14 будет исполнен в двух случаях: если}\EN{The code at 
line 14 is executed in two cases: if} $type=RW$ \RU{или если}\EN{or if} $type=W$.
\RU{Там нет \q{break} для \q{case RW}, и это нормально}\EN{There is no \q{break} 
for \q{case RW}x and that's OK}.

\subsection{MSVC x86}

\lstinputlisting[caption=MSVC 2012]{patterns/08_switch/4_fallthrough/fallthrough_MSVC.asm}

\RU{Код почти полностью повторяет то, что в исходнике.}
\EN{The code mostly resembles what is in the source.}
\RU{Там нет переходов между метками}\EN{There are no jumps between labels} \TT{\$LN4@f} \AndENRU 
\TT{\$LN3@f}: \RU{так что когда управление (code flow) находится на}\EN{so when code flow is at} 
\TT{\$LN4@f}, $read$ \RU{в начале выставляется в 1, затем}\EN{is first set to 1, then} $write$.
\EN{This is why it's called fall-through: code flow falls through one piece of code
(setting $read$) to another (setting $write$).}
\RU{Наверное, поэтому всё это и называется \q{проваливаться}: управление проваливается через
один фрагмент кода (выставляющий $read$) в другой (выставляющий $write$).}
\RU{Если}\EN{If} $type=W$, \RU{мы оказываемся на}\EN{we land at} \TT{\$LN3@f}, 
\RU{так что код выставляющий $read$ в 1 не исполнится}\EN{so no code setting $read$ to 1 
is executed}.

\ifdefined\IncludeARM
\subsection{ARM64}

\lstinputlisting[caption=GCC (Linaro) 4.9]{patterns/08_switch/4_fallthrough/fallthrough_ARM64.s.\LANG}

\RU{Почти то же самое}\EN{Merely the same thing}.
\RU{Здесь нет переходов между метками}\EN{There are no jumps between labels} \TT{.L4} 
\AndENRU \TT{.L3}.
\fi


\ifdefined\IncludeExercises
\section{\Exercises}

\subsection{\Exercise \#1}
\label{exercise_switch_1}

\RU{Вполне возможно переделать пример на Си в листинге \ref{switch_lot_c} так, чтобы при компиляции
получалось даже еще меньше кода, но работать всё будет точно так же.}
\EN{It's possible to rework the C example in \ref{switch_lot_c} in such way that the compiler
will produce even smaller code, but will work just the same.}
\RU{Попробуйте этого добиться}\EN{Try to achieve it}.

\RU{Подсказка}\EN{Hint}: \ref{exercise_solutions_switch_1}.
\fi

\mysection{\Loops}
\label{sec:loops}

% sections
\section{\RU{Простой пример}\EN{Simple example}}

% subsections
\input{patterns/09_loops/simple/x86}
\ifdefined\IncludeARM
\input{patterns/09_loops/simple/ARM/main}
\fi
\ifdefined\IncludeMIPS
\input{patterns/09_loops/simple/MIPS}
\fi

\subsection{\RU{Ещё кое-что}\EN{One more thing}}

\RU{По генерируемому коду мы видим следующее}\EN{In the generated code we can see}: 
\RU{после инициализации}\EN{after initializing} $i$%
\RU{, тело цикла не исполняется. Исполняется сразу 
проверка условия $i$, а лишь затем исполняется тело цикла.}%
\EN{, the body of the loop is not to be executed,
as the condition for $i$ is checked first, and only after that loop body can be executed.}
\RU{Это правильно.}\EN{And that is correct.} 
\RU{Потому что если условие в самом начале не выполняется, тело цикла исполнять нельзя.}
\EN{Because, if the loop condition is
not met at the beginning, the body of the loop must not be executed.}
\RU{Так может быть, например, в таком случае:}\EN{This is possible in the following case:}

\lstinputlisting{patterns/09_loops/simple/loops_3.c.\LANG}

\RU{Если}\EN{If} \IT{total\_entries\_to\_process} \RU{равно}\EN{is} 0,
\RU{тело цикла не должно исполниться ни разу}\EN{the body of the loo must not be executed at all}.
\RU{Поэтому проверка условия происходит перед тем как исполнить само тело.}
\EN{This is why the condition checked before
the execution.}

\RU{Впрочем, оптимизирующий компилятор может переставить проверку условия и тело цикла местами, если он уверен,
что описанная здесь ситуация невозможна, как в случае с нашим простейшим примером и компиляторами 
Keil, Xcode (LLVM), MSVC и GCC в режиме оптимизации.}
\EN{However, an optimizing compiler may swap the condition check and loop body,
if it sure that the situation described here is
not possible (like in the case of our very simple example and Keil, Xcode (LLVM), MSVC in optimization mode).}

\section{\RU{Функция копирования блоков памяти}\EN{Memory blocks copying routine}}
\label{loop_memcpy}

\RU{Настоящие функции копирования памяти могут копировать по 4 или 8 байт на каждой итерации, использовать \ac{SIMD},
векторизацию, \etc{}.}
\EN{Real-world memory copy routines may copy 4 or 8 bytes at each iteration, use \ac{SIMD}, 
vectorization, \etc{}.}
\RU{Но ради простоты, этот пример настолько прост, насколько это возможно.}
\EN{But for the sake of simplicity, this example is the simplest possible.}

\lstinputlisting{memcpy.c}

\subsection{\RU{Простейшая реализация}\EN{Straight-forward implementation}}

\lstinputlisting[caption=GCC 4.9 x64 \RU{оптимизация по размеру}\EN{optimized for size} (-Os)]{patterns/09_loops/memcpy/memcpy_GCC49_x64_Os.s.\LANG}

\ifdefined\IncludeARM

\lstinputlisting[caption=GCC 4.9 ARM64 \RU{оптимизация по размеру}\EN{optimized for size} (-Os)]{patterns/09_loops/memcpy/memcpy_GCC49_ARM64_Os.s.\LANG}

\lstinputlisting[caption=\OptimizingKeilVI (\ThumbMode)]{patterns/09_loops/memcpy/memcpy_Keil_Thumb_O3.s.\LANG}

\subsection{ARM \RU{в режиме ARM}\EN{in ARM mode}}

\RU{Keil в режиме ARM пользуется условными суффиксами:}
\EN{Keil in ARM mode takes full advantage of conditional suffixes:}

\lstinputlisting[caption=\OptimizingKeilVI (\ARMMode)]{patterns/09_loops/memcpy/memcpy_Keil_ARM_O3.s.\LANG}

\RU{Вот почему здесь только одна инструкция перехода вместо двух.}
\EN{That's why there is only one branch instruction instead of 2.}

\fi

\ifdefined\IncludeMIPS
\subsection{MIPS}

\lstinputlisting[caption=GCC 4.4.5 \RU{оптимизация по размеру}\EN{optimized for size} (-Os) (IDA)]{patterns/09_loops/memcpy/memcpy_MIPS_Os_IDA.lst.\LANG}

\index{MIPS!\Instructions!LBU}
\index{MIPS!\Instructions!SB}
\RU{Здесь две новых для нас инструкций:}
\EN{Here we have two new instructions:} LBU (\q{Load Byte Unsigned}) \AndENRU SB (\q{Store Byte}).
\RU{Так же как и в ARM, все регистры в MIPS имеют длину в 32 бита. Здесь нет частей регистров равных байту,
как в x86.}
\EN{Just like in ARM, all MIPS registers are 32-bit wide, there are no byte-wide parts like in x86.}
\RU{Так что когда нужно работать с байтами, приходится выделять целый 32-битный регистр для этого.}
\EN{So when dealing with single bytes, we have to allocate whole 32-bit registers for them.}
\RU{LBU загружает байт и сбрасывает все остальные биты (\q{Unsigned}).}
\EN{LBU loads a byte and clears all other bits (\q{Unsigned}).}
\index{MIPS!\Instructions!LB}
\RU{И напротив, инструкция LB (\q{Load Byte}) расширяет байт до 32-битного значения учитывая знак.}
\EN{On the other hand, LB (\q{Load Byte}) instruction sign-extends the loaded byte to a 32-bit value.}
\RU{SB просто записывает байт из младших 8 бит регистра в память.}
\EN{SB just writes a byte from lowest 8 bits of register to memory.}

\fi

\ifx\LITE\undefined
\subsection{\RU{Векторизация}\EN{Vectorization}}

\Optimizing GCC \RU{может из этого примера сделать намного больше}\EN{can do much more on this example}: 
\myref{vec_memcpy}.
\fi

\EN{\input{patterns/09_loops/cond_check/main_EN}}
\RU{\input{patterns/09_loops/cond_check/main_RU}}

\EN{\input{patterns/09_loops/conclusion_EN}}
\RU{\input{patterns/09_loops/conclusion_RU}}
\DE{\input{patterns/09_loops/conclusion_DE}}
\FR{\input{patterns/09_loops/conclusion_FR}}


\section{\Exercises}

\subsection{\Exercise \#1}

\index{x86!\Instructions!LOOP}
\RU{Почему инструкция}\EN{Why} \LOOP \RU{больше не используется современными 
компиляторами}\EN{instruction is not used by modern compilers anymore}?

\subsection{\Exercise \#2}

\RU{Возьмите пример рассмотренный в этой секции}\EN{Take a loop example from this section} 
(\ref{loops_src}), 
\RU{скомпилируйте его в вашей любимой}\EN{compile it in your favorite} \ac{OS}
\RU{и компиляторе, и модифицируйте исполняемый файл так, чтобы цикл был в пределах}\EN{and compiler 
and modify (patch) executable file, so the loop range will be} [6..20].


\chapter{\SimpleStringsProcessings}
\index{\CStandardLibrary!strlen()}
\index{\CLanguageElements!while}

% sections
\section{strlen()}
\index{\CStandardLibrary!strlen()}

\RU{Ещё немного о циклах. Часто функция \TT{strlen()}\footnote{подсчет длины строки в Си} 
реализуется при помощи \TT{while()}.}
\EN{Let's talk about loops one more time. Often, the \TT{strlen()} 
function\footnote{counting the characters in a string in the C language} is implemented using a \TT{while()} 
statement.}
\RU{Например, вот как это сделано в стандартных библиотеках MSVC:}
\EN{Here is how it is done in the MSVC standard libraries:}

\lstinputlisting{patterns/10_strings/1_strlen/ex1.c}

% subsections
\input{patterns/10_strings/1_strlen/x86}
\ifdefined\IncludeARM
\input{patterns/10_strings/1_strlen/ARM/main}
\fi
\ifdefined\IncludeMIPS
\input{patterns/10_strings/1_strlen/MIPS}
\fi

\chapter{\RU{Файл сохранения состояния в игре Millenium}\EN{Millenium game save file}}
\label{Millenium_DOS_game}
\index{MS-DOS}

\RU{Игра}\EN{The} \q{Millenium Return to Earth} \RU{под DOS довольно древняя (1991), позволяющая
добывать ресурсы, строить корабли, снаряжать их на другие планеты,\etc{}.}
\EN{is an ancient DOS game (1991), that allows you to mine resources, build ships,
equip them on other planets, and so on}\footnote{\RU{Её можно скачать бесплатно}\EN{It can be downloaded for free}
\href{http://go.yurichev.com/17316}{\RU{здесь}\EN{here}}}.

\RU{Как и многие другие игры, она позволяет сохранять состояние игры в файл.}
\EN{Like many other games, it allows you to save all game state into a file.}

\RU{Посмотрим, сможем ли мы найти что-нибудь в нем}\EN{Let's see if we can find something in it}.

\clearpage
\RU{В игре есть шахта}\EN{So there is a mine in the game}.
\RU{Шахты на некоторых планетах работают быстрее, на некоторых других --- медленнее}\EN{Mines at some planets 
work faster, or slower on others}. 
\RU{Набор ресурсов также разный}\EN{The set of resources is also different}.

\RU{Здесь видно, какие ресурсы добыты в этот момент}\EN{Here we can see what resources are mined at the time}: 

\begin{figure}[H]
\centering
\includegraphics[scale=\FigScale]{ff/millenium/1.png}
\caption{\RU{Шахта: первое состояние}\EN{Mine: state 1}}
\label{fig:mill_1}
\end{figure}

\RU{Сохраним состояние игры}\EN{Let's save a game state}.
\RU{Это файл размером}\EN{This is a file of size} 9538 \RU{байт}\EN{bytes}.

\RU{Подождем несколько \q{дней} здесь в игре и теперь в шахте добыто больше ресурсов}%
\EN{Let's wait some \q{days} here in the game, and now we've got more resources from the mine}:

\begin{figure}[H]
\centering
\includegraphics[scale=\FigScale]{ff/millenium/2.png}
\caption{\RU{Шахта: второе состояние}\EN{Mine: state 2}}
\label{fig:mill_2}
\end{figure}

\RU{Снова сохраним состояние игры}\EN{Let's sav game state again}.

\RU{Теперь просто попробуем сравнить оба файла побайтово используя простую утилиту FC под DOS/Windows:}
\EN{Now let's try to just do binary comparison of the save files using the simple DOS/Windows FC utility:}

\lstinputlisting{ff/millenium/fc_result.txt}

\RU{Вывод здесь неполный, там было больше отличий, но мы обрежем результат до самого интересного.}%
\EN{The output is incomplete here, there are more differences, but we will cut result to show the most interesting.}

\RU{В первой версии у нас было 14 единиц водорода (hydrogen) и 102 --- кислорода (oxygen).}
\EN{In the first state, we have 14 \q{units} of hydrogen and 102 \q{units} of oxygen.}
\RU{Во второй версии у нас 22 и 155 единиц соответственно.}
\EN{We have 22 and 155 \q{units} respectively in the second state.}
\RU{Если эти значения сохраняются в файл, мы должны увидеть разницу}\EN{If these values are saved into 
the save file, we would see this in the difference}.
\RU{И она действительно есть}\EN{And indeed we do}. 
\RU{Там}\EN{There is} 0x0E (14) \RU{на позиции}\EN{at position} 0xBDA \RU{и это значение}\EN{and this value is} 
0x16 (22) \RU{в новой версии файла}\EN{in the new version of the file}.
\RU{Это, наверное, водород}\EN{This is probably hydrogen}.
\RU{Там также}\EN{There is} 0x66 (102) \RU{на позиции}\EN{at position} 0xBDC \RU{в старой версии и}\EN{in the old 
version and} 0x9B (155) \RU{в новой версии файла}\EN{in the new version of the file}. 
\RU{Это, наверное, кислород}\EN{This seems to be the oxygen}.

\RU{Обе версии файла доступны на сайте, для тех кто хочет их изучить (или поэкспериментировать)}%
\EN{Both files are available on the website for those who wants to inspect them (or experiment) more}: 
\href{http://go.yurichev.com/17212}{beginners.re}.

\clearpage
\RU{Новую версию файла откроем в Hiew и отметим значения, связанные с ресурсами, добытыми на шахте в игре}%
\EN{Here is the new version of file opened in Hiew, we marked the values related to the resources mined in the game}: 

\begin{figure}[H]
\centering
\includegraphics[scale=\FigScale]{ff/millenium/hiew3.png}
\caption{Hiew: \RU{первое состояние}\EN{state 1}}
\label{fig:mill_hiew3}
\end{figure}

\RU{Проверим каждое, и это они}\EN{Let's check each, and these are}.
\RU{Это явно 16-битные значения: не удивительно для 16-битной программы под DOS, где \Tint имел длину в 16 бит.}
\EN{These are clearly 16-bit values: not a strange thing for 16-bit DOS software where the \Tint type has 16-bit width.}

\clearpage
\RU{Проверим наши предположения}\EN{Let's check our assumptions}.
\RU{Запишем 1234 (0x4D2) на первой позиции (это должен быть водород)}%
\EN{We will write the 1234 (0x4D2) value at the first position (this must be hydrogen)}:

\begin{figure}[H]
\centering
\includegraphics[scale=\FigScale]{ff/millenium/hiew4.png}
\caption{Hiew: \RU{запишем там}\EN{let's write 1234} (0x4D2)\EN{ there}}
\label{fig:mill_hiew4}
\end{figure}

\RU{Затем загрузим измененный файл в игру и посмотрим на статистику в шахте}%
\EN{Then we will load the changed file in the game and took a look at mine statistics}:

\begin{figure}[H]
\centering
\includegraphics[scale=\FigScale]{ff/millenium/5.png}
\caption{\RU{Проверим значение водорода}\EN{Let's check for hydrogen value}}
\label{fig:mill_5}
\end{figure}

\RU{Так что да, это оно}\EN{So yes, this is it}.

\clearpage
\RU{Попробуем пройти игру как можно быстрее, установим максимальные значения везде}\EN{Now let's try to 
finish the game as soon as possible, set the maximal values everywhere}:

\begin{figure}[H]
\centering
\includegraphics[scale=\FigScale]{ff/millenium/hiew7.png}
\caption{Hiew: \RU{установим максимальные значения}\EN{let's set maximal values}}
\label{fig:mill_hiew7}
\end{figure}

0xFFFF \RU{это}\EN{is} 65535, \RU{так что да, у нас много ресурсов теперь}\EN{so yes, we now have a 
lot of resources}:

\begin{figure}[H]
\centering
\includegraphics[scale=\FigScale]{ff/millenium/6.png}
\caption{\RU{Все ресурсы теперь действительно}\EN{All resources are} 65535 (0xFFFF)\EN{ indeed}}
\label{fig:mill_6}
\end{figure}

\clearpage
\RU{Пропустим еще несколько \q{дней} в игре и видим что-то неладное}\EN{Let's skip some \q{days} in the game and oops}! 
\RU{Некоторых ресурсов стало меньше}\EN{We have a lower amount of some resources}:

\begin{figure}[H]
\centering
\includegraphics[scale=\FigScale]{ff/millenium/8.png}
\caption{\RU{Переполнение переменных ресурсов}\EN{Resource variables overflow}}
\label{fig:mill_8}
\end{figure}

\RU{Это просто переполнение}\EN{That's just overflow}. 
\RU{Разработчик игры вероятно никогда не думал, что значения ресурсов будут такими большими,
так что, здесь, наверное, нет проверок на переполнение, но шахта в игре \q{работает}, ресурсы добавляются,
отсюда и переполнение.}
\EN{The game's developer probably didn't think about such high amounts of resources,
so there are probably no overflow checks, but the mine is \q{working} in the game, resources are added,
hence the overflows.}
\RU{Вероятно, не нужно было жадничать}\EN{Apparently, it was a bad idea to be that greedy}.

\RU{Здесь наверняка еще какие-то значения в этом файле}\EN{There are probably a lot of more values 
saved in this file}.

\RU{Так что это очень простой способ читинга в играх}\EN{So this is very simple method of cheating in games}.
\RU{Файл с таблицей очков также можно легко модифицировать}\EN{High score files often can be easily 
patched like that}.

\EN{More about files and memory snapshots comparing}\RU{Еще насчет сравнения файлов и снимков памяти}: 
\myref{snapshots_comparing}.

\section{\Exercises}

\subsection{\Exercise \#1}
\label{exercise_strlen_1}

\WhatThisCodeDoes\

\begin{lstlisting}[caption=\Optimizing MSVC 2010]
_s$ = 8			
_f	PROC
	mov	edx, DWORD PTR _s$[esp-4]
	mov	cl, BYTE PTR [edx]
	xor	eax, eax
	test	cl, cl
	je	SHORT $LN2@f
	npad	4 ; align next label
$LL4@f:
	cmp	cl, 32	
	jne	SHORT $LN3@f
	inc	eax
$LN3@f:
	mov	cl, BYTE PTR [edx+1]
	inc	edx
	test	cl, cl
	jne	SHORT $LL4@f
$LN2@f:
	ret	0
_f	ENDP
\end{lstlisting}

\begin{lstlisting}[caption=GCC 4.8.1 -O3]
f:
.LFB24:
	push	ebx
	mov	ecx, DWORD PTR [esp+8]
	xor	eax, eax
	movzx	edx, BYTE PTR [ecx]
	test	dl, dl
	je	.L2
.L3:
	cmp	dl, 32
	lea	ebx, [eax+1]
	cmove	eax, ebx
	add	ecx, 1
	movzx	edx, BYTE PTR [ecx]
	test	dl, dl
	jne	.L3
.L2:
	pop	ebx
	ret
\end{lstlisting}

\begin{lstlisting}[caption=\OptimizingKeilVI (\ARMMode)]
f PROC
        MOV      r1,#0
|L0.4|
        LDRB     r2,[r0,#0]
        CMP      r2,#0
        MOVEQ    r0,r1
        BXEQ     lr
        CMP      r2,#0x20
        ADDEQ    r1,r1,#1
        ADD      r0,r0,#1
        B        |L0.4|
        ENDP
\end{lstlisting}

\begin{lstlisting}[caption=\OptimizingKeilVI (\ThumbMode)]
f PROC
        MOVS     r1,#0
        B        |L0.12|
|L0.4|
        CMP      r2,#0x20
        BNE      |L0.10|
        ADDS     r1,r1,#1
|L0.10|
        ADDS     r0,r0,#1
|L0.12|
        LDRB     r2,[r0,#0]
        CMP      r2,#0
        BNE      |L0.4|
        MOVS     r0,r1
        BX       lr
        ENDP
\end{lstlisting}

\begin{lstlisting}[caption=\Optimizing GCC 4.9 (ARM64)]
f:
	ldrb	w1, [x0]
	cbz	w1, .L4
	mov	w2, 0
.L3:
	cmp	w1, 32
	ldrb	w1, [x0,1]!
	csinc	w2, w2, w2, ne
	cbnz	w1, .L3
.L2:
	mov	w0, w2
	ret
.L4:
	mov	w2, w1
	b	.L2
\end{lstlisting}

\lstinputlisting[caption=\Optimizing GCC 4.4.5 (MIPS) (IDA)]{patterns/10_strings/ex_MIPS_O3_IDA.lst}

\Answer\: \myref{exercise_solutions_strlen_1}.


\section{\ArithOptimizations}

\ifdefined\ENGLISH
In the pursuit of optimization, one instruction may be replaced by another, or even with a group of instructions.
For example, \ADD and \SUB can replace each other:
line 18 in \lstref{neg_array_c}.

For example, the \LEA instruction is often used for simple arithmetic calculations: \myref{sec:LEA}.
\fi % ENGLISH

\ifdefined\RUSSIAN
В целях оптимизации одна инструкция может быть заменена другой, или даже группой инструкций.
Например, \ADD и \SUB могут заменять друг друга:
строка 18 в \lstref{neg_array_c}.

Более того, не всегда замена тривиальна. Инструкция \LEA, несмотря на оригинальное назначение, нередко применяется для простых арифметических действий: \myref{sec:LEA}.
\fi % RUSSIAN

\ifdefined\GERMAN

\DEph{}

\fi % GERMAN

% sections
\EN{\section{Multiplication}

\subsection{Multiplication using addition}

Here is a simple example:

\begin{lstlisting}[caption=\Optimizing MSVC 2010]
unsigned int f(unsigned int a)
{
	return a*8;
};
\end{lstlisting}

Multiplication by 8 is replaced by 3 addition instructions, which do the same.
Apparently, MSVC's optimizer decided that this code can be faster.

\begin{lstlisting}
_TEXT	SEGMENT
_a$ = 8							; size = 4
_f	PROC
; File c:\polygon\c\2.c
	mov	eax, DWORD PTR _a$[esp-4]
	add	eax, eax
	add	eax, eax
	add	eax, eax
	ret	0
_f	ENDP
_TEXT	ENDS
END
\end{lstlisting}

\subsection{Multiplication using shifting}
\label{subsec:mult_using_shifts}

Multiplication and division instructions by a numbers that's a power of 2 are often replaced by shift instructions.

\begin{lstlisting}
unsigned int f(unsigned int a)
{
	return a*4;
};
\end{lstlisting}

\begin{lstlisting}[caption=\NonOptimizing MSVC 2010]
_a$ = 8		; size = 4
_f	PROC
	push	ebp
	mov	ebp, esp
	mov	eax, DWORD PTR _a$[ebp]
	shl	eax, 2
	pop	ebp
	ret	0
_f	ENDP
\end{lstlisting}


Multiplication by 4 is just shifting the number to the left by 2 bits
and inserting 2 zero bits at the right (as the last two bits).
It is just like multiplying 3 by 100~---we need to just add two zeroes at the right.

That's how the shift left instruction works:

\myindex{x86!\Instructions!SHL}
\input{shift_left}

The added bits at right are always zeroes.

\ifdefined\IncludeARM
Multiplication by 4 in ARM:

\begin{lstlisting}[caption=\NonOptimizingKeilVI (\ARMMode)]
f PROC
        LSL      r0,r0,#2
        BX       lr
        ENDP
\end{lstlisting}
\fi

\ifdefined\IncludeMIPS
Multiplication by 4 in MIPS:

\lstinputlisting[caption=\Optimizing GCC 4.4.5 (IDA)]{patterns/11_arith_optimizations/MIPS_SLL.lst}

\myindex{MIPS!\Instructions!SLL}
\INS{SLL} is \q{Shift Left Logical}.
\fi

\subsection{Multiplication using shifting, subtracting, and adding}
\label{multiplication_using_shifts_adds_subs}

It's still possible to get rid of the multiplication operation when you multiply by numbers like
7 or 17 again by using shifting.
The mathematics used here is relatively easy.

\subsubsection{32-bit}

\lstinputlisting{patterns/11_arith_optimizations/mult_shifts.c}

\myparagraph{x86}

\lstinputlisting[caption=\Optimizing MSVC 2012]{patterns/11_arith_optimizations/mult_shifts_MSVC_2012_Ox.asm}

\ifdefined\IncludeARM
\myparagraph{ARM}

Keil for ARM mode takes advantage of the second operand's shift modifiers:

\lstinputlisting[caption=\OptimizingKeilVI (\ARMMode)]{patterns/11_arith_optimizations/mult_shifts_Keil_ARM_O3.s}

But there are no such modifiers in Thumb mode.
It also can't optimize \TT{f2()}:

\lstinputlisting[caption=\OptimizingKeilVI (\ThumbMode)]{patterns/11_arith_optimizations/mult_shifts_Keil_thumb_O3.s}
\fi

\ifdefined\IncludeMIPS
\myparagraph{MIPS}

\lstinputlisting[caption=\Optimizing GCC 4.4.5 (IDA)]{patterns/11_arith_optimizations/mult_shifts_MIPS_O3_IDA.lst}
\fi

\subsubsection{64-bit}

\lstinputlisting{patterns/11_arith_optimizations/mult_shifts_64.c}

\myparagraph{x64}

\lstinputlisting[caption=\Optimizing MSVC 2012]{patterns/11_arith_optimizations/mult_shifts_64_GCC49_x64_O3.s}

\ifdefined\IncludeARM
\myparagraph{ARM64}

\ifdefined\IncludeGCC

GCC 4.9 for ARM64 is also terse, thanks to the shift modifiers:

\lstinputlisting[caption=\Optimizing GCC (Linaro) 4.9 ARM64]{patterns/11_arith_optimizations/mult_shifts_64_GCC49_ARM64.s}
\fi
\fi
}
\RU{\section{Умножение}

\subsection{Умножение при помощи сложения}

Вот простой пример:

\begin{lstlisting}[caption=\Optimizing MSVC 2010]
unsigned int f(unsigned int a)
{
	return a*8;
};
\end{lstlisting}

Умножение на 8 заменяется на три инструкции сложения, делающих то же самое.
Должно быть, оптимизатор в MSVC решил, что этот код может быть быстрее.


\begin{lstlisting}
_TEXT	SEGMENT
_a$ = 8							; size = 4
_f	PROC
; File c:\polygon\c\2.c
	mov	eax, DWORD PTR _a$[esp-4]
	add	eax, eax
	add	eax, eax
	add	eax, eax
	ret	0
_f	ENDP
_TEXT	ENDS
END
\end{lstlisting}

\subsection{Умножение при помощи сдвигов}
\label{subsec:mult_using_shifts}

Ещё очень часто умножения и деления на числа вида $2^{n}$ заменяются на инструкции сдвигов.

\begin{lstlisting}
unsigned int f(unsigned int a)
{
	return a*4;
};
\end{lstlisting}

\begin{lstlisting}[caption=\NonOptimizing MSVC 2010]
_a$ = 8		; size = 4
_f	PROC
	push	ebp
	mov	ebp, esp
	mov	eax, DWORD PTR _a$[ebp]
	shl	eax, 2
	pop	ebp
	ret	0
_f	ENDP
\end{lstlisting}

Умножить на 4 это просто сдвинуть число на 2 бита влево, 
вставив 2 нулевых бита справа (как два самых младших бита). 
Это как умножить 3 на 100~--- нужно просто дописать два нуля справа.

Вот как работает инструкция сдвига влево:

\myindex{x86!\Instructions!SHL}
\input{shift_left}

Добавленные биты справа~--- всегда нули.

Умножение на 4 в ARM:

\begin{lstlisting}[caption=\NonOptimizingKeilVI (\ARMMode)]
f PROC
        LSL      r0,r0,#2
        BX       lr
        ENDP
\end{lstlisting}

Умножение на 4 в MIPS:

\lstinputlisting[caption=\Optimizing GCC 4.4.5 (IDA)]{patterns/11_arith_optimizations/MIPS_SLL.lst}

\myindex{MIPS!\Instructions!SLL}
\INS{SLL} это \q{Shift Left Logical}.

\subsection{Умножение при помощи сдвигов, сложений и вычитаний}
\label{multiplication_using_shifts_adds_subs}

Можно избавиться от операции умножения, если вы умножаете на числа вроде 7 или 17,
и использовать сдвиги.

Здесь используется относительно простая математика.

\subsubsection{32-бита}

\lstinputlisting{patterns/11_arith_optimizations/mult_shifts.c}

\myparagraph{x86}

\lstinputlisting[caption=\Optimizing MSVC 2012]{patterns/11_arith_optimizations/mult_shifts_MSVC_2012_Ox.asm}

\myparagraph{ARM}

Keil, генерируя код для режима ARM, использует модификаторы инструкции, в которых можно задавать
сдвиг для второго операнда:

\lstinputlisting[caption=\OptimizingKeilVI (\ARMMode)]{patterns/11_arith_optimizations/mult_shifts_Keil_ARM_O3.s}

Но таких модификаторов в режиме Thumb нет.

И он также не смог оптимизировать функцию \TT{f2()}:

\lstinputlisting[caption=\OptimizingKeilVI (\ThumbMode)]{patterns/11_arith_optimizations/mult_shifts_Keil_thumb_O3.s}

\myparagraph{MIPS}

\lstinputlisting[caption=\Optimizing GCC 4.4.5 (IDA)]{patterns/11_arith_optimizations/mult_shifts_MIPS_O3_IDA.lst}

\subsubsection{64-бита}

\lstinputlisting{patterns/11_arith_optimizations/mult_shifts_64.c}

\myparagraph{x64}

\lstinputlisting[caption=\Optimizing MSVC 2012]{patterns/11_arith_optimizations/mult_shifts_64_GCC49_x64_O3.s}

\myparagraph{ARM64}

GCC 4.9 для ARM64 также очень лаконичен благодаря модификаторам сдвига:

\lstinputlisting[caption=\Optimizing GCC (Linaro) 4.9 ARM64]{patterns/11_arith_optimizations/mult_shifts_64_GCC49_ARM64.s}
}
\DE{\subsection{Multiplikation}

\subsubsection{Multiplikation durch Addition}

Hier ist ein einfaches Beispiel:

\begin{lstlisting}[style=customc]
unsigned int f(unsigned int a)
{
	return a*8;
};
\end{lstlisting}

Die Multiplikation mit 8 wird durch 3 Additionsbefehle ersetzt, welche das
gleiche Ergebnis erzielen. Offenbar hat der MSVC Optimierer entschieden, dass
der Code so schneller sein kann.

\begin{lstlisting}[caption=\Optimizing MSVC 2010,style=customasmx86]
_TEXT	SEGMENT
_a$ = 8		; size = 4
_f	PROC
; File c:\polygon\c\2.c
	mov	eax, DWORD PTR _a$[esp-4]
	add	eax, eax
	add	eax, eax
	add	eax, eax
	ret	0
_f	ENDP
_TEXT	ENDS
END
\end{lstlisting}

\subsubsection{Multiplikation durch Verschieben}
\label{subsec:mult_using_shifts}
Multiplikation mit und Divisionen durch Zahlen, die Potenzen von 2 sind, werden
oft durch Schiebebefehle (oft auch Shifting genannt) ersetzt.

\begin{lstlisting}[style=customc]
unsigned int f(unsigned int a)
{
	return a*4;
};
\end{lstlisting}

\begin{lstlisting}[caption=\NonOptimizing MSVC 2010,style=customasmx86]
_a$ = 8		; size = 4
_f	PROC
	push	ebp
	mov	ebp, esp
	mov	eax, DWORD PTR _a$[ebp]
	shl	eax, 2
	pop	ebp
	ret	0
_f	ENDP
\end{lstlisting}

Die Multiplikation mit 4 entspricht einer Linksverschiebung der Zahl um 2 Bit
und Einfügen zweier Nullen an der rechten Seite (an den niederwertigsten beiden
Bits). Das Prinzip ist das gleiche wie bei der dezimalen Multiplikation von 3
mit 100~--wir schreiben einfach zwei Nullen rechts an die Zahl.

Der Befehl für Linksverschiebung funktioniert wie folgt:

\myindex{x86!\Instructions!SHL}
\input{shift_left}

Die beiden rechts angefügten Bits sind stets Nullen.

Multiplikation mit 4 in ARM:

\begin{lstlisting}[caption=\NonOptimizingKeilVI (\ARMMode),style=customasmARM]
f PROC
        LSL      r0,r0,#2
        BX       lr
        ENDP
\end{lstlisting}

Multiplikation mit 4 in MIPS:

\lstinputlisting[caption=\Optimizing GCC 4.4.5 (IDA),style=customasmMIPS]{patterns/11_arith_optimizations/MIPS_SLL.lst}

\myindex{MIPS!\Instructions!SLL}
\INS{SLL} bedeutet \q{Shift Left Logical}.

\subsubsection{Multiplikation durch Verschieben, Subtrahieren und Addieren}
\label{multiplication_using_shifts_adds_subs}
Es ist auch möglich die Multiplikation zu ersetzen, wenn man mit Zahlen wie 7
oder 17 multipliziert, wenn Verschiebung verwendet wird.
Die zugrundeliegende Mathematik ist relativ einfach.

\myparagraph{32-bit}

\lstinputlisting[style=customc]{patterns/11_arith_optimizations/mult_shifts.c}

\mysubparagraph{x86}

\lstinputlisting[caption=\Optimizing MSVC 2012,style=customasmx86]{patterns/11_arith_optimizations/mult_shifts_MSVC_2012_Ox.asm}

\mysubparagraph{ARM}

Keil im ARM mode benutzt den Umwandler zur Verschiebung im zweiten Operanden:

\lstinputlisting[caption=\OptimizingKeilVI (\ARMMode),style=customasmARM]{patterns/11_arith_optimizations/mult_shifts_Keil_ARM_O3.s}

Da es im Thumb mode keine solchen Umwandler gibt, kann folglich \TT{f2()} nicht
optimiert werden:

\lstinputlisting[caption=\OptimizingKeilVI (\ThumbMode),style=customasmARM]{patterns/11_arith_optimizations/mult_shifts_Keil_thumb_O3.s}

\mysubparagraph{MIPS}

\lstinputlisting[caption=\Optimizing GCC 4.4.5 (IDA),style=customasmMIPS]{patterns/11_arith_optimizations/mult_shifts_MIPS_O3_IDA.lst}

\myparagraph{64-bit}

\lstinputlisting[style=customc]{patterns/11_arith_optimizations/mult_shifts_64.c}

\mysubparagraph{x64}

\lstinputlisting[caption=\Optimizing MSVC 2012,style=customasmx86]{patterns/11_arith_optimizations/mult_shifts_64_GCC49_x64_O3.s}

\mysubparagraph{ARM64}

GCC 4.9 für ARM64 fasst sich dank der Verschiebe-Umwandler ebenfalls kurz:

\lstinputlisting[caption=\Optimizing GCC (Linaro) 4.9 ARM64,style=customasmARM]{patterns/11_arith_optimizations/mult_shifts_64_GCC49_ARM64.s}

\myparagraph{Booths Multiplikationsalgorithmus}

\myindex{Data general Nova}
\myindex{Booth's multiplication algorithm}
Es gab Zeiten, in denen Computer groß und so teuer waren, dass einige von ihnen
keinen Hardwaresupport für die Multiplikation in der \ac{CPU} besaßen, so zum
Beispiel der Data General Nova. 
Wenn dort eine Multiplikation benötigt wurde, musste diese softwareseitig
abgebildet werden, zum Beispiel durch Booths Multiplikationsalgorithmus. 
Dabei handelt es sich um einen Algorithmus zur Multiplikation, welcher lediglich
Addtionen und Verschiebeoperationen verwendet. 

Zwar gehen moderne optimierende Compiler hier anders vor, aber das Ziel (die
Multiplikation) und die Ressourcenfrage (schnellere Operationen) sind gleich.

}

\EN{\subsection{Division}

\subsubsection{Division using shifts}
\label{division_by_shifting}

Example of division by 4:

\begin{lstlisting}[style=customc]
unsigned int f(unsigned int a)
{
	return a/4;
};
\end{lstlisting}

We get (MSVC 2010):

\begin{lstlisting}[caption=MSVC 2010,style=customasmx86]
_a$ = 8		; size = 4
_f	PROC
	mov	eax, DWORD PTR _a$[esp-4]
	shr	eax, 2
	ret	0
_f	ENDP
\end{lstlisting}

\label{SHR}
\myindex{x86!\Instructions!SHR}

The \SHR (\IT{SHift Right}) instruction in this example is shifting a number by 2 bits to the right.
The two freed bits at left (e.g., two most significant bits) are set to zero.
The two least significant bits are dropped.
In fact, these two dropped bits are the division operation remainder.

\myindex{x86!\Instructions!SHR}

The \SHR instruction works just like \SHL, but in the other direction.

\input{shift_right}

It is easy to understand if you imagine the number 23 in the decimal numeral system.
23 can be easily divided by 10 just by dropping last digit (3---division remainder). 
2 is left after the operation as a \gls{quotient}.

So the remainder is dropped, but that's OK, we work on integer values anyway, 
these are not a \glslink{real number}{real numbers}!

Division by 4 in ARM:

\begin{lstlisting}[caption=\NonOptimizingKeilVI (\ARMMode),style=customasmARM]
f PROC
        LSR      r0,r0,#2
        BX       lr
        ENDP
\end{lstlisting}

Division by 4 in MIPS:

\begin{lstlisting}[caption=\Optimizing GCC 4.4.5 (IDA),style=customasmMIPS]
        jr      $ra
        srl     $v0, $a0, 2 ; branch delay slot
\end{lstlisting}

\myindex{MIPS!\Instructions!SRL}
The SRL instruction is \q{Shift Right Logical}.
}
\RU{\subsection{Деление}

\subsubsection{Деление используя сдвиги}
\label{division_by_shifting}

Например, возьмем деление на 4:

\begin{lstlisting}[style=customc]
unsigned int f(unsigned int a)
{
	return a/4;
};
\end{lstlisting}

Имеем в итоге (MSVC 2010):

\begin{lstlisting}[caption=MSVC 2010,style=customasmx86]
_a$ = 8		; size = 4
_f	PROC
	mov	eax, DWORD PTR _a$[esp-4]
	shr	eax, 2
	ret	0
_f	ENDP
\end{lstlisting}

\label{SHR}
\myindex{x86!\Instructions!SHR}
Инструкция \SHR (\IT{SHift Right}) в данном примере сдвигает число на 2 бита вправо. 
При этом освободившиеся два бита слева (т.е. самые 
старшие разряды) выставляются в нули. А самые младшие 2 бита выкидываются. 
Фактически, эти два выкинутых бита~--- остаток от деления.

\myindex{x86!\Instructions!SHR}
Инструкция \SHR работает так же как и \SHL, только в другую сторону.

\input{shift_right}

Для того, чтобы это проще понять, представьте себе десятичную систему счисления и число 23. 
23 можно разделить на 10 просто откинув последний разряд (3~--- это остаток от деления). 
После этой операции останется 2 как \glslink{quotient}{частное}.

Так что остаток выбрасывается, но это нормально, мы все-таки работаем с целочисленными
значениями, а не с \glslink{real number}{вещественными}!

Деление на 4 в ARM:

\begin{lstlisting}[caption=\NonOptimizingKeilVI (\ARMMode),style=customasmARM]
f PROC
        LSR      r0,r0,#2
        BX       lr
        ENDP
\end{lstlisting}

Деление на 4 в MIPS:

\begin{lstlisting}[caption=\Optimizing GCC 4.4.5 (IDA),style=customasmMIPS]
        jr      $ra
        srl     $v0, $a0, 2 ; branch delay slot
\end{lstlisting}

\myindex{MIPS!\Instructions!SRL}
Инструкция SRL это \q{Shift Right Logical}.
}
\DE{\subsection{Division}

\subsubsection{Division durch Verschieben}
\label{division_by_shifting}

Beispiel der Division durch 4:

\begin{lstlisting}[style=customc]
unsigned int f(unsigned int a)
{
	return a/4;
};
\end{lstlisting}

Wir betrachten (MSVC 2010):

\begin{lstlisting}[caption=MSVC 2010,style=customasmx86]
_a$ = 8		; size = 4
_f	PROC
	mov	eax, DWORD PTR _a$[esp-4]
	shr	eax, 2
	ret	0
_f	ENDP
\end{lstlisting}

\label{SHR}
\myindex{x86!\Instructions!SHR}

Der Befehl \SHR (\IT{Shift Right}) verschiebt die Zahl in diesem Beispiel um 2
Bits nach rechts. Die beiden freien Bits am linken Rand (dies sind die beiden
höchstwertigsten Bits) werden auf null gesetzt. 
Die beiden niederwertigsten Bits werden entfernt. 
Diese beiden entfernten Bits entsprechen genau dem Rest der Division.

\myindex{x86!\Instructions!SHR}

Der \SHR Befehl funktioniert genau wie \SHL, aber in die entgegengesetzte
Richtung.

\input{shift_right}
Das Vorgehen kann leicht verdeutlicht werden, wenn wir es an der Zahl 23 im
Dezimalsystem veranschaulichen. Die 23 kann einfach durch 10 geteilt werden,
indem die letzte Ziffer (3--Rest der Division) entfernt wird. Die 2 bleibt bei
der Division als ganzzahliger \glslink{quotient}{Quotient} übrig.

Der Rest wird also entfernt, was aber kein Problem darstellt, da wir hier
ausschließlich mit ganzzahligen Werten arbeiten und nicht mit \glslink{real number}{reellen Zahlen}.

Division durch 4 in ARM:

\begin{lstlisting}[caption=\NonOptimizingKeilVI (\ARMMode),style=customasmARM]
f PROC
        LSR      r0,r0,#2
        BX       lr
        ENDP
\end{lstlisting}

Division by 4 in MIPS:

\begin{lstlisting}[caption=\Optimizing GCC 4.4.5 (IDA),style=customasmMIPS]
        jr      $ra
        srl     $v0, $a0, 2 ; branch delay slot
\end{lstlisting}

\myindex{MIPS!\Instructions!SRL}
Der Befehl SRL steht für \q{Shift Right Logical}.
}

\section{\Exercises}

% 1

\subsection{\Exercise \#2}
\label{exercise_arith_optimizations_2}

\WhatThisCodeDoes\

\begin{lstlisting}[caption=\Optimizing MSVC 2010]
_a$ = 8
_f	PROC
	mov	ecx, DWORD PTR _a$[esp-4]
	lea	eax, DWORD PTR [ecx*8]
	sub	eax, ecx
	ret	0
_f	ENDP
\end{lstlisting}

\begin{lstlisting}[caption=\NonOptimizingKeilVI (\ARMMode)]
f PROC
        RSB      r0,r0,r0,LSL #3
        BX       lr
        ENDP
\end{lstlisting}

\begin{lstlisting}[caption=\NonOptimizingKeilVI (\ThumbMode)]
f PROC
        LSLS     r1,r0,#3
        SUBS     r0,r1,r0
        BX       lr
        ENDP
\end{lstlisting}

\begin{lstlisting}[caption=\Optimizing GCC 4.9 (ARM64)]
f:
	lsl	w1, w0, 3
	sub	w0, w1, w0
	ret
\end{lstlisting}

\begin{lstlisting}[caption=\Optimizing GCC 4.4.5 (MIPS) (IDA)]
f:
                sll     $v0, $a0, 3
                jr      $ra
                subu    $v0, $a0
\end{lstlisting}

\Answer\: \myref{exercise_solutions_arith_optimizations_2}.


\ifx\LITE\undefined
\chapter{\FPUChapterName}
\label{sec:FPU}

\newcommand{\FNURLSTACK}{\footnote{\href{http://go.yurichev.com/17123}{wikipedia.org/wiki/Stack\_machine}}}
\newcommand{\FNURLFORTH}{\footnote{\href{http://go.yurichev.com/17124}{wikipedia.org/wiki/Forth\_(programming\_language)}}}
\newcommand{\FNURLIEEE}{\footnote{\href{http://go.yurichev.com/17125}{wikipedia.org/wiki/IEEE\_floating\_point}}}
\newcommand{\FNURLSP}{\footnote{\href{http://go.yurichev.com/17126}{wikipedia.org/wiki/Single-precision\_floating-point\_format}}}
\newcommand{\FNURLDP}{\footnote{\href{http://go.yurichev.com/17127}{wikipedia.org/wiki/Double-precision\_floating-point\_format}}}
\newcommand{\FNURLEP}{\footnote{\href{http://go.yurichev.com/17128}{wikipedia.org/wiki/Extended\_precision}}}

\RU{\ac{FPU}\EMDASH блок в процессоре работающий с числами с плавающей запятой.}
\EN{The \ac{FPU} is a device within the main \ac{CPU}, specially designed to deal with floating point numbers.}
\RU{Раньше он назывался \q{сопроцессором} и он стоит немного в стороне от \ac{CPU}.}
\EN{It was called \q{coprocessor} in the past and it stays somewhat aside of the main \ac{CPU}.}

\section{IEEE 754}

\RU{Число с плавающей точкой в формате IEEE 754 состоит из \IT{знака}, \IT{мантиссы}\footnote{\IT{significand} или \IT{fraction} 
в англоязычной литературе} и \IT{экспоненты}.}
\EN{A number in the IEEE 754 format consists of a \IT{sign}, a \IT{significand} (also called \IT{fraction}) and an \IT{exponent}.}

\section{x86}

\RU{Перед изучением \ac{FPU} в x86 полезно ознакомиться с тем как работают стековые машины\FNURLSTACK 
или ознакомиться с основами языка Forth\FNURLFORTH.}
\EN{It is worth looking into stack machines\FNURLSTACK or learning the basics of the Forth language\FNURLFORTH,
before studying the \ac{FPU} in x86.}

\index{Intel!80486}
\index{Intel!FPU}
\RU{Интересен факт, что в свое время (до 80486) сопроцессор был отдельным чипом на материнской плате, 
и вследствие его высокой цены, он не всегда присутствовал. Его можно было докупить и установить отдельно}%
\EN{It is interesting to know that in the past (before the 80486 CPU) the coprocessor was a separate chip 
and it was not always pre-installed on the motherboard. It was possible to buy it separately and install it}%
\footnote{\RU{Например, Джон Кармак использовал в своей игре Doom числа с фиксированной запятой 
(\href{http://go.yurichev.com/17357}{ru.wikipedia.org/wiki/Число\_с\_фиксированной\_запятой}), хранящиеся
в обычных 32-битных \ac{GPR} (16 бит на целую часть и 16 на дробную),
чтобы Doom работал на 32-битных компьютерах без FPU, т.е. 80386 и 80486 SX.}
\EN{For example, John Carmack used fixed-point arithmetic 
(\href{http://go.yurichev.com/17356}{wikipedia.org/wiki/Fixed-point\_arithmetic}) values in his Doom video game, stored in 
32-bit \ac{GPR} registers (16 bit for integral part and another 16 bit for fractional part), so Doom
could work on 32-bit computers without FPU, i.e., 80386 and 80486 SX.}}.
\RU{Начиная с 80486 DX в состав процессора всегда входит FPU.}
\EN{Starting with the 80486 DX CPU, the \ac{FPU} is integrated in the \ac{CPU}.}

\index{x86!\Instructions!FWAIT}
\RU{Этот факт может напоминать такой рудимент как наличие инструкции \TT{FWAIT}, 
которая заставляет
\ac{CPU} ожидать, пока \ac{FPU} закончит работу}\EN{The \TT{FWAIT} instruction reminds us of that fact---it
switches the \ac{CPU} to a waiting state, so it can wait until the \ac{FPU} is done with its work}.
\RU{Другой рудимент это тот факт, что опкоды \ac{FPU}-инструкций начинаются с т.н. \q{escape}-опкодов 
(\TT{D8..DF}) как опкоды, передающиеся в отдельный сопроцессор.}
\EN{Another rudiment is the fact that the \ac{FPU} instruction 
opcodes start with the so called \q{escape}-opcodes (\TT{D8..DF}), i.e., 
opcodes passed to a separate coprocessor.}

\index{IEEE 754}
\label{FPU_is_stack}
\RU{FPU имеет стек из восьми 80-битных регистров:}
\EN{The FPU has a stack capable to holding 8 80-bit registers, and each register can hold a number 
in the IEEE 754\FNURLIEEE format.}
\RU{\ST{0}..\ST{7}. Для краткости, IDA и \olly отображают \ST{0} как \TT{ST},
что в некоторых учебниках и документациях означает \q{Stack Top} (\q{вершина стека}).}
\RU{Каждый регистр может содержать число в формате IEEE 754\FNURLIEEE.}
\EN{They are \ST{0}..\ST{7}. For brevity, IDA and \olly show \ST{0} as \TT{ST}, 
which is represented in some textbooks and manuals as \q{Stack Top}.}

\section{ARM, MIPS, x86/x64 SIMD}

\RU{В ARM и MIPS FPU это не стек, а просто набор регистров.}
\EN{In ARM and MIPS the FPU is not a stack, but a set of registers.}
\RU{Такая же идеология применяется в расширениях SIMD в процессорах x86/x64.}
\EN{The same ideology is used in the SIMD extensions of x86/x64 CPUs.}

\section{\CCpp}

\index{float}
\index{double}
\RU{В стандартных \CCpp имеются два типа для работы с числами с плавающей запятой: 
\Tfloat (\IT{число одинарной точности}\FNURLSP, 32 бита)
\footnote{Формат представления чисел с плавающей точкой одинарной точности затрагивается в разделе 
\IT{\WorkingWithFloatAsWithStructSubSubSectionName}~(\myref{sec:floatasstruct}).}
и \Tdouble (\IT{число двойной точности}\FNURLDP, 64 бита).}
\EN{The standard \CCpp languages offer at least two floating number types, \Tfloat (\IT{single-precision}\FNURLSP, 32 bits)
\footnote{the single precision floating point number format is also addressed in 
the \IT{\WorkingWithFloatAsWithStructSubSubSectionName}~(\myref{sec:floatasstruct}) section}
and \Tdouble (\IT{double-precision}\FNURLDP, 64 bits).}

\index{long double}
\RU{GCC также поддерживает тип \IT{long double} (\IT{extended precision}\FNURLEP, 80 бит), но MSVC~--- нет.}
\EN{GCC also supports the \IT{long double} type (\IT{extended precision}\FNURLEP, 80 bit), which MSVC doesn't.}

\RU{Несмотря на то, что \Tfloat занимает столько же места, сколько и \Tint на 32-битной архитектуре, 
представление чисел, разумеется, совершенно другое.}
\EN{The \Tfloat type requires the same number of bits as the \Tint type in 32-bit environments, 
but the number representation is completely different.}

\section{\RU{Простой пример}\EN{Simple example}}

\RU{Рассмотрим простой пример}\EN{Let's consider this simple example}:

\lstinputlisting{patterns/12_FPU/1_simple/simple.c}

\input{patterns/12_FPU/1_simple/x86}
\ifdefined\IncludeARM
\input{patterns/12_FPU/1_simple/ARM/main}
\fi
\ifdefined\IncludeMIPS
\input{patterns/12_FPU/1_simple/MIPS}
\fi

\subsection{\RU{Передача чисел с плавающей запятой в аргументах}\EN{Passing floating point numbers via arguments}\DEph{}}
\myindex{\CStandardLibrary!pow()}

\lstinputlisting[style=customc]{patterns/12_FPU/2_passing_floats/pow.c}

\EN{\input{patterns/12_FPU/2_passing_floats/x86_EN}}
\RU{\input{patterns/12_FPU/2_passing_floats/x86_RU}}
\DE{\input{patterns/12_FPU/2_passing_floats/x86_DE}}

\EN{\input{patterns/12_FPU/2_passing_floats/ARM_EN}}
\RU{\input{patterns/12_FPU/2_passing_floats/ARM_RU}}
\DE{\input{patterns/12_FPU/2_passing_floats/ARM_DE}}

\EN{\input{patterns/12_FPU/2_passing_floats/MIPS_EN}}
\RU{\input{patterns/12_FPU/2_passing_floats/MIPS_RU}}
\DE{\input{patterns/12_FPU/2_passing_floats/MIPS_DE}}


\section{\RU{Пример с сравнением}\EN{Comparison example}}

\RU{Попробуем теперь вот это:}\EN{Let's try this:}

\lstinputlisting{patterns/12_FPU/3_comparison/d_max.c}

\RU{Несмотря на кажущуюся простоту этой функции, понять, как она работает, будет чуть сложнее.}
\EN{Despite the simplicity of the function, it will be harder to understand how it works.}

% subsections
\input{patterns/12_FPU/3_comparison/x86/main}
\ifdefined\IncludeARM
\input{patterns/12_FPU/3_comparison/ARM/ARM32}
\input{patterns/12_FPU/3_comparison/ARM/ARM64}
\fi
\ifdefined\IncludeMIPS
\input{patterns/12_FPU/3_comparison/MIPS}
\fi


\section{\RU{Стек, калькуляторы и обратная польская запись}\EN{Stack, calculators and reverse Polish notation}}

\index{\RU{Обратная польская запись}\EN{Reverse Polish notation}}
\RU{Теперь понятно, почему некоторые старые калькуляторы использовали обратную польскую запись%
\footnote{\href{http://go.yurichev.com/17355}{ru.wikipedia.org/wiki/Обратная\_польская\_запись}}.}
\EN{Now we undestand why some old calculators used reverse Polish notation
\footnote{\href{http://go.yurichev.com/17354}{wikipedia.org/wiki/Reverse\_Polish\_notation}}.}
\RU{Например для сложения 12 и 34 нужно было набрать 12, потом 34, потом нажать знак \q{плюс}.}
\EN{For example, for addition of 12 and 34 one has to enter 12, then 34, then press \q{plus} sign.}
\RU{Это потому что старые калькуляторы просто реализовали стековую машину и это было куда проще, 
чем обрабатывать сложные выражения со скобками.}
\EN{It's because old calculators were just stack machine implementations, and this was much simpler
than to handle complex parenthesized expressions.}
\section{x64}

\RU{О том, как происходит работа с числами с плавающей запятой в x86-64, читайте здесь: \myref{floating_SIMD}.}
\EN{On how floating point numbers are processed in x86-64, read more here: \myref{floating_SIMD}.}

% sections
\ifdefined\IncludeExercises
\section{\Exercises}

\subsection{\Exercise \#1}

\RU{Избавтесь от инструкции}\EN{Eliminate} FXCH \RU{в примере}\EN{instruciton in example} 
\ref{gcc481_o3} \RU{и протестируйте его}\EN{and test it}.

\subsection{\Exercise \#2}
\label{exercise_FPU_2}

\WhatThisCodeDoes\

\begin{lstlisting}[caption=\Optimizing MSVC 2010]
__real@4014000000000000 DQ 04014000000000000r	; 5

_a1$ = 8	; size = 8
_a2$ = 16	; size = 8
_a3$ = 24	; size = 8
_a4$ = 32	; size = 8
_a5$ = 40	; size = 8
_f	PROC
	fld	QWORD PTR _a1$[esp-4]
	fadd	QWORD PTR _a2$[esp-4]
	fadd	QWORD PTR _a3$[esp-4]
	fadd	QWORD PTR _a4$[esp-4]
	fadd	QWORD PTR _a5$[esp-4]
	fdiv	QWORD PTR __real@4014000000000000
	ret	0
_f	ENDP
\end{lstlisting}

\begin{lstlisting}[caption=\NonOptimizingKeilVI (\ThumbMode{} / \RU{скомпилировано для}\EN{compiled for} Cortex-R4F CPU)]
f PROC
        VADD.F64 d0,d0,d1
        VMOV.F64 d1,#5.00000000
        VADD.F64 d0,d0,d2
        VADD.F64 d0,d0,d3
        VADD.F64 d2,d0,d4
        VDIV.F64 d0,d2,d1
        BX       lr
        ENDP
\end{lstlisting}

\Answer\: \ref{exercise_solutions_FPU_2}.

\fi

\fi
\chapter{\Arrays}
\label{arrays}

\RU{Массив, это просто набор переменных в памяти, 
обязательно лежащих рядом, и обязательно одного типа
\footnote{\ac{AKA} ``гомогенный контейнер''}.}
\EN{Array is just a set of variables in memory, 
always lying next to each other, always has same type
\footnote{\ac{AKA} ``homogeneous container''}.}

% sections
\subsection{\RU{Простой пример}\EN{Simple example}}

\label{arrays_simple}
\lstinputlisting[style=customc]{patterns/13_arrays/1_simple/simple.c}

\EN{\input{patterns/13_arrays/1_simple/x86_EN}}\RU{\input{patterns/13_arrays/1_simple/x86_RU}}
\EN{\input{patterns/13_arrays/1_simple/ARM_EN}}\RU{\input{patterns/13_arrays/1_simple/ARM_RU}}
\EN{\input{patterns/13_arrays/1_simple/MIPS_EN}}\RU{\input{patterns/13_arrays/1_simple/MIPS_RU}}


\subsection{\RU{Переполнение буфера}\EN{Buffer overflow}}
\label{subsec:bufferoverflow}
\myindex{\BufferOverflow}

\EN{\input{patterns/13_arrays/2_BO/reading_EN}}
\RU{\input{patterns/13_arrays/2_BO/reading_RU}}
\DE{\input{patterns/13_arrays/2_BO/reading_DE}}

\EN{\input{patterns/13_arrays/2_BO/writing_EN}}
\RU{\input{patterns/13_arrays/2_BO/writing_RU}}
\DE{\input{patterns/13_arrays/2_BO/writing_DE}}

\section{\RU{Защита от переполнения буфера}\EN{Buffer overflow protection methods}}
\label{subsec:BO_protection}

\RU{В наше время пытаются бороться с переполнением буфера невзирая на халатность программистов на \CCpp. 
В MSVC есть опции вроде}%
\EN{There are several methods to protect against this scourge, regardless of the \CCpp programmers' negligence.
MSVC has options like}\footnote{
\RU{описания защит, которые компилятор может вставлять в код}%
\EN{compiler-side buffer overflow protection methods}:
\href{http://go.yurichev.com/17133}{wikipedia.org/wiki/Buffer\_overflow\_protection}}:

\begin{lstlisting}
 /RTCs Stack Frame runtime checking
 /GZ Enable stack checks (/RTCs)
\end{lstlisting}

\index{x86!\Instructions!RET}
\index{Function prologue}
\index{Security cookie}
\RU{Одним из методов является вставка в прологе функции некоего случайного значения в область локальных переменных 
и проверка этого значения в эпилоге функции перед выходом. 
Если проверка не прошла, то не выполнять инструкцию \RET, а остановиться (или зависнуть). 
Процесс зависнет, но это лучше, чем удаленная атака на ваш компьютер.}
\EN{One of the methods is to write a random value between the local variables in stack at function prologue 
and to check it in function epilogue before the function exits.
If value is not the same, do not execute the last instruction \RET, but stop (or hang).
The process will halt, but that is much better than a remote attack to your host.}
    
\newcommand{\CANARYURL}{\RU{\href{http://go.yurichev.com/17135}{miningwiki.ru/wiki/Канарейка\_в\_шахте}}%
\EN{\href{http://go.yurichev.com/17134}{wikipedia.org/wiki/Domestic\_canary\#Miner.27s\_canary}}}

\index{Canary}
\RU{Это случайное значение иногда называют \q{канарейкой}%
\footnote{\q{canary} в англоязычной литературе}, 
по аналогии с шахтной канарейкой\footnote{\CANARYURL}.
Раньше использовали шахтеры, чтобы определять, есть ли в шахте опасный газ.
}
\EN{This random value is called a \q{canary} sometimes, it is related to the miners' canary\footnote{\CANARYURL},
they were used by miners in the past days in order to detect poisonous gases quickly.}
\RU{Канарейки очень к нему чувствительны и либо проявляли сильное беспокойство, либо гибли от газа.}
\EN{Canaries are very sensitive to mine gases, they become very agitated in case of danger, or even die.}

\RU{Если скомпилировать наш простейший пример работы с массивом}
\EN{If we compile our very simple array example}~(\myref{arrays_simple}) \InENRU \ac{MSVC}
\RU{с опцией RTC1 или RTCs}\EN{with RTC1 and RTCs option}, \RU{в конце нашей функции будет вызов 
функции}\EN{you can see a call to}
\TT{@\_RTC\_CheckStackVars@8}\RU{, проверяющей корректность \q{канарейки}.}
\EN{ a function at the end of the function that checks if the \q{canary} is correct.}

\RU{Посмотрим, как дела обстоят в GCC}\EN{Let's see how GCC handles this}. 
\RU{Возьмем пример из секции про}\EN{Let's take an} \TT{alloca()}~(\myref{alloca})\EN{ example}:

\lstinputlisting{patterns/02_stack/04_alloca/2_1.c}

\RU{По умолчанию, без дополнительных ключей, GCC 4.7.3 вставит в код проверку \q{канарейки}:}
\EN{By default, without any additional options, GCC 4.7.3 inserts a \q{canary} check into the code:}

\lstinputlisting[caption=GCC 4.7.3]{patterns/13_arrays/3_BO_protection/gcc_canary.asm.\LANG}

\index{x86!\Registers!GS}
\RU{Случайное значение находится в}\EN{The random value is located in} \TT{gs:20}. 
\RU{Оно записывается в стек, затем, в конце функции, значение в стеке
сравнивается с корректной \q{канарейкой} в}\EN{It gets written on the stack and then at the end of the function
the value in the stack is compared with the correct \q{canary} in} \TT{gs:20}. 
\RU{Если значения не равны, будет вызвана функция}\EN{If the values are not equal, the} 
\TT{\_\_stack\_chk\_fail} \RU{и в консоли мы увидим что-то вроде такого}
\EN{function is called and we can see in the console something like that} (Ubuntu 13.04 x86):

\begin{lstlisting}
*** buffer overflow detected ***: ./2_1 terminated
======= Backtrace: =========
/lib/i386-linux-gnu/libc.so.6(__fortify_fail+0x63)[0xb7699bc3]
/lib/i386-linux-gnu/libc.so.6(+0x10593a)[0xb769893a]
/lib/i386-linux-gnu/libc.so.6(+0x105008)[0xb7698008]
/lib/i386-linux-gnu/libc.so.6(_IO_default_xsputn+0x8c)[0xb7606e5c]
/lib/i386-linux-gnu/libc.so.6(_IO_vfprintf+0x165)[0xb75d7a45]
/lib/i386-linux-gnu/libc.so.6(__vsprintf_chk+0xc9)[0xb76980d9]
/lib/i386-linux-gnu/libc.so.6(__sprintf_chk+0x2f)[0xb7697fef]
./2_1[0x8048404]
/lib/i386-linux-gnu/libc.so.6(__libc_start_main+0xf5)[0xb75ac935]
======= Memory map: ========
08048000-08049000 r-xp 00000000 08:01 2097586    /home/dennis/2_1
08049000-0804a000 r--p 00000000 08:01 2097586    /home/dennis/2_1
0804a000-0804b000 rw-p 00001000 08:01 2097586    /home/dennis/2_1
094d1000-094f2000 rw-p 00000000 00:00 0          [heap]
b7560000-b757b000 r-xp 00000000 08:01 1048602    /lib/i386-linux-gnu/libgcc_s.so.1
b757b000-b757c000 r--p 0001a000 08:01 1048602    /lib/i386-linux-gnu/libgcc_s.so.1
b757c000-b757d000 rw-p 0001b000 08:01 1048602    /lib/i386-linux-gnu/libgcc_s.so.1
b7592000-b7593000 rw-p 00000000 00:00 0
b7593000-b7740000 r-xp 00000000 08:01 1050781    /lib/i386-linux-gnu/libc-2.17.so
b7740000-b7742000 r--p 001ad000 08:01 1050781    /lib/i386-linux-gnu/libc-2.17.so
b7742000-b7743000 rw-p 001af000 08:01 1050781    /lib/i386-linux-gnu/libc-2.17.so
b7743000-b7746000 rw-p 00000000 00:00 0
b775a000-b775d000 rw-p 00000000 00:00 0
b775d000-b775e000 r-xp 00000000 00:00 0          [vdso]
b775e000-b777e000 r-xp 00000000 08:01 1050794    /lib/i386-linux-gnu/ld-2.17.so
b777e000-b777f000 r--p 0001f000 08:01 1050794    /lib/i386-linux-gnu/ld-2.17.so
b777f000-b7780000 rw-p 00020000 08:01 1050794    /lib/i386-linux-gnu/ld-2.17.so
bff35000-bff56000 rw-p 00000000 00:00 0          [stack]
Aborted (core dumped)
\end{lstlisting}

\index{MS-DOS}
gs \RU{это так называемый сегментный регистр. Эти регистры широко использовались во времена MS-DOS 
и DOS-экстендеров.}\EN{is the so-called segment register. These registers were used widely in MS-DOS and DOS-extenders
times.}
\RU{Сейчас их функция немного изменилась.}\EN{Today, its function is different.}
\index{TLS}
\index{Windows!TIB}
\RU{Если говорить кратко, в Linux \TT{gs} всегда указывает на \ac{TLS}~(\myref{TLS})~--- там находится различная 
информация, специфичная для выполняющегося потока.}
\EN{To say it briefly, the \TT{gs} register in Linux always points to the
\ac{TLS}~(\myref{TLS})---some information specific to thread is stored there.}
\RU{Кстати, в win32 эту же роль играет сегментный регистр \TT{fs},
он всегда указывает на}\EN{By the way, in win32
the \TT{fs} register plays the same role, pointing to}
\ac{TIB} \footnote{\href{http://go.yurichev.com/17104}{wikipedia.org/wiki/Win32\_Thread\_Information\_Block}}. 

\RU{Больше информации можно почерпнуть из исходных кодов Linux (по крайней мере, в версии 3.11): 
в файле}\EN{More information can be found in the Linux kernel source code (at least in 3.11 version), in}
\IT{arch/x86/include/asm/stackprotector.h}\RU{ в комментариях описывается эта переменная.}
\EN{ this variable is described in the comments.}

\ifdefined\IncludeARM
\input{patterns/13_arrays/3_BO_protection/ARM.tex}
\fi

\section{\RU{Еще немного о массивах}\EN{One more word about arrays}}

\RU{Теперь понятно, почему нельзя написать в исходном коде на \CCpp что-то вроде
\footnote{Впрочем, по стандарту C99 это возможно\cite[6.7.5/2]{C99TC3}: 
GCC может это сделать выделяя место под массив динамически в стеке (как alloca()~(\ref{alloca}))}}
\EN{Now we understand, why it is impossible to write something like that in \CCpp code
\footnote{However, it is possible in C99 standard\cite[6.7.5/2]{C99TC3}: 
GCC is actually do this by allocating array dynamically on the stack (like alloca()~(\ref{alloca}))}}:

\begin{lstlisting}
void f(int size)
{
    int a[size];
...
};
\end{lstlisting}

\RU{Все просто потому, чтобы выделять место под массив в локальном стеке, 
компилятору нужно знать его размер, чего он, на стадии компиляции, 
разумеется, знать не может.}
\EN{That's just because compiler must know exact array size to allocate space for 
it in local stack layout on compiling stage.}

\index{\CLanguageElements!C99!variable length arrays}
\index{\CStandardLibrary!alloca()}
\RU{Если вам нужен массив произвольной длины, то выделите столько, сколько нужно, через \TT{malloc()}, 
затем обращайтесь к выделенному блоку байт как к массиву того типа, который вам нужен.
Либо используйте возможность стандарта C99\cite[6.7.5/2]{C99TC3}, 
но внутри это очень похоже на alloca()~(\ref{alloca})}
\EN{If you need array of arbitrary size, allocate it by \TT{malloc()}, then access allocated memory block
as array of variables of type you need.
Or use C99 standard feature\cite[6.7.5/2]{C99TC3}, 
but it looks like alloca()~(\ref{alloca}) internally.}

\section{\RU{Многомерные массивы}\EN{Multidimensional arrays}}

\RU{Внутри, многомерный массив выглядит так же, как и линейный.}
\EN{Internally, a multidimensional array is essentially the same thing as a linear array.}

\RU{Ведь память компьютера линейная, это одномерный массив.
Но для удобства, этот одномерный массив легко представить как многомерный.}
\EN{Since the computer memory is linear, it is an one-dimensional array.
For convenience, this multi-dimensional array can be easily represented as one-dimensional.}

\RU{К примеру, вот как элементы массива $a[3][4]$ расположены в одномерном массиве из 12-и ячеек:}
\EN{For example, thit is how the elements of the $a[3][4]$ array are placed in one-dimensional array of 12 cells:}

\begin{table}[H]
\centering
\begin{tabular}{ | l | }
\hline
[0][0] \\
\hline
[0][1] \\
\hline
[0][2] \\
\hline
[0][3] \\
\hline
[1][0] \\
\hline
[1][1] \\
\hline
[1][2] \\
\hline
[1][3] \\
\hline
[2][0] \\
\hline
[2][1] \\
\hline
[2][2] \\
\hline
[2][3] \\
\hline
\end{tabular}
\caption{\RU{Двухмерный массив представляется в памяти как одномерный}
\EN{Two-dimensional array represented in memory as one-dimensional}}
\end{table}

\RU{Вот по каким адресам в памяти располагается каждая ячейка двухмерного массива 3*4:}
\EN{Here is how each cell of 3*4 array are placed in memory:}

\begin{table}[H]
\centering
\begin{tabular}{ | l | l | l | l | }
\hline                        
0 & 1 & 2 & 3 \\
\hline  
4 & 5 & 6 & 7 \\
\hline  
8 & 9 & 10 & 11 \\
\hline  
\end{tabular}
\caption{\RU{Адреса в памяти каждой ячейки двухмерного массива}
\EN{Memory addresses of each cell of two-dimensional array}}
\end{table}

\index{row-major order}
\RU{То есть, чтобы вычислить адрес нужного элемента, в начале умножаем первый индекс на 4 (ширину матрицы), 
затем прибавляем второй индекс.}
\EN{So, in order to calculate the address of the element we need, we first multiply the first index by
4 (matrix width) and then add the second index.}
\RU{Это называется}\EN{That's called} \IT{row-major order}, 
\RU{и такой способ представления массивов и матриц используется по крайней мере в}
\EN{and this method of array and matrix representation is used in at least} \CCpp \AndENRU Python. 
\EN{The term}\RU{Термин} \IT{row-major order} \RU{означает по-русски
примерно следующее: ``в начале записываем элементы первой строки, затем второй \dots и элементы последней 
строки в самом конце''.}
\EN{in plain English language means: ``first, write the elements of the first row, then the second row \dots 
and finally the elements of the last row''.}

\index{column-major order}
\index{FORTRAN}
\RU{Другой способ представления называется}\EN{Another method for representation is called} 
\IT{column-major order} 
(\RU{индексы массива используются в обратном порядке}\EN{the array indices are used in reverse order}) 
\RU{и это используется по крайней мере в}\EN{and it is used at least in} FORTRAN, MATLAB \AndENRU R. 
\RU{Термин }\IT{column-major order} \RU{означает по-русски
следующее: ``в начале записываем элементы первого столбца, затем второго \dots и элементы последнего столбца
в самом конце''.}
\EN{term in plain English language means: ``first, write the elements of the first column, then the second column \dots
and finally the elements of the last column''.}

\RU{Какой из способов лучше}\EN{Which method is better}?
\RU{Вообще, в терминах производительности и кэш-памяти, лучший метод организации данных это тот,
при котором к данным обращаются последовательно.}
\EN{In general, in terms of performance and cache memory, 
the best scheme for data organization is the one,
in which the elements are accessed sequentially.}
\RU{Так что если ваша функция обращается к данным построчно, то \IT{row-major order} лучше,
и наоборот.}
\EN{So if your function accesses data per row, \IT{row-major order} is better, and vice versa.}

% subsections
\input{patterns/13_arrays/5_multidimensional/2D}
\input{patterns/13_arrays/5_multidimensional/2D_as_1D}
\input{patterns/13_arrays/5_multidimensional/3D}

\ifx\LITE\undefined
\subsection{\RU{Еще примеры}\EN{More examples}}

\RU{Компьютерный экран представляет собой двухмерный массив, но видеобуфер это линейный
одномерный массив}\EN{The computer screen is represented as a 2D array, but the video-buffer is 
a linear 1D array}. 
\RU{Мы рассматриваем это здесь}\EN{We talk about it here}: \myref{Mandelbrot_demo}.
\fi

\section{\Exercises}

\begin{itemize}
	\item \url{http://challenges.re/62}
	\item \url{http://challenges.re/63}
	\item \url{http://challenges.re/64}
	\item \url{http://challenges.re/65}
	\item \url{http://challenges.re/66}
\end{itemize}



\EN{\mysection{\BitfieldsChapter}
\label{sec:bitfields}

A lot of functions define their input arguments as flags in bit fields.
\myindex{\CLanguageElements!C99!bool}

Of course, they could be substituted by a set of \Tbool-typed variables, but it is not frugally.

% sections
\input{patterns/14_bitfields/1_check/main}
\input{patterns/14_bitfields/2_set_reset/main}
\input{patterns/14_bitfields/3_shifts/main}
\input{patterns/14_bitfields/35_set_reset_FPU/main}
\input{patterns/14_bitfields/4_popcnt/main}
\input{patterns/14_bitfields/conclusion_EN}
\input{patterns/14_bitfields/exercises}
}
\RU{\mysection{\BitfieldsChapter}
\label{sec:bitfields}

Немало функций задают различные флаги в аргументах при помощи битовых полей\footnote{bit fields в англоязычной литературе}.

\myindex{\CLanguageElements!C99!bool}
Наверное, вместо этого можно было бы использовать набор переменных типа \Tbool, но это было бы 
не очень экономно.

% sections
\input{patterns/14_bitfields/1_check/main}
\input{patterns/14_bitfields/2_set_reset/main}
\input{patterns/14_bitfields/3_shifts/main}
\input{patterns/14_bitfields/35_set_reset_FPU/main}
\input{patterns/14_bitfields/4_popcnt/main}
\input{patterns/14_bitfields/conclusion_RU}
\input{patterns/14_bitfields/exercises}
}
\DE{\mysection{\BitfieldsChapter}
\label{sec:bitfields}
Eine Menge Funktionen definiert ihre Eingabeargumente als Flags in Bitfields.

\myindex{\CLanguageElements!C99!bool}
Natürlich können diese auch durch Variablen von Typ \Tbool ersetzt werden; das
wäre jedoch umständlicher als nötig.

% sections
\input{patterns/14_bitfields/1_check/main}
\input{patterns/14_bitfields/2_set_reset/main}
\input{patterns/14_bitfields/3_shifts/main}
\input{patterns/14_bitfields/35_set_reset_FPU/main}
\input{patterns/14_bitfields/4_popcnt/main}
\input{patterns/14_bitfields/conclusion_DE}
\input{patterns/14_bitfields/exercises}
}
\FR{\mysection{\BitfieldsChapter}
\label{sec:bitfields}

Beaucoup de fonctions définissent leurs arguments comme des flags dans un champ
de bits.
\myindex{\CLanguageElements!C99!bool}

Bien sûr, ils pourraient être substitués par un ensemble de variables de type \Tbool,
mais ce n'est pas frugal.

% sections
\input{patterns/14_bitfields/1_check/main}
\input{patterns/14_bitfields/2_set_reset/main}
\input{patterns/14_bitfields/3_shifts/main}
\input{patterns/14_bitfields/35_set_reset_FPU/main}
\input{patterns/14_bitfields/4_popcnt/main}
\input{patterns/14_bitfields/conclusion_FR}
\input{patterns/14_bitfields/exercises}
}


\chapter[\RU{Линейный конгруэнтный генератор}\EN{Linear congruential generator}]
{\RU{Линейный конгруэнтный генератор как генератор псевдослучайных чисел}\EN{Linear congruential generator as pseudorandom number generator}}
\index{\CStandardLibrary!rand()}
\label{LCG_simple}

\RU{Линейный конгруэнтный генератор, пожалуй, самый простой способ генерировать псевдослучайные числа.}
\EN{The linear congruential generator is probably the simplest possible way to generate random numbers.}
\RU{Он не в почете в наше время\footnote{Вихрь Мерсенна куда лучше}, но он настолько прост
(только одно умножение, одно сложение и одна операция \q{И}),
что мы можем использовать его в качестве примера.}
\EN{It's not in favour in modern times\footnote{Mersenne twister is better}, but it's so simple 
(just one multiplication, one addition and one AND operation), 
we can use it as an example.}

\lstinputlisting{patterns/145_LCG/rand.c.\LANG}

\RU{Здесь две функции: одна используется для инициализации внутреннего состояния, а вторая
вызывается собственно для генерации псевдослучайных чисел.}
\EN{There are two functions: the first one is used to initialize the internal state, and the second one is called
to generate pseudorandom numbers.}

\RU{Мы видим что в алгоритме применяются две константы}\EN{We see that two constants are used in the algorithm}.
\RU{Они взяты из}\EN{They are taken from} \cite{Numerical}.
\RU{Определим их используя выражение \CCpp \TT{\#define}. Это макрос.}
\EN{Let's define them using a \TT{\#define} \CCpp statement. It's a macro.}
\RU{Разница между макросом в \CCpp и константой в том, что все макросы заменяются на значения препроцессором
\CCpp и они не занимают места в памяти как переменные.}
\EN{The difference between a \CCpp macro and a constant is that all macros are replaced 
with their value by \CCpp preprocessor,
and they don't take any memory, unlike variables.}
\RU{А константы, напротив, это переменные только для чтения.}
\EN{In contrast, a constant is a read-only variable.}
\RU{Можно взять указатель (или адрес) переменной-константы, но это невозможно сделать с макросом.}
\EN{It's possible to take a pointer (or address) of a constant variable, but impossible to do so with a macro.}

\RU{Последняя операция \q{И} нужна, потому что согласно стандарту Си \TT{my\_rand()} должна возвращать значение в пределах
0..32767.}
\EN{The last AND operation is needed because by C-standard \TT{my\_rand()} has to return a value in 
the 0..32767 range.}
\RU{Если вы хотите получать 32-битные псевдослучайные значения, просто уберите последнюю операцию \q{И}.}
\EN{If you want to get 32-bit pseudorandom values, just omit the last AND operation.}

\section{x86}

\lstinputlisting[caption=\Optimizing MSVC 2013]{patterns/145_LCG/rand_MSVC_2013_x86_Ox.asm}

\RU{Вот мы это и видим: обе константы встроены в код.}
\EN{Here we see it: both constants are embedded into the code.}
\RU{Память для них не выделяется.}\EN{There is no memory allocated for them.}
\RU{Функция \TT{my\_srand()} просто копирует входное значение во внутреннюю переменную \TT{rand\_state}.}
\EN{The \TT{my\_srand()} function just copies its input value into the internal \TT{rand\_state} variable.}

\RU{\TT{my\_rand()} берет её, вычисляет следующее состояние \TT{rand\_state}, 
обрезает его и оставляет в регистре EAX.}
\EN{\TT{my\_rand()} takes it, calculates the next \TT{rand\_state}, cuts it and leaves it in the EAX register.}

\RU{Неоптимизированная версия побольше}\EN{The non-optimized version is more verbose}:

\lstinputlisting[caption=\NonOptimizing MSVC 2013]{patterns/145_LCG/rand_MSVC_2013_x86.asm}

\section{x64}

\RU{Версия для x64 почти такая же, и использует 32-битные регистры вместо 64-битных
(потому что мы работаем здесь с переменными типа \Tint).}
\EN{The x64 version is mostly the same and uses 32-bit registers instead of 64-bit ones 
(because we are working with \Tint values here).}
\RU{Но функция \TT{my\_srand()} берет входной аргумент из регистра \ECX, а не из стека:}
\EN{But \TT{my\_srand()} takes its input argument from the \ECX register rather than from stack:}

\lstinputlisting[caption=\Optimizing MSVC 2013 x64]{patterns/145_LCG/rand_MSVC_2013_x64_Ox.asm.\LANG}

\ifdefined\IncludeGCC
\RU{GCC делает почти такой же код}\EN{GCC compiler generates mostly the same code}.
\fi

\ifdefined\IncludeARM
\section{32-bit ARM}

\lstinputlisting[caption=\OptimizingKeilVI (\ARMMode)]{patterns/145_LCG/rand.s_Keil_ARM_O3.s.\LANG}

\RU{В ARM инструкцию невозможно встроить 32-битную константу, так что Keil-у приходится размещать
их отдельно и дополнительно загружать.}
\EN{It's not possible to embed 32-bit constants into ARM instructions, so Keil has to place them externally
and load them additionally.}

\RU{Вот еще что интересно: константу 0x7FFF также нельзя встроить.}
\EN{One interesting thing is that it's not possible to embed the 0x7FFF constant as well.}
\RU{Поэтому Keil сдвигает \TT{rand\_state} влево на 17 бит и затем сдвигает вправо на 17 бит.}
\EN{So what Keil does is shifting \TT{rand\_state} left by 17 bits and then shifting it right by 17 bits.}
\RU{Это аналогично \CCpp{}-выражению $(rand\_state \ll 17) \gg 17$.}
\EN{This is analogous to the $(rand\_state \ll 17) \gg 17$ statement in \CCpp.}
\RU{Выглядит как бессмысленная операция, но тем не менее, что она делает это очищает старшие 17 бит, оставляя
младшие 15 бит нетронутыми, и это наша цель, в конце концов.}
\EN{It seems to be useless operation, but
what it does is clearing the high 17 bits, leaving the low 15 bits intact, and that's our goal after all.}
\ESph{}\PTBRph{}\PLph{}\ITAph{}\\
\\
\Optimizing Keil \RU{для режима Thumb делает почти такой же код}\EN{for Thumb mode generates mostly the same code}.
\fi

\ifdefined\IncludeMIPS
\section{MIPS}

\lstinputlisting[caption=\Optimizing GCC 4.4.5 (IDA)]{patterns/145_LCG/MIPS_O3_IDA.lst.\LANG}

\RU{Ух, мы видим здесь только одну константу}
\EN{Wow, we see here only one constant} (0x3C6EF35F \OrENRU 1013904223).
\RU{Где же вторая}\EN{Where is another one} (1664525)?

\RU{Похоже, умножение на 1664525 сделано только при помощи сдвигов и прибавлений!}
\EN{It seems, multiplication by 1664525 is done using just shifts and additions!}
\RU{Проверим эту версию}\EN{Let's check this assumption}:

\lstinputlisting{patterns/145_LCG/test.c}

\lstinputlisting[caption=\Optimizing GCC 4.4.5 (IDA)]{patterns/145_LCG/test_O3_MIPS.lst}

\RU{Действительно}\EN{Indeed}!

\subsection{\RU{Релоки в MIPS}\EN{MIPS relocations}}

\RU{Еще поговорим о том, как на самом деле происходят операции загрузки из памяти и запись в память.}
\EN{I would also focus on how such operations as load from memory and store to memory are actually works.}
\RU{Листинги здесь были сделаны в IDA, которая убирает немного деталей.}
\EN{The listings here are produced by IDA, which hiding some details.}

\RU{Я запущу objdump дважды, чтобы получить дизассемблированный листинг и еще список релоков:}
\EN{I'll run objdump twice to get disassembled listing and also relocations list:}

\lstinputlisting[caption=\Optimizing GCC 4.4.5 (objdump)]{patterns/145_LCG/MIPS_O3_objdump.txt}

\RU{Рассмотрим два релока для ф-ции \TT{my\_srand()}.}
\EN{Let's consider two relocations for the \TT{my\_srand()} function.}
\RU{Первый, для адреса 0, имеет тип \TT{R\_MIPS\_HI16}, и второй, для адреса 8, имеет тип \TT{R\_MIPS\_LO16}.}
\EN{First for address 0 has type of \TT{R\_MIPS\_HI16} and second for address 8 has types \TT{R\_MIPS\_LO16}.}
\RU{Это значит, что адрес начала сегмента .bss будет записан в инструкцию по адресу 0 (старшая часть адреса)
и по адресу 8 (младшая асть адреса).}
\EN{That means that address of beginning of .bss segment will be written into instructions at
address of 0 (high part of address) and 8 (low part of address).}

\RU{Ведь переменная \TT{rand\_state} находится в самом начале сегмента .bss.}
\EN{\TT{rand\_state} variable is at the very start of .bss segment.}

\RU{Так что мы видим нули в операндах инструкций LUI и SW потому что там пока ничего нет --- 
компилятор не знает, что туда записать.}
\EN{So we see zeroes in the operands of instructions LUI and SW, because nothing is there yet --- 
compiler don't know what to write there.}
\RU{Линкер это исправит и старшая часть адреса будет записана в операнд инструкции LUI и младшая часть адреса ---
в операнд инструкции SW.}
\EN{Linker will fix this and high part of address will be written into LUI instruction operand and
low part of address --- to operand of SW instruction.}
\RU{SW просуммирует младшую асть адреса и то что находится в регистре \$V0 (там старшая часть).}
\EN{SW will sum up low part of address and what is in \$V0 register (high part is there).}

\RU{Та же история и с ф-цией my\_rand(): релок R\_MIPS\_HI16 указывает линкеру записать старшую часть
адреса сегмента .bss в инструкцию LUI.}
\EN{The same story about my\_rand() function: R\_MIPS\_HI16 relocation instructs linker to write high part
of .bss segment address into LUI instruction.}
\RU{Так что старшая часть адреса переменной rand\_state будет находится в регистре \$V1.}
\EN{So, high part of rand\_state variable address will reside in \$V1 register.}
\RU{Инструкция LW по адресу 0x10 просуммирует старшую и младшую часть и загрузит значение переменной 
rand\_state в \$V1.}
\EN{LW instruction at address 0x10 will sum up high and low part and load value of rand\_state 
variable into \$V1.}
\RU{Инструкция SW по адресу 0x54 также просуммирует и затем запишет новое значение в глобальную переменную
rand\_state.}
\EN{SW instruction at address 0x54 will also do the summing and then will store new value 
to rand\_state global variable.}

\RU{Так что IDA обрабатывает релоки при загрузке, и таким образом эти детали скрываются.}
\EN{So, IDA processes relocations while loading, thus hiding these details.}
\RU{Но мы должны о них помнить.}\EN{But we should remember about them.}

% TODO add example of compiled binary, GDB example, etc...

\fi

\ifx\LITE\undefined
\section{\RU{Версия этого примера для многопоточной среды}\EN{Thread-safe version of the example}}

\RU{Версия примера для многопоточной среды будет рассмотрена позже}%
\EN{The thread-safe version of the example is to be demonstrated later}: \myref{LCG_TLS}.
\fi

\section{\IFRU{Структуры}{Structures}}

\IFRU{В принципе, структура в \CCpp это, с некоторыми допущениями, просто всегда лежащий рядом, 
и в той же последовательности, набор переменных, не обязательно одного типа
\footnote{\ac{AKA} ``гетерогенный контейнер''}.}
{It can be defined the \CCpp structure, with some assumptions, just a set of variables, always stored
in memory together, not necessary of the same type
\footnote{\ac{AKA} ``heterogeneous container''}.}

\section{\RU{Пример SYSTEMTIME}\EN{SYSTEMTIME example}}

\newcommand{\FNSYSTEMTIME}{\footnote{\href{http://msdn.microsoft.com/en-us/library/ms724950(VS.85).aspx}{MSDN: SYSTEMTIME structure}}}

\RU{Возьмем, к примеру, структуру SYSTEMTIME\FNSYSTEMTIME{} из win32 описывающую время.}
\EN{Let's take SYSTEMTIME\FNSYSTEMTIME{} win32 structure describing time.}

\RU{Она объявлена так:}\EN{That's how it is defined:}

\begin{lstlisting}[caption=WinBase.h]
typedef struct _SYSTEMTIME {
  WORD wYear;
  WORD wMonth;
  WORD wDayOfWeek;
  WORD wDay;
  WORD wHour;
  WORD wMinute;
  WORD wSecond;
  WORD wMilliseconds;
} SYSTEMTIME, *PSYSTEMTIME;
\end{lstlisting}

\RU{Пишем на Си функцию для получения текущего системного времени:}
\EN{Let's write a C function to get current time:}

\lstinputlisting{patterns/15_structs/systemtime.c}

\RU{Что в итоге}\EN{We got} (MSVC 2010):

\lstinputlisting[caption=MSVC 2010]{patterns/15_structs/systemtime.asm}

\RU{Под структуру в стеке выделено 16 байт ~--- именно столько будет \TT{sizeof(WORD)*8}
(в структуре 8 переменных с типом WORD).}
\EN{16 bytes are allocated for this structure in local stack~---that is exactly \TT{sizeof(WORD)*8}
(there are 8 WORD variables in the structure).}

\newcommand{\FNMSDNGST}{\footnote{\href{http://msdn.microsoft.com/en-us/library/ms724390(VS.85).aspx}{MSDN: GetSystemTime function}}}

\RU{Обратите внимание на тот факт, что структура начинается с поля \TT{wYear}. 
Можно сказать, что в качестве аргумента для \TT{GetSystemTime()}\FNMSDNGST передается указатель на структуру 
SYSTEMTIME, но можно также сказать, что передается указатель на поле \TT{wYear}, 
что одно и тоже! 
\TT{GetSystemTime()} пишет текущий год в тот WORD на который указывает переданный указатель, 
затем сдвигается на 2 байта вправо, пишет текущий месяц, и т.д., и т.д.}
\EN{Pay attention to the fact the structure beginning with \TT{wYear} field.
It can be said, an pointer to SYSTEMTIME structure is passed to the \TT{GetSystemTime()}\FNSYSTEMTIME,
but it is also can be said, pointer to the \TT{wYear} field is passed, and that is the same!
\TT{GetSystemTime()} writes current year to the WORD pointer pointing to, then shifts 2 bytes
ahead, then writes current month, etc, etc.}

\RU{Тот факт, что поля структуры это просто переменные расположенные рядом, 
я могу проиллюстрировать следующим образом.}
\EN{The fact the structure fields are just variables located side-by-side, 
I can demonstrate by the following technique.}
\RU{Глядя на описание структуры}\EN{Keeping in ming} \TT{SYSTEMTIME}\RU{, я могу переписать этот простой пример так:}
\EN{ structure description, I can rewrite this simple example like this:}

\lstinputlisting{patterns/15_structs/systemtime2.c}

\RU{Компилятор немного поворчит:}\EN{Compiler will grumble for a little:}

\begin{lstlisting}
systemtime2.c(7) : warning C4133: 'function' : incompatible types - from 'WORD [8]' to 'LPSYSTEMTIME'
\end{lstlisting}

\RU{Тем не менее, выдаст такой код}\EN{But nevertheless, it will produce this code}:

\lstinputlisting[caption=MSVC 2010]{patterns/15_structs/systemtime2.asm}

\RU{И это работает так же}\EN{And it works just as the same}!

\RU{Любопытно что результат на ассемблере неотличим от предыдущего}
\EN{It is very interesting fact the
result in assembly form cannot be distinguished from the result of previous compilation}.
\RU{Таким образом, глядя на этот код, 
никогда нельзя сказать с уверенностью, была ли там объявлена структура, либо просто набор переменных.}
\EN{So by looking at this code, one cannot say for sure, was there structure declared, or just pack of variables.} 

\RU{Тем не менее, никто в здравом уме делать так не будет}\EN{Nevertheless, no one will do it in sane state of mind}.
\RU{Потому что это неудобно}\EN{Since it is not convenient}. 
\RU{К тому же, иногда, поля в структуре могут меняться разработчиками, 
переставляться местами, и т.д}\EN{Also structure fields may be changed by developers, swapped, etc}.


\subsection{\IFRU{Выделяем место для структуры через malloc()}{Let's allocate place for structure using malloc()}}

\IFRU{Однако, бывает и так, что проще хранить структуры не в стеке а в куче\footnote{heap}:}
{However, sometimes it's simpler to place structures not in local stack, but in heap:}

\lstinputlisting{15_structs/systemtime_malloc.c}

\IFRU{Скомпилируем на этот раз с оптимизацией (\Ox) чтобы было проще увидеть то, что нам нужно.}
{Let's compile it now with optimization (\Ox) so to easily see what we need.}

\lstinputlisting[caption=\Optimizing MSVC]{15_structs/systemtime_malloc.asm}

\index{\CLanguageElements!malloc()}
\IFRU{Итак, \TT{sizeof(SYSTEMTIME) = 16}, именно столько байт выделяется при помощи \TT{malloc()}. 
Она возвращает указатель на только что выделенный блок памяти в \EAX, который копируется в \ESI. 
Win32 функция \TT{GetSystemTime()} обязуется сохранить состояние \ESI, 
поэтому здесь оно нигде не сохраняется и продолжает использоваться после вызова \TT{GetSystemTime()}.}
{So, \TT{sizeof(SYSTEMTIME) = 16}, that's exact number of bytes to be allocated by \TT{malloc()}.
It return the pointer to freshly allocated memory block in \EAX, which is then moved into \ESI.
\TT{GetSystemTime()} win32 function undertake to save \ESI value, 
and that's why it is not saved here and continue to be used after \TT{GetSystemTime()} call.}

\index{x86!\Instructions!MOVZX}
\IFRU{
Новая инструкция ~--- \MOVZX (\IT{Move with Zero eXtent}). 
Она нужна почти там же где и \MOVSX, 
только всегда очищает остальные биты в $0$. Дело в том что \printf требует 32-битный тип \Tint, 
а в структуре лежит WORD ~--- это 16-битный беззнаковый тип. Поэтому копируя значение из WORD в \Tint, 
нужно очистить биты от 16 до 31, иначе там будет просто случайный мусор, оставшийся от предыдущих действий 
с регистрами.}
{New instruction ~--- \MOVZX (\IT{Move with Zero eXtent}).
It may be used almost in those cases as \MOVSX, but, it clearing other bits to $0$.
That's because \printf require 32-bit \Tint, but we got WORD in structure ~--- that's 16-bit unsigned type.
That's why by copying value from WORD into \Tint{}, bits from 16 to 31 should be cleared, 
because there will be random noise otherwise, leaved from previous operations on registers.}

\IFRU{В этом примере я тоже могу представить структуру как массив WORD-ов}{In this example, I can represent
structure as array of WORD-s}:

\lstinputlisting{15_structs/systemtime_malloc2.c}

\IFRU{Получим такое}{We got}:

\lstinputlisting[caption=\Optimizing MSVC]{15_structs/systemtime_malloc2.asm}

\IFRU{И снова мы получаем идетичный код, неотличимый от предыдущего}{Again, we got a code that cannot be distinguished
from previous}.
\IFRU{Но и снова я должен отметить, что в реальности так лучше не делать}{And again I should note, one shouldn't do
this in practice}.


\subsection{struct tm}

\subsubsection{Linux}

\IFRU{В Линуксе, для примера, возьем структуру \TT{tm} из \TT{time.h}:}
{As of Linux, let's take \TT{tm} structure from \TT{time.h} for example:}

\lstinputlisting{15_structs/GCC_tm.c}

\IFRU{Компилируем при помощи}{Let's compile it in} GCC 4.4.1:

\IFRU{\lstinputlisting[caption=GCC 4.4.1]{15_structs/GCC_tm_ru.asm}}{\lstinputlisting{15_structs/GCC_tm_en.asm}}

\IFRU{К сожалению, по какой-то причине, \IDA не сформировала названия локальных переменных в стеке. 
Но так как мы уже опытные реверсеры :-) то можем обойтись и без этого в таком простом примере.}
{Somehow, \IDA didn't created local variables names in local stack.
But since we already experienced reverse engineers :-) we may do it without this information in 
this simple example.}

\IFRU{Обратите внимание на \TT{lea edx, [eax+76Ch]} ~--- эта инструкция прибавляет $0x76C$ к \EAX, 
но не модифицирует флаги. См. также соответствующий раздел об инструкции \LEA{}~\ref{sec:LEA}.}
{Please also pay attention to \TT{lea edx, [eax+76Ch]} ~--- this instruction just adding $0x76C$ to \EAX,
but not modify any flags. See also relevant section about \LEA{}~\ref{sec:LEA}.}

Чтобы проиллюстрировать то что структура это просто набор переменных лежащих в одном месте, переделаем немного
пример, заглянув предварительно в файл time.h:

\begin{lstlisting}[caption=time.h]
struct tm
{
  int	tm_sec;
  int	tm_min;
  int	tm_hour;
  int	tm_mday;
  int	tm_mon;
  int	tm_year;
  int	tm_wday;
  int	tm_yday;
  int	tm_isdst;
};
\end{lstlisting}

\lstinputlisting{15_structs/GCC_tm2.c}

Обратите внимание на то что в \TT{localtime\_r} передается указатель именно на \TT{tm\_sec}, 
т.е., на первый элемент ``структуры''.

В итоге, и этот компилятор поворчит:

\begin{lstlisting}[caption=GCC 4.7.3]
GCC_tm2.c: In function 'main':
GCC_tm2.c:11:5: warning: passing argument 2 of 'localtime_r' from incompatible pointer type [enabled by default]
In file included from GCC_tm2.c:2:0:
/usr/include/time.h:59:12: note: expected 'struct tm *' but argument is of type 'int *'
\end{lstlisting}

Тем не менее, сгенерирует такоу:

\lstinputlisting[caption=GCC 4.7.3]{15_structs/GCC_tm2.asm}

Этот код почти идентичен уже рассмотренному, и нельзя сказать, была ли структура
в оригинальном исходном коде либо набор переменных.

И это работает. Однако, в реальности так лучше не делать. Обычно, компилятор располагает переменные в локальном
стеке в том же порядке, в котором они объявляются в функции. Тем не менее, никакой гарантии нет.

Кстати, какой-нибудь другой компилятор может предупредить, что переменные \TT{tm\_year}, \TT{tm\_mon}, \TT{tm\_mday},
\TT{tm\_hour}, \TT{tm\_min}, но не \TT{tm\_sec}, используются без инициализации. 
Действительно, ведь компилятор не знает
что они будут заполнены при вызове функции \TT{localtime\_r()}.

Я выбрал именно этот пример для иллюстрации, потому что члены структуры имеют тип \Tint, а члены структуры
\TT{SYSTEMTIME} ~--- 16-битные \TT{WORD}, и если их объявлять так же, то они будут выровнены по 32-битной границе 
и ничего не выйдет (потому что \TT{GetSystemTime()} заполнит их неверно). Читайте об этом в следующей секции
``\StructurePackingSectionName''.

\index{\SyntacticSugar}
Так что, структура это просто набор переменных лежащих в одном месте, рядом. Я мог бы сказать что структура
это такой синтаксический сахар, заставляющий компилятор удерживать их в одном месте. Впрочем, я не специалист
по языкам программирования, так что, скорее всего, ошибаюсь с этим термином.
Кстати, когда-то, в очень ранних версиях Си (перед 1972) структур не 
было вовсе\cite{Ritchie:1993:DCL:155360.155580}.

\subsubsection{ARM + \OptimizingKeil + \ThumbMode}

Этот же пример:

\lstinputlisting[caption=\OptimizingKeil + \ThumbMode]{15_structs/tm_ARM_keil_thumb.asm}

\subsubsection{ARM + \OptimizingXcode + \ThumbTwoMode}

\IDA ``узнала'' структуру tm (потому что \IDA ``знает'' типы аргументов библиотечных функций, 
таких как \TT{localtime\_r()}), поэтому показала здесь обращения к элементам структуры.

\lstinputlisting[caption=\OptimizingXcode + \ThumbTwoMode]{15_structs/tm_ARM_xcode_thumb.asm}


\section{\StructurePackingSectionName}

\RU{Достаточно немаловажный момент, это упаковка полей в структурах\footnote{См. также: \URLWPDA}.}
\EN{One important thing is fields packing in structures\footnote{See also: \URLWPDA}.}

\RU{Возьмем простой пример:}\EN{Let's take a simple example:}

\lstinputlisting{patterns/15_structs/packing.c}

\RU{Как видно, мы имеем два поля \Tchar (занимающий один байт) и еще два ~--- \Tint (по 4 байта).}
\EN{As we see, we have two \Tchar fields (each is exactly one byte) and two more~---\Tint (each - 4 bytes).}

\subsection{x86}

\RU{Компилируется это все в:}\EN{That's all compiling into:}

\lstinputlisting{patterns/15_structs/packing.asm}

\RU{Мы видим здесь что адрес каждого поля в структуре выравнивается по 4-байтной границе. 
Так что каждый \Tchar здесь занимает те же 4 байта что и \Tint. Зачем? 
Затем что процессору удобнее обращаться по таким адресам и кэшировать данные из памяти.}
\EN{As we can see, each field's address is aligned on a 4-bytes border.
That's why each \Tchar occupies 4 bytes here (like \Tint). Why?
Thus it is easier for CPU to access memory at aligned addresses and to cache data from it.}

\RU{Но это не экономично по размеру данных.}\EN{However, it is not very economical in size sense.}

\RU{Попробуем скомпилировать тот же исходник с опцией}\EN{Let's try to compile it with option} (\TT{/Zp1}) 
(\IT{/Zp[n] pack structures on n-byte boundary}).

\lstinputlisting[caption=MSVC /Zp1]{patterns/15_structs/packing_msvc_Zp1.asm}

\RU{Теперь структура занимает 10 байт и все \Tchar занимают по байту. Что это дает? 
Экономию места. Недостаток ~--- процессор будет обращаться к этим полям не так эффективно 
по скорости, как мог бы.}
\EN{Now the structure takes only 10 bytes and each \Tchar value takes 1 byte. What it give to us?
Size economy. And as drawback~---CPU will access these fields without maximal performance it can.}

\RU{Как нетрудно догадаться, если структура используется много в каких исходниках и объектных файлах, 
все они должны быть откомпилированы с одним и тем же соглашением об упаковке структур.}
\EN{As it can be easily guessed, if the structure is used in many source and object files,
all these must be compiled with the same convention about structures packing.}

\newcommand{\FNURLMSDNZP}{\footnote{\href{http://msdn.microsoft.com/en-us/library/ms253935.aspx}
{MSDN: Working with Packing Structures}}}
\newcommand{\FNURLGCCPC}{\footnote{\href{http://gcc.gnu.org/onlinedocs/gcc/Structure_002dPacking-Pragmas.html}
{Structure-Packing Pragmas}}}

\RU{Помимо ключа MSVC \TT{/Zp}, указывающего, по какой границе упаковывать поля структур, есть также 
опция компилятора \TT{\#pragma pack}, её можно указывать прямо в исходнике. 
Это справедливо и для MSVC\FNURLMSDNZP и GCC\FNURLGCCPC{}.}
\EN{Aside from MSVC \TT{/Zp} option which set how to align each structure field, here is also
\TT{\#pragma pack} compiler option, it can be defined right in source code.
It is available in both MSVC\FNURLMSDNZP and GCC\FNURLGCCPC{}.}

\RU{Давайте теперь вернемся к \TT{SYSTEMTIME}, которая состоит из 16-битных полей. 
Откуда наш компилятор знает что их надо паковать по однобайтной границе?}
\EN{Let's back to the \TT{SYSTEMTIME} structure consisting in 16-bit fields.
How our compiler know to pack them on 1-byte alignment boundary?}

\RU{В файле \TT{WinNT.h} попадается такое:}\EN{\TT{WinNT.h} file has this:}

\begin{lstlisting}[caption=WinNT.h]
#include "pshpack1.h"
\end{lstlisting}

\RU{И такое:}\EN{And this:}

\begin{lstlisting}[caption=WinNT.h]
#include "pshpack4.h"                   // 4 byte packing is the default
\end{lstlisting}

\RU{Сам файл PshPack1.h выглядит так:}\EN{The file PshPack1.h looks like:}

\begin{lstlisting}[caption=PshPack1.h]
#if ! (defined(lint) || defined(RC_INVOKED))
#if ( _MSC_VER >= 800 && !defined(_M_I86)) || defined(_PUSHPOP_SUPPORTED)
#pragma warning(disable:4103)
#if !(defined( MIDL_PASS )) || defined( __midl )
#pragma pack(push,1)
#else
#pragma pack(1)
#endif
#else
#pragma pack(1)
#endif
#endif /* ! (defined(lint) || defined(RC_INVOKED)) */
\end{lstlisting}

\RU{Собственно, так и задается компилятору, как паковать объявленные после \TT{\#pragma pack} структуры.}
\EN{That's how compiler will pack structures defined after \TT{\#pragma pack}.}

\subsection{ARM + \OptimizingKeil + \ThumbMode}

\lstinputlisting[caption=\OptimizingKeil + \ThumbMode]{patterns/15_structs/packing_Keil_thumb.asm}

\RU{Как мы помним, здесь передается не указатель на структуру, а сама структура, а так как в ARM первые 4 аргумента
функции передаются через регистры, то поля структуры передаются через}
\EN{As we may recall, here a structure passed instead of pointer to structure,
and since first 4 function arguments in ARM are passed via registers,
so then structure fields are passed via} \TT{R0-R3}.

\index{ARM!\Instructions!LDRB}
\index{x86!\Instructions!MOVSX}
\RU{Инструкция }\TT{LDRB} \RU{загружает один байт из памяти и расширяет до 32-бит учитывая знак.}
\EN{loads one byte from memory and extending it to 32-bit, taking into account its sign.}
\RU{Это то же что и инструкция}\EN{This is akin to} \MOVSX \RU{в}\EN{instruction in} x86.
\RU{Она здесь применяется для загрузки полей}\EN{Here it is used for loading fields} $a$ \AndENRU $c$ 
\RU{из структуры}\EN{from structure}.

\index{Function epilogue}
\RU{Еще что бросается в глаза, так это то что вместо эпилога функции, переход на эпилог другой функции!}
\EN{One more thing we spot easily, instead of function epilogue, here is jump to another function's epilogue!}
\RU{Действительно, то была совсем другая, не относящаяся к этой, функция, однако, она имела точно такой же
эпилог}\EN{Indeed, that was quite different function, not related in any way to our function, however, it has exactly
the same epilogue} 
(\RU{видимо, тоже хранила в стеке 5 локальных переменных}\EN{probably because, it hold 5 local variables too} 
($5*4=0x14$)).
\RU{К тому же, она находится рядом (обратите внимание на адреса).}
\EN{Also it is located nearly (take a look on addresses).}
\RU{Действительно, нет никакой разницы, какой эпилог исполнять, если он работает так же, как нам нужно.}
\EN{Indeed, there is no difference, which epilogue to execute,
if it works just as we need.}
\RU{Keil решил использовать часть другой ф-ции, вероятно, из-за экономии.}
\EN{Apparently, Keil decides to reuse a part of another function by a reason of economy.}
\RU{Эпилог занимает 4 байта, а переход ~--- только 2.}
\EN{Epilogue takes 4 bytes while jump~---only 2.}

\subsection{ARM + \OptimizingXcode + \ThumbTwoMode}

\lstinputlisting[caption=\OptimizingXcode + \ThumbTwoMode]{patterns/15_structs/packing_Xcode_thumb.asm}

\index{ARM!\Instructions!SXTB}
\index{x86!\Instructions!MOVSX}
\TT{SXTB} (\IT{Signed Extend Byte}) \RU{это также аналог}\EN{is analogous to} \MOVSX \InENRU 
x86\RU{, только работает не с памятью, а с регистром.}\EN{ as well, but works not with memory, but with register.}
\RU{Всё остальное ~--- так же.}\EN{All the rest~---just the same.}


\subsection{\IFRU{Вложенные структуры}{Nested structures}}

\IFRU{Теперь, как насчет ситуаций, когда одна структура определяет внутри себя еще одну структуру?}
{Now what about situations when one structure defines another structure inside?}

\lstinputlisting{patterns/15_structs/nested.c}

\dots \IFRU{в этом случае, оба поля \TT{inner\_struct} просто будут располагаться между полями a,b и d,e в 
\TT{outer\_struct}.}
{in this case, both \TT{inner\_struct} fields will be placed between a,b and d,e fields of
\TT{outer\_struct}.}

\IFRU{Компилируем}{Let's compile} (MSVC 2010):

\lstinputlisting[caption=MSVC 2010]{patterns/15_structs/nested_msvc.asm}

\IFRU{Очень любопытный момент в том, что глядя на этот код на ассемблере, мы даже не видим, 
что была использована какая-то еще другая структура внутри этой!
Так что, пожалуй, можно сказать, что все вложенные структуры в итоге разворачиваются в одну, \IT{линейную} 
или \IT{одномерную} структуру.}
{One curious point here is that by looking onto this assembly code, we do not even see that
another structure was used inside of it!
Thus, we would say, nested structures are finally unfolds into \IT{linear} or \IT{one-dimensional} structure.}

\IFRU{Конечно, если заменить объявление \TT{struct inner\_struct c;} на \TT{struct inner\_struct *c;} 
(объявляя таким образом указатель), ситауция будет совсем иная.}
{Of course, if to replace \TT{struct inner\_struct c;} declaration to \TT{struct inner\_struct *c;} 
(thus making a pointer here) situation will be quite different.}


\subsection{\IFRU{Работа с битовыми полями в структуре}{Bit fields in structure}}

\subsubsection{\IFRU{Пример CPUID}{CPUID example}}

\IFRU{Язык \CCpp позволяет указывать, сколько именно бит отвести для каждого поля структуры. 
Это удобно если нужно экономить место в памяти. К примеру, для переменной типа \Tbool достаточно одного бита.
Но, это не очень удобно, если нужна скорость.}
{\CCpp language allow to define exact number of bits for each structure fields.
It is very useful if one needs to save memory space. 
For example, one bit is enough for variable of \Tbool type.
But of course, it is not rational if speed is important.}

\newcommand{\FNCPUID}{\footnote{\url{http://en.wikipedia.org/wiki/CPUID}}}

\index{x86!\Instructions!CPUID}
\label{cpuid}
\IFRU{Рассмотрим пример с инструкцией \CPUID\FNCPUID. 
Эта инструкция возвращает информацию о том, какой процессор имеется в наличии и какие возможности он имеет.}
{Let's consider \CPUID\FNCPUID instruction example.
This instruction returning information about current CPU and its features.}

\IFRU{Если перед исполнением инструкции в \EAX будет 1, 
то \CPUID вернет упакованную в \EAX такую информацию о процессоре:}
{If the \EAX is set to 1 before instruction execution, 
\CPUID will return this information packed into the \EAX register:}

\begin{center}
\begin{tabular}{ | l | l | }
\hline
3:0 & Stepping \\
7:4 & Model \\
11:8 & Family \\
13:12 & Processor Type \\
19:16 & Extended Model \\
27:20 & Extended Family \\
\hline
\end{tabular}
\end{center}

\newcommand{\FNGCCAS}{\footnote{\href{http://www.ibiblio.org/gferg/ldp/GCC-Inline-Assembly-HOWTO.html}
{\IFRU{Подробнее о встроенном ассемблере GCC}{More about internal GCC assembler}}}}

\IFRU{MSVC 2010 имеет макрос для \CPUID, а GCC 4.4.1 ~--- нет. 
Поэтому для GCC сделаем эту функцию сами, используя его встроенный ассемблер\FNGCCAS.}
{MSVC 2010 has \CPUID macro, but GCC 4.4.1~---has not.
So let's make this function by yourself for GCC with the help of its built-in assembler\FNGCCAS.}

\lstinputlisting{patterns/15_structs/CPUID.c}

\IFRU{После того как \CPUID заполнит \EAX/\EBX/\ECX/\EDX, у нас они отразятся в массиве \TT{b[]}. 
Затем, мы имеем указатель на структуру \TT{CPUID\_1\_EAX}, и мы указываем его на значение 
\EAX из массива \TT{b[]}.}
{After \CPUID will fill \EAX/\EBX/\ECX/\EDX, these registers will be reflected in the \TT{b[]} array.
Then, we have a pointer to the \TT{CPUID\_1\_EAX} structure and we point it to the value in the \EAX from \TT{b[]} array.}

\IFRU{Иными словами, мы трактуем 32-битный \Tint как структуру.}
{In other words, we treat 32-bit \Tint value as a structure.}

\IFRU{Затем мы читаем из структуры.}{Then we read from the stucture.}

\IFRU{Компилируем в MSVC 2008 с опцией \Ox}{Let's compile it in MSVC 2008 with \Ox option}:

\lstinputlisting[caption=\Optimizing MSVC 2008]{patterns/15_structs/CPUID_msvc_Ox.asm}

\index{x86!\Instructions!SHR}
\IFRU{Инструкция \TT{SHR} сдвигает значение из \EAX на то количество бит, 
которое нужно \IT{пропустить}, то есть, мы игнорируем некоторые биты \IT{справа}.}
{\TT{SHR} instruction shifting value in the \EAX register by number of bits must be
\IT{skipped}, e.g., we ignore a bits \IT{at right}.}

\index{x86!\Instructions!AND}
\IFRU{А инструкция \ANDIns очищает биты \IT{слева} которые нам не нужны, или же, говоря иначе, 
она оставляет по маске только те биты в \EAX, которые нам сейчас нужны.}
{\ANDIns instruction clears bits not needed \IT{at left}, or, in other words, 
leaves only those bits in the \EAX register we need now.}

\IFRU{Попробуем GCC 4.4.1 с опцией \Othree.}{Let's try GCC 4.4.1 with \Othree option.}

\lstinputlisting[caption=\Optimizing GCC 4.4.1]{patterns/15_structs/CPUID_gcc_O3.asm}

\IFRU{Практически, то же самое. Единственное что стоит отметить это то, что GCC решил зачем-то объединить 
вычисление \TT{extended\_model\_id} и \TT{extended\_family\_id} в один блок, 
вместо того чтобы вычислять их перед соответствующим вызовом \printf.}
{Almost the same.
The only thing worth noting is the GCC somehow united calculation of
\TT{extended\_model\_id} and \TT{extended\_family\_id} into one block,
instead of calculating them separately, before corresponding each \printf call.}

\subsubsection{\WorkingWithFloatAsWithStructSubSubSectionName}
\label{sec:floatasstruct}

\IFRU{Как уже раннее указывалось в секции о FPU~(\ref{sec:FPU}), и \Tfloat и \Tdouble содержат в себе знак, 
мантиссу и экспоненту. 
Однако, можем ли мы работать с этими полями напрямую? Попробуем с \Tfloat.}
{As it was already noted in section about FPU~(\ref{sec:FPU}), both \Tfloat and \Tdouble types consisted of sign,
significand (or fraction) and exponent.
But will we able to work with these fields directly? Let's try with \Tfloat.}

\bigskip
% a hack used here! http://tex.stackexchange.com/questions/73524/bytefield-package
\begin{center}
\begin{bytefield}{32}
	\bitheader[endianness=big]{0,22,23,30,31} \\
	\bitbox{1}{S} & 
	\bitbox{8}{\IFRU{экспонента}{exponent}} & 
	\bitbox{23}{\IFRU{мантисса}{mantissa or fraction}}
\end{bytefield}
\end{center}

\begin{center}
( S\EMDASH{}\IFRU{знак}{sign} )
\end{center}

\lstinputlisting{patterns/15_structs/float_en.c}

\IFRU{Структура \TT{float\_as\_struct} занимает в памяти столько же места сколько и \Tfloat, 
то есть 4 байта или 32 бита.}
{\TT{float\_as\_struct} structure occupies as much space is memory as \Tfloat, e.g., 4 bytes or 32 bits.}

\IFRU{Далее мы выставляем во входящем значении отрицательный знак, 
а также прибавляя двойку к экспоненте, мы тем 
самым умножаем всё значение на \TT{$2^2$}, то есть на 4.}
{Now we setting negative sign in input value and also by adding 2 to exponent we thereby multiplicating
the whole number by \TT{$2^2$}, e.g., by 4.}

\IFRU{Компилируем в MSVC 2008 без оптимизации:}{Let's compile in MSVC 2008 without optimization:}

\lstinputlisting[caption=\NonOptimizing MSVC 2008]{patterns/15_structs/float_msvc_\LANG.asm}

\IFRU{Слегка избыточно. В версии скомпилированной с флагом \Ox нет вызовов \TT{memcpy()}, 
там работа происходит сразу с переменной f. Но по неоптимизированной версии будет проще понять.}
{Redundant for a bit.
If it is compiled with \Ox flag there is no \TT{memcpy()} call,
\TT{f} variable is used directly.
But it is easier to understand it all considering unoptimized version.}

\IFRU{А что сделает GCC 4.4.1 с опцией \Othree?}{What GCC 4.4.1 with \Othree will do?}

\lstinputlisting[caption=\Optimizing GCC 4.4.1]{patterns/15_structs/float_gcc_O3_\LANG.asm}

\IFRU{Да, функция \TT{f()} в целом понятна. Однако, что интересно, еще при компиляции, 
не взирая на мешанину с полями структуры, GCC умудрился вычислить результат функции \TT{f(1.234)} и 
сразу подставить его в аргумент для \printf{}!}
{The \TT{f()} function is almost understandable. However, what is interesting, GCC was able to calculate
\TT{f(1.234)} result during compilation stage despite all this hodge-podge with structure fields
and prepared this argument to the \printf{} as precalculated!}





\ifx\LITE\undefined
\chapter{\RU{Объединения (union)}\EN{Unions}}

\EN{\CCpp \IT{union} is mostly used for interpreting a variable (or memory block) of one data type as a variable of another data type.}
\RU{\IT{union} в \CCpp используется в основном для интерпертации переменной (или блока памяти) одного типа как переменной другого типа.}

% sections
\section{\RU{Пример генератора случайных чисел}\EN{Pseudo-random number generator example}}
\label{FPU_PRNG}

\RU{Если нам нужны случайные значения с плавающей запятой в интервале от 0 до 1, самое простое это взять
\ac{PRNG} вроде Mersenne twister.
Он выдает случайные 32-битные числа в виде DWORD.
Затем мы можем преобразовать это число в \Tfloat и затем разделить на \TT{RAND\_MAX} (\TT{0xFFFFFFFF} в данном случае)\EMDASH{}
полученное число будет в интервале от 0 до 1.}
\EN{If we need float random numbers between 0 and 1, the simplest thing is to use a \ac{PRNG} like
the Mersenne twister. 
It produces random 32-bit values in DWORD form. 
Then we can transform this value to \Tfloat and then
divide it by \TT{RAND\_MAX} (\TT{0xFFFFFFFF} in our case)\EMDASH{}
we getting a value in the 0..1 interval.}

\RU{Но как известно, операция деления\EMDASH{}это медленная операция. 
Да и вообще хочется избежать лишних операций с FPU.
Сможем ли мы избежать деления?}
\EN{But as we know, division is slow.
Also, we would like to issue as few FPU operations as possible.
Can we get rid of the division?}

\index{IEEE 754}
\RU{Вспомним состав числа с плавающей запятой: это бит знака, биты мантиссы и биты экспоненты. 
Для получения случайного числа, нам нужно просто заполнить случайными битами все биты мантиссы!}
\EN{Let's recall what a floating point number consists of: sign bit, significand bits and exponent bits.
We just need to store random bits in all significand bits to get a random float number!}

\RU{Экспонента не может быть нулевой (иначе число с плавающей точкой будет денормализованным), 
так что в эти биты мы запишем \TT{01111111}\EMDASH{}
это будет означать что экспонента равна единице. Далее заполняем мантиссу случайными битами, 
знак оставляем в виде 0 (что значит наше число положительное), и вуаля. 
Генерируемые числа будут в интервале от 1 до 2, так что нам еще нужно будет отнять единицу.}
\EN{The exponent cannot be zero (the floating number is denormalized in this case), so we are storing \TT{01111111} 
to exponent\EMDASH{}this means that the exponent is 1. 
Then we filling the significand with random bits, set the sign bit to
0 (which means a positive number) and voilà.
The generated numbers is to be between 1 and 2, so we must also subtract 1.}

\newcommand{\URLXOR}{\url{http://go.yurichev.com/17308}}

\RU{В моем примере\footnote{идея взята здесь: \URLXOR} 
применяется очень простой линейный конгруэнтный генератор случайных чисел, выдающий 32-битные числа.
Генератор инициализируется текущим временем в стиле UNIX.}
\EN{A very simple linear congruential random numbers generator is used in my 
example\footnote{the idea was taken from: \URLXOR}, it produces 32-bit numbers. 
The \ac{PRNG} is initialized with the current time in UNIX timestamp format.}

\RU{Далее, тип \Tfloat представляется в виде \IT{union}\EMDASH{}это конструкция \CCpp позволяющая 
интерпретировать часть памяти по-разному. В нашем случае, мы можем создать переменную типа \TT{union} 
и затем обращаться к ней как к \Tfloat или как к \IT{uint32\_t}. Можно сказать, что это хак, причем грязный.}
\EN{Here we represent the \Tfloat type as an \IT{union}\EMDASH{}it is the \CCpp construction that enables us
to interpret a piece of memory as different types.
In our case, we are able to create a variable
of type \TT{union} and then access to it as it is \Tfloat or as it is \IT{uint32\_t}. 
It can be said, it is just a hack. A dirty one.}

% WTF?
\RU{Код целочисленного \ac{PRNG} точно такой же, как мы уже рассматривали ранее:}
\EN{The integer \ac{PRNG} code is the same as we already considered:} \myref{LCG_simple}.
\RU{Так что и в скомпилированном виде этот код будет опущен.}
\EN{So this code in compiled form is omitted.}

\lstinputlisting{patterns/17_unions/FPU_PRNG/FPU_PRNG.cpp.\LANG}

\subsection{x86}

\lstinputlisting[caption=\Optimizing MSVC 2010]{patterns/17_unions/FPU_PRNG/MSVC2010_Ox_Ob0.asm.\LANG}

\EN{Function names are so strange here because this example was compiled as C++ and this is name mangling in C++,
we will talk about it later:}%
\RU{Имена функций такие странные, потому что этот пример был скомпилирован как Си++, и это манглинг имен в Си++, 
мы будем рассматривать это позже:} \myref{namemangling}.

\RU{Если скомпилировать это в MSVC 2012, компилятор будет использовать SIMD-инструкции для FPU, читайте об этом
здесь:}
\EN{If we compile this in MSVC 2012, it uses the SIMD instructions for the FPU, read more about it here:}
\myref{FPU_PRNG_SIMD}.

\subsection{MIPS}

\lstinputlisting[caption=\Optimizing GCC 4.4.5]{patterns/17_unions/FPU_PRNG/MIPS_O3_IDA.lst.\LANG}

\EN{There is also an useless LUI instruction added for some weird reason.}
\RU{Здесь снова зачем-то добавлена инструкция LUI, которая ничего не делает.}
\EN{We considered this artifact earlier:}
\RU{Мы уже рассматривали этот артефакт ранее:} \myref{MIPS_FPU_LUI}.

\subsection{ARM (\ARMMode)}

\lstinputlisting[caption=\Optimizing GCC 4.6.3 (IDA)]{patterns/17_unions/FPU_PRNG/raspberry_GCC_O3_IDA.lst.\LANG}

\index{objdump}
\index{binutils}
\index{IDA}
\RU{Мы также сделаем дамп в objdump и увидим что FPU-инструкции имеют немного другие имена чем в \IDA.}%
\EN{We'll also make a dump in objdump and we'll see that the FPU instructions have different names than in \IDA.}
\EN{Apparently, IDA and binutils developers used different manuals?}
\RU{Наверное, разработчики IDA и binutils пользовались разной документацией?}
\EN{Perhaps, it would be good to know both instruction name variants.}
\RU{Должно быть, будет полезно знать оба варианта названий инструкций.}

\lstinputlisting[caption=\Optimizing GCC 4.6.3 (objdump)]{patterns/17_unions/FPU_PRNG/raspberry_GCC_O3_objdump.lst}

\EN{The instructions at 0x5c in float\_rand() and at 0x38 in main() are random noise.}
\RU{Инструкции по адресам 0x5c в float\_rand() и 0x38 в main() это случайный мусор.}

\ifdefined\RUSSIAN
\else
\section{Calculating machine epsilon}

\subsection{x86}

Machine epsilon is a smallest possible granule \ac{FPU} can work with\RU{\footnote{В русскоязычной
литературе встречается также термин ``машинный ноль''.}}.
The more bits allocated for floating point number, the smaller machine epsilon.
It is $2^{-23} = 1.19e-07$ for \Tfloat and $2^{-52} = 2.22e-16$ for double.

It's interesting, how easy it's possible to calculate machine epsilon:

\lstinputlisting{patterns/17_unions/epsilon/float.c}

What we do here is just treating fraction part of IEE 754 number as integer and adding 1 to it.
Resulting number will be $starting\_value+machine\_epsilon$, so we just need to subtract
starting value (using floating point arithmetics) to measure, what number one bit reflects
in the single precision (\Tfloat).

union serves here as a way to access IEEE 754 number as a regular integer.
Adding 1 to it is in fact adds 1 to \IT{fraction} part of number, however, needless to say,
overflow is possible, which will add yet another bit to exponent part.

\lstinputlisting[caption=\Optimizing MSVC 2010]{patterns/17_unions/epsilon/float_MSVC_2010_Ox.asm}

Second FST instruction is redundant: there are no need to store input value to the same
place (compiler decided to allocate $v$ variable at the same point of local stack as input 
argument).

Then it is incremented with INC, as it is usual integer variable.
Then it is loaded into FPU as it is 32-bit IEEE 754 number, FSUBR do the job and resulting
value is in the ST0.

Two last FSTP/FLD instruction pair is redundant, but compiler didn't optimized them.

\ifdefined\IncludeARM
\subsection{ARM64}

Let's extend our example to 64-bit:

\lstinputlisting[label=machine_epsilon_double_c]{patterns/17_unions/epsilon/double.c}

ARM64 has no instruction which can add a number to FPU D-register, 
so input value (came in D0) is first copied into GPR,
incremented, copied to FPU register D1, then subtraction occurred.

\lstinputlisting[caption=\Optimizing GCC 4.9 ARM64]{patterns/17_unions/epsilon/double_GCC49_ARM64_O3.s}

See also this example compiled for x64 with SIMD instructions: \ref{machine_epsilon_x64_and_SIMD}.
\fi

\subsection{Conclusion}

It's hard to say, whether someone will need this trickery in real-world code, 
but as I write many times in this book, this example is serving well 
for explaining IEEE 754 format and union feature of \CCpp.
\fi


\section{\RU{Быстрое вычисление квадратного корня}\EN{Fast square root calculation}}

\RU{Вот где еще можно на практике применить трактовку типа \Tfloat как целочисленного, это быстрое вычисление квадратного корня.}%
\EN{Another well-known algorithm where \Tfloat is interpreted as integer is fast calculation of square root.}

\begin{lstlisting}[caption=\EN{The source code is taken from Wikipedia}\RU{Исходный код взят из Wikipedia}: \url{http://go.yurichev.com/17364}]
/* Assumes that float is in the IEEE 754 single precision floating point format
 * and that int is 32 bits. */
float sqrt_approx(float z)
{
    int val_int = *(int*)&z; /* Same bits, but as an int */
    /*
     * To justify the following code, prove that
     *
     * ((((val_int / 2^m) - b) / 2) + b) * 2^m = ((val_int - 2^m) / 2) + ((b + 1) / 2) * 2^m)
     *
     * where
     *
     * b = exponent bias
     * m = number of mantissa bits
     *
     * .
     */
 
    val_int -= 1 << 23; /* Subtract 2^m. */
    val_int >>= 1; /* Divide by 2. */
    val_int += 1 << 29; /* Add ((b + 1) / 2) * 2^m. */
 
    return *(float*)&val_int; /* Interpret again as float */
}
\end{lstlisting}

\RU{В качестве упражнения, вы можете попробовать скомпилировать эту функцию и разобраться, как она работает.}
\EN{As an exercise, you can try to compile this function and to understand, how it works.}\ESph{}\PTBRph{}\PLph{}\ITAph{}\\
\\
\RU{Имеется также известный алгоритм быстрого вычисления}\EN{There is also well-known algorithm of fast calculation of} $\frac{1}{\sqrt{x}}$.
\index{Quake III Arena}
\RU{Алгоритм стал известным, вероятно потому, что был применен в Quake III Arena.}%
\EN{Algorithm became popular, supposedly, because it was used in Quake III Arena.}

\RU{Описание алгоритма есть в}\EN{Algorithm description is present in} Wikipedia:
\EN{\url{http://go.yurichev.com/17360}}\RU{\url{http://go.yurichev.com/17361}}.


\newcommand{\comp}{\TT{comp()}\xspace}
\chapter{\RU{Указатели на функции}\EN{Pointers to functions}}
\label{sec:pointerstofunctions}

\index{\CLanguageElements!\Pointers}
\RU{Указатель на функцию, в целом, как и любой другой указатель, просто адрес указывающий на начало функции 
в сегменте кода.}
\EN{Pointer to function, as any other pointer, is just an address of function beginning in its code segment.}

\index{Callbacks}
\RU{Это применяется часто в т.н. callback-ах}\EN{It is often used in callbacks}
\footnote{\url{http://en.wikipedia.org/wiki/Callback_(computer_science)}}.

\RU{Известные примеры:}\EN{Well-known examples are:}

\begin{itemize}
\item
\qsort\footnote{\url{http://en.wikipedia.org/wiki/Qsort_(C_standard_library)}},
{\TT{atexit()}}\footnote{\url{http://www.opengroup.org/onlinepubs/009695399/functions/atexit.html}} \RU{из стандартной библиотеки Си}\EN{from the standard C library}; 

\item
\RU{сигналы в *NIX ОС}\EN{*NIX OS signals}\footnote{\url{http://en.wikipedia.org/wiki/Signal.h}};

\item
\RU{запуск тредов}\EN{thread starting}: \TT{CreateThread()} (win32), \TT{pthread\_create()} (POSIX);

\item
\RU{множество функций win32, например}\EN{a lot of win32 functions, e.g.} \TT{EnumChildWindows()}\footnote{\url{http://msdn.microsoft.com/en-us/library/ms633494(VS.85).aspx}}.

\item
\EN{a lot of places in Linux kernel, for example, filesystem driver functions are called via
callbacks}\RU{множество мест в ядре Linux, например, ф-ции драйверов файловой системы вызываются
через callback-и}: 
\url{http://lxr.free-electrons.com/source/include/linux/fs.h?v=3.14\#L1525}

\item
\EN{GCC plugin functions are also called via callbacks}\RU{ф-ции плагинов GCC также вызываются
через callback-и}: 
\url{https://gcc.gnu.org/onlinedocs/gccint/Plugin-API.html\#Plugin-API}

\ifdefined\RUSSIAN
\else
\item
One example of function pointers is a table in ``dwm'' Linux window manager, 
consisting of shortcuts. 
Each shortcut has corresponding function to call if the specific key has been pressed:\\
\url{https://github.com/cdown/dwm/blob/master/config.def.h#L117}\\
As one may see, such table is much more easier to handle then large switch() statement.
\fi
\end{itemize}

\index{\CStandardLibrary!qsort()}
\RU{Итак, функция \qsort это реализация алгоритма ``быстрой сортировки''. 
Функция может сортировать что угодно, 
любые типы данных, но при условии, что вы имеете функцию сравнения этих двух элементов данных и 
\qsort может вызывать её.}
\EN{So, \qsort function is a \CCpp standard library quicksort implementation. 
The functions is able to sort anything, any types of data, 
as long as you have a function for these two elements comparison, 
and \qsort is able to call it.}

\RU{Эта функция сравнения может определяться так:}\EN{The comparison function can be defined as:}

\begin{lstlisting}
int (*compare)(const void *, const void *)
\end{lstlisting}

\RU{Воспользуемся немного модифицированным примером, который я нашел вот}
\EN{Let's use slightly modified example I found} \href{http://cplus.about.com/od/learningc/ss/pointers2_8.htm}
{\RU{здесь}\EN{here}}:

\lstinputlisting[numbers=left,label=qsort_c_src]{patterns/18_pointers_to_functions/17_1.c}

\section{MSVC}

\RU{Компилируем в MSVC 2010 (я убрал некоторые части для краткости) с опцией \Ox}
\EN{Let's compile it in MSVC 2010 (I omitted some parts for the sake of brevity) with \Ox option}:

\lstinputlisting[caption=\Optimizing MSVC 2010: /GS- /MD]{patterns/18_pointers_to_functions/17_2_msvc_Ox.asm}

\RU{Ничего особо удивительного здесь мы не видим. В качестве четвертого аргумента, 
в \qsort просто передается адрес метки \TT{\_comp}, где собственно и располагается функция \comp,
или, можно сказать, самая первая инструкция этой ф-ции.}
\EN{Nothing surprising so far.
As a fourth argument, an address of label \TT{\_comp} is passed, that is just a place
where function \comp located, or, in other words, address of the very first instruction of 
this function.}

\RU{Как \qsort вызывает её?}\EN{How \qsort calling it?}

\index{Windows!MSVCR80.DLL}
\RU{Посмотрим в MSVCR80.DLL (эта DLL куда в MSVC вынесены функции из стандартных библиотек Си):}
\EN{Let's take a look into this function located in MSVCR80.DLL (a MSVC DLL module with C standard library functions):}

\lstinputlisting[caption=MSVCR80.DLL]{patterns/18_pointers_to_functions/17_3_MSVCR.lst}

\TT{comp}\EMDASH{}\RU{это четвертый аргумент функции. 
Здесь просто передается управление по адресу указанному в \TT{comp}. 
Перед этим подготавливается два аргумента для функции \comp. 
Далее, проверяется результат её выполнения.}
\EN{is fourth function argument.
Here the control is just passed to the address in the \TT{comp} argument.
Before it, two arguments prepared for \comp. Its result is checked after its execution.}

\RU{Вот почему использование указателей на функции ~--- это опасно. 
Во-первых, если вызвать \qsort с неправильным указателем на функцию, 
то \qsort, дойдя до этого вызова, может передать управление неизвестно куда, 
процесс упадет, и эту ошибку можно будет найти не сразу.}
\EN{That's why it is dangerous to use pointers to functions.
First of all, if you call \qsort with incorrect pointer to function, \qsort may pass control
to incorrect point, a process may crash and this bug will be hard to find.}

\RU{Во-вторых, типизация callback-функции должна строго соблюдаться, 
вызов не той функции с не теми аргументами не того типа, 
может привести к плачевным результатам, 
хотя падение процесса это и не проблема, проблема ~--- это найти ошибку, ведь компилятор 
на стадии компиляции может вас и не предупредить о потенциальных неприятностях.}
\EN{Second reason is the callback function types must comply strictly, calling wrong function
with wrong arguments of wrong types may lead to serious problems, however, process crashing is not a 
big problem~---big problem is to determine a reason of crashing~---because compiler may be 
silent about potential trouble while compiling.}

\ifdefined\IncludeOlly
\clearpage
\subsection{MSVC + \olly}
\index{\olly}

\RU{Загрузим наш пример в \olly и установим брякпойнт на ф-ции \comp}
\EN{Let's load our example into \olly and set breakpoint on \comp function}.

\RU{Как значения сравниваются, мы можем увидеть во время самого первого вызова \comp}
\EN{How values are compared we can see at the very first \comp call}:

\begin{figure}[H]
\centering
\includegraphics[scale=\FigScale]{patterns/18_pointers_to_functions/olly1.png}
\caption{\olly: \RU{первый вызов}\EN{first call of} \comp}
\label{fig:qsort_olly1}
\end{figure}

\RU{Для удобства, }\olly \RU{показывает сравниваемые значения в окне под окном кода}
\EN{shows compared values in the window under code window, for convenience}.
\RU{Мы можем так же увидеть что}\EN{We can also see that the} \ac{SP} \RU{указывает на}\EN{pointing to} 
\ac{RA} \RU{где находится место в ф-ции}\EN{where the place in} 
\qsort \EN{function is }(\RU{на самом деле, находится в}\EN{actually located in} \TT{MSVCR100.DLL}).

\clearpage
\RU{Трассируя}\EN{By tracing} (F8) \RU{до инструкции}\EN{until} \TT{RETN} 
\RU{и нажав F8 еще один раз, мы возвращаемся в ф-цию}\EN{instruction, and pressing F8 one more time, 
we returning into} \qsort\EN{ function}:

\begin{figure}[H]
\centering
\includegraphics[scale=\FigScale]{patterns/18_pointers_to_functions/olly2.png}
\caption{\olly: \RU{код в}\EN{the code in} \qsort \RU{сразу после вызова}\EN{right after} \comp\EN{ call}}
\label{fig:qsort_olly2}
\end{figure}

\RU{Это был вызов ф-ции сравнения}\EN{That was a call to comparison function}.

\clearpage
\RU{Вот также скриншот момента второго вызова ф-ции}\EN{Here is also screenshot of the moment of the 
second call of} \comp\EMDASH{}\RU{теперь сравниваемые значения другие}
\EN{now values to be compared are different}:

\begin{figure}[H]
\centering
\includegraphics[scale=\FigScale]{patterns/18_pointers_to_functions/olly3.png}
\caption{\olly: \RU{второй вызов}\EN{second call of} \comp}
\label{fig:qsort_olly3}
\end{figure}

\subsection{MSVC + tracer}
\index{tracer}

\RU{Посмотрим, какие пары сравниваются}\EN{Let's also see, which pairs are compared}.
\RU{Эти 10 чисел будут сортироваться}\EN{These 10 numbers are being sorted}: 
1892, 45, 200, -98, 4087, 5, -12345, 1087, 88, -100000.

\RU{Я нашел адрес первой инструкции}\EN{I found the address of the first} \CMP 
\RU{в}\EN{instruction in} \comp, \RU{и это}\EN{it is} \TT{0x0040100C} 
\RU{и я ставлю брякпойнт на нем}\EN{and I'm setting breakpoint on it}:

\begin{lstlisting}
tracer.exe -l:17_1.exe bpx=17_1.exe!0x0040100C
\end{lstlisting}

\RU{Получаю информацию о регистрах на брякпойнте}
\EN{I'm getting information about registers at breakpoint}:

\begin{lstlisting}
PID=4336|New process 17_1.exe
(0) 17_1.exe!0x40100c
EAX=0x00000764 EBX=0x0051f7c8 ECX=0x00000005 EDX=0x00000000
ESI=0x0051f7d8 EDI=0x0051f7b4 EBP=0x0051f794 ESP=0x0051f67c
EIP=0x0028100c
FLAGS=IF
(0) 17_1.exe!0x40100c
EAX=0x00000005 EBX=0x0051f7c8 ECX=0xfffe7960 EDX=0x00000000
ESI=0x0051f7d8 EDI=0x0051f7b4 EBP=0x0051f794 ESP=0x0051f67c
EIP=0x0028100c
FLAGS=PF ZF IF
(0) 17_1.exe!0x40100c
EAX=0x00000764 EBX=0x0051f7c8 ECX=0x00000005 EDX=0x00000000
ESI=0x0051f7d8 EDI=0x0051f7b4 EBP=0x0051f794 ESP=0x0051f67c
EIP=0x0028100c
FLAGS=CF PF ZF IF
...
\end{lstlisting}

\RU{Я отфильтровал}\EN{I filtered out} \TT{EAX} \AndENRU \TT{ECX} \RU{и получил}\EN{and got}:

\begin{lstlisting}
EAX=0x00000764 ECX=0x00000005
EAX=0x00000005 ECX=0xfffe7960
EAX=0x00000764 ECX=0x00000005
EAX=0x0000002d ECX=0x00000005
EAX=0x00000058 ECX=0x00000005
EAX=0x0000043f ECX=0x00000005
EAX=0xffffcfc7 ECX=0x00000005
EAX=0x000000c8 ECX=0x00000005
EAX=0xffffff9e ECX=0x00000005
EAX=0x00000ff7 ECX=0x00000005
EAX=0x00000ff7 ECX=0x00000005
EAX=0xffffff9e ECX=0x00000005
EAX=0xffffff9e ECX=0x00000005
EAX=0xffffcfc7 ECX=0xfffe7960
EAX=0x00000005 ECX=0xffffcfc7
EAX=0xffffff9e ECX=0x00000005
EAX=0xffffcfc7 ECX=0xfffe7960
EAX=0xffffff9e ECX=0xffffcfc7
EAX=0xffffcfc7 ECX=0xfffe7960
EAX=0x000000c8 ECX=0x00000ff7
EAX=0x0000002d ECX=0x00000ff7
EAX=0x0000043f ECX=0x00000ff7
EAX=0x00000058 ECX=0x00000ff7
EAX=0x00000764 ECX=0x00000ff7
EAX=0x000000c8 ECX=0x00000764
EAX=0x0000002d ECX=0x00000764
EAX=0x0000043f ECX=0x00000764
EAX=0x00000058 ECX=0x00000764
EAX=0x000000c8 ECX=0x00000058
EAX=0x0000002d ECX=0x000000c8
EAX=0x0000043f ECX=0x000000c8
EAX=0x000000c8 ECX=0x00000058
EAX=0x0000002d ECX=0x000000c8
EAX=0x0000002d ECX=0x00000058
\end{lstlisting}

\RU{Это}\EN{That's} 34 \RU{пары}\EN{pairs}.
\RU{Следовательно, алгоритму быстрой сортировки нужно 34 операции сравнения для сортировки этих
10-и чисел}\EN{Therefore, quick sort algorithm needs 34 comparison operations for sorting these 10 numbers}.

\clearpage
\subsection{MSVC + tracer (code coverage)}
\index{tracer}

\RU{Но можно также и воспользоваться возможностью tracer накапливать все возможные состояния регистров
и показать их в \IDA}\EN{We can also use tracer's feature to collect all possible register's values
and show them in \IDA}.

\RU{Трассируем все инструкции в ф-ции \comp}\EN{Let's trace all instructions in \comp function}:

\begin{lstlisting}
tracer.exe -l:17_1.exe bpf=17_1.exe!0x00401000,trace:cc
\end{lstlisting}

\RU{Получем .idc-скрипт для загрузки в \IDA и загружаем его}
\EN{We getting .idc-script for loading into \IDA and load it}:

\begin{figure}[H]
\centering
\includegraphics[scale=\FigScale]{patterns/18_pointers_to_functions/tracer_cc.png}
\caption{tracer \AndENRU IDA. N.B.: 
\RU{некоторые значения обрезаны справа}\EN{some values are cutted at right}}
\label{fig:qsort_tracer_cc}
\end{figure}

\RU{Имя этой ф-ции (PtFuncCompare) дала \IDA}\EN{\IDA gave the function name (PtFuncCompare)}
\EMDASH{}\RU{видимо, потому что видит что указатель на эту ф-цию передается в \qsort}\EN{it seems,
because \IDA sees that pointer to this function is passed into \qsort}.

\RU{Мы видим что указатели $a$ и $b$ указывают на разные места внутри массива, 
но шаг между указателями --- 4, что логично, ведь в массиве хранятся 32-битные значения}
\EN{We see that $a$ and $b$ pointers are pointing to various places in array, but step between
points is 4---indeed, 32-bit values are stored in the array}.

\RU{Видно что инструкции по адресам}\EN{We see that the instructions at} \TT{0x401010} \AndENRU 
\TT{0x401012} \RU{никогда не исполнялись}\EN{was never executed} 
(\RU{они и остались белыми}\EN{so they leaved as white}): 
\RU{действительно, ф-ция}\EN{indeed,} \comp \RU{никогда не возвращала 0,
потому что в массиве нет одинаковых элементов}\EN{was never returned 0, because there no equal elements in array}.

\fi

\section{GCC}

\RU{Не слишком большая разница}\EN{Not a big difference}:

\begin{lstlisting}[caption=GCC]
                lea     eax, [esp+40h+var_28]
                mov     [esp+40h+var_40], eax
                mov     [esp+40h+var_28], 764h
                mov     [esp+40h+var_24], 2Dh
                mov     [esp+40h+var_20], 0C8h
                mov     [esp+40h+var_1C], 0FFFFFF9Eh
                mov     [esp+40h+var_18], 0FF7h
                mov     [esp+40h+var_14], 5
                mov     [esp+40h+var_10], 0FFFFCFC7h
                mov     [esp+40h+var_C], 43Fh
                mov     [esp+40h+var_8], 58h
                mov     [esp+40h+var_4], 0FFFE7960h
                mov     [esp+40h+var_34], offset comp
                mov     [esp+40h+var_38], 4
                mov     [esp+40h+var_3C], 0Ah
                call    _qsort
\end{lstlisting}

\RU{Функция \comp}\EN{\comp function}:

\begin{lstlisting}
                public comp
comp            proc near

arg_0           = dword ptr  8
arg_4           = dword ptr  0Ch

                push    ebp
                mov     ebp, esp
                mov     eax, [ebp+arg_4]
                mov     ecx, [ebp+arg_0]
                mov     edx, [eax]
                xor     eax, eax
                cmp     [ecx], edx
                jnz     short loc_8048458
                pop     ebp
                retn
loc_8048458:
                setnl   al
                movzx   eax, al
                lea     eax, [eax+eax-1]
                pop     ebp
                retn
comp            endp
\end{lstlisting}

\index{Linux!libc.so.6}
\RU{Реализация \qsort находится в \TT{libc.so.6}, и представляет собой просто wrapper
\footnote{понятие близкое к \gls{thunk function}} для \TT{qsort\_r()}.}
\EN{\qsort implementation is located in the \TT{libc.so.6} and it is in fact just a wrapper
\footnote{a concept like \gls{thunk function}} for \TT{qsort\_r()}.}

\RU{Она, в свою очередь, вызывает \TT{quicksort()}, где есть вызовы определенной нами функции через 
переданный указатель:}
\EN{It will call then \TT{quicksort()}, where our defined function will be called via passed pointer:}

\begin{lstlisting}[caption=
(\RU{файл libc.so.6{,} версия glibc}\EN{file libc.so.6{,} glibc version}\EMDASH{}2.10.1)]

.text:0002DDF6                 mov     edx, [ebp+arg_10]
.text:0002DDF9                 mov     [esp+4], esi
.text:0002DDFD                 mov     [esp], edi
.text:0002DE00                 mov     [esp+8], edx
.text:0002DE04                 call    [ebp+arg_C]
...
\end{lstlisting}

\ifdefined\IncludeGDB
\subsection{GCC + GDB (\RU{с исходными кодами}\EN{with source code})}
\index{GDB}

\RU{Очевидно, у нас есть исходный код нашего примера на Си (\ref{qsort_c_src}), 
так что мы можем установить брякпойнт ($b$) на
номере строки}\EN{Obviously, we have a C-source code of our example (\ref{qsort_c_src}), 
so we can set breakpoint ($b$) on line number}
(\RU{11-й --- это номер строки где происходит первое сравнение}\EN{11th---the line where 
first comparison is occurred}).
\RU{Нам также нужно скомпилировать наш пример с ключом \TT{-g}, чтобы в исполняемом файле была
полная отладочная информация}\EN{We also need to compile example with debugging information 
included (\TT{-g}), so the table
with addresses and corresponding line numbers is present}.
\RU{Мы можем так же выводить значения используя имена переменных}
\EN{We can also print values by variable name} (\TT{p}):
\RU{отладочная информация также содержит информацию о том, в каком регистре и/или элементе локального
стека находится какая переменная}\EN{debugging information also has information about which register and/or 
local stack element contain which variable}.

\index{Glibc}
\RU{Мы можем также увидеть стек}\EN{We can also see stack} (\TT{bt}) 
\RU{и обнаружить что в Glibc используется какая-то вспомогательная ф-ция с именем}
\EN{and find out that there are some intermediate function} 
\TT{msort\_with\_tmp()}\EN{ used in Glibc}.

\lstinputlisting[caption=GDB\RU{-сессия}\EN{ session}]{patterns/18_pointers_to_functions/GDB_source.txt}

\subsection{GCC + GDB (\RU{без исходных кодов}\EN{no source code})}
\index{GDB}

\RU{Но часто никаких исходных кодов нет вообще, так что мы можем дизассемблировать ф-цию \comp}
\EN{But often there are no source code at all, so we can disassemble \comp function} (\TT{disas}), 
\RU{найти самую первую инструкцию \CMP и установить брякпойнт}\EN{find the very first
\CMP instruction and set breakpoint} ($b$) \RU{по этому адресу}\EN{at that address}.
\RU{На каждом брякпойнте мы будем видеть содержимое регистров}
\EN{At each breakpoint, we will dump all register contents} (\TT{info registers}).
\RU{Информация из стека так же доступна}\EN{Stack information is also available} (\TT{bt}), 
\RU{но частичная: здесь нет номеров строк для ф-ции \comp}
\EN{but partial: there are no line number information for \comp function}.

\lstinputlisting[caption=GDB\RU{-сессия}\EN{ session}]{patterns/18_pointers_to_functions/GDB_no_source.txt}
\fi

\fi
\ifdefined\ENGLISH
\section{64-bit values in 32-bit environment}
\label{sec:64bit_in_32_env}

In a 32-bit environment, \ac{GPR}'s are 32-bit, so 64-bit values are stored and passed as 32-bit value pairs
\footnote{By the way, 32-bit values are passed as pairs in 16-bit environment in the same way: \myref{win16_32bit_values}}.
\fi

\ifdefined\RUSSIAN
\section{64-битные значения в 32-битной среде}
\label{sec:64bit_in_32_env}

В среде, где \ac{GPR}-ы 32-битные, 64-битные значения хранятся и передаются как пары 32-битных значений
\footnote{Кстати, в 16-битной среде, 32-битные значения передаются 16-битными парами точно так же: \myref{win16_32bit_values}}.
\fi

\ifdefined\GERMAN
\section{64-Bit-Werte in 32-Bit-Umgebungen}
\label{sec:64bit_in_32_env}

In einer 32-Bit-Umgebung sind \ac{GPR} 32 Bit groß. Also werden 64-Bit-Werte in
32-Bit-Wertepaaren gespeichert und übergeben\footnote{Übrigens, 32-Bit-Werte werden
als Paare in 16--Bit-Umgebungen auf der gleiche Art übergeben: \myref{win16_32bit_values}}.
\fi

\EN{\input{patterns/185_64bit_in_32_env/ret/main_EN}}
\RU{\input{patterns/185_64bit_in_32_env/ret/main_RU}}
\DE{\input{patterns/185_64bit_in_32_env/ret/main_DE}}

\EN{\input{patterns/185_64bit_in_32_env/passing_add_sub/main_EN}}
\RU{\input{patterns/185_64bit_in_32_env/passing_add_sub/main_RU}}
\DE{\input{patterns/185_64bit_in_32_env/passing_add_sub/main_DE}}

\section{\RU{Умножение, деление}\EN{Multiplication, division}}

\lstinputlisting{patterns/185_64bit_in_32_env/multdiv/2.c}

\subsection{x86}

\lstinputlisting[caption=\Optimizing MSVC 2013 /Ob1]{patterns/185_64bit_in_32_env/multdiv/2_MSVC.asm.\LANG}

\RU{Умножение и деление --- это более сложная операция, так что обычно, компилятор встраивает вызовы библиотечных функций,
делающих это}\EN{Multiplication and division are more complex operations, so usually the compiler embeds calls to
a library functions doing that}.

\ifx\LITE\undefined
\RU{Значение этих библиотечных функций, здесь}\EN{These functions are described here}: \myref{sec:MSVC_library_func}.
\fi

\ifdefined\IncludeGCC
\lstinputlisting[caption=\Optimizing GCC 4.8.1 -fno-inline]{patterns/185_64bit_in_32_env/multdiv/2_GCC.asm.\LANG}

\RU{GCC делает почти то же самое, тем не менее,
встраивает код умножения прямо в функцию, посчитав что так будет эффективнее}\EN{GCC does the expected, but the multiplication
code is inlined right in the function, thinking it could be more efficient}.
\RU{У GCC другие имена библиотечных функций}\EN{GCC has different library function names}: \myref{sec:GCC_library_func}.
\fi

\ifdefined\IncludeARM
\subsection{ARM}

\RU{Keil для режима Thumb вставляет вызовы библиотечных функций:}
\EN{Keil for Thumb mode inserts library subroutine calls:}

\lstinputlisting[caption=\OptimizingKeilVI (\ThumbMode)]{patterns/185_64bit_in_32_env/multdiv/Keil_thumb_O3.s}

\RU{Keil для режима ARM, тем не менее, может сгенерировать код для умножения 64-битных чисел:}
\EN{Keil for ARM mode, on the other hand, is able to produce 64-bit multiplication code:}

\lstinputlisting[caption=\OptimizingKeilVI (\ARMMode)]{patterns/185_64bit_in_32_env/multdiv/Keil_ARM_O3.s}
% TODO add explanation
\fi

\ifdefined\IncludeMIPS
\subsection{MIPS}

\Optimizing GCC \ForENRU MIPS 
\EN{can generate 64-bit multiplication code, but has to call a library routine for 64-bit division:}
\RU{может генерировать код для 64-битного умножения, но для 64-битного деления приходится вызывать библиотечную функцию:}

\lstinputlisting[caption=\Optimizing GCC 4.4.5 (IDA)]{patterns/185_64bit_in_32_env/multdiv/MIPS_O3_IDA.lst}

\RU{Тут также много \ac{NOP}-ов, это возможно заполнение delay slot-ов после инструкции умножения (она ведь работает
медленнее прочих инструкций).}
\EN{There are a lot of \ac{NOP}s, probably delay slots filled after the multiplication instruction (it's slower
than other instructions, after all).}

% TODO add explanation
\fi

\EN{\input{patterns/185_64bit_in_32_env/shifting/main_EN}}
\RU{\input{patterns/185_64bit_in_32_env/shifting/main_RU}}
\DE{\input{patterns/185_64bit_in_32_env/shifting/main_DE}}
\section{\RU{Конвертирование 32-битного значения в 64-битное}\EN{Converting 32-bit value into 64-bit one}}
\label{subsec:sign_extending_32_to_64}

\lstinputlisting{patterns/185_64bit_in_32_env/conversion/4.c}

\subsection{x86}

\lstinputlisting[caption=\Optimizing MSVC 2012]{patterns/185_64bit_in_32_env/conversion/MSVC2012_Ox.asm}

\RU{Здесь появляется необходимость расширить 32-битное знаковое значение в 64-битное знаковое.}
\EN{Here we also run into necessity to extend 32-bit signed value into 64-bit signed.}
\RU{Конвертировать беззнаковые значения очень просто: нужно просто выставить в 0 все биты в старшей части}
\EN{Unsigned values are converted straightforwardly: all bits in higher part must be set to 0}.
\RU{Но для знаковых типов это не подходит: знак числа должен быть скопирован в старшую часть числа-результата}
\EN{But it is not appropriate for signed data types: sign should be copied into higher part of resulting number}.
\index{x86!\Instructions!CDQ}
\RU{Здесь это делает инструкция \TT{CDQ}, она берет входное значение в \EAX{}, расширяет его до 64-битного,
и оставляет его в паре регистров \EDX{}:\EAX{}}
\EN{\TT{CDQ} instruction doing that here, it takes input value in \EAX{}, extending it to 64-bit and leaving it
in the \EDX{}:\EAX{} registers pair}.
\RU{Иными словами, инструкция \TT{CDQ} узнает знак числа в \EAX{} (просто берет самый старший бит в \EAX{}) и в зависимости от этого,
выставляет все 32 бита в \EDX{} в 0 или в 1}\EN{In other words, \TT{CDQ} instruction gets number sign in \EAX{} (by getting just
most significant bit in \EAX{}), and depending of it, setting all 32-bits in \EDX{} to 0 or 1}.
\RU{Её работа в каком-то смысле напоминает работу инструкции \MOVSX{}}\EN{Its operation is somewhat
similar to the \MOVSX{} instruction}.

\ifdefined\IncludeARM
\subsection{ARM}

\lstinputlisting[caption=\OptimizingKeilVI (\ARMMode)]{patterns/185_64bit_in_32_env/conversion/Keil_ARM_O3.s}

\RU{Keil для ARM работает иначе: он просто сдвигает (арифметически) входное значение на 31 бит вправо.}
\EN{Keil for ARM is different: it just arithmetically shifts input value by 31 bit right.}
\RU{Как мы знаем, бит знака это \ac{MSB}, и арифметический сдвиг копирует бит знака в ``появляющихся'' битах.}
\EN{As we know, sign bit is \ac{MSB}, and arithmetical shift copies sign bit into ``emerged'' bits.}
\RU{Так что после инструкции ``ASR r1,r0,\#31'', R1 будет содержать 0xFFFFFFFF если входное значение
было отрицательным, или 0 в противном случае.}
\EN{So after ``ASR r1,r0,\#31'' instruction, R1 will contain 0xFFFFFFFF if input value was negative
and 0 otherwise.}
\RU{R1 содержит старшую часть возвращаемого 64-битного значения.}
\EN{R1 contain high part of resulting 64-bit value.}

\RU{Другими словами, этот код просто копирует \ac{MSB} (бит знака) из входного значения в R0 во все
биты старшей 32-битной части итогового 64-битного значения.}
\EN{In other words, this code just copies \ac{MSB} (sign bit) from input value in R0 into all bits
of high 32-bit part of resulting 64-bit value.}

\fi



\ifx\LITE\undefined
\section{SIMD}

\label{SIMD_x86}
\ac{SIMD} \IFRU{это акроним:}{is just acronym:} \IT{Single Instruction, Multiple Data}.

\IFRU{Как можно судить по названию, это обработка множества данных исполняя только одну инструкцию.}
{As it is said, it is multiple data processing using only one instruction.}

\IFRU{Как и \ac{FPU}, эта подсистема процессора выглядит также отдельным процессором внутри x86.}
{Just as \ac{FPU}, that \ac{CPU} subsystem looks like separate processor inside x86.}

\index{x86!MMX}
\IFRU{SIMD в x86 начался с MMX. Появилось 8 64-битных регистров MM0-MM7.}
{SIMD began as MMX in x86. 8 new 64-bit registers appeared: MM0-MM7.}

\IFRU{Каждый MMX-регистр может содержать 2 32-битных значения, 4 16-битных или же 8 байт. 
Например, складывая значения двух MMX-регистров, можно складывать одновременно 8 8-битных значений.}
{Each MMX register may hold 2 32-bit values, 4 16-bit values or 8 bytes.
For example, it is possible to add 8 8-bit values (bytes) simultaneously by adding two values in MMX-registers.}

\IFRU{Простой пример, это некий графический редактор, который хранит открытое изображение как двумерный массив. 
Когда пользователь меняет яркость изображения, редактору нужно, например, прибавить некий коэффициент 
ко всем пикселям, или отнять. 
Для простоты можно представить, что изображение у нас бело-серо-черное и каждый пиксель занимает один байт, 
то с помощью MMX можно менять яркость сразу у восьми пикселей.}
{One simple example is graphics editor, representing image as a two dimensional array.
When user change image brightness, the editor must add a coefficient to each pixel value, or to subtract.
For the sake of brevity, our image may be grayscale and each pixel defined by one 8-bit byte, then it is possible
to change brightness of 8 pixels simultaneously.}

\IFRU{Когда MMX только появилось, эти регистры на самом деле располагались в FPU-регистрах. 
Можно было использовать 
либо FPU либо MMX в одно и то же время. Можно подумать, что Intel решило немного сэкономить на транзисторах, 
но на самом деле причина такого симбиоза проще ~--- более старая \ac{OS} не знающая о дополнительных 
регистрах процессора не будет сохранять их во время переключения задач, а вот регистры FPU сохранять будет. 
Таким образом, процессор с MMX + старая \ac{OS} + задача использующая возможности MMX = все 
это может работать вместе.}
{When MMX appeared, these registers was actually located in FPU registers. 
It was possible to use either FPU or MMX at the same time. One might think, Intel saved on transistors,
but in fact, the reason of such symbiosis is simpler~---older \ac{OS} may not aware 
of additional CPU registers would not save them at the context switching, but will save FPU registers.
Thus, MMX-enabled CPU + old \ac{OS} + process utilizing MMX features = that all will work together.}

\index{x86!SSE}
\index{x86!SSE2}
SSE\EMDASH\IFRU{это расширение регистров до 128 бит, теперь уже отдельно от FPU.}{is extension of SIMD registers up to 128 bits, now separately from FPU.}

\index{x86!AVX}
AVX\EMDASH\IFRU{расширение регистров до 256 бит.}{another extension to 256 bits.}

\IFRU{Немного о практическом применении.}{Now about practical usage.}

\IFRU{Конечно же, копирование блоков в памяти (\TT{memcpy}), сравнение (\TT{memcmp}), и подобное.}
{Of course, memory copying (\TT{memcpy}), memory comparing (\TT{memcmp}) and so on.}

\index{DES}
\IFRU{Еще пример: имеется алгоритм шифрования DES, который берет 64-битный блок, 56-битный ключ, 
шифрует блок с ключом и образуется 64-битный результат.
Алгоритм DES можно легко представить в виде очень большой электронной цифровой схемы, 
с проводами, элементами И, ИЛИ, НЕ.}
{One more example: we got DES encryption algorithm, it takes 64-bit block, 56-bit key, encrypt block and produce 64-bit result.
DES algorithm may be considered as a very large electronic circuit, with wires and AND/OR/NOT gates.}

\label{bitslicedes}
\newcommand{\URLBS}{\url{http://www.darkside.com.au/bitslice/}}

\IFRU{Идея bitslice DES\footnote{\URLBS} ~--- это обработка сразу группы блоков и ключей одновременно. 
Скажем, на x86 переменная типа \IT{unsigned int} вмещает в себе 32 бита, так что там можно хранить 
промежуточные результаты сразу для 32-х блоков-ключей, используя 64+56 переменных типа \IT{unsigned int}.}
{Bitslice DES\footnote{\URLBS}~---is an idea of processing group of blocks and keys simultaneously.
Let's say, variable of type \IT{unsigned int} on x86 may hold up to 32 bits, so, it is possible to store there
intermediate results for 32 blocks-keys pairs simultaneously, using 64+56 variables of \IT{unsigned int} type.}

\index{Oracle RDBMS}
\IFRU{Я написал утилиту для перебора паролей/хешей Oracle RDBMS (которые основаны на алгоритме DES), 
переделав алгоритм bitslice DES для SSE2 и AVX ~--- и теперь возможно шифровать одновременно 
128 или 256 блоков-ключей:}
{I wrote an utility to brute-force Oracle RDBMS passwords/hashes (ones based on DES),
slightly modified bitslice DES algorithm for SSE2 and AVX~---now it is possible to encrypt 128 
or 256 block-keys pairs simultaneously.}

\url{http://conus.info/utils/ops_SIMD/}
 
\subsection{\IFRU{Векторизация}{Vectorization}}

\newcommand{\URLVEC}{\href{http://en.wikipedia.org/wiki/Vectorization_(computer_science)}{Wikipedia: vectorization}}

\IFRU{Векторизация\footnote{\URLVEC} это когда у вас есть цикл, который берет на вход несколько массивов и выдает, 
например, один массив данных. 
Тело цикла берет некоторые элементы из входных массивов, что-то делает с ними и помещает в выходной. 
Важно, что операция применяемая ко всем элементам одна и та же. 
Векторизация ~--- это обрабатывать несколько элементов одновременно.}
{Vectorization\footnote{\URLVEC}, for example, is when you have a loop taking couple of arrays at input and produces one array.
Loop body takes values from input arrays, do something and put result into output array.
It is important that there is only one single operation applied to each element.
Vectorization~---is to process several elements simultaneously.}

\IFRU{Векторизация ~--- это не самая новая технология: автор сих строк видел её по крайней мере на 
линейке суперкомпьютеров Cray Y-MP от 1988, когда работал на его версии-``лайт'' Cray Y-MP EL
\footnote{Удаленно. Он находится в музее суперкомпьютеров: \url{http://www.cray-cyber.org}}}
{Vectorization is not very fresh technology: author of this textbook saw it at least on Cray Y-MP 
supercomputer line from 1988 when played with its ``lite'' version Cray Y-MP EL
\footnote{Remotely. It is installed in the museum of supercomputers: \url{http://www.cray-cyber.org}}}.

\IFRU{Например:}{For example:}

\begin{lstlisting}
for (i = 0; i < 1024; i++)
{
    C[i] = A[i]*B[i];
}
\end{lstlisting}

\IFRU{Этот фрагмент кода берет элементы из A и B, перемножает и сохраняет результат в C.}
{This fragment of code takes elements from A and B, multiplies them and save result into C.}

\index{x86!\Instructions!PLMULLD}
\index{x86!\Instructions!PLMULHW}
\newcommand{\PMULLD}{\IT{PMULLD} (\IT{\IFRU{Перемножить запакованные знаковые DWORD и сохранить младшую часть результата}
{Multiply Packed Signed Dword Integers and Store Low Result}})}
\newcommand{\PMULHW}{\TT{PMULHW} (\IT{\IFRU{Перемножить запакованные знаковые DWORD и сохранить старшую часть результата}
{Multiply Packed Signed Integers and Store High Result}})}

\IFRU{Если представить, что каждый элемент массива ~--- это 32-битный \Tint, то их можно загружать сразу 
по 4 из А в 128-битный XMM-регистр, 
из B в другой XMM-регистр и выполнив инструкцию \PMULLD{} и \PMULHW{}, можно получить 4 64-битных 
\glslink{product}{произведения} сразу.}
{If each array element we have is 32-bit \Tint, then it is possible to load 4 elements from A into 128-bit 
XMM-register, from B to another XMM-registers, and by executing \PMULLD{} and \PMULHW{}, 
it is possible to get 4 64-bit \glspl{product} at once.}

\IFRU{Таким образом, тело цикла исполняется $1024/4$ раза вместо 1024, что в 4 раза меньше, и, конечно, быстрее.}
{Thus, loop body count is $1024/4$ instead of $1024$, that is 4 times less and, of course, faster.}

\newcommand{\URLINTELVEC}{\href{http://www.intel.com/intelpress/sum_vmmx.htm}{Excerpt: Effective Automatic Vectorization}}

\index{Intel C++}
\IFRU{Некоторые компиляторы умеют делать автоматическую векторизацию в простых случаях, 
например Intel C++\footnote{Еще о том, как Intel C++ умеет автоматически векторизовать циклы: \URLINTELVEC}.}
{Some compilers can do vectorization automatically in a simple cases, 
e.g., Intel C++\footnote{More about Intel C++ automatic vectorization: \URLINTELVEC}.}

\IFRU{Я написал очень простую функцию:}{I wrote tiny function:}

\begin{lstlisting}
int f (int sz, int *ar1, int *ar2, int *ar3)
{
	for (int i=0; i<sz; i++)
		ar3[i]=ar1[i]+ar2[i];

	return 0;
};
\end{lstlisting}

\subsubsection{Intel C++}

\IFRU{Компилирую при помощи}{Let's compile it with} Intel C++ 11.1.051 win32:

\begin{verbatim}
icl intel.cpp /QaxSSE2 /Faintel.asm /Ox
\end{verbatim}

\IFRU{Имеем такое (в \IDA):}{We got (in \IDA):}

\lstinputlisting{patterns/19_SIMD/18_1_en.asm}

\IFRU{Инструкции, имеющие отношение к SSE2 это:}{SSE2-related instructions are:}
\index{x86!\Instructions!MOVDQA}
\index{x86!\Instructions!MOVDQU}
\index{x86!\Instructions!PADDD}
\begin{itemize}
\item
\MOVDQU (\IT{Move Unaligned Double Quadword})\EMDASH\IFRU{она просто загружает 16 байт из памяти в XMM-регистр}
{it just load 16 bytes from memory into a XMM-register}.

\item
\PADDD (\IT{Add Packed Integers})\EMDASH\IFRU{складывает сразу 4 пары 32-битных чисел и оставляет в первом операнде результат. 
Кстати, если произойдет переполнение, то исключения не произойдет и никакие флаги не установятся, 
запишутся просто младшие 32 бита результата. 
Если один из операндов \PADDD ~--- адрес значения в памяти, 
то требуется чтобы адрес был выровнен по 16-байтной границе. Если он не выровнен, произойдет исключение
\footnote{О выравнивании данных см. также: \URLWPDA}.}
{adding 4 pairs of 32-bit numbers and leaving result in first operand.
By the way, no exception raised in case of overflow and no flags will be set, just low 32-bit of result will
be stored.
If one of \PADDD operands is address of value in memory,
then address must be aligned on a 16-byte boundary. If it is not aligned, exception will be occurred
\footnote{More about data aligning: \URLWPDA}.}

\item
\MOVDQA (\IT{Move Aligned Double Quadword})\EMDASH\IFRU{тоже что и \MOVDQU, только подразумевает 
что адрес в памяти выровнен по 16-байтной границе. 
Если он не выровнен, произойдет исключение. 
\MOVDQA работает быстрее чем \MOVDQU, но требует вышеозначенного.}
{the same as \MOVDQU, but requires address of value in memory to be aligned on a 16-bit border.
If it is not aligned, exception will be raised.
\MOVDQA works faster than \MOVDQU, but requires aforesaid.}

\end{itemize}

\IFRU{Итак, эти SSE2-инструкции исполнятся только в том случае если еще осталось просуммировать 
4 пары переменных типа \Tint плюс если указатель \TT{ar3} выровнен по 16-байтной границе.}
{So, these SSE2-instructions will be executed only in case if there are more 4 pairs to work on
plus pointer \TT{ar3} is aligned on a 16-byte boundary.}

\IFRU{Более того, если еще и \TT{ar2} выровнен по 16-байтной границе, то будет выполняться этот фрагмент кода:}
{More than that, if \TT{ar2} is aligned on a 16-byte boundary as well, this fragment of code will be executed:}

\begin{lstlisting}
movdqu  xmm0, xmmword ptr [ebx+edi*4] ; ar1+i*4
paddd   xmm0, xmmword ptr [esi+edi*4] ; ar2+i*4
movdqa  xmmword ptr [eax+edi*4], xmm0 ; ar3+i*4
\end{lstlisting}

\IFRU{А иначе, значение из \TT{ar2} загрузится в \XMMZERO используя инструкцию \MOVDQU, 
которая не требует выровненного указателя, зато может работать чуть медленнее:}
{Otherwise, value from \TT{ar2} will be loaded into \XMMZERO using \MOVDQU,
it does not require aligned pointer, but may work slower:}

\begin{lstlisting}
movdqu  xmm1, xmmword ptr [ebx+edi*4] ; ar1+i*4
movdqu  xmm0, xmmword ptr [esi+edi*4] ; ar2+i*4 is not 16-byte aligned, so load it to xmm0
paddd   xmm1, xmm0
movdqa  xmmword ptr [eax+edi*4], xmm1 ; ar3+i*4
\end{lstlisting}

\IFRU{А во всех остальных случаях, будет исполняться код, который был бы, как если бы не была 
включена поддержка SSE2.}
{In all other cases, non-SSE2 code will be executed.}

\subsubsection{GCC}

\newcommand{\URLGCCVEC}{\url{http://gcc.gnu.org/projects/tree-ssa/vectorization.html}}

\IFRU{Но и GCC умеет кое-что векторизировать\footnote{Подробнее о векторизации в GCC: \URLGCCVEC}, 
если компилировать с опциями \Othree и включить поддержку SSE2: \TT{-msse2}.}
{GCC may also vectorize in a simple cases\footnote{More about GCC vectorization support: \URLGCCVEC},
if to use \Othree option and to turn on SSE2 support: \TT{-msse2}.}

\IFRU{Вот что вышло}{What we got} (GCC 4.4.1):

\lstinputlisting{patterns/19_SIMD/18_2_gcc_O3.asm}

\IFRU{Почти то же самое, хотя и не так дотошно как Intel C++.}
{Almost the same, however, not as meticulously as Intel C++ doing it.}

\subsection{\IFRU{Реализация \strlen при помощи SIMD}{SIMD \strlen implementation}}

\newcommand{\URLMSDNSSE}{\href{http://msdn.microsoft.com/en-us/library/y0dh78ez(VS.80).aspx}{MSDN: MMX, SSE, and SSE2 Intrinsics}}

\IFRU{Прежде всего, следует заметить, что SIMD-инструкции можно вставлять в \CCpp код при помощи специальных 
макросов\footnote{\URLMSDNSSE}. В MSVC, часть находится в файле \TT{intrin.h}.}
{It should be noted the \ac{SIMD}-instructions may be inserted into \CCpp code via 
special macros\footnote{\URLMSDNSSE}.
As of MSVC, some of them are located in the \TT{intrin.h} file.}

\index{\CStandardLibrary!strlen()}
\IFRU{Имеется возможность реализовать функцию \strlen\footnote{strlen() ~--- стандартная функция Си 
для подсчета длины строки} при помощи SIMD-инструкций, работающий в 2-2.5 раза быстрее обычной реализации. 
Эта функция будет загружать в XMM-регистр сразу 16 байт и проверять каждый на ноль.}
{It is possible to implement \strlen function\footnote{strlen()~---standard C library function for calculating
string length} using SIMD-instructions, working 2-2.5 times faster than common implementation.
This function will load 16 characters into a XMM-register and check each against zero.}

\lstinputlisting{patterns/19_SIMD/18_3.c}

\newcommand{\URLSTRLEN}{http://www.strchr.com/sse2\_optimised\_strlen}

\IFRU{(пример базируется на исходнике \href{\URLSTRLEN}{отсюда}).}
{(the example is based on source code from \href{\URLSTRLEN}{there}).}

\IFRU{Компилируем в MSVC 2010 с опцией \Ox:}{Let's compile in MSVC 2010 with \Ox option:}

\lstinputlisting{patterns/19_SIMD/18_4_msvc_Ox.asm}

\IFRU{Итак, прежде всего, мы проверяем указатель \TT{str}, выровнен ли он по 16-байтной границе. 
Если нет, то мы вызовем обычную реализацию \strlen.}
{First of all, we check \TT{str} pointer, if it is aligned on a 16-byte boundary.
If not, let's call generic \strlen implementation.}

\IFRU{Далее мы загружаем по 16 байт в регистр \XMMONE при помощи команды \MOVDQA.}
{Then, load next 16 bytes into the \XMMONE register using \MOVDQA instruction.}

\IFRU{Наблюдательный читатель может спросить, почему в этом месте мы не можем использовать \MOVDQU, 
которая может загружать откуда угодно не взирая на факт, выровнен ли указатель?}
{Observant reader might ask, why \MOVDQU cannot be used here since it can load data from the memory
regardless the fact if the pointer aligned or not.}

\IFRU{Да, можно было бы сделать вот как: если указатель выровнен, загружаем используя \MOVDQA, 
иначе используем работающую чуть медленнее \MOVDQU.}
{Yes, it might be done in this way: if pointer is aligned, load data using \MOVDQA,
if not~---use slower \MOVDQU.}

\IFRU{Однако здесь кроется не сразу заметная проблема, которая проявляется вот в чем:}
{But here we are may stick into hard to notice caveat:}

\index{Page (memory)}
\newcommand{\URLPAGE}{\url{http://en.wikipedia.org/wiki/Page_(computer_memory)}}

\IFRU{В \ac{OS} линии \gls{Windows NT}, и не только, память выделяется страницами по 4 KiB (4096 байт). 
Каждый win32-процесс якобы имеет в наличии 4 GiB, но на самом деле, 
только некоторые части этого адресного пространства присоединены к реальной физической памяти. 
Если процесс обратится к блоку памяти, которого не существует, сработает исключение. 
Так работает виртуальная память\footnote{\URLPAGE}.}
{In \gls{Windows NT} line of \ac{OS} but not limited to it, memory allocated by pages of 4 KiB (4096 bytes).
Each win32-process has ostensibly 4 GiB, but in fact, only some parts
of address space are connected to real physical memory.
If the process accessing to the absent memory block, exception will be raised.
That's how virtual memory works\footnote{\URLPAGE}.}

\IFRU{Так вот, функция, читающая сразу по 16 байт, имеет возможность нечаянно вылезти за границу 
выделенного блока памяти. 
Предположим, \ac{OS} выделила программе 8192 (0x2000) байт по адресу 0x008c0000. 
Таким образом, блок занимает байты с адреса 0x008c0000 по 0x008c1fff включительно.}
{So, a function loading 16 bytes at once, may step over a border of allocated memory block.
Let's consider, \ac{OS} allocated 8192 (0x2000) bytes at the address 0x008c0000.
Thus, the block is the bytes starting from address 0x008c0000 to 0x008c1fff inclusive.}

\IFRU{За этим блоком, то есть начиная с адреса 0x008c2000 нет вообще ничего, т.е., \ac{OS} не выделяла там память. 
Обращение к памяти начиная с этого адреса вызовет исключение.}
{After the block, that is, starting from address 0x008c2000 there is nothing at all, e.g., \ac{OS} not allocated
any memory there.
Attempt to access a memory starting from the address will raise exception.}

\IFRU{И предположим, что программа хранит некую строку из, скажем, пяти символов почти в самом конце блока, 
что не является преступлением:}
{And let's consider, the program holding a string containing 5 characters almost at the end of block,
and that is not a crime.}

\begin{center}
  \begin{tabular}{ | l | l | }
    \hline
        0x008c1ff8 & 'h' \\
        0x008c1ff9 & 'e' \\
        0x008c1ffa & 'l' \\
        0x008c1ffb & 'l' \\
        0x008c1ffc & 'o' \\
        0x008c1ffd & '\textbackslash{}x00' \\
        0x008c1ffe & \IFRU{здесь случайный мусор}{random noise} \\
        0x008c1fff & \IFRU{здесь случайный мусор}{random noise} \\
    \hline
  \end{tabular}
\end{center}

\IFRU{В обычных условиях, программа вызывает \strlen передав ей указатель на строку \TT{'hello'} 
лежащую по адресу 0x008c1ff8. 
\strlen будет читать по одному байту до 0x008c1ffd, где ноль, и здесь она закончит работу.}
{So, in common conditions the program calling \strlen passing it a pointer to string \TT{'hello'} 
lying in memory at address 0x008c1ff8.
\strlen will read one byte at a time until 0x008c1ffd, where zero-byte, and so here it will stop working.}

\IFRU{Теперь, если мы напишем свою реализацию \strlen читающую сразу по 16 байт, с любого адреса, 
будь он выровнен по 16-байтной границе или нет, 
\MOVDQU попытается загрузить 16 байт с адреса 0x008c1ff8 по 0x008c2008, и произойдет исключение. 
Это ситуация которой, конечно, хочется избежать.}
{Now if we implement our own \strlen reading 16 byte at once, starting at any address, will it be aligned or not,
\MOVDQU may attempt to load 16 bytes at once at address 0x008c1ff8 up to 0x008c2008, 
and then exception will be raised.
That's the situation to be avoided, of course.}

\IFRU{Поэтому мы будем работать только с адресами, выровненными по 16 байт, что в сочетании со знанием 
что размер страницы \ac{OS} также как правило выровнен по 16 байт, 
даст некоторую гарантию что наша функция не будет пытаться читать из мест в невыделенной памяти.}
{So then we'll work only with the addresses aligned on a 16 byte boundary, what in combination with a knowledge
of \ac{OS} page size is usually aligned on a 16-byte boundary too, give us some warranty our function will not
read from unallocated memory.}

\IFRU{Вернемся к нашей функции}{Let's back to our function}.

\index{x86!\Instructions!PXOR}
\verb|_mm_setzero_si128()|\EMDASH\IFRU{это макрос, генерирующий \TT{pxor xmm0, xmm0} ~--- инструкция просто обнуляет регистр \XMMZERO.}
{is a macro generating \TT{pxor xmm0, xmm0}~---instruction just clears the \XMMZERO register}

\verb|_mm_load_si128()|\EMDASH\IFRU{это макрос для \MOVDQA, он просто загружает 16 байт по адресу из указателя в \XMMONE.}
{is a macro for \MOVDQA, it just loading 16 bytes from the address in the \XMMONE register.}

\index{x86!\Instructions!PCMPEQB}
\verb|_mm_cmpeq_epi8()|\EMDASH\IFRU{это макрос для \PCMPEQB, это инструкция которая 
побайтово сравнивает значения из двух XMM регистров.} 
{is a macro for \PCMPEQB, is an instruction comparing two XMM-registers bytewise.}

\IFRU{И если какой-то из байт равен другому, то в результирующем значении будет выставлено на месте этого 
байта \TT{0xff}, либо 0, если байты не были равны.}
{And if some byte was equals to other, there will be \TT{0xff} at this point in the result or 0 if otherwise.}

\IFRU{Например.}{For example.}

\begin{verbatim}
XMM1: 11223344556677880000000000000000
XMM0: 11ab3444007877881111111111111111
\end{verbatim}

\IFRU{После исполнения \TT{pcmpeqb xmm1, xmm0}, регистр \XMMONE будет содержать:}
{After \TT{pcmpeqb xmm1, xmm0} execution, the \XMMONE register shall contain:}

\begin{verbatim}
XMM1: ff0000ff0000ffff0000000000000000
\end{verbatim}

\IFRU{Эта инструкция в нашем случае, сравнивает каждый 16-байтный блок с блоком состоящим из 16-и нулевых байт, 
выставленным в \XMMZERO при помощи \TT{pxor xmm0, xmm0}.}
{In our case, this instruction comparing each 16-byte block with the block of 16 zero-bytes,
was set in the \XMMZERO register by \TT{pxor xmm0, xmm0}.}

\index{x86!\Instructions!PMOVMSKB}
\IFRU{Следующий макрос \TT{\_mm\_movemask\_epi8()} ~--- это инструкция \TT{PMOVMSKB}.}
{The next macro is \TT{\_mm\_movemask\_epi8()}~---that is \TT{PMOVMSKB} instruction.}

\IFRU{Она очень удобна как раз для использования в паре с \PCMPEQB.}
{It is very useful if to use it with \PCMPEQB.}

\TT{pmovmskb eax, xmm1}

\IFRU{Эта инструкция выставит самый первый бит \EAX в единицу, если старший бит первого байта в 
регистре \XMMONE является единицей. 
Иными словами, если первый байт в регистре \XMMONE является \TT{0xff}, то первый бит в \EAX будет также единицей, 
иначе нулем.}
{This instruction will set first \EAX bit into 1 if most significant bit of the first byte in the \XMMONE is $1$.
In other words, if first byte of the \XMMONE register is \TT{0xff}, first \EAX bit will be set to 1 too.}

\IFRU{Если второй байт в регистре \XMMONE является \TT{0xff}, то второй бит в \EAX также будет единицей. 
Иными словами, инструкция отвечает на вопрос, \IT{какие из байт в \XMMONE являются \TT{0xff}?}
В результате приготовит 16 бит и запишет в \EAX. Остальные биты в \EAX обнулятся.}
{If second byte in the \XMMONE register is \TT{0xff}, then second \EAX bit will be set to 1 too.
In other words, the instruction is answer to the question \IT{which bytes in the \XMMONE are \TT{0xff?}}
And will prepare 16 bits in the \EAX register. Other bits in the \EAX register are to be cleared.}

\IFRU{Кстати, не забывайте также вот о какой особенности нашего алгоритма:}
{By the way, do not forget about this feature of our algorithm:}

\IFRU{На вход может прийти 16 байт вроде}{There might be 16 bytes on input like} \TT{hello\textbackslash{}x00garbage\textbackslash{}x00ab}

\IFRU{Это строка \TT{'hello'}, после нее терминирующий ноль, затем немного мусора в памяти.}
{It is a \TT{'hello'} string, terminating zero, and also a random noise in memory.}

\newcommand{\MSBFOOTNOTE}{\footnote{most significant bit}}
\newcommand{\LSBFOOTNOTE}{\footnote{least significant bit}}

\IFRU{Если мы загрузим эти 16 байт в \XMMONE и сравним с нулевым \XMMZERO, то в итоге получим такое 
(я использую здесь порядок с MSB\MSBFOOTNOTE до LSB\LSBFOOTNOTE):}
{If we load these 16 bytes into \XMMONE and compare them with zeroed \XMMZERO, we will get something like
(I use here order from MSB\MSBFOOTNOTE to LSB\LSBFOOTNOTE):}

\begin{verbatim}
XMM1: 0000ff00000000000000ff0000000000
\end{verbatim}

\IFRU{Это означает что инструкция сравнения обнаружила два нулевых байта, что и не удивительно.}
{This means, the instruction found two zero bytes, and that is not surprising.}

\IFRU{\TT{PMOVMSKB} в нашем случае подготовит \EAX вот так (в двоичном представлении):} 
{\TT{PMOVMSKB} in our case will prepare \EAX like (in binary representation):} \IT{0010000000100000b}.

\IFRU{Совершенно очевидно, что далее наша функция должна учитывать только первый встретившийся
нулевой бит и игнорировать все остальное.}
{Obviously, our function must consider only first zero bit and ignore the rest ones.}

\index{x86!\Instructions!BSF}
\label{instruction_BSF}
\IFRU{Следующая инструкция}{The next instruction}\EMDASH\TT{BSF} (\IT{Bit Scan Forward}). 
\IFRU{Это инструкция находит самый младший бит во втором операнде и записывает его позицию в первый операнд.}
{This instruction find first bit set to 1 and stores its position into first operand.}

\begin{verbatim}
EAX=0010000000100000b
\end{verbatim}

\IFRU{После исполнения этой инструкции \TT{bsf eax, eax}, в \EAX будет 5, что означает, 
что единица найдена в пятой позиции (считая с нуля).}
{After \TT{bsf eax, eax} instruction execution, \EAX will contain 5, this means, 
1 found at 5th bit position (starting from zero).}

\IFRU{Для использования этой инструкции, в MSVC также имеется макрос}
{MSVC has a macro for this instruction:} \TT{\_BitScanForward}.

\IFRU{А дальше все просто. Если нулевой байт найден, его позиция прибавляется к тому что 
мы уже насчитали и возвращается результат.}
{Now it is simple. If zero byte found, its position added to what we already counted and now we have 
ready to return result.}

\IFRU{Почти всё.}{Almost all.}

\IFRU{Кстати, следует также отметить, что компилятор MSVC сгенерировал два тела цикла сразу, для оптимизации.}
{By the way, it is also should be noted, MSVC compiler emitted two loop bodies side by side, for optimization.}

\IFRU{Кстати, в SSE 4.2 (который появился в Intel Core i7) все эти манипуляции со строками могут быть еще проще:}
{By the way, SSE 4.2 (appeared in Intel Core i7) offers more instructions where these string manipulations might be
even easier:} \url{http://www.strchr.com/strcmp\_and\_strlen\_using\_sse\_4.2}


\fi
\chapter{\IFRU{64 бита}{64 bits}}

\section{x86-64}
\index{x86-64}
\label{x86-64}

\IFRU{Это расширение x86-архитуктуры до 64 бит.}{It is a 64-bit extension to x86-architecture.}

\IFRU{С точки зрения начинающего reverse engineer-а, наиболее важные отличия от 32-битного x86 это:}
{From the reverse engineer's perspective, most important differences are:}

\index{\CLanguageElements!\Pointers}
\begin{itemize}

\item
\IFRU{Почти все регистры (кроме FPU и SIMD) расширены до 64-бит и получили префикс r-. 
И еще 8 регистров добавлено. 
В итоге имеются эти \ac{GPR}-ы:}
{Almost all registers (except FPU and SIMD) are extended to 64 bits and got r- prefix.
8 additional registers added.
Now \ac{GPR}'s are:} \RAX, \RBX, \RCX, \RDX, 
\RBP, \RSP, \RSI, \RDI, \Reg{8}, \Reg{9}, \Reg{10}, 
\Reg{11}, \Reg{12}, \Reg{13}, \Reg{14}, \Reg{15}. 

\IFRU{К ним также можно обращаться так же, как и прежде. Например, для доступа к младшим 32 битам \TT{RAX} 
можно использовать \EAX.}
{It is still possible to access to \IT{older} register parts as usual. 
For example, it is possible to access lower 32-bit part of the \TT{RAX} register using \EAX.}

\IFRU{У новых регистров \TT{r8-r15} также имеются их \IT{младшие части}: \TT{r8d-r15d} 
(младшие 32-битные части), 
\TT{r8w-r15w} (младшие 16-битные части), \TT{r8b-r15b} (младшие 8-битные части).}
{New \TT{r8-r15} registers also has its \IT{lower parts}: \TT{r8d-r15d} (lower 32-bit parts),
\TT{r8w-r15w} (lower 16-bit parts), \TT{r8b-r15b} (lower 8-bit parts).}

\IFRU{Удвоено количество SIMD-регистров: с 8 до 16:}
{SIMD-registers number are doubled: from 8 to 16:} \TT{XMM0-XMM15}.

\item
\IFRU{В win64 передача всех параметров немного иная, это немного похоже на fastcall~(\ref{fastcall}).
Первые 4 аргумента записываются в регистры \RCX, \RDX, \Reg{8}, \Reg{9}, а остальные ~--- в стек. 
Вызывающая функция также должна подготовить место из 32 байт чтобы вызываемая функция могла сохранить 
там первые 4 аргумента и использовать эти регистры по своему усмотрению. 
Короткие функции могут использовать аргументы прямо из регистров, но б\'{о}льшие функции могут сохранять 
их значения на будущее.}
{In Win64, function calling convention is slightly different, somewhat resembling fastcall~(\ref{fastcall}).
First 4 arguments stored in the \RCX, \RDX, \Reg{8}, \Reg{9} registers, others~---in the stack.
\Gls{caller} function must also allocate 32 bytes so the \gls{callee} may save there 4 first arguments and use these 
registers for own needs.
Short functions may use arguments just from registers, but larger may save their values on the stack.}

\RU{Соглашение }System V AMD64 ABI (Linux, *BSD, \MacOSX)\cite{SysVABI} \IFRU{также напоминает}{also somewhat resembling}
fastcall, \IFRU{использует 6 регистров}{it uses 6 registers} 
\RDI, \RSI, \RDX, \RCX, \Reg{8}, \Reg{9} \IFRU{для первых шести аргументов}{for the first 6 arguments}.
\IFRU{Остальные передаются через стек}{All the rest are passed in the stack}.

\IFRU{См. также в соответствующем разделе о способах передачи аргументов через стек}
{See also section about calling conventions}~(\ref{sec:callingconventions}).

\item
\IFRU{Сишный \Tint остается 32-битным для совместимости.}
{C \Tint type is still 32-bit for compatibility.}

\item
\IFRU{Все указатели теперь 64-битные}{All pointers are 64-bit now}.

% to be proofreaded (begin)
\IFRU{На это иногда сетуют: ведь теперь для хранения всех указателей нужно в 2 раза больше места 
в памяти, в т.ч. и в кэш-памяти, не смотря на то что x64-процессоры адресуют только 48 бит
внешней \ac{RAM}}
{This provokes irritation sometimes: now one need twice as much memory for storing pointers,
including, cache memory, despite the fact x64 \ac{CPU}s addresses only 48 bits of external 
\ac{RAM}}.
% to be proofreaded (end)

\end{itemize}

\index{Register allocation}
\IFRU{Из-за того, что регистров общего пользования теперь вдвое больше, у компиляторов теперь больше 
свободного места для маневра называемого \glslink{register allocator}{register allocation}.
Для нас это означает, что в итоговом коде будет меньше локальных переменных.}
{Since now registers number are doubled, compilers has more space now for maneuvering calling 
\glslink{register allocator}{register allocation}.
What it meanings for us, emitted code will contain less local variables.}

\index{DES}
\IFRU{Для примера, функция вычисляющая первый S-бокс алгоритма шифрования DES, 
она обрабатывает сразу 32/64/128/256 значений, в зависимости от типа \TT{DES\_type} (uint32, uint64, SSE2 или AVX), 
методом bitslice DES (больше об этом методе читайте здесь~(\ref{bitslicedes})):}
{For example, function calculating first S-box of DES encryption algorithm, it processing
32/64/128/256 values at once (depending on \TT{DES\_type} type (uint32, uint64, SSE2 or AVX)) 
using bitslice DES method
(read more about this technique here ~(\ref{bitslicedes})):}

\lstinputlisting{patterns/20_x64/19_1.c}

\IFRU{Здесь много локальных переменных. Конечно, далеко не все они будут в локальном стеке. 
Компилируем обычным MSVC 2008 с опцией \Ox:}
{There is a lot of local variables. Of course, not all those will be in local stack.
Let's compile it with MSVC 2008 with \Ox option:}

\lstinputlisting[caption=\Optimizing MSVC 2008]{patterns/20_x64/19_2_msvc_Ox.asm}

\IFRU{5 переменных компилятору пришлось разместить в локальном стеке.}
{5 variables was allocated in local stack by compiler.}

\IFRU{Теперь попробуем то же самое только в 64-битной версии MSVC 2008:}
{Now let's try the same thing in 64-bit version of MSVC 2008:}

\lstinputlisting[caption=\Optimizing MSVC 2008]{patterns/20_x64/19_3_msvc_x64.asm}

\IFRU{Компилятор ничего не выделил в локальном стеке, а \TT{x36} это синоним для \TT{a5}.}
{Nothing allocated in local stack by compiler, \TT{x36} is synonym for \TT{a5}.}

\IFRU{Кстати, видно, что функция сохраняет регистры \RCX, \RDX в отведенных для 
этого вызываемой функцией местах, 
а \Reg{8} и \Reg{9} не сохраняет, а начинает использовать их сразу.}
{By the way, we can see here, the function saved \RCX and \RDX registers in allocated by \gls{caller} space,
but \Reg{8} and \Reg{9} are not saved but used from the beginning.}

\IFRU{Кстати, существуют процессоры с еще большим количеством \ac{GPR}, например, 
Itanium ~--- 128 регистров.}
{By the way, there are CPUs with much more \ac{GPR}'s, e.g. Itanium (128 registers).}

\section{ARM}

\IFRU{64-битные инструкции в ARM появились в}{In ARM, 64-bit instructions are appeared in} ARMv8.

\section{\IFRU{Числа с плавающей запятой}{Float point numbers}}

\IFRU{О том как происходит работа с числами с плавающей запятой в x86-64, читайте здесь: 
\ref{floating_SIMD}.}
{Read more here\ref{floating_SIMD} about how float point numbers are processed in x86-64.}

\ifx\LITE\undefined
% FIXME1 divide this file into separate ones...
\chapter{\RU{Работа с числами с плавающей запятой используя SIMD}\EN{Working with floating point numbers using SIMD}}

\label{floating_SIMD}
\index{IEEE 754}
\index{SIMD}
\index{SSE}
\index{SSE2}
\RU{Разумеется, FPU остался в x86-совместимых процессорах в то время, когда ввели расширения \ac{SIMD}}
\EN{Of course, the \ac{FPU} has remained in x86-compatible processors when the \ac{SIMD} extensions were added}.

\EN{The }\ac{SIMD}\RU{-расширения}\EN{ extensions} (SSE2) \RU{позволяют удобнее работать с числами с плавающей 
запятой}\EN{offer an easier way to work with floating-point numbers}.

\RU{Формат чисел остается тот же}\EN{The number format remains the same} (IEEE 754).

\index{x86-64}
\RU{Так что современные компиляторы (включая те, что компилируют под x86-64) 
обычно используют \ac{SIMD}-инструкции вместо FPU-инструкций.}\EN{So, modern compilers (including those generating
for x86-64) usually use \ac{SIMD} instructions instead of FPU ones.}

\RU{Это, можно сказать, хорошая новость, потому что работать с ними легче}
\EN{It can be said that it's good news, because it's easier to work with them}.

\RU{Примеры будем использовать из секции о FPU}
\EN{We are going to reuse the examples from the FPU section here}: \myref{sec:FPU}.

\section{\RU{Простой пример}\EN{Simple example}}

\lstinputlisting{patterns/12_FPU/1_simple/simple.c}

\subsection{x64}

\lstinputlisting[caption=\Optimizing MSVC 2012 x64]{patterns/205_floating_SIMD/simple_MSVC_2012_x64_Ox.asm}

\RU{Собственно, входные значения с плавающей запятой передаются через регистры \XMM{0}-\XMM{3}, 
а остальные --- через стек}\EN{The input floating point values are passed in the \XMM{0}-\XMM{3} registers,
all the rest---via the stack}
\footnote{\href{http://go.yurichev.com/17263}{MSDN: Parameter Passing}}.

$a$ \RU{передается через}\EN{is passed in} \XMM{0}, $b$\EMDASH{}\RU{через}\EN{via} \XMM{1}.
\RU{Но XMM-регистры (как мы уже знаем из секции о \ac{SIMD}: \myref{SIMD_x86}) 128-битные, 
а значения типа \Tdouble --- 64-битные,
так что используется только младшая половина регистра}
\EN{The XMM-registers are 128-bit (as we know from the section about \ac{SIMD}: \myref{SIMD_x86}), 
but the \Tdouble values are 64 bit, so only lower register half is used}.

\index{x86!\Instructions!DIVSD}
\TT{DIVSD} \RU{это SSE-инструкция, означает}\EN{is an SSE-instruction that stands for} 
\q{Divide Scalar Double-Precision Floating-Point Values}, 
\RU{и просто делит значение типа \Tdouble на другое, лежащие в младших половинах операндов}\EN{it just divides
one value of type \Tdouble by another, stored in the lower halves of operands}.

\RU{Константы закодированы компилятором в формате IEEE 754}\EN{The constants are encoded by compiler in IEEE 754 format}.

\index{x86!\Instructions!MULSD}
\index{x86!\Instructions!ADDSD}
\TT{MULSD} \AndENRU \TT{ADDSD} \RU{работают так же, только производят умножение и сложение}
\EN{work just as the same, but do multiplication and addition}.

\RU{Результат работы функции типа \Tdouble функция оставляет в регистре \XMM{0}}
\EN{The result of the function's execution in type \Tdouble is left in the in \XMM{0} register}.\\
\\
\RU{Как работает неоптимизирующий MSVC}\EN{That is how non-optimizing MSVC works}:

\lstinputlisting[caption=MSVC 2012 x64]{patterns/205_floating_SIMD/simple_MSVC_2012_x64.asm}

\index{Shadow space}
\RU{Чуть более избыточно}\EN{Slightly redundant}. 
\RU{Входные аргументы сохраняются в}\EN{The input arguments are saved in the} \q{shadow space} (\myref{shadow_space}), 
\RU{причем, только младшие половины регистров, т.е. только 64-битные значения типа \Tdouble}
\EN{but only their lower register halves, i.e., only 64-bit values of type \Tdouble}.
\ifdefined\IncludeGCC
\RU{Результат работы компилятора GCC точно такой же}\EN{GCC produces the same code}.
\fi

\subsection{x86}

\RU{Скомпилируем этот пример также и под x86. MSVC 2012 даже генерируя под x86, использует SSE2-инструкции:}
\EN{Let's also compile this example for x86. Despite the fact it's generating for x86, MSVC 2012 uses SSE2 instructions:}

\lstinputlisting[caption=\NonOptimizing MSVC 2012 x86]{patterns/205_floating_SIMD/simple_MSVC_2012_x86.asm}

\lstinputlisting[caption=\Optimizing MSVC 2012 x86]{patterns/205_floating_SIMD/simple_MSVC_2012_x86_Ox.asm}

\RU{Код почти такой же, правда есть пара отличий связанных с соглашениями о вызовах:}
\EN{It's almost the same code, however, there are some differences related to calling conventions:}
1) \RU{аргументы передаются не в XMM-регистрах, а через стек, как и прежде, в примерах с FPU (\myref{sec:FPU});}
\EN{the arguments are passed not in XMM registers, but in the stack, like in the FPU examples (\myref{sec:FPU});}
2) \RU{результат работы функции возвращается через \ST{0} --- для этого он через стек
(через локальную переменную \TT{tv}) копируется из XMM-регистра в \ST{0}.}
\EN{the result of the function is returned in \ST{0} --- in order to do so, it's copied
(through local variable \TT{tv}) from one of the XMM registers to \ST{0}.}

\ifdefined\IncludeOlly
\clearpage
\RU{Попробуем соптимизированный пример в}\EN{Let's try the optimized example in} \olly:

\begin{figure}[H]
\centering
\includegraphics[scale=\FigScale]{patterns/205_floating_SIMD/simple_olly1.png}
\caption{\olly: \TT{MOVSD} \RU{загрузила значение}\EN{loads the value of} $a$ \RU{в}\EN{into} \XMM{1}}
\label{fig:FPU_SIMD_simple_olly1}
\end{figure}

\clearpage
\begin{figure}[H]
\centering
\includegraphics[scale=\FigScale]{patterns/205_floating_SIMD/simple_olly2.png}
\caption{\olly: \TT{DIVSD} \RU{вычислила}\EN{calculated} \gls{quotient} 
\RU{и оставила его в}\EN{and stored it in} \XMM{1}}
\label{fig:FPU_SIMD_simple_olly2}
\end{figure}

\clearpage
\begin{figure}[H]
\centering
\includegraphics[scale=\FigScale]{patterns/205_floating_SIMD/simple_olly3.png}
\caption{\olly: \TT{MULSD} \RU{вычислила}\EN{calculated} \gls{product} \RU{и оставила его в}\EN{and stored it
in} \XMM{0}}
\label{fig:FPU_SIMD_simple_olly3}
\end{figure}

\clearpage
\begin{figure}[H]
\centering
\includegraphics[scale=\FigScale]{patterns/205_floating_SIMD/simple_olly4.png}
\caption{\olly: \TT{ADDSD} \RU{прибавила значение в}\EN{adds value in} \XMM{0} \RU{к}\EN{to} \XMM{1}}
\label{fig:FPU_SIMD_simple_olly4}
\end{figure}

\clearpage
\begin{figure}[H]
\centering
\includegraphics[scale=\FigScale]{patterns/205_floating_SIMD/simple_olly5.png}
\caption{\olly: \FLD \RU{оставляет результат функции в}\EN{left function result in} \ST{0}}
\label{fig:FPU_SIMD_simple_olly5}
\end{figure}

\RU{Видно, что \olly показывает XMM-регистры как пары чисел в формате \Tdouble,
но используется только \IT{младшая} часть.}
\EN{We see that \olly shows the XMM registers as pairs of \Tdouble numbers,
but only the \IT{lower} part is used.}
\RU{Должно быть, \olly показывает их именно так, потому что сейчас исполняются SSE2-инструкции
с суффиксом \TT{-SD}.}
\EN{Apparently, \olly shows them in that format because the SSE2 instructions (suffixed with \TT{-SD}) 
are executed right now.}
\RU{Но конечно же, можно переключить отображение значений в регистрах и посмотреть содержимое
как 4 \Tfloat{}-числа или просто как 16 байт.}
\EN{But of course, it's possible to switch the register format and to see their contents as
4 \Tfloat{}-numbers or just as 16 bytes.}
\fi

\clearpage
\section{\RU{Передача чисел с плавающей запятой в аргументах}\EN{Passing floating point number via arguments}}

\lstinputlisting{patterns/12_FPU/2_passing_floats/pow.c}

\RU{Они передаются в младших половинах регистров}\EN{They are passed in the lower halves
of the} \XMM{0}-\XMM{3}\EN{ registers}.

\lstinputlisting[caption=\Optimizing MSVC 2012 x64]{patterns/205_floating_SIMD/pow_MSVC_2012_x64_Ox.asm}

\index{x86!\Instructions!MOVSD}
\index{x86!\Instructions!MOVSDX}
\RU{Инструкции}\EN{There is no} \TT{MOVSDX} \RU{нет в документации от}\EN{instruction in} 
Intel \cite{Intel} \AndENRU AMD \cite{AMD}\EN{ manuals}, 
\RU{там она называется просто}\EN{there it is called just} \TT{MOVSD}.
\RU{Таким образом, в процессорах x86 две инструкции с одинаковым именем}\EN{So there are two instructions
sharing the same name in x86} (\RU{о второй}\EN{about the other see}: \myref{REP_MOVSx}).
\RU{Возможно, в Microsoft решили избежать
путаницы и переименовали инструкцию в}\EN{Apparently, Microsoft developers wanted to get rid of the mess,
so they renamed it to} \TT{MOVSDX}.
\RU{Она просто загружает значение в младшую половину XMM-регистра}\EN{It just loads a value into
the lower half of a XMM register}.

\RU{Функция }\TT{pow()} \RU{берет аргументы из}\EN{takes arguments from} \XMM{0} \AndENRU \XMM{1}, 
\RU{и возвращает результат в}\EN{and returns result in} \XMM{0}.
\RU{Далее он перекладывается в}\EN{It is then moved to} \RDX \ForENRU \printf. 
\RU{Почему}\EN{Why}? 
\RU{Может быть, это потому что}\EN{Maybe because} 
\printf\EMDASH{}\RU{функция с переменным количеством аргументов}\EN{is a variable arguments function}?

\lstinputlisting[caption=\Optimizing GCC 4.4.6 x64]{patterns/205_floating_SIMD/pow_GCC446_x64_O3.s.\LANG}

GCC \RU{работает понятнее}\EN{generates clearer output}. 
\RU{Значение для}\EN{The value for} \printf \RU{передается в}\EN{is passed in} \XMM{0}. 
\RU{Кстати, вот тот случай, когда в}\EN{By the way, here is a case when 1 is written into} \EAX
\ForENRU \printf \RU{записывается 1 --- это значит, что будет передан один аргумент в векторных регистрах, 
так того требует стандарт}\EN{---this implies that one argument will be passed in vector registers,
just as the standard requires} \cite{SysVABI}.

\section{\RU{Пример с сравнением}\EN{Comparison example}}

\lstinputlisting{patterns/12_FPU/3_comparison/d_max.c}

\subsection{x64}

\lstinputlisting[caption=\Optimizing MSVC 2012 x64]{patterns/205_floating_SIMD/d_max_MSVC_2012_x64_Ox.asm}

\Optimizing MSVC \RU{генерирует очень понятный код}\EN{generates a code very easy to understand}.

\index{x86!\Instructions!COMISD}
\RU{Инструкция }\TT{COMISD} \RU{это}\EN{is} \q{Compare Scalar Ordered Double-Precision Floating-Point 
Values and Set EFLAGS}. \RU{Собственно, это она и делает}\EN{Essentially, that is what it does}.\\
\\
\NonOptimizing MSVC \RU{генерирует более избыточно, но тоже всё понятно}\EN{generates more redundant code,
but it is still not hard to understand}:

\lstinputlisting[caption=MSVC 2012 x64]{patterns/205_floating_SIMD/d_max_MSVC_2012_x64.asm}

\index{x86!\Instructions!MAXSD}
\RU{А вот}\EN{However,} GCC 4.4.6 \RU{дошел в оптимизации дальше и применил инструкцию}
\EN{did more optimizations and used the} \TT{MAXSD} (\q{Return Maximum Scalar 
Double-Precision Floating-Point Value})\RU{, которая просто выбирает максимальное значение}\EN{ instruction,
which just choose the maximum value}!

\lstinputlisting[caption=\Optimizing GCC 4.4.6 x64]{patterns/205_floating_SIMD/d_max_GCC446_x64_O3.s}

\clearpage
\subsection{x86}

\RU{Скомпилируем этот пример в MSVC 2012 с включенной оптимизацией:}
\EN{Let's compile this example in MSVC 2012 with optimization turned on:}

\lstinputlisting[caption=\Optimizing MSVC 2012 x86]{patterns/205_floating_SIMD/d_max_MSVC_2012_x86_Ox.asm}

\RU{Всё то же самое, только значения}\EN{Almost the same, but the values of} $a$ \AndENRU $b$ 
\RU{берутся из стека, а результат функции оставляется в}\EN{are taken from the stack and the function result 
is left in} \ST{0}.

\ifdefined\IncludeOlly
\RU{Если загрузить этот пример в}\EN{If we load this example in} \olly, 
\RU{увидим, как инструкция}\EN{we can see how the} \TT{COMISD} \RU{сравнивает значения и устанавливает/сбрасывает
флаги}\EN{instruction compares values and sets/clears the} \CF \AndENRU \PF\EN{ flags}:

\begin{figure}[H]
\centering
\includegraphics[scale=\FigScale]{patterns/205_floating_SIMD/d_max_olly.png}
\caption{\olly: \TT{COMISD} \RU{изменила флаги}\EN{changed} \CF \AndENRU \PF\EN{ flags}}
\label{fig:FPU_SIMD_d_max_olly}
\end{figure}
\fi

\section{\RU{Вычисление машинного эпсилона}\EN{Calculating machine epsilon}: x64 \AndENRU SIMD}
\label{machine_epsilon_x64_and_SIMD}

\RU{Вернемся к примеру \q{вычисление машинного эпсилона} для \Tdouble \lstref{machine_epsilon_double_c}.}
\EN{Let's revisit the \q{calculating machine epsilon} example for \Tdouble \lstref{machine_epsilon_double_c}.}

\RU{Теперь скомпилируем его для x64}\EN{Now we compile it for x64}:

\lstinputlisting[caption=\Optimizing MSVC 2012 x64]{patterns/205_floating_SIMD/epsilon_double_MSVC_2012_x64_Ox.asm}

\RU{Нет способа прибавить 1 к значению в 128-битном XMM-регистре, так что его нужно в начале поместить в память.}
\EN{There is no way to add 1 to a value in 128-bit XMM register, so it must be placed into memory.}

\RU{Впрочем, есть инструкция ADDSD (\IT{Add Scalar Double-Precision Floating-Point Values}),
которая может прибавить значение к младшей 64-битной части XMM-регистра игнорируя старшую половину,
но наверное MSVC 2012 пока недостаточно хорош для этого}
\EN{There is, however, the ADDSD instruction (\IT{Add Scalar Double-Precision Floating-Point Values}) 
which can add a value to the lowest 64-bit half of a XMM register while ignoring the higher one, 
but MSVC 2012 probably is not that good yet}
\footnote{\RU{В качестве упражнения, вы можете попробовать переработать этот код, чтобы избавиться 
от использования локального стека}\EN{As an exercise, you may try to rework this code to 
eliminate the usage of the local stack}.}.

\RU{Так или иначе, значение затем перезагружается в XMM-регистр и происходит вычитание.}
\EN{Nevertheless, the value is then reloaded to a XMM register and subtraction occurs.}
SUBSD \RU{это}\EN{is} \q{Subtract Scalar Double-Precision Floating-Point Values}, 
\RU{т.е. операция производится над младшей 64-битной частью 128-битного XMM-регистра}
\EN{i.e., it operates on the lower 64-bit part of 128-bit XMM register}.
\RU{Результат возвращается в регистре XMM0}\EN{The result is returned in the XMM0 register}.

\section{\RU{И снова пример генератора случайных чисел}\EN{Pseudo-random number generator example revisited}}
\label{FPU_PRNG_SIMD}

\RU{Вернемся к примеру ``пример генератора случайных чисел'' \lstref{FPU_PRNG}.}
\EN{Let's revisit ``pseudo-random number generator example'' example \lstref{FPU_PRNG}.}

\RU{Если скомпилировать это в MSVC 2012, компилятор будет использовать SIMD-инструкции для FPU.}
\EN{If we compile this in MSVC 2012, it will use the SIMD instructions for the FPU.}

\lstinputlisting[caption=\Optimizing MSVC 2012]{patterns/205_floating_SIMD/FPU_PRNG/MSVC2012_Ox_Ob0.asm.\LANG}

\RU{У всех инструкций суффикс -SS, это означает}\EN{All instructions have the -SS suffix, this means } ``Scalar Single''.
``Scalar'' \RU{означает что только одно значение хранится в регистре}\EN{means that only one value is stored in the register}.
``Single'' \RU{означает что это тип \Tfloat}\EN{means \Tfloat data type}.


\section{\RU{Итог}\EN{Summary}}

\RU{Во всех приведенных примерах, в XMM-регистрах используется только младшая половина регистра, там
хранится значение в формате IEEE 754}\EN{Only the lower half of XMM registers is used in all examples here, 
to store number in IEEE 754 format}.

\RU{Собственно, все инструкции с суффиксом}\EN{Essentially, all instructions prefixed by} 
\TT{-SD} (\q{Scalar Double-Precision})\EMDASH{}\RU{это инструкции для работы с числами с плавающей 
запятой в формате IEEE 754, 
хранящиеся в младшей 64-битной половине XMM-регистра}\EN{are instructions working with floating point numbers
in IEEE 754 format, stored in the lower 64-bit half of a XMM register}.

\RU{Всё удобнее чем это было в FPU, видимо, сказывается тот факт, что расширения 
SIMD развивались не так хаотично как FPU в прошлом.}
\EN{And it is easier than in the FPU, probably because the SIMD extensions 
were evolved in a less chaotic way than the FPU ones in the past.}
\RU{Стековая модель регистров не используется}\EN{The stack register model is not used}.

\index{x86!\Instructions!ADDSS}
\index{x86!\Instructions!MOVSS}
\index{x86!\Instructions!COMISS}
% TODO1: do this!
\RU{Если вы попробуете заменить в этих примерах}\EN{If you would try to replace} \Tdouble \RU{на}\EN{with} \Tfloat
\RU{, то инструкции будут использоваться те же,
только с суффиксом}
% FIXME1 ... but their -SS versions
\EN{in these examples, the same instructions will be used, but prefixed with} \TT{-SS} 
(\q{Scalar Single-Precision}), \RU{например}\EN{for example}, \TT{MOVSS}, \TT{COMISS}, \TT{ADDSS}, \etc{}.

\q{Scalar} \RU{означает что SIMD-регистр будет хранить только одно значение, вместо нескольких.}
\EN{implies that the SIMD register containing only one value instead of several.}
\RU{Инструкции, работающие с несколькими значениями в регистре одновременно, имеют \q{Packed} в названии}
\EN{Instructions working with several values in a register simultaneously have \q{Packed} in their name}.

\RU{Нужно также обратить внимание, что SSE2-инструкции работают с 64-битными числами (\Tdouble) в формате IEEE 754,
в то время как внутреннее представление в FPU --- 80-битные числа.}
\EN{Needless to say, the SSE2 instructions work with 64-bit IEEE 754 numbers (\Tdouble),
while the internal representation of the floating-point numbers in FPU is 80-bit numbers.}
\RU{Поэтому ошибок округления (\IT{round-off error}) в FPU может быть меньше чем в SSE2,
как следствие, можно сказать, работа с FPU может давать более точные результаты вычислений.}
\EN{Hence, the FPU may produce less round-off errors and as a consequence, FPU may give more precise
calculation results.}

\fi
\ifdefined\IncludeARM
\chapter{\EN{More about ARM}\RU{Еще немного об ARM}}

\section{\RU{Загрузка констант в регистр}\EN{Loading constants into register}}

\subsection{\RU{32-битный}\EN{32-bit} ARM}
\label{ARM_big_constants_loading}

\RU{Как мы уже знаем, все инструкции имеют длину в 4 байта в режиме ARM и 2 байта в режиме Thumb.}
\EN{Aa we already know, all instructions has length of 4 bytes in ARM mode and 2 bytes in Thumb mode.}
\RU{Как в таком случае записать в регистр 32-битное число, если его невозможно закодировать
внутри одной инструкции?}
\EN{How to load 32-bit value into register, if it's not possible to encode it inside one instruction?}

\RU{Попробуем}\EN{Let's try}:

\begin{lstlisting}
unsigned int f()
{
	return 0x12345678;
};
\end{lstlisting}

\begin{lstlisting}[caption=GCC 4.6.3 -O3 \ARMMode]
f:
        ldr     r0, .L2
        bx      lr
.L2:
        .word   305419896 ; 0x12345678
\end{lstlisting}

\RU{Т.е., значение \TT{0x12345678} просто записано в памяти отдельно и загружается, если нужно.}
\EN{So, the \TT{0x12345678} value just stored aside in memory and loads if it needs.}
\RU{Но можно обойтись и без дополнительного обращения к памяти.}
\EN{But it's possible to get rid of additional memory access.}

\begin{lstlisting}[caption=GCC 4.6.3 -O3 -march=armv7-a \ARMMode]
movw    r0, #22136      ; 0x5678
movt    r0, #4660       ; 0x1234
bx      lr
\end{lstlisting}
% FIXME: what is MOVW? MOVT?

\RU{Видно что число загружается в регистр по частям, в начале младшая часть, затем старшая.}
\EN{We see that value is loaded into register by parts, lower part first, then higher.}

\RU{Следовательно, нужно 2 инструкции в режиме ARM, чтобы записать 32-битное число в регистр.}
\EN{It means, 2 instructions are necessary in ARM mode for loading 32-bit value into register.}
\RU{Это не так уж и страшно, потому что в реальном коде не так уж и много констант (кроме 0 и 1).}
\EN{It's not a real problem, because in fact there are not much constants in the real code (except of 0 and 1).}
\RU{Значит ли это, что это исполняется медленнее чем одна инструкция, как две инструкции?}
\EN{Does it mean it executes slower then one instruction, as two instructions?}
\RU{Врядли, наверняка современные процессоры ARM наверняка умеют распознавать такие 
последовательности и исполнять их быстро.}
\EN{Doubtfully. Most likely, modern ARM processors are able to detect such sequences and execute
them fast.}

\RU{А \IDA легко распознает подобные паттерны в коде и дизассемблирует эту ф-цию как:}
\EN{On the other hand, \IDA is able to detect such patterns in the code and disassembles this function as:}

\begin{lstlisting}
MOV    R0, 0x12345678
BX     LR
\end{lstlisting}

\subsection{ARM64}

\begin{lstlisting}
uint64_t f()
{
	return 0x12345678ABCDEF01;
};
\end{lstlisting}

\begin{lstlisting}[caption=GCC 4.9.1 -O3]
mov	x0, 61185   ; 0xef01
movk	x0, 0xabcd, lsl 16
movk	x0, 0x5678, lsl 32
movk	x0, 0x1234, lsl 48
ret
\end{lstlisting}

\TT{MOVK} \RU{означает}\EN{means} ``MOV Keep'', \RU{т.е., она записывает 16-битное значение в регистр, не трогая
при этом остальные биты.}\EN{i.e., it writes 16-bit value into register, not touching other bits at the same 
time.}
\RU{Суффикс }\TT{LSL} \RU{сдвигает значение в каждом случае влево на 16, 32 и 48 бит. Сдвиг происходит
перед загрузкой.}\EN{suffix shifts value left by 16, 32 and 48 bits at each step. Shifting done before loading.}
\RU{Таким образом, нужно 4 инструкции, чтобы записать в регистр 64-битное значение.}
\EN{This means, 4 instructions are necessary to load 64-bit value into register.}

\subsubsection{\RU{Записать числа с плавающей точкой в регистр}\EN{Storing floating number into register}}

\RU{Некоторые числа можно записывать в D-регистр при помощи только одной инструкции.}
\EN{It's possible to store a floating number into D-register using only one instruction.}

\RU{Например}\EN{For example}:

\begin{lstlisting}
double a()
{
	return 1.5;
};
\end{lstlisting}

\begin{lstlisting}[caption=GCC 4.9.1 -O3 + objdump]
0000000000000000 <a>:
   0:   1e6f1000        fmov    d0, #1.500000000000000000e+000
   4:   d65f03c0        ret
\end{lstlisting}

\RU{Число $1.5$ действительно было закодировано в 32-битной инструкции.}
\EN{$1.5$ number was indeed encoded in 32-bit instruction.}
\RU{Но как}\EN{But how}?
\RU{В ARM64, инструкцию \TT{FMOV} есть 8 бит для кодирования некоторых чисел с плавающей запятой.}
\EN{In ARM64, there are 8 bits in \TT{FMOV} instruction for encoding some float point numbers.}
\RU{В \cite{ARM64ref} алгоритм называется \TT{VFPExpandImm()}.}
\EN{The algorithm is called \TT{VFPExpandImm()} in \cite{ARM64ref}.}
\RU{Я попробовал разные: $30.0$ и $31.0$ компилятору удается закодировать, а $32.0$ уже нет, для него
приходится выделять 8 байт в памяти и записать его там в формате IEEE 754:}
\EN{I tried different: compiler is able to encode $30.0$ and $31.0$, but it couldn't encode $32.0$,
an 8 bytes should be allocated to this number in IEEE 754 format:}

\begin{lstlisting}
double a()
{
	return 32;
};
\end{lstlisting}

\begin{lstlisting}[caption=GCC 4.9.1 -O3]
a:
	ldr	d0, .LC0
	ret
.LC0:
	.word	0
	.word	1077936128
\end{lstlisting}

\section{\RU{Релоки}\EN{Relocs} \InENRU ARM64}
\label{ARM64_relocs}

\RU{Как известно, в ARM64 инструкции 4-байтные, так что записать длинное число в регистр одной инструкцией нельзя.}
\EN{As we know, there are 4-byte instructions in ARM64, so it is impossible to write large number into register
using single instruction.}
\RU{Тем не менее, файл может быть загружен по произвольному адресу в памяти, для этого релоки и нужны.}
\EN{Nevertheless, image may be loaded at random address in memory, so that's why relocs are existing.}
\RU{Больше о них (в связи с Win32 PE)}\EN{Read more about them (in relation to Win32 PE)}: \ref{subsec:relocs}.

\RU{В ARM64 принят следующий метод: адрес формируется при помощи пары инструкций: \TT{ADRP} и \ADD.}
\EN{ARM64 method is to form address using \TT{ADRP} and \ADD instructions pair.}
\RU{Первая загружает в регистр адрес 4Kb-страницы, а вторая прибавляет остаток.}
\EN{The first loads 4Kb-page address and the second adding remainder.}
\RU{Я скомпилировал пример из}\EN{I compiled example from} ``\HelloWorldSectionName'' 
(\lstref{hw_c}) \InENRU GCC (Linaro) 4.9 \RU{под}\EN{under} win32:

\begin{lstlisting}
...>aarch64-linux-gnu-gcc.exe hw.c -c

...>aarch64-linux-gnu-objdump.exe -d hw.o

...

0000000000000000 <main>:
   0:   a9bf7bfd        stp     x29, x30, [sp,#-16]!
   4:   910003fd        mov     x29, sp
   8:   90000000        adrp    x0, 0 <main>
   c:   91000000        add     x0, x0, #0x0
  10:   94000000        bl      0 <printf>
  14:   52800000        mov     w0, #0x0                        // #0
  18:   a8c17bfd        ldp     x29, x30, [sp],#16
  1c:   d65f03c0        ret

...>aarch64-linux-gnu-objdump.exe -r hw.o

...

RELOCATION RECORDS FOR [.text]:
OFFSET           TYPE              VALUE
0000000000000008 R_AARCH64_ADR_PREL_PG_HI21  .rodata
000000000000000c R_AARCH64_ADD_ABS_LO12_NC  .rodata
0000000000000010 R_AARCH64_CALL26  printf
\end{lstlisting}

\RU{Итак, в этом объектом файле три релока.}
\EN{So there are 3 relocs in this object file.}

\begin{itemize}
\item \RU{Самый первый записывает 21-битный адрес страницы в битовые поля инструкции \TT{ADRP}.}
\EN{The very first writes 21-bit page address into \TT{ADRP} instruction bit fields.}

\item \RU{Второй ---- 12 бит адреса, относительного от начала страницы, в поля инструкции \ADD.}
\EN{Second---12 bit of address relative to page start, into \ADD instruction bit fields.}

\item \RU{Последний, 26-битный, накладывается на инструкцию по адресу \TT{0x10}, где переход на ф-цию \printf.}
\EN{Last, 26-bit one, is applied to the instruction at \TT{0x10} address where the 
jump to the \printf function is.}
\RU{Из-за того что в ARM64 (да и в ARM в режиме ARM) невозможен переход по адресу не кратному 4,
так что доступное адресное пространство становится не 26 бит, а 28.}
\EN{Since it's not possible in ARM64 (and in ARM in ARM mode) to jump to the address not multiple of 4,
so the available address space is not 26 bits, but 28.}
\end{itemize}
% FIXME: objdump of resulting image
\RU{Если запустить пример в GDB под Linux и посмотреть опкоды:}
\EN{Here is what we see if to load the example in GDB under Linux and to see opcodes:}

\begin{lstlisting}
(gdb) disas/r main
Dump of assembler code for function main:
   0x0000000000400590 <+0>:     fd 7b bf a9     stp     x29, x30, [sp,#-16]!
   0x0000000000400594 <+4>:     fd 03 00 91     mov     x29, sp
   0x0000000000400598 <+8>:     00 00 00 90     adrp    x0, 0x400000
   0x000000000040059c <+12>:    00 20 19 91     add     x0, x0, #0x648
=> 0x00000000004005a0 <+16>:    a0 ff ff 97     bl      0x400420 <puts@plt>
   0x00000000004005a4 <+20>:    00 00 80 52     mov     w0, #0x0                        // #0
   0x00000000004005a8 <+24>:    fd 7b c1 a8     ldp     x29, x30, [sp],#16
   0x00000000004005ac <+28>:    c0 03 5f d6     ret
End of assembler dump.
\end{lstlisting}

\RU{Видно что}\EN{We see that} \IT{immhi} \AndENRU \IT{immlo} \RU{в инструкции}\EN{in} 
\TT{ADRP}\EN{ instruction}\EMDASH{}
\RU{нули, т.е., строка по нужному адресу находится в той же 4Kb-странице что и код ф-ции \main.}
\EN{are zeroes, i.e., the string is located in the same 4Kb-page as a \main function code.}

\RU{Дизассемблируем инструкцию \ADD вручную используя}\EN{Let's disassemble \ADD instruction manually 
using} \cite{ARM64ref}\footnote{N.B.: \RU{GDB показывает байты опкодов в том же порядке, в котором
они расположены в памяти, но документация от ARM показывает их наоборот}\EN{GDB shows opcode bytes in the same
order as they located in memory, but ARM manuals shows them contrariwise}}, \RU{и получим}\EN{and we will see}:

\begin{lstlisting}
91          19          20       00
10010001    00011001    00100000 00000000
sf=1 0 0 10001  shift=00 imm12=011001001000 (0x648) Rn=00000 (X0) Rd=00000 (X0)
\end{lstlisting}

\RU{Т.е., загрузчик Linux дописал \TT{0x648} на место бит \TT{imm12} в этой инструкции.}
\EN{So, the Linux loader writes \TT{0x648} on the \TT{imm12} bits place in this instruction.}
\TT{0x648} \RU{это адрес строки}\EN{is the address of} ``Hello!'' \RU{относительно начала 4Kb-страницы}\EN{string
relatively to the start of 4Kb-page}:

\lstinputlisting{patterns/ARM/gdb2.txt}

\RU{Действительно}\EN{Indeed}, \main \RU{начинается с}\EN{is beginning at} \TT{0x400590}, \RU{а строка}
\EN{and the} ``Hello!'' \RU{с}\EN{string at} \TT{0x400648}, 
\RU{страница начинается с}\EN{the page beginning at} \TT{0x400000}, \RU{а следующая страница 
уже с}\EN{and the next page at} \TT{0x401000}.

\RU{Больше о релоках связанных с ARM64}\EN{More about ARM64-related relocs}: \cite{ARM64_ELF}.

\fi
\ifdefined\IncludeMIPS
\chapter{\EN{More about}\RU{Еще немного о} MIPS}

\section{Instructions}

There are 3 kinds of instructions:

\begin{itemize}

\item R-type: those which has 3 registers.
R-instruction are often has the following form:

\begin{lstlisting}
instruction destination, source1, source2
\end{lstlisting}

One important thing to remember is that when first and second register is the same, 
IDA may show it in shorter form:

\begin{lstlisting}
instruction destination/source1, source2
\end{lstlisting}

That somewhat reminds us Intel-syntax of x86 assembly language.

\item I-type: those which has 16-bit immediate value.
\item J-type: jump/branch instructions, has 26 bits for offset encoding.

\end{itemize}

% sections
\ifx\RUSSIAN\undefined
\section{\RU{Загрузка констант в регистр}\EN{Loading constants into register}}

\begin{lstlisting}
unsigned int f()
{
	return 0x12345678;
};
\end{lstlisting}

MIPS, just like ARM, has all instructions with size of 32-bit, so it's not possible to
embed 32-bit constant into one instruction.
\index{MIPS!\Instructions!LI}
\index{MIPS!\Instructions!ORI}
So this translates to at least two instructions: 
first loads high part of 32-bit number and the second
one apply OR operation which effectively sets low 16-bit part of the target register:

\begin{lstlisting}[caption=GCC 4.4.5 -O3 (assembly output)]
        li      $2,305397760                    # 0x12340000
        j       $31
        ori     $2,$2,0x5678
\end{lstlisting}

\IDA is fully aware of such code patterns, so it shows the last ORI instruction as LI pseudoinstruction
allegedly it loads full 32-bit number into V0 register --- for convenience.

\index{MIPS!\Instructions!LUI}

\begin{lstlisting}[caption=GCC 4.4.5 -O3 (IDA)]
         lui     $v0, 0x1234
         jr      $ra
         li      $v0, 0x12345678
\end{lstlisting}

GCC assembly output has LI instruction, but in fact, LUI (``Load Upper Imeddiate'') there,
which stores 16-bit value into high part of register.

\fi


\section{Further reading about MIPS}

\cite{MIPSRun}.

\fi

\ifx\LITE\undefined
\part{\IFRU{Важные фундаментальные вещи}{Important fundamentals}}

\chapter{\SignedNumbersSectionName}
\label{sec:signednumbers}
\index{Signed numbers}

\newcommand{\URLS}{\href{http://go.yurichev.com/17117}{wikipedia}}

\RU{Методов представления чисел с знаком \q{плюс} или \q{минус} несколько\footnote{\URLS}, 
но в компьютерах обычно применяется метод \q{дополнительный код} или \q{two's complement}.}%
\EN{There are several methods for representing signed numbers\footnote{\URLS}, 
but \q{two's complement} is the most popular one in computers.}%
\ES{Hay distintos m\'etodos para representar n\'umeros con signo\footnote{\URLS},
pero el \q{complemento a dos} es el m\'as popular en las computadoras.}%
\PTBRph{}%
\DEph{}\PLph{}%
\ITAph{}

\RU{Вот таблица некоторые значений байтов:}%
\EN{Here is a table for some byte values:}%
\ES{Aqu\'i hay una tabla con los valores de algunos bytes:}%
\PTBRph{}%
\DEph{}\PLph{}%
\ITAph{}

\begin{center}
\begin{tabular}{ | l | l | l | l | }
\hline
\cellcolor{blue!25}
	\RU{двоичное}%
	\EN{binary}%
	\ES{binario}%
	\PTBRph{}%
	\DEph{}\PLph{}%
	\ITAph{}
& 
\cellcolor{blue!25}
	\RU{шестнадцатеричное}%
	\EN{hexadecimal}%
	\ES{hexadecimal}%
	\PTBRph{}%
	\DEph{}\PLph{}%
	\ITAph{}
& 
\cellcolor{blue!25}
	\RU{беззнаковое}%
	\EN{unsigned}%
	\ES{sin signo}%
	\PTBRph{}%
	\DEph{}\PLph{}%
	\ITAph{}
&
\cellcolor{blue!25}
	\RU{знаковое}%
	\EN{signed}%
	\ES{con signo}%
	\PTBRph{}%
	\DEph{}\PLph{}%
	\ITAph{}
(%
	\RU{дополнительный код}%
	\EN{2's complement}%
	\ES{complemento a dos}%
	\PTBRph{}%
	\DEph{}\PLph{}%
	\ITAph{}%
) \\
\hline
01111111 & 0x7f & 127 & 127 \\
\hline
01111110 & 0x7e & 126 & 126 \\
\hline
\multicolumn{4}{ |c| }{...} \\
\hline
00000110 & 0x6 & 6 & 6 \\
\hline
00000101 & 0x5 & 5 & 5 \\
\hline
00000100 & 0x4 & 4 & 4 \\
\hline
00000011 & 0x3 & 3 & 3 \\
\hline
00000010 & 0x2 & 2 & 2 \\
\hline
00000001 & 0x1 & 1 & 1 \\
\hline
00000000 & 0x0 & 0 & 0 \\
\hline
11111111 & 0xff & 255 & -1 \\
\hline
11111110 & 0xfe & 254 & -2 \\
\hline
11111101 & 0xfd & 253 & -3 \\
\hline
11111100 & 0xfc & 252 & -4 \\
\hline
11111011 & 0xfb & 251 & -5 \\
\hline
11111010 & 0xfa & 250 & -6 \\
\hline
\multicolumn{4}{ |c| }{...} \\
\hline
10000010 & 0x82 & 130 & -126 \\
\hline
10000001 & 0x81 & 129 & -127 \\
\hline
10000000 & 0x80 & 128 & -128 \\
\hline
\end{tabular}
\end{center}

\index{x86!\Instructions!JA}
\index{x86!\Instructions!JB}
\index{x86!\Instructions!JL}
\index{x86!\Instructions!JG}
\RU{Разница в подходе к знаковым/беззнаковым числам, собственно, нужна потому что, например, 
если представить \TT{0xFFFFFFFE} и \TT{0x00000002} как беззнаковое, то первое число (4294967294) больше второго (2). 
Если их оба представить как знаковые, то первое будет $-2$, которое, разумеется, меньше чем второе (2).
Вот почему инструкции для условных переходов~(\myref{sec:Jcc}) представлены в обоих версиях ~--- 
и для знаковых сравнений (например, \JG, \JL) и для беззнаковых (\JA, \JB).}%
\EN{The difference between signed and unsigned numbers is that if we represent \TT{0xFFFFFFFE} and \TT{0x00000002} 
as unsigned, then the first number (4294967294) is bigger than the second one (2). 
If we represent them both as signed, the first one becomes $-2$, and it is smaller than the second (2). 
That is the reason why conditional jumps~(\myref{sec:Jcc}) are present both for signed (e.g. \JG, \JL) 
and unsigned (\JA, \JB) operations.}%
\ES{La diferencia entre n\'umeros con signo y sin signo est\'a en que si representamos \TT{0xFFFFFFFE} y \TT{0x00000002}
sin signo, el primero (4294967294) es mayor que el segundo (2).
Pero si representamos ambos con signo, el primero se vuelve $-2$, y es menor que el segundo (2).
Esa es la raz\'on por la que se tienen saltos condicionales~(\myref{sec:Jcc}) tanto para operaciones con signo (e.g. \JG, \JL)
como sin signo (\JA, \JB).}%

\par
\RU{Для простоты, вот что нужно знать:}%
\EN{For the sake of simplicity, this is what one needs to know:}%
\ES{Con el fin de la simplicidad, esto es lo que uno debe de saber:}%
\PTBRph{}%
\DEph{}\PLph{}%
\ITAph{}
\begin{itemize}
\item
	\RU{Числа бывают знаковые и беззнаковые.}%
	\EN{Numbers can be signed or unsigned.}%
	\ES{Los n\'umeros pueden ser con signo y sin signo.}%
	\PTBRph{}%
	\DEph{}\PLph{}%
	\ITAph{}

\item 
	\RU{Знаковые типы в \CCpp:}%
	\EN{\CCpp signed types:}%
	\ES{Tipos con signo en \CCpp:}%
	\PTBRph{}%
	\DEph{}\PLph{}%
	\ITAph{}
	
  \begin{itemize}
    \item \TT{int64\_t} (-9,223,372,036,854,775,808..9,223,372,036,854,775,807) (-~9.2..~9.2
		\RU{квинтиллионов}%
		\EN{quintillions}%
		\ES{billones}%
		\PTBRph{}%
		\DEph{}\PLph{}%
		\ITAph{}%
		) \OrENRU \\
                \TT{0x8000000000000000..0x7FFFFFFFFFFFFFFF}),
    \item \Tint (-2,147,483,648..2,147,483,647 (-~2.15..~2.15Gb) \OrENRU \TT{0x80000000..0x7FFFFFFF}),
    \item \Tchar (-128..127 \OrENRU \TT{0x80..0x7F}),
    \item \TT{ssize\_t}.
   \end{itemize}

	\RU{Беззнаковые:}%
	\EN{Unsigned:}%
	\ES{Sin signo:}%
	\PTBRph{}%
	\DEph{}\PLph{}%
	\ITAph{}
  \begin{itemize}
	  \item \TT{uint64\_t} (0..18,446,744,073,709,551,615 (~18
		\RU{квинтиллионов}%
		\EN{quintillions}%
		\ES{billones}%
		\PTBRph{}%
		\DEph{}\PLph{}%
		\ITAph{}%
	) \OrENRU \TT{0..0xFFFFFFFFFFFFFFFF}),
   \item \TT{unsigned int} (0..4,294,967,295 (~4.3Gb) \OrENRU \TT{0..0xFFFFFFFF}),
   \item \TT{unsigned char} (0..255 \OrENRU \TT{0..0xFF}), 
   \item \TT{size\_t}.
  \end{itemize}

\item
	\RU{У знаковых чисел знак определяется самым старшим битом: 1 означает \q{минус}, 0 означает \q{плюс}.}%
	\EN{Signed types have the sign in the most significant bit: 1 mean \q{minus}, 0 mean \q{plus}.}%
	\ES{En los tipos con signo, el signo se encuentra en el bit m\'as significativo: 1 significa \q{menos}, 0 significa \q{m\'as}.}%
	\PTBRph{}%
	\DEph{}\PLph{}%
	\ITAph{}

\item
    \RU{Преобразование в б\'{о}льшие типы данных обходится легко:}%
	\EN{Promoting to a larger data types is simple:}%
	\ES{Promover a un tipo de dato m\'as grande es sencillo:}%
	\PTBRph{}%
	\DEph{}\PLph{}%
	\ITAph{}
\myref{subsec:sign_extending_32_to_64}.

\label{sec:signednumbers:negation}
\item
	\RU{Изменить знак легко: просто инвертируйте все биты и прибавьте 1.}%
	\EN{Negation is simple: just invert all bits and add 1.}%
	\ES{La negaci\'on es simple: s\'olo invierte todos los bits y suma 1.}%
	\PTBRph{}%
	\DEph{}\PLph{}%
	\ITAph{}
\RU{Мы можем заметить, что число другого знака находится на другой стороне на том же расстоянии от нуля.}%
\EN{We can remember that a number of inverse sign is located
on the opposite side at the same proximity from zero.}%
\ES{Podemos recordar que un n\'umero con signo contrario se localiza en el lado opuesto a la misma distancia de cero.}%
\PTBRph{}%
\DEph{}\PLph{}%
\ITAph{}
\RU{Прибавление единицы необходимо из-за присутствия нуля посредине.}%
\EN{The addition of one is needed because zero is present in the middle.}%
\ES{La suma de 1 es necesaria porque el cero se localiza en medio.}%
\PTBRph{}%
\DEph{}\PLph{}%
\ITAph{}

\index{x86!\Instructions!IDIV}
\index{x86!\Instructions!DIV}
\index{x86!\Instructions!IMUL}
\index{x86!\Instructions!MUL}
\index{x86!\Instructions!CBW}
\index{x86!\Instructions!CWD}
\index{x86!\Instructions!CWDE}
\index{x86!\Instructions!CDQ}
\index{x86!\Instructions!CDQE}
\index{x86!\Instructions!MOVSX}
\index{x86!\Instructions!SAR}
\item 
	\RU{Инструкции сложения и вычитания работают одинаково хорошо и для знаковых и для беззнаковых значений.}%
	\EN{The addition and subtraction operations work well for both signed and unsigned values.}%
	\ES{Las operaciones de suma y resta funcionan bien para valores tanto con signo como sin signo.}%
	\PTBRph{}%
	\DEph{}\PLph{}%
	\ITAph{}
	\RU{Но для операций умножения и деления, в x86 имеются разные инструкции:}%
	\EN{But for multiplication and division operations, x86 has different instructions:}%
	\ES{Sin embargo, para las operaciones de multiplicaci\'on y divisi\'on x86 tiene instrucciones distintas:}%
	\PTBRph{}%
	\DEph{}\PLph{}%
	\ITAph{}
	\TT{IDIV}/\TT{IMUL}
		\RU{для знаковых}%
		\EN{for signed}%
		\ES{para n\'umeros con signo}%
		\PTBRph{}%
		\DEph{}\PLph{}%
		\ITAph{}
	\AndENRU \TT{DIV}/\TT{MUL}
		\RU{для беззнаковых.}%
		\EN{for unsigned.}%
		\ES{para n\'umeros sin signo.}%
		\PTBRph{}%
		\DEph{}\PLph{}%
		\ITAph{}
\ifx\LITE\undefined
\item
	\RU{Еще инструкции работающие с знаковыми числами:}%
	\EN{Here are some more instructions that work with signed numbers:}%
	\ES{\'Estas son algunas otras instruccciones que funcionan con n\'umeros con signo:}%
	\PTBRph{}%
	\DEph{}\PLph{}%
	\ITAph{}
	\TT{CBW/CWD/CWDE/CDQ/CDQE} (\myref{ins:CBW_CWD_etc}), \TT{MOVSX} (\myref{MOVSX}), \TT{SAR} (\myref{ins:SAR}).
\fi
\end{itemize}

\iffalse
% TODO rework!
\section{%
	\RU{Переполнение integer}%
	\EN{Integer overflow}%
	\ES{Desbordamiento de enteros}%
	\PTBRph{}%
	\DEph{}\PLph{}%
	\ITAph{}%
}

\RU{Бывает так, что ошибки представления знаковых/беззнаковых могут привести к уязвимости \IT{переполнение integer}.}%
\EN{It is worth noting that the incorrect representation of a number can lead to integer overflow vulnerabilities.}%
\ES{Es necesario aclarar que la representaci\'on incorrecta de un n\'umero puede desenbocar en vulnerabilidades de desbordamiento de enteros.}%
\PTBRph{}%
\DEph{}\PLph{}%
\ITAph{}

\RU{Например, есть некий сервис, который принимает по сети некие пакеты. 
В пакете есть заголовок где указана длина пакета. Это 32-битное значение. 
В процессе приема пакета, 
сервис проверяет это значение и сверяет, больше ли оно чем максимальный размер пакета, скажем, константа
\TT{MAX\_PACKET\_SIZE} (например, 10 килобайт), и если да, то пакет отвергается как некорректный. 
Сравнение знаковое. Злоумышленник подставляет значение \TT{0xFFFFFFFF}. Это число трактуется как знаковое $-1$ 
и оно меньше чем 10000. Проверка проходит. Продолжаем дальше и копируем этот пакет куда-нибудь себе 
в сегмент данных\dots вызов функции \TT{memcpy (dst, src, 0xFFFFFFFF)} скорее всего, 
затрет много чего внутри процесса.}%
\EN{For example, we have a network service and it receives network packets. 
In the packets there is a field where the subpacket length is encoded. 
It is a 32-bit value. 
After receiving the network packet, the service checks the field and if it is larger than 
some \TT{MAX\_PACKET\_SIZE} (let's say, 10 kilobytes), the packet is rejected as incorrect.
The comparison is signed. The intruder set this value to \TT{0xFFFFFFFF}.
While comparing, this number is considered as signed $-1$ and it is less than 10 kilobytes. 
No error here. 
The service would then like to copy the subpacket to another place in memory and calls the 
\TT{memcpy (dst, src, 0xFFFFFFFF)} function: this operation, rapidly garbles a lot of 
process memory.}%
\ES{Por ejemplo, tenemos un servicio de red que recibe paquetes de red.
En los paquetes se tiene un campo donde la longitud del subpaquete est\'a codificada;
tal campo es un valor de 32 bits.
Tras recibir el paquete de red, el servicio verifica que el campo sea menor a alg\'un valor
\TT{MAX\_PACKET\_SIZE} (digamos, 10 kilobytes), y de ser mayor lo descarta como incorrecto.
La comparaci\'on se realiza con signo. El intruso le asigna el valor de \TT{0xFFFFFFFF}.
Mientras se compara, ete n\'umero es considerado como $-1$ (con signo) y es menor que 10
kilobytes. No hay error ah\'i.
El servicio tratar\'a de copiar el subpaquete a otro lugar en memoria y llama a la funci\'on
\TT{memcpy (dst, src, 0xFFFFFFFF)}: dicha operaci\'on gasta mucha memoria del proceso r\'apidamente.}%
\PTBRph{}%
\DEph{}\PLph{}%
\ITAph{}

\RU{Немного подробнее:}%
\EN{More about it:}%
\ES{M\'as acerca de ello:}%
\PTBRph{}%
\DEph{}\PLph{}%
\ITAph{}
\cite{Phrack3C0A}.
% TODO example!
\fi

\chapter{Endianness\RU{ (порядок байт)}}
\label{sec:endianness}

\RU{Endianness (порядок байт) это способ представления чисел в памяти}
\EN{Endianness is a way of representing values in memory}.

\section{Big-endian\RU{ (от старшего к младшему)}}

\RU{Число}\EN{A} \TT{0x12345678} \RU{будет представлено в памяти так}\EN{value will be represented in memory as}:

\begin{center}
\begin{tabular}{ | l | l | }
\hline
\cellcolor{blue!25} \RU{адрес в памяти}\EN{address in memory} & \cellcolor{blue!25} \RU{значение байта}\EN{byte value} \\
\hline
+0 & 0x12 \\
\hline
+1 & 0x34 \\
\hline
+2 & 0x56 \\
\hline
+3 & 0x78 \\
\hline
\end{tabular}
\end{center}

\RU{CPU с таким порядком включают в себя}\EN{Big-endian CPUs include} Motorola 68k, IBM POWER.

\section{Little-endian\RU{ (от младшего к старшему)}}

\RU{Число}\EN{A} \TT{0x12345678} \RU{будет представлено в памяти так}\EN{value will be represented in memory as}:

\begin{center}
\begin{tabular}{ | l | l | }
\hline
\cellcolor{blue!25} \RU{адрес в памяти}\EN{address in memory} & \cellcolor{blue!25} \RU{значение байта}\EN{byte value} \\
\hline
+0 & 0x78 \\
\hline
+1 & 0x56 \\
\hline
+2 & 0x34 \\
\hline
+3 & 0x12 \\
\hline
\end{tabular}
\end{center}

\RU{CPU с таким порядком байт включают в себя}\EN{Little-endian CPUs include} Intel x86.

\section{\Example}

\RU{У меня есть big-endian Linux для MIPS заинсталированный в QEMU}
\EN{I've got big-endianness MIPS Linux installed and ready for QEMU}
\footnote{\RU{Я скачал его здесь}\EN{I've got it here}: \url{http://go.yurichev.com/17008}}.

\RU{И вот я сделал простой пример}\EN{And I compiled simple example}:

\begin{lstlisting}
#include <stdio.h>

int main()
{
	int v, i;

	v=123;

	printf ("%02X %02X %02X %02X\n", 
		*(char*)&v,
		*(((char*)&v)+1),
		*(((char*)&v)+2),
		*(((char*)&v)+3));
};
\end{lstlisting}

\RU{И запустил его}\EN{And run it}:

\begin{lstlisting}
root@debian-mips:~# ./a.out 
00 00 00 7B
\end{lstlisting}

\RU{Это оно и есть}\EN{That is}.
0x7B \RU{это}\EN{is} 123 \RU{в десятичном виде}\EN{in decimal}.
\RU{В little-endian-архитектуре, 7B будет первым байтом (вы можете это проверить в x86 или x86-64),
но здесь он последний, потому что старший байт идет первым.}
\EN{In little-endianness architecture, 7B will be the first byte (you can check on x86 or x86-64), 
but here it is the last one, because highest byte goes first.}

\RU{Вот почему имеются разные дистрибутивы Linux для MIPS}
\EN{That's why there are separate Linux distributions exist for MIPS} 
(``mips'' (big-endian) \AndENRU ``mipsel'' (little-endian)).
\RU{Программа скомпилированная для одного соглашения об endiannes, не сможет работать в OS использующей
другое соглашение.}
\EN{It is impossible for binary compiled for one endianness to work on the OS with different endianness.}\\
\\
\RU{Еще один пример связанный с big-endian в MIPS в этой книге:}
\EN{Another example of MIPS big-endiannes in this book:} \ref{MIPS_structure_big_endian}.

\section{Bi-endian\RU{ (переключаемый порядок)}}

\RU{CPU поддерживающие оба порядка, и его можно переключать, включают в себя}
\EN{CPUs which may switch between endianness are} ARM, PowerPC, SPARC, MIPS, \ac{IA64}, \RU{и т.д.}\EN{etc.}

\section{\RU{Конвертирование}\EN{Converting data}}

\index{TCP/IP}
\RU{Сетевые пакеты TCP/IP используют соглашение big-endian, вот почему программа работающая на little-endian архитектуре
должна конвертировать значения используя ф-ции}
\EN{TCP/IP network data packets use big-endian conventions, so that is why a program working on little-endian architecture
should convert values using} \TT{htonl()} \AndENRU \TT{htons()}\EN{ functions}.

\RU{Порядок байт big-endian в среде TCP/IP также называется}
\EN{In TCP/IP, big-endian is also called} ``network byte order'',
\RU{а}\EN{while} little-endian\EMDASH{}``host byte order''.

\index{x86!\Instructions!BSWAP}
\RU{Инструкция }{The }\TT{BSWAP} \RU{также может использоваться для конвертирования}
\EN{instruction can also be used for conversion}.



\part{\RU{Поиск в коде того что нужно}\EN{Finding important/interesting stuff in the code}}

\RU{Современное ПО, в общем-то, минимализмом не отличается.}
\EN{Minimalism it is not a prominent feature of modern software.}

\index{\Cpp!STL}
\RU{Но не потому, что программисты слишком много пишут, 
а потому что к исполняемым файлам обыкновенно прикомпилируют все подряд библиотеки. 
Если бы все вспомогательные библиотеки всегда выносили во внешние DLL, мир был бы иным.
(Еще одна причина для Си++ ~--- STL и прочие библиотеки шаблонов.)}
\EN{But not because programmers are writing a lot, but in a reason that all libraries are commonly linked statically
to executable files.
If all external libraries were shifted into external DLL files, the world would be different.
(Another reason for C++~---STL and other template libraries.)}

\newcommand{\FOOTNOTEBOOST}{\footnote{\url{http://www.boost.org/}}}
\newcommand{\FOOTNOTELIBPNG}{\footnote{\url{http://www.libpng.org/pub/png/libpng.html}}}

\RU{Таким образом, очень полезно сразу понимать, какая функция из стандартной библиотеки или 
более-менее известной (как Boost\FOOTNOTEBOOST, libpng\FOOTNOTELIBPNG), 
а какая ~--- имеет отношение к тому что мы пытаемся найти в коде.}
\EN{Thus, it is very important to determine origin of a function, if it is from standard library or 
well-known library (like Boost\FOOTNOTEBOOST, libpng\FOOTNOTELIBPNG),
and which one~---is related to what we are trying to find in the code.}

\RU{Переписывать весь код на \CCpp, чтобы разобраться в нем, безусловно, не имеет никакого смысла.}
\EN{It is just absurd to rewrite all code to \CCpp to find what we're looking for.}

\RU{Одна из важных задач reverse engineer-а это быстрый поиск в коде того что собственно его интересует.}
\EN{One of the primary reverse engineer's task is to find quickly the code he/she needed.}

\index{\GrepUsage}
\RU{Дизассемблер \IDA позволяет делать поиск как минимум строк, последовательностей байт, констант.
Можно даже сделать экспорт кода в текстовый файл .lst или .asm и затем натравить на него \TT{grep}, \TT{awk}, и т.д.}
\EN{\IDA disassembler allow us search among text strings, byte sequences, constants.
It is even possible to export the code into .lst or .asm text file and then use \TT{grep}, \TT{awk}, etc.}

\RU{Когда вы пытаетесь понять, что делает тот или иной код, это запросто может быть какая-то 
опенсорсная библиотека вроде libpng. Поэтому, когда находите константы, или текстовые строки которые 
выглядят явно знакомыми, всегда полезно их \IT{погуглить}.
А если вы найдете искомый опенсорсный проект где это используется, 
то тогда будет достаточно будет просто сравнить вашу функцию с ней. 
Это решит часть проблем.}
\EN{When you try to understand what a code is doing, this easily could be some open-source library like libpng.
So when you see some constants or text strings which looks familiar, it is always worth to \IT{google} it.
And if you find the opensource project where it is used, 
then it will be enough just to compare the functions.
It may solve some part of the problem.}

\RU{К примеру, если программа использует какие-то XML-файлы, первым шагом может быть
установление, какая именно XML-библиотека для этого используется, ведь часто используется какая-то
стандартная (или очень известная) вместо самодельной.}
\EN{For example, if program use a XML files, the first step may be determining, which
XML-library is used for processing, since standard (or well-known) libraries are usually used
instead of self-made one.}

\index{SAP}
\index{Windows!PDB}
\RU{К примеру, однажды я пытался разобраться как происходит компрессия/декомпрессия сетевых пакетов в SAP 6.0. 
Это очень большая программа, но к ней идет подробный .\gls{PDB}-файл с отладочной информацией, и это очень удобно. 
Я в конце концов пришел к тому что одна из функций декомпрессирующая пакеты называется CsDecomprLZC(). 
Не сильно раздумывая, я решил погуглить и оказалось что функция с таким же названием имеется в MaxDB
(это опен-сорсный проект SAP)\footnote{Больше об этом в соответствующей секции~(\ref{sec:SAPGUI})}.}
\EN{For example, once upon a time I tried to understand how SAP 6.0 network packets compression/decompression 
was working.
It is a huge software, but a detailed .\gls{PDB} with debugging information is present, 
and that is cozily.
I finally came to idea that one of the functions doing decompressing of network packet called CsDecomprLZC().
Immediately I tried to google its name and I quickly found the function named as the same is used in MaxDB
(it is open-source SAP project)\footnote{More about it in relevant section~(\ref{sec:SAPGUI})}.}

\url{http://www.google.com/search?q=CsDecomprLZC}

\RU{Каково же было мое удивление, когда оказалось, что в MaxDB используется точно такой же алгоритм, 
скорее всего, с таким же исходником.}
\EN{Astoundingly, MaxDB and SAP 6.0 software shared likewise code for network packets compression/decompression.}

\section{\IFRU{Идентификация исполняемых файлов}{Identification of executable files}}

\subsection{Microsoft Visual C++}

\IFRU{Версии MSVC и DLL которые могут быть импортированы}{MSVC versions and DLLs which may be imported}:

\begin{center}
\begin{tabular}{ | l | l | l | l | l | }
\hline
\cellcolor{blue!25} \IFRU{Маркетинговая версия}{Marketing version} & 
\cellcolor{blue!25} \IFRU{Внутренняя версия}{Internal version} & 
\cellcolor{blue!25} \IFRU{Версия }{}CL.EXE\IFRU{}{ version} &
\cellcolor{blue!25} \IFRU{Импортируемые DLL}{DLLs may be imported} &
\cellcolor{blue!25} \IFRU{Дата выхода}{Release date} \\
\hline
% 4.0, April 1995
% 97 & 5.0 & February 1997
6		&  6.0 & 12.00 & msvcrt.dll, msvcp60.dll    & June 1998 \\
\hline
.NET (2002)	&  7.0 & 13.00 & msvcr70.dll, msvcp70.dll   & February 13, 2002 \\
\hline
.NET 2003	&  7.1 & 13.10 & msvcr71.dll, msvcp71.dll   & April 24, 2003 \\
\hline
2005		&  8.0 & 14.00 & msvcr80.dll, msvcp80.dll   & November 7, 2005 \\
\hline
2008		&  9.0 & 15.00 & msvcr90.dll, msvcp90.dll   & November 19, 2007 \\
\hline
2010		& 10.0 & 16.00 & msvcr100.dll, msvcp100.dll & April 12, 2010 \\
\hline
2012		& 11.0 & 17.00 & msvcr110.dll, msvcp110.dll & September 12, 2012 \\
\hline
2013		& 12.0 & 18.00 & msvcr120.dll, msvcp120.dll & October 17, 2013 \\
\hline
\end{tabular}
\end{center}

msvcp*.dll \IFRU{содержит ф-ции связанные с Си++, так что если она импортируется, скорее всего, 
вы имеете дело с программой на Си++}{contain C++-related functions, so, if it is imported, 
this is probably C++ program}.

\subsubsection{Name mangling}

\IFRU{Имена обычно начинаются с символа}{Names are usually started with} \TT{?}\IFRU{}{ symbol}.

\IFRU{О}{Read more about MSVC} \gls{name mangling} \IFRU{в MSVC читайте также здесь}{here}: \ref{namemangling}.

\subsection{GCC}
\index{GCC}

\IFRU{Кроме компиляторов под *NIX, GCC имеется также и для win32-окружения: в виде}
{Aside from *NIX targets, GCC is also present in win32 environment: in form of} Cygwin \AndENRU MinGW.

\subsubsection{Name mangling}

\IFRU{Имена обычно начинаются с символов}{Names are usually started with} \TT{\_Z}\IFRU{}{ symbols}.

\IFRU{О}{Read more about GCC} \gls{name mangling} \IFRU{в GCC читайте также здесь}{here}: \ref{namemangling}.

\subsubsection{Cygwin}
\index{Cygwin}

cygwin1.dll \IFRU{часто импортируется}{is often imported}.

\subsubsection{MinGW}
\index{MinGW}

msvcrt.dll \IFRU{может импортироваться}{may be imported}.

\subsection{Borland}
\index{Borland Delphi}
\index{Borland C++Builder}

\IFRU{Вот пример \gls{name mangling} в Borland Delphi и C++Builder}
{Here is an example of Borland Delphi and C++Builder \gls{name mangling}}:

\lstinputlisting{digging_into_code/identification/borland_mangling.txt}

\IFRU{Имена всегда начинаются с символа}{Names are always started with} \TT{@} 
\IFRU{затем следует имя класса, имя метода
и закодированные типы аргументов}{symbol, then class name came, method name, and encoded method argument types}.

\IFRU{Эти имена могут присутствовать с импортах .exe, экспортах .dll, отладочной информации, итд}
{These names can be in .exe imports, .dll exports, debug data, etc}.

Borland Visual Component Libraries (VCL) \IFRU{находятся в файлах .bpl вместо .dll, например}
{are stored in .bpl files instead of .dll ones, for example}, vcl50.dll, rtl60.dll.

\IFRU{Другие DLL которые могут импортироваться}{Other DLL might be imported}: BORLNDMM.DLL.

\subsubsection{Delphi}

\IFRU{Почти все исполняемые файлы имеют текстовую строку}{Almost all Delphi executables has} ``Boolean'' 
\IFRU{в самом начале сегмента кода, среди остальных имен типов}
{text string at the very beginning of code segment, along with other type names}.

\IFRU{Вот очень характерное для Delphi начало сегмента \TT{.text}, 
этот блок следует сразу за заголовком win32 PE-файла}
{This is a very typical beginning of \TT{.text} 
segment of a Delphi program, this block came right after win32 PE file header}:

\lstinputlisting{digging_into_code/identification/delphi.txt}

% binary files might be also here
\chapter{\RU{Связь с внешним миром}\EN{Communication with the outer world} (win32)}

\RU{Иногда, чтобы понять что делает та или иная функция, можно её не разбирать, а просто посмотреть на её входы и выходы.}%
\EN{Sometimes it's enough to observe some function's inputs and outputs in order to understand what it does.}
\RU{Так можно сэкономить время}\EN{That way you can save time}.

\RU{Обращения к файлам и реестру}\EN{Files and registry access}: 
\RU{для самого простого анализа может помочь утилита}%
\EN{for the very basic analysis, } Process Monitor\footnote{\url{http://go.yurichev.com/17301}}
\RU{от}\EN{utility from} SysInternals\EN{ can help}.

\RU{Для анализа обращения программы к сети, может помочь}%
\EN{For the basic analysis of network accesses,} Wireshark\footnote{\url{http://go.yurichev.com/17303}}\EN{ can be useful}.

\RU{Затем всё-таки придётся смотреть внутрь}\EN{But then you will have to to look inside anyway}. \\
\\
\RU{Первое на что нужно обратить внимание, это какие функции из \ac{API} \ac{OS} и какие функции стандартных библиотек используются.}%
\EN{The first thing to look for is which functions from the \ac{OS}'s \ac{API}s and standard libraries are used.}

\RU{Если программа поделена на главный исполняемый файл и группу DLL-файлов, то имена функций в этих DLL, бывает так, могут помочь.}%
\EN{If the program is divided into a main executable file and a group of DLL files, sometimes the names of the functions in these DLLs can help.}

\RU{Если нас интересует, что именно приводит к вызову \TT{MessageBox()} с определенным текстом, 
то первое что можно попробовать сделать: найти в сегменте данных этот текст, найти ссылки на него, и найти, 
откуда может передаться управление к интересующему нас вызову \TT{MessageBox()}.}%
\EN{If we are interested in exactly what can lead to a call to \TT{MessageBox()} with specific text, 
we can try to find this text in the data segment, find the references to it and find the points
from which the control may be passed to the \TT{MessageBox()} call we're interested in.}

\index{\CStandardLibrary!rand()}
\RU{Если речь идет о компьютерной игре, и нам интересно какие события в ней более-менее случайны, 
мы можем найти функцию \rand или её заменитель (как алгоритм Mersenne twister), и посмотреть, 
из каких мест эта функция вызывается и что самое главное: как используется результат этой функции.}%
\EN{If we are talking about a video game and we're interested in which events are more or less random in it,
we may try to find the \rand function or its replacements (like the Mersenne twister algorithm) and find the places
from which those functions are called, and more importantly, how are the results used.}
\ifx\LITE\undefined
% BUG in varioref: http://tex.stackexchange.com/questions/104261/varioref-vref-or-vpageref-at-page-boundary-may-loop
\RU{Один пример}\EN{One example}: \ref{chap:color_lines}. 
\fi

\RU{Но если это не игра, а \rand используется, то также весьма любопытно, зачем. 
Бывают неожиданные случаи вроде использования \rand в алгоритме для сжатия данных (для имитации шифрования):}%
\EN{But if it is not a game, and \rand is still used, it is also interesting to know why.
There are cases of unexpected \rand usage in data compression algorithms (for encryption imitation):}
\href{http://go.yurichev.com/17221}{blog.yurichev.com}.

\section{\RU{Часто используемые функции}\EN{Often used functions in the} Windows API}

\RU{Это функции которые можно увидеть в числе импортируемых}\EN{These functions may be among the imported}.
\RU{Но также нельзя забывать, что далеко не все они были использованы в коде написанном автором}%
\EN{It is worth to note that not every function might be used in the code that was written by the programmer}.
\RU{Немалая часть может вызываться из библиотечных функций и}%
\EN{A lot of functions might be called from library functions and} \ac{CRT}\RU{-кода}\EN{ code}.

\begin{itemize}

\item
\RU{Работа с реестром}\EN{Registry access} (advapi32.dll): 
RegEnumKeyEx\footnote{\href{http://go.yurichev.com/17228}{MSDN}}
\footnote{
	\RU{Может иметь суффикс -A для ASCII-версии и -W для Unicode-версии}
	\EN{May have the -A suffix for the ASCII version and -W for the Unicode version}
	\label{note1}},
RegEnumValue\footnote{\href{http://go.yurichev.com/17229}{MSDN}}
\footnoteref{note1},
RegGetValue\footnote{\href{http://go.yurichev.com/17230}{MSDN}}
\footnoteref{note1},
RegOpenKeyEx\footnote{\href{http://go.yurichev.com/17231}{MSDN}}
\footnoteref{note1},
RegQueryValueEx\footnote{\href{http://go.yurichev.com/17232}{MSDN}}
\footnoteref{note1}.

\item
\RU{Работа с текстовыми .ini-файлами}\EN{Access to text .ini-files} (kernel32.dll): 
GetPrivateProfileString
\footnote{\href{http://go.yurichev.com/17233}{MSDN}}
\footnoteref{note1}.

\item
\RU{Диалоговые окна}\EN{Dialog boxes} (user32.dll): 
MessageBox
\footnote{\href{http://go.yurichev.com/17234}{MSDN}}
\footnoteref{note1}, 
MessageBoxEx
\footnote{\href{http://go.yurichev.com/17235}{MSDN}}
\footnoteref{note1},
SetDlgItemText
\footnote{\href{http://go.yurichev.com/17236}{MSDN}}
\footnoteref{note1},
GetDlgItemText
\footnote{\href{http://go.yurichev.com/17237}{MSDN}}
\footnoteref{note1}.

\item
\RU{Работа с ресурсами}\EN{Resources access} 
\ifx\LITE\undefined
(\myref{PEresources})
\fi
: (user32.dll): LoadMenu
\footnote{\href{http://go.yurichev.com/17238}{MSDN}}
\footnoteref{note1}.
\item
\RU{Работа с TCP/IP-сетью}\EN{TCP/IP networking} (ws2\_32.dll):
WSARecv
\footnote{\href{http://go.yurichev.com/17239}{MSDN}},
WSASend
\footnote{\href{http://go.yurichev.com/17240}{MSDN}}.

\item
\RU{Работа с файлами}\EN{File access} (kernel32.dll):
CreateFile
\footnote{\href{http://go.yurichev.com/17241}{MSDN}}
\footnoteref{note1},
ReadFile
\footnote{\href{http://go.yurichev.com/17242}{MSDN}},
ReadFileEx
\footnote{\href{http://go.yurichev.com/17243}{MSDN}},
WriteFile
\footnote{\href{http://go.yurichev.com/17244}{MSDN}},
WriteFileEx
\footnote{\href{http://go.yurichev.com/17245}{MSDN}}.

\item
\RU{Высокоуровневая работа с}\EN{High-level access to the} Internet
(wininet.dll):
WinHttpOpen
\footnote{\href{http://go.yurichev.com/17246}{MSDN}}.

\item
\RU{Проверка цифровой подписи исполняемого файла}
\EN{Checking the digital signature of an executable file} (wintrust.dll):
WinVerifyTrust
\footnote{\href{http://go.yurichev.com/17247}{MSDN}}.

\item
\RU{Стандартная библиотека MSVC (в случае динамического связывания)}%
\EN{The standard MSVC library (if it's linked dynamically)} (msvcr*.dll):
assert, itoa, ltoa, open, printf, read, strcmp, atol, atoi, fopen, fread, fwrite, memcmp, rand,
strlen, strstr, strchr.

\end{itemize}

\section{tracer: \RU{Перехват всех функций в отдельном модуле}%
\EN{Intercepting all functions in specific module}}
\index{tracer}

\index{x86!\Instructions!INT3}
\RU{В \tracer есть поддержка точек останова INT3, хотя и срабатывающие только один раз, но зато их можно установить на все
сразу функции в некоей DLL.}%
\EN{There are INT3 breakpoints in the \tracer, that are triggered only once, however, they can be set for all functions
in a specific DLL.}

\begin{lstlisting}
--one-time-INT3-bp:somedll.dll!.*
\end{lstlisting}

\RU{Либо, поставим INT3-прерывание на все функции, имена которых начинаются с префикса \TT{xml}:}%
\EN{Or, let's set INT3 breakpoints on all functions with the \TT{xml} prefix in their name:}

\begin{lstlisting}
--one-time-INT3-bp:somedll.dll!xml.*
\end{lstlisting}

\RU{В качестве обратной стороны медали, такие прерывания срабатывают только один раз.}%
\EN{On the other side of the coin, such breakpoints are triggered only once.}

\RU{Tracer покажет вызов какой-либо функции, если он случится, но только один раз.}%
\EN{Tracer will show the call of a function, if it happens, but only once.}
\RU{Еще один недостаток\EMDASH{}увидеть аргументы функции также нельзя.}%
\EN{Another drawback\EMDASH{}it is impossible to see the function's arguments.}

\RU{Тем не менее, эта возможность очень удобна для тех ситуаций, 
когда вы знаете что некая программа использует некую DLL,
но не знаете какие именно функции в этой DLL.}%
\EN{Nevertheless, this feature is very useful when you know that the program uses a DLL,
but you do not know which functions are actually used.}
\RU{И функций много.}\EN{And there are a lot of functions.}\PTBRph{}\ESph{}\PLph{}\ITAph{} \\
\\
\index{Cygwin}
\RU{Например, попробуем узнать, что использует cygwin-утилита uptime:}%
\EN{For example, let's see, what does the uptime utility from cygwin use:}

\begin{lstlisting}
tracer -l:uptime.exe --one-time-INT3-bp:cygwin1.dll!.*
\end{lstlisting}

\RU{Так мы можем увидеть все функции из библиотеки cygwin1.dll, которые были вызваны хотя бы один раз, и откуда}%
\EN{Thus we may see all that cygwin1.dll library functions that were called at least once, and where from}:

\lstinputlisting{digging_into_code/uptime_cygwin.txt}


\section{\IFRU{Строки}{Strings}}
\label{sec:digging_strings}

\IFRU{Очень сильно помогают отладочные сообщения, если они имеются. В некотором смысле, отладочные сообщения, 
это отчет о том, что сейчас происходит в программе. Зачастую, это \printf-подобные функции, 
которые пишут куда-нибудь в лог, а бывает так что и не пишут ничего, но вызовы остались, так как эта сборка ~--- 
не отладочная, а release.}
{Debugging messages are often very helpful if present.
In some sense, debugging messages are reporting
about what's going on in program right now. Often these are \printf-like functions,
which writes to log-files, and sometimes, not writing anything but calls are still present 
since this build is not a debug build but release one.}
\index{\oracle}
\IFRU{Если в отладочных сообщениях дампятся значения некоторых локальных или глобальных переменных, 
это тоже может помочь, как минимум, узнать их имена. 
Например, в \oracle одна из таких функций: \TT{ksdwrt()}.}
{If local or global variables are dumped in debugging messages, it might be helpful as well 
since it is possible to get variable names at least.
For example, one of such functions in \oracle is \TT{ksdwrt()}.}

\newcommand{\CONUSONE}{http://blog.yurichev.com/node/32}
\newcommand{\CONUSTWO}{http://blog.yurichev.com/node/43}

\IFRU{Осмысленные текстовые строки вообще очень сильно могут помочь. 
Дизассемблер \IDA может сразу указать, из какой функции и из какого её места используется эта строка. 
Встречаются и \href{\CONUSONE}{смешные случаи}.}
{Meaningful text strings are often helpful.
\IDA disassembler may show from which function and from which point this specific string is used.
Funny cases \href{\CONUSONE}{sometimes happen}.}

\IFRU{Сообщения об ошибках также могут помочь найти то что нужно. 
В \oracle сигнализация об ошибках проходит при помощи вызова некоторой группы функций. 
\href{\CONUSTWO}{Тут еще немного об этом}.}
{Error messages may help us as well.
In \oracle, errors are reporting using group of functions.
\href{\CONUSTWO}{More about it}.}

\index{Error messages}
\IFRU{Можно довольно быстро найти, какие функции сообщают о каких ошибках, и при каких условиях.}
{It is possible to find very quickly, which functions reporting about errors and in which conditions.}
\IFRU{Это, кстати, одна из причин, почему в защите софта от копирования, 
бывает так, что сообщение об ошибке заменяется 
невнятным кодом или номером ошибки. Мало кому приятно, если взломщик быстро поймет, 
из за чего именно срабатывает защита от копирования, просто по сообщению об ошибке.}
{By the way, it is often a reason why copy-protection systems has inarticulate cryptic error messages 
or just error numbers. No one happy when software cracker quickly understand why copy-protection
is triggered just by error message.}

% TODO software protection... set ref to section about dongle for SCO UNIX...

\chapter{\RU{Вызовы assert()}\EN{Calls to assert()}}
\index{\CStandardLibrary!assert()}
\RU{Может также помочь наличие \TT{assert()} в коде: обычно этот макрос оставляет название файла-исходника, 
номер строки, и условие.}
\EN{Sometimes \TT{assert()} macro presence is useful too: 
commonly this macro leaves source file name, line number and condition in code.}

\RU{Наиболее полезная информация содержится в assert-условии, по нему можно судить по именам переменных
или именам полей структур. Другая полезная информация ~--- это имена файлов, по их именам можно попытаться
предположить, что там за код. Также, по именам файлов можно опознать какую-либо очень известную опен-сорсную
библиотеку.}
\EN{Most useful information is contained in assert-condition, we can deduce variable names, or structure field
names from it. Another useful piece of information is file names~---we can try to deduce what type of
code is here.
Also by file names it is possible to recognize a well-known open-source libraries.}

\lstinputlisting[caption=\RU{Пример информативных вызовов assert()}
\EN{Example of informative assert() calls}]{digging_into_code/assert_examples.lst}

\RU{Полезно ``гуглить'' и условия и имена файлов, это может вывести вас к опен-сорсной бибилотеке.
Например, если ``погуглить'' ``sp->lzw\_nbits <= BITS\_MAX'', 
это вполне предсказуемо выводит на опенсорсный код, что-то связанное с LZW-компрессией.}
\EN{It is advisable to ``google'' both conditions and file names, that may lead us to open-source library.
For example, if to ``google'' ``sp->lzw\_nbits <= BITS\_MAX'', this predictably 
give us some open-source code, something related to LZW-compression.}

\chapter{\RU{Константы}\EN{Constants}}

\RU{Люди, включая программистов, часто используют круглые числа вроде}
\EN{Humans, including programmers, often use round numbers like} 10, 100, 1000, 
\RU{в т.ч. и в коде}\EN{in real life as well as in the code}.

\RU{Практикующие реверсеры, обычно, хорошо знают их в шестнадцатеричном представлении}
\EN{The practicing reverse engineer usually know them well in hexadecimal representation}:
10=0xA, 100=0x64, 1000=0x3E8, 10000=0x2710.

\RU{Иногда попадаются константы}\EN{The constants} \TT{0xAAAAAAAA} (10101010101010101010101010101010) \AndENRU \\
\TT{0x55555555} (01010101010101010101010101010101) \RU{\EMDASH{}это чередующиеся биты}\EN{ are also popular\EMDASH{}those
are composed of alternating bits}.
\RU{Это помогает отличить некоторый сигнал от сигнала где все биты включены (1111 \dots) или выключены (0000 \dots).}
\EN{That may help to distinguish some signal from the signal where all bits are turned on (1111 \dots) or off (0000 \dots).}
\RU{Например, константа}\EN{For example, the} \TT{0x55AA} \RU{используется как минимум в бут-секторе}\EN{constant
is used at least in the boot sector}, \ac{MBR}, 
\AndENRU \InENRU \EN{the }\ac{ROM} \RU{плат-расширений IBM-компьютеров}\EN{of IBM-compatible extension cards}.

\RU{Некоторые алгоритмы, особенно криптографические, используют хорошо различимые константы, 
которые при помощи \IDA легко находить в коде.}
\EN{Some algorithms, especially cryptographical ones use distinct constants, which are easy to find
in code using \IDA.}

\index{MD5}
\newcommand{\URLMD}{\RU{http://go.yurichev.com/17110}\EN{http://go.yurichev.com/17111}}

\RU{Например, алгоритм MD5\footnote{\href{\URLMD}{wikipedia}} инициализирует свои внутренние переменные так:}
\EN{For example, the MD5\footnote{\href{\URLMD}{wikipedia}} algorithm initializes its own internal variables like this:}

\begin{verbatim}
var int h0 := 0x67452301
var int h1 := 0xEFCDAB89
var int h2 := 0x98BADCFE
var int h3 := 0x10325476
\end{verbatim}

\RU{Если в коде найти использование этих четырех констант подряд\EMDASH{} очень высокая вероятность что эта функция имеет отношение к MD5.}
\EN{If you find these four constants used in the code in a row, it is very highly probable that this function is related to MD5.}\PTBRph{}\ESph{}\PLph{}\ITAph{} \\
\\
\RU{Еще такой пример это алгоритмы CRC16/CRC32, часто, алгоритмы вычисления контрольной суммы по CRC 
используют заранее заполненные таблицы, вроде}\EN{Another example are the CRC16/CRC32 algorithms, 
whose calculation algorithms often use precomputed tables like this one}:

\begin{lstlisting}[caption=linux/lib/crc16.c]
/** CRC table for the CRC-16. The poly is 0x8005 (x^16 + x^15 + x^2 + 1) */
u16 const crc16_table[256] = {
	0x0000, 0xC0C1, 0xC181, 0x0140, 0xC301, 0x03C0, 0x0280, 0xC241,
	0xC601, 0x06C0, 0x0780, 0xC741, 0x0500, 0xC5C1, 0xC481, 0x0440,
	0xCC01, 0x0CC0, 0x0D80, 0xCD41, 0x0F00, 0xCFC1, 0xCE81, 0x0E40,
	...
\end{lstlisting}

\ifx\LITE\undefined
\RU{См. также таблицу CRC32}\EN{See also the precomputed table for CRC32}: \myref{sec:CRC32}.
\fi

\section{Magic numbers}

\newcommand{\FNURLMAGIC}{\footnote{\href{http://go.yurichev.com/17112}{wikipedia}}}

\RU{Немало форматов файлов определяет стандартный заголовок файла где используются \IT{magic number}\FNURLMAGIC{}, один или даже несколько.}
\EN{A lot of file formats define a standard file header where a \IT{magic number(s)}\FNURLMAGIC{} is used, single one or even several.}

\index{MS-DOS}
\RU{Скажем, все исполняемые файлы для Win32 и MS-DOS начинаются с двух символов}
\EN{For example, all Win32 and MS-DOS executables start with the two characters} \q{MZ}\footnote{\href{http://go.yurichev.com/17113}{wikipedia}}.

\index{MIDI}
\RU{В начале MIDI-файла должно быть \q{MThd}. Если у нас есть использующая для чего-нибудь MIDI-файлы программа
очень вероятно, что она будет проверять MIDI-файлы на правильность хотя бы проверяя первые 4 байта.}
\EN{At the beginning of a MIDI file the \q{MThd} signature must be present. 
If we have a program which uses MIDI files for something,
it's very likely that it must check the file for validity by checking at least the first 4 bytes.}

\RU{Это можно сделать при помощи:}\EN{This could be done like this:}

\RU{(\IT{buf} указывает на начало загруженного в память файла)}
\EN{(\IT{buf} points to the beginning of the loaded file in memory)}

\begin{lstlisting}
cmp [buf], 0x6468544D ; "MThd"
jnz _error_not_a_MIDI_file
\end{lstlisting}

\index{\CStandardLibrary!memcmp()}
\index{x86!\Instructions!CMPSB}
\RU{\dots либо вызвав функцию сравнения блоков памяти \TT{memcmp()} или любой аналогичный код, 
вплоть до инструкции \TT{CMPSB} 
\ifx\LITE\undefined
(\myref{REPE_CMPSx})
\fi
.}
\EN{\dots or by calling a function for comparing memory blocks like \TT{memcmp()} or any other equivalent code
up to a \TT{CMPSB} 
\ifx\LITE\undefined
(\myref{REPE_CMPSx}) 
\fi
instruction.}

\RU{Найдя такое место мы получаем как минимум информацию о том, где начинается загрузка MIDI-файла, во-вторых, 
мы можем увидеть где располагается буфер с содержимым файла, и что еще оттуда берется, и как используется.}
\EN{When you find such point you already can say where the loading of the MIDI file starts,
also, we could see the location
of the buffer with the contents of the MIDI file, what is used from the buffer, and how.}

\subsection{DHCP}

\RU{Это касается также и сетевых протоколов. 
Например, сетевые пакеты протокола DHCP содержат так называемую \IT{magic cookie}: \TT{0x63538263}. 
Какой-либо код, генерирующий пакеты по протоколу DHCP где-то и как-то должен внедрять в пакет также и эту константу. 
Найдя её в коде мы сможем найти место где происходит это и не только это. 
Любая программа, получающая DHCP-пакеты, должна где-то как-то проверять \IT{magic cookie}, 
сравнивая это поле пакета с константой.}
\EN{This applies to network protocols as well.
For example, the DHCP protocol's network packets contains the so-called \IT{magic cookie}: \TT{0x63538263}.
Any code that generates DHCP packets somewhere must embed this constant into the packet.
If we find it in the code we may find where this happens and, not only that.
Any program which can receive DHCP packet must verify the \IT{magic cookie}, comparing it with the constant.}

\RU{Например, берем файл dhcpcore.dll из Windows 7 x64 и ищем эту константу. 
И находим, два раза: оказывается, эта константа используется в функциях с красноречивыми 
названиями}
\EN{For example, let's take the dhcpcore.dll file from Windows 7 x64 and search for the constant.
And we can find it, twice:
it seems that the constant is used in two functions with descriptive names 
like} \TT{DhcpExtractOptionsForValidation()} \AndENRU \TT{DhcpExtractFullOptions()}:

\begin{lstlisting}[caption=dhcpcore.dll (Windows 7 x64)]
.rdata:000007FF6483CBE8 dword_7FF6483CBE8 dd 63538263h          ; DATA XREF: DhcpExtractOptionsForValidation+79
.rdata:000007FF6483CBEC dword_7FF6483CBEC dd 63538263h          ; DATA XREF: DhcpExtractFullOptions+97
\end{lstlisting}

\RU{А вот те места в функциях где происходит обращение к константам:}
\EN{And here are the places where these constants are accessed:}

\begin{lstlisting}[caption=dhcpcore.dll (Windows 7 x64)]
.text:000007FF6480875F  mov     eax, [rsi]
.text:000007FF64808761  cmp     eax, cs:dword_7FF6483CBE8
.text:000007FF64808767  jnz     loc_7FF64817179
\end{lstlisting}

\RU{И:}\EN{And:}

\begin{lstlisting}[caption=dhcpcore.dll (Windows 7 x64)]
.text:000007FF648082C7  mov     eax, [r12]
.text:000007FF648082CB  cmp     eax, cs:dword_7FF6483CBEC
.text:000007FF648082D1  jnz     loc_7FF648173AF
\end{lstlisting}

\section{\RU{Поиск констант}\EN{Searching for constants}}

\RU{В \IDA это очень просто, Alt-B или Alt-I.}
\EN{It is easy in \IDA: Alt-B or Alt-I.}
\index{binary grep}
\RU{А для поиска константы в большом количестве файлов, либо для поиска их в неисполняемых файлах, 
имеется небольшая утилита}%
\EN{And for searching for a constant in a big pile of files, or for searching in non-executable files,
there is a small utility called}
\IT{binary grep}\footnote{\BGREPURL}.

\chapter{\RU{Поиск нужных инструкций}\EN{Finding the right instructions}}

\RU{Если программа использует инструкции сопроцессора, и их не очень много, 
то можно попробовать вручную проверить отладчиком какую-то из них.}
\EN{If the program is utilizing FPU instructions and there are very few of them in the code,
one can try to check each one manually with a debugger.}\PTBRph{}\ESph{}\PLph{}\ITAph{}\\
\\
\RU{К примеру, нас может заинтересовать, при помощи чего Microsoft Excel считает 
результаты формул, введенных пользователем. Например, операция деления.}
\EN{For example, we may be interested how Microsoft Excel calculates the formulae entered by user.
For example, the division operation.}

\index{\GrepUsage}
\index{x86!\Instructions!FDIV}
\RU{Если загрузить excel.exe (из Office 2010) версии 14.0.4756.1000 в \IDA, затем сделать полный листинг 
и найти все инструкции \FDIV (но кроме тех, которые в качестве второго операнда используют константы\EMDASH{}они, 
очевидно, не подходят нам):}
\EN{If we load excel.exe (from Office 2010) version 14.0.4756.1000 into \IDA, make a full listing
and to find every \FDIV instruction (except the ones which use constants as a second 
operand\EMDASH{}obviously, they do not suit us):}\PTBRph{}\ESph{}\PLph{}\ITAph{}\\

\begin{lstlisting}
cat EXCEL.lst | grep fdiv | grep -v dbl_ > EXCEL.fdiv
\end{lstlisting}

\RU{\dots то окажется, что их всего 144.}\EN{\dots then we see that there are 144 of them.}\PTBRph{}\ESph{}\PLph{}\ITAph{}\\
\\
\RU{Мы можем вводить в Excel строку вроде \TT{=(1/3)} и проверить все эти инструкции.}
\EN{We can enter a string like \TT{=(1/3)} in Excel and check each instruction.}\PTBRph{}\ESph{}\PLph{}\ITAph{}\\
\\
\index{tracer}
\RU{Проверяя каждую инструкцию в отладчике или \tracer 
(проверять эти инструкции можно по 4 за раз), 
окажется, что нам везет и срабатывает всего лишь 14-я по счету:}
\EN{By checking each instruction in a debugger or \tracer
(one may check 4 instruction at a time),
we get lucky and the sought-for instruction is just the 14th:}

\begin{lstlisting}
.text:3011E919 DC 33                                fdiv    qword ptr [ebx]
\end{lstlisting}

\begin{lstlisting}
PID=13944|TID=28744|(0) 0x2f64e919 (Excel.exe!BASE+0x11e919)
EAX=0x02088006 EBX=0x02088018 ECX=0x00000001 EDX=0x00000001
ESI=0x02088000 EDI=0x00544804 EBP=0x0274FA3C ESP=0x0274F9F8
EIP=0x2F64E919
FLAGS=PF IF
FPU ControlWord=IC RC=NEAR PC=64bits PM UM OM ZM DM IM 
FPU StatusWord=
FPU ST(0): 1.000000
\end{lstlisting}

\RU{В \ST{0} содержится первый аргумент (1), второй содержится в}
\EN{\ST{0} holds the first argument (1) and second one is in} \TT{[EBX]}.\\
\\
\index{x86!\Instructions!FDIV}
\RU{Следующая за \FDIV инструкция (\TT{FSTP}) записывает результат в память:}
\EN{The instruction after \FDIV (\TT{FSTP}) writes the result in memory:}\\

\begin{lstlisting}
.text:3011E91B DD 1E                                fstp    qword ptr [esi]
\end{lstlisting}

\RU{Если поставить breakpoint на ней, то мы можем видеть результат:}
\EN{If we set a breakpoint on it, we can see the result:}

\begin{lstlisting}
PID=32852|TID=36488|(0) 0x2f40e91b (Excel.exe!BASE+0x11e91b)
EAX=0x00598006 EBX=0x00598018 ECX=0x00000001 EDX=0x00000001
ESI=0x00598000 EDI=0x00294804 EBP=0x026CF93C ESP=0x026CF8F8
EIP=0x2F40E91B
FLAGS=PF IF
FPU ControlWord=IC RC=NEAR PC=64bits PM UM OM ZM DM IM 
FPU StatusWord=C1 P 
FPU ST(0): 0.333333
\end{lstlisting}

\RU{А также, в рамках пранка\footnote{practical joke}, модифицировать его на лету:}
\EN{Also as a practical joke, we can modify it on the fly:}\PTBRph{}\ESph{}\PLph{}\ITAph{}\\

\begin{lstlisting}
tracer -l:excel.exe bpx=excel.exe!BASE+0x11E91B,set(st0,666)
\end{lstlisting}

\begin{lstlisting}
PID=36540|TID=24056|(0) 0x2f40e91b (Excel.exe!BASE+0x11e91b)
EAX=0x00680006 EBX=0x00680018 ECX=0x00000001 EDX=0x00000001
ESI=0x00680000 EDI=0x00395404 EBP=0x0290FD9C ESP=0x0290FD58
EIP=0x2F40E91B
FLAGS=PF IF
FPU ControlWord=IC RC=NEAR PC=64bits PM UM OM ZM DM IM 
FPU StatusWord=C1 P 
FPU ST(0): 0.333333
Set ST0 register to 666.000000
\end{lstlisting}

\RU{Excel показывает в этой ячейке 666, что окончательно убеждает нас в том, что мы нашли нужное место.}
\EN{Excel shows 666 in the cell, finally convincing us that we have found the right point.}

\begin{figure}[H]
\centering
\includegraphics[scale=\NormalScale]{digging_into_code/Excel_prank.png}
\caption{\RU{Пранк сработал}\EN{The practical joke worked}}
\end{figure}

\RU{Если попробовать ту же версию Excel, только x64, то окажется что там инструкций \FDIV всего 12, 
причем нужная нам\EMDASH{}третья по счету.}
\EN{If we try the same Excel version, but in x64,
we will find only 12 \FDIV instructions there,
and the one we looking for is the third one.}

\begin{lstlisting}
tracer.exe -l:excel.exe bpx=excel.exe!BASE+0x1B7FCC,set(st0,666)
\end{lstlisting}

\index{x86!\Instructions!DIVSD}
\RU{Видимо, все дело в том, что много операций деления переменных типов \Tfloat и \Tdouble 
компилятор заменил на SSE-инструкции вроде \TT{DIVSD}, 
коих здесь теперь действительно много (\TT{DIVSD} присутствует в количестве 268 инструкций).}
\EN{It seems that a lot of division operations of \Tfloat and \Tdouble types, were replaced by the compiler with SSE instructions
like \TT{DIVSD} (\TT{DIVSD} is present 268 times in total).}

\chapter{\RU{Подозрительные паттерны кода}\EN{Suspicious code patterns}}

\section{\RU{Инструкции XOR}\EN{XOR instructions}}
\index{x86!\Instructions!XOR}

\RU{Инструкции вроде}\EN{Instructions like} \TT{XOR op, op} (\RU{например}\EN{for example}, \TT{XOR EAX, EAX}) 
\RU{обычно используются для обнуления регистра,
однако, если операнды разные, то применяется операция именно}\EN{are usually used for setting the register value
to zero, but if the operands are different, the} \q{\RU{исключающего или}\EN{exclusive or}}\EN{ operation
is executed}.
\RU{Эта операция очень редко применяется в обычном программировании, но применяется очень часто в криптографии,
включая любительскую.}
\EN{This operation is rare in common programming, but widespread in cryptography,
including amateur one.}
\RU{Особенно подозрительно, если второй операнд\EMDASH{}это большое число}\EN{It's especially suspicious if the
second operand is a big number}.
\RU{Это может указывать на шифрование, вычисление контрольной суммы,}
\EN{This may point to encrypting/decrypting, checksum computing,}\etc{}.\\
\\
\ifx\LITE\undefined
\RU{Одно из исключений из этого наблюдения о котором стоит сказать, то, что генерация и проверка значения \q{канарейки}
(\myref{subsec:BO_protection}) часто происходит, используя инструкцию \XOR.}
\EN{One exception to this observation worth noting is the \q{canary} (\myref{subsec:BO_protection}). 
Its generation and checking are often done using the \XOR instruction.} \\
\\
\fi
\index{AWK}
\RU{Этот AWK-скрипт можно использовать для обработки листингов (.lst) созданных \IDA{}}
\EN{This AWK script can be used for processing \IDA{} listing (.lst) files}:

\begin{lstlisting}
gawk -e '$2=="xor" { tmp=substr($3, 0, length($3)-1); if (tmp!=$4) if($4!="esp") if ($4!="ebp") { print $1, $2, tmp, ",", $4 } }' filename.lst
\end{lstlisting}

\ifx\LITE\undefined
\RU{Нельзя также забывать,
что если использовать подобный скрипт, то, возможно, он захватит и неверно дизассемблированный
код}\EN{It is also worth noting that this kind of script can also match incorrectly disassembled code} 
(\myref{sec:incorrectly_disasmed_code}).
\fi

\section{\RU{Вручную написанный код на ассемблере}\EN{Hand-written assembly code}}

\index{Function prologue}
\index{Function epilogue}
\index{x86!\Instructions!LOOP}
\index{x86!\Instructions!RCL}
\RU{Современные компиляторы не генерируют инструкции \TT{LOOP} и \TT{RCL}. 
С другой стороны, эти инструкции хорошо знакомы кодерам, предпочитающим писать прямо на ассемблере. 
\ifx\LITE\undefined
Подобные инструкции отмечены как (M) в списке инструкций в приложении: 
\myref{sec:x86_instructions}.
\fi
Если такие инструкции встретились, можно сказать с какой-то вероятностью, что этот фрагмент кода написан вручную.}
\EN{Modern compilers do not emit the \TT{LOOP} and \TT{RCL} instructions.
On the other hand, these instructions are well-known to coders who like to code directly in assembly language.
If you spot these, it can be said that there is a high probability that this fragment of code was hand-written.
\ifx\LITE\undefined
Such instructions are marked as (M) in the instructions list in this appendix: 
\myref{sec:x86_instructions}.
\fi
}\PTBRph{}\ESph{}\PLph{}\ITAph{}\\
\\
\RU{Также, пролог/эпилог функции обычно не встречается в ассемблерном коде, написанном вручную.}
\EN{Also the function prologue/epilogue are not commonly present in hand-written assembly.}\\
\\
\RU{Как правило, в вручную написанном коде, нет никакого четкого метода передачи аргументов в 
функцию}
\EN{Commonly there is no fixed system for passing arguments to functions in the hand-written
code}.\\
\\
\RU{Пример из ядра}\EN{Example from the} Windows 2003\EN{ kernel} 
(\RU{файл }ntoskrnl.exe\EN{ file}):

\begin{lstlisting}
MultiplyTest proc near               ; CODE XREF: Get386Stepping
             xor     cx, cx
loc_620555:                          ; CODE XREF: MultiplyTest+E
             push    cx
             call    Multiply
             pop     cx
             jb      short locret_620563
             loop    loc_620555
             clc
locret_620563:                       ; CODE XREF: MultiplyTest+C
             retn
MultiplyTest endp

Multiply     proc near               ; CODE XREF: MultiplyTest+5
             mov     ecx, 81h
             mov     eax, 417A000h
             mul     ecx
             cmp     edx, 2
             stc
             jnz     short locret_62057F
             cmp     eax, 0FE7A000h
             stc
             jnz     short locret_62057F
             clc
locret_62057F:                       ; CODE XREF: Multiply+10
                                     ; Multiply+18
             retn
Multiply     endp
\end{lstlisting}

\RU{Действительно, если заглянуть в исходные коды}\EN{Indeed, if we look in the} 
\ac{WRK} v1.2\RU{, данный код можно найти в файле}\EN{ source code, this code
can be found easily in file} 
\IT{WRK-v1.2\textbackslash{}base\textbackslash{}ntos\textbackslash{}ke\textbackslash{}i386\textbackslash{}cpu.asm}.

\chapter{\RU{Использование magic numbers для трассировки}\EN{Using magic numbers while tracing}}

\RU{Нередко бывает нужно узнать, как используется то или иное значение, прочитанное из файла либо взятое из пакета,
принятого по сети. Часто, ручное слежение за нужной переменной это трудный процесс. Один из простых методов (хотя и не
полностью надежный на 100\%) это использование вашей собственной \IT{magic number}.}
\EN{Often, our main goal is to understand how the program uses a value that was either read from file or received via network. 
The manual tracing of a value is often a very labour-intensive task. One of the simplest techniques for this (although not 100\% reliable) 
is to use your own \IT{magic number}.}

\RU{Это чем-то напоминает компьютерную томографию: пациенту перед сканированием вводят в кровь 
рентгеноконтрастный препарат, хорошо отсвечивающий в рентгеновских лучах.
Известно, как кровь нормального человека
расходится, например, по почкам, и если в этой крови будет препарат, то при томографии будет хорошо видно,
достаточно ли хорошо кровь расходится по почкам и нет ли там камней, например, и прочих образований.}
\EN{This resembles X-ray computed tomography is some sense: a radiocontrast agent is injected into the patient's blood,
which is then used to improve the visibility of the patient's internal structure in to the X-rays.
It is well known how the blood of healthy humans
percolates in the kidneys and if the agent is in the blood, it can be easily seen on tomography, how blood is percolating,
and are there any stones or tumors.}

\RU{Мы можем взять 32-битное число вроде \TT{0x0badf00d}, либо чью-то дату рождения вроде \TT{0x11101979} 
и записать это, занимающее 4 байта число, в какое-либо место файла используемого исследуемой нами программой.}
\EN{We can take a 32-bit number like \TT{0x0badf00d}, or someone's birth date like \TT{0x11101979}
and write this 4-byte number to some point in a file used by the program we investigate.}

\index{\GrepUsage}
\index{tracer}
\RU{Затем, при трассировки этой программы, в том числе, при помощи \tracer в режиме 
\IT{code coverage}, а затем при помощи
\IT{grep} или простого поиска по текстовому файлу с результатами трассировки, мы можем легко увидеть, в каких местах кода использовалось 
это значение, и как.}
\EN{Then, while tracing this program with \tracer in \IT{code coverage} mode, with the help of \IT{grep}
or just by searching in the text file (of tracing results), we can easily see where the value was used and how.}

\RU{Пример результата работы \tracer в режиме \IT{cc}, к которому легко применить утилиту \IT{grep}}\EN{Example 
of \IT{grepable} \tracer results in \IT{cc} mode}:

\begin{lstlisting}
0x150bf66 (_kziaia+0x14), e=       1 [MOV EBX, [EBP+8]] [EBP+8]=0xf59c934 
0x150bf69 (_kziaia+0x17), e=       1 [MOV EDX, [69AEB08h]] [69AEB08h]=0 
0x150bf6f (_kziaia+0x1d), e=       1 [FS: MOV EAX, [2Ch]] 
0x150bf75 (_kziaia+0x23), e=       1 [MOV ECX, [EAX+EDX*4]] [EAX+EDX*4]=0xf1ac360 
0x150bf78 (_kziaia+0x26), e=       1 [MOV [EBP-4], ECX] ECX=0xf1ac360 
\end{lstlisting}
% TODO: good example!
\RU{Это справедливо также и для сетевых пакетов.
Важно только, чтобы наш \IT{magic number} был как можно более уникален и не присутствовал в самом коде.}
\EN{This can be used for network packets as well.
It is important for the \IT{magic number} to be unique and not to be present in the program's code.}

\newcommand{\DOSBOXURL}{\href{http://go.yurichev.com/17222}{blog.yurichev.com}}

\index{DosBox}
\index{MS-DOS}
\RU{Помимо \tracer, такой эмулятор MS-DOS как DosBox, в режиме heavydebug, может писать в отчет информацию обо всех
состояниях регистра на каждом шаге исполнения программы\footnote{См. также мой пост в блоге об этой возможности в 
DosBox: \DOSBOXURL{}}, так что этот метод может пригодиться и для исследования программ под DOS.}\EN{Aside of 
the \tracer, DosBox (MS-DOS emulator) in heavydebug mode
is able to write information about all registers' states for each executed instruction of the program to a plain text file\footnote{See also my 
blog post about this DosBox feature: \DOSBOXURL{}}, so this technique may be useful for DOS programs as well.}



\chapter{\RU{Прочее}\EN{Other things}}

\section{\EN{General idea}\RU{Общая идея}}

\RU{Нужно стараться как можно чаще ставить себя на место программиста и задавать себе вопрос, 
как бы вы сделали ту или иную вещь в этом случае и в этой программе.}
\EN{Reverse engineer should try to be in programmer's shoes as often as possible. 
To take his/her viewpoint and ask himself, how one solve some task here in this case.}

\section{\Cpp}

\ac{RTTI}~(\ref{RTTI})-\RU{информация также может быть полезна для идентификации 
классов в \Cpp}\EN{data may be also useful for \Cpp classes identification}.

\chapter{\RU{Старые методы, тем не менее, интересные}
\EN{Old-school techniques, nevertheless, interesting to know}}

\section{\RU{Сравнение ``снимков'' памяти}\EN{Memory ``snapshots'' comparing}}

\RU{Метод простого сравнения двух снимков памяти для поиска изменений часто применялся для взлома игр 
на 8-битных компьютерах и взлома файлов с записанными рекордными очками.}
\EN{The technique of straightforward two memory snapshots comparing in order to see changes, was often used to hack
8-bit computer games and hacking ``high score'' files.}

\RU{К примеру, если вы имеете загруженную игру на 8-битном компьютере (где самой памяти не очень много, но игра
занимает еще меньше), и вы знаете что сейчас у вас, условно, 100 пуль, вы можете сделать ``снимок'' всей
памяти и сохранить где-то. Затем просто стреляете куда угодно, у вас станет 99 пуль, сделать второй ``снимок'',
и затем сравнить эти два снимка: где-то наверняка должен быть байт, который в начале был 100, а затем стал 99.}
\EN{For example, if you got a loaded game on 8-bit computer (it is not much memory on these, but game is usually
consumes even less memory) and you know that you have now, let's say, 100 bullets, you can do a ``snapshot''
of all memory and back it up to some place. Then shoot somewhere, bullet count now 99, do second ``snapshot''
and then compare both: somewhere must be a byte which was 100 in the beginning and now it is 99.}
\RU{Если учесть, что игры на тех маломощных домашних компьютерах обычно были написаны на ассемблере и подобные
переменные там были глобальные, то можно с уверенностью сказать, какой адрес в памяти всегда отвечает за количество
пуль. Если поискать в дизассемблированном коде игры все обращения по этому адресу, несложно найти код,
отвечающий за уменьшение пуль и записать туда инструкцию \gls{NOP}
или несколько \gls{NOP}-в, так мы получим игру в которой у игрока всегда будет 100 пуль, например.}
\EN{Considering a fact these 8-bit games were often written in assembly language and such variables were global,
it can be said for sure, which address in memory holding bullets count. If to search all references to the
address in disassembled game code, it is not very hard to find a piece of code \glslink{decrement}{decrementing} bullets count,
write \gls{NOP} instruction there, or couple of \gls{NOP}-s, 
we'll have a game with e.g 100 bullets forever.}
\index{BASIC!POKE}
\RU{А так как игры на тех домашних 8-битных 
компьютерах всегда загружались по одним и тем же адресам, и версий одной игры редко когда было больше одной продолжительное время,
то геймеры-энтузиасты знали, по какому адресу (используя инструкцию языка BASIC \gls{POKE}) что записать после загрузки
игры, чтобы хакнуть её. Это привело к появлению списков ``читов'' состоящих из инструкций \gls{POKE}, публикуемых
в журналах посвященным 8-битным играм. См. также:}\EN{Games on these 8-bit computers was commonly loaded on the same
address, also, there were no much different versions of each game (commonly just one version was popular for a long span of time),
enthusiastic gamers knew, which byte must be written (using BASIC instruction \gls{POKE}) to which address in
order to hack it. This led to ``cheat'' lists containing of \gls{POKE} instructions published in magazines related to
8-bit games. See also:} \url{http://en.wikipedia.org/wiki/PEEK\_and\_POKE}.

\index{MS-DOS}
\RU{Точно так же легко модифицировать файлы с сохраненными рекордами (кто сколько очков набрал), впрочем, это может
сработать не только с 8-битными играми. Нужно заметить, какой у вас сейчас рекорд и где-то сохранить файл
с очками. Затем, когда очков станет другое количество, просто сравнить два файла, можно даже
DOS-утилитой FC\footnote{утилита MS-DOS для сравнения двух файлов побайтово} (файлы рекордов, часто, бинарные).}
\EN{Likewise, it is easy to modify ``high score'' files, this may work not only with 8-bit games. Let's notice 
your score count and back the file up somewhere. When ``high score'' count will be different, just compare two files,
it can be even done with DOS-utility FC\footnote{MS-DOS utility for binary files comparing} (``high score'' files
are often in binary form).}
\RU{Где-то будут отличаться несколько байт, и легко будет увидеть, какие именно отвечают за количество очков. 
Впрочем, разработчики игр осведомлены о таких хитростях и могут защититься от этого.}
\EN{There will be a point where couple of bytes will be different and it will be easy to see which ones are
holding score number.
However, game developers are aware of such tricks and may protect against it.}

\RU{В каком-то смысле похожий пример в этой книге здесь}
\EN{Somewhat similar example here in the book is}: \ref{Millenium_DOS_game}.

% TODO: пример с какой-то простой игрушкой?

\subsection{\RU{Реестр Windows}\EN{Windows registry}}

\RU{А еще можно вспомнить сравнение реестра Windows до инсталляции программы и после}
\EN{It is also possible to compare Windows registry before and after a program installation}.
\RU{Это также популярный метод поиска, какие элементы реестра программа будет использовать}
\EN{It is very popular method of finding, which registry elements a program will use}.
\EN{Probably, this is a reason why ``windows registy cleaner'' shareware is so popluar.}
\RU{Наверное это причина, почему так популярны shareware-программы для очистки реестра в Windows.}


\chapter{\RU{Специфичное для ОС}\EN{OS-specific}\DE{Betriebssystem-spezifische Themen}}
\EN{\section{Arguments passing methods (calling conventions)}
\label{sec:callingconventions}

\subsection{cdecl}
\myindex{cdecl}
\label{cdecl}

This is the most popular method for passing arguments to functions in the \CCpp languages.

The gls{caller} also must return the value of the \gls{stack pointer} (\ESP) to its initial state after the \gls{callee} function exits.

\begin{lstlisting}[caption=cdecl,style=customasmx86]
push arg3
push arg2
push arg1
call function
add esp, 12 ; returns ESP
\end{lstlisting}

\subsection{stdcall}
\label{sec:stdcall}
\myindex{stdcall}

\newcommand{\SIZEOFINT}{The size of an \Tint type variable is 4 in x86 systems and 8 in x64 systems}

It's almost the same as \IT{cdecl}, with the exception that the \gls{callee} must set \ESP to the initial state by executing the \TT{RET x} instruction instead of \RET, \\
where \TT{x = arguments number * sizeof(int)\footnote{\SIZEOFINT}}.
The \gls{caller} is not adjusting the \gls{stack pointer}, 
there are no \TT{add esp, x} instruction.

\begin{lstlisting}[caption=stdcall,style=customasmx86]
push arg3
push arg2
push arg1
call function

function:
... do something ...
ret 12
\end{lstlisting}

The method is ubiquitous in win32 standard libraries, but not in win64 (see below about win64).

\par For example, we can take the function from \myref{src:passing_arguments_ex} and change it slightly by adding the \TT{\_\_stdcall} modifier:

\begin{lstlisting}[style=customc]
int __stdcall f2 (int a, int b, int c)
{
	return a*b+c;
};
\end{lstlisting}

It is to be compiled in almost the same way as \myref{src:passing_arguments_ex_MSVC_cdecl}, but you will see \TT{RET 12} instead of \TT{RET}.
\ac{SP} is not updated in the \gls{caller}.

As a consequence, 
the number of function arguments can be easily deduced from the \TT{RETN n} instruction: just divide $n$ by 4.

\lstinputlisting[caption=MSVC 2010,style=customasmx86]{OS/calling_conventions/stdcall_ex.asm}

\subsubsection{Functions with variable number of arguments}

\printf-like functions are, probably, the only case of functions with a variable number of arguments in \CCpp,
but it is easy to illustrate an important difference between \IT{cdecl} and \IT{stdcall} with their help.
Let's start with the idea that the compiler knows the argument count of each \printf function call.

However, the called \printf, which is already compiled and located in MSVCRT.DLL (if we talk about Windows),
does not have any information about how much arguments were passed, however it can determine it from the format string.

Thus, if \printf would be a \IT{stdcall} function and restored \gls{stack pointer} to its initial state by counting
the number of arguments in the format string, this could be a dangerous situation, when one programmer's typo can
provoke a sudden program crash.
Thus it is not suitable for such functions to use \IT{stdcall}, \IT{cdecl} is better.

\subsection{fastcall}
\label{fastcall}
\myindex{fastcall}

That's the general naming for the method of passing some arguments via registers and the 
rest via the stack. It worked faster than \IT{cdecl}/\IT{stdcall} on older CPUs 
(because of smaller stack pressure).
It may not help to gain any significant performance on latest (much more complex) \ac{CPU}s, however.

It is not standardized, so the various compilers can do it differently.
It's a well known caveat: if you have two DLLs and the one uses another one, and they are built by different compilers with 
different \IT{fastcall} calling conventions, you can expect problems.

Both MSVC and GCC pass the first and second arguments via \ECX and \EDX and the rest of the arguments via the stack.

The \gls{stack pointer} must be restored to its initial state by the \gls{callee} (like in \IT{stdcall}).

\begin{lstlisting}[caption=fastcall,style=customasmx86]
push arg3
mov edx, arg2
mov ecx, arg1
call function

function:
.. do something ..
ret 4
\end{lstlisting}

For example, we may take the function from \myref{src:passing_arguments_ex} and change it slightly by adding a \TT{\_\_fastcall} modifier:

\begin{lstlisting}[style=customc]
int __fastcall f3 (int a, int b, int c)
{
	return a*b+c;
};
\end{lstlisting}

Here is how it is to be compiled:

\lstinputlisting[caption=\Optimizing MSVC 2010 /Ob0,style=customasmx86]{OS/calling_conventions/fastcall_ex.asm}

We see that the \gls{callee} returns \ac{SP} by using the \TT{RETN} instruction with an operand.

Which implies that the number of arguments can be deduced easily here as well.

\subsubsection{GCC regparm}

\newcommand{\URLREGPARMM}{\url{http://go.yurichev.com/17040}}

It is the evolution of \IT{fastcall}\footnote{\URLREGPARMM} in some sense.
With the \TT{-mregparm} option it is possible to set how many arguments are to be passed via registers (3 is the maximum).
Thus, the \EAX, \EDX and \ECX registers are to be used.

Of course, if the number the of arguments is less than 3, not all 3 registers are to be used.

The \gls{caller} restores the \gls{stack pointer} to its initial state.

For example, see (\myref{regparm}).

\subsubsection{Watcom/OpenWatcom}
\myindex{OpenWatcom}

Here it is called \q{register calling convention}.
The first 4 arguments are passed via the \EAX, \EDX, \EBX and \ECX registers.
All the rest---via the stack.

These functions has an underscore appended to the function name in order to distinguish them from 
those having a different calling convention.

\subsection{thiscall}
\myindex{thiscall}

This is passing the object's \ITthis pointer to the function-method, in \Cpp.

In MSVC, \ITthis is usually passed in the \ECX register.

In GCC, the \ITthis pointer is passed as the first function-method argument.
Thus it will be very visible that internally: all function-methods have an extra argument.

For an example, see (\myref{thiscall}).

\subsection{x86-64}
\myindex{x86-64}

\subsubsection{Windows x64}
\label{sec:callingconventions_win64}

The method of for passing arguments in Win64 somewhat resembles \TT{fastcall}.
The first 4 arguments are passed via \RCX, \RDX, \Reg{8} and \Reg{9}, the rest---via the stack.
The \gls{caller} also must prepare space for 32 bytes or 4 64-bit values,
so then the \gls{callee} can save there the first 4 arguments.
Short functions may use the arguments' values just from the registers,
but larger ones may save their values for further use.

The \gls{caller} also must return the \gls{stack pointer} into its initial state.

This calling convention is also used in Windows x86-64 system DLLs 
(instead of \IT{stdcall} in win32).

Example:

\lstinputlisting[style=customc]{OS/calling_conventions/x64.c}

\lstinputlisting[caption=MSVC 2012 /0b,style=customasmx86]{OS/calling_conventions/x64_MSVC_Ob.asm}

\myindex{Scratch space}

Here we clearly see how 7 arguments are passed: 4 via registers and the remaining 3 via the stack.

The code of the f1() function's prologue saves the arguments in the \q{scratch space}---a space in the stack
intended exactly for this purpose.

This is arranged so because the compiler cannot be sure that there will be enough registers to use without these 4,
which will otherwise be occupied by the arguments until the function's execution end.

The \q{scratch space} allocation in the stack is the caller's duty.

\lstinputlisting[caption=\Optimizing MSVC 2012 /0b,style=customasmx86]{OS/calling_conventions/x64_MSVC_Ox_Ob.asm}

If we compile the example with optimizations, it is to be almost the same, 
but the \q{scratch space} will not be used, because it won't be needed.

\myindex{x86!\Instructions!LEA}
\label{using_MOV_and_pack_of_LEA_to_load_values}

Also take a look on how MSVC 2012 optimizes the loading of primitive values into registers by using \LEA (\myref{sec:LEA}).
\INS{MOV} would be 1 byte longer here (5 instead of 4).

Another example of such thing is: \myref{TaskMgr_LEA}.

\myparagraph{Windows x64: Passing \ITthis (\CCpp)}

The \ITthis pointer is passed in \RCX, the first argument of the method is in \RDX, etc.
For an example see: \myref{simple_CPP_MSVC_x64}.
 
\subsubsection{Linux x64}

The way arguments are passed in Linux for x86-64 is almost the same as in Windows, but 6 registers are
used instead of 4 (\RDI, \RSI, \RDX, \RCX, \Reg{8}, \Reg{9}) and there is no \q{scratch space}, 
although the \gls{callee} may save the register values in the stack, if it needs/wants to.

\lstinputlisting[caption=\Optimizing GCC 4.7.3,style=customasmx86]{OS/calling_conventions/x64_linux_O3.s}

\myindex{AMD}

N.B.: here the values are written into the 32-bit parts of the registers (e.g., EAX) but not in the whole 64-bit 
register (RAX).
This is because each write to the low 32-bit part of a register automatically clears the high 32 bits.
Supposedly, it was decided in AMD to do so to simplify porting code to x86-64.

\subsection{Return values of \Tfloat and \Tdouble type}
\myindex{float}
\myindex{double}

In all conventions except in Win64, the values of type \Tfloat or \Tdouble are returned via the FPU register \ST{0}.

In Win64, the values of \Tfloat and \Tdouble types are returned 
in the low 32 or 64 bits of the \XMM{0} register.

\subsection{Modifying arguments}

Sometimes, \CCpp{} programmers (not limited to these \ac{PL}s, though),
may ask, what can happen if they modify the arguments?

The answer is simple: the arguments are stored in the stack, 
that is where the modification takes place.

The calling functions is not using them after the \gls{callee}'s exit (the author of these lines has never seen any such case in his practice).

\lstinputlisting[style=customc]{OS/calling_conventions/change_arguments.c}

\lstinputlisting[caption=MSVC 2012,style=customasmx86]{OS/calling_conventions/change_arguments.asm}

% TODO (OllyDbg) пример как в стеке меняется $a$

So yes, one can modify the arguments easily.
Of course, if it is not \IT{references} in \Cpp{} (\myref{cpp_references}),
and if you don't modify data to which a pointer points to, 
then the effect will not propagate outside the current function.

Theoretically, after the \gls{callee}'s return, 
the \gls{caller} could get the modified argument and use it somehow.
Maybe if it is written directly in assembly language.

For example, code like this will be generated by usual \CCpp compiler:

\begin{lstlisting}[style=customasmx86]
	push	456	; will be b
	push	123	; will be a
	call	f	; f() modifies its first argument
	add	esp, 2*4
\end{lstlisting}

We can rewrite this code like:

\begin{lstlisting}[style=customasmx86]
	push	456	; will be b
	push	123	; will be a
	call	f	; f() modifies its first argument
	pop	eax
	add	esp, 4
	; EAX=1st argument of f() modified in f()
\end{lstlisting}

Hard to imagine, why anyone would need this, but this is possible in practice.
Nevertheless, the \CCpp languages standards don't offer any way to do so.

% sections
\subsection{Taking a pointer to function argument}
\label{pointer_to_argument}

\dots even more than that, it's possible to take a pointer to the function's argument and pass
it to another function:

\lstinputlisting[style=customc]{OS/calling_conventions/ptr_to_argument/10.c}

It's hard to understand how it works until we can see the code:

\lstinputlisting[caption=\Optimizing MSVC 2010,style=customasmx86]{OS/calling_conventions/ptr_to_argument/MSVC_2010_O3.asm}

The address of the place in the stack where $a$ has been passed is just passed to another function.
It modifies the value addressed by the pointer and then \printf prints the modified value.

\par The observant reader might ask, what about calling conventions where the function's arguments are
passed in registers?

That's a situation where the \IT{Shadow Space} is used.

The input value is copied from the register
to the \IT{Shadow Space} in the local stack, and then this address is passed to the other function:

\lstinputlisting[caption=\Optimizing MSVC 2012 x64,style=customasmx86]{OS/calling_conventions/ptr_to_argument/MSVC_2012_x64_O3.asm}

GCC also stores the input value in the local stack:

\lstinputlisting[caption=\Optimizing GCC 4.9.1 x64,style=customasmx86]{OS/calling_conventions/ptr_to_argument/GCC491_x64_O3.s}

GCC for ARM64 does the same, but this space is called \IT{Register Save Area} here:

\lstinputlisting[caption=\Optimizing GCC 4.9.1 ARM64,style=customasmARM]{OS/calling_conventions/ptr_to_argument/GCC49_ARM64_O3.s}

By the way, a similar usage of the \IT{Shadow Space} is also considered here: \myref{variadic_arith_registers}.



}
\RU{\section{Способы передачи аргументов при вызове функций}
\label{sec:callingconventions}

\subsection{cdecl}
\myindex{cdecl}
\label{cdecl}

Этот способ передачи аргументов через стек чаще всего используется в языках \CCpp.

Вызывающая функция заталкивает в стек аргументы в обратном порядке: сначала последний аргумент в стек, 
затем предпоследний, и в самом конце --- первый аргумент. 
Вызывающая функция должна также затем вернуть \glslink{stack pointer}{указатель стека} в нормальное состояние, 
после возврата вызываемой функции.

\begin{lstlisting}[caption=cdecl,style=customasmx86]
push arg3
push arg2
push arg1
call function
add esp, 12 ; returns ESP
\end{lstlisting}

\subsection{stdcall}
\label{sec:stdcall}
\myindex{stdcall}

\newcommand{\SIZEOFINT}{Размер переменной типа \Tint --- 4 в x86-системах и 8 в x64-системах}

Это почти то же что и \IT{cdecl}, за исключением того, что вызываемая функция сама возвращает \ESP 
в нормальное состояние, выполнив инструкцию \TT{RET x} вместо \RET, \\
где \TT{x = количество\_аргументов * sizeof(int)\footnote{\SIZEOFINT}}.
Вызывающая функция не будет корректировать \glslink{stack pointer}{указатель стека},
там нет инструкции \TT{add esp, x}.

\begin{lstlisting}[caption=stdcall,style=customasmx86]
push arg3
push arg2
push arg1
call function

function:
... do something ...
ret 12
\end{lstlisting}

Этот способ используется почти везде в системных библиотеках win32, 
но не в win64 (о win64 смотрите ниже).

\par Например, мы можем взять функцию из 
\myref{src:passing_arguments_ex} и изменить её немного добавив модификатор \TT{\_\_stdcall}:

\begin{lstlisting}[style=customc]
int __stdcall f2 (int a, int b, int c)
{
	return a*b+c;
};
\end{lstlisting}

Он будет скомпилирован почти так же как и \myref{src:passing_arguments_ex_MSVC_cdecl},
но вы увидите \TT{RET 12} вместо \TT{RET}. 
\ac{SP} не будет корректироваться в \glslink{caller}{вызывающей функции}.

Как следствие, количество аргументов функции легко узнать из инструкции \TT{RETN n} просто разделите
$n$ на 4.

\lstinputlisting[caption=MSVC 2010,style=customasmx86]{OS/calling_conventions/stdcall_ex.asm}

\subsubsection{Функции с переменным количеством аргументов}

Функции вроде \printf, должно быть, единственный случай функций в \CCpp с переменным количеством аргументов,
но с их помощью можно легко проследить очень важную разницу между \IT{cdecl} и \IT{stdcall}.
Начнем с того, что компилятор знает сколько аргументов было у \printf.

Однако, вызываемая функция \printf, которая уже давно скомпилирована 
и находится в системной библиотеке MSVCRT.DLL (если говорить о Windows), 
не знает сколько аргументов ей передали, хотя может установить их количество по строке формата.

Таким образом, если бы \printf была \IT{stdcall}-функцией и возвращала \glslink{stack pointer}{указатель стека} в первоначальное состояние 
подсчитав количество аргументов в строке формата, это была бы потенциально опасная ситуация, 
когда одна опечатка программиста могла бы вызывать неожиданные падения программы. 
Таким образом, для таких функций \IT{stdcall} явно не подходит, а подходит \IT{cdecl}.

\subsection{fastcall}
\label{fastcall}
\myindex{fastcall}

Это общее название для передачи некоторых аргументов через регистры, а всех остальных --- через стек.
На более старых процессорах, это работало потенциально быстрее чем \IT{cdecl}/\IT{stdcall} (ведь стек в памяти использовался меньше).
Впрочем, на современных (намного более сложных) CPU, существенного выигрыша может и не быть.

Это не стандартизированный способ, поэтому разные компиляторы делают это по-своему. 
Разумеется, если у вас есть, скажем, две DLL, одна использует другую, и обе они собраны с \IT{fastcall}
но разными компиляторами, очень вероятно, будут проблемы.

MSVC и GCC передает первый и второй аргумент через \ECX и \EDX а остальные аргументы через стек.

\glslink{stack pointer}{Указатель стека} должен быть возвращен в первоначальное состояние вызываемой функцией, 
как в случае \IT{stdcall}.

\begin{lstlisting}[caption=fastcall,style=customasmx86]
push arg3
mov edx, arg2
mov ecx, arg1
call function

function:
.. do something ..
ret 4
\end{lstlisting}

Например, мы можем взять функцию из 
\myref{src:passing_arguments_ex} и изменить её немного добавив модификатор \TT{\_\_fastcall}:

\begin{lstlisting}[style=customc]
int __fastcall f3 (int a, int b, int c)
{
	return a*b+c;
};
\end{lstlisting}

Вот как он будет скомпилирован:

\lstinputlisting[caption=\Optimizing MSVC 2010 /Ob0,style=customasmx86]{OS/calling_conventions/fastcall_ex.asm}

Видно, что \glslink{callee}{вызываемая функция} сама возвращает
 \ac{SP} 
при помощи инструкции \TT{RETN} с операндом.
Так что и здесь можно легко вычислять количество аргументов.

\subsubsection{GCC regparm}

\newcommand{\URLREGPARMM}{\url{http://go.yurichev.com/17040}}

Это в некотором роде, развитие \IT{fastcall}\footnote{\URLREGPARMM}. 
Опцией \TT{-mregparm=x} можно указывать, 
сколько аргументов компилятор будет передавать через регистры. Максимально 3. 
В этом случае будут задействованы регистры \EAX, \EDX и \ECX.

Разумеется, если аргументов у функции меньше трех, то будет задействована только часть регистров.

Вызывающая функция возвращает \glslink{stack pointer}{указатель стека} в первоначальное состояние.

Для примера, см. (\myref{regparm}).

\subsubsection{Watcom/OpenWatcom}
\myindex{OpenWatcom}

Здесь это называется \q{register calling convention}.
Первые 4 аргумента передаются через регистры
\EAX, \EDX, \EBX and \ECX.
Все остальные --- через стек.
Эти функции имеют символ подчеркивания, добавленный к концу имени функции, для отличия их от тех,
которые имеют другой способ передачи аргументов.

\subsection{thiscall}
\myindex{thiscall}

В \Cpp, это передача в функцию-метод указателя \ITthis на объект.

В MSVC указатель \ITthis обычно передается в регистре \ECX.

В GCC указатель \ITthis обычно передается как самый первый аргумент. 
Таким образом, внутри будет видно: у всех функций-методов на один аргумент больше.

Для примера, см. (\myref{thiscall}).

\subsection{x86-64}
\myindex{x86-64}

\subsubsection{Windows x64}
\label{sec:callingconventions_win64}

В win64 метод передачи всех параметров немного похож на \TT{fastcall}. 
Первые 4 аргумента записываются в регистры \RCX, \RDX, \Reg{8}, \Reg{9}, а остальные --- в стек. 
Вызывающая функция также должна подготовить место из 32 байт или для четырех 64-битных значений, 
чтобы вызываемая функция могла сохранить там первые 4 аргумента. 
Короткие функции могут использовать переменные прямо из регистров, 
но б\'{о}льшие могут сохранять их значения на будущее.

Вызывающая функция должна вернуть \glslink{stack pointer}{указатель стека} 
в первоначальное состояние.

Это же соглашение используется и в системных библиотеках Windows x86-64 
(вместо \IT{stdcall} в win32).

Пример:

\lstinputlisting[style=customc]{OS/calling_conventions/x64.c}

\lstinputlisting[caption=MSVC 2012 /0b,style=customasmx86]{OS/calling_conventions/x64_MSVC_Ob.asm}

\myindex{Scratch space}
Здесь мы легко видим, как 7 аргументов передаются: 4 через регистры и остальные 3 через стек.
Код пролога функции f1() сохраняет аргументы в \q{scratch space} --- место в стеке предназначенное
именно для этого.
Это делается потому что компилятор может быть не уверен, достаточно ли ему будет остальных регистров
для работы исключая эти 4, которые иначе будут заняты аргументами до конца исполнения функции.
Выделение \q{scratch space} в стеке лежит на ответственности вызывающей функции.

\lstinputlisting[caption=\Optimizing MSVC 2012 /0b,style=customasmx86]{OS/calling_conventions/x64_MSVC_Ox_Ob.asm}

Если компилировать этот пример с оптимизацией, то выйдет почти то же самое, 
только \q{scratch space} не используется, потому что незачем.

\myindex{x86!\Instructions!LEA}
\label{using_MOV_and_pack_of_LEA_to_load_values}
Обратите также внимание на то как MSVC 2012 оптимизирует примитивную загрузку значений в регистры
используя \LEA (\myref{sec:LEA}).
\INS{MOV} здесь был бы на 1 байт длиннее (5 вместо 4).

Еще один пример подобного: \myref{TaskMgr_LEA}.

\myparagraph{Windows x64: Передача \ITthis (\CCpp)}

Указатель \ITthis передается через \RCX, первый аргумент метода через \RDX, итд.
Для примера, см. также: \myref{simple_CPP_MSVC_x64}.
 
\subsubsection{Linux x64}

Метод передачи аргументов в Linux для x86-64 почти такой же, как и в Windows, но 6 регистров
используется вместо 4 (\RDI, \RSI, \RDX, \RCX, \Reg{8}, \Reg{9}), и здесь нет \q{scratch space}, 
но \gls{callee} может сохранять значения регистров в стеке, если ему это нужно.

\lstinputlisting[caption=\Optimizing GCC 4.7.3,style=customasmx86]{OS/calling_conventions/x64_linux_O3.s}

\myindex{AMD}
N.B.: здесь значения записываются в 32-битные части регистров (например EAX) а не в весь 64-битный
регистр (RAX).
Это связано с тем что в x86-64,
запись в младшую 32-битную часть 64-битного регистра автоматически обнуляет старшие 32 бита.
Должно быть, это так решили в AMD для упрощения портирования кода под x86-64.

\subsection{Возвращение переменных типа \Tfloat, \Tdouble}
\myindex{float}
\myindex{double}

Во всех соглашениях кроме Win64, переменная типа \Tfloat или \Tdouble возвращается через регистр FPU \ST{0}.

В Win64 переменные типа \Tfloat и \Tdouble возвращаются в младших 16-и или 32-х битах 
регистра \XMM{0}.

\subsection{Модификация аргументов}

Иногда программисты на \CCpp{} (и не только этих \ac{PL}) задаются вопросом,
что может случиться, если модифицировать аргументы?

Ответ прост: аргументы хранятся в стеке, именно там и будет происходит модификация.

А вызывающие функции не использует их после вызова функции (автор этих строк никогда не видел в своей практике обратного случая).

\lstinputlisting[style=customc]{OS/calling_conventions/change_arguments.c}

\lstinputlisting[caption=MSVC 2012,style=customasmx86]{OS/calling_conventions/change_arguments.asm}

% TODO (OllyDbg) пример как в стеке меняется $a$

Следовательно, модифицировать аргументы функции можно запросто.
Разумеется, если это не \IT{references} в \Cpp{} (\myref{cpp_references}),
и если вы не модифицируете данные по указателю, 
то эффект не будет распространяться за пределами текущей функции.

Теоретически, после возврата из \gls{callee},
функция-\gls{caller} могла бы получить модифицированный аргумент и использовать его как-то.
Может быть, если бы она была написана на языке ассемблера.

Например, такой код генерирует обычный компилятор \CCpp:

\begin{lstlisting}[style=customasmx86]
	push	456	; will be b
	push	123	; will be a
	call	f	; f() modifies its first argument
	add	esp, 2*4
\end{lstlisting}

Мы можем переписать так:

\begin{lstlisting}[style=customasmx86]
	push	456	; will be b
	push	123	; will be a
	call	f	; f() modifies its first argument
	pop	eax
	add	esp, 4
	; EAX=1st argument of f() modified in f()
\end{lstlisting}

Трудно представить, кому может это понадобиться, но на практике это возможно.
Так или иначе, стандарты языков \CCpp не предлагают никакого способа это сделать.

% sections
\subsection{Указатель на аргумент функции}
\label{pointer_to_argument}

\dots и даже более того, можно взять указатель на аргумент функции и передать его в другую функцию:

\lstinputlisting[style=customc]{OS/calling_conventions/ptr_to_argument/10.c}

Трудно понять, как это работает, пока мы не посмотрим на код:

\lstinputlisting[caption=\Optimizing MSVC 2010,style=customasmx86]{OS/calling_conventions/ptr_to_argument/MSVC_2010_O3.asm}

Адрес места в стеке где была передана $a$ просто передается в другую функцию.
Она модифицирует переменную по этому адресу, и затем \printf выведет модифицированное значение.

\par Наблюдательный читатель может спросить, а что насчет тех соглашений о вызовах, где аргументы функции
передаются в регистрах?

Это та ситуация где используется \IT{Shadow Space}.

Так что входящее значение копируется из регистра в \IT{Shadow Space} в локальном стеке и затем это адрес
передается в другую функцию:

\lstinputlisting[caption=\Optimizing MSVC 2012 x64,style=customasmx86]{OS/calling_conventions/ptr_to_argument/MSVC_2012_x64_O3.asm}

GCC также записывает входное значение в локальный стек:

\lstinputlisting[caption=\Optimizing GCC 4.9.1 x64,style=customasmx86]{OS/calling_conventions/ptr_to_argument/GCC491_x64_O3.s}

GCC для ARM64 делает то же самое, но это пространство здесь называется \IT{Register Save Area}:

\lstinputlisting[caption=\Optimizing GCC 4.9.1 ARM64,style=customasmARM]{OS/calling_conventions/ptr_to_argument/GCC49_ARM64_O3.s}

Кстати, похожее использование \IT{Shadow Space} разбирается здесь: \myref{variadic_arith_registers}.



}
\DE{\section{Methoden zur Argumentenübergabe (Aufrufkonventionen)}
\label{sec:callingconventions}

\subsection{cdecl}
\myindex{cdecl}
\label{cdecl}

Hierbei handelt es sich um die am weitesten verbreitete Methode um in \CCpp-Sprachen
Argumente an Funktionen zu übergeben.

Der \gls{caller} muss den Wert des \gls{stack pointer} (\ESP) auf den ursprünglichen Stand
bringen, nachdem \gls{callee}-Funktion beendet wurde.

\begin{lstlisting}[caption=cdecl]
push Argument3
push Argument2
push Argument1
call Funktion
add esp, 12 ; gibt ESP zurueck
\end{lstlisting}

\subsection{stdcall}
\label{sec:stdcall}
\myindex{stdcall}

\newcommand{\SIZEOFINT}{Die Größe einer Variablen vom Datentyp \Tint ist 4 in x86-Systemen und 8 in x64-Systemen}

Dies ist fast gleich zu der \IT{cdecl}-Aufrufkonvention, mit Ausnahme, dass die
\gls{callee} den Wert von \ESP auf den ursprünglichen Wert setzen muss. Dies geschieht
durch die \TT{RET x} anstatt \RET, wobei gilt \TT{x = Nummer des Arguments * sizeof(int)\footnote{\SIZEOFINT}}.

Der \gls{caller} passt den \gls{stack pointer} nicht an, es sind also keine \TT{add esp, x}-Anweisungen vorhanden.

\begin{lstlisting}[caption=stdcall]
push Argument3
push Argument2
push Argument1
call Funktion

Funktion:
... tue etwas ...
ret 12
\end{lstlisting}

Diese Methode ist in Win32-Standard-Bibliotheken allgegenwärtig, fehlt jedoch in Win64 (siehe unten).

\par Beispielsweise kann die Funktion von \myref{src:passing_arguments_ex} genommen werden und durch
Hinzufügen des \TT{\_\_stdcall}-Modifizierers leicht verändert werden:

\begin{lstlisting}
int __stdcall f2 (int a, int b, int c)
{
	return a*b+c;
};
\end{lstlisting}

Das Kompilat ist fast das gleiche wie bei \myref{src:passing_arguments_ex_MSVC_cdecl},
jedoch wird \TT{RET 12} anstatt \TT{RET} genutzt. Der \ac{SP} wird im \gls{caller} nicht
aktualisiert.

Als Konsequenz daraus kann die Anzahl der Funktionsargumente einfach von der \TT{RETN n}-Anweisung
abgeleitet werden, indem $n$ durch 4 geteilt wird

%TODO Compilation of german version fails with style=customasm
%\lstinputlisting[caption=MSVC 2010,style=customasm]{OS/calling_conventions/stdcall_ex.asm}
\lstinputlisting[caption=MSVC 2010]{OS/calling_conventions/stdcall_ex.asm}

\subsubsection{Funktionen mit einer variablen Anzahl von Argumenten}

\printf-ähnliche Funktionen sind vielleicht die einzigen Funktionen in \CCpp mit einer Variablen
Anzahl von Argumenten, aber mit ihnen kann ein wichtiger Unterschied zwischen \IT{cdecl} und
\IT{stdcall} veranschaulicht werden.
Beginnen wir mit der Vorstellung, dass der Compiler die Anzahl der Argumente für jeden Aufruf
von \printf kennt.

Die aufgerufene und bereits kompilierte \printf-Funktion befindet sich in der Datei MSVCRT.DLL
(wenn über Windows geredet wird) und hat aber keinerlei Informationen darüber wie viele Argumente
übergeben wurden; dies kann jedoch über den Formatstring herausgefunden werden.

%TODO: The following sentence is much too long.
Wenn \printf eine \IT{stdcall}-Funktion wäre und den \gls{stack pointer} durch Zählen der Zahl
der Argumente im Formatstring auf den ursprünglichen Wert setzen würde, kann eine gefährliche
Situation entstehen, da ein Schreibfehler des Programmierers zu einem plötzliche Programmabsturz
führen könnte.
Aus diesem Grund ist für diese Art von Funktionen \IT{stdcall} ungeeignet und \IT{cdecl} ist
zu bevorzugen.

\subsection{fastcall}
\label{fastcall}
\myindex{fastcall}

Dies ist der allgemeine Name für eine Methode, in der einige Argumente mittels Registern und
der Rest über den Stack übergeben werden. Für ältere CPUs ist \IT{fastcall} schneller als
\IT{cdecl} und \IT{stdcall} wegen der geringeren Stack-Nutzung.
Auf neueren \ac{CPU}s wird dieser Ansatz vermutlich keine signifikante Geschwindigkeitserhöhung
nach sich ziehen.

\IT{fastcall} ist nicht standardisiert, so das verschiedene Compiler eine unterschiedliche
Umsetzung machen können.
Dies kann zu Problemen führen, wenn zwei DLLs genutzt werden, von denen eine die andere nutzt
und durch das Nutzen verschiedener Compiler unterschiedliche \IT{fastcall}-Aufrufkonventionen
genutzt werden.

Sowohl MSVC als auch GCC übergeben das erste und zweite Argument über \ECX und \EDX und den Rest
der Arguments mittels des Stacks.

Der \gls{stack pointer} muss vom \gls{callee} auf den ursprünglichen Wert gesetzt werden,
wie in \IT{stdcall} auch.

\begin{lstlisting}[caption=fastcall]
push Argument3
mov edx, Argument2
mov ecx, Argument1
call Funktion

Funktion:
.. tue etwas ..
ret 4
\end{lstlisting}

Beispielsweise kann die Funktion von \myref{src:passing_arguments_ex} genommen werden und durch
Hinzufügen des \TT{\_\_fastcall}-Modifizierers leicht verändert werden:

\begin{lstlisting}
int __fastcall f3 (int a, int b, int c)
{
	return a*b+c;
};
\end{lstlisting}

Nachfolgend das Kompilat:

%TODO Compilation of german version fails with style=customasm
%\lstinputlisting[caption=\Optimizing MSVC 2010 /Ob0,style=customasm]{OS/calling_conventions/fastcall_ex.asm}
\lstinputlisting[caption=\Optimizing MSVC 2010 /Ob0]{OS/calling_conventions/fastcall_ex.asm}

Es ist erkennbar, dass der \gls{callee} den \ac{SP} mit der \TT{RETN}-Anweisung
und einem Operanden auf den ursprünglichen Wert setzt.

Dies bedeutet, dass die Anzahl der Argumente ebenfalls einfach abgeleitet werden kann.

\subsubsection{GCC regparm}

\newcommand{\URLREGPARMM}{\url{http://go.yurichev.com/17040}}

Dies ist in gewisser Weise die Weiterentwicklung von \IT{fastcall}\footnote{\URLREGPARMM}.
Mit der \TT{-mregparm}-Option ist es möglich festzulegen, wieviele Argumente per Register
übergeben werden (maximal 3).
Aus diesem Grund werden die Register \EAX, \EDX und \ECX genutzt.

Natürlich werden, wenn die Anzahl der Argumente kleiner als drei ist, nicht alle drei
Register genutzt.

Der \gls{caller} setzt den \gls{stack pointer} auf den initialen Zustand.

Als Beispiel siehe (\myref{regparm}).

\subsubsection{Watcom/OpenWatcom}
\myindex{OpenWatcom}

Hier erfolgt der Aufruf mit der \q{Register-Aufruf-Konvention}.
Die ersten vier Argumente werden in den Registern \EAX, \EDX, \EBX und \ECX übergeben,
der Rest auf dem Stack.

Diese Funktionen haben einen Unterstrich an den Funktionsnamen angehängt, um sie von
den anderen Aufrufkonventionen unterscheiden zu können.

\subsection{thiscall}
\myindex{thiscall}

Hier wird der \ITthis-Zeiger des Objekts an die Methode in \Cpp übergeben.

In MSVC wird \ITthis üblicherweise im \ECX-Register übergeben.

In GCC wird der \ITthis-Zeiger im ersten Argument der Methode übergeben.
Es ist sehr offensichtlich dass intern alle Methoden ein zusätzliches Argument haben.

Als Beispiel siehe (\myref{thiscall}).

\subsection{x86-64}
\myindex{x86-64}

\subsubsection{Windows x64}
\label{sec:callingconventions_win64}

Die Art Argumente zu Übergeben ähnelt in Win64 in gewisser Weise \TT{fastcall}.
Die ersten vier Argumente werden in den Registern \RCX, \RDX, \Reg{8} und \Reg{9}
übergeben und der Rest auf dem Stack.
Der \gls{caller} muss Platz für 32 Byte, also 4 64-Bit-Werte bereitstellen, so
dass der \gls{callee} dort die ersten vier Argumente speichern kann.
Kurze Funktionen können die Werte der Argumente direkt aus den Registern lesen,
während längere Funktionen diese für späteren Gebrauch zwischenspeichern sollten.

Der \gls{caller} muss den \gls{stack pointer} auf den vorherigen Zustand zurücksetzen.

Diese Aufrufkonvention wird auch in den Windows x86-64-System-DLLs genutzt
(anstatt \IT{stdcall} in Win32).

Beispiel:

\lstinputlisting[style=customc]{OS/calling_conventions/x64.c}

%TODO Compilation of german version fails with style=customasm
%\lstinputlisting[caption=MSVC 2012 /0b,style=customasm]{OS/calling_conventions/x64_MSVC_Ob.asm}
\lstinputlisting[caption=MSVC 2012 /0b]{OS/calling_conventions/x64_MSVC_Ob.asm}

\myindex{Scratch space}

Es ist hier klar erkennbar, wie sieben Argumente übergeben werden: vier in den
Registern und die drei restlichen auf dem Stack.

Der Code des f1()-Funktionsprolog sichert die Argumente in dem \q{Scratch Space},
einer Stelle auf dem Stack, die genau für diese Zwecke existiert.

Dies ist so realisiert, weil der Compiler nicht sicher sein kann, dass genug Register
ohne diese vier genutzt werden können und sie sonst durch die Argumente verändert
werden können, bis die Funktion beendet wird.

Die Bereitstellung des \q{Scratch Space} auf dem Stack ist Aufgabe der aufrufenden Funktion.

%TODO Compilation of german version fails with style=customasm
%\lstinputlisting[caption=\Optimizing MSVC 2012 /0b,style=customasm]{OS/calling_conventions/x64_MSVC_Ox_Ob.asm}
\lstinputlisting[caption=\Optimizing MSVC 2012 /0b]{OS/calling_conventions/x64_MSVC_Ox_Ob.asm}

Wenn das Beispiel ohne Optimierung compiliert wird, ist das Ergebnis fast das gleiche,
lediglich der \q{Scratch Space} ist unnötig und wird nicht genutzt.

\myindex{x86!\Instructions!LEA}
\label{using_MOV_and_pack_of_LEA_to_load_values}

Beachtenswert ist auch, wie MSVC 2012 das Laden von einfachen Werten in Register
durch das Nutzen von \LEA (\myref{sec:LEA}) optimiert.
\INS{MOV} wäre hier ein Byte länger (5 anstatt 4).

Ein weiteres Beispiel so eines Sachverhalts ist: \myref{TaskMgr_LEA}.

\myparagraph{Windows x64: Übergeben von \ITthis (\CCpp)}

Der \ITthis-Zeiger wird in \RCX übergeben, das erste Argument der Methode ist \RDX, usw.
Ein Beispiel ist hier zu sehen: \myref{simple_CPP_MSVC_x64}.

\subsubsection{Linux x64}

Die Art wie Linux für x86-64 Argumente übergibt ist fast die Gleiche wie in Windows,
jedoch werden sechs anstatt vier Register genutzt (\RDI, \RSI, \RDX, \RCX, \Reg{8}, \Reg{9})
und es gibt keinen \q{Scratch Space}, auch wenn der \gls{callee} die Registerwerte auf
dem Stack speichern kann wenn er dies will oder muss.

%TODO Compilation of german version fails with style=customasm
%\lstinputlisting[caption=\Optimizing GCC 4.7.3,style=customasm]{OS/calling_conventions/x64_linux_O3.s}
\lstinputlisting[caption=\Optimizing GCC 4.7.3]{OS/calling_conventions/x64_linux_O3.s}

\myindex{AMD}
Zur Beachtung: die Werte werden hier in die 32-Bit-Teile der Register (z.B.EAX) geschrieben,
aber nicht in die kompletten 64-Bit-Register (RAX).
Dies wird gemacht, weil jeder Schreibzugang auf die niederwertigen 32-Bit-Teile eines
Registers automatisch den höherwertigen Teil zurücksetzt.

Vermutlich wurde dies bei AMD so eingeführt um die Portierung des Codes zu x86-64 zu vereinfachen.

\subsection{Rückgabewerte von \Tfloat- und \Tdouble-Typen}
\myindex{float}
\myindex{double}

In allen Konventionen außer in Win64, werden di Werte vom Typ \Tfloat oder \Tdouble
in dem FPU-Register \ST{0} zurückgegeben.

In Win64 werden die Werte vom Typ \Tfloat oder \Tdouble in den niederwertigen 32 oder
64 Bit des \XMM{0}-Registers zurückgegeben.

\subsection{Verändern von Argumenten}

Manchmal fragen \CCpp{}-Programmierer (obwohl nicht auf diese \ac{PL} beschränkt),
was passieren kann wenn die Funktionsargumente verändert werden.

Die Antwort ist einfach: die Argumente sind auf dem Stack gespeichert und hier
werden auch die Veränderungen vorgenommen.

Die aufzurufende Funktion wird diese nicht nach dem Verlassen des \gls{callee}
nutzen (der Autor dieser Linien hat so einen Fall in der Praxis noch nie gesehen).

\lstinputlisting[style=customc]{OS/calling_conventions/change_arguments.c}

%TODO Compilation of german version fails with style=customasm
%\lstinputlisting[caption=MSVC 2012,style=customasm]{OS/calling_conventions/change_arguments.asm}
\lstinputlisting[caption=MSVC 2012]{OS/calling_conventions/change_arguments.asm}

% TODO (OllyDbg) пример как в стеке меняется $a$

Also: ja, die Argumente können einfach modifiziert werden.
Natürlich, wenn diese keine \IT{Referenz} in  \Cpp{} (\myref{cpp_references}) ist,
und nicht die Daten verändert werden auf die der Zeiger zeigt, wird der Effekt nicht
außerhalb der aktuellen Funktion sichtbar sein.

Theoretisch kann der \gls{caller} die modifizierten Argumente auf irgendeine Weise
nutzen nachdem der \gls{callee} beendet wurde.
Vielleicht wenn dies direkt in Assembler programmiert ist.

Beispielsweise wird Code wie folgt von gewöhnlichen \CCpp-Compilern erzeugt:

\begin{lstlisting}
	push	456	; wird gleich b sein
	push	123	; wird gleich a sein
	call	f	; f() modifiziert das erste Argument
	add	esp, 2*4
\end{lstlisting}

Der Code kann wie folgt neu geschrieben werden:

\begin{lstlisting}
	push	456	; wird gleich b sein
	push	123	; wird gleich a sein
	call	f	; f() modifiziert das erste Argument
	pop	eax
	add	esp, 4
	; EAX=Erstes Argument von f() modifiziert in f()
\end{lstlisting}

Es ist scher vorzustellen warum jemand dies tun sollte, aber in der Praxis ist es möglich.
Nichtsdestotrotz bietet der \CCpp-Standard keinen Möglichkeit dies zu tun.

% sections
\subsection{Einen Zeiger auf ein Argument verarbeiten}
\label{pointer_to_argument}

\dots mehr als das ist es sogar möglich, einen Zeiger auf ein Funktionsargument
zu nehmen und an eine weitere Funktion zu übergeben:

\lstinputlisting[style=customc]{OS/calling_conventions/ptr_to_argument/10.c}

Es ist schwierig die Funktionsweise zu verstehen, aber der folgende Code bring Klarheit:

\lstinputlisting[caption=\Optimizing MSVC 2010,style=customasmx86]{OS/calling_conventions/ptr_to_argument/MSVC_2010_O3.asm}

Die Adresse der Stelle im Stack an der $a$ übergeben wird, wird lediglich an eine
weitere Funktion übergeben.
Diese verändert den Wert der mit dem Zeiger übergeben wird und \printf gibt anschließend
den veränderten Wert aus.

\par Der aufmerksame Leser mag sich fragen, was mit der Aufrufkonvention ist, in der
Funktionsargumente in Registern übergeben werden.

Das ist eine Situation, in der \IT{Shadow Space} genutzt wird.

Der Eingangswert wird vom Register in den \IT{Shadow Space} des lokalen Stacks
kopiert und dann diese Adresse an die andere Funktion übergeben:

\lstinputlisting[caption=\Optimizing MSVC 2012 x64,style=customasmx86]{OS/calling_conventions/ptr_to_argument/MSVC_2012_x64_O3.asm}

GCC sichert den Eingangswert ebenfalls auf dem lokalen Stack:

\lstinputlisting[caption=\Optimizing GCC 4.9.1 x64,style=customasmx86]{OS/calling_conventions/ptr_to_argument/GCC491_x64_O3.s}

GCC für ARM64 tut genau das gleiche, allerdings wird hier der Platz \IT{Register Save Area} genannt:

\lstinputlisting[caption=\Optimizing GCC 4.9.1 ARM64,style=customasmARM]{OS/calling_conventions/ptr_to_argument/GCC49_ARM64_O3.s}

Übrigens eine gleiche Nutzung des \IT{Shadow Space} wird auch hier beschrieben: \myref{variadic_arith_registers}.

}

\chapter{Thread Local Storage}
\label{TLS}
\index{TLS}

\RU{Это область данных, отдельная для каждого треда. Каждый тред может хранить там то, что ему нужно}
\EN{TLS is a data area, specific to each thread. Every thread can store what it needs there}.
\RU{Один из известных примеров, это стандартная глобальная переменная в Си}%
\EN{One well-known example is the C standard global variable} \IT{errno}. 
\RU{Несколько тредов одновременно могут вызывать функции
возвращающие код ошибки в \IT{errno}, поэтому глобальная переменная здесь не будет работать корректно, 
для мультитредовых программ \IT{errno} нужно хранить в в \ac{TLS}.}
\EN{Multiple threads may simultaneously call functions
which return an error code in \IT{errno}, so a global variable will not work correctly here for multi-threaded programs,
so \IT{errno} must be stored in the \ac{TLS}.} \\
\\
\index{\Cpp!C++11}
\RU{В}\EN{In the} C++11 \RU{ввели модификатор}\EN{standard, a new} \IT{thread\_local} 
\RU{, показывающий что каждый тред будет иметь свою версию этой переменной}
\EN{modifier was added, showing that each thread has its own version of the variable},
\RU{и её можно инициализировать, и она расположена в}\EN{it can be initialized, and it is located in the} \ac{TLS}
\footnote{
\index{C11}
\RU{В C11 также есть поддержка тредов, хотя и опциональная}
\EN{C11 also has thread support, optional though}}:

\begin{lstlisting}[caption=C++11]
#include <iostream>
#include <thread>

thread_local int tmp=3;

int main()
{
	std::cout << tmp << std::endl;
};
\end{lstlisting}

\RU{Компилируется в}\EN{Compiled in} MinGW GCC 4.8.1, \RU{но не в}\EN{but not in} MSVC 2012.

\RU{Если говорить о PE-файлах, то в исполняемом файле значение}
\EN{If we talk about PE files, in the resulting executable file, the} \IT{tmp} 
\RU{будет размещено именно в секции отведенной}
\EN{variable is to be allocated in the section devoted to the} \ac{TLS}.

\section{\RU{Вернемся к линейному конгруэнтному генератору}\EN{Linear congruential generator revisited}}
\label{LCG_TLS}

\RU{Рассмотренный ранее \myref{LCG_simple} генератор псевдослучайных чисел имеет недостаток:}
\EN{The pseudorandom number generator we considered earlier \myref{LCG_simple} has a flaw:}
\RU{он не пригоден для многопоточной среды, потому что переменная его внутреннего состояния может быть
прочитана и/или модифицирована в разных потоках одновременно.}
\EN{it's not thread-safe, because it has an internal state variable which can be read and/or 
modified in different threads simultaneously.}

% subsections
\subsection{Win32}

\subsubsection{\RU{Неинициализированные данные в \ac{TLS}}\EN{Uninitialized \ac{TLS} data}}

\RU{Одно из решений это добавить модификатор \TT{\_\_declspec( thread )} к глобальной переменной, и теперь
она будет выделена в \ac{TLS} (строка 9):}
\EN{One solution is to add \TT{\_\_declspec( thread )} modifier to the global variable, now it will be allocated
in \ac{TLS} (line 9):}

\lstinputlisting[numbers=left]{OS/TLS/win32/rand_uninit.c}

\RU{Hiew показывает что в исполняемом файле теперь есть новая PE-секция:}
\EN{Hiew shows us that there are new PE section in the executable file:} \TT{.tls}.
% TODO hiew screenshot?

\lstinputlisting[caption=\Optimizing MSVC 2013 x86]{OS/TLS/win32/rand_x86_uninit.asm}

\RU{\TT{rand\_state} теперь в \ac{TLS}-сегменте и у каждого потка есть своя версия этой переменной.}
\EN{\TT{rand\_state} is now in \ac{TLS} segment, and each thread has its own version of this variable.}
\RU{Вот как к ней обращаться: загрузить адрес \ac{TIB} из FS:2Ch, затем прибавить дополнительный индекс 
(если нужно), затем вычислить адрес \ac{TLS}-сегмента.}
\EN{Here is how it's accessed: load address of \ac{TIB} from FS:2Ch, then add additional index (if needed),
then calculate address of \ac{TLS} segment.}

\RU{Затем можно обращаться к переменной \TT{rand\_state} через регистр ECX, который указывает на свою
область в каждом потоке.}
\EN{Then it's possible to access \TT{rand\_state} variable through ECX register, which points to unique area
in each thread.}

\index{x86!\Registers!FS}
\RU{Селектор \TT{FS:} знаком любому reverse engineer-у, он всегда указывает на \ac{TIB}, чтобы всегда можно было
загружать данные специфичные для текущего потока.}
\EN{\TT{FS:} selector is familiar to any reverse engineer, it is specially used to always point to \ac{TIB},
so it would be fast to load thread-specific data.}

\index{x86!\Registers!GS}
\RU{В Win64 используется селектор \TT{GS:} и адрес \ac{TLS} теперь 0x58:}
\EN{\TT{GS:} selector used in Win64 and address of \ac{TLS} is 0x58:}

\lstinputlisting[caption=\Optimizing MSVC 2013 x64]{OS/TLS/win32/rand_x64_uninit.asm}

\subsubsection{\RU{Инициализированные данные в \ac{TLS}}\EN{Initialized \ac{TLS} data}}

\RU{Скажем, мы хотим чтобы в переменной \TT{rand\_state} в самом начале было какое-то значение, 
и если программист забудет инициализировать генератор, то \TT{rand\_state} все же будет инициализирована какой-то
константой (строка 9):}
\EN{Let's say, we want to set some fixed value to \TT{rand\_state} so in case of forgetfulness of programmer,
the \TT{rand\_state} variable would be initialized to some constant anyway (line 9):}

\lstinputlisting[numbers=left]{OS/TLS/win32/rand_init.c}

\RU{Код ничем не отличается от того, что мы уже видели, но вот что мы видим в IDA:}
\EN{The code is no differ from what we already saw, but what we see in IDA:}

\lstinputlisting{OS/TLS/win32/rand_init_IDA.lst}

\RU{Там 1234 и теперь, во время запуска каждого нового потока, новый \ac{TLS} будет выделен для нового потока,
и все эти данные, включая 1234, будут туда скопированы.}
\EN{1234 is there and while any new thread starting, new \ac{TLS} will be allocated for the new thread, 
and all this data, including 1234, will be copied.}

\RU{Вот типичный сценарий}\EN{This is typical scenario}:

\begin{itemize}
\item \RU{Запустился поток А. \ac{TLS} создался для него, 1234 скопировалось в \TT{rand\_state}.}
\EN{Thread A is started. \ac{TLS} is created for it, 1234 is copied to \TT{rand\_state}.}

\item \RU{Ф-ция \TT{my\_rand()} была вызвана несколько раз в потоке А. 
\TT{rand\_state} теперь содержит что-то неравное 1234.}
\EN{\TT{my\_rand()} function called several times in thread A. \TT{rand\_state} is different from 1234.}

\item 
\RU{Запустился поток Б. \ac{TLS} создался для него, 1234 скопировалось в \TT{rand\_state}, 
а в это же время, поток А имеет какое-то другое значение в этой переменной.}
\EN{Thread B is started. \ac{TLS} is created for it, 1234 is copied to \TT{rand\_state}, 
while thread A has some other value in this variable.}
\end{itemize}

\subsubsection{\RU{ac{TLS}-коллбэки}\EN{\ac{TLS} callbacks}}
\index{TLS!\RU{Коллбэки}\EN{Callbacks}}

\RU{Но что если переменные в \ac{TLS} должны быть установлены в значения, которые должны быть подготовлены
каким-то необычным образом?}
\EN{But what if \ac{TLS} variables should be filled with some data that must be prepared in some unusual way?}
\RU{Скажем, у нас есть следующая задача:
программист может забыть вызвать ф-цию \TT{my\_srand()} для инициализации \ac{PRNG}, но генератор должен быть
инициализирован на старте чем-то по-настоящему случайным а не 1234.}
\EN{Let's say, we've got the following task:
programmer may forget to call \TT{my\_srand()} function to initialize \ac{PRNG}, but generator should be 
initialized at start with something truly random, rather than 1234.}
\RU{Вот случай где можно применить \ac{TLS}-коллбэки}\EN{Here is a moment when \ac{TLS} callbacks can be used}.

\RU{Нижеследующий код не очень портабельный из-за хака, но тем не менее, вы поймете идею.}
\EN{The following code is not very portable due to the hack, but nevertheless, you've got the idea.}
\RU{Мы здесь добавляем ф-цию (\TT{tls\_callback()}), которая вызывается \IT{перед} стартом процесса и/или потока.}
\EN{What we do here is define a function (\TT{tls\_callback()}) which will be called \IT{before} 
process and/or thread start.}
\RU{Ф-ция будет инициализировать \ac{PRNG} значением из GetTickCount().}
\EN{The function will initialize \ac{PRNG} with GetTickCount() value.}

\lstinputlisting{OS/TLS/win32/rand_TLS_callback.c}

\RU{Посмотрим в}\EN{Let's see it in} IDA:

\lstinputlisting[caption=\Optimizing MSVC 2013]{OS/TLS/win32/rand_TLS_callback.lst}

\RU{TLS-коллбэки иногда используются в процедурах распаковки для запутывания их работы.}
\EN{TLS callback functions are sometimes used in unpacking routines to obscure its processings.}
\RU{Некоторые люди могут быть в неведении что какой-то код уже был исполнен прямо перед \ac{OEP}.}
\EN{Some people may be confused and be in the dark that some code was already executed right before \ac{OEP}.}

\subsection{Linux}

\RU{Вот как глобальная переменная локальная для потока определяется в GCC:}
\EN{Here is how thread-local global variable declared in GCC:}

\begin{lstlisting}
__thread uint32_t rand_state=1234;
\end{lstlisting}

\RU{Этот модификатор не стандартный для \CCpp, он присутствует только в GCC}
\EN{This is not standard \CCpp modifier, but rather GCC-specific}
\footnote{\url{https://gcc.gnu.org/onlinedocs/gcc-3.3/gcc/C99-Thread-Local-Edits.html}}.

\index{x86!\Registers!GS}
\RU{Селектор \TT{GS:} также используется для доступа к \ac{TLS}, но немного иначе:}
\EN{\TT{GS:} selector is also used to \ac{TLS} access, but in some different way:}

\lstinputlisting[caption=\Optimizing GCC 4.8.1 x86]{OS/TLS/linux/rand.lst}

% ??? Uninitialized data is allocated in \TT{.tbss} section, initialized --- in \TT{.tdata} section.

\RU{Еще об этом}\EN{More about it}: \cite{DrepperTLS}.



\EN{\section{System calls (syscall-s)}

\label{syscalls}
\myindex{syscall}

\myindex{kernel space}
\myindex{user space}
As we know, all running processes inside an \ac{OS} are divided into two categories:
those having full access to the hardware (\q{kernel space}) 
and those that do not (\q{user space}).

The \ac{OS} kernel and usually the drivers are in the first category.

All applications are usually in the second category.

\myindex{Glibc}
For example, Linux kernel is in \IT{kernel space}, but Glibc in \IT{user space}.

This separation is crucial for the safety of the \ac{OS}: it is very important not to give to any process the possibility to screw up
something in other processes or even in the \ac{OS} kernel.
\myindex{kernel panic}
\myindex{BSoD}
On the other hand, a failing driver or error inside the \ac{OS}'s kernel usually leads to a kernel panic or \ac{BSOD}.

The protection in the x86 processors allows to separate everything into 4 levels of protection (rings), but both in Linux
and in Windows only two are used: ring0 (\q{kernel space}) and ring3 (\q{user space}).

System calls (syscall-s)
are a point where these two areas are connected.

It can be said that this is the main \ac{API} provided to applications.

As in \gls{Windows NT}, the syscalls table resides in the \ac{SSDT}.

\myindex{Shellcode}

The usage of syscalls is very popular among shellcode and computer viruses authors, 
because it is hard to determine the addresses of
needed functions in the system libraries, but it is easier to use syscalls. However, much more code has to be
written due to the lower level of abstraction of the \ac{API}.

It is also worth noting that the syscall numbers may be different in various OS versions.

\subsection{Linux}
\label{linux_syscall}

\myindex{x86!\Instructions!INT!INT 0x80}
In Linux, a syscall is usually called via \TT{int 0x80}.
The call's number is passed in the \EAX register, and any other parameters~---in the other registers.

\lstinputlisting[caption=A simple example of the usage of two syscalls,style=customasmx86]{OS/linux_syscall.s}

Compilation:

\begin{lstlisting}
nasm -f elf32 1.s
ld 1.o
\end{lstlisting}

The full list of syscalls in Linux: \url{http://go.yurichev.com/17319}.

For system calls interception and tracing in Linux, strace(\myref{strace}) can be used.

\subsection{Windows}

\myindex{x86!\Instructions!INT!INT 0x2e}
\myindex{x86!\Instructions!SYSENTER}

Here they are called via \TT{int 0x2e} 
or using the special x86 instruction \TT{SYSENTER}.

The full list of syscalls in Windows: \url{http://go.yurichev.com/17320}.

Further reading:

\q{Windows Syscall Shellcode} by Piotr Bania: \url{http://go.yurichev.com/17321}.

}
\RU{\section{Системные вызовы (syscall-ы)}

\label{syscalls}
\myindex{syscall}

\myindex{kernel space}
\myindex{user space}
Как известно, все работающие процессы в \ac{OS} делятся на две категории:
имеющие полный доступ ко всему \q{железу} (\q{kernel space}) 
и не имеющие (\q{user space}).

В первой категории ядро \ac{OS} и, обычно, драйвера.

Во второй категории всё прикладное ПО.

\myindex{Glibc}

Например, ядро Linux в \IT{kernel space}, но Glibc в \IT{user space}.

Это разделение очень важно для безопасности \ac{OS}:
очень важно чтобы никакой процесс не мог испортить что-то в других процессах
или даже в самом ядре \ac{OS}.
\myindex{kernel panic}
\myindex{BSoD}
С другой стороны, падающий драйвер или ошибка внутри ядра \ac{OS} обычно приводит к kernel panic или \ac{BSOD}.

Защита x86-процессора устроена так что возможно разделить всё на 4 слоя защиты (rings), но и в Linux,
и в Windows, используются только 2: ring0 (\q{kernel space}) и ring3 (\q{user space}).

Системные вызовы (syscall-ы)
это точка где соединяются вместе оба эти пространства.
Это, можно сказать, самое главное \ac{API} предоставляемое прикладному ПО.

В \gls{Windows NT} таблица сисколлов находится в \ac{SSDT}.

\myindex{Shellcode}
Работа через syscall-ы популярна у авторов шеллкодов и вирусов,
потому что там обычно бывает трудно определить адреса нужных функций в системных библиотеках,
а syscall-ами проще пользоваться, хотя и придется писать больше
кода из-за более низкого уровня абстракции этого \ac{API}.
Также нельзя еще забывать, что номера syscall-ов могут отличаться от версии к версии OS.

\subsection{Linux}
\label{linux_syscall}

\myindex{x86!\Instructions!INT!INT 0x80}
В Linux вызов syscall-а обычно происходит через \TT{int 0x80}.
В регистре \EAX передается номер вызова,
в остальных регистрах --- параметры.

\lstinputlisting[caption=Простой пример использования пары syscall-ов,style=customasmx86]{OS/linux_syscall.s}

Компиляция:

\begin{lstlisting}
nasm -f elf32 1.s
ld 1.o
\end{lstlisting}

Полный список syscall-ов в Linux: \url{http://go.yurichev.com/17319}.

Для перехвата и трассировки системных вызовов в Linux, можно применять strace(\myref{strace}).

\subsection{Windows}

\myindex{x86!\Instructions!INT!INT 0x2e}
\myindex{x86!\Instructions!SYSENTER}

Вызов происходит через \TT{int 0x2e} 
либо используя специальную x86-инструкцию \TT{SYSENTER}.

Полный список syscall-ов в Windows: \url{http://go.yurichev.com/17320}.

Смотрите также:

\q{Windows Syscall Shellcode} by Piotr Bania: \url{http://go.yurichev.com/17321}.

}
\DE{\section{Systemaufrufe}

\label{syscalls}
\myindex{syscall}

\myindex{kernel space}
\myindex{user space}
Wie bekannt werden alle laufende Prozesse in einem \ac{OS} in zwei Kategorien unterteilt:
diejenigen, die vollen Zugriff auf die Hardware haben (\q{kernel space}) und diejenigen
die dies nicht haben (\q{user space}).

Der Kernel des Betriebssystems sowie die Treiber gehören in der Regel zur ersten Kategorie.

Alle Anwendungen gehören in der Regel der zweiten Kategorie an.

\myindex{Glibc}
Beispielsweise gehört der Linux-Kernel in den \IT{kernel space} während Glibc im \IT{user space}
ausgeführt wird.

Diese Unterscheidung ist von großer Bedeutung für die Sicherheit eines \ac{OS}:
es ist sehr wichtig nicht jedem Prozess die Möglichkeit etwas in einem anderen Prozess oder
sogar im \ac{OS}-Kernel zum Absturz zu bringen.
\myindex{kernel panic}
\myindex{BSoD}
Auf der anderen Seite führt ein Fehler im Treiber oder innerhalb des \ac{OS}-Kernels in der Regel
zu einem Kernel-Panic oder \ac{BSOD}.

Der Schutz im x86-Prozessor erlaubt die Unterscheidung in vier unterschiedliche Schutzlevel (Ringe).
Sowohl von Linux als auch von Windows werden jedoch nur zwei genutzt:
Ring 0 (\q{kernel space}) und Ring 3 (\q{user space}).

Systemaufrufe (syscalls) sind die Stelle an der diese beiden Bereiche miteinander verbunden werden.

In diesem Sinne sind die Systemaufrufe die Haupt-\ac{API} für die Anwendungen.

Unter \gls{Windows NT}, befindet sich die Tabelle mit Systemaufrufen in der \ac{SSDT}.

\myindex{Shellcode}

Die Nutzung von Systemaufrufen ist sehr verbreitet bei Shellcode- und Viren-Programmierern,
weil es schwieriger ist die Adresse einer benötigten Funktion herauszufinden als einen
Systemaufruf zu nutzen.
Die Kehrseite ist, das viel mehr Code geschrieben werden muss, aufgrund dem geringeren Grad
an Abstraktion der \ac{API}.

Erwähnenswert ist es das die Anzahl der Systemaufrufe bei den unterschiedlichen Betriebssystemen
variieren kann.

\subsection{Linux}
\label{linux_syscall}

\myindex{x86!\Instructions!INT!INT 0x80}
Unter Linux wird ein Systemaufruf in der Regel mit \TT{int 0x80} aufgerufen.
Die Nummer des Aufrufs wird im \EAX-Register übergeben und die restlichen Parameter in anderen Registern.

%TODO: style=customasm does not compile in german version
%\lstinputlisting[caption=A simple example of the usage of two syscalls,style=customasm]{OS/linux_syscall.s}
\lstinputlisting[caption=Ein einfaches Beispiel zur Nutzung zweier Systemaufrufe]{OS/linux_syscall.s}

Kompilation:

\begin{lstlisting}
nasm -f elf32 1.s
ld 1.o
\end{lstlisting}

Eine vollständige Liste von Systemaufrufen unter Linux: \url{http://go.yurichev.com/17319}.

Um Systemaufrufe unter Linux zu unterbrechen und nachverfolgen zu können, kann strace(\myref{strace})
genutzt werden.

\subsection{Windows}

\myindex{x86!\Instructions!INT!INT 0x2e}
\myindex{x86!\Instructions!SYSENTER}

Hier werden die Systemaufrufe via \TT{int 0x2e} aufgerufen oder über die spezielle x86-Anweisung \TT{SYSENTER}.

Eine vollständige Liste von Systemaufrufen unter Windows: \url{http://go.yurichev.com/17320}.

Weitere Informationen:

\q{Windows Syscall Shellcode} von Piotr Bania: \url{http://go.yurichev.com/17321}.
}

\section{Linux}
\EN{\input{OS/PIC_EN}}
\RU{\subsection{\CapitalPICcode}
\myindex{\PICcode}
\myindex{Linux}
\label{sec:PIC}

Во время анализа динамических библиотек (.so) в Linux, часто можно заметить такой шаблонный код:

\begin{lstlisting}[caption=libc-2.17.so x86,style=customasmx86]
.text:0012D5E3 __x86_get_pc_thunk_bx proc near         ; CODE XREF: sub_17350+3
.text:0012D5E3                                         ; sub_173CC+4 ...
.text:0012D5E3                 mov     ebx, [esp+0]
.text:0012D5E6                 retn
.text:0012D5E6 __x86_get_pc_thunk_bx endp

...

.text:000576C0 sub_576C0       proc near               ; CODE XREF: tmpfile+73

...

.text:000576C0                 push    ebp
.text:000576C1                 mov     ecx, large gs:0
.text:000576C8                 push    edi
.text:000576C9                 push    esi
.text:000576CA                 push    ebx
.text:000576CB                 call    __x86_get_pc_thunk_bx
.text:000576D0                 add     ebx, 157930h
.text:000576D6                 sub     esp, 9Ch

...

.text:000579F0                 lea     eax, (a__gen_tempname - 1AF000h)[ebx] ; "__gen_tempname"
.text:000579F6                 mov     [esp+0ACh+var_A0], eax
.text:000579FA                 lea     eax, (a__SysdepsPosix - 1AF000h)[ebx] ; "../sysdeps/posix/tempname.c"
.text:00057A00                 mov     [esp+0ACh+var_A8], eax
.text:00057A04                 lea     eax, (aInvalidKindIn_ - 1AF000h)[ebx] ; "! \"invalid KIND in __gen_tempname\""
.text:00057A0A                 mov     [esp+0ACh+var_A4], 14Ah
.text:00057A12                 mov     [esp+0ACh+var_AC], eax
.text:00057A15                 call    __assert_fail
\end{lstlisting}

Все указатели на строки корректируются при помощи некоторой константы из регистра \EBX, которая вычисляется в начале каждой функции.

Это так называемый адресно-независимый код (\ac{PIC}), он предназначен для исполнения будучи расположенным по любому адресу в памяти, вот почему он не содержит никаких абсолютных адресов в памяти.

\ac{PIC} был очень важен в ранних компьютерных системах и важен сейчас во встраиваемых\footnote{embedded}, не имеющих поддержки виртуальной памяти (все процессы расположены в одном непрерывном блоке памяти).
Он до сих пор используется в *NIX системах для динамических библиотек, потому что динамическая библиотека может использоваться одновременно в нескольких процессах, будучи загружена в память только один раз.
Но все эти процессы могут загрузить одну и ту же динамическую библиотеку по разным адресам, вот почему динамическая библиотека должна работать корректно, не привязываясь к абсолютным адресам.

Простой эксперимент:

\begin{lstlisting}[style=customc]
#include <stdio.h>

int global_variable=123;

int f1(int var)
{
    int rt=global_variable+var;
    printf ("returning %d\n", rt);
    return rt;
};
\end{lstlisting}

Скомпилируем в GCC 4.7.3 и посмотрим итоговый файл .so в \IDA:

\begin{lstlisting}
gcc -fPIC -shared -O3 -o 1.so 1.c
\end{lstlisting}

\begin{lstlisting}[caption=GCC 4.7.3,style=customasmx86]
.text:00000440                 public __x86_get_pc_thunk_bx
.text:00000440 __x86_get_pc_thunk_bx proc near         ; CODE XREF: _init_proc+4
.text:00000440                                         ; deregister_tm_clones+4 ...
.text:00000440                 mov     ebx, [esp+0]
.text:00000443                 retn
.text:00000443 __x86_get_pc_thunk_bx endp

.text:00000570                 public f1
.text:00000570 f1              proc near
.text:00000570
.text:00000570 var_1C          = dword ptr -1Ch
.text:00000570 var_18          = dword ptr -18h
.text:00000570 var_14          = dword ptr -14h
.text:00000570 var_8           = dword ptr -8
.text:00000570 var_4           = dword ptr -4
.text:00000570 arg_0           = dword ptr  4
.text:00000570
.text:00000570                 sub     esp, 1Ch
.text:00000573                 mov     [esp+1Ch+var_8], ebx
.text:00000577                 call    __x86_get_pc_thunk_bx
.text:0000057C                 add     ebx, 1A84h
.text:00000582                 mov     [esp+1Ch+var_4], esi
.text:00000586                 mov     eax, ds:(global_variable_ptr - 2000h)[ebx]
.text:0000058C                 mov     esi, [eax]
.text:0000058E                 lea     eax, (aReturningD - 2000h)[ebx] ; "returning %d\n"
.text:00000594                 add     esi, [esp+1Ch+arg_0]
.text:00000598                 mov     [esp+1Ch+var_18], eax
.text:0000059C                 mov     [esp+1Ch+var_1C], 1
.text:000005A3                 mov     [esp+1Ch+var_14], esi
.text:000005A7                 call    ___printf_chk
.text:000005AC                 mov     eax, esi
.text:000005AE                 mov     ebx, [esp+1Ch+var_8]
.text:000005B2                 mov     esi, [esp+1Ch+var_4]
.text:000005B6                 add     esp, 1Ch
.text:000005B9                 retn
.text:000005B9 f1              endp
\end{lstlisting}

\newcommand{\retstring}{\IT{<<returning \%d\textbackslash{}n>>}}
\newcommand{\globvar}{\IT{global\_variable}}

Так и есть: указатели на строку \retstring{} и переменную \globvar{} корректируются при каждом исполнении функции.

\par Функция \TT{\_\_x86\_get\_pc\_thunk\_bx()} возвращает адрес точки после вызова самой себя (здесь: \TT{0x57C}) в \EBX.
Это очень простой способ получить значение указателя на текущую инструкцию (\EIP) в произвольном месте.

Константа \TT{0x1A84} связана с разницей между началом этой функции и так называемой
\IT{Global Offset Table Procedure Linkage Table} (GOT PLT), секцией, сразу же за \IT{Global Offset Table} (GOT), где находится указатель на \globvar{}.
\IDA показывает смещения уже обработанными, чтобы их было проще понимать, но на самом деле код такой:

\begin{lstlisting}[style=customasmx86]
.text:00000577                 call    __x86_get_pc_thunk_bx
.text:0000057C                 add     ebx, 1A84h
.text:00000582                 mov     [esp+1Ch+var_4], esi
.text:00000586                 mov     eax, [ebx-0Ch]
.text:0000058C                 mov     esi, [eax]
.text:0000058E                 lea     eax, [ebx-1A30h]
\end{lstlisting}

Так что, \EBX указывает на секцию \TT{GOT PLT} и для вычисления указателя на \globvar{}, которая хранится в \TT{GOT}, нужно вычесть 0xC.
А чтобы вычислить указатель на \retstring{}, нужно вычесть \TT{0x1A30}.

\myindex{x86-64}
\myindex{x86!\Registers!RIP}
Кстати, вот зачем в AMD64 появилась поддержка адресации относительно RIP\footnote{указатель инструкций в AMD64}, просто для упрощения PIC-кода.

Скомпилируем тот же код на Си при помощи той же версии GCC, но для x64.

\myindex{objdump}
\IDA упростит код на выходе убирая упоминания RIP, так что будем использовать \IT{objdump} вместо нее:

\begin{lstlisting}[style=customasmx86]
0000000000000720 <f1>:
 720:	48 8b 05 b9 08 20 00 	mov    rax,QWORD PTR [rip+0x2008b9]        # 200fe0 <_DYNAMIC+0x1d0>
 727:	53                   	push   rbx
 728:	89 fb                	mov    ebx,edi
 72a:	48 8d 35 20 00 00 00 	lea    rsi,[rip+0x20]        # 751 <_fini+0x9>
 731:	bf 01 00 00 00       	mov    edi,0x1
 736:	03 18                	add    ebx,DWORD PTR [rax]
 738:	31 c0                	xor    eax,eax
 73a:	89 da                	mov    edx,ebx
 73c:	e8 df fe ff ff       	call   620 <__printf_chk@plt>
 741:	89 d8                	mov    eax,ebx
 743:	5b                   	pop    rbx
 744:	c3                   	ret    
\end{lstlisting}

\TT{0x2008b9} это разница между адресом инструкции по \TT{0x720} и \globvar{}, 
а \TT{0x20} это разница между инструкцией по \TT{0x72A} и строкой \retstring{}.

Как видно, необходимость очень часто пересчитывать адреса делает исполнение немного медленнее 
(хотя это и стало лучше в x64).
Так что если вы заботитесь о скорости исполнения, то, наверное, нужно задуматься о статической компоновке (static linking)
\InSqBrackets{см. \AgnerFogCPP}.

\subsubsection{Windows}
\myindex{Windows!Win32}

Такой механизм не используется в Windows DLL. Если загрузчику в Windows приходится загружать DLL 
в другое место, он \q{патчит} DLL прямо в памяти (на местах \IT{FIXUP}-ов) чтобы скорректировать 
все адреса.
Это приводит к тому что загруженную один раз DLL нельзя использовать одновременно в разных 
процессах, желающих расположить её по разным адресам --- потому что каждый загруженный в память 
экземпляр DLL \IT{доводится} до того чтобы работать только по этим адресам.

}
\DE{\subsection{\CapitalPICcode}
\myindex{\PICcode}
\myindex{Linux}
\label{sec:PIC}

Wenn der Code von Shared Libraries (.so) unter Linux analysiert wird, findet
man häufig das folgende Code-Muster:

\begin{lstlisting}[caption=libc-2.17.so x86]
.text:0012D5E3 __x86_get_pc_thunk_bx proc near         ; CODE XREF: sub_17350+3
.text:0012D5E3                                         ; sub_173CC+4 ...
.text:0012D5E3                 mov     ebx, [esp+0]
.text:0012D5E6                 retn
.text:0012D5E6 __x86_get_pc_thunk_bx endp

...

.text:000576C0 sub_576C0       proc near               ; CODE XREF: tmpfile+73

...

.text:000576C0                 push    ebp
.text:000576C1                 mov     ecx, large gs:0
.text:000576C8                 push    edi
.text:000576C9                 push    esi
.text:000576CA                 push    ebx
.text:000576CB                 call    __x86_get_pc_thunk_bx
.text:000576D0                 add     ebx, 157930h
.text:000576D6                 sub     esp, 9Ch

...

.text:000579F0                 lea     eax, (a__gen_tempname - 1AF000h)[ebx] ; "__gen_tempname"
.text:000579F6                 mov     [esp+0ACh+var_A0], eax
.text:000579FA                 lea     eax, (a__SysdepsPosix - 1AF000h)[ebx] ; "../sysdeps/posix/tempname.c"
.text:00057A00                 mov     [esp+0ACh+var_A8], eax
.text:00057A04                 lea     eax, (aInvalidKindIn_ - 1AF000h)[ebx] ; "! \"invalid KIND in __gen_tempname\""
.text:00057A0A                 mov     [esp+0ACh+var_A4], 14Ah
.text:00057A12                 mov     [esp+0ACh+var_AC], eax
.text:00057A15                 call    __assert_fail
\end{lstlisting}

Alle Zeiger auf Zeichenketten sind durch Konstanten und den Wert in \EBX korrigiert,
welcher zu Beginn jeder Funktion berechnet wird.

Dies ist sogenannter \ac{PIC}, und hat den Zweck an jeder beliebigen Stelle im Speicher
ausführbar zu sein. Aus diesem Grund können keine absoluten Speicheradressen verwendet werden.

\ac{PIC} war entscheidend in früheren Computer-Systemen und ist es immer noch in Eingebetteten
Systeme ohne virtuelle Speicherverwaltung in denen sich alle Prozesse in einem einzigen durchgängigen
Speicherbereich befinden.

Diese Technik wird auch heute noch in *NIX-Systemen für Shared Libraries verwendet
da diese von mehreren Prozessen genutzt, aber nur einmal in den Speicher geladen werden.
Jeder Prozess kann jedoch die gleiche Bibliothek an verschiedene Adressen \q{mappen}.
Aus diesem Grund muss diese Bibliothek auch ohne Verwendung absoluter Adressen funktionieren.

Machen wir ein sehr einfaches Experiment:

\begin{lstlisting}
#include <stdio.h>

int global_variable=123;

int f1(int var)
{
    int rt=global_variable+var;
    printf ("returning %d\n", rt);
    return rt;
};
\end{lstlisting}

Nachfolgende die kompilierte .so-Datei von GCC 4.7.3. in \IDA:

\begin{lstlisting}
gcc -fPIC -shared -O3 -o 1.so 1.c
\end{lstlisting}

\begin{lstlisting}[caption=GCC 4.7.3]
.text:00000440                 public __x86_get_pc_thunk_bx
.text:00000440 __x86_get_pc_thunk_bx proc near         ; CODE XREF: _init_proc+4
.text:00000440                                         ; deregister_tm_clones+4 ...
.text:00000440                 mov     ebx, [esp+0]
.text:00000443                 retn
.text:00000443 __x86_get_pc_thunk_bx endp

.text:00000570                 public f1
.text:00000570 f1              proc near
.text:00000570
.text:00000570 var_1C          = dword ptr -1Ch
.text:00000570 var_18          = dword ptr -18h
.text:00000570 var_14          = dword ptr -14h
.text:00000570 var_8           = dword ptr -8
.text:00000570 var_4           = dword ptr -4
.text:00000570 arg_0           = dword ptr  4
.text:00000570
.text:00000570                 sub     esp, 1Ch
.text:00000573                 mov     [esp+1Ch+var_8], ebx
.text:00000577                 call    __x86_get_pc_thunk_bx
.text:0000057C                 add     ebx, 1A84h
.text:00000582                 mov     [esp+1Ch+var_4], esi
.text:00000586                 mov     eax, ds:(global_variable_ptr - 2000h)[ebx]
.text:0000058C                 mov     esi, [eax]
.text:0000058E                 lea     eax, (aReturningD - 2000h)[ebx] ; "returning %d\n"
.text:00000594                 add     esi, [esp+1Ch+arg_0]
.text:00000598                 mov     [esp+1Ch+var_18], eax
.text:0000059C                 mov     [esp+1Ch+var_1C], 1
.text:000005A3                 mov     [esp+1Ch+var_14], esi
.text:000005A7                 call    ___printf_chk
.text:000005AC                 mov     eax, esi
.text:000005AE                 mov     ebx, [esp+1Ch+var_8]
.text:000005B2                 mov     esi, [esp+1Ch+var_4]
.text:000005B6                 add     esp, 1Ch
.text:000005B9                 retn
.text:000005B9 f1              endp
\end{lstlisting}

\newcommand{\retstring}{\IT{<<returning \%d\textbackslash{}n>>}}
\newcommand{\globvar}{\IT{global\_variable}}

Das ist es: die Zeiger auf \retstring{} und \globvar{} werden bei jedem Funktionsaufruf korrigiert.

\par Die \TT{\_\_x86\_get\_pc\_thunk\_bx()}-Funktion gibt in \EBX die Adresse auf eine Stelle nach einen
Aufruf von sich selbst zurück (hier \TT{0x57C}).

Dies ist eine einfache Möglichkeit um den Wert des Programmzählers (\EIP) an einer beliebigen Stelle zu erhalten.
Die Konstante \TT{0x1A84} gehört zu dem Unterschied des Funktionsanfangs und der sogenannten
\IT{Global Offset Table Procedure Linkage Table} (GOT PLT), die Sektion direkt hinter der \IT{Global Offset Table}
(GOT), an der der Zeiger auf \globvar{}.
\IDA zeigt diese Offsets aus Gründen des einfacheren Verständnisses in einer verarbeiteten Form an,
der Code ist aber wie folgt:

\begin{lstlisting}
.text:00000577                 call    __x86_get_pc_thunk_bx
.text:0000057C                 add     ebx, 1A84h
.text:00000582                 mov     [esp+1Ch+var_4], esi
.text:00000586                 mov     eax, [ebx-0Ch]
.text:0000058C                 mov     esi, [eax]
.text:0000058E                 lea     eax, [ebx-1A30h]
\end{lstlisting}

Hier zeigt \EBX auf die \TT{GOT PLT}-Sektion und um den Zeiger auf \globvar{} zu berechnen
(welcher in der \TT{GOT} gesichert ist) muss \TT{0xC} subtrahiert werden.

Um den Zeiger auf die Zeichenkette \retstring{} zu berechnen muss \TT{0x1A30} abgezogen werden.

\myindex{x86-64}
\myindex{x86!\Registers!RIP}

Übrigens ist dies der Grund warum die AMD64-Anweisungen RIP\footnote{Programmzähler in AMD64}-relative Adressierung
unterstützen: sie vereinfachen den PIC-Code.

Nachfolgend der selbe C-Code mit der gleichen GCC-Version, jedoch für x64 kompiliert.

\myindex{objdump}
\IDA würde den resultierenden Code vereinfachen aber auch die Details zur RIP-relativen Adressierung
unterdrücken. Um alles sehen zu können wird hier \IT{objdump} anstatt IDA genutzt.

\begin{lstlisting}
0000000000000720 <f1>:
 720:	48 8b 05 b9 08 20 00 	mov    rax,QWORD PTR [rip+0x2008b9]        # 200fe0 <_DYNAMIC+0x1d0>
 727:	53                   	push   rbx
 728:	89 fb                	mov    ebx,edi
 72a:	48 8d 35 20 00 00 00 	lea    rsi,[rip+0x20]        # 751 <_fini+0x9>
 731:	bf 01 00 00 00       	mov    edi,0x1
 736:	03 18                	add    ebx,DWORD PTR [rax]
 738:	31 c0                	xor    eax,eax
 73a:	89 da                	mov    edx,ebx
 73c:	e8 df fe ff ff       	call   620 <__printf_chk@plt>
 741:	89 d8                	mov    eax,ebx
 743:	5b                   	pop    rbx
 744:	c3                   	ret    
\end{lstlisting}

\TT{0x2008b9} ist der Unterschied zwischen der Adresse der Anweisung an \TT{0x720} und \globvar{},
und \TT{0x20} ist der Unterschied zwischen der Adresse der Anweisung an \TT{0x72A} und der
Zeichenkette \retstring{}.

Wie zu sehen ist, die Notwendigkeit die Adressen regelmäßig neu zu berechnen macht die Ausführung
etwas langsamer (auch wenn dies in x64 etwas besser ist).

Es mag also von Vorteil sein, statisch zu linken wenn Geschwindigkeit eine Rolle spielt \InSqBrackets{siehe: \AgnerFogCPP}.

\subsubsection{Windows}
\myindex{Windows!Win32}

Der PIC-Mechanismus wird in Windows-DLLs nicht genutzt. Wenn der Windows-Loader eine DLL an
eine andere Basisadresse laden muss, \q{patched} er diese im Speicher (and den \IT{FIXUP}-Platz)
um alle Adressen zu korrigieren.

Dies bedeutet, dass mehrere Windows-Prozesse eine einmal geladene DLL nicht teilen können
wenn diese an verschiedenen Adressen in verschiedenen Prozess-Speichern sein muss, da
jede Instanz im Speicher die Funktionen an einer festen Adresse erwartet.
}

\EN{\subsection{\IT{LD\_PRELOAD} hack in Linux}

\myindex{LD\_PRELOAD}
\label{ld_preload}

This allows us to load our own dynamic libraries before others, even before system ones, like libc.so.6.

This, in turn, allows us to \q{substitute} our written functions before the original ones in the system libraries.
For example, it is easy to intercept all calls to 
time(), read(), write(), etc. \\
\\
\myindex{uptime}
Let's see if we can fool the \IT{uptime} utility.
As we know, it tells how long the computer has been working.
\myindex{strace}
With the help of strace(\myref{strace}), it is possible to see that the utility takes this information the \TT{/proc/uptime} file:

\begin{lstlisting}
$ strace uptime 
...
open("/proc/uptime", O_RDONLY)          = 3
lseek(3, 0, SEEK_SET)                   = 0
read(3, "416166.86 414629.38\n", 2047)  = 20
...
\end{lstlisting}

It is not a real file on disk, it is a virtual one and its contents are generated on fly in the Linux kernel.
There are just two numbers:

\begin{lstlisting}
$ cat /proc/uptime
416690.91 415152.03
\end{lstlisting}

What we can learn from Wikipedia
\footnote{\href{http://go.yurichev.com/17043}{wikipedia}}:

\begin{framed}
\begin{quotation}
The first number is the total number of seconds the system has been up.
The second number is how much of that time the machine has spent idle, in seconds.
\end{quotation}
\end{framed}

\myindex{\CStandardLibrary!open()}
\myindex{\CStandardLibrary!read()}
\myindex{\CStandardLibrary!close()}

Let's try to write our own dynamic library with the open(), read(), close() 
functions working as we need.

At first, our open() will compare the name of the file to be opened with what we need and if it is so,
it will write down the descriptor of the file opened.

Second, read(), if called for this file descriptor, will substitute the output,
and in the rest of the cases will call the original read() from libc.so.6.
And also close(), 
will note if the file we are currently following is to be closed.

\myindex{dlopen()}
\myindex{dlsym()}

We are going to use the dlopen() and dlsym() functions to determine the original function addresses in libc.so.6.

We need them because we must pass control to the \q{real} functions.

\myindex{\CStandardLibrary!strcmp()}

On the other hand, if we intercepted strcmp() and monitored each string
comparisons in the program, then we would have to implement our version of strcmp(), and not
use the original function
\footnote{For example, here is how simple strcmp() interception works in this article
\footnote{\href{http://go.yurichev.com/17143}{yurichev.com}}
written by Yong Huang}, that would be easier.

\lstinputlisting[style=customc]{OS/LD_PRELOAD/fool_uptime.c}
( \href{https://github.com/dennis714/RE-for-beginners/blob/master/OS/LD_PRELOAD/fool_uptime.c}{Source code at GitHub} )
% FIXME go.yurichev.com...

Let's compile it as common dynamic library:

\begin{lstlisting}
gcc -fpic -shared -Wall -o fool_uptime.so fool_uptime.c -ldl
\end{lstlisting}

Let's run \IT{uptime}
while loading our library before the others:

\begin{lstlisting}
LD_PRELOAD=`pwd`/fool_uptime.so uptime
\end{lstlisting}

And we see:

\begin{lstlisting}
 01:23:02 up 24855 days,  3:14,  3 users,  load average: 0.00, 0.01, 0.05
\end{lstlisting}

If the \IT{LD\_PRELOAD} 

environment variable always points to the filename and path of our library, 
it is to be loaded for all starting programs. \\
\\
More examples:

\begin{itemize}

\item
Very simple interception of the strcmp() (Yong Huang) 
\url{http://go.yurichev.com/17143}

\item
Kevin Pulo---Fun with LD\_PRELOAD. A lot of examples and ideas.
\href{http://go.yurichev.com/17145}{yurichev.com}

\item
File functions interception for compression/decompression files on fly (zlibc). \url{http://go.yurichev.com/17146}

\end{itemize}
}
\RU{\subsection{Трюк с \IT{LD\_PRELOAD} в Linux}

\myindex{LD\_PRELOAD}
\label{ld_preload}

Это позволяет загружать свои динамические библиотеки перед другими, даже перед системными, такими как libc.so.6.

Что в свою очередь, позволяет \q{подставлять} написанные нами функции перед оригинальными из системных библиотек.

Например, легко перехватывать все вызовы к time(), read(), write(), итд. \\
\\
\myindex{uptime}
Попробуем узнать, сможем ли мы обмануть утилиту \IT{uptime}.
Как известно, она сообщает, как долго компьютер работает.
\myindex{strace}
При помощи strace(\myref{strace}), можно увидеть, что эту информацию утилита получает из файла \TT{/proc/uptime}:

\begin{lstlisting}
$ strace uptime 
...
open("/proc/uptime", O_RDONLY)          = 3
lseek(3, 0, SEEK_SET)                   = 0
read(3, "416166.86 414629.38\n", 2047)  = 20
...
\end{lstlisting}

Это не реальный файл на диске, это виртуальный файл, содержимое которого генерируется на лету в ядре Linux.

Там просто два числа:

\begin{lstlisting}
$ cat /proc/uptime
416690.91 415152.03
\end{lstlisting}

Из Wikipedia, можно узнать
\footnote{\href{http://go.yurichev.com/17043}{wikipedia}}:

\begin{framed}
\begin{quotation}
The first number is the total number of seconds the system has been up.
The second number is how much of that time the machine has spent idle, in seconds.
\end{quotation}
\end{framed}

\myindex{\CStandardLibrary!open()}
\myindex{\CStandardLibrary!read()}
\myindex{\CStandardLibrary!close()}
Попробуем написать свою динамическую библиотеку, в которой будет open(), read(), close() с нужной нам функциональностью.

Во-первых, наш open() будет сравнивать имя открываемого файла с тем что нам нужно, и если да, 
то будет запоминать дескриптор открытого файла.

Во-вторых, read(), если будет вызываться для этого дескриптора, будет подменять вывод,
а в остальных случаях, будет вызывать настоящий read() из libc.so.6.
А также close(), будет следить, закрывается ли файл за которым мы следим.

\myindex{dlopen()}
\myindex{dlsym()}
Для того чтобы найти адреса настоящих функций в libc.so.6, используем dlopen() и dlsym().

Нам это нужно, потому что нам нужно передавать управление \q{настоящим} функциями.

\myindex{\CStandardLibrary!strcmp()}
С другой стороны, если бы мы перехватывали, скажем, strcmp(), и следили бы за всеми сравнениями строк в программе, 
то, наверное, strcmp() можно было бы и самому реализовать, не пользуясь настоящей функцией
\footnote{Например, посмотрите как обеспечивается простейший перехват strcmp() в статье
\footnote{\href{http://go.yurichev.com/17143}{yurichev.com}} написанной Yong Huang}, так было бы проще.

\lstinputlisting[style=customc]{OS/LD_PRELOAD/fool_uptime.c}
( \href{https://github.com/dennis714/RE-for-beginners/blob/master/OS/LD_PRELOAD/fool_uptime.c}{Исходный код на GitHub} )
% FIXME go.yurichev.com...

Компилируем как динамическую библиотеку:

\begin{lstlisting}
gcc -fpic -shared -Wall -o fool_uptime.so fool_uptime.c -ldl
\end{lstlisting}

Запускаем \IT{uptime}, подгружая нашу библиотеку перед остальными:

\begin{lstlisting}
LD_PRELOAD=`pwd`/fool_uptime.so uptime
\end{lstlisting}

Видим такое:

\begin{lstlisting}
 01:23:02 up 24855 days,  3:14,  3 users,  load average: 0.00, 0.01, 0.05
\end{lstlisting}

Если переменная окружения \IT{LD\_PRELOAD} 
будет всегда указывать на путь и имя файла нашей библиотеки, то она будет
загружаться для всех запускаемых программ.  \\
\\
Еще примеры:

\begin{itemize}
\item
Перехват time() в Sun Solaris \href{http://go.yurichev.com/17144}{yurichev.com}

\item
Очень простой перехват strcmp() (Yong Huang) 
\url{http://go.yurichev.com/17143}

\item
Kevin Pulo --- Fun with LD\_PRELOAD. Много примеров и идей.
\href{http://go.yurichev.com/17145}{yurichev.com}

\item
Перехват функций работы с файлами для компрессии и декомпрессии файлов на лету (zlibc). \url{http://go.yurichev.com/17146}

\end{itemize}
}
\DE{\subsection{\IT{LD\_PRELOAD}-Hack in Linux}

\myindex{LD\_PRELOAD}
\label{ld_preload}

Diese Technik erlaubt es eigene, dynamische Bibliotheken vor anderen zu laden\dots{}
sogar vor denen des Systems, wie libc.so.6.

Dies wiederum erlaubt es die eigenen Funktionen für die des Systems zu \q{ersetzen}.
Es ist zum Beispiel einfach alle Aufrufe zu time(), read(), write(), usw. abzufangen.

\myindex{uptime}
Sehen wir uns einmal an, wie das Tool \IT{uptime} ausgetrickst werden kann.
Wie bekannt ist, zeigt dieses Programm an, wie lange der Computer schon arbeitet.
\myindex{strace}
Mithilfe von strace(\myref{strace}), ist es möglich zu sehen, dass das Tool die
Informationen aus der Datei \TT{/proc/uptime} ausliest:

\begin{lstlisting}
$ strace uptime 
...
open("/proc/uptime", O_RDONLY)          = 3
lseek(3, 0, SEEK_SET)                   = 0
read(3, "416166.86 414629.38\n", 2047)  = 20
...
\end{lstlisting}

Es handelt sich dabei nicht um eine reale Datei auf der Festplatte sondern um eine
virtuelle, bei der die Dateien on-the-fly im Linux Kernel erstellt werden.
Es gibt zwei Zahlen:

\begin{lstlisting}
$ cat /proc/uptime
416690.91 415152.03
\end{lstlisting}

Nachfolgend, der englischen Wikipedia\footnote{\href{http://go.yurichev.com/17043}{Wikipedia}}:

\begin{framed}
\begin{quotation}
The first number is the total number of seconds the system has been up.
The second number is how much of that time the machine has spent idle, in seconds.
\end{quotation}
\end{framed}

\myindex{\CStandardLibrary!open()}
\myindex{\CStandardLibrary!read()}
\myindex{\CStandardLibrary!close()}

Versuchen wir eine eigene dynamische Bibliothek mit den Funktionen open(), read()
und close() zu schreiben die so funktionieren wie wir es gerne hätten.

Zunächst wird unsere open()-Funktion den Namen der zu öffnenden Datei mit dem was
wir brauchen vergleichen. Ist dies der Fall, soll der Deskriptor der geöffnete Datei
geschrieben werden.

Als zweites read(): Wenn diese Funktion für den Datei-Deskriptor aufgerufen wird,
soll die Ausgaben ersetzt werden und der Rest dem original read() aus libc.so.6
entsprechen.
close() wird eine Meldung geben wenn die Datei der zur Zeit gefolgt wird geschlossen
wurde.

\myindex{dlopen()}
\myindex{dlsym()}

Wir werden die dlopen()- und dlsym()-Funktionen nutzen, um die Adressen der Original-Funktionen
in libc.so.6 herauszufinden.

Diese werden benötigt, weil die Ausführkontrolle wieder an die \q{realen} Funktionen
übergeben werden müssen.

\myindex{\CStandardLibrary!strcmp()}

Auf der anderen Seite: wenn wir strcmp() unterbrechen und jeden einzelnen Vergleich
von Zeichenketten im Programm untersuchen, müssten wir eine eigene strcmp()-Variante
schreiben und nicht die Original-Funktion
nutzen\footnote{Als Beispiel, wie einfach strcmp()-Unterbrechung funktioniert
\footnote{\href{http://go.yurichev.com/17143}{yurichev.com}} von Yong Huang}, was
einfacher wäre.

\lstinputlisting[style=customc]{OS/LD_PRELOAD/fool_uptime.c}
( \href{https://github.com/dennis714/RE-for-beginners/blob/master/OS/LD_PRELOAD/fool_uptime.c}{Quellcode auf GitHub} )
% FIXME go.yurichev.com...

Kompilieren wir den Code als gemeinsame, dynamische Bibliothek:

\begin{lstlisting}
gcc -fpic -shared -Wall -o fool_uptime.so fool_uptime.c -ldl
\end{lstlisting}

Jetzt starten wir \IT{uptime} während unsere Bibliothek vor den anderen geladen wird:

\begin{lstlisting}
LD_PRELOAD=`pwd`/fool_uptime.so uptime
\end{lstlisting}

Und wir sehen:

\begin{lstlisting}
 01:23:02 up 24855 days,  3:14,  3 users,  load average: 0.00, 0.01, 0.05
\end{lstlisting}

Wenn die \IT{LD\_PRELOAD}-Umgebungsvariable immer auf den Dateinamen und -pfad unserer
Bibliothek zeigt, wird diese vor allen anderen gestarteten Programmen geladen.

Weitere Beispiele:

\begin{itemize}

\item
Sehr einfache Unterbrechung von strcmp() (Yong Huang)
\url{http://go.yurichev.com/17143}

\item
Kevin Pulo---Fun with LD\_PRELOAD. Viele Beispiele und Ideen.
\href{http://go.yurichev.com/17145}{yurichev.com}

\item
Datei-Funktionen unterbrechen beim Komprimieren/Entkomprimieren on-the-fly (zlibc). \url{http://go.yurichev.com/17146}
\end{itemize}
}

\section{Windows NT}
\EN{\subsection{CRT (win32)}
\label{sec:CRT}
\myindex{CRT}

Does the program execution start right at the \main{} function?
No, it does not.

If we would open any executable file in \IDA or HIEW, 
we can see \ac{OEP} pointing to some another code block.

This code is doing some maintenance and preparations before passing control flow to our code.
It is called startup-code or CRT code (C RunTime). \\
\\
The \main{} function takes an array of the arguments passed on the command line, and also
one with environment variables.
But in fact a generic string is passed to the program,
the CRT code finds the spaces in it and cuts it in parts.
The CRT code also prepares the environment
variables array \TT{envp}.

As for \ac{GUI} win32 applications, \TT{WinMain} is used instead of \main{}, having its own arguments:

\begin{lstlisting}[style=customc]
int CALLBACK WinMain(
  _In_  HINSTANCE hInstance,
  _In_  HINSTANCE hPrevInstance,
  _In_  LPSTR lpCmdLine,
  _In_  int nCmdShow
);
\end{lstlisting}

The CRT code prepares them as well.

Also, the number returned by the \main{} function is the exit code.

It may be passed in CRT to the \TT{ExitProcess()} function, which takes the exit code as an argument. \\
\\
Usually, each compiler has its own CRT code. \\
\\
Here is a typical CRT code for MSVC 2008.

\lstinputlisting[numbers=left,style=customasmx86]{OS/win32_CRT/crt_msvc_2008.asm}

Here we can see calls to \TT{GetCommandLineA()} (line 62),
then to \TT{setargv()} (line 66) and \TT{setenvp()} (line 74),
which apparently fill the global variables
\TT{argc}, \TT{argv}, \TT{envp}.

Finally, \main{} is called with these arguments (line 97).

There are also calls to functions
with self-describing names like \TT{heap\_init()} (line 35), \TT{ioinit()} (line 54).

The \glslink{heap}{heap} is indeed initialized in the \ac{CRT}.
If you try to use \TT{malloc()} in a program without CRT, it will exit abnormally with the following error:

\begin{lstlisting}
runtime error R6030
- CRT not initialized
\end{lstlisting}

Global object initializations in \Cpp is also occur in the \ac{CRT} before the execution of \main{}: 
\myref{sec:std_string_as_global_variable}.

The value that \main{} returns is passed to \TT{cexit()}, 
or in \TT{\$LN32}, which in turn calls \TT{doexit()}.

Is it possible to get rid of the \ac{CRT}?
Yes, if you know what you are doing.

The \ac{MSVC}'s linker has the \TT{/ENTRY} option for setting an entry point.

\begin{lstlisting}[style=customc]
#include <windows.h>

int main()
{
	MessageBox (NULL, "hello, world", "caption", MB_OK);
};
\end{lstlisting}

Let's compile it in MSVC 2008.

\begin{lstlisting}
cl no_crt.c user32.lib /link /entry:main
\end{lstlisting}

We are getting a runnable .exe with size 2560 bytes, that has a PE header in it, instructions calling
\TT{MessageBox}, two strings in the data segment,
the \TT{MessageBox} function imported from \TT{user32.dll} and nothing else.

This works, but you cannot write \TT{WinMain} with its 4 arguments instead of \main{}.

To be precise, you can, but the arguments are not prepared at the moment of execution.

By the way, it is possible to make the .exe even 
shorter by aligning the \ac{PE} sections at less than the default 4096 bytes.

\begin{lstlisting}
cl no_crt.c user32.lib /link /entry:main /align:16
\end{lstlisting}

Linker says:

\begin{lstlisting}
LINK : warning LNK4108: /ALIGN specified without /DRIVER; image may not run
\end{lstlisting}

We get an .exe that's 720 bytes.
It can be executed in Windows 7 x86, but not in x64 
(an error message will be shown when you try to execute it).

With even more efforts, it is possible
to make the executable even shorter, but as you can see, compatibility problems arise quickly.

}
\RU{\subsection{CRT (win32)}
\label{sec:CRT}
\myindex{CRT}

Начинается ли исполнение программы прямо с функции \main{}?
Нет, не начинается.
Если открыть любой исполняемый файл в \IDA или Hiew, 
то \ac{OEP} указывает на какой-то совсем другой код.

Это код, который делает некоторые приготовления перед тем как запустить ваш код.
Он называется стартап-код или CRT-код (C RunTime). \\
\\
Функция \main{} принимает на вход массив из параметров, переданных в командной строке, а также
переменные окружения.
Но в реальности в программу передается командная строка в виде простой строки, это именно
CRT-код находит там пробелы и разрезает строку на части.
CRT-код также готовит массив переменных окружения \TT{envp}.
В \ac{GUI}-приложениях win32, вместо \main{} имеется функция \TT{WinMain} со своими аргументами:

\begin{lstlisting}[style=customc]
int CALLBACK WinMain(
  _In_  HINSTANCE hInstance,
  _In_  HINSTANCE hPrevInstance,
  _In_  LPSTR lpCmdLine,
  _In_  int nCmdShow
);
\end{lstlisting}

CRT-код готовит и их.

А также, число, возвращаемое функцией \main{}, это код ошибки возвращаемый программой.
В CRT это значение передается в \TT{ExitProcess()}, принимающей в качестве аргумента код ошибки. \\
\\
Как правило, каждый компилятор имеет свой CRT-код. \\
\\
Вот типичный для MSVC 2008 CRT-код.

\lstinputlisting[numbers=left,style=customasmx86]{OS/win32_CRT/crt_msvc_2008.asm}

Здесь можно увидеть по крайней мере вызов
функции \TT{GetCommandLineA()} (строка 62), 
затем \TT{setargv()} (строка 66) и \TT{setenvp()} (строка 74),
которые, видимо, заполняют глобальные переменные-указатели
\TT{argc}, \TT{argv}, \TT{envp}.

В итоге, вызывается \main{} с этими аргументами (строка 97).

Также имеются вызовы функций с говорящими именами вроде \TT{heap\_init()} (строка 35), \TT{ioinit()} (строка 54).

\glslink{heap}{Куча} действительно инициализируется в \ac{CRT}.
Если вы попытаетесь использовать \TT{malloc()} в программе без CRT, программа упадет с такой ошибкой:

\begin{lstlisting}
runtime error R6030
- CRT not initialized
\end{lstlisting}

Инициализация глобальных объектов в \Cpp происходит до вызова \main{}, именно в \ac{CRT}: 
\myref{sec:std_string_as_global_variable}.

Значение, возвращаемое из \main{} передается или в \TT{cexit()}, 
или же в \TT{\$LN32}, которая далее вызывает \TT{doexit()}.

Можно ли обойтись без \ac{CRT}? Можно, если вы знаете что делаете.

В линкере от \ac{MSVC} точка входа задается опцией \TT{/ENTRY}.

\begin{lstlisting}[style=customc]
#include <windows.h>

int main()
{
	MessageBox (NULL, "hello, world", "caption", MB_OK);
};
\end{lstlisting}

Компилируем в MSVC 2008.

\begin{lstlisting}
cl no_crt.c user32.lib /link /entry:main
\end{lstlisting}

Получаем вполне работающий .exe размером 2560 байт, внутри которого есть только PE-заголовок, инструкции, 
вызывающие \TT{MessageBox},
две строки в сегменте данных, импортируемая из \TT{user32.dll} функция \TT{MessageBox}, и более ничего.

Это работает, но вы уже не сможете вместо \main{} написать \TT{WinMain} с его четырьмя аргументами.
Вернее, если быть точным, написать-то сможете, но доступа к этим аргументам не будет, 
потому что они не подготовлены на момент исполнения.

Кстати, можно еще короче сделать .exe если уменьшить 
выравнивание \ac{PE}-секций (которое, по умолчанию, 4096 байт).

\begin{lstlisting}
cl no_crt.c user32.lib /link /entry:main /align:16
\end{lstlisting}

Линкер скажет:

\begin{lstlisting}
LINK : warning LNK4108: /ALIGN specified without /DRIVER; image may not run
\end{lstlisting}

Получим .exe размером 720 байт.
Он запускается в Windows 7 x86, но не x64 
(там выдает ошибку при загрузке).
При желании, размер можно еще сильнее ужать, но, как видно, 
возникают проблемы с совместимостью с разными версиями Windows.

}
\DE{\subsection{CRT (win32)}
\label{sec:CRT}
\myindex{CRT}

Startet ein Programm genau bei der \main{}-Funktion?
Nein, tut es nicht!

Würden wir jede ausführbare Datei in \IDA oder HIEW öffnen können, würde wir sehen,
dass \ac{OEP} auf einen anderen Code-Block zeigt.

Dieser Code erledigt einige Vorbereitungen bevor die Ausführungskontrolle an unseren
Code übergeben wird.
Dies ist der sogenannte Startup- oder CRT-Code (C RunTime).

Die \main{}-Funktion nimmt ein Array mit den Argumenten entgegen, die auf der Kommandozeile
übergeben wurde, sowie eins mit den Umgebungsvariablen.
Genaugenommen wird eine normale Zeichenkette an das Programm übergeben und der CRT-Code
unterteilt diesen anhand der Leerzeichen in seine Bestandteile.
Der CRT-Code bereitet auch das Array \TT{envp} vor, welches die Umgebungsvariablen enthält.

Für \ac{GUI}-Programme unter Win32 wird \TT{WinMain} anstatt \main{} genutzt, welches eigene
Argumente hat:

\begin{lstlisting}
int CALLBACK WinMain(
  _In_  HINSTANCE hInstance,
  _In_  HINSTANCE hPrevInstance,
  _In_  LPSTR lpCmdLine,
  _In_  int nCmdShow
);
\end{lstlisting}

Der CRT-Code bereitet diese ebenfalls vor.

Die Zahl die von \main{} zurückgegeben wird ist der Exit-Code.

Diese kann in der CRT für die Funktion \TT{ExitProcess()} werden, die diesen Exit-Code als
Argument entgegennimmt.

In der Regel hat jeder Compiler seinen eigenen CRT-Code.

Nachfolgend eine typischer CRT-Code für MSVC 2008.

%TODO In german translation the customasm style can not be compiled.
%\lstinputlisting[numbers=left,style=customasm]{OS/win32_CRT/crt_msvc_2008.asm}
\lstinputlisting[numbers=left]{OS/win32_CRT/crt_msvc_2008.asm}

Wir sehen hier die Aufrufe zu \TT{GetCommandLineA()} (Zeile 62), anschließend zu \TT{setargv()}
(Zeile 66) und \TT{setenvp()} (Zeile 74), was offensichtlich die globalen Variablen \TT{argc},
\TT{argv} und  \TT{envp} initialisiert.

Zum Schluss wird \main{} mit diesen Argumenten aufgerufen (Zeile 97).

Es sind ebenfalls Aufrufe zu Funktionen mit selbsterklärenden Namen zu finden, wie \TT{heap\_init()}
(Zeile 35) und \TT{ioinit()} (Zeile 54).

Der \glslink{heap}{heap} wird jedoch vom \ac{CRT} initialisiert.
Wenn man versucht \TT{malloc()} in einem Programm ohne CRT zu nutzen, wird dieses mit dem folgenden
Fehler abstürzen:

\begin{lstlisting}
runtime error R6030
- CRT not initialized
\end{lstlisting}

Die Initialisierung von globalen Objekten in \Cpp passiert ebenfalls in der \ac{CRT} vor der
Ausführung von \main{}:
\myref{sec:std_string_as_global_variable}.

%TODO SLN32 not knwon in german translation
%Der Wert den \main{} zurückgibt wird an \TT{cexit()} übergeben oder in \TT{\SLN32}, welches
%\TT{doexit()} aufruft.

Ist es möglich die \ac{CRT} loszuwerden?
Ja, wenn man genau weiß, was man tut.

Der Linker von \ac{MSVC} hat die \TT{/ENTRY}-Option um den Einsprungpunkt festzulegen.

\begin{lstlisting}
#include <windows.h>

int main()
{
	MessageBox (NULL, "hello, world", "caption", MB_OK);
};
\end{lstlisting}

Kompilieren wir dies in MSVC 2008.

\begin{lstlisting}
cl no_crt.c user32.lib /link /entry:main
\end{lstlisting}

Wir bekommen eine lauffähige .exe-Datei mit der Größe 2560 Byte mit einem PE-Header,
Anweisungen die \TT{MessageBox} aufrufen, zwei Zeichenketten im Datensegment, die aus
\TT{user32.dll} importierte Funktion \TT{MessageBox} und sonst nichts.

Dies funktioniert, jedoch kann nicht \TT{WinMain} mit den 4 Argumenten anstatt \main{}
genutzt werden.

Um genau zu sein, wäre dies zwar möglich, allerdings wurden die Argumente nicht vorbereitet
um sie zu nutzen.

Es ist übrigens auch möglich die .exe-Datei kleiner machen indem die \ac{PE}-Sektion
an weniger als den standardmäßigen 4096 Byte auszurichten.

\begin{lstlisting}
cl no_crt.c user32.lib /link /entry:main /align:16
\end{lstlisting}

Der Linker gibt aus:

\begin{lstlisting}
LINK : warning LNK4108: /ALIGN specified without /DRIVER; image may not run
\end{lstlisting}

Die Ausgabe ist eine .exe-Datei mit 720 Byte.
Sie kann unter Windows 7 x86, jedoch nicht x64 ausgeführt werden (beim Versuch wird
eine Fehlermeldung erscheinen).

Mit mehr Aufwand ist es möglich die Datei noch weiter zu verkleinern, aber wie zu
sehen ist, bekommt man schnell Kompatibilitätsprobleme.

}

\EN{\subsection{Win32 PE}
\label{win32_pe}
\myindex{Windows!Win32}

\acs{PE} is an executable file format used in Windows.
The difference between .exe, .dll and .sys is that .exe and .sys usually do not have exports, only imports.

\myindex{OEP}

A \ac{DLL}, just like any other PE-file, has an entry point (\ac{OEP}) (the function DllMain() is located there)
but this function usually does nothing.
.sys is usually a device driver.
As of drivers, Windows requires the checksum to be present in the PE file and for it to be correct
\footnote{For example, Hiew(\myref{Hiew}) can calculate it}.

\myindex{Windows!Windows Vista}
Starting at Windows Vista, a driver's files must also be signed with a digital signature. It will fail to load otherwise.

\myindex{MS-DOS}
Every PE file begins with tiny DOS program that prints a
message like \q{This program cannot be run in DOS mode.}---if you run this program in DOS or Windows 3.1 (\ac{OS}-es which are not aware of the PE format),
this message will be printed.

\subsubsection{Terminology}

\myindex{VA}
\myindex{Base address}
\myindex{RVA}
\myindex{Windows!IAT}
\myindex{Windows!INT}

\begin{itemize}
\item Module---a separate file, .exe or .dll.

\item Process---a program loaded into memory and currently running.  Commonly consists of one .exe file and bunch of .dll files.

\item Process memory---the memory a process works with.  Each process has its own.
There usually are loaded modules, memory of the stack, \gls{heap}(s), etc.

\item \ac{VA}---an address which is to be used in program while runtime.

\item Base address (of module)---the address within the process memory at which the module is to be loaded.
\ac{OS} loader may change it, if the base address is already occupied by another module just loaded before.

\item \ac{RVA}---the \ac{VA}-address minus the base address.

Many addresses in PE-file tables use \ac{RVA}-addresses.

%\item
%Data directory\EMDASH{}...

\item \ac{IAT}---an array of addresses of imported symbols \footnote{\PietrekPE}.
Sometimes, the \TT{IMAGE\_DIRECTORY\_ENTRY\_IAT} data directory points at the \ac{IAT}.
\label{IDA_idata}
It is worth noting that \ac{IDA} (as of 6.1) may allocate a pseudo-section named \TT{.idata} for
\ac{IAT}, even if the \ac{IAT} is a part of another section!

\item \ac{INT}---an array of names of symbols to be imported\footnote{\PietrekPE}.
\end{itemize}

\subsubsection{Base address}

The problem is that several module authors can prepare DLL files for others to use and it is not possible
to reach an agreement which addresses is to be assigned to whose modules.

So that is why if two necessary DLLs for a process have the same base address,
one of them will be loaded at this base address, and the other---at some other free space in process memory,
and each virtual addresses in the second DLL will be corrected.

\par With \ac{MSVC} the linker often generates the .exe files with a base address of \TT{0x400000}
\footnote{The origin of this address choice is described here: \href{http://go.yurichev.com/17041}{MSDN}},
and with the code section starting at \TT{0x401000}.
This means that the \ac{RVA} of the start of the code section is \TT{0x1000}.

DLLs are often generated by MSVC's linker with a base address of \TT{0x10000000}
\footnote{This can be changed by the /BASE linker option}.

\myindex{ASLR}

There is also another reason to load modules at various base addresses, in this case random ones.
It is \ac{ASLR}\footnote{\href{http://go.yurichev.com/17140}{wikipedia}}.

\myindex{Shellcode}

A shellcode trying to get executed on a compromised system must call system functions, hence, know their addresses.

In older \ac{OS} (in \gls{Windows NT} line: before Windows Vista),
system DLL (like kernel32.dll, user32.dll) were always loaded at known addresses,
and if we also recall
that their versions rarely changed, the addresses of functions were
fixed and shellcode could call them directly.

In order to avoid this, the \ac{ASLR}
method loads your program and all modules it needs at random base addresses, different every time.

\ac{ASLR} support is denoted in a PE file by setting the flag
\par \TT{IMAGE\_DLL\_CHARACTERISTICS\_DYNAMIC\_BASE} \InSqBrackets{see \Russinovich}.

\subsubsection{Subsystem}

There is also a \IT{subsystem} field, usually it is:

\myindex{Native API}

\begin{itemize}
\item native\footnote{Meaning, the module use Native API instead of Win32} (.sys-driver),

\item console (console application) or

\item \ac{GUI} (non-console).
\end{itemize}

\subsubsection{OS version}

A PE file also specifies the minimal Windows version it needs in order to be loadable.

The table of version numbers stored in the PE file and corresponding Windows codenames is
here\footnote{\href{http://go.yurichev.com/17044}{wikipedia}}.

\myindex{Windows!Windows NT4}
\myindex{Windows!Windows 2000}
For example, \ac{MSVC} 2005 compiles .exe files for running on Windows NT4 (version 4.00), but \ac{MSVC} 2008 does not
(the generated files have a version of 5.00, at least Windows 2000 is needed to run them).

\myindex{Windows!Windows XP}

\ac{MSVC} 2012 generates .exe files of version 6.00 by default,
targeting at least Windows Vista.
However, by changing the compiler's options\footnote{\href{http://go.yurichev.com/17045}{MSDN}},
it is possible to force it to compile for Windows XP.

\subsubsection{Sections}

Division in sections, as it seems, is present in all executable file formats.

It is devised in order to separate code from data, and data---from constant data.

\begin{itemize}
\item Either the \IT{IMAGE\_SCN\_CNT\_CODE} or \IT{IMAGE\_SCN\_MEM\_EXECUTE} flags will be set on the code section---this is executable code.

\item On data section---\IT{IMAGE\_SCN\_CNT\_INITIALIZED\_DATA},\\
\IT{IMAGE\_SCN\_MEM\_READ} and \IT{IMAGE\_SCN\_MEM\_WRITE} flags.

\item On an empty section with uninitialized data---\\
\IT{IMAGE\_SCN\_CNT\_UNINITIALIZED\_DATA}, \IT{IMAGE\_SCN\_MEM\_READ} \\
        and \IT{IMAGE\_SCN\_MEM\_WRITE}.

\item On a constant data section (one that's protected from writing), the flags \\
\IT{IMAGE\_SCN\_CNT\_INITIALIZED\_DATA} and \IT{IMAGE\_SCN\_MEM\_READ} can be set, \\
but not \IT{IMAGE\_SCN\_MEM\_WRITE}.
A process going to crash if it tries to write to this section.
\end{itemize}

\myindex{TLS}
\myindex{BSS}
Each section in PE-file may have a name, however, it is not very important.
Often (but not always) the code section is named \TT{.text},
the data section---\TT{.data}, the constant data section --- \TT{.rdata} \IT{(readable data)}.
Other popular section names are:

\myindex{MIPS}
\begin{itemize}
\item \TT{.idata}---imports section.
\ac{IDA} may create a pseudo-section named like this: \myref{IDA_idata}.
\item \TT{.edata}---exports section (rare)
\item \TT{.pdata}---section holding all information about exceptions in Windows NT for MIPS, \ac{IA64} and x64: \myref{SEH_win64}
\item \TT{.reloc}---relocs section
\item \TT{.bss}---uninitialized data (\ac{BSS})
\item \TT{.tls}---thread local storage (\ac{TLS})
\item \TT{.rsrc}---resources
\item \TT{.CRT}---may present in binary files compiled by ancient MSVC versions
\end{itemize}

PE file packers/encryptors often garble section names or replace the names with their own.

\ac{MSVC} allows you to declare data in arbitrarily named section
\footnote{\href{http://go.yurichev.com/17047}{MSDN}}.

Some compilers and linkers can add a section with debugging symbols and
other debugging information (MinGW for instance).
\myindex{Windows!PDB}
However it is not so in latest versions of \ac{MSVC} (separate \gls{PDB} files are used there for this purpose).\\
\\
That is how a PE section is described in the file:

\begin{lstlisting}[style=customc]
typedef struct _IMAGE_SECTION_HEADER {
  BYTE  Name[IMAGE_SIZEOF_SHORT_NAME];
  union {
    DWORD PhysicalAddress;
    DWORD VirtualSize;
  } Misc;
  DWORD VirtualAddress;
  DWORD SizeOfRawData;
  DWORD PointerToRawData;
  DWORD PointerToRelocations;
  DWORD PointerToLinenumbers;
  WORD  NumberOfRelocations;
  WORD  NumberOfLinenumbers;
  DWORD Characteristics;
} IMAGE_SECTION_HEADER, *PIMAGE_SECTION_HEADER;
\end{lstlisting}
\footnote{\href{http://go.yurichev.com/17048}{MSDN}}

\myindex{Hiew}
A word about terminology: \IT{PointerToRawData} is called \q{Offset} in Hiew
and \IT{VirtualAddress} is called \q{RVA} there.

\subsubsection{Data section}

Data section in file can be smaller than in memory.
For example, some variables can be initialized, some are not.
Compiler and linker will collect them all into one section, but the first part of it is initialized and allocated in file,
while another is absent in file (of course, to make it smaller).
\IT{VirtualSize} will be equal to the size of section in memory, and \IT{SizeOfRawData} --- to
size of section in file.

IDA can show the border between initialized and not initialized parts like that:

\begin{lstlisting}[style=customasmx86]
...

.data:10017FFA                 db    0
.data:10017FFB                 db    0
.data:10017FFC                 db    0
.data:10017FFD                 db    0
.data:10017FFE                 db    0
.data:10017FFF                 db    0
.data:10018000                 db    ? ;
.data:10018001                 db    ? ;
.data:10018002                 db    ? ;
.data:10018003                 db    ? ;
.data:10018004                 db    ? ;
.data:10018005                 db    ? ;

...
\end{lstlisting}

\subsubsection{Relocations (relocs)}
\label{subsec:relocs}

\ac{AKA} FIXUP-s (at least in Hiew).

They are also present in almost all executable file formats
\footnote{Even in .exe files for MS-DOS}.
Exceptions are shared dynamic libraries compiled with \ac{PIC}, or any other \ac{PIC}-code.

What are they for?

Obviously, modules can be loaded on various base addresses, but how to deal with global variables, for example?
They must be accessed by address.  One solution is \PICcode{} (\myref{sec:PIC}).
But it is not always convenient.

That is why a relocations table is present.
There the addresses of points that must be corrected are enumerated,
in case of loading at a different base address.

% TODO тут бы пример с HIEW или objdump..
For example, there is a global variable at address \TT{0x410000} and this is how it is accessed:

\begin{lstlisting}[style=customasmx86]
A1 00 00 41 00         mov         eax,[000410000]
\end{lstlisting}

The base address of the module is \TT{0x400000}, the \ac{RVA} of the global variable is \TT{0x10000}.

If the module is loaded at base address \TT{0x500000}, the real address of the global variable must be \TT{0x510000}.

\myindex{x86!\Instructions!MOV}

As we can see, the address of variable is encoded in the instruction \TT{MOV}, after the byte \TT{0xA1}.

That is why the address of the 4 bytes after \TT{0xA1}, is written in the relocs table.

If the module is loaded at a different base address, the \ac{OS} loader enumerates all addresses in the table,

finds each 32-bit word the address points to, subtracts the original base address from it
(we get the \ac{RVA} here), and adds the new base address to it.

If a module is loaded at its original base address, nothing happens.

All global variables can be treated like that.

Relocs may have various types, however, in Windows for x86 processors, the type is usually \\
\IT{IMAGE\_REL\_BASED\_HIGHLOW}.

\myindex{Hiew}

By the way, relocs are darkened in Hiew, for example: \figref{fig:scanf_ex3_hiew_1}.

\myindex{\olly}
\olly underlines the places in memory to which relocs are to be applied, for example: \figref{fig:switch_lot_olly3}.

\subsubsection{Exports and imports}

\label{PE_exports_imports}
As we all know, any executable program must use the \ac{OS}'s services and other DLL-libraries somehow.

It can be said that functions from one module (usually DLL) must be connected somehow to the points of their
calls in other modules (.exe-file or another DLL).

For this, each DLL has an \q{exports} table, which consists of functions plus their addresses in a module.

And every .exe file or DLL has \q{imports}, a table of functions it needs for execution including
list of DLL filenames.

After loading the main .exe-file, the \ac{OS} loader processes imports table:
it loads the additional DLL-files, finds function names
among the DLL exports and writes their addresses down in the \ac{IAT} of the main .exe-module.

\myindex{Windows!Win32!Ordinal}

As we can see, during loading the loader must compare a lot of function names, but string comparison is not a very
fast procedure, so there is a support for \q{ordinals} or \q{hints},
which are function numbers stored in the table, instead of their names.

That is how they can be located faster when loading a DLL.
Ordinals are always present in the \q{export} table.

\myindex{MFC}
For example, a program using the \ac{MFC} library usually loads mfc*.dll by ordinals,
and in such programs there are no \ac{MFC} function names in \ac{INT}.

% TODO example!
When loading such programs in \IDA, it will ask for a path to the mfc*.dll files
in order to determine the function names.

If you don't tell \IDA the path to these DLLs, there will be \IT{mfc80\_123} instead of function names.

\myparagraph{Imports section}

Often a separate section is allocated for the imports table and everything related to it (with name like \TT{.idata}),
however, this is not a strict rule.

Imports are also a confusing subject because of the terminological mess. Let's try to collect all information in one place.

\begin{figure}[H]
\centering
\myincludegraphics{OS/PE/unnamed0.png}
\caption{
A scheme that unites all PE-file structures related to imports}
\end{figure}

The main structure is the array \IT{IMAGE\_IMPORT\_DESCRIPTOR}.
Each element for each DLL being imported.

Each element holds the \ac{RVA} address of the text string (DLL name) (\IT{Name}).

\IT{OriginalFirstThunk} is the \ac{RVA} address of the \ac{INT} table.
This is an array of \ac{RVA} addresses, each of which points to a text string with a function name.
Each string is prefixed by a 16-bit integer
(\q{hint})---\q{ordinal} of function.

While loading, if it is possible to find a function by ordinal,
then the strings comparison will not occur. The array is terminated by zero.

There is also a pointer to the \ac{IAT} table named \IT{FirstThunk}, it is just the \ac{RVA} address
of the place where the loader writes the addresses of the resolved functions.

The points where the loader writes addresses are marked by \IDA like this: \IT{\_\_imp\_CreateFileA}, etc.

There are at least two ways to use the addresses written by the loader.

\myindex{x86!\Instructions!CALL}
\begin{itemize}
\item The code will have instructions like \IT{call \_\_imp\_CreateFileA},
and since the field with the address of the imported function is a global variable in some sense,
the address of the \IT{call} instruction (plus 1 or 2) is to be added to the relocs table,
for the case when the module is loaded at a different base address.

But, obviously, this may enlarge relocs table significantly.

Because there are might be a lot of calls to imported functions in the module.

Furthermore, large relocs table slows down the process of loading modules.

\myindex{x86!\Instructions!JMP}
\myindex{thunk-functions}
\item For each imported function, there is only one jump allocated, using the \JMP instruction
plus a reloc to it.
Such points are also called \q{thunks}.

All calls to the imported functions are just \CALL instructions to the corresponding \q{thunk}.
In this case, additional relocs are not necessary because these \CALL{}-s
have relative addresses and do not need to be corrected.
\end{itemize}

These two methods can be combined.

Possible, the linker creates individual \q{thunk}s if there are too many calls to the function,
but not done by default. \\
\\
By the way, the array of function addresses to which FirstThunk is pointing is not necessary to be located in the \ac{IAT} section.
For example, the author of these lines once wrote the PE\_add\_import\footnote{\href{http://go.yurichev.com/17049}{yurichev.com}}
utility for adding imports to an existing .exe-file.

Some time earlier, in the previous versions of the utility,
at the place of the function you want to substitute with a call to another DLL,
my utility wrote the following code:

\begin{lstlisting}[style=customasmx86]
MOV EAX, [yourdll.dll!function]
JMP EAX
\end{lstlisting}

FirstThunk points to the first instruction. In other words, when loading yourdll.dll,
the loader writes the address of the \IT{function} function right in the code.

It also worth noting that a code section is usually write-protected, so my utility adds the \\
\IT{IMAGE\_SCN\_MEM\_WRITE}
flag for code section. Otherwise, the program to crash while loading with error code
5 (access denied). \\
\\
One might ask: what if I supply a program with a set of DLL files which is not supposed to change (including addresses of all DLL functions),
is it possible to speed up the loading process?

Yes, it is possible to write the addresses of the functions to be imported into the FirstThunk arrays in advance.
The \IT{Timestamp} field is present in the \\
\IT{IMAGE\_IMPORT\_DESCRIPTOR} structure.

If a value is present there, then the loader compares this value with the date-time of the DLL file.

If the values are equal, then the loader does not do anything, and the loading of the process can be faster.
This is called \q{old-style binding}
\footnote{\href{http://go.yurichev.com/17050}{MSDN}. There is also the \q{new-style binding}.}.
\myindex{BIND.EXE}

The BIND.EXE utility in Windows SDK is for for this.
For speeding up the loading of your program, Matt Pietrek in \PietrekPEURL, suggests to do the binding shortly after your program
installation on the computer of the end user. \\
\\
PE-files packers/encryptors may also compress/encrypt imports table.

In this case, the Windows loader, of course, will not load all necessary DLLs.
\myindex{Windows!Win32!LoadLibrary}
\myindex{Windows!Win32!GetProcAddress}

Therefore, the packer/encryptor does this on its own, with the help of
\IT{LoadLibrary()} and the \IT{GetProcAddress()} functions.

That is why these two functions are often present in \ac{IAT} in packed files.\\
\\
In the standard DLLs from the Windows installation, \ac{IAT} often is located right at the beginning of the PE file.
Supposedly, it is made so for optimization.

While loading, the .exe file is not loaded into memory as a whole (recall huge install programs which are
started suspiciously fast), it is \q{mapped}, and loaded into memory in parts as they are accessed.

Probably, Microsoft developers decided it will be faster.

\subsubsection{Resources}

\label{PEresources}

Resources in a PE file are just a set of icons, pictures, text strings, dialog descriptions.

Perhaps they were separated from the main code, so all these things could be multilingual,
and it would be simpler to pick text or picture for the language that is currently set in the \ac{OS}. \\
\\
As a side effect, they can be edited easily and saved back to the executable file, even if one does not have special knowledge,
by using the ResHack editor, for example (\myref{ResHack}).

\subsubsection{.NET}

\myindex{.NET}

.NET programs are not compiled into machine code but into a special bytecode.
\myindex{OEP}
Strictly speaking, there is bytecode instead of the usual x86 code
in the .exe file, however, the entry point (\ac{OEP}) points to this tiny fragment of x86 code:

\begin{lstlisting}[style=customasmx86]
jmp         mscoree.dll!_CorExeMain
\end{lstlisting}

The .NET loader is located in mscoree.dll, which processes the PE file.
\myindex{Windows!Windows XP}

It was so in all pre-Windows XP \ac{OS}es. Starting from XP, the \ac{OS} loader is able to detect the .NET file
and run it without executing that \JMP instruction
\footnote{\href{http://go.yurichev.com/17051}{MSDN}}.

\myindex{TLS}
\subsubsection{TLS}

This section holds initialized data for the \ac{TLS}(\myref{TLS}) (if needed).
When a new thread start, its \ac{TLS} data is initialized using the data from this section. \\
\\
\myindex{TLS!Callbacks}

Aside from that, the PE file specification also provides initialization of the
\ac{TLS} section, the so-called TLS callbacks.

If they are present, they are to be called before the control is passed to the main entry point (\ac{OEP}).

This is used widely in the PE file packers/encryptors.

\subsubsection{Tools}

\myindex{objdump}
\myindex{Cygwin}
\myindex{Hiew}
\label{ResHack}

\begin{itemize}
\item objdump (present in cygwin) for dumping all PE-file structures.

\item Hiew(\myref{Hiew}) as editor.

\item pefile---Python-library for PE-file processing \footnote{\url{http://go.yurichev.com/17052}}.

\item ResHack \acs{AKA} Resource Hacker---resources editor\footnote{\url{http://go.yurichev.com/17052}}.

\item PE\_add\_import\footnote{\url{http://go.yurichev.com/17049}}---
simple tool for adding symbol(s) to PE executable import table.

\item PE\_patcher\footnote{\href{http://go.yurichev.com/17054}{yurichev.com}}---simple tool for patching PE executables.

\item PE\_search\_str\_refs\footnote{\href{http://go.yurichev.com/17055}{yurichev.com}}---simple tool for searching for a function in PE executables which use some text string.
\end{itemize}

\subsubsection{Further reading}

% FIXME: bibliography per chapter or section
\begin{itemize}
\item Daniel Pistelli---The .NET File Format \footnote{\url{http://go.yurichev.com/17056}}
\end{itemize}

}
\RU{\subsection{Win32 PE}
\label{win32_pe}
\myindex{Windows!Win32}

\acs{PE} это формат исполняемых файлов, принятый в Windows.

Разница между .exe, .dll, и .sys в том, что у .exe и .sys обычно нет экспортов, только импорты.

\myindex{OEP}
У \ac{DLL}, как и у всех PE-файлов, есть точка входа (\ac{OEP})
(там располагается функция DllMain()), но обычно эта функция ничего не делает.

.sys это обычно драйвера устройств.

Для драйверов, Windows требует, чтобы контрольная сумма в PE-файле была проставлена и была верной\footnote{Например, Hiew(\myref{Hiew}) умеет её подсчитывать}.

\myindex{Windows!Windows Vista}
А начиная с Windows Vista, файлы драйверов должны быть также подписаны при помощи электронной подписи, иначе они не будут загружаться.


\myindex{MS-DOS}
В начале всякого PE-файла есть крохотная DOS-программа,
выводящая на консоль сообщение вроде \q{This program cannot be run in DOS mode.} --- если запустить эту программу в DOS либо Windows 3.1 (\ac{OS} не знающие о PE-формате), 
выведется это сообщение.

\subsubsection{Терминология}

\begin{itemize}
\item
Модуль --- это отдельный файл, .exe или .dll.

\item Процесс --- это некая загруженная в память и работающая программа.
Как правило, состоит из одного .exe-файла и массы .dll-файлов.

\item Память процесса --- память с которой работает процесс.
У каждого процесса --- своя.
Там обычно имеются загруженные модули, память стека, \glslink{heap}{кучи}, итд.

\item
\myindex{VA}
\ac{VA} --- это адрес, который будет использоваться в самой программе во время исполнения.

\item
\myindex{Базовый адрес}
Базовый адрес (модуля) --- это адрес, по которому модуль должен быть загружен в пространство процесса.

Загрузчик \ac{OS} может его изменить, если этот базовый адрес уже занят другим модулем, загруженным перед ним.

\item
\myindex{RVA}
\ac{RVA} --- это \ac{VA}-адрес минус базовый адрес. Многие адреса в таблицах PE-файла используют \ac{RVA}-адреса.

%\item
%Data directory\EMDASH{}...

\item 
\myindex{Windows!IAT}
\ac{IAT} --- массив адресов импортированных символов \footnote{\PietrekPE}. \\
Иногда, директория \TT{IMAGE\_DIRECTORY\_ENTRY\_IAT} указывает на \ac{IAT}. 
\label{IDA_idata} Важно отметить, что \ac{IDA} (по крайней мере 6.1) может выделить псевдо-секцию с именем \TT{.idata} для \ac{IAT}, даже если \ac{IAT} является частью совсем другой секции!

\item 
\myindex{Windows!INT}
\ac{INT} --- массив имен символов для импортирования \footnote{\PietrekPE}.
\end{itemize}

\subsubsection{Базовый адрес}

Дело в том, что несколько авторов модулей могут готовить DLL-файлы для других, и нет возможности договориться о том, какие адреса и кому будут отведены.

Поэтому, если у двух необходимых для загрузки процесса DLL одинаковые базовые адреса,
одна из них будет загружена по этому базовому адресу, 
а вторая --- по другому свободному месту в памяти процесса, и все виртуальные адреса во второй DLL будут скорректированы.

\par Очень часто линкер в \ac{MSVC} генерирует .exe-файлы с базовым адресом  \TT{0x400000}\footnote{Причина выбора такого адреса описана здесь: \href{http://go.yurichev.com/17041}{MSDN}},
и с секцией кода начинающейся с \TT{0x401000}.
Это значит, что \ac{RVA} начала секции кода --- \TT{0x1000}.
А \ac{DLL} часто генерируются MSVC-линкером с базовым адресом \TT{0x10000000}\footnote{Это можно изменять опцией /BASE в линкере}.

\myindex{ASLR}
Помимо всего прочего, есть еще одна причина намеренно загружать модули по разным адресам, а точнее, по случайным.

Это \ac{ASLR}\footnote{\href{http://go.yurichev.com/17042}{wikipedia}}.

\myindex{Shellcode}
Дело в том, что некий шеллкод, пытающийся исполниться на зараженной системе, должен вызывать какие-то системные функции, а следовательно, знать их адреса.

И в старых \ac{OS} (в линейке \gls{Windows NT}: до Windows Vista),
системные DLL (такие как kernel32.dll, user32.dll) загружались все время
по одним и тем же адресам, 
а если еще и вспомнить, что версии этих DLL редко менялись, то адреса отдельных
функций, можно сказать, фиксированы и шеллкод может вызывать их напрямую.

Чтобы избежать этого, методика \ac{ASLR}
загружает и вашу программу, и все модули ей необходимые, по случайным адресам, разным при каждом запуске.

В PE-файлах, поддержка \ac{ASLR} отмечается выставлением флага \\
\TT{IMAGE\_DLL\_CHARACTERISTICS\_DYNAMIC\_BASE} \InSqBrackets{см: \Russinovich}.

\subsubsection{Subsystem}

Имеется также поле \IT{subsystem}, обычно это:

\myindex{Native API}
\begin{itemize}
\item native\footnote{Что означает, что модуль использует Native API а не Win32} (.sys-драйвер), 

\item console (консольное приложение) или

\item \ac{GUI} (не консольное).
\end{itemize}

\subsubsection{Версия ОС}

PE-файле также задает минимальный номер версии Windows, необходимый для загрузки модуля.

Соответствие номеров версий в файле и кодовых наименований Windows, можно посмотреть
здесь\footnote{\href{http://go.yurichev.com/17044}{wikipedia}}.

\myindex{Windows!Windows NT4}
\myindex{Windows!Windows 2000}
Например, \ac{MSVC} 2005 еще компилирует .exe-файлы запускающиеся на Windows NT4 (версия 4.00), а вот \ac{MSVC} 2008 уже нет 
(генерируемые файлы имеют версию 5.00, для запуска необходима как минимум Windows 2000).

\myindex{Windows!Windows XP}
\ac{MSVC} 2012 по умолчанию генерирует .exe-файлы версии 6.00, для запуска нужна как минимум Windows Vista. 
Хотя, изменив настройки компиляции\footnote{\href{http://go.yurichev.com/17045}{MSDN}},
можно заставить генерировать и под Windows XP.

\subsubsection{Секции}

Разделение на секции присутствует, по-видимому, во всех форматах исполняемых файлов.

Придумано это для того, чтобы отделить код от данных, а данные --- от константных данных.


\begin{itemize}
\item На секции кода будет стоять флаг \IT{IMAGE\_SCN\_CNT\_CODE} или \IT{IMAGE\_SCN\_MEM\_EXECUTE} --- это исполняемый код.

\item На секции данных --- флаги \IT{IMAGE\_SCN\_CNT\_INITIALIZED\_DATA}, \IT{IMAGE\_SCN\_MEM\_READ} и \\
\IT{IMAGE\_SCN\_MEM\_WRITE}.

\item На пустой секции с неинициализированными данными --- \IT{IMAGE\_SCN\_CNT\_UNINITIALIZED\_DATA}, \IT{IMAGE\_SCN\_MEM\_READ} и \IT{IMAGE\_SCN\_MEM\_WRITE}.

\item А на секции с константными данными, то есть, защищенными от записи, могут быть флаги \\
\IT{IMAGE\_SCN\_CNT\_INITIALIZED\_DATA} и \IT{IMAGE\_SCN\_MEM\_READ} без \IT{IMAGE\_SCN\_MEM\_WRITE}. 
Если попытаться записать что-то в эту секцию, процесс упадет.
\end{itemize}

\myindex{TLS}
\myindex{BSS}
В PE-файле можно задавать название для секции, но это не важно.
Часто (но не всегда) секция кода называется \TT{.text}, секция данных --- \TT{.data}, константных данных --- \TT{.rdata} \IT{(readable data)}.
Еще популярные имена секций: 

\myindex{MIPS}
\begin{itemize}
\item \TT{.idata} --- секция импортов. \ac{IDA} может создавать псевдо-секцию с этим же именем: \myref{IDA_idata}.
\item \TT{.edata} --- секция экспортов (редко встречается)
\item \TT{.pdata} --- секция содержащая информацию об исключениях в Windows NT для MIPS, \ac{IA64} и x64: \myref{SEH_win64}
\item \TT{.reloc} --- секция релоков
\item \TT{.bss} --- неинициализированные данные
\item \TT{.tls} --- thread local storage (\ac{TLS})
\item \TT{.rsrc} --- ресурсы
\item \TT{.CRT} --- может присутствует в бинарных файлах, скомпилированных очень старыми версиями MSVC
\end{itemize}

Запаковщики/зашифровщики PE-файлов часто затирают имена секций, или меняют на свои.

В \ac{MSVC} можно объявлять данные в произвольно названной секции\footnote{\href{http://go.yurichev.com/17047}{MSDN}}.

Некоторые компиляторы и линкеры могут добавлять также секцию с отладочными символами 
и вообще отладочной информацией (например, MinGW).

\myindex{Windows!PDB}
Хотя это не так в современных версиях \ac{MSVC} (там принято отладочную информацию сохранять в отдельных \gls{PDB}-файлах).\\
\\
Вот как PE-секция описывается в файле:

\begin{lstlisting}[style=customc]
typedef struct _IMAGE_SECTION_HEADER {
  BYTE  Name[IMAGE_SIZEOF_SHORT_NAME];
  union {
    DWORD PhysicalAddress;
    DWORD VirtualSize;
  } Misc;
  DWORD VirtualAddress;
  DWORD SizeOfRawData;
  DWORD PointerToRawData;
  DWORD PointerToRelocations;
  DWORD PointerToLinenumbers;
  WORD  NumberOfRelocations;
  WORD  NumberOfLinenumbers;
  DWORD Characteristics;
} IMAGE_SECTION_HEADER, *PIMAGE_SECTION_HEADER;
\end{lstlisting}
\footnote{\href{http://go.yurichev.com/17048}{MSDN}}

\myindex{Hiew}
Еще немного терминологии: \IT{PointerToRawData} называется \q{Offset} в Hiew и \IT{VirtualAddress} называется \q{RVA} там же.

\subsubsection{Секция данных}

Секция данных в файле может быть меньше, чем в памяти.
Например, некоторые переменные могут быть инициализированы, а некоторые --- нет.
Тогда компилятор и линкер объединяют их все в одну секцию, но первая часть секции инициализирована и находится в файле,
а вторая отсутствует в файле (конечно, для экономии места).
\IT{VirtualSize} будет равен размеру секции в памяти, а \IT{SizeOfRawData} будет равен размеру секции в файле.

IDA будет показывать границу между инициализированной и неинициализированной частями так:

\begin{lstlisting}[style=customasmx86]
...

.data:10017FFA                 db    0
.data:10017FFB                 db    0
.data:10017FFC                 db    0
.data:10017FFD                 db    0
.data:10017FFE                 db    0
.data:10017FFF                 db    0
.data:10018000                 db    ? ;
.data:10018001                 db    ? ;
.data:10018002                 db    ? ;
.data:10018003                 db    ? ;
.data:10018004                 db    ? ;
.data:10018005                 db    ? ;

...
\end{lstlisting}

\subsubsection{Релоки}
\label{subsec:relocs}

Также известны как FIXUP-ы (по крайней мере в Hiew).

Это также присутствует почти во всех форматах загружаемых и исполняемых файлов
\footnote{Даже .exe-файлы в MS-DOS}.

Исключения это динамические библиотеки явно скомпилированные с \ac{PIC} или любой другой \ac{PIC}-код.

Зачем они нужны?
Как видно, модули могут загружаться по другим базовым адресам,
но как же тогда работать с глобальными переменными, например?

Ведь нужно обращаться к ним по адресу.
Одно из решений --- это \PICcode{} (\myref{sec:PIC}).  Но это далеко не всегда удобно.

Поэтому имеется таблица релоков. 
Там просто перечислены адреса мест в модуле подлежащими коррекции при загрузке по другому базовому адресу.

% TODO тут бы пример с HIEW или objdump..
Например, по \TT{0x410000} лежит некая глобальная переменная, и вот как обеспечивается её чтение:

\begin{lstlisting}[style=customasmx86]
A1 00 00 41 00         mov         eax,[000410000]
\end{lstlisting}

Базовый адрес модуля \TT{0x400000}, а \ac{RVA} глобальной переменной \TT{0x10000}.

Если загружать модуль по базовому адресу \TT{0x500000}, нужно чтобы адрес этой переменной в этой инструкции стал \TT{0x510000}.

\myindex{x86!\Instructions!MOV}
Как видно, адрес переменной закодирован в самой инструкции \TT{MOV}, после байта \TT{0xA1}.

Поэтому адрес четырех байт, после \TT{0xA1}, записывается в таблицу релоков.

Если модуль загружается по другому базовому адресу,
загрузчик \ac{OS} обходит все адреса в таблице, 
находит каждое 32-битное слово по этому адресу,
отнимает от него настоящий, оригинальный базовый адрес
(в итоге получается \ac{RVA}), и прибавляет к нему новый базовый адрес.

А если модуль загружается по своему оригинальному базовому адресу, ничего не происходит.

Так можно обходиться со всеми глобальными переменными.

Релоки могут быть разных типов, однако в Windows для x86-процессоров, тип обычно \\
\IT{IMAGE\_REL\_BASED\_HIGHLOW}.

\myindex{Hiew}
Кстати, релоки маркируются темным в Hiew, например: \figref{fig:scanf_ex3_hiew_1}.

\myindex{\olly}
\olly подчеркивает места в памяти, к которым будут применены релоки, например: \figref{fig:switch_lot_olly3}.

\subsubsection{Экспорты и импорты}

\label{PE_exports_imports}
Как известно, любая исполняемая программа должна как-то пользоваться сервисами \ac{OS} и прочими DLL-библиотеками.

Можно сказать, что нужно связывать функции из одного модуля (обычно DLL) и места их вызовов в другом модуле (.exe-файл или другая DLL).

Для этого, у каждой DLL есть \q{экспорты}, это таблица функций плюс их адреса в модуле.

А у .exe-файла, либо DLL, есть \q{импорты}, это таблица функций требующихся для исполнения
включая список имен DLL-файлов.

Загрузчик \ac{OS}, после загрузки основного .exe-файла, проходит по таблице импортов:
загружает дополнительные DLL-файлы, 
находит имена функций среди экспортов в DLL и прописывает их адреса в \ac{IAT} в головном .exe-модуле.

\myindex{Windows!Win32!Ordinal}
Как видно, во время загрузки, загрузчику нужно много сравнивать одни имена функций с другими,
а сравнение строк --- это не очень быстрая процедура, так что,
имеется также поддержка \q{ординалов} или
\q{hint}-ов, это когда в таблице импортов проставлены номера функций вместо их имен.

Так их быстрее находить в загружаемой DLL.
В таблице экспортов ординалы присутствуют всегда.

\myindex{MFC}
К примеру, программы использующие библиотеки \ac{MFC}, обычно загружают mfc*.dll по ординалам, и в таких программах, в \ac{INT}, нет имен функций \ac{MFC}.

% TODO example!
При загрузке такой программы в \IDA, она спросит у вас путь к файлу mfc*.dll,
чтобы установить имена функций.
Если в \IDA не указать путь к этой DLL, то вместо имен функций будет что-то вроде \IT{mfc80\_123}.

\myparagraph{Секция импортов}

Под таблицу импортов и всё что с ней связано иногда отводится отдельная секция 
(с названием вроде \TT{.idata}),
но это не обязательно.

Импорты --- это запутанная тема еще и из-за терминологической путаницы. Попробуем собрать всё в одно место.

\begin{figure}[H]
\centering
\myincludegraphics{OS/PE/unnamed0.png}
\caption{схема, объединяющая все структуры в PE-файлы, связанные с импортами}
\end{figure}

Самая главная структура --- это массив \IT{IMAGE\_IMPORT\_DESCRIPTOR}.
Каждый элемент на каждую импортируемую DLL.

У каждого элемента есть \ac{RVA}-адрес текстовой строки (имя DLL) (\IT{Name}).

\IT{OriginalFirstThunk} это \ac{RVA} -адрес таблицы \ac{INT}.
Это массив \ac{RVA}-адресов, каждый из которых указывает на текстовую строку где записано имя функции. 
Каждую строку предваряет 16-битное число (\q{hint}) --- \q{ординал} функции.

Если при загрузке удается найти функцию по ординалу, тогда сравнение текстовых строк не будет происходить.
Массив оканчивается нулем.

Есть также указатель на таблицу \ac{IAT} с названием \IT{FirstThunk}, это просто \ac{RVA}-адрес места, где загрузчик будет проставлять адреса найденных функций.

Места где загрузчик проставляет адреса, \IDA именует их так: \IT{\_\_imp\_CreateFileA}, etc.

Есть по крайней мере два способа использовать адреса, проставленные загрузчиком.

\myindex{x86!\Instructions!CALL}
\begin{itemize}
\item
В коде будут просто инструкции вроде \IT{call \_\_imp\_CreateFileA}, а так как, поле с адресом импортируемой функции это как бы глобальная переменная, 
то в таблице релоков добавляется адрес (плюс 1 или 2) в инструкции \IT{call},
на случай если модуль будет загружен по другому базовому адресу.

Но как видно, это приводит к увеличению таблицы релоков.
Ведь вызовов импортируемой функции у вас в модуле может быть очень много.
К тому же, чем больше таблица релоков, тем дольше загрузка.

\myindex{x86!\Instructions!JMP}
\myindex{thunk-функции}
\item
На каждую импортируемую функцию выделяется только один переход на импортируемую функцию используя
инструкцию \JMP плюс релок на эту инструкцию.
Такие места-\q{переходники} называются также \q{thunk}-ами.
А все вызовы импортируемой функции это просто инструкция \CALL на соответствующий \q{thunk}.
В данном случае, дополнительные релоки не нужны, потому что эти CALL-ы имеют относительный адрес,
и корректировать их не надо.
\end{itemize}

Оба этих два метода могут комбинироваться.
Надо полагать, линкер создает отдельный \q{thunk}, если вызовов слишком много, но по умолчанию --- не создает. \\
\\
Кстати, массив адресов функций, на который указывает FirstThunk,
не обязательно может быть в секции \ac{IAT}.
К примеру, автор сих строк написал утилиту
PE\_add\_import\footnote{\href{http://go.yurichev.com/17049}{yurichev.com}} 
для добавления импорта в уже существующий .exe-файл.
Раньше, в прошлых версиях утилиты, на месте функции, вместо которой вы хотите подставить вызов в другую DLL,
моя утилита вписывала такой код:

\begin{lstlisting}[style=customasmx86]
MOV EAX, [yourdll.dll!function]
JMP EAX
\end{lstlisting}

При этом, FirstThunk указывает прямо на первую инструкцию.
Иными словами, загрузчик, загружая yourdll.dll, прописывает адрес функции \IT{function} прямо в коде.

Надо также отметить что обычно секция кода защищена от записи, так что, моя утилита добавляет флаг \\
\IT{IMAGE\_SCN\_MEM\_WRITE} 
для секции кода. Иначе при загрузке такой программы, она упадет с ошибкой 5 (access denied). \\
\\
Может возникнуть вопрос: а что если я поставляю программу с набором DLL,
которые никогда не будут меняться (в т.ч., адреса всех функций в этих DLL), может как-то можно ускорить процесс загрузки?

Да, можно прописать адреса импортируемых функций в массивы FirstThunk заранее.
Для этого в структуре \\
\IT{IMAGE\_IMPORT\_DESCRIPTOR} имеется поле \IT{Timestamp}.
И если там присутствует какое-то значение, то загрузчик сверяет это значение с датой-временем DLL-файла.
И если они равны, то загрузчик больше ничего не делает, и загрузка может происходить быстрее.

Это называется \q{old-style binding}
\footnote{\href{http://go.yurichev.com/17050}{MSDN}.
Существует также \q{new-style binding}.}.
\myindex{BIND.EXE}
В Windows SDK для этого имеется утилита BIND.EXE.
Для ускорения загрузки вашей программы, 
Matt Pietrek в \PietrekPEURL, предлагает делать binding сразу после инсталляции
вашей программы на компьютере конечного пользователя. \\
\\
Запаковщики/зашифровщики PE-файлов могут также сжимать/шифровать таблицу импортов.
В этом случае, загрузчик Windows, конечно же, не загрузит все нужные DLL.
\myindex{Windows!Win32!LoadLibrary}
\myindex{Windows!Win32!GetProcAddress}
Поэтому распаковщик/расшифровщик делает это сам, при помощи вызовов \IT{LoadLibrary()} и \IT{GetProcAddress()}.
Вот почему в запакованных файлах эти две функции часто присутствуют в \ac{IAT}. \\
\\
В стандартных DLL входящих в состав Windows, часто, \ac{IAT} находится в самом начале PE-файла.

Возможно это для оптимизации.
Ведь .exe-файл при загрузке не загружается в память весь 
(вспомните что инсталляторы огромного размера подозрительно быстро запускаются), он \q{мапится} (map), 
и подгружается в память частями по мере обращения к этой памяти.

И возможно в Microsoft решили, что так будет быстрее.

\subsubsection{Ресурсы}

\label{PEresources}
Ресурсы в PE-файле --- это набор иконок, картинок, текстовых строк, описаний диалогов.
Возможно, их в свое время решили отделить от основного кода, чтобы все эти вещи были многоязычными,
и было проще выбирать текст или картинку того языка, который установлен в \ac{OS}. \\
\\
В качестве побочного эффекта, их легко редактировать и сохранять обратно в исполняемый файл,
даже не обладая специальными знаниями, например, редактором ResHack (\myref{ResHack}).

\subsubsection{.NET}

\myindex{.NET}
Программы на .NET компилируются не в машинный код, а в свой собственный байткод.
\myindex{OEP}
Собственно, в .exe-файлы байткод вместо обычного кода, однако, точка входа (\ac{OEP}) 
указывает на крохотный фрагмент x86-кода:

\begin{lstlisting}[style=customasmx86]
jmp         mscoree.dll!_CorExeMain
\end{lstlisting}

А в mscoree.dll и находится .NET-загрузчик, который уже сам будет работать с PE-файлом.

\myindex{Windows!Windows XP}
Так было в \ac{OS} до Windows XP. Начиная с XP, загрузчик \ac{OS} уже сам определяет, что это
.NET-файл и запускает его не исполняя этой инструкции \JMP
\footnote{\href{http://go.yurichev.com/17051}{MSDN}}.

\myindex{TLS}
\subsubsection{TLS}

Эта секция содержит в себе инициализированные данные для \ac{TLS}(\myref{TLS}) (если нужно).
При старте нового треда, его \ac{TLS}-данные инициализируются данными из этой секции. \\
\\
\myindex{TLS!Callbacks}
Помимо всего прочего, спецификация PE-файла предусматривает инициализацию \ac{TLS}-секции, т.н., TLS callbacks.
Если они присутствуют, то они будут вызваны перед тем как передать управление на главную точку входа (\ac{OEP}).
Это широко используется запаковщиками/зашифровщиками PE-файлов.

\subsubsection{Инструменты}

\begin{itemize}
\item
\myindex{objdump}
\myindex{Cygwin}
objdump (имеется в cygwin) для вывода всех структур PE-файла.

\item
\myindex{Hiew}
Hiew(\myref{Hiew}) как редактор.

\item pefile --- Python-библиотека для работы с PE-файлами
\footnote{\url{http://go.yurichev.com/17052}}.

\item
\label{ResHack}
ResHack \acs{AKA} Resource Hacker --- редактор ресурсов
\footnote{\url{http://go.yurichev.com/17052}}.

\item PE\_add\_import\footnote{\url{http://go.yurichev.com/17049}} --- простая утилита для добавления символа/-ов в таблицу импортов PE-файла.

\item PE\_patcher\footnote{\href{http://go.yurichev.com/17054}{yurichev.com}} --- простая утилита для модификации PE-файлов.

\item PE\_search\_str\_refs\footnote{\href{http://go.yurichev.com/17055}{yurichev.com}} --- простая утилита для поиска функции в PE-файле, где используется некая текстовая строка.
\end{itemize}

\subsubsection{Further reading}

% FIXME: bibliography per chapter or section
\begin{itemize}
\item Daniel Pistelli --- The .NET File Format \footnote{\url{http://go.yurichev.com/17056}}
\end{itemize}

}


\subsection{Windows SEH}
\label{sec:SEH}
\myindex{Windows!Structured Exception Handling}

\newcommand{\HandlerFunction}{
\RU{функция-обработчик}
\EN{handler function}
\DE{Handler-Funktionen}
\FR{Gestionnaire d'exception}
}
\newcommand{\MoreEntries}{
\RU{остальные элементы}
\EN{more entries}
\DE{weitere Einträge}
\FR{en savoir plus}
}
\newcommand{\FilterFunction}{
\RU{функция-фильтр}
\EN{filter function}
\DE{Filter-Funktionen}
\FR{fonction de filtrage}
}
\newcommand{\HandlerFinallyFunction}{
\RU{функция-обработчик/finally-обработчик}
\EN{handler/finally function}
\DE{Handler/finale Funktion}
\FR{handler/finaliseur}
}
\newcommand{\ScopeTable}{
\RU{таблица scope}
\EN{scope table}
\DE{Geltungsbereich-Tabelle}
\FR{scope table}
}

% subsections
\EN{\subsubsection{Let's forget about MSVC}

In Windows, the \ac{SEH} is intended for exceptions handling, nevertheless, it is language-agnostic,
not related to \Cpp or \ac{OOP} in any way.

Here we are going to take a look at \ac{SEH} in its isolated (from C++ and MSVC extensions) form.

\myindex{Windows!TIB}
\myindex{Windows!Win32!RaiseException()}

Each running process has a chain of \ac{SEH} handlers, \ac{TIB} has the address of the last handler.

When an exception occurs (division by zero, incorrect address access, user exception triggered by
calling the \TT{RaiseException()} function), the \ac{OS} finds the last handler in the \ac{TIB} and calls it,
passing all information about the \ac{CPU} state (register values, etc.) at the moment of the exception.

The exception handler considering the exception, does it see something familiar?
If so, it handles the exception.

If not, it signals to the \ac{OS} that it
cannot handle it and the \ac{OS} calls the next handler in the chain,
until a handler which is able to handle the exception is be found.

At the very end of the chain there a standard handler that shows the well-known dialog box, informing the user about a
process crash, some technical information about the \ac{CPU} state at the time of the crash,
and offering to collect all information and send it to developers in Microsoft. 

\begin{figure}[H]
\centering
\includegraphics[width=0.6\textwidth]{OS/SEH/1/crash_xp1.png}
\caption{Windows XP}
\end{figure}

\begin{figure}[H]
\centering
\includegraphics[width=0.6\textwidth]{OS/SEH/1/crash_xp2.png}
\caption{Windows XP}
\end{figure}

\begin{figure}[H]
\centering
\includegraphics[width=0.6\textwidth]{OS/SEH/1/crash_win7.png}
\caption{Windows 7}
\end{figure}

\begin{figure}[H]
\centering
\includegraphics[width=0.6\textwidth]{OS/SEH/1/crash_win81.png}
\caption{Windows 8.1}
\end{figure}

Earlier, this handler was called Dr. Watson
\footnote{\href{http://go.yurichev.com/17046}{wikipedia}}.

By the way, some developers make their own handler that sends information about the program crash to themselves.
\myindex{Windows!Win32!SetUnhandledExceptionFilter()}
It is registered with the help of \TT{SetUnhandledExceptionFilter()} 
and to be called if the \ac{OS} does not have any other way to handle the exception.
\myindex{\oracle}
An example is \oracle---it saves huge dumps reporting all possible information about the \ac{CPU} and memory state.

Let's write our own primitive exception handler.
This example is based on the example from \PietrekSEH.
It must be compiled with the SAFESEH option: \TT{cl seh1.cpp /link /safeseh:no}.
More about SAFESEH here: \href{http://go.yurichev.com/17252}{MSDN}.

\lstinputlisting[style=customc]{OS/SEH/1/1.cpp}

The FS: segment register is pointing to the \ac{TIB} in win32.

The very first element in the \ac{TIB} is a pointer to the last handler in the chain.
We save it in the stack and store the address of our handler there.
The structure is named \TT{\_EXCEPTION\_REGISTRATION}, it is a simple singly-linked list and its elements are stored right in the stack.

\begin{lstlisting}[caption=MSVC/VC/crt/src/exsup.inc,style=customasmx86]
\_EXCEPTION\_REGISTRATION struc
     prev    dd      ?
     handler dd      ?
\_EXCEPTION\_REGISTRATION ends
\end{lstlisting}

So each \q{handler} field points to a handler and an each \q{prev} field points to the previous record in the stack.
The last record has \TT{0xFFFFFFFF} (-1) in the \q{prev} field.

\input{OS/SEH/1/tikz}

After our handler is installed, we call \TT{RaiseException()}
\footnote{\href{http://go.yurichev.com/17253}{MSDN}}.
This is an user exception. 
The handler checks the code.
If the code is \TT{0xE1223344}, it returning \TT{ExceptionContinueExecution},
which means that handler corrected the CPU state (it is usually a correction of the EIP/ESP registers) and the \ac{OS} can resume the execution of the.
If you alter slightly the code so the handler returns \TT{ExceptionContinueSearch},

then the \ac{OS} will call the other handlers, and it's unlikely that one who can handle it will be found, since
no one will have any information about it (rather about its code).
You will see the standard Windows dialog about a process crash.

What is the difference between a system exceptions and a user one? Here are the system ones:

\small
\begin{center}
\begin{tabular}{ | l | l | l | }
\hline
\HeaderColor as defined in WinBase.h & 
\HeaderColor as defined in ntstatus.h & 
\HeaderColor value \\
\hline
EXCEPTION\_ACCESS\_VIOLATION          & STATUS\_ACCESS\_VIOLATION           & 0xC0000005 \\
\hline
EXCEPTION\_DATATYPE\_MISALIGNMENT     & STATUS\_DATATYPE\_MISALIGNMENT      & 0x80000002 \\
\hline
EXCEPTION\_BREAKPOINT                & STATUS\_BREAKPOINT                 & 0x80000003 \\
\hline
EXCEPTION\_SINGLE\_STEP               & STATUS\_SINGLE\_STEP                & 0x80000004 \\
\hline
EXCEPTION\_ARRAY\_BOUNDS\_EXCEEDED     & STATUS\_ARRAY\_BOUNDS\_EXCEEDED      & 0xC000008C \\
\hline
EXCEPTION\_FLT\_DENORMAL\_OPERAND      & STATUS\_FLOAT\_DENORMAL\_OPERAND     & 0xC000008D \\
\hline
EXCEPTION\_FLT\_DIVIDE\_BY\_ZERO        & STATUS\_FLOAT\_DIVIDE\_BY\_ZERO       & 0xC000008E \\
\hline
EXCEPTION\_FLT\_INEXACT\_RESULT        & STATUS\_FLOAT\_INEXACT\_RESULT       & 0xC000008F \\
\hline
EXCEPTION\_FLT\_INVALID\_OPERATION     & STATUS\_FLOAT\_INVALID\_OPERATION    & 0xC0000090 \\
\hline
EXCEPTION\_FLT\_OVERFLOW              & STATUS\_FLOAT\_OVERFLOW             & 0xC0000091 \\
\hline
EXCEPTION\_FLT\_STACK\_CHECK           & STATUS\_FLOAT\_STACK\_CHECK          & 0xC0000092 \\
\hline
EXCEPTION\_FLT\_UNDERFLOW             & STATUS\_FLOAT\_UNDERFLOW            & 0xC0000093 \\
\hline
EXCEPTION\_INT\_DIVIDE\_BY\_ZERO        & STATUS\_INTEGER\_DIVIDE\_BY\_ZERO     & 0xC0000094 \\
\hline
EXCEPTION\_INT\_OVERFLOW              & STATUS\_INTEGER\_OVERFLOW           & 0xC0000095 \\
\hline
EXCEPTION\_PRIV\_INSTRUCTION          & STATUS\_PRIVILEGED\_INSTRUCTION     & 0xC0000096 \\
\hline
EXCEPTION\_IN\_PAGE\_ERROR             & STATUS\_IN\_PAGE\_ERROR              & 0xC0000006 \\
\hline
EXCEPTION\_ILLEGAL\_INSTRUCTION       & STATUS\_ILLEGAL\_INSTRUCTION        & 0xC000001D \\
\hline
EXCEPTION\_NONCONTINUABLE\_EXCEPTION  & STATUS\_NONCONTINUABLE\_EXCEPTION   & 0xC0000025 \\
\hline
EXCEPTION\_STACK\_OVERFLOW            & STATUS\_STACK\_OVERFLOW             & 0xC00000FD \\
\hline
EXCEPTION\_INVALID\_DISPOSITION       & STATUS\_INVALID\_DISPOSITION        & 0xC0000026 \\
\hline
EXCEPTION\_GUARD\_PAGE                & STATUS\_GUARD\_PAGE\_VIOLATION       & 0x80000001 \\
\hline
EXCEPTION\_INVALID\_HANDLE            & STATUS\_INVALID\_HANDLE             & 0xC0000008 \\
\hline
EXCEPTION\_POSSIBLE\_DEADLOCK         & STATUS\_POSSIBLE\_DEADLOCK          & 0xC0000194 \\
\hline
CONTROL\_C\_EXIT                      & STATUS\_CONTROL\_C\_EXIT             & 0xC000013A \\
\hline
\end{tabular}
\end{center}
\normalsize

That is how the code is defined:

\begin{center}
\begin{bytefield}[bitwidth=0.03\linewidth]{32}
\bitheader[endianness=big]{31,29,28,27,16,15,0} \\
\bitbox{2}{S} & 
\bitbox{1}{U} &
\bitbox{1}{0} & 
\bitbox{12}{Facility code} &
\bitbox{16}{Error code}
\end{bytefield}
\end{center}

S is a basic status code: 
11---error;
10---warning;
01---informational;
00---success.
U---whether the code is user code.

That is why we chose 0xE1223344---E\textsubscript{16} (1110\textsubscript{2}) 0xE (1110b) 
means that it is 1) user exception; 2) error.

But to be honest, this example works fine without these high bits.

Then we try to read a value from memory at address 0.

Of course, there is nothing at this address in win32, so an exception is raised.

The very first handler is to be called---yours, and it will know about it first, by checking
the code if it's equal to the \TT{EXCEPTION\_ACCESS\_VIOLATION} constant.

The code that's reading from memory at address 0 is looks like this:

\lstinputlisting[style=customasmx86,caption=MSVC 2010]{OS/SEH/1/1_fragment.asm}

Will it be possible to fix this error \q{on the fly} and to continue with program execution?

Yes, our exception handler can fix the \EAX value and let the \ac{OS} execute this instruction once again.
So that is what we do. \printf prints 1234, because after the execution of our handler \EAX is not 0,
but contains the address of the global variable \TT{new\_value}.
The execution will resume.

That is what is going on: the memory manager in the \ac{CPU} signals about an error, the \ac{CPU} suspends the thread,
finds the exception handler in the Windows kernel, 
which, in turn, starts to call all handlers in the \ac{SEH} chain, one by one.

We use MSVC 2010 here, but of course, there is no any guarantee that \EAX will be used for this pointer.

This address replacement trick is showy, and we considering it here as an illustration of \ac{SEH}'s internals.
Nevertheless, it's hard to recall any case where it is used for \q{on-the-fly} error fixing.

Why SEH-related records are stored right in the stack instead of some other place?

Supposedly because the \ac{OS} is not needing to care about freeing this information, 
these records are simply disposed when the function finishes its execution.
\myindex{\CStandardLibrary!alloca()}
This is somewhat like alloca(): (\myref{alloca}).

}
\RU{\subsubsection{Забудем на время о MSVC}

\ac{SEH} в Windows предназначен для обработки исключений, тем не менее, с \Cpp и \ac{OOP} он никак не связан.
Здесь мы рассмотрим \ac{SEH} изолированно от Си++ и расширений MSVC.

\myindex{Windows!TIB}
\myindex{Windows!Win32!RaiseException()}
Каждый процесс имеет цепочку \ac{SEH}-обработчиков, и адрес последнего записан в \ac{TIB}.
Когда происходит исключение (деление на ноль, обращение по неверному адресу в памяти, 
пользовательское исключение, поднятое при помощи \TT{RaiseException()}),
\ac{OS} находит последний обработчик в \ac{TIB} и вызывает его, 
передав ему информацию о состоянии \ac{CPU} в момент исключения
(все значения регистров, итд.).
Обработчик выясняет, то ли это исключение, для которого он создавался?

Если да, то он обрабатывает исключение.
Если нет, то показывает \ac{OS} что он не может его обработать и \ac{OS} вызывает следующий обработчик
в цепочке, и так до тех пор, пока не найдется обработчик способный обработать исключение.

В самом конце цепочки находится стандартный обработчик, показывающий всем очень известное окно, 
сообщающее что процесс упал, 
сообщает также состояние \ac{CPU} в момент падения и позволяет собрать и отправить информацию обработчикам 
в Microsoft. 

\begin{figure}[H]
\centering
\includegraphics[width=0.6\textwidth]{OS/SEH/1/crash_xp1.png}
\caption{Windows XP}
\end{figure}

\begin{figure}[H]
\centering
\includegraphics[width=0.6\textwidth]{OS/SEH/1/crash_xp2.png}
\caption{Windows XP}
\end{figure}

\begin{figure}[H]
\centering
\includegraphics[width=0.6\textwidth]{OS/SEH/1/crash_win7.png}
\caption{Windows 7}
\end{figure}

\begin{figure}[H]
\centering
\includegraphics[width=0.6\textwidth]{OS/SEH/1/crash_win81.png}
\caption{Windows 8.1}
\end{figure}

Раньше этот обработчик назывался Dr. Watson
\footnote{\href{http://go.yurichev.com/17046}{wikipedia}}.

Кстати, некоторые разработчики делают свой собственный обработчик,
отправляющий информацию о падении программы им самим.\\
\myindex{Windows!Win32!SetUnhandledExceptionFilter()}
Он регистрируется при помощи функции \TT{SetUnhandledExceptionFilter()} 
и будет вызван если \ac{OS} не знает, как иначе обработать исключение.

\myindex{\oracle}
А, например, \oracle в этом случае генерирует огромные дампы, 
содержащие всю возможную информацию и состоянии \ac{CPU} и памяти.

Попробуем написать свой примитивный обработчик исключений.
Этот пример основан на примере из \PietrekSEH.
Он должен компилироваться с опцией SAFESEH: \TT{cl seh1.cpp /link /safeseh:no}.
Подробнее об опции SAFESEH здесь: \href{http://go.yurichev.com/17252}{MSDN}.
	
\lstinputlisting[style=customc]{OS/SEH/1/1.cpp}

Сегментный регистр FS: в win32 указывает на \ac{TIB}.
Самый первый элемент \ac{TIB} это указатель на последний обработчик в цепочке.
Мы сохраняем его в стеке и записываем туда адрес своего обработчика.
Эта структура называется \TT{\_EXCEPTION\_REGISTRATION}, 
это простейший односвязный список, и эти элементы хранятся прямо в стеке.

\begin{lstlisting}[caption=MSVC/VC/crt/src/exsup.inc,style=customasmx86]
\_EXCEPTION\_REGISTRATION struc
     prev    dd      ?
     handler dd      ?
\_EXCEPTION\_REGISTRATION ends
\end{lstlisting}

Так что каждое поле \q{handler} указывает на обработчик,
а каждое поле \q{prev} указывает на предыдущую структуру в стеке.
Самая последняя структура имеет \TT{0xFFFFFFFF} (-1) в поле \q{prev}.

\input{OS/SEH/1/tikz}

После инсталляции своего обработчика, вызываем \TT{RaiseException()}\footnote{\href{http://go.yurichev.com/17253}{MSDN}}.
Это пользовательские исключения. 
Обработчик проверяет код.\\
Если код \TT{0xE1223344}, то он возвращает \TT{ExceptionContinueExecution},
что сигнализирует системе что обработчик скорректировал состояние CPU (обычно это регистры EIP/ESP) и что \ac{OS} может
возобновить исполнение треда.
Если вы немного измените код так что обработчик будет возвращать \TT{ExceptionContinueSearch},
то \ac{OS} будет вызывать остальные
обработчики в цепочке, и вряд ли найдется тот, кто обработает ваше исключение,
ведь информации о нем (вернее, его коде) ни у кого нет.
Вы увидите стандартное окно Windows о падении процесса.

Какова разница между системными исключениями и пользовательскими? Вот системные:

\small
\begin{center}
\begin{tabular}{ | l | l | l | }
\hline
\HeaderColor как определен в WinBase.h & 
\HeaderColor как определен в ntstatus.h & 
\HeaderColor численное значение \\
\hline
EXCEPTION\_ACCESS\_VIOLATION          & STATUS\_ACCESS\_VIOLATION           & 0xC0000005 \\
\hline
EXCEPTION\_DATATYPE\_MISALIGNMENT     & STATUS\_DATATYPE\_MISALIGNMENT      & 0x80000002 \\
\hline
EXCEPTION\_BREAKPOINT                & STATUS\_BREAKPOINT                 & 0x80000003 \\
\hline
EXCEPTION\_SINGLE\_STEP               & STATUS\_SINGLE\_STEP                & 0x80000004 \\
\hline
EXCEPTION\_ARRAY\_BOUNDS\_EXCEEDED     & STATUS\_ARRAY\_BOUNDS\_EXCEEDED      & 0xC000008C \\
\hline
EXCEPTION\_FLT\_DENORMAL\_OPERAND      & STATUS\_FLOAT\_DENORMAL\_OPERAND     & 0xC000008D \\
\hline
EXCEPTION\_FLT\_DIVIDE\_BY\_ZERO        & STATUS\_FLOAT\_DIVIDE\_BY\_ZERO       & 0xC000008E \\
\hline
EXCEPTION\_FLT\_INEXACT\_RESULT        & STATUS\_FLOAT\_INEXACT\_RESULT       & 0xC000008F \\
\hline
EXCEPTION\_FLT\_INVALID\_OPERATION     & STATUS\_FLOAT\_INVALID\_OPERATION    & 0xC0000090 \\
\hline
EXCEPTION\_FLT\_OVERFLOW              & STATUS\_FLOAT\_OVERFLOW             & 0xC0000091 \\
\hline
EXCEPTION\_FLT\_STACK\_CHECK           & STATUS\_FLOAT\_STACK\_CHECK          & 0xC0000092 \\
\hline
EXCEPTION\_FLT\_UNDERFLOW             & STATUS\_FLOAT\_UNDERFLOW            & 0xC0000093 \\
\hline
EXCEPTION\_INT\_DIVIDE\_BY\_ZERO        & STATUS\_INTEGER\_DIVIDE\_BY\_ZERO     & 0xC0000094 \\
\hline
EXCEPTION\_INT\_OVERFLOW              & STATUS\_INTEGER\_OVERFLOW           & 0xC0000095 \\
\hline
EXCEPTION\_PRIV\_INSTRUCTION          & STATUS\_PRIVILEGED\_INSTRUCTION     & 0xC0000096 \\
\hline
EXCEPTION\_IN\_PAGE\_ERROR             & STATUS\_IN\_PAGE\_ERROR              & 0xC0000006 \\
\hline
EXCEPTION\_ILLEGAL\_INSTRUCTION       & STATUS\_ILLEGAL\_INSTRUCTION        & 0xC000001D \\
\hline
EXCEPTION\_NONCONTINUABLE\_EXCEPTION  & STATUS\_NONCONTINUABLE\_EXCEPTION   & 0xC0000025 \\
\hline
EXCEPTION\_STACK\_OVERFLOW            & STATUS\_STACK\_OVERFLOW             & 0xC00000FD \\
\hline
EXCEPTION\_INVALID\_DISPOSITION       & STATUS\_INVALID\_DISPOSITION        & 0xC0000026 \\
\hline
EXCEPTION\_GUARD\_PAGE                & STATUS\_GUARD\_PAGE\_VIOLATION       & 0x80000001 \\
\hline
EXCEPTION\_INVALID\_HANDLE            & STATUS\_INVALID\_HANDLE             & 0xC0000008 \\
\hline
EXCEPTION\_POSSIBLE\_DEADLOCK         & STATUS\_POSSIBLE\_DEADLOCK          & 0xC0000194 \\
\hline
CONTROL\_C\_EXIT                      & STATUS\_CONTROL\_C\_EXIT             & 0xC000013A \\
\hline
\end{tabular}
\end{center}
\normalsize

Так определяется код:

\begin{center}
\begin{bytefield}[bitwidth=0.03\linewidth]{32}
\bitheader[endianness=big]{31,29,28,27,16,15,0} \\
\bitbox{2}{S} & 
\bitbox{1}{U} &
\bitbox{1}{0} & 
\bitbox{12}{Facility code} &
\bitbox{16}{Error code}
\end{bytefield}
\end{center}

S это код статуса: 
11 --- ошибка;
10 --- предупреждение;
01 --- информация;
00 --- успех.
U ---- является ли этот код пользовательским, а не системным.

Вот почему мы выбрали 0xE1223344 --- E\textsubscript{16} (1110\textsubscript{2}) 0xE (1110b)
означает, что это 1) пользовательское исключение; 2) ошибка.
Хотя, если быть честным, этот пример нормально работает и без этих старших бит.

Далее мы пытаемся прочитать значение из памяти по адресу 0.
Конечно, в win32 по этому адресу обычно ничего нет, и сработает исключение.
Однако, первый обработчик, который будет заниматься этим делом --- ваш, и он узнает об этом
первым, проверяя код на соответствие с константной \TT{EXCEPTION\_ACCESS\_VIOLATION}.

А если заглянуть в то что получилось на ассемблере,
то можно увидеть, что код читающий из памяти по адресу 0, выглядит так:

\lstinputlisting[caption=MSVC 2010,style=customasmx86]{OS/SEH/1/1_fragment.asm}

Возможно ли \q{на лету} исправить ошибку и предложить программе исполняться далее?
Да, наш обработчик может изменить значение в \EAX и предложить \ac{OS} исполнить эту же инструкцию еще раз.
Что мы и делаем. \printf напечатает 1234, потому что после работы нашего обработчика, \EAX будет не 0,
а будет содержать адрес глобальной переменной \TT{new\_value}.
Программа будет исполняться далее.

Собственно, вот что происходит: срабатывает защита менеджера памяти в \ac{CPU}, 
он останавливает работу треда, отыскивает в ядре Windows обработчик исключений, 
тот, в свою очередь, начинает вызывать обработчики из цепочки \ac{SEH}, по одному.

Мы компилируем это всё в MSVC 2010, но конечно же, нет никакой гарантии 
что для указателя будет использован именно регистр \EAX.

Этот трюк с подменой адреса эффектно выглядит, и мы рассматриваем его здесь для наглядной иллюстрации работы \ac{SEH}.

Тем не менее, трудно припомнить, применяется ли где-то подобное на практике для исправления ошибок \q{на лету}.

Почему SEH-записи хранятся именно в стеке а не в каком-то другом месте?
Возможно, потому что \ac{OS} не нужно заботиться об освобождении этой информации, эти записи
просто пропадают как ненужные когда функция заканчивает работу.

\myindex{\CStandardLibrary!alloca()}
Это чем-то похоже на alloca(): (\myref{alloca}).

}
\DE{\subsubsection{Vergessen wir MSVC}

Unter Windows ist der Zweck von \ac{SEH} die Ausnahmebehandlung. Nichtsdestotrotz
ist es sprachunabhängig und nicht in irgendeiner Weise an \Cpp oder \ac{OOP} gebunden.

Wir betrachten \ac{SEH} hier in einer isolierten Form und nicht im Zusammenhang mit
C++ oder MSVC-Erweiterungen.

\myindex{Windows!TIB}
\myindex{Windows!Win32!RaiseException()}

Jeder laufende Prozess hat eine Kette von \ac{SEH}-Handles, \ac{TIB} beinhaltet die Adresse
des letzten Handlers.

Wenn eine Ausnahme auftritt (Division durch Null, Zugriff auf fehlerhafte Adresse,
Benutzer-Ausnahme durch Aufruf \TT{RaiseException()}-Funktion), findet das \ac{OS} 
den letzten Handler in der \ac{TIB} und ruft ihn auf. Dabei werden alle Informationen
über den Zustand der \ac{CPU} (Register-Werte usw.) im Moment der Ausnahme übergeben.

Wenn der Ausnahme-Handler eine bekannte Ausnahme sieht wird sie von ihm behandelt.

Ist dies nicht der Fall, wird das \ac{OS} darüber informiert, dass keine Ausnahmebehandlung
stattfand und das \ac{OS} ruft den nächsten Handler in der Kette, bis ein Handler
gefunden wird, der die Ausnahme behandeln kann.

Am Ende der Kette ist ein Standard-Handler, der den wohlbekannten Dialog anzeigt,
welcher den Benutzer über den Prozessabsturz informiert. Zusätzlich werden einige
technische Informationen wie der \ac{CPU}-Status beim Zeitpunkt des Absturzes
und die Möglichkeit zum Senden der Infos an Microsoft-Entwickler angezeigt.

\begin{figure}[H]
\centering
\includegraphics[width=0.6\textwidth]{OS/SEH/1/crash_xp1.png}
\caption{Windows XP}
\end{figure}

\begin{figure}[H]
\centering
\includegraphics[width=0.6\textwidth]{OS/SEH/1/crash_xp2.png}
\caption{Windows XP}
\end{figure}

\begin{figure}[H]
\centering
\includegraphics[width=0.6\textwidth]{OS/SEH/1/crash_win7.png}
\caption{Windows 7}
\end{figure}

\begin{figure}[H]
\centering
\includegraphics[width=0.6\textwidth]{OS/SEH/1/crash_win81.png}
\caption{Windows 8.1}
\end{figure}

Früher wurde dieser Handler Dr. Watson
\footnote{\href{http://go.yurichev.com/17046}{Wikipedia}}
genannt.

%By the way, some developers make their own handler that sends information about the program crash to themselves.
%\myindex{Windows!Win32!SetUnhandledExceptionFilter()}
%It is registered with the help of \TT{SetUnhandledExceptionFilter()} 
%and to be called if the \ac{OS} does not have any other way to handle the exception.
%\myindex{\oracle}
%An example is \oracle---it saves huge dumps reporting all possible information about the \ac{CPU} and memory state.
%
%Let's write our own primitive exception handler.
%This example is based on the example from \PietrekSEH.
%It must be compiled with the SAFESEH option: \TT{cl seh1.cpp /link /safeseh:no}.
%More about SAFESEH here: \href{http://go.yurichev.com/17252}{MSDN}.
%
%\lstinputlisting[style=customc]{OS/SEH/1/1.cpp}
%
%The FS: segment register is pointing to the \ac{TIB} in win32.
%
%The very first element in the \ac{TIB} is a pointer to the last handler in the chain.
%We save it in the stack and store the address of our handler there.
%The structure is named \TT{\_EXCEPTION\_REGISTRATION}, it is a simple singly-linked list and its elements are stored right in the stack.
%
%\begin{lstlisting}[caption=MSVC/VC/crt/src/exsup.inc,style=customasmx86]
%\_EXCEPTION\_REGISTRATION struc
%     prev    dd      ?
%     handler dd      ?
%\_EXCEPTION\_REGISTRATION ends
%\end{lstlisting}
%
%So each \q{handler} field points to a handler and an each \q{prev} field points to the previous record in the stack.
%The last record has \TT{0xFFFFFFFF} (-1) in the \q{prev} field.
%
%\input{OS/SEH/1/tikz}
%
%After our handler is installed, we call \TT{RaiseException()}
%\footnote{\href{http://go.yurichev.com/17253}{MSDN}}.
%This is an user exception. 
%The handler checks the code.
%If the code is \TT{0xE1223344}, it returning \TT{ExceptionContinueExecution},
%which means that handler corrected the CPU state (it is usually a correction of the EIP/ESP registers) and the \ac{OS} can resume the execution of the.
%If you alter slightly the code so the handler returns \TT{ExceptionContinueSearch},
%
%then the \ac{OS} will call the other handlers, and it's unlikely that one who can handle it will be found, since
%no one will have any information about it (rather about its code).
%You will see the standard Windows dialog about a process crash.
%
%What is the difference between a system exceptions and a user one? Here are the system ones:
%
%\small
%\begin{center}
%\begin{tabular}{ | l | l | l | }
%\hline
%\HeaderColor as defined in WinBase.h & 
%\HeaderColor as defined in ntstatus.h & 
%\HeaderColor value \\
%\hline
%EXCEPTION\_ACCESS\_VIOLATION          & STATUS\_ACCESS\_VIOLATION           & 0xC0000005 \\
%\hline
%EXCEPTION\_DATATYPE\_MISALIGNMENT     & STATUS\_DATATYPE\_MISALIGNMENT      & 0x80000002 \\
%\hline
%EXCEPTION\_BREAKPOINT                & STATUS\_BREAKPOINT                 & 0x80000003 \\
%\hline
%EXCEPTION\_SINGLE\_STEP               & STATUS\_SINGLE\_STEP                & 0x80000004 \\
%\hline
%EXCEPTION\_ARRAY\_BOUNDS\_EXCEEDED     & STATUS\_ARRAY\_BOUNDS\_EXCEEDED      & 0xC000008C \\
%\hline
%EXCEPTION\_FLT\_DENORMAL\_OPERAND      & STATUS\_FLOAT\_DENORMAL\_OPERAND     & 0xC000008D \\
%\hline
%EXCEPTION\_FLT\_DIVIDE\_BY\_ZERO        & STATUS\_FLOAT\_DIVIDE\_BY\_ZERO       & 0xC000008E \\
%\hline
%EXCEPTION\_FLT\_INEXACT\_RESULT        & STATUS\_FLOAT\_INEXACT\_RESULT       & 0xC000008F \\
%\hline
%EXCEPTION\_FLT\_INVALID\_OPERATION     & STATUS\_FLOAT\_INVALID\_OPERATION    & 0xC0000090 \\
%\hline
%EXCEPTION\_FLT\_OVERFLOW              & STATUS\_FLOAT\_OVERFLOW             & 0xC0000091 \\
%\hline
%EXCEPTION\_FLT\_STACK\_CHECK           & STATUS\_FLOAT\_STACK\_CHECK          & 0xC0000092 \\
%\hline
%EXCEPTION\_FLT\_UNDERFLOW             & STATUS\_FLOAT\_UNDERFLOW            & 0xC0000093 \\
%\hline
%EXCEPTION\_INT\_DIVIDE\_BY\_ZERO        & STATUS\_INTEGER\_DIVIDE\_BY\_ZERO     & 0xC0000094 \\
%\hline
%EXCEPTION\_INT\_OVERFLOW              & STATUS\_INTEGER\_OVERFLOW           & 0xC0000095 \\
%\hline
%EXCEPTION\_PRIV\_INSTRUCTION          & STATUS\_PRIVILEGED\_INSTRUCTION     & 0xC0000096 \\
%\hline
%EXCEPTION\_IN\_PAGE\_ERROR             & STATUS\_IN\_PAGE\_ERROR              & 0xC0000006 \\
%\hline
%EXCEPTION\_ILLEGAL\_INSTRUCTION       & STATUS\_ILLEGAL\_INSTRUCTION        & 0xC000001D \\
%\hline
%EXCEPTION\_NONCONTINUABLE\_EXCEPTION  & STATUS\_NONCONTINUABLE\_EXCEPTION   & 0xC0000025 \\
%\hline
%EXCEPTION\_STACK\_OVERFLOW            & STATUS\_STACK\_OVERFLOW             & 0xC00000FD \\
%\hline
%EXCEPTION\_INVALID\_DISPOSITION       & STATUS\_INVALID\_DISPOSITION        & 0xC0000026 \\
%\hline
%EXCEPTION\_GUARD\_PAGE                & STATUS\_GUARD\_PAGE\_VIOLATION       & 0x80000001 \\
%\hline
%EXCEPTION\_INVALID\_HANDLE            & STATUS\_INVALID\_HANDLE             & 0xC0000008 \\
%\hline
%EXCEPTION\_POSSIBLE\_DEADLOCK         & STATUS\_POSSIBLE\_DEADLOCK          & 0xC0000194 \\
%\hline
%CONTROL\_C\_EXIT                      & STATUS\_CONTROL\_C\_EXIT             & 0xC000013A \\
%\hline
%\end{tabular}
%\end{center}
%\normalsize
%
%That is how the code is defined:
%
%\begin{center}
%\begin{bytefield}[bitwidth=0.03\linewidth]{32}
%\bitheader[endianness=big]{31,29,28,27,16,15,0} \\
%\bitbox{2}{S} & 
%\bitbox{1}{U} &
%\bitbox{1}{0} & 
%\bitbox{12}{Facility code} &
%\bitbox{16}{Error code}
%\end{bytefield}
%\end{center}
%
%S is a basic status code: 
%11---error;
%10---warning;
%01---informational;
%00---success.
%U---whether the code is user code.
%
%That is why we chose 0xE1223344---E\textsubscript{16} (1110\textsubscript{2}) 0xE (1110b) 
%means that it is 1) user exception; 2) error.
%
%But to be honest, this example works fine without these high bits.
%
%Then we try to read a value from memory at address 0.
%
%Of course, there is nothing at this address in win32, so an exception is raised.
%
%The very first handler is to be called---yours, and it will know about it first, by checking
%the code if it's equal to the \TT{EXCEPTION\_ACCESS\_VIOLATION} constant.
%
%The code that's reading from memory at address 0 is looks like this:
%
%\lstinputlisting[style=customasmx86,caption=MSVC 2010]{OS/SEH/1/1_fragment.asm}
%
%Will it be possible to fix this error \q{on the fly} and to continue with program execution?
%
%Yes, our exception handler can fix the \EAX value and let the \ac{OS} execute this instruction once again.
%So that is what we do. \printf prints 1234, because after the execution of our handler \EAX is not 0,
%but contains the address of the global variable \TT{new\_value}.
%The execution will resume.
%
%That is what is going on: the memory manager in the \ac{CPU} signals about an error, the \ac{CPU} suspends the thread,
%finds the exception handler in the Windows kernel, 
%which, in turn, starts to call all handlers in the \ac{SEH} chain, one by one.
%
%We use MSVC 2010 here, but of course, there is no any guarantee that \EAX will be used for this pointer.
%
%This address replacement trick is showy, and we considering it here as an illustration of \ac{SEH}'s internals.
%Nevertheless, it's hard to recall any case where it is used for \q{on-the-fly} error fixing.
%
%Why SEH-related records are stored right in the stack instead of some other place?
%
%Supposedly because the \ac{OS} is not needing to care about freeing this information, 
%these records are simply disposed when the function finishes its execution.
%\myindex{\CStandardLibrary!alloca()}
%This is somewhat like alloca(): (\myref{alloca}).
%
}
\FR{\subsubsection{Oublions MSVC}

Sous Windows, \ac{SEH} concerne la gestion d'exceptions. Ce mécanisme est indépendant du langage 
de programmation et n'est en aucun cas spécifique ni à \Cpp, ni à l'\ac{OOP}.

Nous nous intéressons donc au \ac{SEH} indépendamment du C++ et des extensions du langage dans MSVC.

\myindex{Windows!TIB}
\myindex{Windows!Win32!RaiseException()}

A chaque processus est associé une chaîne de gestionnaires \ac{SEH}. A tout moment, chaque \ac{TIB} 
référence le gestionnaire le plus récent de la chaîne.

Dès qu'une exception intervient (division par zéro, violation d'accès mémoire, exception explicitement 
déclenchée par le programe en appelant la fonction \TT{RaiseException()} ...), l'OS retrouve dans le 
\ac{TIB} le gestionnaire le plus récemment déclaré et l'appelle. Il lui fourni l'état de la \ac{CPU} 
(contenu des registres ...) tels qu'ils étaient lors du déclenchement de l'exception.

Le gestionnaire examine le type de l'exception et décide de la traiter s'il la reconnaît. Sinon, il 
signale à l'\ac{OS} qu'il passe la main et celui-ci appelle le prochain gestionnaire dans la chaîne, 
et ainsi de suite jusuqu'à ce qu'il trouve un gestionnaire qui soit capable de la traiter.

A la toute fin de la chaîne se trouve un gestionnaire standard qui affiche la fameuse boîte de 
dialogue qui informe l'utilisateur que le processus va être avorté. Ce dialogue fournit quelques
informations techniques au sujet de l'état de la \ac{CPU} au moment du crash, et propose de collecter 
des informations techniques qui seront envoyées aux développeurs de Microsoft. 

\begin{figure}[H]
\centering
\includegraphics[width=0.6\textwidth]{OS/SEH/1/crash_xp1.png}
\caption{Windows XP}
\end{figure}

\begin{figure}[H]
\centering
\includegraphics[width=0.6\textwidth]{OS/SEH/1/crash_xp2.png}
\caption{Windows XP}
\end{figure}

\begin{figure}[H]
\centering
\includegraphics[width=0.6\textwidth]{OS/SEH/1/crash_win7.png}
\caption{Windows 7}
\end{figure}

\begin{figure}[H]
\centering
\includegraphics[width=0.6\textwidth]{OS/SEH/1/crash_win81.png}
\caption{Windows 8.1}
\end{figure}

Historiquement, cette chaîne de gestionnaires était connue sous l'appelation Dr. Watson
\footnote{\href{http://go.yurichev.com/17046}{wikipedia}}.

Certains développeurs ont eu l'idée d'écrire leur propre gestionnaire d'exceptions pour recevoir 
les informations relatives au crash.
\myindex{Windows!Win32!SetUnhandledExceptionFilter()}
Ils enregistrent leur gestionnaire en appelant la fonction \TT{SetUnhandledExceptionFilter()} qui 
sera alors appelée si l'\ac{OS} ne trouve aucun autre gestionnaire qui souhaite gérer l'exception.

\myindex{\oracle}
\oracle--- en est un bon exemple qui sauvegarde un énorme fichier dump collectant toutes les 
informations possibles concernant la \ac{CPU} et l'état de la mémoire.

Ecrivons notre propre gestionnaire d'exception. Cet exemple s'appuie sur celui de \PietrekSEH.
Pour le compiler, il faut utiliser l'option SAFESEH : \TT{cl seh1.cpp /link /safeseh:no}.
Vous trouverez plus d'informations concernant SAFESEH à: \href{http://go.yurichev.com/17252}{MSDN}.

\lstinputlisting[style=customc]{OS/SEH/1/1.cpp}

En environnement win32, le registre de segment FS: contient l'adresse du \ac{TIB}.

Le tout premier élément de la structure \ac{TIB} est un pointeur sur le premier gestionnaire de la 
chaîne de traitement des exceptions.
Nous le sauvegardons sur la pile et remplaçons la valeur par celle de notre propre gestionnaire.
La structure est du type \TT{\_EXCEPTION\_REGISTRATION}. Il s'agit d'une simple liste chaînée dont 
les éléments sont conservés sur la pile.

\begin{lstlisting}[caption=MSVC/VC/crt/src/exsup.inc,style=customasmx86]
\_EXCEPTION\_REGISTRATION struc
     prev    dd      ?
     handler dd      ?
\_EXCEPTION\_REGISTRATION ends
\end{lstlisting}

Le champ \q{handler} contient l'adresse du gestionnaire et le champ \q{prev} celle de 
l'enregistrement suivant dans la chaîne.
Le dernier enregistremThe last record contient la valeur \TT{0xFFFFFFFF} (-1) dans son champ \q{prev}.

\input{OS/SEH/1/tikz}

Une fois notre gestionnaire installé, nous invoquons la fonction \TT{RaiseException()}
\footnote{\href{http://go.yurichev.com/17253}{MSDN}}.
Il s'agit d'une exception utilisateur. 
Le gestionnaire vérifie le code.
Si le code est égal à \TT{0xE1223344}, il retourne la valeur \TT{ExceptionContinueExecution} qui 
signifie que le gestionnaire a corrigé le contenu de la structure passée en paramètre qui décrit 
l'état de la CPU. La modification concerne généralement les registres EIP/ESP. L'\ac{OS} peut alors 
reprendre l'exécution du thread.

Si vous modifiez légèrement le code pour que le gestionaire retourne la valeur \TT{ExceptionContinueSearch},
l'\ac{OS} appelera les gestionnaires suivants dans la liste. Il est peu probable que l'un d'eux sache 
la gérer puisqu'aucun d'eux ne la comprend, ni ne connait le code exception.
Vous verrez donc apparaître la boîte de dialogue Windows précurseure du crash.

Quelles sont les différences entre les exceptions système et les exceptions utilisateur ?
Les exceptions système sont listées ci-dessous:

\small
\begin{center}
\begin{tabular}{ | l | l | l | }
\hline
\HeaderColor as defined in WinBase.h & 
\HeaderColor as defined in ntstatus.h & 
\HeaderColor value \\
\hline
EXCEPTION\_ACCESS\_VIOLATION          & STATUS\_ACCESS\_VIOLATION           & 0xC0000005 \\
\hline
EXCEPTION\_DATATYPE\_MISALIGNMENT     & STATUS\_DATATYPE\_MISALIGNMENT      & 0x80000002 \\
\hline
EXCEPTION\_BREAKPOINT                & STATUS\_BREAKPOINT                 & 0x80000003 \\
\hline
EXCEPTION\_SINGLE\_STEP               & STATUS\_SINGLE\_STEP                & 0x80000004 \\
\hline
EXCEPTION\_ARRAY\_BOUNDS\_EXCEEDED     & STATUS\_ARRAY\_BOUNDS\_EXCEEDED      & 0xC000008C \\
\hline
EXCEPTION\_FLT\_DENORMAL\_OPERAND      & STATUS\_FLOAT\_DENORMAL\_OPERAND     & 0xC000008D \\
\hline
EXCEPTION\_FLT\_DIVIDE\_BY\_ZERO        & STATUS\_FLOAT\_DIVIDE\_BY\_ZERO       & 0xC000008E \\
\hline
EXCEPTION\_FLT\_INEXACT\_RESULT        & STATUS\_FLOAT\_INEXACT\_RESULT       & 0xC000008F \\
\hline
EXCEPTION\_FLT\_INVALID\_OPERATION     & STATUS\_FLOAT\_INVALID\_OPERATION    & 0xC0000090 \\
\hline
EXCEPTION\_FLT\_OVERFLOW              & STATUS\_FLOAT\_OVERFLOW             & 0xC0000091 \\
\hline
EXCEPTION\_FLT\_STACK\_CHECK           & STATUS\_FLOAT\_STACK\_CHECK          & 0xC0000092 \\
\hline
EXCEPTION\_FLT\_UNDERFLOW             & STATUS\_FLOAT\_UNDERFLOW            & 0xC0000093 \\
\hline
EXCEPTION\_INT\_DIVIDE\_BY\_ZERO        & STATUS\_INTEGER\_DIVIDE\_BY\_ZERO     & 0xC0000094 \\
\hline
EXCEPTION\_INT\_OVERFLOW              & STATUS\_INTEGER\_OVERFLOW           & 0xC0000095 \\
\hline
EXCEPTION\_PRIV\_INSTRUCTION          & STATUS\_PRIVILEGED\_INSTRUCTION     & 0xC0000096 \\
\hline
EXCEPTION\_IN\_PAGE\_ERROR             & STATUS\_IN\_PAGE\_ERROR              & 0xC0000006 \\
\hline
EXCEPTION\_ILLEGAL\_INSTRUCTION       & STATUS\_ILLEGAL\_INSTRUCTION        & 0xC000001D \\
\hline
EXCEPTION\_NONCONTINUABLE\_EXCEPTION  & STATUS\_NONCONTINUABLE\_EXCEPTION   & 0xC0000025 \\
\hline
EXCEPTION\_STACK\_OVERFLOW            & STATUS\_STACK\_OVERFLOW             & 0xC00000FD \\
\hline
EXCEPTION\_INVALID\_DISPOSITION       & STATUS\_INVALID\_DISPOSITION        & 0xC0000026 \\
\hline
EXCEPTION\_GUARD\_PAGE                & STATUS\_GUARD\_PAGE\_VIOLATION       & 0x80000001 \\
\hline
EXCEPTION\_INVALID\_HANDLE            & STATUS\_INVALID\_HANDLE             & 0xC0000008 \\
\hline
EXCEPTION\_POSSIBLE\_DEADLOCK         & STATUS\_POSSIBLE\_DEADLOCK          & 0xC0000194 \\
\hline
CONTROL\_C\_EXIT                      & STATUS\_CONTROL\_C\_EXIT             & 0xC000013A \\
\hline
\end{tabular}
\end{center}
\normalsize

Le code de chaque exception se décompose comme suit:

\begin{center}
\begin{bytefield}[bitwidth=0.03\linewidth]{32}
\bitheader[endianness=big]{31,29,28,27,16,15,0} \\
\bitbox{2}{S} & 
\bitbox{1}{U} &
\bitbox{1}{0} & 
\bitbox{12}{Facility code} &
\bitbox{16}{Error code}
\end{bytefield}
\end{center}

S est un code status de base: 
11---erreur;
10---warning;
01---information;
00---succès.
U---lorsqu'il s'agit d'un code utilisateur.

Voici pourquoi nous choisissons le code 0xE1223344---E\textsubscript{16} (1110\textsubscript{2}) 0xE (1110b) 
qui signifie 1) qu'il s'agit d'une exception utilisateur; 2) qu'il s'agit d'une erreur.

Pour être honnête, l'exemple fonctionne aussi bien sans ces bits de poids fort.

Tentons maintenant de lire la valeur à l'adresse mémoire 0.

Bien entendu, dans win32 il n'existe rien à cette adresse, ce qui déclenche une exception.

Le premier gestionnaire à être invoqué est le vôtre. Il reconnaît l'exception car il compare
le code avec celui de la constante \TT{EXCEPTION\_ACCESS\_VIOLATION}.

Le code qui lit la mémoire à l'adresse 0 ressemble à ceci:

\lstinputlisting[style=customasmx86,caption=MSVC 2010]{OS/SEH/1/1_fragment.asm}

Serait-il possible de corriger cette erreur \q{on the fly} afin de continuer l'exécution du programme?

Notre gestionnaire d'exception peut modifier la valeur du registre \EAX puis laisser l'\ac{OS} 
exécuter de nouveau l'instruction fautive.
C'est ce que nous faisons et la raison pour laquelle \printf affiche 1234. Lorsque notre gestionnaire 
a fini son travail, la valeur de \EAX n'est plus 0 mais l'adresse de la variable globale \TT{new\_value}.
L'exécution du programme se poursuit donc.

Voici ce qui se passe: le gestionnaire mémoire de la \ac{CPU} signale une erreur, la \ac{CPU} suspend 
le thread, trouve le gestionnaire d'exception dans le noyau Windows, lequel à son tour appelle les 
gestionnaires de la chaîne \ac{SEH} un par un.

Nous utilisons ici le compilateur MSVC 2010. Bien entendu, il n'y a aucune garantie que celui-ci 
décide d'utiliser le registre \EAX pour conserver la valeur du pointeur.

Le truc du remplacement du contenu du registre n'est qu'une illustration de ce que peut être le 
fonctionnement interne des \ac{SEH}.
En pratique, il est très rare qu'il soit utilisé pour corriger \q{on-the-fly} une erreur.

Pourquoi les enregistrements SEH sont-ils conservés directement sur la pile et non pas à un autre 
endroit?

L'explication la plus plausible est que l'\ac{OS} n'a ainsi pas besoin de libérer l'espace qu'ils 
utilisent. Ces enregistrements sont automatiquement supprimés lorsque la fonction se termine.
\myindex{\CStandardLibrary!alloca()}
C'est un peu comme la fonction alloca(): (\myref{alloca}).

}

\EN{\subsubsection{Now let's get back to MSVC}

\myindex{\Cpp!exceptions}

Supposedly, Microsoft programmers needed exceptions in C, but not in \Cpp, so they added a non-standard C extension
to MSVC\footnote{\href{http://go.yurichev.com/17057}{MSDN}}.
It is not related to C++ \ac{PL} exceptions.

% FIXME russian listing:
\begin{lstlisting}[style=customc]
__try
{
    ...
}
__except(filter code)
{
    handler code
}
\end{lstlisting}

\q{Finally} block may be instead of handler code:

\begin{lstlisting}[style=customc]
__try
{
    ...
}
__finally
{
    ...
}
\end{lstlisting}


The filter code is an expression, telling whether this handler code corresponds to the exception raised.

If your code is too big and cannot fit into one expression, a separate filter function can be defined.\\
\\
There are a lot of such constructs in the Windows kernel.
Here are a couple of examples from there (\ac{WRK}):

\lstinputlisting[caption=WRK-v1.2/base/ntos/ob/obwait.c,style=customc]{OS/SEH/2/wrk_ex1.c}

\lstinputlisting[caption=WRK-v1.2/base/ntos/cache/cachesub.c,style=customc]{OS/SEH/2/wrk_ex2.c}

Here is also a filter code example:

\lstinputlisting[caption=WRK-v1.2/base/ntos/cache/copysup.c,style=customc]{OS/SEH/2/wrk_ex3.c}

Internally, SEH is an extension of the OS-supported exceptions.
But the handler function is \TT{\_except\_handler3} (for SEH3) or \TT{\_except\_handler4} (for SEH4).

The code of this handler is MSVC-related, it is located in its libraries, or in msvcr*.dll.
It is very important to know that SEH is a MSVC thing.

Other win32-compilers may offer something completely different.

\myparagraph{SEH3}

SEH3 has \TT{\_except\_handler3} 
as a handler function, and extends the \TT{\_EXCEPTION\_REGISTRATION} table, adding
a pointer to the \IT{scope table} and \IT{previous try level} variable.
SEH4 extends the \IT{scope table} 
by 4 values for buffer overflow protection.\\
\\
The \IT{scope table} is a table that consists of pointers to the filter and handler code blocks, for each nested level of \IT{try/except}.

\input{OS/SEH/2/tikz}

Again, it is very important to understand that the \ac{OS} takes care only of the \IT{prev/handle} fields, and nothing more.\\
It is the job of the \TT{\_except\_handler3} function to read the other fields and \IT{scope table}, and decide
which handler to execute and when.\\
\\
\myindex{Wine}
\myindex{ReactOS}
The source code of the \TT{\_except\_handler3} function is closed.

However, Sanos OS, which has a win32 compatibility layer, has the same
functions reimplemented, which are somewhat equivalent to those in Windows
\footnote{\url{http://go.yurichev.com/17058}}.
Another reimplementation is present in 
Wine\footnote{\href{http://go.yurichev.com/17059}{GitHub}}
and ReactOS\footnote{\url{http://go.yurichev.com/17060}}.\\
\\
If the \IT{filter} pointer is NULL, the \IT{handler} 
pointer is the pointer to the \IT{finally} code block.\\
\\
During execution, the \IT{previous try level} value in the stack changes, so
\TT{\_except\_handler3} can get information about the current level of nestedness, 
in order to know which \IT{scope table} entry to use.

\myparagraph{SEH3: one try/except block example}

\lstinputlisting[style=customc]{OS/SEH/2/2.c}

\lstinputlisting[caption=MSVC 2003,style=customasmx86]{OS/SEH/2/2_SEH3.asm}

Here we see how the SEH frame is constructed in the stack.
The \IT{scope table} is located in the \TT{CONST} segment---indeed, these fields are not to be changed.
An interesting thing is how the \IT{previous try level} variable has changed.
The initial value is \TT{0xFFFFFFFF} ($-1$).
The moment when the body of the \TT{try} 
statement is opened is marked with an instruction that writes 0 to the variable.
The moment when the body of the \TT{try} statement is closed, $-1$ 
is written back to it.
We also see the addresses of filter and handler code.

Thus we can easily see the structure of the \IT{try/except} constructs in the function.\\
\\
Since the SEH setup code in the function prologue may be shared between many functions,
sometimes the compiler inserts a call to the \TT{SEH\_prolog()} function in the prologue, which does just that.

The SEH cleanup code is in the \TT{SEH\_epilog()} function.\\
\\
Let's try to run this example in \tracer{}:
\myindex{tracer}

\begin{lstlisting}
tracer.exe -l:2.exe --dump-seh
\end{lstlisting}

\begin{lstlisting}[caption=tracer.exe output]
EXCEPTION_ACCESS_VIOLATION at 2.exe!main+0x44 (0x401054) ExceptionInformation[0]=1
EAX=0x00000000 EBX=0x7efde000 ECX=0x0040cbc8 EDX=0x0008e3c8
ESI=0x00001db1 EDI=0x00000000 EBP=0x0018feac ESP=0x0018fe80
EIP=0x00401054
FLAGS=AF IF RF
* SEH frame at 0x18fe9c prev=0x18ff78 handler=0x401204 (2.exe!_except_handler3)
SEH3 frame. previous trylevel=0
scopetable entry[0]. previous try level=-1, filter=0x401070 (2.exe!main+0x60) handler=0x401088 (2.exe!main+0x78)
* SEH frame at 0x18ff78 prev=0x18ffc4 handler=0x401204 (2.exe!_except_handler3)
SEH3 frame. previous trylevel=0
scopetable entry[0]. previous try level=-1, filter=0x401531 (2.exe!mainCRTStartup+0x18d) handler=0x401545 (2.exe!mainCRTStartup+0x1a1)
* SEH frame at 0x18ffc4 prev=0x18ffe4 handler=0x771f71f5 (ntdll.dll!__except_handler4)
SEH4 frame. previous trylevel=0
SEH4 header:	GSCookieOffset=0xfffffffe GSCookieXOROffset=0x0
		EHCookieOffset=0xffffffcc EHCookieXOROffset=0x0
scopetable entry[0]. previous try level=-2, filter=0x771f74d0 (ntdll.dll!___safe_se_handler_table+0x20) handler=0x771f90eb (ntdll.dll!_TppTerminateProcess@4+0x43)
* SEH frame at 0x18ffe4 prev=0xffffffff handler=0x77247428 (ntdll.dll!_FinalExceptionHandler@16)
\end{lstlisting}

We see that the SEH chain consists of 4 handlers.\\
\\
\myindex{CRT}
The first two are located in our example. Two?
But we made only one?
Yes, another one has been set up in the \ac{CRT} function \TT{\_mainCRTStartup()}, and as it seems that it handles at least \ac{FPU} exceptions.
Its source code can found in the MSVC installation: \TT{crt/src/winxfltr.c}.\\
\\
The third is the SEH4 one in ntdll.dll, 
and the fourth handler is not MSVC-related and is located in ntdll.dll, and has a self-describing function name.\\
\\
As you can see, there are 3 types of handlers in one chain:

one is not related to MSVC at all (the last one) and two MSVC-related: SEH3 and SEH4.

\myparagraph{SEH3: two try/except blocks example}

\lstinputlisting[style=customc]{OS/SEH/2/3.c}

Now there are two \TT{try} blocks.
So the \IT{scope table} now has two entries, one for each block.
\IT{Previous try level} changes as execution flow enters or exits the \TT{try} block.

\lstinputlisting[caption=MSVC 2003,style=customasmx86]{OS/SEH/2/3_SEH3.asm}

If we set a breakpoint on the \printf{} function, which is called from the handler, 
we can also see how yet another SEH handler is added.

Perhaps it's another machinery inside the SEH handling process.
Here we also see our \IT{scope table} consisting of 2 entries.

\begin{lstlisting}
tracer.exe -l:3.exe bpx=3.exe!printf --dump-seh
\end{lstlisting}

\begin{lstlisting}[caption=tracer.exe output]
(0) 3.exe!printf
EAX=0x0000001b EBX=0x00000000 ECX=0x0040cc58 EDX=0x0008e3c8
ESI=0x00000000 EDI=0x00000000 EBP=0x0018f840 ESP=0x0018f838
EIP=0x004011b6
FLAGS=PF ZF IF
* SEH frame at 0x18f88c prev=0x18fe9c handler=0x771db4ad (ntdll.dll!ExecuteHandler2@20+0x3a)
* SEH frame at 0x18fe9c prev=0x18ff78 handler=0x4012e0 (3.exe!_except_handler3)
SEH3 frame. previous trylevel=1
scopetable entry[0]. previous try level=-1, filter=0x401120 (3.exe!main+0xb0) handler=0x40113b (3.exe!main+0xcb)
scopetable entry[1]. previous try level=0, filter=0x4010e8 (3.exe!main+0x78) handler=0x401100 (3.exe!main+0x90)
* SEH frame at 0x18ff78 prev=0x18ffc4 handler=0x4012e0 (3.exe!_except_handler3)
SEH3 frame. previous trylevel=0
scopetable entry[0]. previous try level=-1, filter=0x40160d (3.exe!mainCRTStartup+0x18d) handler=0x401621 (3.exe!mainCRTStartup+0x1a1)
* SEH frame at 0x18ffc4 prev=0x18ffe4 handler=0x771f71f5 (ntdll.dll!__except_handler4)
SEH4 frame. previous trylevel=0
SEH4 header:	GSCookieOffset=0xfffffffe GSCookieXOROffset=0x0
		EHCookieOffset=0xffffffcc EHCookieXOROffset=0x0
scopetable entry[0]. previous try level=-2, filter=0x771f74d0 (ntdll.dll!___safe_se_handler_table+0x20) handler=0x771f90eb (ntdll.dll!_TppTerminateProcess@4+0x43)
* SEH frame at 0x18ffe4 prev=0xffffffff handler=0x77247428 (ntdll.dll!_FinalExceptionHandler@16)
\end{lstlisting}

\myparagraph{SEH4}

\myindex{\BufferOverflow}
\myindex{Security cookie}
During a buffer overflow (\myref{subsec:bufferoverflow}) attack, the address of the \IT{scope table} 
can be rewritten, so starting from MSVC 2005, SEH3 was upgraded to SEH4 in order to have buffer overflow protection.
The pointer to the \IT{scope table} is now \glslink{xoring}{xored} with a \gls{security cookie}.
The \IT{scope table} was extended to have a header consisting of two pointers to \IT{security cookies}.

Each element has an offset inside the stack of another value: 
the address of the \gls{stack frame} (\EBP) \glslink{xoring}{xored} with the \TT{security\_cookie} , placed in the stack.

This value will be read during exception handling and checked for correctness.
The \IT{security cookie} in the stack is random each time, so hopefully a remote attacker can't predict it. \\
\\
The initial \IT{previous try level} is $-2$ in SEH4 instead of $-1$.

\def\SEHfour{1}
\input{OS/SEH/2/tikz}

Here are both examples compiled in MSVC 2012 with SEH4:

\lstinputlisting[caption=MSVC 2012: one try block example,style=customasmx86]{OS/SEH/2/2_SEH4.asm}

\lstinputlisting[caption=MSVC 2012: two try blocks example,style=customasmx86]{OS/SEH/2/3_SEH4.asm}

Here is the meaning of the \IT{cookies}: \TT{Cookie Offset} 
is the difference between the address of the saved EBP value in the stack
and the $EBP \oplus security\_cookie$ value in the stack.
\TT{Cookie XOR Offset} is an additional difference between the 
$EBP \oplus security\_cookie$ value and what is
stored in the stack.

If this equation is not true, the process is to halt due to stack corruption:

\begin{center}
$security\_cookie \oplus (CookieXOROffset + address\_of\_saved\_EBP) == stack[address\_of\_saved\_EBP + CookieOffset]$
\end{center}

If \TT{Cookie Offset} is $-2$, this implies that it is not present.

\myindex{tracer}
\IT{Cookies} checking is also implemented in my \tracer{},
see \href{http://go.yurichev.com/17061}{GitHub} for details.\\
\\
It is still possible to fall back to SEH3 in the compilers after 
(and including) MSVC 2005 by setting the \TT{/GS-} option,
however, the \ac{CRT} code use SEH4 anyway.

}
\RU{\subsubsection{Теперь вспомним MSVC}

\myindex{\Cpp!исключения}
Должно быть, программистам Microsoft были нужны исключения в Си, но не в \Cpp, так что они добавили нестандартное расширение Си в MSVC
\footnote{\href{http://go.yurichev.com/17057}{MSDN}}.
Оно не связано с исключениями в Си++.

% FIXME russian listing:
\begin{lstlisting}[style=customc]
__try
{
    ...
}
__except(filter code)
{
    handler code
}
\end{lstlisting}

Блок \q{finally} может присутствовать вместо код обработчика:

\begin{lstlisting}[style=customc]
__try
{
    ...
}
__finally
{
    ...
}
\end{lstlisting}

Код-фильтр --- это выражение, отвечающее на вопрос, соответствует ли код этого обработчика к поднятому исключению.
Если ваш код слишком большой и не помещается в одно выражение, отдельная функция-фильтр может быть определена.\\
\\
Таких конструкций много в ядре Windows.
Вот несколько примеров оттуда (\ac{WRK}):

\lstinputlisting[caption=WRK-v1.2/base/ntos/ob/obwait.c,style=customc]{OS/SEH/2/wrk_ex1.c}

\lstinputlisting[caption=WRK-v1.2/base/ntos/cache/cachesub.c,style=customc]{OS/SEH/2/wrk_ex2.c}

Вот пример кода-фильтра:

\lstinputlisting[caption=WRK-v1.2/base/ntos/cache/copysup.c,style=customc]{OS/SEH/2/wrk_ex3.c}

Внутри, SEH это расширение исключений поддерживаемых OS.

Но функция обработчик теперь или \TT{\_except\_handler3} (для SEH3) или \TT{\_except\_handler4} (для SEH4).
Код обработчика от MSVC, расположен в его библиотеках, или же в  msvcr*.dll.
Очень важно понимать, что SEH это специфичное для MSVC.
Другие win32-компиляторы могут предлагать что-то совершенно другое.

\myparagraph{SEH3}

SEH3 имеет \TT{\_except\_handler3} как функцию-обработчик, и расширяет структуру \\
\TT{\_EXCEPTION\_REGISTRATION} добавляя указатель на \IT{scope table}
и переменную \IT{previous try level}.
SEH4 расширяет \IT{scope table} добавляя еще 4 значения связанных с защитой от переполнения буфера.\\
\\
\IT{scope table} это таблица, состоящая из указателей на код фильтра и обработчика, для каждого уровня вложенности \IT{try/except}.

\input{OS/SEH/2/tikz}

И снова, очень важно понимать, что OS заботится только о полях \IT{prev/handle}, и больше ничего.
Это работа функции \TT{\_except\_handler3} читать другие поля, читать \IT{scope table} и решать, какой обработчик исполнять и когда.\\
\\
\myindex{Wine}
\myindex{ReactOS}
Исходный код функции \TT{\_except\_handler3} закрыт.
Хотя, Sanos OS, имеющая слой совместимости с win32, имеет некоторые функции написанные заново, которые
в каком-то смысле эквивалентны тем что в Windows
\footnote{\url{http://go.yurichev.com/17058}}.
Другие попытки реализации имеются в 
Wine\footnote{\href{http://go.yurichev.com/17059}{GitHub}}
и ReactOS\footnote{\url{http://go.yurichev.com/17060}}.\\
\\
Если указатель \IT{filter} ноль, \IT{handler} указывает на код \IT{finally} .\\
\\
Во время исполнения, значение \IT{previous try level} в стеке меняется, чтобы функция
\TT{\_except\_handler3} знала о текущем уровне вложенности, чтобы знать, какой элемент таблицы
\IT{scope table} использовать.

\myparagraph{SEH3: пример с одним блоком try/except}

\lstinputlisting[style=customc]{OS/SEH/2/2.c}

\lstinputlisting[caption=MSVC 2003,style=customasmx86]{OS/SEH/2/2_SEH3.asm}

Здесь мы видим, как структура SEH конструируется в стеке.
\IT{Scope table} расположена в сегменте \TT{CONST} --- действительно, эти поля не будут меняться.
Интересно, как меняется переменная \IT{previous try level}.
Исходное значение \TT{0xFFFFFFFF} ($-1$).
Момент, когда тело \TT{try} открывается, обозначен инструкцией, записывающей 0 в эту переменную.
В момент, когда тело \TT{try} закрывается, $-1$ возвращается в нее назад.
Мы также видим адреса кода фильтра и обработчика.
Так мы можем легко увидеть структуру конструкций \IT{try/except} в функции.\\
\\
Так как код инициализации SEH-структур в прологе функций может быть общим для нескольких функций, иногда компилятор
вставляет в прологе вызов функции \TT{SEH\_prolog()}, которая всё это делает.
А код для деинициализации SEH в функции \TT{SEH\_epilog()}.\\
\\
Запустим этот пример в \tracer{}:
\myindex{tracer}

\begin{lstlisting}
tracer.exe -l:2.exe --dump-seh
\end{lstlisting}

\begin{lstlisting}[caption=tracer.exe output]
EXCEPTION_ACCESS_VIOLATION at 2.exe!main+0x44 (0x401054) ExceptionInformation[0]=1
EAX=0x00000000 EBX=0x7efde000 ECX=0x0040cbc8 EDX=0x0008e3c8
ESI=0x00001db1 EDI=0x00000000 EBP=0x0018feac ESP=0x0018fe80
EIP=0x00401054
FLAGS=AF IF RF
* SEH frame at 0x18fe9c prev=0x18ff78 handler=0x401204 (2.exe!_except_handler3)
SEH3 frame. previous trylevel=0
scopetable entry[0]. previous try level=-1, filter=0x401070 (2.exe!main+0x60) handler=0x401088 (2.exe!main+0x78)
* SEH frame at 0x18ff78 prev=0x18ffc4 handler=0x401204 (2.exe!_except_handler3)
SEH3 frame. previous trylevel=0
scopetable entry[0]. previous try level=-1, filter=0x401531 (2.exe!mainCRTStartup+0x18d) handler=0x401545 (2.exe!mainCRTStartup+0x1a1)
* SEH frame at 0x18ffc4 prev=0x18ffe4 handler=0x771f71f5 (ntdll.dll!__except_handler4)
SEH4 frame. previous trylevel=0
SEH4 header:	GSCookieOffset=0xfffffffe GSCookieXOROffset=0x0
		EHCookieOffset=0xffffffcc EHCookieXOROffset=0x0
scopetable entry[0]. previous try level=-2, filter=0x771f74d0 (ntdll.dll!___safe_se_handler_table+0x20) handler=0x771f90eb (ntdll.dll!_TppTerminateProcess@4+0x43)
* SEH frame at 0x18ffe4 prev=0xffffffff handler=0x77247428 (ntdll.dll!_FinalExceptionHandler@16)
\end{lstlisting}

Мы видим, что цепочка SEH состоит из 4-х обработчиков.\\
\\
\myindex{CRT}
Первые два расположены в нашем примере. Два?
Но ведь мы же сделали только один?
Да, второй был установлен в \ac{CRT}-функции 
\TT{\_mainCRTStartup()}, и судя по всему, он обрабатывает как минимум исключения связанные с \ac{FPU}.
Его код можно посмотреть в инсталляции MSVC: \TT{crt/src/winxfltr.c}.\\
\\
Третий это SEH4 в ntdll.dll, и четвертый это обработчик, не имеющий отношения к MSVC, расположенный в ntdll.dll, имеющий \q{говорящее} название функции.\\
\\
Как видно, в цепочке присутствуют обработчики трех типов:
один не связан с MSVC вообще (последний) и два связанных с MSVC: SEH3 и SEH4.

\myparagraph{SEH3: пример с двумя блоками try/except}

\lstinputlisting[style=customc]{OS/SEH/2/3.c}

Теперь здесь два блока \TT{try}.
Так что \IT{scope table} теперь содержит два элемента, один элемент на каждый блок.
\IT{Previous try level} меняется вместе с тем, как исполнение доходит до очередного \TT{try}-блока, либо выходит из него.

\lstinputlisting[caption=MSVC 2003,style=customasmx86]{OS/SEH/2/3_SEH3.asm}

Если установить точку останова на функцию \printf{} вызываемую из обработчика,
мы можем увидеть, что добавился еще один SEH-обработчик.
Наверное, это еще какая-то дополнительная механика, скрытая внутри процесса обработки исключений.
Тут мы также видим \IT{scope table} состоящую из двух элементов.

\begin{lstlisting}
tracer.exe -l:3.exe bpx=3.exe!printf --dump-seh
\end{lstlisting}

\begin{lstlisting}[caption=tracer.exe output]
(0) 3.exe!printf
EAX=0x0000001b EBX=0x00000000 ECX=0x0040cc58 EDX=0x0008e3c8
ESI=0x00000000 EDI=0x00000000 EBP=0x0018f840 ESP=0x0018f838
EIP=0x004011b6
FLAGS=PF ZF IF
* SEH frame at 0x18f88c prev=0x18fe9c handler=0x771db4ad (ntdll.dll!ExecuteHandler2@20+0x3a)
* SEH frame at 0x18fe9c prev=0x18ff78 handler=0x4012e0 (3.exe!_except_handler3)
SEH3 frame. previous trylevel=1
scopetable entry[0]. previous try level=-1, filter=0x401120 (3.exe!main+0xb0) handler=0x40113b (3.exe!main+0xcb)
scopetable entry[1]. previous try level=0, filter=0x4010e8 (3.exe!main+0x78) handler=0x401100 (3.exe!main+0x90)
* SEH frame at 0x18ff78 prev=0x18ffc4 handler=0x4012e0 (3.exe!_except_handler3)
SEH3 frame. previous trylevel=0
scopetable entry[0]. previous try level=-1, filter=0x40160d (3.exe!mainCRTStartup+0x18d) handler=0x401621 (3.exe!mainCRTStartup+0x1a1)
* SEH frame at 0x18ffc4 prev=0x18ffe4 handler=0x771f71f5 (ntdll.dll!__except_handler4)
SEH4 frame. previous trylevel=0
SEH4 header:	GSCookieOffset=0xfffffffe GSCookieXOROffset=0x0
		EHCookieOffset=0xffffffcc EHCookieXOROffset=0x0
scopetable entry[0]. previous try level=-2, filter=0x771f74d0 (ntdll.dll!___safe_se_handler_table+0x20) handler=0x771f90eb (ntdll.dll!_TppTerminateProcess@4+0x43)
* SEH frame at 0x18ffe4 prev=0xffffffff handler=0x77247428 (ntdll.dll!_FinalExceptionHandler@16)
\end{lstlisting}

\myparagraph{SEH4}

\myindex{\BufferOverflow}
\myindex{Security cookie}
Во время атаки переполнения буфера (\myref{subsec:bufferoverflow})
адрес \IT{scope table} может быть перезаписан, так что начиная с MSVC 2005, SEH3 был дополнен защитой от переполнения буфера, до SEH4.
Указатель на \IT{scope table} теперь \glslink{xoring}{про-XOR-ен} с \gls{security cookie}.

\IT{Scope table} расширена, теперь имеет заголовок, содержащий 2 указателя на \IT{security cookies}.
Каждый элемент имеет смешение внутри стека на другое значение: это адрес \glslink{stack frame}{фрейма} (\EBP) также \glslink{xoring}{про-XOR-еный} с 
\TT{security\_cookie} расположенный в стеке.
Это значение будет прочитано во время обработки исключения и проверено на правильность.

\IT{Security cookie} в стеке случайное каждый раз, так что атакующий, как мы надеемся, не может предсказать его.\\
\\
Изначальное значение \IT{previous try level} это $-2$ в SEH4 вместо $-1$.

\def\SEHfour{1}
\input{OS/SEH/2/tikz}

Оба примера скомпилированные в MSVC 2012 с SEH4:

\lstinputlisting[caption=MSVC 2012: one try block example,style=customasmx86]{OS/SEH/2/2_SEH4.asm}

\lstinputlisting[caption=MSVC 2012: two try blocks example,style=customasmx86]{OS/SEH/2/3_SEH4.asm}

Вот значение \IT{cookies}: \TT{Cookie Offset} 
это разница между адресом записанного в стеке значения EBP и значения $EBP \oplus security\_cookie$ в стеке.
\TT{Cookie XOR Offset} это дополнительная разница между значением $EBP \oplus security\_cookie$ и тем что записано в стеке.
Если это уравнение не верно, то процесс остановится из-за разрушения стека:

\begin{center}
$security\_cookie \oplus (CookieXOROffset + address\_of\_saved\_EBP) == stack[address\_of\_saved\_EBP + CookieOffset]$
\end{center}

Если \TT{Cookie Offset} равно $-2$, это значит, что оно не присутствует.

\myindex{tracer}
Проверка \IT{cookies} также реализована в моем \tracer{},
смотрите \href{http://go.yurichev.com/17061}{GitHub} для деталей.\\
\\
Возможность переключиться назад на SEH3 все еще присутствует в компиляторах после (и включая) MSVC 2005, нужно включить
опцию \TT{/GS-}, впрочем, \ac{CRT}-код будет продолжать использовать SEH4.

}
\DE{\subsubsection{Zurück zu MSVC}

\myindex{\Cpp!exceptions}
Offensichtlich benötigten die Microsoft-Entwickler Ausnahmen in C aber nicht in
\Cpp und führten eine nicht-standardisierte C-Erweiterung ein \footnote{\href{http://go.yurichev.com/17057}{MSDN}}.
Diese hat aber keinen Zusammenhang zu  C++ \ac{PL}-Ausnahmen.

% FIXME russian listing:
\begin{lstlisting}[style=customc]
__try
{
    ...
}
__except(filter code)
{
    handler code
}
\end{lstlisting}

Der \q{Finally}-Block kann anstelle des Handler-Codes stehen:

\begin{lstlisting}[style=customc]
__try
{
    ...
}
__finally
{
    ...
}
\end{lstlisting}

Der Filter-Code ist ein Ausdruck, der anzeigt, ob dieser Handler-Code zu der
geworfenen Ausnahme passt.

If der Code zu groß ist und nicht in einen Ausdruck passt, kann eine separate
Filter-Funktion definiert werden.

Im Windows-Kernel existieren eine Reihe solcher Konstrukte.
Nachfolgend einige Beispiel von dort (\ac{WRK}):

\lstinputlisting[caption=WRK-v1.2/base/ntos/ob/obwait.c,style=customc]{OS/SEH/2/wrk_ex1.c}

\lstinputlisting[caption=WRK-v1.2/base/ntos/cache/cachesub.c,style=customc]{OS/SEH/2/wrk_ex2.c}

Hier ist ein Filter-Code-Beispiel:

\lstinputlisting[caption=WRK-v1.2/base/ntos/cache/copysup.c,style=customc]{OS/SEH/2/wrk_ex3.c}

Intern ist SEH eine Erweiterung der vom \ac{OS}-unterstützten Ausnahmen,
aber die Handler-Funktion ist \TT{\_except\_handler3} (für SEH3) oder \TT{\_except\_handler4} (für SEH4).

Der Code dieses Handlers ist MSVC-spezifisch und befindet sich in dessen Bibliotheken
oder in der msvcr*.dll.
Es ist wichtig zu wissen, dass SEH eine MSVC-spezifische Sache ist.

Andere Win32-Compiler bieten möglicherweise etwas völlig anderes an.

\myparagraph{SEH3}

SEH3 hat \TT{\_except\_handler3} als Handler-Funktion und erweitert die
\TT{\_EXCEPTION\_REGISTRATION}-Tabelle indem ein Zeiger zur \IT{Scope-Tabelle} und
der \IT{previous try level}-Variablen hinzugefügt wird.
SEH4 erweitert die \IT{Scope-Tabelle} um vier Werte für Schutz vor Speicherüberläufen.

Die \IT{Scope-Tabelle} ist eine Tabelle die aus Zeigern auf Filter und Handler-Code-Blöcken
für jede verschachtelte Ebene für \IT{try/except} besteht.

\input{OS/SEH/2/tikz}

Auch hier ist es wieder sehr wichtig zu verstehen, dass das \ac{OS} sich lediglich
um die \IT{prev/handle}-Felder kümmert und sonst nichts.

Es ist Aufgaben der \TT{\_except\_handler3}-Funktion die anderen Felder und die
\IT{Scope-Tabelle} zu lesen und zu entscheiden, welcher Handler wann aufgerufen
werden muss.

\myindex{Wine}
\myindex{ReactOS}
Der Quellcode der \TT{\_except\_handler3}-Funktion ist nicht offen.

Sanos OS, welches einen Win32-Kompatibilitäts-Layer hat, hat die gleiche Funktion
implementiert, welche ähnlich ist zu der unter Windows\footnote{\url{http://go.yurichev.com/17058}}.
Eine weitere Implementierung existiert in Wine\footnote{\href{http://go.yurichev.com/17059}{GitHub}}
und ReactOS\footnote{\url{http://go.yurichev.com/17060}}.

Wenn der \IT{Filter}-Zeiger NULL ist, ist der \IT{Handler}-Zeiger ein Zeiger auf
den \IT{finally}-Code-Block.

Während der Ausführung verändert sich der Wert des \IT{previous try level}, so dass
der \TT{\_except\_handler3} Information über den aktuellen Verschachtelungslevel
hat, um zu wissen welcher Eintrag der \IT{Scope-Tabelle} zu nutzen ist.

\myparagraph{SEH3: one try/except block example}

\lstinputlisting[style=customc]{OS/SEH/2/2.c}

\lstinputlisting[caption=MSVC 2003,style=customasmx86]{OS/SEH/2/2_SEH3.asm}

Hier ist zu sehen wie der SEH-Frame auf dem Stack aufgebaut ist.
Die \IT{Scope-Tabelle} befindet sich im \TT{CONST}-Segment, diese Felder werden
nicht verändert.
Eine interessante Sache ist es, wie die \IT{previous try level}-Variable sich geändert hat.
Der Wert zu Beginn ist \TT{0xFFFFFFFF} ($-1$).
Der Moment, in dem der Body der \TT{try}-Anweisung betreten wird, ist mit einer Anweisung
gekennzeichnet, die 0 in die Variable schreibt.
In dem Moment in dem der Body der \TT{try}-Anweisung geschlossen wird, wird der Wert $-1$
dorthin zurückgeschrieben.
Es sind ebenso die Adressen der Filter- und Handler-Codes zu sehen.

Wir können sehr einfach die Struktur des \IT{try/except}-Konstrukts in der Funktion erkennen.

Da der SEH-Setup-Code im Funktionsprolog von mehreren Funktionen geteilt werden kann,
fügt der Compiler manchmal einen Aufruf zur \TT{SEH\_prolog()}-Funktion in den Prolog ein,
welcher genau dieses tut.

Der SEH-Aufräumcode ist in der \TT{SEH\_epilog()}-Funktion.

Versuchen wir dieses Beispiel in \tracer{} laufen zu lassen:
\myindex{tracer}

\begin{lstlisting}
tracer.exe -l:2.exe --dump-seh
\end{lstlisting}

\begin{lstlisting}[caption=tracer.exe output]
EXCEPTION_ACCESS_VIOLATION at 2.exe!main+0x44 (0x401054) ExceptionInformation[0]=1
EAX=0x00000000 EBX=0x7efde000 ECX=0x0040cbc8 EDX=0x0008e3c8
ESI=0x00001db1 EDI=0x00000000 EBP=0x0018feac ESP=0x0018fe80
EIP=0x00401054
FLAGS=AF IF RF
* SEH frame at 0x18fe9c prev=0x18ff78 handler=0x401204 (2.exe!_except_handler3)
SEH3 frame. previous trylevel=0
scopetable entry[0]. previous try level=-1, filter=0x401070 (2.exe!main+0x60) handler=0x401088 (2.exe!main+0x78)
* SEH frame at 0x18ff78 prev=0x18ffc4 handler=0x401204 (2.exe!_except_handler3)
SEH3 frame. previous trylevel=0
scopetable entry[0]. previous try level=-1, filter=0x401531 (2.exe!mainCRTStartup+0x18d) handler=0x401545 (2.exe!mainCRTStartup+0x1a1)
* SEH frame at 0x18ffc4 prev=0x18ffe4 handler=0x771f71f5 (ntdll.dll!__except_handler4)
SEH4 frame. previous trylevel=0
SEH4 header:	GSCookieOffset=0xfffffffe GSCookieXOROffset=0x0
		EHCookieOffset=0xffffffcc EHCookieXOROffset=0x0
scopetable entry[0]. previous try level=-2, filter=0x771f74d0 (ntdll.dll!___safe_se_handler_table+0x20) handler=0x771f90eb ntdll.dll!_TppTerminateProcess@4+0x43)
* SEH frame at 0x18ffe4 prev=0xffffffff handler=0x77247428 (ntdll.dll!_FinalExceptionHandler@16)
\end{lstlisting}

Es ist zu erkennen, dass die SEH-Kette aus vier Handlern besteht.

\myindex{CRT}
Die ersten zwei sind in unserem Beispiel zu finden. Zwei?
Es wurde doch nur einer erstellt?!
Das stimmt, jedoch wurde ein weiterer in der \ac{CRT}-Funktion \TT{\_mainCRTStartup()}
erstellt und es scheint so, dass hier zumindest \ac{FPU}-Ausnahmen behandelt.
Der Quellcode kann in der MSVC-Installation gefunden werden: \TT{crt/src/winxfltr.c}.

Der dritte ist SEH4 in ntdll.dll und der vierte Handler ist nicht MSVC-spezifisch,
befindet sich in der ntdll.ll und hat einen selbsterklärenden Funktionsnamen.

Es ist zu erkennen, dass es drei Arten von Handlern in einer Kette gibt:

Einer ist in keiner Weise in Verbindung zu MVSC (der letzte) und zwei sind MSVC-spezifisch:
SEH3 und SEH4.

\myparagraph{SEH3: two try/except blocks example}

\lstinputlisting[style=customc]{OS/SEH/2/3.c}

Es existieren jetzt zwei \TT{try}-Blöcke.
Die \IT{Scope-Tabelle} hat jetzt zwei Einträge, einen für jeden Block.
\IT{Previous try level} verändert sich wenn die Ausführung einen \TT{try}-Block
betritt oder verlässt.

\lstinputlisting[caption=MSVC 2003,style=customasmx86]{OS/SEH/2/3_SEH3.asm}

Wenn ein Breakpoint auf die \printf{}-Funktion gesetzt wird, die vom Handler
aufgerufen wird, ist auch sichtbar, wie ein neuer SEH-Handler hinzugefügt wird.

Möglicherweise ist innerhalb des SEH Handling-Prozesses noch eine andere Funktion.
Es sind hier in der \IT{Scope-Tabelle} zwei Einträge zu sehen.

\begin{lstlisting}
tracer.exe -l:3.exe bpx=3.exe!printf --dump-seh
\end{lstlisting}

\begin{lstlisting}[caption=tracer.exe output]
(0) 3.exe!printf
EAX=0x0000001b EBX=0x00000000 ECX=0x0040cc58 EDX=0x0008e3c8
ESI=0x00000000 EDI=0x00000000 EBP=0x0018f840 ESP=0x0018f838
EIP=0x004011b6
FLAGS=PF ZF IF
* SEH frame at 0x18f88c prev=0x18fe9c handler=0x771db4ad (ntdll.dll!ExecuteHandler2@20+0x3a)
* SEH frame at 0x18fe9c prev=0x18ff78 handler=0x4012e0 (3.exe!_except_handler3)
SEH3 frame. previous trylevel=1
scopetable entry[0]. previous try level=-1, filter=0x401120 (3.exe!main+0xb0) handler=0x40113b (3.exe!main+0xcb)
scopetable entry[1]. previous try level=0, filter=0x4010e8 (3.exe!main+0x78) handler=0x401100 (3.exe!main+0x90)
* SEH frame at 0x18ff78 prev=0x18ffc4 handler=0x4012e0 (3.exe!_except_handler3)
SEH3 frame. previous trylevel=0
scopetable entry[0]. previous try level=-1, filter=0x40160d (3.exe!mainCRTStartup+0x18d) handler=0x401621 (3.exe!mainCRTStartup+0x1a1)
* SEH frame at 0x18ffc4 prev=0x18ffe4 handler=0x771f71f5 (ntdll.dll!__except_handler4)
SEH4 frame. previous trylevel=0
SEH4 header:	GSCookieOffset=0xfffffffe GSCookieXOROffset=0x0
		EHCookieOffset=0xffffffcc EHCookieXOROffset=0x0
scopetable entry[0]. previous try level=-2, filter=0x771f74d0 (ntdll.dll!___safe_se_handler_table+0x20) handler=0x771f90eb ntdll.dll!_TppTerminateProcess@4+0x43)
* SEH frame at 0x18ffe4 prev=0xffffffff handler=0x77247428 (ntdll.dll!_FinalExceptionHandler@16)
\end{lstlisting}

\myparagraph{SEH4}

\myindex{\BufferOverflow}
\myindex{Security cookie}
Bei einer Pufferüberlauf-Attacke (\myref{subsec:bufferoverflow}), kann die Adresse der \IT{Scope-Tabelle}
überschrieben werden. Aus diesem Grund wird seit MSVC 2005 SEH3 auf SEH4 aktualiiert um einen Schutz gegen
diese Attacken zu haben.
Der Zeiger auf die \IT{Scope-Tabelle} wird jetzt mit einem Security-Cookie \glslink{xoring}{xored}.

Jedes Element hat einen Offset innerhalb des Stacks mit einem anderen Wert:
die Adresse des \gls{stack frame} (\EBP) \glslink{xoring}{xored} mit dem Security-Cookie im Stack.

Dieser Wert wird während der Ausführung der Ausnahmebehandlung ausgelesen und auf
Korrektheit überprüft.
Das \IT{Security-Cookie} im Stack ist jedes Mal zufällig, so dass ein Angreifer den
Wert hoffentlich nicht voraussehen kann.

Der initiale \IT{previous try level} ist $-2$ in SEH4 anstatt $-1$.

\def\SEHfour{1}
\input{OS/SEH/2/tikz}

Hier sind beide Beispiele mit MSVC 2012 und SEH4 kompiliert:

\lstinputlisting[caption=MSVC 2012: one try block example,style=customasmx86]{OS/SEH/2/2_SEH4.asm}

\lstinputlisting[caption=MSVC 2012: two try blocks example,style=customasmx86]{OS/SEH/2/3_SEH4.asm}

Die Bedeutung des \IT{Cookies} ist wie folgt:
Der \TT{Cookie Offset} ist die Differenz zwischen der Adresse des gespeicherten EBP-Wertes
auf dem Stack und des $EBP \oplus security\_cookie$-Werts auf dem Stack.
Der  \TT{Cookie XOR Offset} ist eine zusätzliche Differenz zwischen dem $EBP \oplus security\_cookie$-Wert
und was auf dem Stack gespeichert ist.

Wenn diese Gleichung nicht richtig ist, wird der Prozess aufgrund eines korrupten Stack angehalte.

\begin{center}
$security\_cookie \oplus (CookieXOROffset + address\_of\_saved\_EBP) == stack[address\_of\_saved\_EBP + CookieOffset]$
\end{center}

Wenn der \TT{Cookie Offset} gleich $-2$ ist, impliziert dies, dass er nicht vorhanden ist.

\myindex{tracer}
\IT{Cookie}-Überprüfung ist auch in dem \tracer{} implementiert.
Siehe \href{http://go.yurichev.com/17061}{GitHub} für Details.

Es ist immer noch möglich, SEH3 im Compiler zu nutzen wenn eine neuere Version als MSVC 2005
genutzt wird, durch setzen der \TT{/GS-}-Option.
Der \ac{CRT}-Code nutzt SEH4 auf jeden Fall.
}
\FR{\subsubsection{Retour à MSVC}

\myindex{\Cpp!exceptions}

Microsoft a ajouté un mécanisme non standard de gestion d'exceptions à 
MSVC\footnote{\href{http://go.yurichev.com/17057}{MSDN}} essentiellement à l'usage des programmeurs C.
Ce mécanisme est totalement distinct de celui définit par le standard ISO du langage C++.

% FIXME russian listing:
\begin{lstlisting}[style=customc]
__try
{
    ...
}
__except(filter code)
{
    handler code
}
\end{lstlisting}

A la place du gestionnaire d'exception, on peut trouver un block \q{Finally}:

\begin{lstlisting}[style=customc]
__try
{
    ...
}
__finally
{
    ...
}
\end{lstlisting}

Le code de filtrage est une expression. L'évaluation de celle-ci permet de définir si le gestionnaire 
reconnaît l'exception qui a été déclenchée.

Si votre filtre est trop complexe pour tenir dans une seule expression, une fonction de filtrage 
séparée peut être définie.\\
\\
Il existe de nombreuses constructions de ce type dans le noyau Windows.
En voici quelques exemples (\ac{WRK}):

\lstinputlisting[caption=WRK-v1.2/base/ntos/ob/obwait.c,style=customc]{OS/SEH/2/wrk_ex1.c}

\lstinputlisting[caption=WRK-v1.2/base/ntos/cache/cachesub.c,style=customc]{OS/SEH/2/wrk_ex2.c}

Voici aussi un exemple de code de filtrage:

\lstinputlisting[caption=WRK-v1.2/base/ntos/cache/copysup.c,style=customc]{OS/SEH/2/wrk_ex3.c}

En interne, SEH est une extension du mécanisme de gestion des exceptions implémenté par l'OS.
La fonction de gestion d'exceptions est \TT{\_except\_handler3} (pour SEH3) ou 
\TT{\_except\_handler4} (pour SEH4).

Le code de ce gestionnaire est propre à MSVC et est situé dans ses librairies, ou dans msvcr*.dll.
Il est essentiel de comprendre que SEH est purement lié à MSVC.

D'autres compilateurs win32 peuvent choisir un modèle totalement différent.

\myparagraph{SEH3}

SEH3 est géré par la fonction \TT{\_except\_handler3}. Il ajoute à la structure \TT{\_EXCEPTION\_REGISTRATION} 
un pointeur vers une \IT{scope table} et une variable \IT{previous try level}.
SEH4 de son côté ajoute 4 valeurs à la structure \IT{scope table} pour la gestion des dépassements 
de buffer.\\
\\
La structure \IT{scope table} est un ensemble de pointeurs vers les blocs de code du filtre et du 
gestionnaire de chaque niveau \IT{try/except} imbriqué.

\input{OS/SEH/2/tikz}

Il est essentiel de comprendre que l'\ac{OS} ne se préoccupe que des champs \IT{prev/handle} et de 
rien d'autre.\\
Les autres champs sont exploités par la fonction \TT{\_except\_handler3}, de même que le contenu 
de la structure \IT{scope table} afin de décider quel gestionnaire exécuter et quand.\\
\\
\myindex{Wine}
\myindex{ReactOS}
Le code source de la fonction \TT{\_except\_handler3} n'est pas public.

Cependant, le système d'exploitation Sanos, possède un mode de compatibilité win32.
Celui-ci réimplémente les mêmes fonctions d'une manière quasi équivalente à celle de Windows
\footnote{\url{http://go.yurichev.com/17058}}.
On trouve une autre réimplementation dans Wine\footnote{\href{http://go.yurichev.com/17059}{GitHub}}
ainsi que dans ReactOS\footnote{\url{http://go.yurichev.com/17060}}.\\
\\
Lorsque le champ \IT{filter} est un pointeur NULL, le champ \IT{handler} est un pointeur vers un 
bloc de code \IT{finally}.\\
\\
Au cours de l'exécution, la valeur du champe \IT{previous try level} change. Ceci permet à la 
fonction \TT{\_except\_handler3} de connaître le niveau d'imbrication et donc de savoir quelle 
entrée de la table \IT{scope table} utiliser en cas d'exception.

\myparagraph{SEH3: exemple de bloc try/except}

\lstinputlisting[style=customc]{OS/SEH/2/2.c}

\lstinputlisting[caption=MSVC 2003,style=customasmx86]{OS/SEH/2/2_SEH3.asm}

Nous voyons ici la manière dont le bloc SEH est construit sur la pile.
La structure \IT{scope table} est présente dans le segment \TT{CONST} du programme--- ce qui est 
normal puisque son contenu n'a jamais besoin d'être changé.
Un point intéressant est la manière dont la valeur de la variable \IT{previous try level} évolue.
Sa valeur initiale est \TT{0xFFFFFFFF} ($-1$).
L'entrée dans le bloc \TT{try} débute par l'écriture de la valeur 0 dans la variable.
La sortie du bloc \TT{try} est marquée par la restauration de la valeur $-1$.
Nous voyons également l'adresse du bloc de filtrage et de celui du gestionnaire.

Nous pouvons donc observer facilement la présence de blocs \IT{try/except} dans la fonction.\\
\\
Le code d'initialisation des structures SEH dans le prologue de la fonction peut être partagé par 
de nombreuses fonctions. Le compilateur choisi donc parfois d'insérer dans le prologue d'une fonction 
un appel à la fonction \TT{SEH\_prolog()} qui assure cette initialisation.

Le code de nettoyage des structures SEH se trouve quant à lui dans la fonction \TT{SEH\_epilog()}.\\
\\
Tentons d'exécuter cet exemple dans \tracer{}:
\myindex{tracer}

\begin{lstlisting}
tracer.exe -l:2.exe --dump-seh
\end{lstlisting}

\begin{lstlisting}[caption=tracer.exe output]
EXCEPTION_ACCESS_VIOLATION at 2.exe!main+0x44 (0x401054) ExceptionInformation[0]=1
EAX=0x00000000 EBX=0x7efde000 ECX=0x0040cbc8 EDX=0x0008e3c8
ESI=0x00001db1 EDI=0x00000000 EBP=0x0018feac ESP=0x0018fe80
EIP=0x00401054
FLAGS=AF IF RF
* SEH frame at 0x18fe9c prev=0x18ff78 handler=0x401204 (2.exe!_except_handler3)
SEH3 frame. previous trylevel=0
scopetable entry[0]. previous try level=-1, filter=0x401070 (2.exe!main+0x60) handler=0x401088 (2.exe!main+0x78)
* SEH frame at 0x18ff78 prev=0x18ffc4 handler=0x401204 (2.exe!_except_handler3)
SEH3 frame. previous trylevel=0
scopetable entry[0]. previous try level=-1, filter=0x401531 (2.exe!mainCRTStartup+0x18d) handler=0x401545 (2.exe!mainCRTStartup+0x1a1)
* SEH frame at 0x18ffc4 prev=0x18ffe4 handler=0x771f71f5 (ntdll.dll!__except_handler4)
SEH4 frame. previous trylevel=0
SEH4 header:	GSCookieOffset=0xfffffffe GSCookieXOROffset=0x0
		EHCookieOffset=0xffffffcc EHCookieXOROffset=0x0
scopetable entry[0]. previous try level=-2, filter=0x771f74d0 (ntdll.dll!___safe_se_handler_table+0x20) handler=0x771f90eb (ntdll.dll!_TppTerminateProcess@4+0x43)
* SEH frame at 0x18ffe4 prev=0xffffffff handler=0x77247428 (ntdll.dll!_FinalExceptionHandler@16)
\end{lstlisting}

Nous constatons que la chaîne SEH est constituée de 4 gestionnaires.\\
\\
\myindex{CRT}
Le deux premiers sont situés dans le code de notre exemple. Deux? Mais nous n'en avons défini qu'un!
Effectivement, mais un second a été initialisé dans la fonction \TT{\_mainCRTStartup()} du \ac{CRT}.
Il semble que celui-ci gère au moins les exceptions \ac{FPU}.
Son code source figure dans le fichier \TT{crt/src/winxfltr.c} fournit avec l'installation de MSVC.\\
\\
Le troisième est le gestionnaire SEH4 dans ntdll.dll.
Le quatrième n'est pas lié à MSVC et se situe dans ntdll.dll. Son nom suffit à en décrire l'utilité.\\
\\
Comme vous le constatez, nous avons 3 types de gestionnaire dans la même chaîne:

L'un n'a rien à voir avec MSVC (le dernier) et deux autres sont liés à MSVC: SEH3 et SEH4.

\myparagraph{SEH3: exemple de deux blocs try/except}

\lstinputlisting[style=customc]{OS/SEH/2/3.c}

Nous avons maintenant deux blocs \TT{try}.
La structure \IT{scope table} possède donc deux entrées, une pour chaque bloc.
La valeur de \IT{Previous try level} change selon que l'exécution entre ou sort des blocs \TT{try}.

\lstinputlisting[caption=MSVC 2003,style=customasmx86]{OS/SEH/2/3_SEH3.asm}

Si nous positionnons un point d'arrêt sur la fonction \printf{} qui est appelée par le gestionnaire, 
nous pouvons constater comment un nouveau gestionnaire SEH est ajouté.

Il s'agit peut-être d'un autre mécanisme interne de la gestion SEH.
Nous constatons aussi que notre structure \IT{scope table} contient 2 entrées.

\begin{lstlisting}
tracer.exe -l:3.exe bpx=3.exe!printf --dump-seh
\end{lstlisting}

\begin{lstlisting}[caption=tracer.exe output]
(0) 3.exe!printf
EAX=0x0000001b EBX=0x00000000 ECX=0x0040cc58 EDX=0x0008e3c8
ESI=0x00000000 EDI=0x00000000 EBP=0x0018f840 ESP=0x0018f838
EIP=0x004011b6
FLAGS=PF ZF IF
* SEH frame at 0x18f88c prev=0x18fe9c handler=0x771db4ad (ntdll.dll!ExecuteHandler2@20+0x3a)
* SEH frame at 0x18fe9c prev=0x18ff78 handler=0x4012e0 (3.exe!_except_handler3)
SEH3 frame. previous trylevel=1
scopetable entry[0]. previous try level=-1, filter=0x401120 (3.exe!main+0xb0) handler=0x40113b (3.exe!main+0xcb)
scopetable entry[1]. previous try level=0, filter=0x4010e8 (3.exe!main+0x78) handler=0x401100 (3.exe!main+0x90)
* SEH frame at 0x18ff78 prev=0x18ffc4 handler=0x4012e0 (3.exe!_except_handler3)
SEH3 frame. previous trylevel=0
scopetable entry[0]. previous try level=-1, filter=0x40160d (3.exe!mainCRTStartup+0x18d) handler=0x401621 (3.exe!mainCRTStartup+0x1a1)
* SEH frame at 0x18ffc4 prev=0x18ffe4 handler=0x771f71f5 (ntdll.dll!__except_handler4)
SEH4 frame. previous trylevel=0
SEH4 header:	GSCookieOffset=0xfffffffe GSCookieXOROffset=0x0
		EHCookieOffset=0xffffffcc EHCookieXOROffset=0x0
scopetable entry[0]. previous try level=-2, filter=0x771f74d0 (ntdll.dll!___safe_se_handler_table+0x20) handler=0x771f90eb (ntdll.dll!_TppTerminateProcess@4+0x43)
* SEH frame at 0x18ffe4 prev=0xffffffff handler=0x77247428 (ntdll.dll!_FinalExceptionHandler@16)
\end{lstlisting}

\myparagraph{SEH4}

\myindex{\BufferOverflow}
\myindex{Security cookie}
Lors d'une attaque par dépassement de buffer (\myref{subsec:bufferoverflow}), l'adresse de la 
\IT{scope table} peut être modifiée. C'est pourquoi à partir de MSVC 2005 SEH3 a été amélioré vers 
SEH4 pour ajouter une protection contre ce type d'attaque.
Le pointeur vers la structure \IT{scope table} est désormais \glslink{xoring}{xored} avec la valeur 
d'un \gls{security cookie}.
Par ailleurs, la structure \IT{scope table} a été étendue avec une en-tête contenant deux pointeurs 
vers des \IT{security cookies}.

Chaque élément contient un offset dans la pile d'une valeur correspondant à: 
adresse du \gls{stack frame} (\EBP) \glslink{xoring}{xored} avec la valeur du \TT{security\_cookie} 
lui aussi situé sur la pile.

Durant la gestion d'exception, l'intégrité de cette valeur est vérifiée.
La valeur de chaque \IT{security cookie} situé sur la pile est aléatoire. Une attaque à distance ne 
pourra donc pas la deviner. \\
\\
Avec SEH4, la valeur initiale de \IT{previous try level} est de $-2$ et non de $-1$.

\def\SEHfour{1}
\input{OS/SEH/2/tikz}

Voici deux exemples compilés avec MSVC 2012 et SEH4:

\lstinputlisting[caption=MSVC 2012: exemple bloc try unique,style=customasmx86]{OS/SEH/2/2_SEH4.asm}

\lstinputlisting[caption=MSVC 2012: exemple de deux blocs try,style=customasmx86]{OS/SEH/2/3_SEH4.asm}

La signification de \IT{cookies} est la suivante: \TT{Cookie Offset} est la différence entre l'adresse 
dans la pile de la dernière valeur sauvegardée du registre EBP et de l'adresse dans la pile du 
résultat de l'addition $EBP \oplus security\_cookie$.
\TT{Cookie XOR Offset} est quant à lui la différence entre $EBP \oplus security\_cookie$ et la valeur 
conservée sur la pile.

Si le prédicat ci-dessous n'est pas respecté, le processus est arrêté du fait d'une corruption 
de la pile:

\begin{center}
$security\_cookie \oplus (CookieXOROffset + address\_of\_saved\_EBP) == stack[address\_of\_saved\_EBP + CookieOffset]$
\end{center}

Lorsque \TT{Cookie Offset} vaut $-2$, ceci indique qu'il n'est pas présent.

\myindex{tracer}
La vérification des \IT{Cookies} est aussi implémentée dans mon \tracer{},
voir \href{http://go.yurichev.com/17061}{GitHub} pour les détails.\\
\\
Pour les versions à partir de MSVC 2005, il est toujours possible de revenir à la version SEH3 
en utilisant l'option \TT{/GS-}. Toutefois, le \ac{CRT} continue à utiliser SEH4.

}

\EN{\subsubsection{Windows x64}

\label{SEH_win64}

As you might think, it is not very fast to set up the SEH frame at each function prologue.
Another performance problem is changing the 
\IT{previous try level} value many times during the function's execution.

So things are changed completely in x64: now all pointers to \TT{try} blocks, filter and handler functions are stored
in another PE segment \TT{.pdata}, 
and from there the \ac{OS}'s exception handler takes all the information.

Here are the two examples from the previous section compiled for x64:

\lstinputlisting[caption=MSVC 2012,style=customasmx86]{OS/SEH/3/2_x64.asm}

\lstinputlisting[caption=MSVC 2012,style=customasmx86]{OS/SEH/3/3_x64.asm}

Read \IgorSkochinsky for more detailed information about this.

Aside from exception information, \TT{.pdata}
is a section that contains the addresses of almost all function starts and ends,
hence it may be useful for a tools targeted at automated analysis.

}
\RU{\subsubsection{Windows x64}

\label{SEH_win64}
Как видно, это не самая быстрая штука, устанавливать SEH-структуры в каждом прологе функции.
Еще одна проблема производительности --- это менять переменную 
\IT{previous try level} много раз в течении исполнении функции.
Так что в x64 всё сильно изменилось, теперь все указатели на \TT{try}-блоки, функции фильтров и обработчиков,
теперь записаны в другом PE-сегменте
 \TT{.pdata}, откуда обработчик исключений \ac{OS} берет всю информацию.

Вот два примера из предыдущей секции, скомпилированных для x64:

\lstinputlisting[caption=MSVC 2012,style=customasmx86]{OS/SEH/3/2_x64.asm}

\lstinputlisting[caption=MSVC 2012,style=customasmx86]{OS/SEH/3/3_x64.asm}

Смотрите \IgorSkochinsky для более детального описания.

Помимо информации об исключениях, секция \TT{.pdata} 
также содержит начала и концы почти всех функций, так что эту информацию можно использовать в каких-либо
утилитах, предназначенных для автоматизации анализа.

}
\DE{\subsubsection{Windows x64}

\label{SEH_win64}

Wie man sich vielleicht denken kann, ist es nicht sehr schnell bei jedem Funktionsprolog
einen SEH-Frame aufzubauen.
Ein weiteres Geschwindigkeitsproblem ist das häufige Ändern des \IT{previous try level}-Werts
während der Ausführung einer Funktion.

Also haben sich die Dinge in x64 komplett geändert: alle Zeiger auf einen \TT{try}-Block,
Filter und Handler-Funktionen sind im einem PE-Segment \TT{.pdata} gesichert.
Von hier nimmt die \ac{OS}-Ausnahmebehandlung alle Informationen.

Hier sind zwei Beispiele aus dem letzten Abschnitt, für x64 kompiliert:

\lstinputlisting[caption=MSVC 2012,style=customasmx86]{OS/SEH/3/2_x64.asm}

\lstinputlisting[caption=MSVC 2012,style=customasmx86]{OS/SEH/3/3_x64.asm}

In \IgorSkochinsky gibt es eine Reihe weiterer, detaillierte Information über dieses Thema.

Neben den Ausnahme-Informationen, beinhaltet \TT{.pdata} die Adressen von fast allen
Funktionsbeginn- und enden, da dies für Tools die für automatische Analysen nützlich
sein kann.
}
\FR{\subsubsection{Windows x64}

\label{SEH_win64}

Vous imaginez bien qu'il n'est pas très performant de construire le contexte SEH dans le prologue 
de chaque fonction. S'y ajoute les nombreux changements de la valeur de \IT{previous try level} 
durant l'exécution de la fonction.

C'est pourquoi avec x64, la manière de faire a complèetement changé. Tous les pointeurs vers les 
blocs \TT{try}, les filtres et les gestionnaires sont désormais stockés dans une nouveau segment 
PE: \TT{.pdata} à partir duquel les gestionnaires d'exception de l'\ac{OS} récupèreront les 
informations.

Voici deux exemples tirés de la section précédente et compilés pour x64:

\lstinputlisting[caption=MSVC 2012,style=customasmx86]{OS/SEH/3/2_x64.asm}

\lstinputlisting[caption=MSVC 2012,style=customasmx86]{OS/SEH/3/3_x64.asm}

Pour plus d'informations sur le sujet, lisez \IgorSkochinsky.

Hormis les informations d'exception, \TT{.pdata} est aussi une section qui contient les adresses 
de début et de fin de toutes les fonctions. Elle revêt donc un intérêt particulier dans le cadre 
d'une analyse automatique d'un programme.

}

\subsubsection{
\RU{Больше о}
\EN{Read more about}
\DE{Mehr über}
\FR{En lire plus sur} SEH
}

\PietrekSEH, \IgorSkochinsky.


\EN{\subsection{Windows NT: Critical section}
\myindex{Windows}

\label{critical_sections}


Critical sections in any \ac{OS} are very important in multithreaded environment,
mostly for giving a guarantee
that only one thread can access some data in a single moment of time, 
while blocking other threads and interrupts.

\par
That is how a \TT{CRITICAL\_SECTION} 
structure is declared in \gls{Windows NT} line OS:

\begin{lstlisting}[caption=(Windows Research Kernel v1.2) public/sdk/inc/nturtl.h,style=customc]
typedef struct _RTL_CRITICAL_SECTION {
    PRTL_CRITICAL_SECTION_DEBUG DebugInfo;

    //
    //  The following three fields control entering and exiting the critical
    //  section for the resource
    //

    LONG LockCount;
    LONG RecursionCount;
    HANDLE OwningThread;        // from the thread's ClientId->UniqueThread
    HANDLE LockSemaphore;
    ULONG_PTR SpinCount;        // force size on 64-bit systems when packed
} RTL_CRITICAL_SECTION, *PRTL_CRITICAL_SECTION;
\end{lstlisting}

That's is how EnterCriticalSection() function works:

\myindex{x86!\Instructions!LOCK}
\begin{lstlisting}[caption=Windows 2008/ntdll.dll/x86 (begin),style=customasmx86]
_RtlEnterCriticalSection@4

var_C           = dword ptr -0Ch
var_8           = dword ptr -8
var_4           = dword ptr -4
arg_0           = dword ptr  8

                mov     edi, edi
                push    ebp
                mov     ebp, esp
                sub     esp, 0Ch
                push    esi
                push    edi
                mov     edi, [ebp+arg_0]
                lea     esi, [edi+4] ; LockCount
                mov     eax, esi
                lock btr dword ptr [eax], 0
                jnb     wait ; jump if CF=0

loc_7DE922DD:
                mov     eax, large fs:18h
                mov     ecx, [eax+24h]
                mov     [edi+0Ch], ecx
                mov     dword ptr [edi+8], 1
                pop     edi
                xor     eax, eax
                pop     esi
                mov     esp, ebp
                pop     ebp
                retn    4

... skipped
\end{lstlisting}

\myindex{x86!\Instructions!BTR}
\myindex{x86!\Prefixes!LOCK}

The most important instruction in this code fragment is \TT{BTR} 
(prefixed with \TT{LOCK}): 

the zeroth bit is stored in the CF flag and cleared in memory.
This is an \gls{atomic operation}, 

blocking all other CPUs' access to this piece of memory 
(see the \TT{LOCK} prefix before the \TT{BTR} instruction).
If the bit at \TT{LockCount} is 1, 

fine, reset it and return from the function: we are in a critical section.

If not---the critical section is already occupied by other thread, so wait. \\
The wait is performed there using WaitForSingleObject(). \\
\\
And here is how the LeaveCriticalSection() function works:

\begin{lstlisting}[caption=Windows 2008/ntdll.dll/x86 (begin),style=customasmx86]
_RtlLeaveCriticalSection@4 proc near

arg_0           = dword ptr  8

                mov     edi, edi
                push    ebp
                mov     ebp, esp
                push    esi
                mov     esi, [ebp+arg_0]
                add     dword ptr [esi+8], 0FFFFFFFFh ; RecursionCount
                jnz     short loc_7DE922B2
                push    ebx
                push    edi
                lea     edi, [esi+4]    ; LockCount
                mov     dword ptr [esi+0Ch], 0
                mov     ebx, 1
                mov     eax, edi
                lock xadd [eax], ebx
                inc     ebx
                cmp     ebx, 0FFFFFFFFh
                jnz     loc_7DEA8EB7

loc_7DE922B0:
                pop     edi
                pop     ebx

loc_7DE922B2:
                xor     eax, eax
                pop     esi
                pop     ebp
                retn    4

... skipped
\end{lstlisting}

\myindex{x86!\Instructions!XADD}
\TT{XADD} is \q{exchange and add}.

In this case, it adds 1 to \TT{LockCount}, meanwhile saves initial value of \TT{LockCount} in the \TT{EBX} register.
However, value in \TT{EBX} is to incremented with a help of subsequent \INS{INC EBX}, and it also will be equal to
the updated value of \TT{LockCount}.

This operation is atomic since it is prefixed by \TT{LOCK} as well,
meaning that all other CPUs or CPU cores in system are blocked from accessing this point in memory.

The \TT{LOCK} prefix is very important: 

without it two threads, each of which works on separate CPU or CPU core can try to
enter a critical section and to modify the value in memory,
which will result in non-deterministic behavior.

% TODO linux

}
\RU{\subsection{Windows NT: Критические секции}
\myindex{Windows}

\label{critical_sections}

Критические секции в любой \ac{OS} очень важны в мультитредовой среде, используются в основном
для обеспечения гарантии что только один тред будет иметь доступ к данным в один момент времени,
блокируя остальные треды и прерывания.


\par
Вот как объявлена структура \TT{CRITICAL\_SECTION} 
объявлена в линейке OS \gls{Windows NT}:

\begin{lstlisting}[caption=(Windows Research Kernel v1.2) public/sdk/inc/nturtl.h,style=customc]
typedef struct _RTL_CRITICAL_SECTION {
    PRTL_CRITICAL_SECTION_DEBUG DebugInfo;

    //
    //  The following three fields control entering and exiting the critical
    //  section for the resource
    //

    LONG LockCount;
    LONG RecursionCount;
    HANDLE OwningThread;        // from the thread's ClientId->UniqueThread
    HANDLE LockSemaphore;
    ULONG_PTR SpinCount;        // force size on 64-bit systems when packed
} RTL_CRITICAL_SECTION, *PRTL_CRITICAL_SECTION;
\end{lstlisting}

Вот как работает функция EnterCriticalSection():

\myindex{x86!\Instructions!LOCK}
\begin{lstlisting}[caption=Windows 2008/ntdll.dll/x86 (begin),style=customasmx86]
_RtlEnterCriticalSection@4

var_C           = dword ptr -0Ch
var_8           = dword ptr -8
var_4           = dword ptr -4
arg_0           = dword ptr  8

                mov     edi, edi
                push    ebp
                mov     ebp, esp
                sub     esp, 0Ch
                push    esi
                push    edi
                mov     edi, [ebp+arg_0]
                lea     esi, [edi+4] ; LockCount
                mov     eax, esi
                lock btr dword ptr [eax], 0
                jnb     wait ; jump if CF=0

loc_7DE922DD:
                mov     eax, large fs:18h
                mov     ecx, [eax+24h]
                mov     [edi+0Ch], ecx
                mov     dword ptr [edi+8], 1
                pop     edi
                xor     eax, eax
                pop     esi
                mov     esp, ebp
                pop     ebp
                retn    4

... skipped
\end{lstlisting}

\myindex{x86!\Instructions!BTR}
\myindex{x86!\Prefixes!LOCK}
Самая важная инструкция в этом фрагменте кода --- это
 \TT{BTR} 
(с префиксом \TT{LOCK}): 
нулевой бит сохраняется в флаге CF и очищается в памяти
.
Это \glslink{atomic operation}{атомарная операция}, 
блокирующая доступ всех остальных процессоров
к этому значению в памяти (обратите внимание на префикс \TT{LOCK} перед инструкцией \TT{BTR}.

Если бит в \TT{LockCount} является 1, 
хорошо, сбросить его и вернуться из функции: мы в критической секции
.
Если нет --- критическая секция уже занята другим тредом, тогда ждем
. \\
Ожидание там сделано через вызов WaitForSingleObject(). \\
\\
А вот как работает функция LeaveCriticalSection():

\begin{lstlisting}[caption=Windows 2008/ntdll.dll/x86 (begin),style=customasmx86]
_RtlLeaveCriticalSection@4 proc near

arg_0           = dword ptr  8

                mov     edi, edi
                push    ebp
                mov     ebp, esp
                push    esi
                mov     esi, [ebp+arg_0]
                add     dword ptr [esi+8], 0FFFFFFFFh ; RecursionCount
                jnz     short loc_7DE922B2
                push    ebx
                push    edi
                lea     edi, [esi+4]    ; LockCount
                mov     dword ptr [esi+0Ch], 0
                mov     ebx, 1
                mov     eax, edi
                lock xadd [eax], ebx
                inc     ebx
                cmp     ebx, 0FFFFFFFFh
                jnz     loc_7DEA8EB7

loc_7DE922B0:
                pop     edi
                pop     ebx

loc_7DE922B2:
                xor     eax, eax
                pop     esi
                pop     ebp
                retn    4

... skipped
\end{lstlisting}

\myindex{x86!\Instructions!XADD}
\TT{XADD} это \q{обменять и прибавить}.
В данном случае, это значит прибавить 1 к значению в \TT{LockCount}, при этом сохранить изначальное значение \TT{LockCount}
в регистре \TT{EBX}.
Впрочем, значение в \TT{EBX} позже инкрементируется при помощи последующей инструкции \INS{INC EBX},
и оно также будет равно обновленному значению \TT{LockCount}.

Эта операция также атомарная, потому что также имеет префикс \TT{LOCK}, что означает, что другие CPU
или ядра CPU в системе не будут иметь доступа к этой ячейке памяти
.

Префикс \TT{LOCK} очень важен: 
два треда, каждый из которых работает на разных CPU или ядрах CPU, могут попытаться одновременно
войти в критическую секцию, одновременно модифицируя значение в памяти, и это может привести к
непредсказуемым результатам.


% TODO linux

}
\DE{\subsection{Windows NT: Kritischer Abschnitt}
\myindex{Windows}

\label{critical_sections}

Kritische Abschnitte sind in jedem \ac{OS} sehr wichtig bei Multithread-Umgebungen.
Der Zweck besteht darin, einen exklusiven Zugriff auf eine Ressource zu garantieren,
während andere Threads oder Interrupts blockiert sind.

\par
Nachfolgend, wie eine \TT{CRITICAL\_SECTION}-Struktur unter \gls{Windows NT} deklariert wird:

\begin{lstlisting}[caption=(Windows Research Kernel v1.2) public/sdk/inc/nturtl.h,style=customc]
typedef struct _RTL_CRITICAL_SECTION {
    PRTL_CRITICAL_SECTION_DEBUG DebugInfo;

    //
    //  The following three fields control entering and exiting the critical
    //  section for the resource
    //

    LONG LockCount;
    LONG RecursionCount;
    HANDLE OwningThread;        // from the thread's ClientId->UniqueThread
    HANDLE LockSemaphore;
    ULONG_PTR SpinCount;        // force size on 64-bit systems when packed
} RTL_CRITICAL_SECTION, *PRTL_CRITICAL_SECTION;
\end{lstlisting}

Nachfolgend wird gezeigt, wie die Funktion EnterCriticalSection() funktioniert:

\myindex{x86!\Instructions!LOCK}
\begin{lstlisting}[caption=Windows 2008/ntdll.dll/x86 (begin),style=customasmx86]
_RtlEnterCriticalSection@4

var_C           = dword ptr -0Ch
var_8           = dword ptr -8
var_4           = dword ptr -4
arg_0           = dword ptr  8

                mov     edi, edi
                push    ebp
                mov     ebp, esp
                sub     esp, 0Ch
                push    esi
                push    edi
                mov     edi, [ebp+arg_0]
                lea     esi, [edi+4] ; LockCount
                mov     eax, esi
                lock btr dword ptr [eax], 0
                jnb     wait ; jump if CF=0

loc_7DE922DD:
                mov     eax, large fs:18h
                mov     ecx, [eax+24h]
                mov     [edi+0Ch], ecx
                mov     dword ptr [edi+8], 1
                pop     edi
                xor     eax, eax
                pop     esi
                mov     esp, ebp
                pop     ebp
                retn    4

... und so weiter
\end{lstlisting}

\myindex{x86!\Instructions!BTR}
\myindex{x86!\Prefixes!LOCK}

Die wichtigste Funktion in diesem Code-Fragment ist \TT{BTR} (nach dem vorangehenden \TT{LOCK}):

Das nullte Bit wird im CF-Flag gesichert und im Speicher zurückgesetzt.
Dies ist eine \gls{atomic operation} und blockiert alle Zugriffe der CPU auf diesen
Teil des Speichert (siehe \TT{LOCK} vor der \TT{BTR}-Anweisung).
Wenn das Bit in \TT{LockCount} 1 ist, wird es zurückgesetzt und von der Funktion
zurückgekehrt: die CPU befindet sich nun um Kritischen Abschnitt.

Wenn nicht, wurde der Kritische Abschnitt bereits von einem anderen Thread betreten,
also muss gewartet werden.
Das Warten wird durch die Funktion WaitForSingleObject() realisiert.

Hier nun, wie die Funktion LeaveCriticalSection() funktioniert:

\begin{lstlisting}[caption=Windows 2008/ntdll.dll/x86 (begin),style=customasmx86]
_RtlLeaveCriticalSection@4 proc near

arg_0           = dword ptr  8

                mov     edi, edi
                push    ebp
                mov     ebp, esp
                push    esi
                mov     esi, [ebp+arg_0]
                add     dword ptr [esi+8], 0FFFFFFFFh ; RecursionCount
                jnz     short loc_7DE922B2
                push    ebx
                push    edi
                lea     edi, [esi+4]    ; LockCount
                mov     dword ptr [esi+0Ch], 0
                mov     ebx, 1
                mov     eax, edi
                lock xadd [eax], ebx
                inc     ebx
                cmp     ebx, 0FFFFFFFFh
                jnz     loc_7DEA8EB7

loc_7DE922B0:
                pop     edi
                pop     ebx

loc_7DE922B2:
                xor     eax, eax
                pop     esi
                pop     ebp
                retn    4

... und so weiter
\end{lstlisting}

\myindex{x86!\Instructions!XADD}
\TT{XADD} bedeutet \q{exchange and add}.

In diesem Fall wird 1 zu \TT{LockCount} addiert, während der ursprüngliche Wert von \TT{LockCount}
im \TT{EBX}-Register gesichert wird.
Der Wert in \TT{EBX} wird durch aufeinander folgende \INS{INC EBX}-Anweisungen inkrementiert und
wird damit gleich dem aktualisierten Wert von \TT{LockCount}.

Diese Operation ist atomar, da sie ebenfalls mit \TT{LOCK} eingeleitet wird und so alle anderen
CPUs oder CPU-Kerne des Systems für den Zugriff auf diesen Speicherbereich blockiert werden.

Das vorangehende \TT{LOCK} ist sehr wichtig:

ohne diese Anweisung können zwei Threads die auf unterschiedlichen CPUs oder CPU-Kernen laufen,
versuchen den Kritischen Abschnitt zu betreten und den Wert im Speicher zu verändern.
Diese kann zu einem nicht-deterministischen Verhalten führen.

% TODO linux
}

\part{\RU{Инструменты}\EN{Tools}}

\chapter{\RU{Дизассемблер}\EN{Disassembler}}

\section{IDA}

\label{IDA}
\RU{Старая бесплатная версия доступна для скачивания}\EN{Older freeware version is available for downloading}
\footnote{\url{http://www.hex-rays.com/idapro/idadownfreeware.htm}}.

\ShortHotKeyCheatsheet: \ref{sec:IDA_cheatsheet}

\chapter{\RU{Отладчик}\EN{Debugger}}

\section{tracer}

\index{tracer}
\label{tracer}
\RU{Я использую}\EN{I use} \IT{tracer}\footnote{\url{http://yurichev.com/tracer-\LANG.html}}
\RU{вместо отладчика}\EN{instead of debugger}.

\RU{Со временем я отказался использовать отладчик, потому что все что мне нужно от него: это иногда подсмотреть 
какие-либо аргументы какой-либо функции во время исполнения или состояние регистров в определенном месте. 
Каждый раз загружать отладчик для этого это слишком, поэтому я написал очень простую утилиту \IT{tracer}. 
Она консольная, запускается из командной строки, позволяет перехватывать исполнение функций, 
ставить брякпойнты на произвольные места, смотреть состояние регистров, модифицировать их, и так далее.}
\EN{I stopped to use debugger eventually, since all I need from it is to spot a function's arguments while
execution, or registers' state at some point.
To load debugger each time is too much, so I wrote a small utility \IT{tracer}.
It has console-interface, working from command-line, enable us to intercept function execution,
set breakpoints at arbitrary places, spot registers' state, modify it, etc.}

\RU{Но для учебы, очень полезно трассировать код руками в отладчике, наблюдать как меняются значения регистров 
(например, как минимум классический SoftICE, OllyDbg, WinDbg подсвечивают измененные регистры), 
флагов, данные, менять их самому, смотреть реакцию, и т.д.}
\EN{However, as for learning purposes, it is highly advisable to trace code in debugger manually, watch how register's state
changing (e.g. classic SoftICE, OllyDbg, WinDbg highlighting changed registers), flags, data, change them
manually, watch reaction, etc.}

\section{\olly}
\index{\olly}

\RU{Очень популярный отладчик пользовательской среды win32}\EN{Very popular user-mode win32 debugger}:\\
\url{http://www.ollydbg.de/}.

\ShortHotKeyCheatsheet: \label{sec:Olly_cheatsheet}

\section{GDB}
\index{GDB}

\RU{Не очень популярный отладчик у реверсеров, тем не менее, крайне удобный}\EN{Not very popular
debugger among reverse engineers, but very comfortable nevertheless}.
\RU{Некоторые команды}\EN{Some commands}: \ref{sec:GDB_cheatsheet}.

\chapter{\RU{Трассировка системных вызовов}\EN{System calls tracing}}

\label{strace}
\index{strace}
\index{dtruss}
\subsection{strace / dtruss}

\index{syscalls}
\RU{Позволяет показать, какие системные вызовы (syscalls(\ref{syscalls})) прямо сейчас вызывает процесс}
\EN{Will show which system calls (syscalls(\ref{syscalls})) are called by process right now}.
\RU{Например}\EN{For example}:

\begin{lstlisting}
# strace df -h

...

access("/etc/ld.so.nohwcap", F_OK)      = -1 ENOENT (No such file or directory)
open("/lib/i386-linux-gnu/libc.so.6", O_RDONLY|O_CLOEXEC) = 3
read(3, "\177ELF\1\1\1\0\0\0\0\0\0\0\0\0\3\0\3\0\1\0\0\0\220\232\1\0004\0\0\0"..., 512) = 512
fstat64(3, {st_mode=S_IFREG|0755, st_size=1770984, ...}) = 0
mmap2(NULL, 1780508, PROT_READ|PROT_EXEC, MAP_PRIVATE|MAP_DENYWRITE, 3, 0) = 0xb75b3000
\end{lstlisting}

\index{\MacOSX}
\RU{В \MacOSX для этого же имеется dtruss}\EN{\MacOSX has dtruss for the same aim}.

\index{cygwin}
\RU{В Cygwin также есть strace, впрочем, если я верно понял, 
он показывает результаты только для .exe-файлов скомпилированных для среды самого cygwin}
\EN{The Cygwin also has strace, but if I understood correctly, it works only for .exe-files
compiled for cygwin environment itself}.

\chapter{\RU{Декомпиляторы}\EN{Decompilers}}

\RU{Пока существует только один, публично доступный, декомпилятор в Си высокого качества}
\EN{There are only one known, publically available, high-quality decompiler to C code}: Hex-Rays:\\
\url{https://www.hex-rays.com/products/decompiler/}

% TODO Java, .NET, VB, etc

\chapter{\RU{Прочие инструменты}\EN{Other tools}}

\begin{itemize}
\item
Microsoft Visual Studio Express\footnote{\url{http://www.microsoft.com/express/Downloads/}}:
\RU{Усеченная бесплатная версия Visual Studio, пригодная для простых экспериментов}
\EN{Stripped-down free Visual Studio version, convenient for simple experiments}.
Some useful options: \ref{sec:MSVC_options}.

\item
\label{Hiew}
Hiew\footnote{\url{http://www.hiew.ru/}} \RU{для мелкой модификации кода в исполняемых файлах}
\EN{for small modifications of code in binary files}.

\item
\index{binary grep}
binary grep: \RU{небольшая утилита для поиска констант (либо просто последовательности байт)
в большом  кол-ве файлов, включая неисполняемые: \BGREPURL.}
\EN{the small utility for constants searching (or just any byte sequence) in a big pile of files, 
including non-executable: \BGREPURL.}
\end{itemize}


\part{\RU{Еще примеры}\EN{More examples}}

% chapters here
\chapter{\EN{Task manager practical joke}\RU{Шутка с task manager} (Windows Vista)}

\RU{У меня только 4 ядра в процессоре в компьютере, так что Task Manager в Windows показывает только 4
графика загрузки процессора.}
\EN{I have only 4 CPU cores on my computer, so the Windows Task Manager shows only 4 CPU load graphs.}

\RU{Посмотрим, сможем ли мы немного хакнуть Task Manager, чтобы он находил больше ядер в компьютере.}
\EN{Let's see if it's possible to hack Task Manager slightly so it would detect more CPU cores on a computer.}

\RU{В начале задумаемся, откуда Task Manager знает количество ядер?}
\EN{Let us first think, how Task Manager would know number of cores?}
\RU{В win32 имеется ф-ция \TT{GetSystemInfo()}, при помощи которой можно узнать.}
\EN{There are \TT{GetSystemInfo()} win32 function present in win32 userspace which can tell us this.}
\RU{Но она не импортируется в}\EN{But it's not imported in} \TT{taskmgr.exe}.
\RU{Есть еще одна в \gls{NTAPI}, \TT{NtQuerySystemInformation()}, которая используется в 
\TT{taskmgr.exe} в ряде мест.}
\EN{There are, however, another one in \gls{NTAPI}, \TT{NtQuerySystemInformation()}, 
which is used in \TT{taskmgr.exe} in several places.}
\RU{Чтобы узнать количество ядер, нужно вызвать эту ф-цию с константной \TT{SystemBasicInformation} в 
первом аргументе (а это ноль}
\EN{To get number of cores, one should call this function with \TT{SystemBasicInformation} constant 
in first argument (which is zero}
\footnote{MSDN: \url{http://msdn.microsoft.com/en-us/library/windows/desktop/ms724509(v=vs.85).aspx}}).

\RU{Второй аргумент должен указывать на буфер, который примет всю информацию.}
\EN{Second argument should point to the buffer, which will receive all the information.}

\RU{Так что нам нужно найти все вызовы ф-ции \TT{NtQuerySystemInformation(0, ?, ?, ?)}.}
\EN{So we need to find all calls to the \TT{NtQuerySystemInformation(0, ?, ?, ?)} function.}
\RU{Откроем}\EN{Let's open} \TT{taskmgr.exe} \InENRU IDA. 
\RU{Что всегда хорошо с исполняемыми файлами от Microsoft, это то что IDA может скачать соответствующуий 
\gls{PDB}-файл именно для этого файла и добавить все имена ф-ций.}
\EN{What is always good about Microsoft executables is that IDA can download corresponding \gls{PDB} 
file for exactly this executable and add all function names.}
\RU{Видимо, Task Manager написан на C++ и некоторые ф-ции и классы имеют говорящие за себя имена.}
\EN{It seems, Task Manager written in C++ and some of function names and classes are really 
speaking for themselves.}
\RU{Тут есть классы}\EN{There are classes} CAdapter, CNetPage, CPerfPage, CProcInfo, CProcPage, CSvcPage, 
CTaskPage, CUserPage.
\RU{Должно быть, каждый класс соответствует каждой вкладке в Task Manager.}
\EN{Apparently, each class corresponding each tab in Task Manager.}

\RU{Я прошелся по всем вызовам и добавил комментарий с числом, передающимся как первый аргумент.}
\EN{I visited each call and I add comment with a value which is passed as the first function argument.}
\RU{В некоторых местах я написал ``not zero'', потому что значение в тех местах однозначно не ноль, 
но что-то другое (больше об этом во второй части главы).}
\EN{There are ``not zero'' I wrote at some places, because, the value there was not clearly zero, 
but something really different (more about this in the second part of this chapter).}
\RU{А мы все-таки ищем ноль передаваемый как аргумент.}
\EN{And we are looking for zero passed as argument after all.}

\begin{figure}[H]
\centering
\includegraphics[scale=\FigScale]{examples/taskmgr/IDA_xrefs.png}
\caption{IDA: \RU{вызовы ф-ции}\EN{cross references to} NtQuerySystemInformation()}
\end{figure}

\RU{Да, имена действительно говорящие сами за себя.}
\EN{Yes, the names are really speaking for themselves.}

\RU{Когда я внимательно изучил каждое место, где вызывается \TT{NtQuerySystemInformation(0, ?, ?, ?)},
я быстро нашел то что нужно в ф-ции \TT{InitPerfInfo()}:}
\EN{When I closely investigating each place where \TT{NtQuerySystemInformation(0, ?, ?, ?)} is called,
I quickly found what I need in the \TT{InitPerfInfo()} function:}

\lstinputlisting[caption=taskmgr.exe (Windows Vista)]{examples/taskmgr/taskmgr.lst}

\TT{g\_cProcessors} \RU{это глобальная переменная и это имя присвоено IDA в соответствии с \gls{PDB}-файлом,
скачанным с сервера символов Microsoft}\EN{is a global variable, and this name was assigned by 
IDA according to \gls{PDB} loaded from the Microsoft symbol server}.

\RU{Байт берется из}\EN{The byte is taken from} \TT{var\_C20}. 
\RU{И}\EN{And} \TT{var\_C58} \RU{передается в}\EN{is passed to} \TT{NtQuerySystemInformation()} 
\RU{как указатель на принимающий буфер}\EN{as a pointer to the receiving buffer}.
\RU{Разница между}\EN{The difference between} 0xC20 \AndENRU 0xC58 \RU{это}\EN{is} 0x38 (56).
\RU{Посмотрим на формат структуры, который можно найти в MSDN:}
\EN{Let's take a look at returning structure format, which we can find in MSDN:}

\begin{lstlisting}
typedef struct _SYSTEM_BASIC_INFORMATION {
    BYTE Reserved1[24];
    PVOID Reserved2[4];
    CCHAR NumberOfProcessors;
} SYSTEM_BASIC_INFORMATION;
\end{lstlisting}

\RU{Это система x64, так что каждый PVOID занимает здесь 8 байт.}
\EN{This is x64 system, so each PVOID takes 8 byte here.}
\RU{Так что все \IT{reserved}-поля занимают $24+4*8=56$.}
\EN{So all \IT{reserved} fields in the structure takes $24+4*8=56$.}
\RU{О да, это значит что }\EN{Oh yes, this means, }\TT{var\_C20} \RU{в локальном стеке это именно поле}\EN{is the 
local stack is exactly} \TT{NumberOfProcessors} \RU{структуры}\EN{field of the} 
\TT{SYSTEM\_BASIC\_INFORMATION}\EN{ structure}.

\RU{Проверим, прав ли я}\EN{Let's check if I'm right}.
\RU{Скопируем}\EN{Copy} \TT{taskmgr.exe} \RU{из}\EN{from} \TT{C:\textbackslash{}Windows\textbackslash{}System32} 
\RU{в какую-нибудь другую папку}\EN{to some other folder} 
(\RU{чтобы}\EN{so the} \IT{Windows Resource Protection} \RU{не пыталась восстанавливать измененный}\EN{will not 
try to restore patched} \TT{taskmgr.exe}).

\RU{Откроем его в Hiew и найдем это место:}
\EN{Let's open it in Hiew and find the place:}

\begin{figure}[H]
\centering
\includegraphics[scale=\FigScale]{examples/taskmgr/hiew1.png}
\caption{Hiew: \RU{найдем это место}\EN{find the place to be patched}}
\end{figure}

\RU{Заменим инструкцию \TT{MOVZX} на нашу.}
\EN{Let's replace \TT{MOVZX} instruction by our.}
\RU{Сделаем вид что у нас 64 ядра процессора}\EN{Let's pretend we've got 64 CPU cores}.
\RU{Добавим дополнительную инструкцию \ac{NOP} (потому что наша инструкция короче чем та что там сейчас):}
\EN{Add one additional \ac{NOP} (because our instruction is shorter than original one):}

\begin{figure}[H]
\centering
\includegraphics[scale=\FigScale]{examples/taskmgr/hiew1.png}
\caption{Hiew: \RU{меняем инструкцию}\EN{patch it}}
\end{figure}

\RU{И это работает}\EN{And it works}!
\RU{Конечно же, данные в графиках неправильные}\EN{Of course, data in graphs is not correct}.
\RU{Иногда, Task Manager даже показывает общую загрузку CPU более 100\%.}
\EN{At times, Task Manager even shows overall CPU load more than 100\%.}

\begin{figure}[H]
\centering
\includegraphics[scale=\FigScale]{examples/taskmgr/taskmgr_64cpu_crop.png}
\caption{\RU{Обманутый}\EN{Fooled} Windows Task Manager}
\end{figure}

\RU{Я выбрал число 64, потому что Task Manager падает если установить б\'{о}льшее значение.}
\EN{I picked number of 64, because Task Manager is crashing if you try to set larger value.}
\RU{Должно быть, Task Manager в Windows Vista не тестировался на компьютерах с большим количеством ядер.}
\EN{Apparently, Task Manager in Windows Vista was not tested on computer with larger count of cores.}
\RU{И наверное там есть внутри какие-то статичные структуры данных, ограниченные до 64-х ядер.}
\EN{So there are probably some static data structures inside it limited to 64 cores.}

\section{\RU{Использование LEA для загрузки значений}\EN{Using LEA to load values}}

\RU{Иногда, \TT{LEA} используется в \TT{taskmgr.exe} вместо \TT{MOV} для установки первого аргумента 
\TT{NtQuerySystemInformation()}:}
\EN{Sometimes, \TT{LEA} is used in \TT{taskmgr.exe} instead of \TT{MOV} to set first argument of 
\TT{NtQuerySystemInformation()}:}

\lstinputlisting[caption=taskmgr.exe (Windows Vista)]{examples/taskmgr/taskmgr2.lst}

\RU{Честно говоря, я не знаю почему, но \ac{MSVC} часто так делает.}
\EN{I honestly, don't know why, but that is what \ac{MSVC} often does.}
\RU{Может быть, это какая-то оптимизация и \TT{LEA} работает быстрее или лучше чем загрузка значения 
используя \TT{MOV}?}
\EN{Maybe this some kind of optiization and \TT{LEA} works faster or better than load 
value using \TT{MOV}?}

\clearpage
\chapter{\RU{Шутка с игрой Color Lines}\EN{Color Lines game practical joke}}
\label{chap:color_lines}

\RU{Это очень популярная игра с большим количеством реализаций}\EN{This is a very popular game with several 
implementations in existence}.
\RU{Возьмем одну из них, с названием}\EN{We can take one of them, called} BallTriX, \RU{от}\EN{from} 1997, 
\RU{доступную бесплатно на}\EN{available freely at} 
\url{http://go.yurichev.com/17311}.
\RU{Вот как она выглядит}\EN{Here is how it looks}:

\begin{figure}[H]
\centering
\includegraphics[scale=\FigScale]{examples/lines/1.png}
\caption{\RU{Обычный вид игры}\EN{How this game looks usually}}
\label{fig:lines_1}
\end{figure}

\clearpage
\index{\CStandardLibrary!rand()}
\RU{Посмотрим, сможем ли мы найти генератор псевдослучайных чисел и и сделать с ним одну шутку.}
\EN{So let's see, is it be possible to find the random generator and do some trick with it.}
\IDA \RU{быстро распознает стандартную функцию}\EN{quickly recognize the standard} \TT{\_rand} \RU{в}\EN{function in} 
\TT{balltrix.exe} \RU{по адресу}\EN{at} \TT{0x00403DA0}.
\IDA \RU{также показывает, что она вызывается только из одного места}\EN{also shows that it is called 
only from one place}:

\lstinputlisting{examples/lines/random.lst}

\RU{Назовем её}\EN{We'll call it} \q{random}.
\RU{Пока не будем концентрироваться на самом коде функции}\EN{Let's not to dive into this function's code yet}.

\RU{Эта функция вызывается из трех мест}\EN{This function is referred from 3 places}.

\RU{Вот первые два}\EN{Here are the first two}:

\lstinputlisting{examples/lines/1.lst}

\EN{Here is the third one}\RU{Вот третье}:

\lstinputlisting{examples/lines/2.lst}

\RU{Так что у функции только один аргумент}\EN{So the function has only one argument}.
\RU{10 передается в первых двух случаях и 5 в третьем.}
\EN{10 is passed in first two cases and 5 in third.}
\RU{Мы также можем заметить, что размер доски 10*10 и здесь 5 возможных цветов}\EN{We can also notice 
that the board has a size of 10*10 and there are 5 possible colors}.
\RU{Это оно}\EN{This is it}!
\RU{Стандартная функция}\EN{The standard} \TT{rand()} \RU{возвращает число в пределах}\EN{function returns 
a number in the} \TT{0..0x7FFF} \RU{и это неудобно, так что многие программисты пишут свою функцию,
возвращающую случайное число в некоторых заданных пределах}\EN{range and this is often inconvenient,
so many programmers implement their own random functions which returns a random number in a specified range}.
\RU{В нашем случае, предел это}\EN{In our case, the range is} $0..n-1$ \AndENRU $n$ \RU{передается как
единственный аргумент в функцию}\EN{is passed as the sole argument of the function}.
\RU{Мы можем быстро проверить это в отладчике}\EN{We can quickly check this in any debugger}.

\RU{Сделаем так, чтобы третий вызов функции всегда возвращал ноль}\EN{So let's fix the third function call to always return zero}.
\RU{В начале заменим три инструкции}\EN{First, we will replace three instructions} (\TT{PUSH/CALL/ADD}) 
\RU{на}\EN{by} \ac{NOP}s.
\RU{Затем добавим инструкцию}\EN{Then we'll add} \INS{XOR EAX, EAX}\RU{, для очистки регистра \EAX}\EN{ instruction, 
to clear the \EAX register}.

\lstinputlisting{examples/lines/fixed.lst}

\RU{Что мы сделали, это заменили вызов функции}\EN{So what we did is we replaced a call to the} \TT{random()} 
\RU{на код, всегда возвращающий ноль}\EN{function by a code which always returns zero}.

\clearpage
\RU{Теперь запустим}\EN{Let's run it now}:

\begin{figure}[H]
\centering
\includegraphics[scale=\FigScale]{examples/lines/2.png}
\caption{\RU{Шутка сработала}\EN{Practical joke works}}
\end{figure}

\RU{О да, это работает}\EN{Oh yes, it works}\footnote{\RU{Автор этой книги однажды сделал это как 
шутку для его сотрудников, в надежде что они перестанут играть. 
Надежды не оправдались.}\EN{Author of this book once did this as a joke for his coworkers with 
the hope that they would stop playing. They didn't.}}.

\RU{Но почему аргументы функции}\EN{But why are the arguments to the} \TT{random()} \RU{это глобальные 
переменные}\EN{functions global variables}?
\RU{Это просто потому что в настройках игры можно изменять размер доски, так что эти параметры не 
фиксированы}\EN{That's just because it's possible to change the board size in the game's settings, 
so these values are not hardcoded}.
\EN{The }10 \AndENRU 5 \RU{это просто значения по умолчанию}\EN{values are just defaults}.

\chapter{\MinesweeperWinXPExampleChapterName}
\label{minesweeper_winxp}
\index{Windows!Windows XP}

\RU{Для тех, кто не очень хорошо играет в Сапёра (Minesweeper), можно попробовать найти все скрытые мины в отладчике.}%
\EN{For those who is not very good at playing Minesweeper, we could try to reveal the hidden mines in the debugger.}

\index{\CStandardLibrary!rand()}
\index{Windows!PDB}
\RU{Как мы знаем, Сапёр располагает мины случайным образом, так что там должен быть генератор случайных чисел
или вызов стандартной функции Си \TT{rand()}.}
\EN{As we know, Minesweeper places mines randomly, so there has to be some kind of random number generator or
a call to the standard \TT{rand()} C-function.}
\RU{Вот что хорошо в реверсинге продуктов от Microsoft, так это то что часто есть \gls{PDB}-файл со всеми
символами (имена функций, \etc{}.).}
\EN{What is really cool about reversing Microsoft products is that there are \gls{PDB} 
file with symbols (function names, \etc{}).}
\RU{Когда мы загружаем}\EN{When we load} \TT{winmine.exe} \RU{в}\EN{into} \IDA, \RU{она скачивает}\EN{it downloads the} 
\gls{PDB} \RU{файл именно для этого исполняемого файла и добавляет все имена}\EN{file exactly for this 
executable and shows all names}.

\RU{И вот оно, только один вызов}\EN{So here it is, the only call to} \TT{rand()} \RU{в этой 
функции}\EN{is this function}:

\begin{lstlisting}
.text:01003940 ; __stdcall Rnd(x)
.text:01003940 _Rnd@4          proc near               ; CODE XREF: StartGame()+53
.text:01003940                                         ; StartGame()+61
.text:01003940
.text:01003940 arg_0           = dword ptr  4
.text:01003940
.text:01003940                 call    ds:__imp__rand
.text:01003946                 cdq
.text:01003947                 idiv    [esp+arg_0]
.text:0100394B                 mov     eax, edx
.text:0100394D                 retn    4
.text:0100394D _Rnd@4          endp
\end{lstlisting}

\RU{Так её назвала \IDA и это было имя данное ей разработчиками Сапёра.}
\EN{\IDA named it so, and it was the name given to it by Minesweeper's developers.}

\RU{Функция очень простая}\EN{The function is very simple}:

\begin{lstlisting}
int Rnd(int limit)
{
    return rand() % limit;
};
\end{lstlisting}

\RU{(В \gls{PDB}-файле не было имени \q{limit}; это мы назвали этот аргумент так, вручную.)}
\EN{(There was no \q{limit} name in the \gls{PDB} file; we manually named this argument like this.)}

\RU{Так что она возвращает случайное число в пределах от нуля до заданного предела}\EN{So it returns 
a random value from 0 to a specified limit}.

\TT{Rnd()} \RU{вызывается только из одного места, это функция с названием}\EN{is called only from one place, 
a function called} \TT{StartGame()}, 
\RU{и как видно, это именно тот код, что расставляет мины}\EN{and as it seems, this is exactly 
the code which place the mines}:

\begin{lstlisting}
.text:010036C7                 push    _xBoxMac
.text:010036CD                 call    _Rnd@4          ; Rnd(x)
.text:010036D2                 push    _yBoxMac
.text:010036D8                 mov     esi, eax
.text:010036DA                 inc     esi
.text:010036DB                 call    _Rnd@4          ; Rnd(x)
.text:010036E0                 inc     eax
.text:010036E1                 mov     ecx, eax
.text:010036E3                 shl     ecx, 5          ; ECX=ECX*32
.text:010036E6                 test    _rgBlk[ecx+esi], 80h
.text:010036EE                 jnz     short loc_10036C7
.text:010036F0                 shl     eax, 5          ; EAX=EAX*32
.text:010036F3                 lea     eax, _rgBlk[eax+esi]
.text:010036FA                 or      byte ptr [eax], 80h
.text:010036FD                 dec     _cBombStart
.text:01003703                 jnz     short loc_10036C7
\end{lstlisting}

\RU{Сапёр позволяет задать размеры доски, так что X (xBoxMac) и Y (yBoxMac) это глобальные переменные.}
\EN{Minesweeper allows you to set the board size, so the X (xBoxMac) and Y (yBoxMac) of the board are global variables.}
\RU{Они передаются в}\EN{They are passed to} \TT{Rnd()} \RU{и генерируются случайные координаты}\EN{and random 
coordinates are generated}.
\RU{Мина устанавливается инструкцией}\EN{A mine is placed by the} \TT{OR} \RU{на}\EN{instruction at} \TT{0x010036FA}. 
\RU{И если она уже была установлена до этого}\EN{And if it was placed before} 
(\RU{это возможно, если пара функций}\EN{it's possible if the pair of} \TT{Rnd()} 
\RU{сгенерирует пару, которая уже была сгенерирована}\EN{generates a coordinates pair which was already 
was generated}), 
\RU{тогда}\EN{then} \TT{TEST} \AndENRU \TT{JNZ} \RU{на}\EN{at} \TT{0x010036E6} 
\RU{перейдет на повторную генерацию пары}\EN{jumps to the generation routine again}.

\TT{cBombStart} \RU{это глобальная переменная, содержащая количество мин. Так что это цикл.}
\EN{is the global variable containing total number of mines. So this is loop.}

\RU{Ширина двухмерного массива это 32 (мы можем это вывести, глядя на инструкцию \TT{SHL}, которая умножает
одну из координат на 32)}\EN{The width of the array is 32 
(we can conclude this by looking at the \TT{SHL} instruction, which multiplies one of the coordinates by 32)}.

\RU{Размер глобального массива}\EN{The size of the} \TT{rgBlk} 
\RU{можно легко узнать по разнице между меткой}\EN{global array can be easily determined by the difference 
between the} \TT{rgBlk} 
\RU{в сегменте данных и следующей известной меткой}\EN{label in the data segment and the next known one}. 
\RU{Это}\EN{It is} 0x360 (864):

\begin{lstlisting}
.data:01005340 _rgBlk          db 360h dup(?)          ; DATA XREF: MainWndProc(x,x,x,x)+574
.data:01005340                                         ; DisplayBlk(x,x)+23
.data:010056A0 _Preferences    dd ?                    ; DATA XREF: FixMenus()+2
...
\end{lstlisting}

$864/32=27$.

\RU{Так что размер массива}\EN{So the array size is} $27*32$?
\RU{Это близко к тому что мы знаем: если попытаемся установить размер доски в установках Сапёра на $100*100$, то он установит размер $24*30$}%
\EN{It is close to what we know: when we try to set board size to $100*100$ in Minesweeper settings, it fallbacks to a board of size $24*30$}.
\RU{Так что это максимальный размер доски здесь}\EN{So this is the maximal board size here}.
\RU{И размер массива фиксирован для доски любого размера}\EN{And the array has a fixed size for any board size}.

\RU{Посмотрим на всё это в}\EN{So let's see all this in} \olly.
\RU{Запустим Сапёр, присоединим (attach) \olly к нему и увидим содержимое памяти по адресу где массив \TT{rgBlk} (\TT{0x01005340})}%
\EN{We will ran Minesweeper, attaching \olly to it and now we can see the memory dump at the address of the \TT{rgBlk} array (\TT{0x01005340})}
\footnote{\RU{Все адреса здесь для Сапёра под}\EN{All addresses here are for Minesweeper for} Windows XP SP3 English. 
\RU{Они могут отличаться для других сервис-паков}\EN{They may differ for other service packs}.}.

\RU{Так что у нас выходит такой дамп памяти массива}\EN{So we got this memory dump of the array}:

\lstinputlisting{examples/minesweeper/1.lst}

\olly, \RU{как и любой другой шестнадцатеричный редактор, показывает 16 байт на строку}\EN{like any other 
hexadecimal editor, shows 16 bytes per line}.
\RU{Так что каждая 32-байтная строка массива занимает ровно 2 строки}\EN{So each 32-byte array row occupies
exactly 2 lines here}.

\RU{Это уровень для начинающих (доска 9*9)}\EN{This is beginner level (9*9 board)}.

\RU{Тут еще какая-то квадратная структура, заметная визуально (байты 0x10)}\EN{There is some square 
structure can be seen visually (0x10 bytes)}.

\RU{Нажмем \q{Run} \InENRU \olly чтобы разморозить процесс Сапёра, потом нажмем в случайное место окна Сапёра, попадаемся на мине, но теперь
видны все мины}%
\EN{We will click \q{Run} \InENRU \olly to unfreeze the Minesweeper process, then we'll clicked randomly at the Minesweeper window 
and trapped into mine, but now all mines are visible}:

\begin{figure}[H]
\centering
\includegraphics[scale=\FigScale]{examples/minesweeper/1.png}
\caption{\RU{Мины}\EN{Mines}}
\label{fig:minesweeper1}
\end{figure}

\RU{Сравнивая места с минами и дамп, мы можем обнаружить что 0x10 это граница, 0x0F\EMDASH{}пустой блок, 
0x8F\EMDASH{}мина.}
\EN{By comparing the mine places and the dump, we can conclude that 0x10 stands for border, 0x0F\EMDASH{}empty block, 0x8F---mine.}

\RU{Теперь добавим комментариев и также заключим все байты 0x8F в квадратные скобки:}%
\EN{Now we'll add comments and also enclose all 0x8F bytes into square brackets:}

\lstinputlisting{examples/minesweeper/2.lst}

\RU{Теперь уберем все байты связанные с границами (0x10) и всё что за ними:}%
\EN{Now we'll remove all \IT{border bytes} (0x10) and what's beyond those:}

\lstinputlisting{examples/minesweeper/3.lst}

\RU{Да, это всё мины, теперь это очень хорошо видно, в сравнении со скриншотом.}
\EN{Yes, these are mines, now it can be clearly seen and compared with the screenshot.}

\clearpage
\RU{Вот что интересно, это то что мы можем модифицировать массив прямо в \olly.}%
\EN{What is interesting is that we can modify the array right in \olly.}
\RU{Уберем все мины заменив все байты 0x8F на 0x0F, и вот что получится в Сапёре}%
\EN{We can remove all mines by changing all 0x8F bytes by 0x0F, and here is what we'll get in Minesweeper}:

\begin{figure}[H]
\centering
\includegraphics[scale=\FigScale]{examples/minesweeper/3.png}
\caption{\RU{Все мины убраны в отладчике}\EN{All mines are removed in debugger}}
\label{fig:minesweeper3}
\end{figure}

\RU{Также уберем их все и добавим их в первом ряду}\EN{We can also move all of them to the first line}: 

\begin{figure}[H]
\centering
\includegraphics[scale=\FigScale]{examples/minesweeper/2.png}
\caption{\RU{Мины, установленные в отладчике}\EN{Mines set in debugger}}
\label{fig:minesweeper2}
\end{figure}

\RU{Отладчик не очень удобен для подсматривания (а это была наша изначальная цель), так что напишем маленькую
утилиту для показа содержимого доски:}%
\EN{Well, the debugger is not very convenient for eavesdropping (which was our goal anyway), so we'll write a small utility
to dump the contents of the board:}

\lstinputlisting{examples/minesweeper/minesweeper_cheater.c}

\RU{Просто установите}\EN{Just set the} \ac{PID}
\footnote{PID \RU{можно увидеть в}\EN{it can be seen in} Task Manager 
(\RU{это можно включить в}\EN{enable it in} \q{View $\rightarrow$ Select Columns})} 
\RU{и адрес массива}\EN{and the address of the array} (\TT{0x01005340} \RU{для}\EN{for} Windows XP SP3 English) 
\RU{и она покажет его}\EN{and it will dump it}
\footnote{\RU{Скомпилированная версия здесь}\EN{The compiled executable is here}: 
\href{http://go.yurichev.com/17165}{beginners.re}}.

\RU{Она подключается к win32-процессу по \ac{PID}-у и просто читает из памяти процесса по этому адресу.}
\EN{It attaches itself to a win32 process by \ac{PID} and just reads process memory an the address.}

\section{\Exercises}

\begin{itemize}

\item \RU{Почему байты описывающие границы (0x10) присутствуют вообще?}
\EN{Why do the \IT{border bytes} (0x10) exist in the array?}
\RU{Зачем они нужны, если они вообще не видимы в интерфейсе Сапёра?}
\EN{What they are for if they are not visible in Minesweeper's interface?}
\RU{Как можно обойтись без них}\EN{How could it work without them}?

\item \RU{Как выясняется, здесь больше возможных значений (для открытых блоков, для тех на которых игрок установил
	флажок, \etc{}.).}
	\EN{As it turns out, there are more values possible (for open blocks, for flagged by user, \etc{}).}
\RU{Попробуйте найти значение каждого}\EN{Try to find the meaning of each one}.

\item \RU{Измените мою утилиту так, чтобы она в запущенном процессе Сапёра убирала все мины, 
или расставляла их в соответствии с каким-то заданным шаблоном.}
\EN{Modify my utility so it can remove all mines or set them in a fixed pattern that you want in the Minesweeper
process currently running.}

\item \RU{Измените мою утилиту так, чтобы она работала без задаваемого адреса массива и без \gls{PDB}-файла.}
\EN{Modify my utility so it can work without the array address specified and without a \gls{PDB} file.}
\RU{Да, вполне возможно автоматически найти информацию о доске в сегменте данных в запущенном процессе Сапёра.}
\EN{Yes, it's possible to find board information in the data segment of Minesweeper's running process automatically.}
%\RU{Подсказка}\EN{Hint}: \myref{minesweeper_winxp_hint}.
%\RU{Подсказка}\EN{Hint}: \RU{подумайте о байтах, описывающих границы (0x10).}
%\EN{think about  \IT{border bytes} (0x10).}

\end{itemize}

\chapter{\IFRU{Ручная декомпиляция + использование SMT-солвера Z3 для взлома любительской криптографии}
{Hand decompiling + using Z3 SMT solver for defeating amateur cryptography}}

\IFRU{Любительская криптография обычно (непреднамеренно) очень слабая и может быть легко сломана ---
для криптографов, конечно}{Amateur cryptography is usually (unintentionally) 
very weak and can be breaked easily---for cryptographers, of course}.

\IFRU{Но представим на время что мы не в числе этих профессионалов}
{But let's pretend we are not among these crypto-professionals}.

\IFRU{Я нашел эту необратимую хэш-ф-цию, которая конвертирует одно 64-битное значение в другое,
и нам нужно попытаться развернуть её работу назад}
{I once found this one-way hash function, converting 64-bit value to another one and we need to try
to reverse its flow back}.

\label{hash_func}
\begin{quotation}
\index{\IFRU{Хеш-функции}{Hash functions}}
\index{CRC32}
\IFRU{Но что такое хеш-функция}{But what is hash-function}?
\IFRU{Простейший пример это CRC32, алгоритм ``более мощный'' чем простая контрольная сумма,
для проверки целостности данных}
{Simplest example is CRC32, an algorithm providing ``stronger'' checksum for integrity checking purposes}.
\IFRU{Невозможно восстановить оригинальный текст из хеша, там просто меньше информации: ведь текст
может быть очень длинным, но результат CRC32 всегда ограничен 32 битами}
{it is impossible to restore original text from the hash value, it just has much less information:
there can be long text, but CRC32 result is always limited to 32 bits}.
\IFRU{Но CRC32 не надежна в криптографическом смысле: известны методы как изменить текст таким образом,
чтобы получить нужный результат}
{But CRC32 is not cryptographically secure: it is known how to alter a text in that way so the resulting
CRC32 hash value will be one we need}.
\IFRU{Криптографические хеш-функции защищены от этого}
{Cryptographical hash functions are protected from this}.
\index{MD5}
\index{SHA1}
\IFRU{Такие ф-ции как MD5, SHA1, и т.д, широко используются для хеширования паролей
для хранения их в базе}
{They are widely used to hash user passwords in order to store them in the database, 
like MD5, SHA1, etc}.
\IFRU{Действительно: БД форума в интернете может и не хранить пароли 
(иначе злоумышленник получивший доступ к БД сможет узнать все пароли), а только хеши}
{Indeed: an internet forum database may not contain user passwords (stolen database will compromise
all user's passwords) but only hashes (a cracker will not be able to reveal passwords)}.
\IFRU{К тому же, скрипту интернет-форума вовсе не обязательно знать ваш пароль, он только должен
cверить его хеш с тем что лежит в БД, и дать вам доступ если cверка проходит}
{Besides, an internet forum engine is not aware of your password, it should only check if its hash
is the same as in the database, then it will give you access in this case}.
\IFRU{Один из самых простых способов взлома это просто перебирать все пароли и ждать пока
результат будет такой же как тот что нам нужен}
{One of the simplest passwords cracking methods is just to brute-force all passwords in order to wait
when resulting value will be the same as we need}.
\IFRU{Другие методы намного сложнее}{Other methods are much more complex}. \\
\end{quotation}

\section{\IFRU{Ручная декомпиляция}{Hand decompiling}}

\IFRU{Вот листинг на ассемблере в}{Here its assembly language listing in} \IDA:

\lstinputlisting{examples/z3/algo_1.asm}

\index{Hex-Rays}
\IFRU{Если вы не имеете}{If} Hex-Rays\IFRU{, либо вы не доверяете его результатам, мы можем попробовать
переписать всё это на Си вручную}{ is not in list of our posessions, or we distrust to it, 
we may try to reverse this code manually}.
\IFRU{Один из методов, это представить регистры \ac{CPU} в виде локальных переменных Си и заменить каждую инструкцию
эквивалентным выражением, например}{One method is to represent \ac{CPU} registers as local C variables and 
replace each instruction by one-line equivalent expression, like}:

\lstinputlisting{examples/z3/algo_2.c}

\IFRU{Если быть очень аккуратным, этот код можно скомпилировать и он даже будет работать, 
точно так же как оригинальный}
{If to be careful enough, this code can be compiled and will even work in the same way as original one}.

\IFRU{Затем, будем переписывать его постепенно, не забывая об использовании регистров}
{Then, we will rewrite it gradually, keeping in mind all registers usage}.
\IFRU{Внимание и фокусирование здесь крайне важно --- любая самая мелкая опечатка может испортить всю работу}
{Attention and focusing is very important here---any tiny typo may ruin all your work}!

\IFRU{Первый шаг}{Here is a first step}:

\lstinputlisting{examples/z3/algo_3.c}

\IFRU{Следующий шаг}{Next step}:

\lstinputlisting{examples/z3/algo_4.c}

\IFRU{Мы находим деление через умножение}{We may spot division using multiplication} (\ref{sec:divisionbynine}).
\index{Wolfram Mathematica}
\IFRU{Действительно, найдем делитель в}{Indeed, let's calculate divider in} Wolfram Mathematica:

\begin{lstlisting}[caption=Wolfram Mathematica]
In[1]:=N[2^(64 + 5)/16^^8888888888888889]
Out[1]:=60.
\end{lstlisting}

\IFRU{Получаем}{We get this}:

\lstinputlisting{examples/z3/algo_5.c}

\IFRU{Еще один шаг}{Another step}:

\lstinputlisting{examples/z3/algo_6.c}

\IFRU{Простым сокращением, мы видим, что вычислялось вовсе не \glslink{quotient}{частное}, а остаток от деления}
{By simple reducing, we finally see that it's not \gls{quotient} calculated, but division remainder}:

\lstinputlisting{examples/z3/algo_7.c}

\IFRU{Заканчиваем на приятно отформатированном исходном коде}{We end up on something fancy formatted source-code}:

\lstinputlisting{examples/z3/algo_src.c}

\IFRU{Так как мы не криптоаналитики, мы не можем найти простой способ найти входное значение
для определенного выходного значения}{Since we are not cryptoanalysts we can't find an easy way to 
generate input value for some specific output value}.
\IFRU{Коэффициенты инструкций сдвигов выглядят очень пугающе --- это гарантия что ф-ция не биективная,
она имеет коллизии, или, говоря проще, возможны несколько значений на входе для одного на выходе}
{Rotate instruction coefficients are look frightening---it's a warranty that the function is not bijective,
it has collisions, or, speaking more simply, many inputs may be possible for one output}.

\IFRU{Брут-форс это тоже не решение, т.к., значения 64-битные, и это совершенно нереально}
{Brute-force is not solution because values are 64-bit ones, that's beyond reality}.

\section{\IFRU{Попробуем Z3 SMT-солвер}{Now let's use Z3 SMT solver}}
\index{Z3}

\IFRU{Но все же, без всяких специальных знаний из криптографии, мы можем попытаться взломать алгоритм при помощи
великолепного SMT-солвера от}{Still, without any special cryptographical knowledge, we may try to break this 
algorithm using excellent SMT solver from} Microsoft Research \IFRU{под названием}{named} 
Z3\footnote{\url{http://z3.codeplex.com/}}.
\IFRU{На самом деле, это автоматический доказыватель теорем, но мы будем использовать его как SMT-солвер}
{It is in fact theorem prover, but we will use it as SMT solver}.
\IFRU{Упрощенно говоря, мы можем думать о нем как о системе, способной решать очень большие системы уравнений}
{In terms of simplicity, we may think about it as a system capable of solving huge equation systems}.

\IFRU{Вот исходный код на Питоне}{Here is a Python source code}:

\lstinputlisting[numbers=left]{examples/z3/1.py}

\IFRU{Это будет наш первый солвер}{This will be our first solver}.

\IFRU{На строке 7 мы видим объявление переменных}{We see variable difinitions on line 7}.
\IFRU{Это просто 64-битные переменные}{These are just 64-bit variables}.
\TT{i1..i6} \IFRU{это промежуточные переменные, отражающие значения в регистрах между исполнениями инструкций}
{are intermediate variables, representing values in registers between instruction executions}.

\IFRU{Потом добавляем т.н. констрайнты, в строках}{Then we add so called constraints on lines} 10..15.
\IFRU{Самый последний констрайнт в строке 17 это наиболее важный: мы будем искать входное значение для
нашего алгоритма, при котором он выдаст на выходе}{The very last constraint at 17 is most important: 
we will try to find input value for which our algorithm will produce} $10816636949158156260$.

\IFRU{Собственно, SMT-солвер ищет (любые) значения, удовлетворяющие всем констрайнтам}
{Essentially, SMT-solver searches for (any) values that satisfy all constraints}.

RotateRight, RotateLeft, URem\EMDASH{}\IFRU{это ф-ции из Питоновского Z3 \ac{API} для описания выражений, 
они не связаны с ЯП Python}
{are functions from Z3 Python \ac{API}, they are not related to Python \ac{PL}}.

\IFRU{Запускаем}{Then we run it}:

\begin{lstlisting}
...>python.exe 1.py
sat
[i1 = 3959740824832824396,
 i3 = 8957124831728646493,
 i5 = 10816636949158156260,
 inp = 1364123924608584563,
 outp = 10816636949158156260,
 i4 = 14065440378185297801,
 i2 = 4954926323707358301]
 inp=0x12EE577B63E80B73
outp=0x961C69FF0AEFD7E4
\end{lstlisting}

``sat'' \IFRU{означает}{mean} ``satisfiable'', \IFRU{т.е., солвер нашел по крайней мере одно решение}
{i.e., solver was able to found at least one solution}.
\IFRU{Решение выведено внутри квадратных скобок}{The solution is printed inside square brackets}.
\IFRU{Две последние строки это пара входного/выходного значения в шестнадцатеричном виде}
{Two last lines are input/output pair in hexadecimal form}.
\IFRU{Да, действительно, если мы запустим нашу ф-цию с}{Yes, indeed, if we run our function with} 
\TT{0x12EE577B63E80B73} \IFRU{на входе, алгоритм выдаст искомое значение}
{on input, the algorithm will produce the value we were looking for}.

\IFRU{Но, как мы заметили раннее, ф-ция не биективная, так что тут могут быть и другие корректные входные значения}
{But, as we are noticed before, the function we work with is not bijective, so there are may be other correct
input values}.
\IFRU{Z3 SMT-солвер не выдает результаты больше одного, но мы можем хакнуть наш пример немного, 
добавив констрайнт в строке 19, означая, что мы ищем какие угодно другие результаты кроме этого}
{Z3 SMT solver is not capable of producing more than one result, but let's hack our example slightly, 
by adding line 19, meaining, look for any other results than this}:

\lstinputlisting[numbers=left]{examples/z3/2.py}

\IFRU{Действительно, получаем еще один верный результат}{Indeed, it found other correct result}:

\begin{lstlisting}
...>python.exe 2.py
sat
[i1 = 3959740824832824396,
 i3 = 8957124831728646493,
 i5 = 10816636949158156260,
 inp = 10587495961463360371,
 outp = 10816636949158156260,
 i4 = 14065440378185297801,
 i2 = 4954926323707358301]
 inp=0x92EE577B63E80B73
outp=0x961C69FF0AEFD7E4
\end{lstlisting}

\IFRU{Это можно автоматизировать}{This can be automated}.
\IFRU{Каждый найденный результат можно добавлять в качестве констрайнта и искать следующий}
{Each found result may be added as constraint and the next result will be searched for}.
\IFRU{Пример немного сложнее}{Here is slightly sophisticated example}:

\lstinputlisting[numbers=left]{examples/z3/3.py}

\IFRU{Получаем}{We got}:

\begin{lstlisting}
1364123924608584563
1234567890
9223372038089343698
4611686019661955794
13835058056516731602
3096040143925676201
12319412180780452009
7707726162353064105
16931098199207839913
1906652839273745429
11130024876128521237
15741710894555909141
6518338857701133333
5975809943035972467
15199181979890748275
10587495961463360371
results total= 16
\end{lstlisting}

\IFRU{Так что имеется 16 верных входных значений для}{So there are 16 correct input values are possible for} 
\TT{0x92EE577B63E80B73} \IFRU{на выходе}{as a result}.

\IFRU{Второй это}{The second is} $1234567890$\EMDASH{}\IFRU{действительно, я это и использовал изначально,
когда готовил этот пример}
{it is indeed a value I used originally while preparing this example}.

\IFRU{Попробуем изучить алгоритм немного больше}{Let's also try to research our algorithm more}.
\IFRU{В порыве садистских желаний, попробуем найти, есть ли здесь какая-нибудь возможная пара входов/выходов,
в которых младшие 32-битные части равны друг другу}
{By some sadistic purposes, let's find, are there any possible input/output pair in 
which lower 32-bit parts are equal to each other}?

\IFRU{Уберем констрайнт}{Let's remove} \IT{outp} \IFRU{и добавим другой, в строке 17}
{constraint and add another, at line 17}:

\lstinputlisting[numbers=left]{examples/z3/4.py}

\IFRU{И действительно}{It is indeed so}:

\begin{lstlisting}
sat
[i1 = 14869545517796235860,
 i3 = 8388171335828825253,
 i5 = 6918262285561543945,
 inp = 1370377541658871093,
 outp = 14543180351754208565,
 i4 = 10167065714588685486,
 i2 = 5541032613289652645]
 inp=0x13048F1D12C00535
outp=0xC9D3C17A12C00535
\end{lstlisting}

\IFRU{Можем упражняться в садизме и далее: пусть последние 16-бит всегда будут}
{Let's be more sadistic and add another constaint: last 16-bit should be} \TT{0x1234}:

\lstinputlisting[numbers=left]{examples/z3/5.py}

\IFRU{Это так же возможно}{Oh yes, this possible as well}:

\begin{lstlisting}
sat
[i1 = 2834222860503985872,
 i3 = 2294680776671411152,
 i5 = 17492621421353821227,
 inp = 461881484695179828,
 outp = 419247225543463476,
 i4 = 2294680776671411152,
 i2 = 2834222860503985872]
 inp=0x668EEC35F961234
outp=0x5D177215F961234
\end{lstlisting}

\IFRU{Z3 работает крайне быстро и это означает что алгоритм слаб, и вообще не относится к криптографическим 
(как и почти вся любительская криптография)}
{Z3 works very fast and it means that algorithm is weak, it is not cryptographical at all
(like the most of amateur cryptography)}.

\IFRU{Можно ли попытаться сделать что-то подобное с настоящими криптоалгоритмами этими методами}
{Will it be possible to tackle real cryptography by these methods}? 
\IFRU{Настоящие алгоритмы, такие как}{Real algorithms like} AES, RSA, \IFRU{итд, так же могут быть представлены
в виде огромных систем уравнений, но они большие настолько, что с ними нельзя работать на компьютерах,
ни сейчас, ни в обозримом будущем}{etc, can also be represented as huge system of equations, 
but these are that huge that are impossible to work with on computers, now or in near future}.
\IFRU{Разумеется, криптографы об этом всем прекрасно знают}{Of course, cryptographers are aware of this}.

\IFRU{Еще одна статья которую я написал о Z3:}{Another article I wrote about Z3 is} \cite{Rockey}.


\chapter{\RU{Донглы}\EN{Dongles}}
\label{dongles}

\RU{Иногда я делаю замену \glslink{dongle}{донглам} или ``эмуляторы донглов'' 
и здесь немного примеров, как это происходит}
\EN{Occasionally I do software copy-protection \gls{dongle} replacements, or ``dongle emulators'' and here
are couple examples of my work}
\footnote{\RU{Больше об этом читайте тут}\EN{Read more about it}: \url{http://yurichev.com/dongles.html}}.

\RU{Об одном неописанном здесь случае вы также можете прочитать здесь}
\EN{About one of not described cases you may also read here}: \cite{Rockey}.

\section{\RU{Пример}\EN{Example} \#1: MacOS Classic \AndENRU PowerPC}

\index{PowerPC}
\index{Mac OS Classic}
\RU{Вот пример программы для}\EN{Here is an example of a program for} MacOS Classic
\footnote{\RU{MacOS перед тем как перейти на UNIX}\EN{pre-UNIX MacOS}}, \RU{для}\EN{for} PowerPC.
\RU{Компания, разработавшая этот продукт, давно исчезла, так что (легальный) пользователь
боялся того что донгла может сломаться}\EN{The company who developed the software product
has disappeared a long time ago, so the (legal) customer was afraid of physical dongle damage}.

\RU{Если запустить программу без подключенной донглы, можно увидеть окно с надписью}
\EN{While running without a dongle connected, a message box with the text}
"Invalid Security Device"\EN{ appeared}.
\RU{Мне повезло потому что этот текст можно было легко найти внутри исполняемого файла}
\EN{Luckily, this text string could easily be found in the executable binary file}.

\RU{Представим, что мы не знакомы ни с Mac OS Classic, ни с PowerPC, но всё-таки попробуем}%
\EN{Let's pretend we are not very familiar both with Mac OS Classic and PowerPC, but will try anyway}.

\ac{IDA} \RU{открывает исполняемый файл легко, показывая его тип как}
\EN{opened the executable file smoothly, reported its type as} 
"PEF (Mac OS or Be OS executable)" (\RU{действительно, это стандартный тип файлов в Mac OS Classic}
\EN{indeed, it is a standard Mac OS Classic file format}).

\RU{В поисках текстовой строки с сообщение об ошибке, мы попадаем на этот фрагмент кода}%
\EN{By searching for the text string with the error message, we've got into this code fragment}:

\lstinputlisting{examples/dongles/1/1.lst}

\index{ARM}
\index{MIPS}
\RU{Да, это код PowerPC}\EN{Yes, this is PowerPC code}.
\RU{Это очень типичный процессор для \ac{RISC} 1990-х}
\EN{The CPU is a very typical 32-bit \ac{RISC} of 1990s era}.
\RU{Каждая инструкция занимает 4 байта (как и в MIPS и ARM) и их имена немного похожи на имена 
инструкций MIPS}
\EN{Each instruction occupies 4 bytes (just as in MIPS and ARM) and the names somewhat resemble
MIPS instruction names}.

\TT{check1()} \RU{это имя которое мы дадим этой функции немного позже}\EN{is a function name we'll give to it later}.
\TT{BL} \RU{это инструкция}\EN{is} \IT{Branch Link} 
\RU{т.е. предназначенная для вызова подпрограмм}\EN{instruction, e.g., intended for calling subroutines}.
\RU{Самое важное место\EMDASH{}это инструкция \ac{BNE}, срабатывающая, если проверка наличия донглы прошла
успешно, либо не срабатывающая в случае ошибки: и тогда адрес текстовой строки с сообщением об ошибке
будет загружен в регистр r3 для последующей передачи в функцию отображения диалогового окна}
\EN{The crucial point is the \ac{BNE} instruction which jumps if the dongle protection check passes 
or not if an error occurs: 
then the address of the text string gets loaded into the r3 register for the subsequent passing into a message box routine}.

\RU{Из}\EN{From the} \cite{PPCABI} \RU{мы узнаем, что регистр r3 используется для возврата
значений (и еще r4 если значение 64-битное)}%
\EN{we will found out that the r3 register is used for return values (and r4, in case of 64-bit values)}.

\index{x86!\Instructions!MOVZX}
\RU{Еще одна, пока что неизвестная инструкция}\EN{Another yet unknown instruction is} \TT{CLRLWI}. 
\RU{Из}\EN{From} \cite{PPC} \RU{мы узнаем, что эта инструкция одновременно и очищает и загружает}%
\EN{we'll learn that this instruction does both clearing and loading}. 
\RU{В нашем случае, она очищает 24 старших бита из значения в r3 и записывает всё это в r0, 
так что это аналог}\EN{In our case, it clears the 24 high bits from the value in r3
and puts them in r0, so it is analogical to} \MOVZX \InENRU x86 (\myref{movzx}),
\RU{но также устанавливает флаги, так что}\EN{but it also sets the flags, so} \ac{BNE} 
\RU{может проверить их потом}\EN{can check them afterwards}.

\RU{Посмотрим внутрь}\EN{Let's take a look into the} \TT{check1()}\EN{ function}:

\lstinputlisting{examples/dongles/1/check1.lst}

\RU{Как можно увидеть в \ac{IDA}, эта функция вызывается из многих мест в программе, но только значение
в регистре r3 проверяется сразу после каждого вызова}
\EN{As you can see in \ac{IDA}, that function is called from many places in the program, but only the r3 register's value
is checked after each call}.
\index{thunk-\RU{функции}\EN{functions}}
\RU{Всё что эта функция делает это только вызывает другую функцию, так что это}
\EN{All this function does is to call the other function, so it is a} \gls{thunk function}: 
\RU{здесь присутствует и пролог функции и эпилог, но регистр r3 не трогается, так что}
\EN{there are function prologue and epilogue, but the r3 register is not touched, so} \TT{checkl()} 
\RU{возвращает то, что возвращает}\EN{returns what} \TT{check2()}\EN{ returns}.

\ac{BLR} \RU{это похоже возврат из функции, но так как IDA делает всю разметку функций автоматически,
наверное, мы можем пока не интересоваться этим}\EN{looks like the return from the function, but since \ac{IDA} does the function layout, we probably do not need
to care about this}.
\RU{Так как это типичный \ac{RISC}, похоже, подпрограммы вызываются, используя}
\EN{Since it is a typical \ac{RISC}, it seems that subroutines are called using a} \gls{link register},
\RU{точно как в}\EN{just like in} ARM.

\EN{The}\RU{Функция} \TT{check2()} \RU{более сложная}\EN{function is more complex}:

\lstinputlisting{examples/dongles/1/check2.lst}

\index{USB}
\RU{Снова повезло: имена некоторых функций оставлены в исполняемом файле
(в символах в отладочной секции? Трудно сказать до тех пор, пока мы не знакомы с этим форматом файлов,
может быть это что-то вроде PE-экспортов (\myref{PE_exports_imports}))?
как например \TT{.RBEFINDNEXT()} and \TT{.RBEFINDFIRST()}.}
\EN{We are lucky again: some function names are left in the executable 
(debug symbols section? Hard to say while we are not very familiar with the file format, maybe it is
some kind of PE exports? (\myref{PE_exports_imports})),
like \TT{.RBEFINDNEXT()} and \TT{.RBEFINDFIRST()}.}
\RU{В итоге, эти функции вызывают другие функции с именами вроде}
\EN{Eventually these functions call other functions with names like} \TT{.GetNextDeviceViaUSB()}, 
\TT{.USBSendPKT()},
\RU{так что они явно работают с каким-то USB-устройством}\EN{so these are clearly dealing with an USB device}.

\RU{Тут даже есть функция с названием}\EN{There is even a function named} 
\TT{.GetNextEve3Device()}\EMDASH\RU{звучит знакомо, в 1990-х годах была донгла}\EN{sounds familiar, there was a} Sentinel Eve3 
\RU{для ADB-порта (присутствующих на Макинтошах)}\EN{dongle for ADB port (present on Macs) in 1990s}.

\RU{В начале посмотрим на то как устанавливается регистр r3 одновременно игнорируя всё остальное}
\EN{Let's first take a look on how the r3 register is set before return, while ignoring everything else}.
\RU{Мы знаем, что \q{хорошее} значение в r3 должно быть не нулевым, а нулевой r3 приведет
к выводу диалогового окна с сообщением об ошибке.}
\EN{We know that a \q{good} r3 value has to be non-zero, zero r3 leads the execution
flow to the message box with an error message.}

\RU{В функции имеются две инструкции}\EN{There are two} \TT{li \%r3, 1} 
\RU{и одна}\EN{instructions present in the function and one} \TT{li \%r3, 0} 
(\IT{Load Immediate}, \RU{т.е. загрузить значение в регистр}\EN{i.e., loading a value into a register}).
\RU{Самая первая инструкция находится на}\EN{The first instruction is at} 
\TT{0x001186B0}\EMDASH\RU{и честно говоря, трудно заранее понять, что это означает}%
\EN{and frankly speaking, it's hard to say what it means}.

\RU{А вот то что мы видим дальше понять проще}\EN{What we see next is, however, easier to understand}: 
\RU{вызывается }\TT{.RBEFINDFIRST()} \RU{и в случае ошибки, 0 будет записан в r3
и мы перейдем на \IT{exit}, а иначе будет вызвана функция \TT{check3()}\EMDASH{}если и она будет
выполнена с ошибкой, будет вызвана}\EN{is called:
if it fails, 0 is written into r3 and we jump to \IT{exit}, otherwise another
function is called (\TT{check3()})\EMDASH{}if it fails too, }
\TT{.RBEFINDNEXT()} \RU{вероятно, для поиска другого USB-устройства}
\EN{is called, probably in order to look for another USB device}.

N.B.: \TT{clrlwi. \%r0, \%r3, 16} \RU{это аналог того что мы уже видели, но она очищает 16 старших бит,
т.е.}\EN{it is analogical to what we already saw, but it clears
16 bits, i.e.}, \TT{.RBEFINDFIRST()} \RU{вероятно возвращает 16-битное значение}
\EN{probably returns a 16-bit value}.

\TT{B} (\RU{означает}\EN{stands for} \IT{branch}) \RU{\EMDASH{}безусловный переход}\EN{unconditional jump}.

\ac{BEQ} \RU{это обратная инструкция от}\EN{is the inverse instruction of} \ac{BNE}.

\RU{Посмотрим на}\EN{Let's see} \TT{check3()}:

\lstinputlisting{examples/dongles/1/check3.lst}

\RU{Здесь много вызовов}\EN{There are a lot of calls to} \TT{.RBEREAD()}. 
\RU{Эта функция вероятно читает какие-то значения из донглы, которые потом сравниваются здесь при помощи}
\EN{The function probably returns some values from the dongle,
so they are compared here with some hard-coded variables using} \TT{CMPLWI}.

\RU{Мы также видим в регистр r3 записывается перед каждым вызовом}
\EN{We also see that the r3 register is also filled before each call to} \TT{.RBEREAD()} 
\RU{одно из этих значений}\EN{with one of these values}: 0, 1, 8, 0xA, 0xB, 0xC, 0xD, 4, 5.
\RU{Вероятно адрес в памяти или что-то в этом роде}\EN{Probably a memory address or something like that}?

\RU{Да, действительно, если погуглить имена этих функций, можно легко найти документацию к}
\EN{Yes, indeed, by googling these function names it is easy to find the} Sentinel Eve3\EN{ dongle manual}!

\RU{Hаверное, уже и не нужно изучать остальные инструкции PowerPC: всё что делает эта функция это просто
вызывает}
\EN{Perhaps, we don't even need to learn any other PowerPC instructions: all this function does is just
call} \TT{.RBEREAD()}, \RU{сравнивает его результаты с константами и возвращает 1 если результат сравнения положительный или 0 в другом случае}\EN{compare its results with the constants and returns 1 if the comparisons
are fine or 0 otherwise}.

\RU{Всё ясно: \TT{check1()} должна всегда возвращать 1 или иное ненулевое значение}
\EN{OK, all we've got is that \TT{check1()} has always to return 1 or any other non-zero value}.
\RU{Но так как мы не очень уверены в своих знаниях инструкций PowerPC, будем осторожны и пропатчим переходы в \TT{check2} на адресах
\TT{0x001186FC} и \TT{0x00118718}.}
\EN{But since we are not very confident in our knowledge of PowerPC instructions, we are going to be careful: we will patch the jumps in \TT{check2()} at
\TT{0x001186FC} and \TT{0x00118718}.}

\RU{На}\EN{At} \TT{0x001186FC} \RU{мы записываем байты}\EN{we'll write bytes} 0x48 \AndENRU 0 
\RU{таким образом превращая инструкцию}\EN{thus converting the} \ac{BEQ} 
\RU{в инструкцию}\EN{instruction in an} 
\TT{B} (\RU{безусловный переход}\EN{unconditional jump}):
\RU{Мы можем заметить этот опкод прямо в коде даже без обращения к}%
\EN{We can spot its opcode in the code without even referring to} \cite{PPC}.

\RU{На}\EN{At} \TT{0x00118718} \RU{мы записываем байт}\EN{we'll write} 0x60 \AndENRU \RU{еще 3 нулевых байта,
таким образом превращая её в инструкцию}\EN{3 zero bytes, thus converting it to a}
\ac{NOP}\EN{ instruction}:
\RU{Этот опкод мы также можем подсмотреть прямо в коде}\EN{Its opcode we could spot in the code too}.

\RU{И всё заработало без подключенной донглы}\EN{And now it all works without a dongle connected}.

\RU{Резюмируя, такие простые модификации можно делать в \ac{IDA} даже с минимальными знаниями
ассемблера}\EN{In summary, such small modifications can be done with \ac{IDA} and minimal assembly language knowledge}.


\subsection{\IFRU{Пример}{Example} \#2: SCO OpenServer}

\index{SCO OpenServer}
\IFRU{Древняя программа для}{An ancient software for} SCO OpenServer \IFRU{от}{from} 1997 
\IFRU{разработанная давно исчезнувшей компанией}{developed
by a company disappeared long time ago}.

\IFRU{Специальный драйвер донглы инсталлируется в системе, он содержит такие текстовые строки}
{There is a special dongle driver to be installed in the system, containing text strings}:
``Copyright 1989, Rainbow Technologies, Inc., Irvine, CA''
\AndENRU
``Sentinel Integrated Driver Ver. 3.0 ''.

\IFRU{После инсталляции драйвера, в /dev появляются такие устройства}
{After driver installation in SCO OpenServer, these device files are appeared in /dev filesystem}:

\begin{lstlisting}
/dev/rbsl8
/dev/rbsl9
/dev/rbsl10
\end{lstlisting}

\IFRU{Без подключенной донглы, программа сообщает об ошибке, но сообщение об ошибке не удается
найти в исполняемых файлах}
{The program without dongle connected reports error, but the error string cannot be found in the executables}.

\index{COFF}
\IFRU{Еще раз спасибо \ac{IDA}, она легко загружает исполняемые файлы формата COFF использующиеся в}
{Thanks to \ac{IDA}, it does its job perfectly working out COFF executable used in} SCO OpenServer.

\IFRU{Я попробовал также поискать строку}{I've tried to find} ``rbsl'' 
\IFRU{, и действительно, её можно найти в таком фрагменте кода}
{and indeed, found it in this code fragment}:

\lstinputlisting{examples/dongles/2/1.lst}

\IFRU{Действительно, должна же как-то программа обмениваться информацией с драйвером}
{Yes, indeed, the program should comminicate with driver somehow and that is how it is}.

\index{thunk-\IFRU{функции}{functions}}
\IFRU{Единственное место где вызывается ф-ция}{The only place} \TT{SSQC()}
\IFRU{это}{function called is the} \gls{thunk function}:

\lstinputlisting{examples/dongles/2/2.lst}

\IFRU{А вот }{}SSQ() \IFRU{вызывается по крайней мерез из двух разных ф-ций}
{is called at least from 2 functions}.

\IFRU{Одна из них}{One of these is}:

\lstinputlisting{examples/dongles/2/check1.lst}

``\TT{3C}'' \AndENRU ``\TT{3E}'' \IFRU{~--- это звучит знакомо: когда-то была донгла}
{are sounds familiar: there was a} Sentinel Pro \IFRU{от Rainbow без памяти,
предоставляющая только одну секретную крипто-хеширующую ф-цию}{dongle by Rainbow with no memory,
providing only one crypto-hashing secret function}.

\begin{quotation}
\index{\IFRU{Хеш-функции}{Hash functions}}
\index{CRC32}
\IFRU{Но что такое хеш-функция}{But what is hash-function}?
\IFRU{Простейший пример это CRC32, алгоритм ``более мощный'' чем простая контрольная сумма,
для проверки целостности данных}
{Simplest example is CRC32, an algorithm providing ``stronger'' checksum for integrity checking purposes}.
\IFRU{Невозмжоно восстановить оригинальный текст из хеша, там просто меньше информации: ведь текст
может быть очень длинным, но результат CRC32 всегда ограничен 32 битами}
{It is not possible to restore original text from the hash value, it just has much less information:
there can be long text, but CRC32 result is always limited to 32 bits}.
\IFRU{Но CRC32 не надежна в криптографическом смысле: известны методы как изменить текст таким образом,
чтобы получить нужный результат}
{But CRC32 is not cryptographically secure: it is known how to alter a text in that way so the resulting
CRC32 hash value will be one we need}.
\IFRU{Криптографические хеш-функции защищены от этого}
{Cryptographical hash functions are protected from this}.
\index{MD5}
\index{SHA1}
\IFRU{Такие ф-ции как MD5, SHA1, итд, широко используются для хеширования паролей
для хранения их в базе}
{They are widely used to hash user passwords in order to store them in the database, 
like MD5, SHA1, etc}.
\IFRU{Действительно: БД форума в интернете может и не хранить пароли 
(иначе злоумышленник получивший доступ к БД сможет узнать все пароли), а только хеши}
{Indeed: an internet forum database may not contain user passwords (stolen database will compromise
all user's passwords) but only hashes (a cracker will not be able to reveal passwords)}.
\IFRU{К тому же, скрипту интернет-форума вовсе не обязательно знать ваш пароль, он только должен
cверить его хеш с тем что лежит в БД, и дать вам доступ если cверка проходит}
{Besides, an internet forum engine is not aware of your password, it should only check if its hash
is the same as in the database, then it will give you access in this case}.
\IFRU{Один из самых простых способов взлома это просто перебирать все пароли и ждать пока
результат будет такой же как тот что нам нужен}
{One of the simplest passwords cracking methods is just to brute-force all passwords in order to wait
when resulting value will be the same as we need}.
\IFRU{Другие методы намного сложнее}{Other methods are much more complex}. \\
\end{quotation}

\IFRU{Но вернемся к нашей программе}{But let's back to the program}.
\IFRU{Так что программа может только проверить подключена ли донгла или нет}
{So the program can only check the presence or absence dongle connected}.
\IFRU{Никакой больше информации в такую донглу без памяти записать нельзя}
{No other information can be written to such dongle with no memory}.
\IFRU{Двухсимвольные коды это команды}{Two-character codes are commands}
(\IFRU{можно увидеть как они обрабатывюатся в ф-ции}{we can see how commands are handled in} 
\TT{SSQC()}\IFRU{}{ function}) 
\IFRU{а все остальные строки хешируются внутри донглы превращаясь в 16-битное число}
{and all other strings are hashed inside the dongle transforming into 16-bit number}.
\IFRU{Алгоритм был секретный, так что нельзя было написать замену драйверу или сделать
электронную копию донглы идеально эмулирующую алгоритм}{The algorithm was secret,
so it was not possible to write driver replacement or to remake dongle hardware emulating it perfectly}.
\IFRU{С другой стороны, всегда можно было перехватить все обращения к ней и найти те константы, с которыми
сравнивается результат хеширования}
{However, it was always possible to intercept all accesses to it and to find what constants
the hash function results compared to}.
\IFRU{Но надо сказать, вполне возможно создать устойчивую защиту от копирования базирующуюся
на секретной хеш-функции: пусть она шифрует все файлы с которыми ваша программа работает}
{Needless to say it is possible to build a robust software copy protection scheme based on secret
cryptographical hash-function: let it to encrypt/decrypt data files your software working with}.

\IFRU{Но вернемся к нашему коду}{But let's back to the code}.

\IFRU{Коды}{Codes} 51/52/53 \IFRU{используются для выбора номера принтеровского LPT-порта}
{are used for LPT printer port selection}.
3x/4x \IFRU{используются для выбора}{is for} ``family'' \IFRU{так донглы Sentinel Pro
можно отличать друг от друга: ведь более одной донглы может быть подключено к LPT-порту}
{selection (that's how Sentinel Pro dongles are differentiated from each other: more than one
dongle can be connected to LPT port)}.

\IFRU{Единственная строка передающаяся в хеш-функцию это}
{The only non-2-character string passed to the hashing function is} "0123456789".
\IFRU{Затем результат сравнивается с несколькими правильными значениями}
{Then, the result is compared against the set of valid results}.
\IFRU{Если результат правилен}{If it is correct},
0xC \OrENRU 0xB \IFRU{будет записано в глобальную переменную}
{is to be written into global variable} \TT{ctl\_model}.

\IFRU{Еще одна строка для хеширования:}{Another text string to be passed is}
"PRESS ANY KEY TO CONTINUE: ", \IFRU{но результат не проверяется}{but the result is not checked}.
\IFRU{Не знаю зачем это, может быть по ошибке}{I don't know why, probably by mistake}.
(\IFRU{Это очень странное чувство: находить ошибки в столь древнем ПО}
{What a strange feeling: to reveal bugs in such ancient software}.)

\IFRU{Давайте посмотрим, где проверяется значение глобальной переменной}
{Let's see where the value from the global variable} \TT{ctl\_mode}\IFRU{}{ is used}.

\IFRU{Одно из таких мест}{One of such places is}:

\lstinputlisting{examples/dongles/2/4.lst}

\IFRU{Если оно 0, шифрованное сообщение об ошибке будет передано в ф-цию дешифрования и оно будет 
показано}{If it is 0, an encrypted error message is passed into decryption routine and printed}.

\index{x86!\Instructions!XOR}
\IFRU{Ф-ция дешифровки сообщений об ошибке похоже применяет простой \ac{XOR}}
{Error strings decryption routine is seems simple xoring}:

\lstinputlisting{examples/dongles/2/err_warn.lst}

% TODO: reverse the function with examples

\IFRU{Вот почему не получилось найти сообщение об ошибке в исполняемых файлах, потому что оно было
зашифровано, это очень популярная практика}
{That's why I was unable to find error messages in the executable files, because they are enrcypted,
this is popular practice}.

\IFRU{Еще один вызов хеширующей ф-ции передает строку}{Another call to \TT{SSQ()} hashing function passes}
``offln'' \IFRU{и сравнивает результат с константами}{string to it and comparing result with}
\TT{0xFE81} \AndENRU \TT{0x12A9}.
\IFRU{Если результат не сходится, происходит работа с какой-то ф-цией}
{If it not so, it deals with some} \TT{timer()} 
\IFRU{(может быть для ожидания плохо подключенной донглы и нового запроса?), затем дешифрует
еще одно сообщение об ошибке и выводит его}{function (maybe waiting for poorly
connected dongle to be reconnected and check again?) and then decrypt another error message to dump}.

\lstinputlisting{examples/dongles/2/check2.lst}

\IFRU{Заставить работать программу без донглы довольно просто: просто пропатчить все места после инструкций
\CMP где происходят соответствующие сравнения}{Dongle bypassing is pretty straightforward: just patch all jumps after \CMP the relevant instructions}.

\IFRU{Еще одна возможность это написать свой драйвер для SCO OpenServer}
{Another option is to write our own SCO OpenServer driver}.


\section{\RU{Пример}\EN{Example} \#3: MS-DOS}
\label{dongle_16bit_dos}

\index{MS-DOS}
\RU{Еще одна очень старая программа для}\EN{Another very old software for} MS-DOS \RU{от}\EN{from} 1995 
\RU{также разработанная давно исчезнувшей компанией}
\EN{also developed by a company that disappeared a long time ago}.

\index{Intel!8086}
\index{Intel!80286}
\RU{Во времена перед DOS-экстендерами, всё ПО для MS-DOS рассчитывалось на процессоры 8086 или 80286,
так что в своей массе весь код был 16-битным}
\EN{In the pre-DOS extenders era, all the software for MS-DOS mostly relied on 16-bit 8086 or 80286 CPUs,
so en masse the code was 16-bit}.
\RU{16-битный код в основном такой же, какой вы уже видели в этой книге, но все регистры 16-битные,
и доступно меньше инструкций}
\EN{The 16-bit code is mostly same as you already saw in this book, but all registers
are 16-bit and there are less instructions available}.

\label{IN_example}
\label{OUT_example}
\index{x86!\Instructions!IN}
\index{x86!\Instructions!OUT}
\RU{Среда MS-DOS не могла иметь никаких драйверов, и ПО работало с \q{голым} железом через порты,
так что здесь вы можете увидеть инструкции \TT{OUT}/\TT{IN}, 
которые в наше время присутствуют в основном только
в драйверах (в современных OS нельзя обращаться на прямую к портам из \gls{user mode})}
\EN{The MS-DOS environment has no system drivers, and any program can deal with the bare hardware via ports,
so here you can see the \TT{OUT}/\TT{IN} instructions, which are present in mostly in drivers in our times
(it is impossible to access ports directly in \gls{user mode} on all modern \ac{OS}es)}.

\RU{Учитывая это, ПО для MS-DOS должно работать с донглой обращаясь к принтерному LPT-порту
напрямую}
\EN{Given that, the MS-DOS program which works with a dongle has to access the LPT printer port directly}.
\RU{Так что мы можем просто поискать эти инструкции. И да, вот они}
\EN{So we can just search for such instructions. And yes, here they are}:

\lstinputlisting{examples/dongles/3/1.lst}

(\RU{Все имена меток в этом примере даны мною}\EN{All label names in this example were given by me}).

\RU{Функция }\TT{out\_port()} \RU{вызывается только из одной функции}
\EN{is referenced only in one function}:

\lstinputlisting{examples/dongles/3/2.lst}

\RU{Это также \q{хеширующая} донгла Sentinel Pro как и в предыдущем примере}
\EN{This is again a Sentinel Pro \q{hashing} dongle as in the previous example}.
\RU{Это заметно по тому что текстовые строки передаются и здесь, 16-битные значения также возвращаются и сравниваются с другими}%
\EN{It is noticeably because text strings are passed here, too, and 16 bit values are returned and compared with others}.

\RU{Так вот как происходит работа с Sentinel Pro через порты}
\EN{So that is how Sentinel Pro is accessed via ports}.
\RU{Адрес выходного порта обычно 0x378, т.е. принтерного порта, данные для него во времена
перед USB отправлялись прямо сюда}
\EN{The output port address is usually 0x378, i.e.,
the printer port, where the data to the old printers in pre-USB era was passed to}.
\RU{Порт однонаправленный, потому что когда его разрабатывали, никто не мог предположить,
что кому-то понадобится получать информацию из принтера}
\EN{The port is uni-directional, because when it was developed, no one imagined that someone
will need to transfer information from the printer}
\footnote{\RU{Если учитывать только Centronics и не учитывать последующий стандарт IEEE 1284\EMDASH{}
в нем из принтера можно получать информацию}
\EN{If we consider Centronics only. The following IEEE 1284 standard allows the transfer of information from
the printer}.}.
\RU{Единственный способ получить информацию из принтера это регистр статуса на порту 0x379,
он содержит такие биты как \q{paper out}, \q{ack}, \q{busy}\EMDASH{}так принтер может сигнализировать
о том, что он готов или нет, и о том, есть ли в нем бумага}
\EN{The only way to get information from the printer is a status register on port 0x379, which contains
such bits as \q{paper out}, \q{ack}, \q{busy}\EMDASH{}thus the printer may signal to the host computer
if it is ready or not and if paper is present in it}.
\RU{Так что донгла возвращает информацию через какой-то из этих бит, по одному биту на каждой
итерации}
\EN{So the dongle returns information from one of these bits, one bit at each iteration}.

\TT{\_in\_port\_2} \RU{содержит адрес статуса}\EN{contains the address of the status word} (0x379) \AndENRU 
\TT{\_in\_port\_1} \RU{содержит адрес управляющего регистра}\EN{contains the control register address} (0x37A).

\RU{Судя по всему, донгла возвращает информацию только через флаг \q{busy} на}
\EN{It seems that the dongle returns information via the \q{busy} flag at} \TT{seg030:00B9}: 
\RU{каждый бит записывается в регистре \TT{DI} позже возвращаемый в самом конце функции}
\EN{each bit is stored in the \TT{DI} register, which is returned at the end of the function}.

\RU{Что означают все эти отсылаемые в выходной порт байты}%
\EN{What do all these bytes sent to output port mean}?
\RU{Трудно сказать. Возможно, команды донглы.}\EN{Hard to say. Probably commands to the dongle.}
\RU{Но честно говоря, нам и не обязательно знать: нашу задачу можно легко решить и без этих знаний}
\EN{But generally speaking, it is not necessary to know: it is easy to solve our task without that knowledge}.

\RU{Вот функция проверки донглы}\EN{Here is the dongle checking routine}:

\lstinputlisting{examples/dongles/3/3.lst}

\RU{А так как эта функция может вызываться слишком часто, например, 
перед выполнением каждой важной возможности ПО,
а обращение к донгле вообще-то медленное (и из-за медленного принтерного порта, и из-за медленного
\ac{MCU} в донгле), так что они, наверное, добавили возможность пропускать проверку донглы слишком часто,
используя текущее время в функции \TT{biostime()}}
\EN{Since the routine can be called very frequently, e.g., before the execution of each important software feature, 
and accessing the dongle is generally slow (because of the slow printer port and also slow
\ac{MCU} in the dongle), they probably added a way to skip some dongle checks,
by checking the current time in the \TT{biostime()} function}.

\index{\CStandardLibrary!rand()}
\EN{The}\RU{Функция} \TT{get\_rand()} \RU{использует стандартную функцию Си}
\EN{function uses the standard C function}:

\lstinputlisting{examples/dongles/3/4.lst}

\RU{Так что текстовая строка выбирается случайно, отправляется в донглу и результат
хеширования сверяется с корректным значением}
\EN{So the text string is selected randomly, passed into the dongle, and then the result of the hashing 
is compared with the correct value}.

\RU{Текстовые строки, похоже, составлялись так же случайно, во время разработки ПО.}%
\EN{The text strings seem to be constructed randomly as well, during software development.}

\RU{И вот как вызывается главная процедура проверки донглы}
\EN{And this is how the main dongle checking function is called}:

\lstinputlisting{examples/dongles/3/5.lst}

\RU{Заставить работать программу без донглы очень просто: просто заставить функцию
\TT{check\_dongle()} возвращать всегда 0}
\EN{Bypassing the dongle is easy, just force the \TT{check\_dongle()} function to always return 0}.

\RU{Например, вставив такой код в самом её начале}\EN{For example, by inserting this code at its beginning}:

\begin{lstlisting}
mov ax,0
retf
\end{lstlisting}

\index{\CStandardLibrary!strcpy()}
\RU{Наблюдательный читатель может заметить, что функция Си \TT{strcpy()} имеет 2 аргумента, но здесь
мы видим, что передается 4}
\EN{The observant reader might recall that the \TT{strcpy()} C function usually requires two pointers in its arguments,
but we see that 4 values are passed}:

\begin{lstlisting}
seg033:088F 1E                          push    ds
seg033:0890 68 22 44                    push    offset aTrupcRequiresA ; "This Software Requires a Software Lock\n"
seg033:0893 1E                          push    ds
seg033:0894 68 60 E9                    push    offset byte_6C7E0 ; dest
seg033:0897 9A 79 65 00+                call    _strcpy
seg033:089C 83 C4 08                    add     sp, 8
\end{lstlisting}

\RU{Это связано с моделью памяти в MS-DOS. Об этом больше читайте здесь}
\EN{This is related to MS-DOS' memory model. You can read more about it here}: 
\myref{8086_memory_model}.

\RU{Так что, \TT{strcpy()}, как и любая другая функция принимающая указатель (-и) в аргументах,
работает с 16-битными парами}
\EN{So as you may see, \TT{strcpy()} and any other function that take pointer(s) in arguments
work with 16-bit pairs}.

\RU{Вернемся к нашему примеру}\EN{Let's get back to our example}.
\TT{DS} \RU{сейчас указывает на сегмент данных размещенный в исполняемом файле, там, где хранится текстовая
строка.}
\EN{is currently set to the data segment located in the executable,
that is where the text string is stored.}

\index{x86!\Instructions!LES}
\RU{В функции \TT{sent\_pro()} каждый байт строки загружается на \TT{seg030:00EF}: инструкция
\TT{LES} загружает из переданного аргумента пару ES:BX одновременно}
\EN{In the \TT{sent\_pro()} function, each byte of the string is loaded at \TT{seg030:00EF}: the \TT{LES} instruction
loads the ES:BX pair simultaneously from the passed argument}.
\RU{\MOV на \TT{seg030:00F5} загружает байт из памяти, на который указывает пара ES:BX.}
\EN{The \MOV at \TT{seg030:00F5} loads the byte from the memory at which the ES:BX pair points.}

% TODO rewrite
%\RU{На \TT{seg030:00F2} \glslink{increment}{инкрементируется} только вторая 16-битная пара адреса.}
%\EN{At \TT{seg030:00F2} only a second 16-bit part of address is \glslink{increment}{incremented}.}
%\RU{Это значит, что переданная в функцию строка не может находиться на границе двух сегментов.}
%\EN{This implies that the string passed to the function cannot be located on the boundary between two data segments.}



\chapter{\RU{\q{QR9}: Любительская криптосистема, вдохновленная кубиком Рубика}
\EN{\q{QR9}: Rubik's cube inspired amateur crypto-algorithm}}

\RU{Любительские криптосистемы иногда встречаются довольно странные.}
\EN{Sometimes amateur cryptosystems appear to be pretty bizarre.}

\RU{Однажды автора сих строк попросили разобраться с одним таким любительским криптоалгоритмом встроенным в 
утилиту для шифрования, исходный код которой был утерян\footnote{Он также получил разрешение от 
клиента на публикацию деталей алгоритма}.}
\EN{The author of this book was once asked to reverse engineer an amateur cryptoalgorithm of some data encryption utility, 
the source code for which was lost\footnote{He also got permission from the customer to publish the algorithm's details}.}

\RU{Вот листинг этой утилиты для шифрования, полученный при помощи \IDA}%
\EN{Here is the listing exported from \IDA for the original encryption utility}:

\lstinputlisting{examples/qr9/qr9_original.lst}

\RU{Все имена функций и меток даны мною в процессе анализа.}
\EN{All function and label names were given by me during the analysis.}

\RU{Начнем с самого верха. Вот функция, берущая на вход два имени файла и пароль.}
\EN{Let's start from the top. Here is a function that takes two file names and password.}

\begin{lstlisting}
.text:00541320 ; int __cdecl crypt_file(int Str, char *Filename, int password)
.text:00541320 crypt_file      proc near
.text:00541320
.text:00541320 Str             = dword ptr  4
.text:00541320 Filename        = dword ptr  8
.text:00541320 password        = dword ptr  0Ch
.text:00541320
\end{lstlisting}

\RU{Открыть файл и сообщить об ошибке в случае ошибки:}\EN{Open the file and report if an error occurs:}

\begin{lstlisting}
.text:00541320                 mov     eax, [esp+Str]
.text:00541324                 push    ebp
.text:00541325                 push    offset Mode     ; "rb"
.text:0054132A                 push    eax             ; Filename
.text:0054132B                 call    _fopen          ; open file
.text:00541330                 mov     ebp, eax
.text:00541332                 add     esp, 8
.text:00541335                 test    ebp, ebp
.text:00541337                 jnz     short loc_541348
.text:00541339                 push    offset Format   ; "Cannot open input file!\n"
.text:0054133E                 call    _printf
.text:00541343                 add     esp, 4
.text:00541346                 pop     ebp
.text:00541347                 retn
.text:00541348
.text:00541348 loc_541348:
\end{lstlisting}

\index{\CStandardLibrary!fseek()}
\index{\CStandardLibrary!ftell()}
\RU{Узнать размер файла используя}\EN{Get the file size via} \TT{fseek()}/\TT{ftell()}:

\lstinputlisting{examples/qr9/1.\LANG}

\RU{Этот фрагмент кода вычисляет длину файла, выровненную по 64-байтной границе.
Это потому что этот алгоритм шифрования работает только с блоками размерами 64 байта.
Работает очень просто: разделить длину файла на 64, забыть об остатке, прибавить 1,
умножить на 64.
Следующий код удаляет остаток от деления, как если бы это значение уже было разделено 
на 64 и добавляет 64. Это почти то же самое.}
\EN{This fragment of code calculates the file size aligned on a 64-byte boundary. 
This is because this cryptographic algorithm works with only 64-byte blocks. 
The operation is pretty straightforward: divide the file size by 64, forget about the remainder and add 1, 
then multiply by 64. 
The following code removes the remainder as if the value was already divided by 64 and adds 64. 
It is almost the same.}

\lstinputlisting{examples/qr9/2.\LANG}

\RU{Выделить буфер с выровненным размером:}\EN{Allocate buffer with aligned size:}

\begin{lstlisting}
.text:00541373                 push    esi             ; Size
.text:00541374                 call    _malloc
\end{lstlisting}

\index{\CStandardLibrary!calloc()}
\RU{Вызвать memset(), т.е. очистить выделенный буфер\footnote{malloc() + memset() можно было бы 
заменить на calloc()}.}\EN{Call memset(), e.g., clear the allocated buffer\footnote{malloc() + memset() could 
be replaced by calloc()}.}

\lstinputlisting{examples/qr9/3.\LANG}

\RU{Чтение файла используя стандартную функцию Си}\EN{Read file via the standard C function} \TT{fread()}.

\begin{lstlisting}
.text:00541392                 mov     eax, [esp+38h+Str]
.text:00541396                 push    eax             ; ElementSize
.text:00541397                 push    ebx             ; DstBuf
.text:00541398                 call    _fread          ; read file
.text:0054139D                 push    ebp             ; File
.text:0054139E                 call    _fclose
\end{lstlisting}

\RU{Вызов \TT{crypt()}. Эта функция берет на вход буфер, длину буфера (выровненную) и строку пароля.}
\EN{Call \TT{crypt()}. This function takes a buffer, buffer size (aligned) and a password string.}

\begin{lstlisting}
.text:005413A3                 mov     ecx, [esp+44h+password]
.text:005413A7                 push    ecx             ; password
.text:005413A8                 push    esi             ; aligned size
.text:005413A9                 push    ebx             ; buffer
.text:005413AA                 call    crypt           ; do crypt
\end{lstlisting}

\RU{Создать выходной файл. Кстати, разработчик забыл вставить проверку, создался ли файл успешно!
Результат открытия файла, впрочем, проверяется.}
\EN{Create the output file. By the way, the developer forgot to check if it is was created correctly! 
The file opening result is being checked, though.}

\begin{lstlisting}
.text:005413AF                 mov     edx, [esp+50h+Filename]
.text:005413B3                 add     esp, 40h
.text:005413B6                 push    offset aWb      ; "wb"
.text:005413BB                 push    edx             ; Filename
.text:005413BC                 call    _fopen
.text:005413C1                 mov     edi, eax
\end{lstlisting}

\RU{Теперь хэндл созданного файла в регистре \EDI. Записываем сигнатуру \q{QR9}.}
\EN{The newly created file handle is in the \EDI register now. Write signature \q{QR9}.}

\begin{lstlisting}
.text:005413C3                 push    edi             ; File
.text:005413C4                 push    1               ; Count
.text:005413C6                 push    3               ; Size
.text:005413C8                 push    offset aQr9     ; "QR9"
.text:005413CD                 call    _fwrite         ; write file signature
\end{lstlisting}

\RU{Записываем настоящую длину файла (не выровненную)}\EN{Write the actual file size (not aligned)}:

\begin{lstlisting}
.text:005413D2                 push    edi             ; File
.text:005413D3                 push    1               ; Count
.text:005413D5                 lea     eax, [esp+30h+Str]
.text:005413D9                 push    4               ; Size
.text:005413DB                 push    eax             ; Str
.text:005413DC                 call    _fwrite         ; write original file size
\end{lstlisting}

\RU{Записываем шифрованный буфер}\EN{Write the encrypted buffer}:

\begin{lstlisting}
.text:005413E1                 push    edi             ; File
.text:005413E2                 push    1               ; Count
.text:005413E4                 push    esi             ; Size
.text:005413E5                 push    ebx             ; Str
.text:005413E6                 call    _fwrite         ; write encrypted file
\end{lstlisting}

\RU{Закрыть файл и освободить выделенный буфер}\EN{Close the file and free the allocated buffer}:

\begin{lstlisting}
.text:005413EB                 push    edi             ; File
.text:005413EC                 call    _fclose
.text:005413F1                 push    ebx             ; Memory
.text:005413F2                 call    _free
.text:005413F7                 add     esp, 40h
.text:005413FA                 pop     edi
.text:005413FB                 pop     esi
.text:005413FC                 pop     ebx
.text:005413FD                 pop     ebp
.text:005413FE                 retn
.text:005413FE crypt_file      endp
\end{lstlisting}

\RU{Переписанный на Си код}\EN{Here is the reconstructed C code}:

\begin{lstlisting}
void crypt_file(char *fin, char* fout, char *pw)
{
	FILE *f;
	int flen, flen_aligned;
	BYTE *buf;

	f=fopen(fin, "rb");
	
	if (f==NULL)
	{
		printf ("Cannot open input file!\n");
		return;
	};

	fseek (f, 0, SEEK_END);
	flen=ftell (f);
	fseek (f, 0, SEEK_SET);

	flen_aligned=(flen&0xFFFFFFC0)+0x40;

	buf=(BYTE*)malloc (flen_aligned);
	memset (buf, 0, flen_aligned);

	fread (buf, flen, 1, f);

	fclose (f);

	crypt (buf, flen_aligned, pw);
	
	f=fopen(fout, "wb");

	fwrite ("QR9", 3, 1, f);
	fwrite (&flen, 4, 1, f);
	fwrite (buf, flen_aligned, 1, f);

	fclose (f);

	free (buf);
};
\end{lstlisting}

\RU{Процедура дешифрования почти такая же}\EN{The decryption procedure is almost the same}:

\begin{lstlisting}
.text:00541400 ; int __cdecl decrypt_file(char *Filename, int, void *Src)
.text:00541400 decrypt_file    proc near
.text:00541400
.text:00541400 Filename        = dword ptr  4
.text:00541400 arg_4           = dword ptr  8
.text:00541400 Src             = dword ptr  0Ch
.text:00541400
.text:00541400                 mov     eax, [esp+Filename]
.text:00541404                 push    ebx
.text:00541405                 push    ebp
.text:00541406                 push    esi
.text:00541407                 push    edi
.text:00541408                 push    offset aRb      ; "rb"
.text:0054140D                 push    eax             ; Filename
.text:0054140E                 call    _fopen
.text:00541413                 mov     esi, eax
.text:00541415                 add     esp, 8
.text:00541418                 test    esi, esi
.text:0054141A                 jnz     short loc_54142E
.text:0054141C                 push    offset aCannotOpenIn_0 ; "Cannot open input file!\n"
.text:00541421                 call    _printf
.text:00541426                 add     esp, 4
.text:00541429                 pop     edi
.text:0054142A                 pop     esi
.text:0054142B                 pop     ebp
.text:0054142C                 pop     ebx
.text:0054142D                 retn
.text:0054142E
.text:0054142E loc_54142E:
.text:0054142E                 push    2               ; Origin
.text:00541430                 push    0               ; Offset
.text:00541432                 push    esi             ; File
.text:00541433                 call    _fseek
.text:00541438                 push    esi             ; File
.text:00541439                 call    _ftell
.text:0054143E                 push    0               ; Origin
.text:00541440                 push    0               ; Offset
.text:00541442                 push    esi             ; File
.text:00541443                 mov     ebp, eax
.text:00541445                 call    _fseek
.text:0054144A                 push    ebp             ; Size
.text:0054144B                 call    _malloc
.text:00541450                 push    esi             ; File
.text:00541451                 mov     ebx, eax
.text:00541453                 push    1               ; Count
.text:00541455                 push    ebp             ; ElementSize
.text:00541456                 push    ebx             ; DstBuf
.text:00541457                 call    _fread
.text:0054145C                 push    esi             ; File
.text:0054145D                 call    _fclose
\end{lstlisting}

\RU{Проверяем сигнатуру (первые 3 байта)}\EN{Check signature (first 3 bytes)}:

\begin{lstlisting}
.text:00541462                 add     esp, 34h
.text:00541465                 mov     ecx, 3
.text:0054146A                 mov     edi, offset aQr9_0 ; "QR9"
.text:0054146F                 mov     esi, ebx
.text:00541471                 xor     edx, edx
.text:00541473                 repe cmpsb
.text:00541475                 jz      short loc_541489
\end{lstlisting}

\RU{Сообщить об ошибке если сигнатура отсутствует}\EN{Report an error if the signature is absent}:

\begin{lstlisting}
.text:00541477                 push    offset aFileIsNotCrypt ; "File is not encrypted!\n"
.text:0054147C                 call    _printf
.text:00541481                 add     esp, 4
.text:00541484                 pop     edi
.text:00541485                 pop     esi
.text:00541486                 pop     ebp
.text:00541487                 pop     ebx
.text:00541488                 retn
.text:00541489
.text:00541489 loc_541489:
\end{lstlisting}

\RU{Вызвать}\EN{Call} \TT{decrypt()}.

\begin{lstlisting}
.text:00541489                 mov     eax, [esp+10h+Src]
.text:0054148D                 mov     edi, [ebx+3]
.text:00541490                 add     ebp, 0FFFFFFF9h
.text:00541493                 lea     esi, [ebx+7]
.text:00541496                 push    eax             ; Src
.text:00541497                 push    ebp             ; int
.text:00541498                 push    esi             ; int
.text:00541499                 call    decrypt
.text:0054149E                 mov     ecx, [esp+1Ch+arg_4]
.text:005414A2                 push    offset aWb_0    ; "wb"
.text:005414A7                 push    ecx             ; Filename
.text:005414A8                 call    _fopen
.text:005414AD                 mov     ebp, eax
.text:005414AF                 push    ebp             ; File
.text:005414B0                 push    1               ; Count
.text:005414B2                 push    edi             ; Size
.text:005414B3                 push    esi             ; Str
.text:005414B4                 call    _fwrite
.text:005414B9                 push    ebp             ; File
.text:005414BA                 call    _fclose
.text:005414BF                 push    ebx             ; Memory
.text:005414C0                 call    _free
.text:005414C5                 add     esp, 2Ch
.text:005414C8                 pop     edi
.text:005414C9                 pop     esi
.text:005414CA                 pop     ebp
.text:005414CB                 pop     ebx
.text:005414CC                 retn
.text:005414CC decrypt_file    endp
\end{lstlisting}

\RU{Переписанный на Си код}\EN{Here is the reconstructed C code}:

\begin{lstlisting}
void decrypt_file(char *fin, char* fout, char *pw)
{
	FILE *f;
	int real_flen, flen;
	BYTE *buf;

	f=fopen(fin, "rb");
	
	if (f==NULL)
	{
		printf ("Cannot open input file!\n");
		return;
	};

	fseek (f, 0, SEEK_END);
	flen=ftell (f);
	fseek (f, 0, SEEK_SET);

	buf=(BYTE*)malloc (flen);

	fread (buf, flen, 1, f);

	fclose (f);

	if (memcmp (buf, "QR9", 3)!=0)
	{
		printf ("File is not encrypted!\n");
		return;
	};

	memcpy (&real_flen, buf+3, 4);

	decrypt (buf+(3+4), flen-(3+4), pw);
	
	f=fopen(fout, "wb");

	fwrite (buf+(3+4), real_flen, 1, f);

	fclose (f);

	free (buf);
};
\end{lstlisting}

\RU{OK, посмотрим глубже}\EN{OK, now let's go deeper}.

\RU{Функция}\EN{Function} \TT{crypt()}:

\begin{lstlisting}
.text:00541260 crypt           proc near
.text:00541260
.text:00541260 arg_0           = dword ptr  4
.text:00541260 arg_4           = dword ptr  8
.text:00541260 arg_8           = dword ptr  0Ch
.text:00541260
.text:00541260                 push    ebx
.text:00541261                 mov     ebx, [esp+4+arg_0]
.text:00541265                 push    ebp
.text:00541266                 push    esi
.text:00541267                 push    edi
.text:00541268                 xor     ebp, ebp
.text:0054126A
.text:0054126A loc_54126A:
\end{lstlisting}

\index{x86!\Instructions!MOVSD}
\RU{Этот фрагмент кода копирует часть входного буфера во внутренний буфер, который мы позже назовем \q{cube64}.}%
\EN{This fragment of code copies a part of the input buffer to an internal array we later name \q{cube64}.}
\RU{Длина в регистре \ECX. \TT{MOVSD} означает \IT{скопировать 32-битное слово}, так что, 16 32-битных слов
это как раз 64 байта.}\EN{The size is in the \ECX register. \TT{MOVSD} stands for \IT{move 32-bit dword}, so, 
16 32-bit dwords are exactly 64 bytes.}

\begin{lstlisting}
.text:0054126A                 mov     eax, [esp+10h+arg_8]
.text:0054126E                 mov     ecx, 10h
.text:00541273                 mov     esi, ebx   ; EBX is pointer within input buffer
.text:00541275                 mov     edi, offset cube64
.text:0054127A                 push    1
.text:0054127C                 push    eax
.text:0054127D                 rep movsd
\end{lstlisting}

\RU{Вызвать}\EN{Call} \TT{rotate\_all\_with\_password()}:

\begin{lstlisting}
.text:0054127F                 call    rotate_all_with_password
\end{lstlisting}

\RU{Скопировать зашифрованное содержимое из \q{cube64} назад в буфер}
\EN{Copy encrypted contents back from \q{cube64} to buffer}:

\begin{lstlisting}
.text:00541284                 mov     eax, [esp+18h+arg_4]
.text:00541288                 mov     edi, ebx
.text:0054128A                 add     ebp, 40h
.text:0054128D                 add     esp, 8
.text:00541290                 mov     ecx, 10h
.text:00541295                 mov     esi, offset cube64
.text:0054129A                 add     ebx, 40h  ; add 64 to input buffer pointer
.text:0054129D                 cmp     ebp, eax  ; EBP contain amount of encrypted data.
.text:0054129F                 rep movsd
\end{lstlisting}

\RU{Если \EBP не больше чем длина во входном аргументе, тогда переходим к следующему блоку.}%
\EN{If \EBP is not bigger that the size input argument, then continue to the next block.}

\begin{lstlisting}
.text:005412A1                 jl      short loc_54126A
.text:005412A3                 pop     edi
.text:005412A4                 pop     esi
.text:005412A5                 pop     ebp
.text:005412A6                 pop     ebx
.text:005412A7                 retn
.text:005412A7 crypt           endp
\end{lstlisting}

\RU{Реконструированная функция \TT{crypt()}}\EN{Reconstructed \TT{crypt()} function}:

\begin{lstlisting}
void crypt (BYTE *buf, int sz, char *pw)
{
	int i=0;
	
	do
	{
		memcpy (cube, buf+i, 8*8);
		rotate_all (pw, 1);
		memcpy (buf+i, cube, 8*8);
		i+=64;
	}
	while (i<sz);
};
\end{lstlisting}

\RU{OK, углубимся в функцию \TT{rotate\_all\_with\_password()}. Она берет на вход два аргумента: 
строку пароля и число.}\EN{OK, now let's go deeper in function \TT{rotate\_all\_with\_password()}. 
It takes two arguments: password string and a number.}
\RU{В функции \TT{crypt()}, число 1 используется и в \TT{decrypt()} (где \TT{rotate\_all\_with\_password()}
функция вызывается также), число 3.}
\EN{In \TT{crypt()}, the number 1 is used, and in the \TT{decrypt()} function (where \TT{rotate\_all\_with\_password()} function 
is called too), the number is 3.}

\begin{lstlisting}
.text:005411B0 rotate_all_with_password proc near
.text:005411B0
.text:005411B0 arg_0           = dword ptr  4
.text:005411B0 arg_4           = dword ptr  8
.text:005411B0
.text:005411B0                 mov     eax, [esp+arg_0]
.text:005411B4                 push    ebp
.text:005411B5                 mov     ebp, eax
\end{lstlisting}

\RU{Проверяем символы в пароле. Если это ноль, выходим:}\EN{Check the current character in the password. If it is zero, exit:}

\begin{lstlisting}
.text:005411B7                 cmp     byte ptr [eax], 0
.text:005411BA                 jz      exit
.text:005411C0                 push    ebx
.text:005411C1                 mov     ebx, [esp+8+arg_4]
.text:005411C5                 push    esi
.text:005411C6                 push    edi
.text:005411C7
.text:005411C7 loop_begin:
\end{lstlisting}

\index{\CStandardLibrary!tolower()}
\RU{Вызываем \TT{tolower()}, стандартную функцию Си.}\EN{Call \TT{tolower()}, a standard C function.}

\begin{lstlisting}
.text:005411C7                 movsx   eax, byte ptr [ebp+0]
.text:005411CB                 push    eax             ; C
.text:005411CC                 call    _tolower
.text:005411D1                 add     esp, 4
\end{lstlisting}

\RU{Хмм, если пароль содержит символ не из латинского алфавита, он пропускается!
Действительно, если мы запускаем утилиту для шифрования используя символы не латинского алфавита, 
похоже, они просто игнорируются.}
\EN{Hmm, if the password contains non-Latin character, it is skipped! 
Indeed, when we run the encryption utility and try non-Latin characters in the password, 
they seem to be ignored.}

\begin{lstlisting}
.text:005411D4                 cmp     al, 'a'
.text:005411D6                 jl      short next_character_in_password
.text:005411D8                 cmp     al, 'z'
.text:005411DA                 jg      short next_character_in_password
.text:005411DC                 movsx   ecx, al
\end{lstlisting}

\RU{Отнимем значение \q{a} (97) от символа.}\EN{Subtract the value of \q{a} (97) from the character.}

\begin{lstlisting}
.text:005411DF                 sub     ecx, 'a'  ; 97
\end{lstlisting}

\RU{После вычитания, тут будет 0 для \q{a}, 1 для \q{b}, и так далее. И 25 для \q{z}.}
\EN{After subtracting, we'll get 0 for \q{a} here, 1 for \q{b}, etc. And 25 for \q{z}.}

\begin{lstlisting}
.text:005411E2                 cmp     ecx, 24
.text:005411E5                 jle     short skip_subtracting
.text:005411E7                 sub     ecx, 24
\end{lstlisting}

\RU{Похоже, символы \q{y} и \q{z} также исключительные.
После этого фрагмента кода, \q{y} становится 0, а \q{z} ~--- 1.
Это значит, что 26 латинских букв становятся значениями в интервале 0..23, (всего 24).}
\EN{It seems, \q{y} and \q{z} are exceptional characters too. 
After that fragment of code, \q{y} becomes 0 and \q{z}~---1. 
This implies that the 26 Latin alphabet symbols become values in the range of 0..23, (24 in total).}

\begin{lstlisting}
.text:005411EA
.text:005411EA skip_subtracting:                       ; CODE XREF: rotate_all_with_password+35
\end{lstlisting}

\RU{Это, на самом деле, деление через умножение.
Читайте об этом больше в секции \q{\DivisionByNineSectionName}~(\myref{sec:divisionbynine}).}
\EN{This is actually division via multiplication. 
You can read more about it in the \q{\DivisionByNineSectionName} section~(\myref{sec:divisionbynine}).}

\RU{Это код, на самом деле, делит значение символа пароля на 3.}
\EN{The code actually divides the password character's value by 3.}
% TODO1: add Mathematica calculations
\begin{lstlisting}
.text:005411EA                 mov     eax, 55555556h
.text:005411EF                 imul    ecx
.text:005411F1                 mov     eax, edx
.text:005411F3                 shr     eax, 1Fh
.text:005411F6                 add     edx, eax
.text:005411F8                 mov     eax, ecx
.text:005411FA                 mov     esi, edx
.text:005411FC                 mov     ecx, 3
.text:00541201                 cdq
.text:00541202                 idiv    ecx
\end{lstlisting}

\RU{\EDX\EMDASH{}остаток от деления.}\EN{\EDX is the remainder of the division.}

\lstinputlisting{examples/qr9/4.\LANG}

\RU{Если остаток 2, вызываем \TT{rotate3()}. 
\EDX это второй аргумент функции \TT{rotate\_all\_with\_password()}. 
Как мы уже заметили, 1 это для шифрования, 3 для дешифрования.
Так что здесь цикл, функции rotate1/2/3 будут вызываться столько же раз, сколько значение переменной
в первом аргументе.}
\EN{If the remainder is 2, call \TT{rotate3()}. 
\EDI is the second argument of the \TT{rotate\_all\_with\_password()} function.
As we already noted, 1 is for the encryption operations and 3 is for the decryption. 
So, here is a loop. When encrypting, rotate1/2/3 are to be called the same number of times as 
given in the first argument.}

\begin{lstlisting}
.text:00541215 call_rotate3:
.text:00541215                 push    esi
.text:00541216                 call    rotate3
.text:0054121B                 add     esp, 4
.text:0054121E                 dec     edi
.text:0054121F                 jnz     short call_rotate3
.text:00541221                 jmp     short next_character_in_password
.text:00541223
.text:00541223 call_rotate2:
.text:00541223                 test    ebx, ebx
.text:00541225                 jle     short next_character_in_password
.text:00541227                 mov     edi, ebx
.text:00541229
.text:00541229 loc_541229:
.text:00541229                 push    esi
.text:0054122A                 call    rotate2
.text:0054122F                 add     esp, 4
.text:00541232                 dec     edi
.text:00541233                 jnz     short loc_541229
.text:00541235                 jmp     short next_character_in_password
.text:00541237
.text:00541237 call_rotate1:
.text:00541237                 test    ebx, ebx
.text:00541239                 jle     short next_character_in_password
.text:0054123B                 mov     edi, ebx
.text:0054123D
.text:0054123D loc_54123D:
.text:0054123D                 push    esi
.text:0054123E                 call    rotate1
.text:00541243                 add     esp, 4
.text:00541246                 dec     edi
.text:00541247                 jnz     short loc_54123D
.text:00541249
\end{lstlisting}

\RU{Достать следующий символ из строки пароля.}\EN{Fetch the next character from the password string.}

\begin{lstlisting}
.text:00541249 next_character_in_password:
.text:00541249                 mov     al, [ebp+1]
\end{lstlisting}

\RU{\glslink{increment}{Инкремент} указателя на символ в строке пароля:}\EN{\Gls{increment} the character pointer in the password string:}

\begin{lstlisting}
.text:0054124C                 inc     ebp
.text:0054124D                 test    al, al
.text:0054124F                 jnz     loop_begin
.text:00541255                 pop     edi
.text:00541256                 pop     esi
.text:00541257                 pop     ebx
.text:00541258
.text:00541258 exit:
.text:00541258                 pop     ebp
.text:00541259                 retn
.text:00541259 rotate_all_with_password endp
\end{lstlisting}

\RU{Реконструированный код на Си:}\EN{Here is the reconstructed C code:}

\begin{lstlisting}
void rotate_all (char *pwd, int v)
{
	char *p=pwd;

	while (*p)
	{
		char c=*p;
		int q;

		c=tolower (c);

		if (c>='a' && c<='z')
		{
			q=c-'a';
			if (q>24)
				q-=24;

			int quotient=q/3;
			int remainder=q % 3;

			switch (remainder)
			{
			case 0: for (int i=0; i<v; i++) rotate1 (quotient); break;
			case 1: for (int i=0; i<v; i++) rotate2 (quotient); break;
			case 2: for (int i=0; i<v; i++) rotate3 (quotient); break;
			};
		};

		p++;
	};
};
\end{lstlisting}

\RU{Углубимся еще дальше и исследуем функции rotate1/2/3.
Каждая функция вызывает еще две.
В итоге мы назовем их \TT{set\_bit()} и \TT{get\_bit()}.}%
\EN{Now let's go deeper and investigate the rotate1/2/3 functions. 
Each function calls another two functions. 
We eventually will name them \TT{set\_bit()} and \TT{get\_bit()}.}

\RU{Начнем с \TT{get\_bit()}:}\EN{Let's start with \TT{get\_bit()}:}

\begin{lstlisting}
.text:00541050 get_bit         proc near
.text:00541050
.text:00541050 arg_0           = dword ptr  4
.text:00541050 arg_4           = dword ptr  8
.text:00541050 arg_8           = byte ptr  0Ch
.text:00541050
.text:00541050                 mov     eax, [esp+arg_4]
.text:00541054                 mov     ecx, [esp+arg_0]
.text:00541058                 mov     al, cube64[eax+ecx*8]
.text:0054105F                 mov     cl, [esp+arg_8]
.text:00541063                 shr     al, cl
.text:00541065                 and     al, 1
.text:00541067                 retn
.text:00541067 get_bit         endp
\end{lstlisting}

\RU{\dots иными словами: подсчитать индекс в массиве cube64}\EN{\dots in other words: calculate an index in 
the cube64 array}: \IT{arg\_4 + arg\_0 * 8}.
\RU{Затем сдвинуть байт из массива вправо на количество бит заданных в arg\_8. 
Изолировать самый младший бит и вернуть его}\EN{Then shift a byte from the array by arg\_8 bits right. 
Isolate the lowest bit and return it.}

\RU{Посмотрим другую функцию}\EN{Let's see another function}, \TT{set\_bit()}:

\begin{lstlisting}
.text:00541000 set_bit         proc near
.text:00541000
.text:00541000 arg_0           = dword ptr  4
.text:00541000 arg_4           = dword ptr  8
.text:00541000 arg_8           = dword ptr  0Ch
.text:00541000 arg_C           = byte ptr  10h
.text:00541000
.text:00541000                 mov     al, [esp+arg_C]
.text:00541004                 mov     ecx, [esp+arg_8]
.text:00541008                 push    esi
.text:00541009                 mov     esi, [esp+4+arg_0]
.text:0054100D                 test    al, al
.text:0054100F                 mov     eax, [esp+4+arg_4]
.text:00541013                 mov     dl, 1
.text:00541015                 jz      short loc_54102B
\end{lstlisting}

\RU{\TT{DL} тут равно 1. Сдвигаем эту единицу на количество, указанное в arg\_8. Например, если в arg\_8 число 4,
тогда значение в \TT{DL} станет 0x10 или 1000b в двоичной системе счисления.}
\EN{The value in the \TT{DL} is 1 here. It gets shifted left by arg\_8.
For example, if arg\_8 is 4, the value in the \TT{DL} register is to be 
0x10 or 1000b in binary form.}

\begin{lstlisting}
.text:00541017                 shl     dl, cl
.text:00541019                 mov     cl, cube64[eax+esi*8]
\end{lstlisting}

\RU{Вытащить бит из массива и явно выставить его.}\EN{Get a bit from array and explicitly set it.} % TODO1: rewrite

\begin{lstlisting}
.text:00541020                 or      cl, dl
\end{lstlisting}

\RU{Сохранить его назад:}\EN{Store it back:} % TODO1: rewrite

\begin{lstlisting}
.text:00541022                 mov     cube64[eax+esi*8], cl
.text:00541029                 pop     esi
.text:0054102A                 retn
.text:0054102B
.text:0054102B loc_54102B:
.text:0054102B                 shl     dl, cl
\end{lstlisting}

\RU{Если arg\_C не ноль\dots}\EN{If arg\_C is not zero\dots}

\begin{lstlisting}
.text:0054102D                 mov     cl, cube64[eax+esi*8]
\end{lstlisting}

\index{x86!\Instructions!NOT}
\RU{\dots инвертировать DL. Например, если состояние DL после сдвига 0x10 или 1000b в двоичной системе,
здесь будет 0xEF после инструкции \NOT или 11101111b в двоичной системе.}
\EN{\dots invert DL. For example, if DL's state after the shift was 0x10 or 1000b in binary form, 
there is 0xEF to be after the \NOT instruction (or 11101111b in binary form).}

\begin{lstlisting}
.text:00541034                 not     dl
\end{lstlisting}

\RU{Эта инструкция сбрасывает бит, иными словами, она сохраняет все биты в \TT{CL} которые также
выставлены в \TT{DL} кроме тех в \TT{DL}, что были сброшены. Это значит, что если в \TT{DL}, например,
11101111b в двоичной системе, все биты будут сохранены кроме пятого (считая с младшего бита).}
\EN{This instruction clears the bit, in other words, it saves all bits in \TT{CL} which are also set in 
\TT{DL} except those in \TT{DL} which are cleared.
This implies that if \TT{DL} is 11101111b in binary form,
all bits are to be saved except the 5th (counting from lowest bit).}

\begin{lstlisting}
.text:00541036                 and     cl, dl
\end{lstlisting}

\RU{Сохранить его назад}\EN{Store it back:}

\begin{lstlisting}
.text:00541038                 mov     cube64[eax+esi*8], cl
.text:0054103F                 pop     esi
.text:00541040                 retn
.text:00541040 set_bit         endp
\end{lstlisting}

\RU{Это почти то же самое что и \TT{get\_bit()}, кроме того, что если arg\_C ноль, тогда функция сбрасывает
указанный бит в массиве, либо же, в противном случае, выставляет его в 1.}
\EN{It is almost the same as \TT{get\_bit()}, except, if arg\_C is zero, the function clears the specific bit in the array, 
or sets it otherwise.}

\RU{Мы также знаем что размер массива 64. Первые два аргумента и у \TT{set\_bit()} и у \TT{get\_bit()}
могут быть представлены как двумерные координаты. Таким образом, массив ~--- это матрица 8*8.}
\EN{We also know that the array's size is 64. The first two arguments both in the \TT{set\_bit()} and \TT{get\_bit()} functions
could be seen as 2D coordinates. Then the array is to be an 8*8 matrix.}

\RU{Представление на Си всего того, что мы уже знаем:}\EN{Here is a C representation of what we know up to now:}

\begin{lstlisting}
#define IS_SET(flag, bit)       ((flag) & (bit))
#define SET_BIT(var, bit)       ((var) |= (bit))
#define REMOVE_BIT(var, bit)    ((var) &= ~(bit))

static BYTE cube[8][8];

void set_bit (int x, int y, int shift, int bit)
{
	if (bit)
		SET_BIT (cube[x][y], 1<<shift);
	else
		REMOVE_BIT (cube[x][y], 1<<shift);
};

bool get_bit (int x, int y, int shift)
{
	if ((cube[x][y]>>shift)&1==1)
		return 1;
	return 0;
};
\end{lstlisting}

\RU{Теперь вернемся к функциям rotate1/2/3.}\EN{Now let's get back to the rotate1/2/3 functions.}

\begin{lstlisting}
.text:00541070 rotate1         proc near
.text:00541070
\end{lstlisting}

\RU{Выделение внутреннего массива размером 64 байта в локальном стеке:}
\EN{Internal array allocation in the local stack, with size of 64 bytes:}

\begin{lstlisting}
.text:00541070 internal_array_64= byte ptr -40h
.text:00541070 arg_0           = dword ptr  4
.text:00541070
.text:00541070                 sub     esp, 40h
.text:00541073                 push    ebx
.text:00541074                 push    ebp
.text:00541075                 mov     ebp, [esp+48h+arg_0]
.text:00541079                 push    esi
.text:0054107A                 push    edi
.text:0054107B                 xor     edi, edi        ; EDI is loop1 counter
\end{lstlisting}

\EBX \RU{указывает на внутренний массив}\EN{is a pointer to the internal array:}

\begin{lstlisting}
.text:0054107D                 lea     ebx, [esp+50h+internal_array_64]
.text:00541081
\end{lstlisting}

\RU{Здесь два вложенных цикла:}\EN{Here we have two nested loops:}

\lstinputlisting{examples/qr9/5.\LANG}

\RU{Мы видим, что оба счетчика циклов в интервале 0..7. 
Также, они используются как первый и второй аргумент \TT{get\_bit()}.
Третий аргумент \TT{get\_bit()} это единственный аргумент \TT{rotate1()}. 
То что возвращает \TT{get\_bit()} будет сохранено во внутреннем массиве.}
\EN{\dots we see that both loops' counters are in the range of 0..7. 
Also they are used as the first and second argument for the \TT{get\_bit()} function.
The third argument to \TT{get\_bit()} is the only argument of \TT{rotate1()}. 
The return value from \TT{get\_bit()} is placed in the internal array.}

\RU{Снова приготовить указатель на внутренний массив:}\EN{Prepare a pointer to the internal array again:}

\lstinputlisting{examples/qr9/6.\LANG}

\RU{\dots этот код помещает содержимое из внутреннего массива в глобальный массив cube используя функцию 
\TT{set\_bit()}, \IT{но}, в обратном порядке!
Теперь счетчик первого цикла в интервале 7 до 0, уменьшается на 1 на каждой итерации!}
\EN{\dots this code is placing the contents of the internal array to the cube global array via the \TT{set\_bit()} function, 
\IT{but} in a different order!
Now the counter of the first loop is in the range of 7 to 0, \glslink{decrement}{decrementing} at each iteration!}

\RU{Представление кода на Си выглядит так:}\EN{The C code representation looks like:}

\begin{lstlisting}
void rotate1 (int v)
{
	bool tmp[8][8]; // internal array
	int i, j;

	for (i=0; i<8; i++)
		for (j=0; j<8; j++)
			tmp[i][j]=get_bit (i, j, v);

	for (i=0; i<8; i++)
		for (j=0; j<8; j++)
			set_bit (j, 7-i, v, tmp[x][y]);
};
\end{lstlisting}

\RU{Не очень понятно, но если мы посмотрим в функцию \TT{rotate2()}:}
\EN{Not very understandable, but if we take a look at \TT{rotate2()} function:}

\lstinputlisting{examples/qr9/7.\LANG}

\RU{\IT{Почти} то же самое, за исключением иного порядка аргументов в \TT{get\_bit()} и \TT{set\_bit()}.
Перепишем это на Си-подобный код:}
\EN{It is \IT{almost} the same, except the order of the arguments of the \TT{get\_bit()} and \TT{set\_bit()} is different. 
Let's rewrite it in C-like code:}

\begin{lstlisting}
void rotate2 (int v)
{
	bool tmp[8][8]; // internal array
	int i, j;

	for (i=0; i<8; i++)
		for (j=0; j<8; j++)
			tmp[i][j]=get_bit (v, i, j);

	for (i=0; i<8; i++)
		for (j=0; j<8; j++)
			set_bit (v, j, 7-i, tmp[i][j]);
};
\end{lstlisting}

\RU{Перепишем так же функцию \TT{rotate3()}:}\EN{Let's also rewrite the \TT{rotate3()} function:}

\begin{lstlisting}
void rotate3 (int v)
{
	bool tmp[8][8];
	int i, j;

	for (i=0; i<8; i++)
		for (j=0; j<8; j++)
			tmp[i][j]=get_bit (i, v, j);

	for (i=0; i<8; i++)
		for (j=0; j<8; j++)
			set_bit (7-j, v, i, tmp[i][j]);
};
\end{lstlisting}

\RU{Теперь всё проще. Если мы представим cube64 как трехмерный куб 8*8*8, где каждый элемент это бит,
то \TT{get\_bit()} и \TT{set\_bit()} просто берут на вход координаты бита.}
\EN{Well, now things are simpler. If we consider cube64 as a 3D cube of size 8*8*8, where each element is a bit, 
\TT{get\_bit()} and \TT{set\_bit()} take just the coordinates of a bit as input.}

\RU{Функции rotate1/2/3 просто поворачивают все биты на определенной плоскости.
Три функции, каждая на каждую сторону куба и аргумент \TT{v} выставляет плоскость в интервале 0..7}
\EN{The rotate1/2/3 functions are in fact rotating all bits in a specific plane. 
These three functions are one for each cube side and the \TT{v} argument sets the plane in the range of 0..7.}


\RU{Может быть, автор алгоритма думал о кубике Рубика 8*8*8}
\EN{Maybe, the algorithm's author was thinking of a 8*8*8 Rubik's cube}
\footnote{\href{http://go.yurichev.com/17115}{wikipedia}}?!

\RU{Да, действительно.}\EN{Yes, indeed.}

\RU{Рассмотрим функцию \TT{decrypt()}, вот её переписанная версия:}%
\EN{Let's look closer into the \TT{decrypt()} function, here is its rewritten version:}

\begin{lstlisting}
void decrypt (BYTE *buf, int sz, char *pw)
{
	char *p=strdup (pw);
	strrev (p);
	int i=0;

	do
	{
		memcpy (cube, buf+i, 8*8);
		rotate_all (p, 3);
		memcpy (buf+i, cube, 8*8);
		i+=64;
	}
	while (i<sz);
	
	free (p);
};
\end{lstlisting}


\RU{Почти то же самое что и crypt(), \IT{но} строка пароля разворачивается стандартной функцией Си}
\EN{It is almost the same as for \TT{crypt()}, \IT{but} the password string is reversed by the}
strrev() \footnote{\href{http://go.yurichev.com/17249}{MSDN}}
\RU{и \TT{rotate\_all()} вызывается с аргументом 3.}
\EN{standard C function and \TT{rotate\_all()} is called with argument 3.} 

\RU{Это значит, что, в случае дешифровки, rotate1/2/3 будут вызываться трижды.}
\EN{This implies that in case of decryption, each corresponding rotate1/2/3 call is to be performed thrice.}

\RU{Это почти кубик Рубика!
Если вы хотите вернуть его состояние назад, делайте то же самое в обратном порядке и направлении!
Чтобы вернуть эффект от поворота плоскости по часовой стрелке, нужно повернуть её же против 
часовой стрелки, либо же трижды по часовой стрелке.}
\EN{This is almost as in Rubik'c cube! 
If you want to get back, do the same in reverse order and direction! 
If you need to undo the effect of rotating one place in clockwise direction, 
rotate it once in counter-clockwise direction, or thrice in clockwise direction.}

\RU{\TT{rotate1()}, вероятно, поворот \q{лицевой} плоскости. 
\TT{rotate2()}, вероятно, поворот \q{верхней} плоскости.
\TT{rotate3()}, вероятно, поворот \q{левой} плоскости.}
\EN{\TT{rotate1()} is apparently for rotating the \q{front} plane. 
\TT{rotate2()} is apparently for rotating the \q{top} plane. 
\TT{rotate3()} is apparently for rotating the \q{left} plane.}

\RU{Вернемся к ядру функции \TT{rotate\_all()}}\EN{Let's get back to the core of the \TT{rotate\_all()} function:}

\begin{lstlisting}
q=c-'a';
if (q>24)
	q-=24;

int quotient=q/3; // in range 0..7
int remainder=q % 3;

switch (remainder)
{
    case 0: for (int i=0; i<v; i++) rotate1 (quotient); break; // front
    case 1: for (int i=0; i<v; i++) rotate2 (quotient); break; // top
    case 2: for (int i=0; i<v; i++) rotate3 (quotient); break; // left
};
\end{lstlisting}

\RU{Так понять проще: каждый символ пароля определяет сторону (одну из трех) и плоскость (одну из восьми).
3*8 = 24, вот почему два последних символа латинского алфавита переопределяются так чтобы алфавит состоял
из 24-х элементов.}
\EN{Now it is much simpler to understand: each password character defines a side (one of three) and a plane (one of 8). 
3*8 = 24, that is why two the last two characters of the Latin alphabet are remapped to fit an alphabet of exactly 
24 elements.}

\RU{Алгоритм очевидно слаб: в случае коротких паролей, в бинарном редакторе файлов можно будет увидеть, 
что в зашифрованных файлах остались незашифрованные символы.}
\EN{The algorithm is clearly weak: in case of short passwords you can see
that in the encrypted file there are 
the original bytes of the original file in a binary file editor.}

\RU{Весь исходный код в реконструированном виде:}\EN{Here is the whole source code reconstructed:}

\lstinputlisting{examples/qr9/qr9.cpp}



\chapter{SAP}

\RU{\input{examples/SAP/sapgui/sapgui_RU}}
\RU{\input{examples/SAP/sapgui/sapgui_EN}}
\section{\RU{Функции проверки пароля в SAP 6.0}\EN{SAP 6.0 password checking functions}}

\index{SAP}
\RU{Когда автор этой книги в очередной раз вернулся к своему SAP 6.0 IDES заинсталлированному в виртуальной машине VMware, он обнаружил что забыл пароль, впрочем, затем он вспомнил его, но теперь получаем такую ошибку:}
\EN{One time when the author of this book have returned again to his SAP 6.0 IDES installed in a VMware box, he figured out that he forgot the password for the SAP* account, then he have remembered it, but then we got this error message} 
\IT{<<Password logon no longer possible - too many failed attempts>>}, 
\RU{потому что были потрачены все попытки на то, чтобы вспомнить его}
\EN{since he've made all these attempts in trying to recall it}.

\index{Windows!PDB}
\RU{Первая очень хорошая новость состоит в том, что с SAP поставляется полный \gls{PDB}-файл \IT{disp+work.pdb}, он содержит все: имена функций, структуры, типы, локальные переменные, имена аргументов, \etc{}. Какой щедрый подарок!}
\EN{The first extremely good news was that the full \IT{disp+work.pdb} \gls{PDB} file is supplied with SAP, and it contain almost everything: function names, structures, types, local variable and argument names, \etc{}. What a lavish gift!}

\RU{Существует утилита}\EN{There is} TYPEINFODUMP\footnote{\url{http://go.yurichev.com/17038}} \RU{для дампа содержимого \gls{PDB}-файлов во что-то более читаемое и grep-абельное}\EN{utility for converting \gls{PDB} files into something readable and grepable}.

\RU{Вот пример её работы: информация о функции + её аргументах + её локальных переменных:}\EN{Here is an example of a function information + its arguments + its local variables:}

\begin{lstlisting}
FUNCTION ThVmcSysEvent 
  Address:         10143190  Size:      675 bytes  Index:    60483  TypeIndex:    60484 
  Type: int NEAR_C ThVmcSysEvent (unsigned int, unsigned char, unsigned short*)
Flags: 0 
PARAMETER events 
  Address: Reg335+288  Size:        4 bytes  Index:    60488  TypeIndex:    60489 
  Type: unsigned int
Flags: d0 
PARAMETER opcode 
  Address: Reg335+296  Size:        1 bytes  Index:    60490  TypeIndex:    60491 
  Type: unsigned char
Flags: d0 
PARAMETER serverName 
  Address: Reg335+304  Size:        8 bytes  Index:    60492  TypeIndex:    60493 
  Type: unsigned short*
Flags: d0 
STATIC_LOCAL_VAR func 
  Address:         12274af0  Size:        8 bytes  Index:    60495  TypeIndex:    60496 
  Type: wchar_t*
Flags: 80 
LOCAL_VAR admhead 
  Address: Reg335+304  Size:        8 bytes  Index:    60498  TypeIndex:    60499 
  Type: unsigned char*
Flags: 90 
LOCAL_VAR record 
  Address: Reg335+64  Size:      204 bytes  Index:    60501  TypeIndex:    60502 
  Type: AD_RECORD
Flags: 90 
LOCAL_VAR adlen 
  Address: Reg335+296  Size:        4 bytes  Index:    60508  TypeIndex:    60509 
  Type: int
Flags: 90 
\end{lstlisting}

\RU{А вот пример дампа структуры:}\EN{And here is an example of some structure:}

\begin{lstlisting}
STRUCT DBSL_STMTID 
Size: 120  Variables: 4  Functions: 0  Base classes: 0
MEMBER moduletype 
  Type:  DBSL_MODULETYPE
  Offset:        0  Index:        3  TypeIndex:    38653
MEMBER module 
  Type:  wchar_t module[40]
  Offset:        4  Index:        3  TypeIndex:      831
MEMBER stmtnum 
  Type:  long
  Offset:       84  Index:        3  TypeIndex:      440
MEMBER timestamp 
  Type:  wchar_t timestamp[15]
  Offset:       88  Index:        3  TypeIndex:     6612
\end{lstlisting}

\RU{Вау!}\EN{Wow!}

\RU{Вторая хорошая новость: \IT{отладочные} вызовы, коих здесь очень много, очень полезны.}\EN{Another good news: \IT{debugging} calls (there are plenty of them) are very useful.} 

\RU{Здесь вы можете увидеть глобальную переменную \IT{ct\_level}}\EN{Here you can also notice the \IT{ct\_level} global variable}\footnote{\RU{Еще об уровне трассировки}\EN{More about trace level}: \url{http://go.yurichev.com/17039}}, \RU{отражающую уровень трассировки}\EN{that reflects the current trace level}.

\RU{В \IT{disp+work.exe} очень много таких отладочных вставок}\EN{There are a lot of debugging inserts in the \IT{disp+work.exe} file}:

\begin{lstlisting}
cmp     cs:ct_level, 1
jl      short loc_1400375DA
call    DpLock
lea     rcx, aDpxxtool4_c ; "dpxxtool4.c"
mov     edx, 4Eh        ; line
call    CTrcSaveLocation
mov     r8, cs:func_48
mov     rcx, cs:hdl     ; hdl
lea     rdx, aSDpreadmemvalu ; "%s: DpReadMemValue (%d)"
mov     r9d, ebx
call    DpTrcErr
call    DpUnlock
\end{lstlisting}

\RU{Если текущий уровень трассировки выше или равен заданному в этом коде порогу, 
отладочное сообщение будет записано в лог-файл вроде \IT{dev\_w0}, \IT{dev\_disp} 
и прочие файлы \IT{dev*}.}
\EN{If the current trace level is bigger or equal to threshold defined in the code here, 
a debugging message is to be written to the log files like \IT{dev\_w0}, \IT{dev\_disp}, 
and other \IT{dev*} files.}

\index{\GrepUsage}
\RU{Попробуем grep-ать файл недавно полученный при помощи утилиты TYPEINFODUMP:}
\EN{Let's try grepping in the file that we have got with the help of the TYPEINFODUMP utility:}

\begin{lstlisting}
cat "disp+work.pdb.d" | grep FUNCTION | grep -i password
\end{lstlisting}

\RU{Получаем:}\EN{We have got:}

\begin{lstlisting}
FUNCTION rcui::AgiPassword::DiagISelection 
FUNCTION ssf_password_encrypt 
FUNCTION ssf_password_decrypt 
FUNCTION password_logon_disabled 
FUNCTION dySignSkipUserPassword 
FUNCTION migrate_password_history 
FUNCTION password_is_initial 
FUNCTION rcui::AgiPassword::IsVisible 
FUNCTION password_distance_ok 
FUNCTION get_password_downwards_compatibility 
FUNCTION dySignUnSkipUserPassword 
FUNCTION rcui::AgiPassword::GetTypeName 
FUNCTION `rcui::AgiPassword::AgiPassword'::`1'::dtor$2 
FUNCTION `rcui::AgiPassword::AgiPassword'::`1'::dtor$0 
FUNCTION `rcui::AgiPassword::AgiPassword'::`1'::dtor$1 
FUNCTION usm_set_password 
FUNCTION rcui::AgiPassword::TraceTo 
FUNCTION days_since_last_password_change 
FUNCTION rsecgrp_generate_random_password 
FUNCTION rcui::AgiPassword::`scalar deleting destructor' 
FUNCTION password_attempt_limit_exceeded 
FUNCTION handle_incorrect_password 
FUNCTION `rcui::AgiPassword::`scalar deleting destructor''::`1'::dtor$1 
FUNCTION calculate_new_password_hash 
FUNCTION shift_password_to_history 
FUNCTION rcui::AgiPassword::GetType 
FUNCTION found_password_in_history 
FUNCTION `rcui::AgiPassword::`scalar deleting destructor''::`1'::dtor$0 
FUNCTION rcui::AgiObj::IsaPassword 
FUNCTION password_idle_check 
FUNCTION SlicHwPasswordForDay 
FUNCTION rcui::AgiPassword::IsaPassword 
FUNCTION rcui::AgiPassword::AgiPassword 
FUNCTION delete_user_password 
FUNCTION usm_set_user_password 
FUNCTION Password_API 
FUNCTION get_password_change_for_SSO 
FUNCTION password_in_USR40 
FUNCTION rsec_agrp_abap_generate_random_password 
\end{lstlisting}

\RU{Попробуем так же искать отладочные сообщения содержащие слова \IT{<<password>>} и \IT{<<locked>>}.}\EN{Let's also try to search for debug messages which contain the words \IT{<<password>>} and \IT{<<locked>>}.}
\RU{Одна из таких это строка}\EN{One of them is the string} \IT{<<user was locked by subsequently failed password logon attempts>>} \RU{на которую есть ссылка в}\EN{, referenced in} \\
\RU{функции}\EN{function} \IT{password\_attempt\_limit\_exceeded()}.

\RU{Другие строки, которые эта найденная функция может писать в лог-файл это}
\EN{Other strings that this function can write to a log file are}: 
\IT{<<password logon attempt will be rejected immediately (preventing dictionary attacks)>>}, \IT{<<failed-logon lock: expired (but not removed due to 'read-only' operation)>>}, \IT{<<failed-logon lock: expired => removed>>}.

\RU{Немного поэкспериментировав с этой функцией, мы быстро понимаем что проблема именно в ней}%
\EN{After playing for a little with this function, we noticed that the problem is exactly in it}.
\RU{Она вызывается из функции \IT{chckpass()}\EMDASH{}одна из функций проверяющих пароль}\EN{It is called from the \IT{chckpass()} function~---one of the password checking functions}.

\RU{В начале, давайте убедимся, что мы на верном пути}%
\EN{First, we would like to make sure that we are at the correct point}:

\RU{Запускаем}\EN{Run} \tracer:
\index{tracer}

\begin{lstlisting}
tracer64.exe -a:disp+work.exe bpf=disp+work.exe!chckpass,args:3,unicode
\end{lstlisting}

\begin{lstlisting}
PID=2236|TID=2248|(0) disp+work.exe!chckpass (0x202c770, L"Brewered1                               ", 0x41) (called from 0x1402f1060 (disp+work.exe!usrexist+0x3c0))
PID=2236|TID=2248|(0) disp+work.exe!chckpass -> 0x35
\end{lstlisting}

\RU{Функции вызываются так}\EN{The call path is}: \IT{syssigni()} -> \IT{DyISigni()} -> \IT{dychkusr()} -> \IT{usrexist()} -> \IT{chckpass()}.

\RU{Число}\EN{The number} 0x35 \RU{возвращается из}\EN{is an error returned in} \IT{chckpass()} \RU{в этом месте}\EN{at that point}:

\begin{lstlisting}
.text:00000001402ED567 loc_1402ED567:                          ; CODE XREF: chckpass+B4
.text:00000001402ED567                 mov     rcx, rbx        ; usr02
.text:00000001402ED56A                 call    password_idle_check
.text:00000001402ED56F                 cmp     eax, 33h
.text:00000001402ED572                 jz      loc_1402EDB4E
.text:00000001402ED578                 cmp     eax, 36h
.text:00000001402ED57B                 jz      loc_1402EDB3D
.text:00000001402ED581                 xor     edx, edx        ; usr02_readonly
.text:00000001402ED583                 mov     rcx, rbx        ; usr02
.text:00000001402ED586                 call    password_attempt_limit_exceeded
.text:00000001402ED58B                 test    al, al
.text:00000001402ED58D                 jz      short loc_1402ED5A0
.text:00000001402ED58F                 mov     eax, 35h
.text:00000001402ED594                 add     rsp, 60h
.text:00000001402ED598                 pop     r14
.text:00000001402ED59A                 pop     r12
.text:00000001402ED59C                 pop     rdi
.text:00000001402ED59D                 pop     rsi
.text:00000001402ED59E                 pop     rbx
.text:00000001402ED59F                 retn
\end{lstlisting}

\RU{Отлично, давайте проверим}\EN{Fine, let's check}:

\begin{lstlisting}
tracer64.exe -a:disp+work.exe bpf=disp+work.exe!password_attempt_limit_exceeded,args:4,unicode,rt:0
\end{lstlisting}

\begin{lstlisting}
PID=2744|TID=360|(0) disp+work.exe!password_attempt_limit_exceeded (0x202c770, 0, 0x257758, 0) (called from 0x1402ed58b (disp+work.exe!chckpass+0xeb))
PID=2744|TID=360|(0) disp+work.exe!password_attempt_limit_exceeded -> 1
PID=2744|TID=360|We modify return value (EAX/RAX) of this function to 0
PID=2744|TID=360|(0) disp+work.exe!password_attempt_limit_exceeded (0x202c770, 0, 0, 0) (called from 0x1402e9794 (disp+work.exe!chngpass+0xe4))
PID=2744|TID=360|(0) disp+work.exe!password_attempt_limit_exceeded -> 1
PID=2744|TID=360|We modify return value (EAX/RAX) of this function to 0
\end{lstlisting}

\RU{Великолепно! Теперь мы можем успешно залогиниться.}\EN{Excellent! We can successfully login now.}

\RU{Кстати, мы можем сделать вид что вообще забыли пароль, заставляя \IT{chckpass()} всегда возвращать ноль, и этого достаточно для отключения проверки пароля:}
\EN{By the way, we can pretend we forgot the password, fixing the \IT{chckpass()} function to return a value of 0 is enough to bypass the check:}

\begin{lstlisting}
tracer64.exe -a:disp+work.exe bpf=disp+work.exe!chckpass,args:3,unicode,rt:0
\end{lstlisting}

\begin{lstlisting}
PID=2744|TID=360|(0) disp+work.exe!chckpass (0x202c770, L"bogus                                   ", 0x41) (called from 0x1402f1060 (disp+work.exe!usrexist+0x3c0))
PID=2744|TID=360|(0) disp+work.exe!chckpass -> 0x35
PID=2744|TID=360|We modify return value (EAX/RAX) of this function to 0
\end{lstlisting}

\RU{Что еще можно сказать, бегло анализируя функцию \IT{password\_attempt\_limit\_exceeded()}, это то, что в начале
можно увидеть следующий вызов:}\EN{What also can be said while analyzing the 
\IT{password\_attempt\_limit\_exceeded()} 
function is that at the very beginning of it, this call can be seen:}

\begin{lstlisting}
lea     rcx, aLoginFailed_us ; "login/failed_user_auto_unlock"
call    sapgparam
test    rax, rax
jz      short loc_1402E19DE
movzx   eax, word ptr [rax]
cmp     ax, 'N'
jz      short loc_1402E19D4
cmp     ax, 'n'
jz      short loc_1402E19D4
cmp     ax, '0'
jnz     short loc_1402E19DE
\end{lstlisting}

\RU{Очевидно, функция \IT{sapgparam()} используется чтобы узнать значение какой-либо переменной конфигурации. Эта функция может вызываться из 1768 разных мест.}
\EN{Obviously, function \IT{sapgparam()} is used to query the value of some configuration parameter. This function can be called from 1768 different places.}
\RU{Вероятно, при помощи этой информации, мы можем легко находить те места кода, на которые влияют определенные переменные конфигурации.}\EN{It seems that with the help of this information, we can easily find the places in code, the control flow of which can be affected by specific configuration parameters.}

\RU{Замечательно! Имена функций очень понятны, куда понятнее чем в \oracle.}
\EN{It is really sweet. The function names are very clear, much clearer than in the \oracle.} 
\index{\Cpp}
\RU{По всей видимости, процесс \IT{disp+work} весь написан на \Cpp. Вероятно, он был переписан не так давно?}
\EN{It seems that the \IT{disp+work} process is written in \Cpp. It was apparently rewritten some time ago?}



\ifdefined\IncludeOracle
\chapter{\oracle}
\label{oracle}

% sections
\section{\IFRU{Таблица \TT{V\$VERSION} в \oracle}{\TT{V\$VERSION} table in the \oracle}}

\index{\oracle}
\index{Linux}
\index{Windows!ntoskrnl.exe}
\IFRU{\oracle 11.2 это очень большая программа, основной модуль \TT{oracle.exe} содержит около 124 тысячи функций.}{\oracle 11.2 is a huge program, main module \TT{oracle.exe} contain approx. 124,000 functions.} \IFRU{Для сравнения, ядро Windows 7 x64 (ntoskrnl.exe) ~--- около 11 тысяч функций, а ядро Linux 3.9.8 (с драйверами по умолчанию) ~--- 31 тысяч функций.}{For comparison, Windows 7 x86 kernel (ntoskrnl.exe)~---approx. 11,000 functions and Linux 3.9.8 kernel (with default drivers compiled)~---31,000 functions.}

\IFRU{Начнем с одного простого вопроса. Откуда \oracle берет информацию, когда мы в SQL*Plus пишем вот такой вот простой запрос:}{Let's start with an easy question. Where \oracle get all this information, when we execute such simple statement in SQL*Plus:}

\begin{lstlisting}
SQL> select * from V$VERSION;
\end{lstlisting}

\IFRU{И получаем}{And we've got}:

\begin{lstlisting}
BANNER
--------------------------------------------------------------------------------

Oracle Database 11g Enterprise Edition Release 11.2.0.1.0 - Production
PL/SQL Release 11.2.0.1.0 - Production
CORE    11.2.0.1.0      Production
TNS for 32-bit Windows: Version 11.2.0.1.0 - Production
NLSRTL Version 11.2.0.1.0 - Production
\end{lstlisting}

\IFRU{Начнем. Где в самом \oracle мы можем найти строку}{Let's start. Where in the \oracle we may find a string} \TT{V\$VERSION}?

\IFRU{Для win32-версии, эта строка имеется в файле \TT{oracle.exe}, это легко увидеть.}
{As of win32-version, \TT{oracle.exe} file contain the string,
which can be investigated easily.}
\IFRU{Но мы так же можем использовать объектные (.o) файлы от версии \oracle для Linux, потому что в них сохраняются имена функций и глобальных переменных, а в \TT{oracle.exe} для win32 этого нет.}{But we can also use object (.o) files from Linux version of \oracle since, unlike win32 version \TT{oracle.exe}, function names (and global variables as well) are preserved there.}

\IFRU{Итак, строка \TT{V\$VERSION} имеется в файле \TT{kqf.o}, в самой главной Oracle-библиотеке \TT{libserver11.a}.}{So, \TT{kqf.o} file contain \TT{V\$VERSION} string.
The object file is in the main Oracle-library \TT{libserver11.a}.}

\IFRU{Ссылка на эту текстовую строку имеется в таблице \TT{kqfviw}, размещенной в этом же файле \TT{kqf.o}}{A reference to this text string we may find in the \TT{kqfviw} table stored in the same file, \TT{kqf.o}}:

\begin{lstlisting}[caption=kqf.o]
.rodata:0800C4A0 kqfviw          dd 0Bh                  ; DATA XREF: kqfchk:loc_8003A6D
.rodata:0800C4A0                                         ; kqfgbn+34
.rodata:0800C4A4                 dd offset _2__STRING_10102_0 ; "GV$WAITSTAT"
.rodata:0800C4A8                 dd 4
.rodata:0800C4AC                 dd offset _2__STRING_10103_0 ; "NULL"
.rodata:0800C4B0                 dd 3
.rodata:0800C4B4                 dd 0
.rodata:0800C4B8                 dd 195h
.rodata:0800C4BC                 dd 4
.rodata:0800C4C0                 dd 0
.rodata:0800C4C4                 dd 0FFFFC1CBh
.rodata:0800C4C8                 dd 3
.rodata:0800C4CC                 dd 0
.rodata:0800C4D0                 dd 0Ah
.rodata:0800C4D4                 dd offset _2__STRING_10104_0 ; "V$WAITSTAT"
.rodata:0800C4D8                 dd 4
.rodata:0800C4DC                 dd offset _2__STRING_10103_0 ; "NULL"
.rodata:0800C4E0                 dd 3
.rodata:0800C4E4                 dd 0
.rodata:0800C4E8                 dd 4Eh
.rodata:0800C4EC                 dd 3
.rodata:0800C4F0                 dd 0
.rodata:0800C4F4                 dd 0FFFFC003h
.rodata:0800C4F8                 dd 4
.rodata:0800C4FC                 dd 0
.rodata:0800C500                 dd 5
.rodata:0800C504                 dd offset _2__STRING_10105_0 ; "GV$BH"
.rodata:0800C508                 dd 4
.rodata:0800C50C                 dd offset _2__STRING_10103_0 ; "NULL"
.rodata:0800C510                 dd 3
.rodata:0800C514                 dd 0
.rodata:0800C518                 dd 269h
.rodata:0800C51C                 dd 15h
.rodata:0800C520                 dd 0
.rodata:0800C524                 dd 0FFFFC1EDh
.rodata:0800C528                 dd 8
.rodata:0800C52C                 dd 0
.rodata:0800C530                 dd 4
.rodata:0800C534                 dd offset _2__STRING_10106_0 ; "V$BH"
.rodata:0800C538                 dd 4
.rodata:0800C53C                 dd offset _2__STRING_10103_0 ; "NULL"
.rodata:0800C540                 dd 3
.rodata:0800C544                 dd 0
.rodata:0800C548                 dd 0F5h
.rodata:0800C54C                 dd 14h
.rodata:0800C550                 dd 0
.rodata:0800C554                 dd 0FFFFC1EEh
.rodata:0800C558                 dd 5
.rodata:0800C55C                 dd 0
\end{lstlisting}

\IFRU{Кстати, нередко, при изучении внутренностей \oracle, появляется вопрос, почему имена функций и глобальных переменных такие странные}{By the way, often, while analysing \oracle internals, you may ask yourself, why functions and global variable names are so weird}. \IFRU{Вероятно, дело в том, что \oracle очень старый продукт сам по себе и писался на Си еще в 1980-х}
{Supposedly, since \oracle is very old product and was developed in C in 1980-s}. \IFRU{А в те времена стандарт Си гарантировал поддержку имен переменных длиной только до шести символов включительно}
{And that was a time when C standard guaranteed function names/variables support only up to 6 characters inclusive}: <<6 significant initial characters in an external identifier>>\footnote{\href{http://yurichev.com/ref/Draft\%20ANSI\%20C\%20Standard\%20(ANSI\%20X3J11-88-090)\%20(May\%2013,\%201988).txt}{Draft ANSI C Standard (ANSI X3J11/88-090) (May 13, 1988)}}

\IFRU{Вероятно, таблица \TT{kqfviw} содержащая в себе многие (а может даже и все) view с префиксом V\$, это служебные view (fixed views), присутствующие всегда.}{Probably, the table \TT{kqfviw} contain most (maybe even all) views prefixed with V\$, these are \IT{fixed views}, present all the time.}
\IFRU{Бегло оценив цикличность данных, мы легко видим, что в каждом элементе таблицы \TT{kqfviw} 12 полей 32-битных полей.}{Superficially, by noticing cyclic recurrence of data, we can easily see that each \TT{kqfviw} table element has 12 32-bit fields.}
\IFRU{В \IDA легко создать структуру из 12-и элементов и применить её ко всем элементам таблицы.}{It is very simple to create a 12-elements structure in \IDA and apply it to all table elements.}
\IFRU{Для версии \oracle 11.2, здесь 1023 элемента в таблице, то есть, здесь описываются 1023 всех возможных \IT{fixed view}.}{As of \oracle version 11.2, there are 1023 table elements, i.e., there are described 1023 of all possible \IT{fixed views}.}
\IFRU{Позже, мы еще вернемся к этому числу.}{We will return to this number later.}

\IFRU{Как видно, мы не очень много можем узнать чисел в этих полях. Самое первое число всегда равно длине строки-названия view (без терминирующего ноля).}
{As we can see, there is not much information in these numbers in fields. The very first number is always equals to name of view (without terminating zero.}
\IFRU{Это справедливо для всех элементов. Но эта информация не очень полезна.}{This is correct for each element. But this information is not very useful.}

\IFRU{Мы также знаем, что информацию обо всех fixed views можно получить из \IT{fixed view} под названием}
{We also know that information about all fixed views can be retrieved from \IT{fixed view} named} \TT{V\$FIXED\_VIEW\_DEFINITION}
\IFRU{(кстати, информация для этого view также берется из таблиц \TT{kqfviw} и \TT{kqfvip}).}{(by the way, the information for this view is also taken from \TT{kqfviw} and \TT{kqfvip} tables.)}
\IFRU{Кстати, там тоже 1023 элемента.}{By the way, there are 1023 elements too.}

\begin{lstlisting}
SQL> select * from V$FIXED_VIEW_DEFINITION where view_name='V$VERSION';

VIEW_NAME
------------------------------
VIEW_DEFINITION
--------------------------------------------------------------------------------

V$VERSION
select  BANNER from GV$VERSION where inst_id = USERENV('Instance')
\end{lstlisting}

\IFRU{Итак, \TT{V\$VERSION} это как бы \IT{thunk view} для другого, с названием \TT{GV\$VERSION}, который, в свою очередь:}
{So, \TT{V\$VERSION} is some kind of \IT{thunk view} for another view, named \TT{GV\$VERSION}, which is, in turn:}

\begin{lstlisting}
SQL> select * from V$FIXED_VIEW_DEFINITION where view_name='GV$VERSION';

VIEW_NAME
------------------------------
VIEW_DEFINITION
--------------------------------------------------------------------------------

GV$VERSION
select inst_id, banner from x$version
\end{lstlisting}

\IFRU{Таблицы с префиксом X\$ в \oracle ~--- это также служебные таблицы, они не документированы, не могут изменятся пользователем, и обновляются динамически.}{Tables prefixed as X\$ in the \oracle -- is service tables too, undocumented, cannot be changed by user and refreshed dynamically.}

\IFRU{Попробуем поискать текст}{Let's also try to search the text} \TT{select  BANNER from GV\$VERSION where inst\_id = USERENV('Instance')} \IFRU{в файле \TT{kqf.o} и находим ссылку на него в таблице \TT{kqfvip}}{in the \TT{kqf.o} file and we find it in the \TT{kqfvip} table}:

.\begin{lstlisting}[caption=kqf.o]
rodata:080185A0 kqfvip          dd offset _2__STRING_11126_0 ; DATA XREF: kqfgvcn+18
.rodata:080185A0                                         ; kqfgvt+F
.rodata:080185A0                                         ; "select inst_id,decode(indx,1,'data bloc"...
.rodata:080185A4                 dd offset kqfv459_c_0
.rodata:080185A8                 dd 0
.rodata:080185AC                 dd 0

...

.rodata:08019570                 dd offset _2__STRING_11378_0 ; "select  BANNER from GV$VERSION where in"...
.rodata:08019574                 dd offset kqfv133_c_0
.rodata:08019578                 dd 0
.rodata:0801957C                 dd 0
.rodata:08019580                 dd offset _2__STRING_11379_0 ; "select inst_id,decode(bitand(cfflg,1),0"...
.rodata:08019584                 dd offset kqfv403_c_0
.rodata:08019588                 dd 0
.rodata:0801958C                 dd 0
.rodata:08019590                 dd offset _2__STRING_11380_0 ; "select  STATUS , NAME, IS_RECOVERY_DEST"...
.rodata:08019594                 dd offset kqfv199_c_0
\end{lstlisting}

\IFRU{Таблица, по всей видимости, имеет 4 поля в каждом элементе. Кстати, здесь также 1023 элемента.}
{The table appear to have 4 fields in each element.
By the way, there are 1023 elements too.}
\IFRU{Второе поле указывает на другую таблицу, содержащую поля этого \IT{fixed view}.}
{The second field pointing to another table, containing table fields for this \IT{fixed view}.}
\IFRU{Для \TT{V\$VERSION}, эта таблица только из двух элементов, первый это $6$ и второй это строка 
\TT{BANNER} (число это длина строки) и далее \IT{терминирующий} элемент содержащий $0$ и \IT{нулевую} 
Си-строку:}{As of \TT{V\$VERSION}, this table contain only two elements, first is $6$ and second is 
\TT{BANNER} string (the number ($6$) is this string length) and after, \IT{terminating} element contain 
$0$ and \IT{null} C-string:}

\begin{lstlisting}[caption=kqf.o]
.rodata:080BBAC4 kqfv133_c_0     dd 6                    ; DATA XREF: .rodata:08019574
.rodata:080BBAC8                 dd offset _2__STRING_5017_0 ; "BANNER"
.rodata:080BBACC                 dd 0
.rodata:080BBAD0                 dd offset _2__STRING_0_0
\end{lstlisting}

\IFRU{Объединив данные из таблиц \TT{kqfviw} и \TT{kqfvip}, мы получим SQL-запросы, которые исполняются, когда пользователь хочет получить информацию из какого-либо \IT{fixed view}.}{By joining data from both \TT{kqfviw} and \TT{kqfvip} tables, we may get SQL-statements which are executed when user wants to query information from specific \IT{fixed view}.}

\IFRU{Я написал программу \oracletables, которая собирает всю эту информацию из объектных файлов от \oracle под Linux.}{So I wrote an \oracletables program, so to gather all this information from \oracle for Linux object files.}
\IFRU{Для \TT{V\$VERSION}, мы можем найти следующее:}{For \TT{V\$VERSION}, we may find this:}

\begin{lstlisting}[caption=\IFRU{Результат работы}{Result of} \OracleTablesName]
kqfviw_element.viewname: [V$VERSION] ?: 0x3 0x43 0x1 0xffffc085 0x4
kqfvip_element.statement: [select  BANNER from GV$VERSION where inst_id = USERENV('Instance')]
kqfvip_element.params:
[BANNER] 
\end{lstlisting}

\AndENRU:

\begin{lstlisting}[caption=\IFRU{Результат работы}{Result of} \OracleTablesName]
kqfviw_element.viewname: [GV$VERSION] ?: 0x3 0x26 0x2 0xffffc192 0x1
kqfvip_element.statement: [select inst_id, banner from x$version]
kqfvip_element.params:
[INST_ID] [BANNER] 
\end{lstlisting}

\IFRU{\IT{Fixed view} \TT{GV\$VERSION} отличается от \TT{V\$VERSION} тем, что содержит еще и поле отражающее идентификатор \IT{instance}.}
{\TT{GV\$VERSION} \IT{fixed view} is distinct from \TT{V\$VERSION} in only that way that it contains one more field with \IT{instance} identifier.}
\IFRU{Но так или иначе, мы теперь упираемся в таблицу \TT{X\$VERSION}. Как и прочие X\$-таблицы, она не документирована, однако, мы можем оттуда что-то прочитать}{Anyway, we stuck at the table \TT{X\$VERSION}. Just like any other X\$-tables, it is undocumented, however, we can query it}:

\begin{lstlisting}
SQL> select * from x$version;

ADDR           INDX    INST_ID
-------- ---------- ----------
BANNER
--------------------------------------------------------------------------------

0DBAF574          0          1
Oracle Database 11g Enterprise Edition Release 11.2.0.1.0 - Production

...
\end{lstlisting}

\IFRU{Эта таблица содержит дополнительные поля вроде \TT{ADDR} и \TT{INDX}.}{This table has additional fields like \TT{ADDR} and \TT{INDX}.}

\IFRU{Бегло листая содержимое файла \TT{kqf.o} в \IDA мы можем увидеть еще одну таблицу где есть ссылка на строку \TT{X\$VERSION}, это \TT{kqftab}:}{While scrolling \TT{kqf.o} in \IDA we may spot another table containing pointer to the \TT{X\$VERSION} string, this is \TT{kqftab}:}

\begin{lstlisting}[caption=kqf.o]
.rodata:0803CAC0                 dd 9                    ; element number 0x1f6
.rodata:0803CAC4                 dd offset _2__STRING_13113_0 ; "X$VERSION"
.rodata:0803CAC8                 dd 4
.rodata:0803CACC                 dd offset _2__STRING_13114_0 ; "kqvt"
.rodata:0803CAD0                 dd 4
.rodata:0803CAD4                 dd 4
.rodata:0803CAD8                 dd 0
.rodata:0803CADC                 dd 4
.rodata:0803CAE0                 dd 0Ch
.rodata:0803CAE4                 dd 0FFFFC075h
.rodata:0803CAE8                 dd 3
.rodata:0803CAEC                 dd 0
.rodata:0803CAF0                 dd 7
.rodata:0803CAF4                 dd offset _2__STRING_13115_0 ; "X$KQFSZ"
.rodata:0803CAF8                 dd 5
.rodata:0803CAFC                 dd offset _2__STRING_13116_0 ; "kqfsz"
.rodata:0803CB00                 dd 1
.rodata:0803CB04                 dd 38h
.rodata:0803CB08                 dd 0
.rodata:0803CB0C                 dd 7
.rodata:0803CB10                 dd 0
.rodata:0803CB14                 dd 0FFFFC09Dh
.rodata:0803CB18                 dd 2
.rodata:0803CB1C                 dd 0
\end{lstlisting}

\IFRU{Здесь очень много ссылок на названия X\$-таблиц, вероятно, на все те что имеются в \oracle этой версии.}
{There are a lot of references to X\$-table names, apparently, to all \oracle 11.2 X\$-tables.}
\IFRU{Но мы снова упираемся в то что не имеем достаточно информации.}{But again, we have not enough information.}
\IFRU{У меня нет никакой идеи, что означает строка \TT{kqvt}.}
{I have no idea, what \TT{kqvt} string means.} 
\IFRU{Вообще, префикс \TT{kq} может означать \IT{kernel} и \IT{query}.}
{\TT{kq} prefix may means \IT{kernel} and \IT{query}.} 
\IFRU{\TT{v}, может быть, \IT{version}, а \TT{t} ~--- \IT{type}?}
{\TT{v}, apparently, means \IT{version} and \TT{t}~---\IT{type}?} 
\IFRU{Я не знаю, честно говоря.}{Frankly speaking, I do not know.}

\IFRU{Таблицу с очень похожим названием мы можем найти в}{The table named similarly can be found in} \TT{kqf.o}:

\begin{lstlisting}[caption=kqf.o]
.rodata:0808C360 kqvt_c_0        kqftap_param <4, offset _2__STRING_19_0, 917h, 0, 0, 0, 4, 0, 0>
.rodata:0808C360                                         ; DATA XREF: .rodata:08042680
.rodata:0808C360                                         ; "ADDR"
.rodata:0808C384                 kqftap_param <4, offset _2__STRING_20_0, 0B02h, 0, 0, 0, 4, 0, 0> ; "INDX"
.rodata:0808C3A8                 kqftap_param <7, offset _2__STRING_21_0, 0B02h, 0, 0, 0, 4, 0, 0> ; "INST_ID"
.rodata:0808C3CC                 kqftap_param <6, offset _2__STRING_5017_0, 601h, 0, 0, 0, 50h, 0, 0> ; "BANNER"
.rodata:0808C3F0                 kqftap_param <0, offset _2__STRING_0_0, 0, 0, 0, 0, 0, 0, 0>
\end{lstlisting}

\IFRU{Она содержит информацию об именах полей в таблице \TT{X\$VERSION}.}{It contain information about all fields in the \TT{X\$VERSION} table.}
\IFRU{Единственная ссылка на эту таблицу имеется в таблице \TT{kqftap}:}{The only reference to this table present in the \TT{kqftap} table:}

\begin{lstlisting}[caption=kqf.o]
.rodata:08042680                 kqftap_element <0, offset kqvt_c_0, offset kqvrow, 0> ; element 0x1f6
\end{lstlisting}

\IFRU{Интересно что здесь этот элемент проходит также под номером \TT{0x1f6} (502-й), как и ссылка на строку 
\TT{X\$VERSION} в таблице \TT{kqftab}.}
{It is interesting that this element here is \TT{0x1f6th} (502nd), just as a pointer to the \TT{X\$VERSION} string in 
the \TT{kqftab} table.}
\IFRU{Вероятно, таблицы \TT{kqftap} и \TT{kqftab} дополняют друг друга, как и \TT{kqfvip} и \TT{kqfviw}.}
{Probably, \TT{kqftap} and \TT{kqftab} tables are complement each other, just like \TT{kqfvip} and \TT{kqfviw}.}
\IFRU{Мы также видим здесь ссылку на функцию с названием \TT{kqvrow()}. А вот это уже кое-что!}
{We also see a pointer to the \TT{kqvrow()} function. Finally, we got something useful!}

\IFRU{Я сделал так чтобы моя программа \oracletables могла дампить и эти таблицы. Для \TT{X\$VERSION} получается:}
{So I added these tables to my \oracletables utility too. For \TT{X\$VERSION} I've got:}

\begin{lstlisting}[caption=\IFRU{Результат работы}{Result of} \OracleTablesName]
kqftab_element.name: [X$VERSION] ?: [kqvt] 0x4 0x4 0x4 0xc 0xffffc075 0x3
kqftap_param.name=[ADDR] ?: 0x917 0x0 0x0 0x0 0x4 0x0 0x0
kqftap_param.name=[INDX] ?: 0xb02 0x0 0x0 0x0 0x4 0x0 0x0
kqftap_param.name=[INST_ID] ?: 0xb02 0x0 0x0 0x0 0x4 0x0 0x0
kqftap_param.name=[BANNER] ?: 0x601 0x0 0x0 0x0 0x50 0x0 0x0
kqftap_element.fn1=kqvrow
kqftap_element.fn2=NULL
\end{lstlisting}

\IFRU{При помощи \tracer, можно легко проверить, что эта ф-ция вызывается 6 раз кряду (из ф-ции \TT{qerfxFetch()}) при получении строк из \TT{X\$VERSION}.}{With the help of \tracer, it is easy to check that this function called 6 times in row (from the \TT{qerfxFetch()} function) while querying \TT{X\$VERSION} table.}

\IFRU{Запустим \tracer в режиме \TT{cc} (он добавит комментарий к каждой исполненной инструкции):}{Let's run \tracer in the \TT{cc} mode (it will comment each executed instruction):}

\begin{lstlisting}
tracer -a:oracle.exe bpf=oracle.exe!_kqvrow,trace:cc
\end{lstlisting}

\lstinputlisting{examples/oracle/VERSION_kqvrow.asm}

\IFRU{Так можно легко увидеть, что номер строки таблицы задается извне. Сама ф-ция возвращает строку, формируя её так:}
{Now it is easy to see that row number is passed from outside of function. The function returns the string constructing it as follows:}

\begin{center}
\begin{tabular}{ | l | l | }
\hline                        
\IFRU{Строка}{String} 1	& \IFRU{Использует глобальные переменные \TT{vsnstr}, \TT{vsnnum}, \TT{vsnban}. Вызывает \TT{sprintf()}.}{Using \TT{vsnstr}, \TT{vsnnum}, \TT{vsnban} global variables. Calling \TT{sprintf()}.} \\
\IFRU{Строка}{String} 2	& \IFRU{Вызывает}{Calling} \TT{kkxvsn()}. \\
\IFRU{Строка}{String} 3	& \IFRU{Вызывает}{Calling} \TT{lmxver()}. \\
\IFRU{Строка}{String} 4	& \IFRU{Вызывает}{Calling} \TT{npinli()}, \TT{nrtnsvrs()}. \\
\IFRU{Строка}{String} 5	& \IFRU{Вызывает}{Calling} \TT{lxvers()}. \\
\hline  
\end{tabular}
\end{center}

\IFRU{Так вызываются соответствующие ф-ции для определения номеров версий отдельных модулей.}
{That's how corresponding functions are called for determining each module's version.}


\input{examples/oracle/2_ksmlru.tex}
\section{\RU{Таблица \TT{V\$TIMER} в}\EN{\TT{V\$TIMER} table in} \oracle}
\index{\oracle}

\TT{V\$TIMER} \RU{это еще один служебный \IT{fixed view}, отражающий какое-то часто меняющееся значение:}
\EN{is another \IT{fixed view} that reflects a rapidly changing value:}

\begin{framed}
\begin{quotation}
V\$TIMER displays the elapsed time in hundredths of a second. Time is measured since the beginning of the epoch, 
which is operating system specific, and wraps around to 0 again whenever the value overflows four bytes 
(roughly 497 days).
\end{quotation}
\end{framed}(\RU{Из документации \oracle}\EN{From \oracle documentation}
\footnote{\url{http://go.yurichev.com/17088}})

\RU{Интересно что периоды разные в Oracle для Win32 и для Linux. Сможем ли мы найти функцию, отвечающую 
за генерирование этого значения?}
\EN{It is interesting that the periods are different for Oracle for win32 and for Linux. 
Will we be able to find the function that generates this value?}

\RU{Как видно, эта информация, в итоге, берется из системной таблицы \TT{X\$KSUTM}.}\EN{As we can see, 
this information is finally taken from the \TT{X\$KSUTM} table.}

\begin{lstlisting}
SQL> select * from V$FIXED_VIEW_DEFINITION where view_name='V$TIMER';

VIEW_NAME
------------------------------
VIEW_DEFINITION
--------------------------------------------------------------------------------

V$TIMER
select  HSECS from GV$TIMER where inst_id = USERENV('Instance')

SQL> select * from V$FIXED_VIEW_DEFINITION where view_name='GV$TIMER';

VIEW_NAME
------------------------------
VIEW_DEFINITION
--------------------------------------------------------------------------------

GV$TIMER
select inst_id,ksutmtim from x$ksutm
\end{lstlisting}

\RU{Здесь мы упираемся в небольшую проблему, в таблицах \TT{kqftab}/\TT{kqftap} нет указателей на функцию, 
которая бы генерировала значение}
\EN{Now we are stuck in a small problem, there are no references to value generating function(s) 
in the tables \TT{kqftab}/\TT{kqftap}}:

\begin{lstlisting}[caption=\RU{Результат работы}\EN{Result of} \OracleTablesName]
kqftab_element.name: [X$KSUTM] ?: [ksutm] 0x1 0x4 0x4 0x0 0xffffc09b 0x3
kqftap_param.name=[ADDR] ?: 0x10917 0x0 0x0 0x0 0x4 0x0 0x0
kqftap_param.name=[INDX] ?: 0x20b02 0x0 0x0 0x0 0x4 0x0 0x0
kqftap_param.name=[INST_ID] ?: 0xb02 0x0 0x0 0x0 0x4 0x0 0x0
kqftap_param.name=[KSUTMTIM] ?: 0x1302 0x0 0x0 0x0 0x4 0x0 0x1e
kqftap_element.fn1=NULL
kqftap_element.fn2=NULL
\end{lstlisting}

\RU{Попробуем в таком случае просто поискать строку \TT{KSUTMTIM}, и находим ссылку на нее в такой функции:}
\EN{When we try to find the string \TT{KSUTMTIM}, we see it in this function:}

\begin{lstlisting}
kqfd_DRN_ksutm_c proc near              ; DATA XREF: .rodata:0805B4E8

arg_0           = dword ptr  8
arg_8           = dword ptr  10h
arg_C           = dword ptr  14h

                push    ebp
                mov     ebp, esp
                push    [ebp+arg_C]
                push    offset ksugtm
                push    offset _2__STRING_1263_0 ; "KSUTMTIM"
                push    [ebp+arg_8]
                push    [ebp+arg_0]
                call    kqfd_cfui_drain
                add     esp, 14h
                mov     esp, ebp
                pop     ebp
                retn
kqfd_DRN_ksutm_c endp
\end{lstlisting}

\RU{Сама функция}\EN{The} \TT{kqfd\_DRN\_ksutm\_c()} \RU{упоминается в таблице}\EN{function is mentioned in the} 
\TT{kqfd\_tab\_registry\_0} \RU{вот так}\EN{table}:

\begin{lstlisting}
dd offset _2__STRING_62_0 ; "X$KSUTM"
dd offset kqfd_OPN_ksutm_c
dd offset kqfd_tabl_fetch
dd 0
dd 0
dd offset kqfd_DRN_ksutm_c
\end{lstlisting}

\RU{Упоминается также некая функция \TT{ksugtm()}}\EN{There is a function \TT{ksugtm()} referenced here}.
\RU{Посмотрим, что там (в Linux x86)}\EN{Let's see what's in it (Linux x86)}:

\begin{lstlisting}[caption=ksu.o]
ksugtm          proc near

var_1C          = byte ptr -1Ch
arg_4           = dword ptr  0Ch

                push    ebp
                mov     ebp, esp
                sub     esp, 1Ch
                lea     eax, [ebp+var_1C]
                push    eax
                call    slgcs
                pop     ecx
                mov     edx, [ebp+arg_4]
                mov     [edx], eax
                mov     eax, 4
                mov     esp, ebp
                pop     ebp
                retn
ksugtm          endp
\end{lstlisting}

\RU{В win32-версии тоже самое}\EN{The code in the win32 version is almost the same}.

\RU{Искомая ли эта функция? Попробуем узнать}\EN{Is this the function we are looking for? Let's see}:
\index{tracer}

\begin{lstlisting}
tracer -a:oracle.exe bpf=oracle.exe!_ksugtm,args:2,dump_args:0x4
\end{lstlisting}

\RU{Пробуем несколько раз}\EN{Let's try again}:

\begin{lstlisting}
SQL> select * from V$TIMER;

     HSECS
----------
  27294929

SQL> select * from V$TIMER;

     HSECS
----------
  27295006

SQL> select * from V$TIMER;

     HSECS
----------
  27295167
\end{lstlisting}

\begin{lstlisting}[caption=\RU{вывод \tracer}\EN{\tracer output}]
TID=2428|(0) oracle.exe!_ksugtm (0x0, 0xd76c5f0) (called from oracle.exe!__VInfreq__qerfxFetch+0xfad (0x56bb6d5))
Argument 2/2
0D76C5F0: 38 C9                                           "8.              "
TID=2428|(0) oracle.exe!_ksugtm () -> 0x4 (0x4)
Argument 2/2 difference
00000000: D1 7C A0 01                                     ".|..            "
TID=2428|(0) oracle.exe!_ksugtm (0x0, 0xd76c5f0) (called from oracle.exe!__VInfreq__qerfxFetch+0xfad (0x56bb6d5))
Argument 2/2
0D76C5F0: 38 C9                                           "8.              "
TID=2428|(0) oracle.exe!_ksugtm () -> 0x4 (0x4)
Argument 2/2 difference
00000000: 1E 7D A0 01                                     ".}..            "
TID=2428|(0) oracle.exe!_ksugtm (0x0, 0xd76c5f0) (called from oracle.exe!__VInfreq__qerfxFetch+0xfad (0x56bb6d5))
Argument 2/2
0D76C5F0: 38 C9                                           "8.              "
TID=2428|(0) oracle.exe!_ksugtm () -> 0x4 (0x4)
Argument 2/2 difference
00000000: BF 7D A0 01                                     ".}..            "
\end{lstlisting}

\RU{Действительно\EMDASH{}значение то, что мы видим в SQL*Plus, и оно возвращается через второй аргумент}
\EN{Indeed\EMDASH{}the value is the same we see in SQL*Plus and it is returned via the second argument}.

\RU{Посмотрим, что в функции}\EN{Let's see what is in} \TT{slgcs()} (Linux x86):

\begin{lstlisting}
slgcs           proc near

var_4           = dword ptr -4
arg_0           = dword ptr  8

                push    ebp
                mov     ebp, esp
                push    esi
                mov     [ebp+var_4], ebx
                mov     eax, [ebp+arg_0]
                call    $+5
                pop     ebx
                nop                     ; PIC mode
                mov     ebx, offset _GLOBAL_OFFSET_TABLE_
                mov     dword ptr [eax], 0
                call    sltrgatime64    ; PIC mode
                push    0
                push    0Ah
                push    edx
                push    eax
                call    __udivdi3       ; PIC mode
                mov     ebx, [ebp+var_4]
                add     esp, 10h
                mov     esp, ebp
                pop     ebp
                retn
slgcs           endp
\end{lstlisting}

(\RU{это просто вызов}\EN{it is just a call to} \TT{sltrgatime64()} \RU{и деление его результата на}
\EN{and division of its result by} 10~(\myref{sec:divisionbynine}))

\RU{И в win32-версии}\EN{And win32-version}:

\begin{lstlisting}
_slgcs          proc near               ; CODE XREF: _dbgefgHtElResetCount+15
                                        ; _dbgerRunActions+1528
                db      66h
                nop
                push    ebp
                mov     ebp, esp
                mov     eax, [ebp+8]
                mov     dword ptr [eax], 0
                call    ds:__imp__GetTickCount@0 ; GetTickCount()
                mov     edx, eax
                mov     eax, 0CCCCCCCDh
                mul     edx
                shr     edx, 3
                mov     eax, edx
                mov     esp, ebp
                pop     ebp
                retn
_slgcs          endp
\end{lstlisting}

\RU{Это просто результат}\EN{It is just the result of} \TT{GetTickCount()
\footnote{\href{http://go.yurichev.com/17248}{MSDN}}} 
\RU{поделенный на}\EN{divided by} 10~(\myref{sec:divisionbynine}).

\RU{Вуаля! Вот почему в win32-версии и версии Linux x86 разные результаты, потому что они получаются разными 
системными функциями \ac{OS}.}\EN{Voilà! That's why the win32 version and the Linux x86 version show different results, 
because they are generated by different \ac{OS} functions.}

\RU{\IT{Drain} по-английски \IT{дренаж, отток, водосток}. Таким образом, возможно имеется ввиду \IT{подключение} 
определенного столбца системной таблице к функции.}
\EN{\IT{Drain} apparently implies \IT{connecting} a specific table column to a specific function.}

\RU{Добавим поддержку таблицы}\EN{We will add support of the table} \TT{kqfd\_tab\_registry\_0} \RU{в}\EN{to} \oracletables, 
\RU{теперь мы можем видеть, при помощи каких функций, столбцы в системных таблицах \IT{подключаются} к значениям, 
например}\EN{now we can see how the table column's variables are \IT{connected} to a specific functions}:

\begin{lstlisting}
[X$KSUTM] [kqfd_OPN_ksutm_c] [kqfd_tabl_fetch] [NULL] [NULL] [kqfd_DRN_ksutm_c]
[X$KSUSGIF] [kqfd_OPN_ksusg_c] [kqfd_tabl_fetch] [NULL] [NULL] [kqfd_DRN_ksusg_c]
\end{lstlisting}

\IT{OPN}, \RU{возможно}\EN{apparently stands for}, \IT{open}, \RU{а}\EN{and} \IT{DRN}, \RU{вероятно, означает}
\EN{apparently, for} \IT{drain}.


\fi

\ifdefined\IncludeMSDOS
\chapter{\RU{Вручную написанный на ассемблере код}\EN{Handwritten assembly code}}

\section{\RU{Тестовый файл} EICAR\EN{ test file}}
\label{subsec:EICAR}

\index{MS-DOS}
\index{EICAR}
\RU{Этот .COM-файл предназначен для тестирования антивирусов, его можно запустить в MS-DOS
и он выведет такую строку}\EN{This .COM-file is intended for testing antivirus software, it is possible to run in
in MS-DOS and it prints this string}: \q{EICAR-STANDARD-ANTIVIRUS-TEST-FILE!}
\footnote{\href{\RU{http://go.yurichev.com/17005}\EN{http://go.yurichev.com/17006}}{wikipedia}}.
% FIXME1 \myref{} -> about .COM files

\RU{Он примечателен тем, что он полностью состоит только из печатных ASCII-символов, следовательно, его можно
набрать в любом текстовом редакторе}\EN{Its important property is that it's consists entirely of printable 
ASCII-symbols, which, in turn, makes it possible to create it in any text editor}:

\begin{lstlisting}
X5O!P%@AP[4\PZX54(P^)7CC)7}$EICAR-STANDARD-ANTIVIRUS-TEST-FILE!$H+H*
\end{lstlisting}

\RU{Попробуем его разобрать}\EN{Let's decompile it}:

\lstinputlisting{examples/handcoding/EICAR.lst.\LANG}

\RU{Добавим везде комментарии, показывающие состояние регистров и стека после каждой инструкции}%
\EN{We will add comments about the registers and stack after each instruction}.

\RU{Собственно, все эти инструкции нужны только для того чтобы исполнить следующий код}\EN{Essentially, all these
instructions are here only to execute this code}:

\begin{lstlisting}
B4 09     MOV AH, 9
BA 1C 01  MOV DX, 11Ch
CD 21     INT 21h
CD 20     INT 20h
\end{lstlisting}

\index{x86!\Instructions!INT}
\TT{INT 21h} \RU{с функцией 9 (переданной в \TT{AH}) просто выводит строку, адрес которой передан в}\EN{with 9th
function (passed in \TT{AH}) just prints a string, the address of which is passed in} \TT{DS:DX}.
\RU{Кстати, строка должна быть завершена символом '\$'}\EN{By the way, the string has to be terminated
with the '\$' sign}.
\RU{Вероятно, это наследие}\EN{Apparently, it's inherited from} \gls{CP/M} 
\RU{и эта функция в DOS осталась для совместимости}\EN{and this function was left in DOS for compatibility}.
\TT{INT 20h} \RU{возвращает управление в}\EN{exits to} DOS.

\RU{Но, как видно, далеко не все опкоды этих инструкций печатные}\EN{But as we can see, these instruction's
opcodes are not strictly printable}.
\RU{Так что основная часть EICAR-файла это}\EN{So the main part of EICAR file is}:

\begin{itemize}
\item \RU{подготовка нужных значений регистров (AH и DX)}\EN{preparing the register (AH and DX) values that we need};
\item \RU{подготовка в памяти опкодов для INT 21 и INT 20}\EN{preparing INT 21 and INT 20 opcodes in memory};
\item \RU{исполнение}\EN{executing} INT 21 \AndENRU INT 20.
\end{itemize}

\index{Shellcode}
\RU{Кстати, подобная техника широко используется для создания шеллкодов, 
где нужно создать x86-код, который будет нужно передать в виде текстовой строки}
\EN{By the way, this technique is widely used in shellcode construction, when one need to pass x86 code
in string form}.

\RU{Здесь также список всех x86-инструкций с печатаемыми опкодоами}\EN{Here is also a list of all 
x86 instructions which have printable opcodes}: \myref{printable_x86_opcodes}.
\fi

\mysection{\RU{Демо}\EN{Demos}}

\RU{Демо (или демомейкинг) были великолепным упражнением в математике, программировании компьютерной графики
и очень плотному программированию на ассемблере вручную}\EN{Demos (or demomaking) were an excellent 
exercise in mathematics, computer graphics programming and very tight x86 hand coding}.

% sections
\EN{\subsection{10 PRINT CHR\$(205.5+RND(1)); : GOTO 10}

All examples here are MS-DOS .COM files.
%FIXME1 -> about .COM files

\myindex{MS-DOS}
In [Nick Montfort et al, \IT{10 PRINT CHR\$(205.5+RND(1)); : GOTO 10}, (The MIT Press:2012)]
\footnote{\AlsoAvailableAs \url{http://go.yurichev.com/17286}}

we can read about one of the most simple possible random maze generators.

It just prints a slash or backslash characters randomly and endlessly, resulting in something like this:

\begin{figure}[H]
\centering
\includegraphics[width=0.6\textwidth]{examples/demos/10print/10print.png}
\end{figure}

There are a few known implementations for 16-bit x86.

\subsubsection{Trixter's 42 byte version}

\newcommand{\FNURLTRIXTER}{\footnote{\url{http://go.yurichev.com/17305}}}

The listing was taken from his website\FNURLTRIXTER, 
but the comments are mine.

\lstinputlisting[style=customasmx86]{examples/demos/10print/10print_42_EN.lst}

\myindex{Intel!8253}
The pseudo-random value here is in fact the time 
that has passed from the system's boot, taken from the 8253 time chip, the value increases by one 18.2 times per second.

By writing zero to port \TT{43h}, 
we send the command \q{select counter 0}, 
"counter latch", 
"binary counter" (not a \ac{BCD} value).

\myindex{x86!\Instructions!POPF}
The interrupts are enabled back with the \TT{POPF} instruction, which restores the \TT{IF} flag as well.

\myindex{x86!\Instructions!IN}
It is not possible to 
use the \TT{IN} instruction with registers other than \TT{AL}, 
hence the shuffling.

\subsubsection{
My attempt to reduce Trixter's version: 27 bytes}

We can say that since we use the timer not 
to get a precise time value, but a pseudo-random one, we do not need
to spend time (and code) to disable the interrupts.

Another thing we can say is that we need only one bit from the low 8-bit part, so let's read only it.

We can reduced the code slightly and we've got 27 bytes:

\lstinputlisting[style=customasmx86]{examples/demos/10print/10print_27_EN.lst}

\subsubsection{
Taking random memory garbage as a source of randomness}

Since it is MS-DOS, there is no memory protection at all, we can read from whatever address we want.
\myindex{x86!\Instructions!LODSB}
Even more than that: a simple \TT{LODSB} 
instruction reads a byte from the \TT{DS:SI} address, but it's not a problem
if the registers' values are not set up, let it read 1) random bytes; 2) from a random place in memory!

It is suggested in Trixter's webpage\FNURLTRIXTER 
to use \TT{LODSB} without any setup.

\myindex{x86!\Instructions!SCASB}
It is also suggested that the \TT{SCASB} 

instruction can be used instead, because it sets a flag according to the byte it reads.


Another idea to minimize the code is to use the \TT{INT 29h} DOS syscall, which just prints the character stored in the \TT{AL} register.

That is what Peter Ferrie and \HERMIT{} did (11 and 10 bytes)
\footnote{\url{http://go.yurichev.com/17087}}:

\lstinputlisting[caption=\HERMIT: 11 bytes,style=customasmx86]{examples/demos/10print/herm1t_11_EN.lst}

\myindex{x86!\Instructions!SCASB}
\TT{SCASB} also uses the value in the \TT{AL}
 register, it subtract a random memory byte's value from the \TT{5Ch} value in \TT{AL}.
\myindex{x86!\Instructions!JP}
\TT{JP} is a rare instruction, here it used for checking the parity flag (PF), 
which is generated by the formulae in the listing.
As a consequence, the output character 
is determined not by some bit in a random memory byte, but by a sum of bits, 
this (hopefully) makes the result more distributed.

\myindex{x86!\Instructions!SALC}
\myindex{x86!\Instructions!SETALC}
\myindex{NEC V20}
It is possible to 
make this even shorter by using the undocumented x86 instruction \TT{SALC} (\ac{AKA} \TT{SETALC}) (\q{Set AL CF}).
It was introduced in the NEC V20 \ac{CPU} and sets \TT{AL} to 
\TT{0xFF} if \TT{CF} is 1 or to 0 if otherwise.

\lstinputlisting[caption=Peter Ferrie: 10 bytes,style=customasmx86]{examples/demos/10print/ferrie_10_EN.lst}

So it is possible to get rid of conditional jumps at all.
The \ac{ASCII} code of backslash (\q{\textbackslash{}}) 
is \TT{0x5C} and \TT{0x2F} for slash (\q{/}).
So we have to convert one (pseudo-random) bit in the \TT{CF} flag to a value of \TT{0x5C} or \TT{0x2F}.

This is done easily: by \TT{AND}-ing all bits in \TT{AL} (where all 8 bits are set or cleared) with \TT{0x2D} we have just 0 or \TT{0x2D}.

By adding \TT{0x2F} to this value, we get \TT{0x5C} or \TT{0x2F}.

Then we just output it to the screen.

\subsubsection{\Conclusion{}}

\myindex{DOSBox}
It is also worth mentioning that the result may 
be different in DOSBox, \gls{Windows NT} and even MS-DOS, 

due to different
conditions: the timer chip can be emulated differently and the initial register contents may be different as well.
}
\RU{\subsection{10 PRINT CHR\$(205.5+RND(1)); : GOTO 10}

Все примеры здесь для .COM-файлов под MS-DOS.
%FIXME1 -> about .COM files

\myindex{MS-DOS}
В [Nick Montfort et al, \IT{10 PRINT CHR\$(205.5+RND(1)); : GOTO 10}, (The MIT Press:2012)]
\footnote{\AlsoAvailableAs \url{http://go.yurichev.com/17286}}
можно прочитать об одном из простейших генераторов случайных лабиринтов.
Он просто бесконечно и случайно печатает символ слэша или обратный слэша, выдавая в итоге что-то вроде:

\begin{figure}[H]
\centering
\includegraphics[width=0.6\textwidth]{examples/demos/10print/10print.png}
\end{figure}

Здесь несколько известных реализаций для 16-битного x86.

\subsubsection{Версия 42-х байт от Trixter}

\newcommand{\FNURLTRIXTER}{\footnote{\url{http://go.yurichev.com/17305}}}

Листинг взят с его сайта\FNURLTRIXTER, но комментарии --- автора.

\lstinputlisting[style=customasmx86]{examples/demos/10print/10print_42_RU.lst}

\myindex{Intel!8253}
Псевдослучайное число на самом деле это время, прошедшее со старта системы, получаемое из чипа таймера 8253, 
это значение
увеличивается на единицу 18.2 раза в секунду.

Записывая ноль в порт \TT{43h}, 
мы имеем ввиду что команда это \q{выбрать счетчик 0}, 
"counter latch", 
"двоичный счетчик" (а не значение \ac{BCD}).

\myindex{x86!\Instructions!POPF}
Прерывания снова разрешаются при помощи инструкции \TT{POPF}, которая
также возвращает флаг \TT{IF}.

\myindex{x86!\Instructions!IN}
Инструкцию \TT{IN} нельзя использовать с другими регистрами кроме \TT{AL}, поэтому здесь перетасовка.

\subsubsection{Моя попытка укоротить версию Trixter: 27 байт}

Мы можем сказать, что мы используем таймер не для того чтобы получить точное время, но псевдослучайное число,
так что мы можем не тратить время (и код) на запрещение прерываний.
Еще можно сказать, что так как мы берем бит из младшей 8-битной части, то мы можем считывать только её.

Немного укоротим код и выходит 27 байт:

\lstinputlisting[style=customasmx86]{examples/demos/10print/10print_27_RU.lst}

\subsubsection{Использование случайного мусора в памяти как источника случайных чисел}

Так как это MS-DOS, защиты памяти здесь нет вовсе, так что мы можем читать с какого
угодно адреса.
\myindex{x86!\Instructions!LODSB}
И даже более того: простая инструкция \TT{LODSB} 
будет читать байт по адресу \TT{DS:SI}, но это не проблема
если правильные значения не установлены в регистры, пусть она читает 1) случайные байты; 2) из случайного
места в памяти!

Так что на странице Trixter-а\FNURLTRIXTER 
можно найти предложение использовать \TT{LODSB} без всякой инициализации.

\myindex{x86!\Instructions!SCASB}
Есть также предложение использовать инструкцию \TT{SCASB} 
вместо, потому что она выставляет флаги в соответствии с прочитанным значением.

Еще одна идея насчет минимизации кода --- это использовать прерывание DOS
 \TT{INT 29h} которое просто печатает символ на экране
из регистра \TT{AL}.

Это то что сделали Peter Ferrie и \HERMIT{} (11 и 10 байт)
\footnote{\url{http://go.yurichev.com/17087}}:

\lstinputlisting[caption=\HERMIT: 11 байт,style=customasmx86]{examples/demos/10print/herm1t_11_RU.lst}

\myindex{x86!\Instructions!SCASB}
\TT{SCASB} также использует значение в регистре \TT{AL}, она вычитает значение
случайного байта в памяти из
 \TT{5Ch} в \TT{AL}.
\myindex{x86!\Instructions!JP}
\TT{JP} это редкая инструкция, здесь она используется для проверки флага четности (PF),
который вычисляется по формуле в листинге.
Как следствие, выводимый символ определяется не каким-то конкретным битом из случайного байта в памяти,
а суммой бит, и это (надеемся) сделает результат более распределенным.

\myindex{x86!\Instructions!SALC}
\myindex{x86!\Instructions!SETALC}
\myindex{NEC V20}
Можно сделать еще короче, если использовать недокументированную x86-инструкцию \TT{SALC} (\ac{AKA} \TT{SETALC}) (\q{Set AL CF}).
Она появилась в \ac{CPU} и выставляет \TT{AL} в 
\TT{0xFF} если \TT{CF} это 1 или 0 если наоборот.

\lstinputlisting[caption=Peter Ferrie: 10 байт,style=customasmx86]{examples/demos/10print/ferrie_10_RU.lst}

Так что можно избавиться и от условных переходов.
\ac{ASCII}-код обратного слэша (\q{\textbackslash{}}) 
это \TT{0x5C} и \TT{0x2F} для слэша (\q{/}).

Так что нам нужно конвертировать один (псевдослучайный) бит из флага \TT{CF} в значение \TT{0x5C} или \TT{0x2F}.

%
Это делается легко: применяя операцию \q{И} ко всем битам в \TT{AL} (где все 8 бит либо выставлены, либо сброшены) с \TT{0x2D} мы имеем просто 0 или \TT{0x2D}.

%
Прибавляя значение \TT{0x2F} к этому значению, мы получаем \TT{0x5C} или \TT{0x2F}.
И просто выводим это на экран.

\subsubsection{\Conclusion{}}

\myindex{DOSBox}
Также стоит отметить, что результат может быть разным в эмуляторе DOSBox, \gls{Windows NT} и даже MS-DOS, 
из-за разных условий:
чип таймера может эмулироваться по-разному, изначальные значения регистров также могут быть разными.
}
\EN{\clearpage
\subsection{Mandelbrot set}
\label{Mandelbrot_demo}

\epigraph{You know, if you magnify the coastline, it still looks like
a coastline, and a lot of other things have this property. Nature has
recursive algorithms that it uses to generate clouds and Swiss cheese
and things like that.}
{Donald Knuth, interview (1993)}

Mandelbrot set is a fractal, which exhibits self-similarity.

When you increase scale, you see that this characteristic pattern repeating infinitely.

Here is a demo\footnote{Download it \href{http://go.yurichev.com/17306} {here},} 
written by \q{Sir\_Lagsalot} in 2009, that draws 
the Mandelbrot set, which is just a x86 program with executable file size of only 64 bytes.
There are only 30 16-bit x86 instructions.

Here it is what it draws:

\begin{figure}[H]
\centering
\myincludegraphics{examples/demos/mandelbrot/1.png}
\end{figure}

Let's try to understand how it works.

\subsubsection{Theory}

\myparagraph{A word about complex numbers}

A complex number is a number that consists of two parts---real (Re) and imaginary (Im).


The complex plane is a two-dimensional plane where any complex number can be placed: the real part is one coordinate
and the imaginary part is the other.

Some basic rules we have to keep in mind:

\begin{itemize}
\item Addition: $(a+bi) + (c+di) = (a+c) + (b+d)i$

In other words:

$\operatorname{Re}(sum) = \operatorname{Re}(a) + \operatorname{Re}(b)$

$\operatorname{Im}(sum) = \operatorname{Im}(a) + \operatorname{Im}(b)$

\item Multiplication: $(a+bi) (c+di) = (ac-bd) + (bc+ad)i$

In other words:

$\operatorname{Re}(product) = \operatorname{Re}(a) \cdot \operatorname{Re}(c) - \operatorname{Re}(b) \cdot \operatorname{Re}(d)$

$\operatorname{Im}(product) = \operatorname{Im}(b) \cdot \operatorname{Im}(c) + \operatorname{Im}(a) \cdot \operatorname{Im}(d)$

\item Square: $(a+bi)^2 = (a+bi) (a+bi) = (a^2-b^2) + (2ab)i$

In other words:

$\operatorname{Re}(square) = \operatorname{Re}(a)^2-\operatorname{Im}(a)^2$

$\operatorname{Im}(square) = 2 \cdot \operatorname{Re}(a) \cdot \operatorname{Im}(a)$

\end{itemize}

\myparagraph{How to draw the Mandelbrot set}

The Mandelbrot set is a set of points for which the $z_{n+1} = {z_n}^2 + c$ recursive sequence
(where $z$ and $c$ are complex numbers and $c$ 
is the starting value)
does not approach infinity.\\
\\
In plain English language: 

\begin{itemize}
\item Enumerate all points on screen. 
\item Check if the specific point 
is in the Mandelbrot set.
\item Here is how to check it:

  \begin{itemize}
  \item Represent the point as a complex number.
  \item Calculate the square of it.
  \item Add the starting value of the point to it.
  \item Does it go off limits? If yes, break.
  \item Move the point to the 
new place at the coordinates we just calculated.
  \item Repeat all this for some reasonable 
number of iterations.
  \end{itemize}

\item The point is still in limits?
Then draw the point.

\item The point has eventually gone off limits?

  \begin{itemize}
    \item (For a black-white image) do not draw anything.
    \item 

(For a colored image) transform the number of iterations to some color. 
      So the color shows the speed with which point has gone off limits.
  \end{itemize}

\end{itemize}

%
Here is Pythonesque algorithm for both complex and integer number representations:

\lstinputlisting[caption=For complex numbers]{examples/demos/mandelbrot/algo_cplx_EN.lst}


The integer version is where the operations on complex numbers are replaced with integer operations according to the rules
which were explained above.

\lstinputlisting[caption=For integer numbers]{examples/demos/mandelbrot/algo_int_EN.lst}

Here is also a C\# source 
which is present in the Wikipedia article\footnote{\href{http://go.yurichev.com/17307}{wikipedia}}, but we'll modify it
so it will print the iteration numbers instead of some symbol
\footnote{Here is also the executable file: 
\href{http://go.yurichev.com/17163}{beginners.re}}:

\lstinputlisting[style=customc]{examples/demos/mandelbrot/dump_iterations.cs}

Here is the resulting file, 
which is too wide to be included here: \\
\href{http://go.yurichev.com/17164}{beginners.re}.

The maximal number of iterations is 40, so when you see 40 in this dump, it means that this point has been wandering
for 40 iterations but never got off limits. 

A number $n$ less than 40 means that point remained inside the bounds only for $n$ iterations, 
then it went outside them.

\clearpage
There is a cool demo available at 
\url{http://go.yurichev.com/17309}, which shows
visually how the point moves on the plane at each iteration for some specific point. 
Here are two screenshots.

%
First, we've clicked inside the yellow area and saw that the trajectory (green line)
eventually swirls at some point inside:

\begin{figure}[H]
\centering
\includegraphics[width=0.7\textwidth]{examples/demos/mandelbrot/demo1.png}
\caption{Click inside yellow area}
\end{figure}


This implies that the point we've clicked belongs to the Mandelbrot set.

\clearpage

Then we've clicked outside the yellow area and saw a much more chaotic point movement, 
which quickly went off bounds:

\begin{figure}[H]
\centering
\includegraphics[width=0.7\textwidth]{examples/demos/mandelbrot/demo2.png}
\caption{Click outside yellow area}
\end{figure}

This means the point doesn't belong to Mandelbrot set.

Another good demo is available here: 
\url{http://go.yurichev.com/17310}.

\clearpage
\subsubsection{Let's get back to the demo}


The demo, although very tiny (just 64 bytes or 30 instructions), implements the common algorithm 
described here, but using some coding tricks.

%
The source code is easily downloadable, so here is it, but let's also add comments:

\lstinputlisting[caption=Commented source code,numbers=left,style=customasmx86]{examples/demos/mandelbrot/Microbrot_commented_EN.asm}

Algorithm:

\begin{itemize}
\item Switch to 320*200 VGA video mode, 256 colors. 
$320*200=64000$ (0xFA00). 

Each pixel is encoded by one byte, so the buffer size is 0xFA00 bytes.
It is addressed using the ES:DI registers pair.

\myindex{x86!\Registers!ES}
ES must be 0xA000 here, because this is the segment address of 
the VGA video buffer, but storing 0xA000 to ES requires at least 4 bytes (\TT{PUSH 0A000h / POP ES}). 
You can read more about the 16-bit MS-DOS memory model here: 
\myref{8086_memory_model}.

\myindex{x86!\Instructions!LES}

Assuming that BX is zero here, and the Program Segment Prefix is at the zeroth
address, the 2-byte \TT{LES AX,[BX]} instruction stores 0x20CD to AX and 0x9FFF to ES.

So the program starts to draw 16 pixels (or bytes) before the actual video buffer.
But this is MS-DOS, 

there is no memory protection, so a write happens into the very end of conventional memory, and usually, there is nothing important.
That's why you see a red strip 16 
pixels wide at the right side.
The whole picture is shifted left by 16 pixels.
This is the price of saving 2 bytes.

\item An infinite loop processes each pixel.

Probably, the most common way to enumerate all pixels on the screen is with two loops: 
one for the X coordinate, another for the Y coordinate.
But then you'll need 
to multiply the coordinates to address a byte in the VGA video buffer.

The author of this demo decided to do it otherwise: enumerate all bytes in the video buffer by using one single loop instead 
of two, and get the coordinates of the current point using division.
The resulting coordinates are: X in the range of $-256..63$ and Y in the range of $-100..99$.
You can see on 
the screenshot that the picture is somewhat shifted to the right part of screen.

That's because the biggest heart-shaped black hole usually appears on coordinates 0,0 and these are shifted
here to right.
Could the author just 
subtract 160 from the value to get X in the range of $-160..159$? 
Yes, but the instruction \TT{SUB DX, 160} takes 4 bytes, 
while \TT{DEC DH}---2 bytes 
(which subtracts 0x100 (256) from DX). 
So the whole picture is shifted for the cost of 
another 2 bytes of saved space.

    \begin{itemize}
    \item Check, if the current 
point is inside the Mandelbrot set.
          The algorithm is the one that has been described here.
\myindex{x86!\Instructions!LOOP}
     \item The loop 
is organized using the \TT{LOOP} instruction, which uses the CX register as counter.

The author could set the number of iterations to some specific number, but he didn't: 320 is already present in CX 
(has been set at line 35), and this is good maximal iteration number anyway.
We save here some space 
by not the reloading CX register with another value.

\myindex{x86!\Instructions!SAR}
     \item 
\TT{IMUL} is used here instead of \TT{MUL}, because we work with signed values: 
keep in mind that the 0,0 coordinates has to be somewhere near the center of the screen.

It's the same with \TT{SAR} (arithmetic shift for signed values): it's used instead of \TT{SHR}.

     \item Another idea is to simplify the bounds check.
We must check a coordinate pair, i.e., two variables.
What the author does is to checks thrice for overflow: two squaring operations and one addition.

Indeed, we use 16-bit registers, which hold signed values in the range of -32768..32767, 
so if any of the coordinates is greater than 32767 during the signed multiplication, this point is definitely out 
of bounds: we jump to the \TT{MandelBreak} label.

     \item 
There is also a division by 64 (SAR instruction). 64 sets scale.

Try to increase the value and you can get a closer look, or to decrease if for a more distant look.

    \end{itemize}

\item We are at the \TT{MandelBreak} label, there are two ways of getting here: 
the loop ended with CX=0 (
the point is inside the Mandelbrot set); or because an overflow has happened (CX still holds some value).
Now we write the low 8-bit part of CX (CL) to the 
video buffer.

The default palette is rough, nevertheless, 0 is black: hence we see black holes in the places where the points are
in the Mandelbrot set.
The palette can be initialized at the program's start, but keep in mind, this is only a 64 bytes program!

\item The program runs in an infinite loop, 
because an additional check where to stop, or any user interface will result in additional instructions.

\end{itemize}

Some other optimization tricks:

\myindex{x86!\Instructions!CWD}
\begin{itemize}
\item The 1-byte CWD is used here 
for clearing DX instead of the 2-byte \TT{XOR DX, DX} or even the 3-byte \TT{MOV DX, 0}.

\item The 1-byte \TT{XCHG AX, CX} is used instead of the 2-byte 
\TT{MOV AX,CX}. 
The current value of AX is not needed here anyway.

\item DI (position in video buffer) is not initialized, and it is 0xFFFE at the start
\footnote{More information about initial register values: 
\url{http://go.yurichev.com/17004}}.

That's OK, because the program works for all DI in the range of 0..0xFFFF eternally, 
and the user can't notice
that it is started off the screen (the last pixel of a 320*200 video buffer is at address 0xF9FF).
So some work is actually done 
off the limits of the screen.

Otherwise, you'll need an additional instructions to set DI to 0 and check for the video buffer's end.

\end{itemize}

\newcommand{\MyFixedVersion}{My \q{fixed} version}
\subsubsection{\MyFixedVersion}

\lstinputlisting[caption=\MyFixedVersion,numbers=left,style=customasmx86]{examples/demos/mandelbrot/my_version_EN.asm}


The author of these lines made an attempt to fix all these oddities: now the palette is smooth grayscale, the video buffer is at the correct place 
(lines 19..20),
the picture is drawn on center of the screen (line 30), the program eventually ends and waits for the user's keypress 
(lines 58..68).

But now it's much bigger: 105 bytes (or 54 instructions)
\footnote{
You can experiment by yourself: get DosBox and NASM and compile it as: 
\TT{nasm fiole.asm -fbin -o file.com}}.

\begin{figure}[H]
\centering
\myincludegraphics{examples/demos/mandelbrot/fixed.png}
\caption{\MyFixedVersion}
\label{fig:mandelbrot_fixed}
\end{figure}
}
\RU{\clearpage
\subsection{Множество Мандельброта}
\label{Mandelbrot_demo}

\epigraph{You know, if you magnify the coastline, it still looks like
a coastline, and a lot of other things have this property. Nature has
recursive algorithms that it uses to generate clouds and Swiss cheese
and things like that.}
{Дональд Кнут, интервью (1993)}

Множество Мандельброта это фрактал, характерное свойство которого это самоподобие.

При увеличении картинки, вы видите, что этот характерный узор повторяется бесконечно.

Вот демо\footnote{Можно скачать \href{http://go.yurichev.com/17306} {здесь},} 
написанное автором по имени \q{Sir\_Lagsalot} в 2009, 
рисующее множество Мандельброта, и это программа для x86 с размером файла всего 64 байта.
Там только 30 16-битных x86-инструкций.

Вот что она рисует:

\begin{figure}[H]
\centering
\myincludegraphics{examples/demos/mandelbrot/1.png}
\end{figure}

Попробуем разобраться, как она работает.

\subsubsection{Теория}

\myparagraph{Немного о комплексных числах}

Комплексное число состоит из двух чисел (вещественная (Re) и мнимая (Im).

Комплексная плоскость --- это двухмерная плоскость, где любое комплексное число может быть расположено:
вещественная часть --- это одна координата и мнимая --- вторая.

Некоторые базовые правила, которые нам понадобятся:

\begin{itemize}
\item Сложение: $(a+bi) + (c+di) = (a+c) + (b+d)i$

Другими словами:

$\operatorname{Re}(sum) = \operatorname{Re}(a) + \operatorname{Re}(b)$

$\operatorname{Im}(sum) = \operatorname{Im}(a) + \operatorname{Im}(b)$

\item Умножение: $(a+bi) (c+di) = (ac-bd) + (bc+ad)i$

Другими словами:

$\operatorname{Re}(product) = \operatorname{Re}(a) \cdot \operatorname{Re}(c) - \operatorname{Re}(b) \cdot \operatorname{Re}(d)$

$\operatorname{Im}(product) = \operatorname{Im}(b) \cdot \operatorname{Im}(c) + \operatorname{Im}(a) \cdot \operatorname{Im}(d)$

\item Возведение в квадрат: $(a+bi)^2 = (a+bi) (a+bi) = (a^2-b^2) + (2ab)i$

Другими словами:

$\operatorname{Re}(square) = \operatorname{Re}(a)^2-\operatorname{Im}(a)^2$

$\operatorname{Im}(square) = 2 \cdot \operatorname{Re}(a) \cdot \operatorname{Im}(a)$

\end{itemize}

\myparagraph{Как нарисовать множество Мандельброта}

Множество Мандельброта --- это набор точек, для которых рекурсивное соотношение
 $z_{n+1} = {z_n}^2 + c$ 
(где $z$ и $c$ это комплексные числа и $c$ это начальное значение) не стремится к бесконечности.\\
\\
Простым русским языком: 

\begin{itemize}
\item Перечисляем все точки на экране. 
\item Проверяем, является ли эта точка в множестве Мандельброта.
\item Вот как проверить:

  \begin{itemize}
  \item Представим точку как комплексное число.
  \item Возведем в квадрат.
  \item Прибавим значение точки в самом начале.
  \item Вышло за пределы? Прерываемся, если да.
  \item Передвигаем точку в новое место, координаты которого только что вычислили.
  \item Повторять всё это некое разумное количество итераций.
  \end{itemize}

\item Двигающаяся точка в итоге не вышла за пределы?
Тогда рисуем точку.

\item Двигающаяся точка в итоге вышла за пределы?

  \begin{itemize}
    \item (Для черно-белого изображения) ничего не рисуем.
    \item 
(Для цветного изображения) преобразуем количество итераций в какой-нибудь цвет.
Так что цвет будет показывать, с какой скоростью точка вышла за пределы.

  \end{itemize}

\end{itemize}

Вот алгоритмы для комплексных и обычных целочисленных чисел (на языке, отдаленно напоминающем Python):%


\lstinputlisting[caption=Для комплексных чисел]{examples/demos/mandelbrot/algo_cplx_RU.lst}

Целочисленная версия, это версия где все операции над комплексными числами заменены на операции 
с целочисленными, в соответствии с изложенными ранее правилами.


\lstinputlisting[caption=Для целочисленных чисел]{examples/demos/mandelbrot/algo_int_RU.lst}

Вот также исходный текст на C\#, который есть в статье в Wikipedia\footnote{\href{http://go.yurichev.com/17307}{wikipedia}}, но мы немного изменим его,
чтобы он выдавал количество итераций, вместо некоторого символа
\footnote{Здесь также и исполняемый файл: 
\href{http://go.yurichev.com/17163}{beginners.re}}:

\lstinputlisting[style=customc]{examples/demos/mandelbrot/dump_iterations.cs}

Вот файл с результатом, который слишком широкий, чтобы привести его здесь: \\
\href{http://go.yurichev.com/17164}{beginners.re}.

Максимальное число итераций 40, так что если вы видите 40 в этом файле, это означает, что точка ходила
40 итераций, но так и не вышла за пределы.
 
Номер $n$ меньше 40 означает, что эта точка оставалась внутри пределов только $n$ итераций, и затем
вышла наружу.


\clearpage
Вот здесь есть неплохая демонстрация: 
\url{http://go.yurichev.com/17309}, она показывает визуально,
как определенная точка двигается по плоскости на каждой итерации. 
Вот два скриншота.

В начале кликаем внутри желтой области, и увидим траекторию (зеленые линии), которая в итоге
закручивается в какой-то точке внутри:%


\begin{figure}[H]
\centering
\includegraphics[width=0.7\textwidth]{examples/demos/mandelbrot/demo1.png}
\caption{Клик внутри желтой области}
\end{figure}

Это значит, что точка на которой кликнули, находится внутри множества Мандельброта.


\clearpage
Затем кликаем снаружи желтой области, и мы видим более хаотичные движения точки, которая быстро выходит
за пределы:


\begin{figure}[H]
\centering
\includegraphics[width=0.7\textwidth]{examples/demos/mandelbrot/demo2.png}
\caption{Клик снаружи желтой области}
\end{figure}

Это значит, что эта точка не принадлежит множеству Мандельброта.

Другая неплохая демонстрация там: 
\url{http://go.yurichev.com/17310}.

\clearpage
\subsubsection{Вернемся к демо}

Демо, хотя и крошечная (только 64 байта или 30 инструкций), реализует общий алгоритм, изложенный
здесь, но с некоторыми трюками.


Исходный код можно скачать, так что вот он, но также снабдим его своими комментариями:

\lstinputlisting[caption=Исходный код с комментариями,numbers=left,style=customasmx86]{examples/demos/mandelbrot/Microbrot_commented_RU.asm}

Алгоритм:

\begin{itemize}
\item Переключаемся в режим VGA 320*200 256 цветов. 
$320*200=64000$ (0xFA00). 
Каждый пиксель кодируется одним байтом, так что размер буфера 0xFA00 байт.

Он адресуется здесь при помощи пары регистров ES:DI.

\myindex{x86!\Registers!ES}
ES должен быть здесь 0xA000, потому что это сегментный адрес видеобуфера, но запись
числа 0xA000 в ES потребует по крайней мере 4 байта (\TT{PUSH 0A000h / POP ES}). 
О 16-битной модели памяти в MS-DOS, читайте больше тут: 
\myref{8086_memory_model}.

\myindex{x86!\Instructions!LES}
Учитывая, что BX здесь 0, и Program Segment Prefix находится по нулевому адресу, 2-байтная инструкция
\TT{LES AX,[BX]} запишет 0x20CD в AX и 0x9FFF в ES.

Так что программа начнет рисовать на 16 пикселей (или байт) перед видеобуфером.

Но это MS-DOS, 
здесь нет защиты памяти, так что запись происходит в самый конец обычной памяти, а там, как правило, ничего важного нет.

Вот почему вы видите красную полосу шириной 16 пикселей справа.
Вся картинка сдвинута налево на 16 пикселей.
Это цена экономии 2-х байт.

\item Вечный цикл, обрабатывающий каждый пиксель.
Наверное, самый общий метод обойти все точки на экране это два цикла:
один для X-координаты, второй для Y-координаты.

Но тогда вам придется перемножать координаты для поиска байта в видеобуфере VGA.
Автор этого демо решил сделать наоборот: перебирать все байты в видеобуфере при помощи одного цикла
вместо двух и затем получать координаты текущей точки при помощи деления.

В итоге координаты такие: X в пределах $-256..63$ и Y 
в пределах $-100..99$.
Вы можете увидеть на скриншоте что картинка как бы сдвинута в правую часть экрана.
Это потому что самая большая черная дыра в форме сердца обычно появляется на координатах 0,0 и они
здесь сдвинуты вправо.

Мог ли автор просто отнять 160 от X, чтобы получилось значение в пределах $-160..159$? 
Да, но инструкция \TT{SUB DX, 160} занимает 4 байта, 
тогда как \TT{DEC DH} --- 2 байта 
(которая отнимает 0x100 (256) от DX). 
Так что картинка сдвинута ценой экономии еще 2-х байт.

    \begin{itemize}
    \item Проверить, является ли текущая точка внутри множества Мандельброта.
          Алгоритм такой же, как и описанный здесь.
\myindex{x86!\Instructions!LOOP}
     \item Цикл организуется инструкцией \TT{LOOP}, которая использует регистр CX как счетчик.
Автор мог бы установить число итераций на какое-то число, но не сделал этого: потому что 320 уже
находится в CX (было установлено на строке 35), и это итак подходящее число как число максимальных
итераций.

Мы здесь экономим немного места, не загружая другое значение в регистр CX.

\myindex{x86!\Instructions!SAR}
     \item Здесь используется \TT{IMUL} вместо \TT{MUL}, потому что мы работаем с знаковыми значениями:
помните, что координаты 0,0 должны быть где-то рядом с центром экрана.

Тоже самое и с \TT{SAR} (арифметический сдвиг для знаковых значений): она используется вместо \TT{SHR}.


     \item Еще одна идея --- это упростить проверку пределов.
Нам бы пришлось проверять пару координат, т.е. две переменных.
Что делает автор это трижды проверяет на переполнение: две операции возведения в квадрат и одно 
прибавление.
Действительно, мы ведь используем 16-битные регистры, содержащие знаковые значения в пределах
 -32768..32767, так что
если любая из координат больше чем 32767 в процессе умножения, точка однозначно вышла за пределы,
и мы переходим на метку \TT{MandelBreak}.


     \item Здесь также имеется деление на 64 (при помощи инструкции SAR). 64 задает масштаб.

Попробуйте увеличить значение и вы получите более увеличенную картинку, или уменьшить для
меньшей.


    \end{itemize}

\item Мы находимся на метке \TT{MandelBreak}, есть только две возможности
попасть сюда: 
цикл закончился с CX=0 (точка внутри множества Мандельброта
); или потому что произошло переполнение (CX все еще содержит 
какое-то значение).
Записываем 8-битную часть CX (CL) в видеобуфер.
Палитра по умолчанию грубая, тем не менее, 0 это черный: поэтому видим черные дыры в местах где точки
внутри множества Мандельброта.

Палитру можно инициализировать в начале программы, но не забывайте, это всего лишь программа на 64 
байта!

\item Программа работает в вечном цикле, потому что дополнительная проверка, где остановится, 
или пользовательский интерфейс, это дополнительные инструкции.

\end{itemize}

Еще оптимизационные трюки:

\myindex{x86!\Instructions!CWD}
\begin{itemize}
\item 1-байтная CWD используется здесь для обнуления DX вместо двухбайтной \TT{XOR DX, DX} или даже трехбайтной \TT{MOV DX, 0}.

\item 1-байтная \TT{XCHG AX, CX} используется вместо двухбайтной 
\TT{MOV AX,CX}. 
Текущее значение в AX все равно уже не нужно.

\item DI (позиция в видеобуфере) не инициализирована, и будет 0xFFFE в
начале
\footnote{Больше о состояниях регистров на старте: 
\url{http://go.yurichev.com/17004}}.
Это нормально, потому что программа работает бесконечно для всех DI в пределах 0..0xFFFF, и пользователь
не может увидеть, что работала началась за экраном (последний пиксель видеобуфера 320*200 имеет адрес 0xF9FF).

Так что некоторая часть работы на самом деле происходит за экраном.
А иначе понадобятся дополнительные инструкции для установки DI в 0; добавить проверку на конец буфера.


\end{itemize}

\newcommand{\MyFixedVersion}{Моя \q{исправленная} версия}
\subsubsection{\MyFixedVersion}

\lstinputlisting[caption=\MyFixedVersion,numbers=left,style=customasmx86]{examples/demos/mandelbrot/my_version_RU.asm}

Автор сих строк попытался исправить все эти странности: теперь палитра плавная черно-белая, видеобуфер на правильном месте
(строки 19..20), картинка рисуется в центре экрана (строка 30), программа в итоге заканчивается и ждет,
пока пользователь нажмет какую-нибудь клавишу (строки 58..68).

Но теперь она намного больше: 105 байт (или 54 инструкции)

\footnote{Можете поэкспериментировать и сами: скачайте DosBox и NASM и компилируйте так:
 
\TT{nasm fiole.asm -fbin -o file.com}}.

\begin{figure}[H]
\centering
\myincludegraphics{examples/demos/mandelbrot/fixed.png}
\caption{\MyFixedVersion}
\label{fig:mandelbrot_fixed}
\end{figure}
}



\chapter{\RU{Примеры разбора закрытых (proprietary) форматов файлов}\EN{Examples of reversing proprietary file formats}\ESph{}\PTBRph{}\PLph{}\ITAph{}
\DE{Beispiele für das Reverse Engineering proprietärer Dateiformate}
\NLph{}}

% chapters
\section{\RU{Примитивное XOR-шифрование}\EN{Primitive XOR-encryption}\FR{Chiffrement primitif avec XOR}}
\label{simple_XOR_encryption}

\ifdefined\RUSSIAN
В русскоязычной литературе также используется термин \IT{гаммирование}.
\fi

% subsections
\EN{% TODO translate
\subsection{Simplest ever XOR encryption}

I once saw a software where all debugging messages has been encrypted using XOR by value of 3.
In other words, two lowest bits of all characters has been flipped.

``Hello, world'' would become ``Kfool/\#tlqog'':

\begin{lstlisting}
#!/usr/bin/python

msg="Hello, world!"

print "".join(map(lambda x: chr(ord(x)^3), msg))
\end{lstlisting}

This is quite interesting encryption (or rather obfuscation), because it has two important properties:
1) single function for encryption/decryption, just apply it again;
2) resulting characters are also printable, so the whole string can be used in source code without escaping characters.

The second property exploits the fact that all printable characters organized in rows: 0x2x-0x7x, and when you 
flip two lowest bits, character \IT{moving} 1 or 3 characters left or right, but never \IT{moved} to another (maybe
non-printable) row:

\begin{figure}[H]
\centering
\includegraphics[width=0.7\textwidth]{ascii_clean.png}
\caption{7-bit \ac{ASCII} table in Emacs}
\end{figure}

\dots with a single exception of 0x7F character.

For example, let's \IT{encrypt} characters in A-Z range:

\begin{lstlisting}
#!/usr/bin/python

msg="@ABCDEFGHIJKLMNO"

print "".join(map(lambda x: chr(ord(x)^3), msg))
\end{lstlisting}

Result:
% FIXME \verb
\begin{lstlisting}
CBA@GFEDKJIHONML
\end{lstlisting}

It's like ``@'' and ``C'' characters has been swapped, and so are ``B'' and ``a''.

Yet again, this is interesting example of exploiting XOR properties, rather than encryption:
the very same effect of \IT{preserving printableness} can be achieved while flipping any of lowest 4 bits,
in any combination.

}\FR{% TODO translate
\subsection{Chiffrement XOR le plus simple}

J'ai vu une fois un logiciel où tous les messages de débogage étaient chiffrés en
utilisant XOR avec une valeur de 3.
Autrement dit, les deux bits les plus bas de chaque caractères étaient inversés.

``Hello, world'' devenait ``Kfool/\#tlqog'':

\begin{lstlisting}
#!/usr/bin/python

msg="Hello, world!"

print "".join(map(lambda x: chr(ord(x)^3), msg))
\end{lstlisting}

Ceci est un chiffrement assez intéressant (ou plutôt une offuscation), car il possède
deux propriétés importantes:
1) fonction unique pour le chiffrement/déchiffrement, il suffit de l'appliquer à nouveau;
2) les caractères résultants sont aussi imprimable, donc la chaîne complète peut être
utilisée dans du code source, sans caractères d'échappement.

La seconde propriété exploite le fait que tous les caractères imprimables sont organisés
en lignes: 0x2x-0x7x, et lorsque vous inversez les deux bits de poids faible, le caractère
est \IT{déplacé} de 1 ou 3 caractères à droite ou à gauche, mais n'est jamais \IT{déplacé}
sur une autre ligne (peut-être non imprimable):

\begin{figure}[H]
\centering
\includegraphics[width=0.7\textwidth]{ascii_clean.png}
\caption{Table \ac{ASCII} 7-bit dans Emacs}
\end{figure}

\dots avec la seule exception du caractère 0x7F.

Par exemple, \IT{chiffrons} les caractères de l'intervalle A-Z:

\begin{lstlisting}
#!/usr/bin/python

msg="@ABCDEFGHIJKLMNO"

print "".join(map(lambda x: chr(ord(x)^3), msg))
\end{lstlisting}

Résultat:
% FIXME \verb
\begin{lstlisting}
CBA@GFEDKJIHONML
\end{lstlisting}

C'est comme si les caractères ``@'' et ``C'' avaient été échangés, ainsi que ``B''
et ``a''.

Encore une fois, ceci est un exemple intéressant de l'exploitation des propriétés
de XOR plutôt qu'un chiffrement:
le même effet de \IT{préservation de l'imprimabilité} peut être obtenu en échangeant
chacun des 4 bits de poids faible, avec n'importe quelle combinaison.

}
\EN{\clearpage
\subsection{Norton Guide: simplest possible 1-byte XOR encryption}
\label{norton_guide}

Norton Guide\footnote{\href{http://go.yurichev.com/17116}{wikipedia}} was popular in the epoch of MS-DOS, it was a resident program that worked as a hypertext reference manual.

Norton Guide's databases are files with the extension .ng, the contents of which look encrypted:

\begin{figure}[H]
\centering
\myincludegraphics{ff/XOR/ng/ng1.png}
\caption{Very typical look}
\end{figure}

Why did we think that it's encrypted but not compressed?

We see that the 0x1A byte (looking like \q{$\rightarrow$}) occurs often, it would not be possible in a compressed file.

We also see long parts that consist only of Latin letters, and they look like strings in an unknown
language.

\clearpage
Since the 0x1A byte occurs so often, we can try to decrypt the file, assuming that it's encrypted
by the simplest XOR-encryption.

If we apply XOR with the 0x1A constant to each byte in Hiew, we can see familiar English text strings:

\begin{figure}[H]
\centering
\myincludegraphics{ff/XOR/ng/ng2.png}
\caption{Hiew XORing with 0x1A}
\end{figure}

XOR encryption with one single constant byte is the simplest possible encryption method, which is, nevertheless,
encountered sometimes.

Now we understand why the 0x1A byte is occurring so often: because there are so many zero bytes and they
were replaced by 0x1A in encrypted form.

But the constant might be different.
In this case, we could try every constant in the 0..255 range and look for something familiar in the decrypted
file. 256 is not so much.

More about Norton Guide's file format: \url{http://go.yurichev.com/17317}.

\subsubsection{Entropy}
\myindex{Wolfram Mathematica}
\myindex{Entropy}

A very important property of such primitive encryption systems is that the information entropy
of the encrypted/decrypted block is the same.

Here is my analysis in Wolfram Mathematica 10.

\begin{lstlisting}[caption=Wolfram Mathematica 10,style=custommath]
In[1]:= input = BinaryReadList["X86.NG"];

In[2]:= Entropy[2, input] // N
Out[2]= 5.62724

In[3]:= decrypted = Map[BitXor[#, 16^^1A] &, input];

In[4]:= Export["X86_decrypted.NG", decrypted, "Binary"];

In[5]:= Entropy[2, decrypted] // N
Out[5]= 5.62724

In[6]:= Entropy[2, ExampleData[{"Text", "ShakespearesSonnets"}]] // N
Out[6]= 4.42366
\end{lstlisting}

What we do here is load the file, get its entropy, decrypt it, save it and get the entropy again (the same!).

Mathematica also offers some well-known English language texts for analysis.

So we also get the entropy of Shakespeare's sonnets, and it is close to the entropy of the file we just analyzed.

The file we analyzed consists of English language sentences, which are close to the language 
of Shakespeare.

And the XOR-ed bitwise English language text has the same entropy.

% I checked!
However, this is not true when the file is XOR-ed with a pattern larger than one byte.

The file we analyzed can be downloaded here: \url{http://go.yurichev.com/17350}.

\myparagraph{One more word about base of entropy}

\newcommand{\FNENTURL}{\footnote{\url{http://www.fourmilab.ch/random/}}}

Wolfram Mathematica calculates entropy with base of $e$ (base of the natural logarithm),
and the UNIX \IT{ent} utility\FNENTURL uses base 2.

So we set base 2 explicitly in \TT{Entropy} command, so Mathematica will give us the same results as the \IT{ent} utility.

}\RU{\clearpage
\subsection{Norton Guide: простейшее однобайтное XOR-шифрование}
\label{norton_guide}

Norton Guide\footnote{\href{http://go.yurichev.com/17116}{wikipedia}} был популярен во времена MS-DOS, это была резидентная программа, работающая как
гипертекстовый справочник.

Базы данных Norton Guide это файлы с расширением .ng, содержимое которых выглядит как зашифрованное:

\begin{figure}[H]
\centering
\myincludegraphics{ff/XOR/ng/ng1.png}
\caption{Очень типичный вид}
\end{figure}

Почему мы думаем, что зашифрованное а не сжатое? 

Мы видим, как слишком часто попадается байт 0x1A (который выглядит как \q{$\rightarrow$}), в сжатом файле такого не было бы никогда.

Во-вторых, мы видим длинные части состоящие только из латинских букв, они выглядят как строки
на незнакомом языке.

\clearpage

Из-за того, что байт 0x1A слишком часто встречается, мы можем попробовать расшифровать файл, полагая
что он зашифрован простейшим XOR-шифрованием.

Применяем XOR с константой 0x1A к каждому байту в Hiew и мы можем видеть знакомые текстовые строки на английском:

\begin{figure}[H]
\centering
\myincludegraphics{ff/XOR/ng/ng2.png}
\caption{Hiew применение XOR с 0x1A}
\end{figure}

XOR-шифрование с одним константным байтом это самый простой способ шифрования, который, тем не менее, иногда
встречается.

Теперь понятно почему байт 0x1A так часто встречался: потому что в файле очень много нулевых байт 
и в зашифрованном виде они везде были заменены на 0x1A.

Но эта константа могла быть другой.

В таком случае, можно было бы попробовать перебрать все 256 комбинаций, и посмотреть содержимое \q{на глаз}, 
а 256 --- это совсем немного.

Больше о формате файлов Norton Guide: \url{http://go.yurichev.com/17317}.

\subsubsection{Энтропия}
\myindex{Wolfram Mathematica}
\myindex{Энтропия}

Очень важное свойство подобного примитивного шифрования в том, что информационная энтропия
зашифрованного/дешифрованного блока точно такая же.
Вот мой анализ в Wolfram Mathematica 10.

\begin{lstlisting}[caption=Wolfram Mathematica 10,style=custommath]
In[1]:= input = BinaryReadList["X86.NG"];

In[2]:= Entropy[2, input] // N
Out[2]= 5.62724

In[3]:= decrypted = Map[BitXor[#, 16^^1A] &, input];

In[4]:= Export["X86_decrypted.NG", decrypted, "Binary"];

In[5]:= Entropy[2, decrypted] // N
Out[5]= 5.62724

In[6]:= Entropy[2, ExampleData[{"Text", "ShakespearesSonnets"}]] // N
Out[6]= 4.42366
\end{lstlisting}

Что мы здесь делаем это загружаем файл, вычисляем его энтропию, дешифруем его, сохраняем, снова вычисляем энтропию (точно такая же!).

Mathematica дает возможность анализировать некоторые хорошо известные англоязычные тексты.

Так что мы вычисляем энтропию сонетов Шейкспира, и она близка к энтропии анализируемого нами файла.

Анализируемый нами файл состоит из предложений на английском языке, которые близки к языку
Шейкспира.

И применение побайтового XOR к тексту на английском языке не меняет энтропию.

% I checked!

Хотя, это не будет справедливо когда файл зашифрован при помощи XOR шаблоном длиннее одного байта.

Файл, который мы анализировали, можно скачать здесь: \url{http://go.yurichev.com/17350}.

\myparagraph{Еще кое-что о базе энтропии}

\newcommand{\FNENTURL}{\footnote{\url{http://www.fourmilab.ch/random/}}}

Wolfram Mathematica вычисляет энтропию с базой $e$ (основание натурального логарифма),
а утилита UNIX \IT{ent}\FNENTURL использует базу 2.

Так что мы явно указываем базу 2 в команде \TT{Entropy}, чтобы Mathematica давала те же результаты, что и утилита \IT{ent}.
}%
\FR{\clearpage
\subsection{Norton Guide: chiffrement XOR à 1 octet le plus simple possible}
\label{norton_guide}

Norton Guide\footnote{\href{http://go.yurichev.com/17116}{wikipédia}} était très
populaire à l'époque de MS-DOS, c'était un programme résident qui fonctionnait comme
un manuel de référence hypertexte.

Les bases de données de Norton Guide étaient des fichiers avec l'extension .ng, dont
le contenu avait l'air chiffré:

\begin{figure}[H]
\centering
\myincludegraphics{ff/XOR/ng/ng1.png}
\caption{Aspect très typique}
\end{figure}

Pourquoi pensons-nous qu'il est chiffré mais pas compressé?

Nous voyons que l'octet 0x1A (ressemblant à \q{$\rightarrow$}) est très fréquent,
ça ne serait pas possible dans un fichier compressé.

Nous voyons aussi de longues parties constituées seulement de lettres latines, et
qui ressemble à des chaînes de caractères dans un langage inconnu.

\clearpage
Puisque l'octet 0x1A revient si souvent, nous pouvons essayer de décrypter le fichier,
en supposant qu'il est chiffré avec le chiffrement XOR le plus simple.

Si nous appliquons un XOR avec la constante 0x1A à chaque octet dans Hiew, nous voyons
des chaînes de texte familières en anglais:

\begin{figure}[H]
\centering
\myincludegraphics{ff/XOR/ng/ng2.png}
\caption{XOR dans Hiew avec 0x1A}
\end{figure}

Le chiffrement XOR avec un seul octet constant est la méthode de chiffrement la plus
simple, que l'on rencontre néanmoins parfois.

Maintenant, nous comprenons pourquoi l'octet 0x1A revenait si souvent: parce qu'il
y a beaucoup d'octets à zéro et qu'ils sont remplacés par 0x1A dans la forme chiffrée.

Mais la constante pourrait être différente.
Dans ce cas, nous pourrions essayer chaque constante dans l'intervalle 0..255 et
chercher quelque chose de familier dans le fichier déchiffré. 256 n'est pas si grand.

Plus d'informations sur le format de fichier de Norton Guide: \url{http://go.yurichev.com/17317}.

\subsubsection{Entropie}
\myindex{Wolfram Mathematica}
\myindex{Entropy}

Une propriété très importante de tels systèmes de chiffrement est que l'entropie
des blocs chiffrés/déchiffrés est la même.

Voici mon analyse faite dans Wolfram Mathematica 10.

\begin{lstlisting}[caption=Wolfram Mathematica 10,style=custommath]
In[1]:= input = BinaryReadList["X86.NG"];

In[2]:= Entropy[2, input] // N
Out[2]= 5.62724

In[3]:= decrypted = Map[BitXor[#, 16^^1A] &, input];

In[4]:= Export["X86_decrypted.NG", decrypted, "Binary"];

In[5]:= Entropy[2, decrypted] // N
Out[5]= 5.62724

In[6]:= Entropy[2, ExampleData[{"Text", "ShakespearesSonnets"}]] // N
Out[6]= 4.42366
\end{lstlisting}

Ici, nous chargeons le fichier, obtenons son entropie, le déchiffrons, le sauvons
et obtenons à nouveau son entropie (la même!).

Mathematica fourni également quelques textes en langue anglaise bien connus pour
analyse.

Nous obtenons ainsi l'entropie de sonnets de Shakespeare, et elle est proche de l'entropie
du fichier que nous venons d'analyser.

Le fichier que nous avons analysé consiste en des phrases en langue anglaise, qui
sont proches du langage de Shakespeare.

Et le texte en langue anglaise XOR-é possède la même entropie.

% I checked!
Toutefois, ceci n'est pas vrai lorsque le fichier est XOR-é avec un pattern de plus
d'un octet.

Le fichier qui vient d'être analysé peut être téléchargé ici: \url{http://go.yurichev.com/17350}.

\myparagraph{Encore un mot sur la base de l'entropie}

\newcommand{\FNENTURL}{\footnote{\url{http://www.fourmilab.ch/random/}}}

Wolfram Mathematica calcule l'entropie avec une base $e$ (base des logarithmes naturels),
et l'utilitaire\FNENTURL UNIX \IT{ent} utilise une base 2.

Donc, nous avons mis explicitement une base 2 dans la commande \TT{Entropy}, donc
Mathematica nous donne le même résultat que l'utilitaire \IT{ent}.

}
\EN{\clearpage
\subsection{Simplest possible 4-byte XOR encryption}

If a longer pattern was used for XOR-encryption, for example a 4 byte pattern, it's easy to spot as well.

For example, here is the beginning of the kernel32.dll file (32-bit version from Windows Server 2008):

\begin{figure}[H]
\centering
\myincludegraphics{ff/XOR/4byte/original1.png}
\caption{Original file}
\end{figure}

\clearpage

Here it is \q{encrypted} with a 4-byte key:

\begin{figure}[H]
\centering
\myincludegraphics{ff/XOR/4byte/encrypted1.png}
\caption{\q{Encrypted} file}
\end{figure}

It's very easy to spot the recurring 4 symbols.

Indeed, the header of a PE-file has a lot of long zero areas, which are the reason for the key to become visible.

\clearpage

Here is the beginning of a PE-header in hexadecimal form:

\begin{figure}[H]
\centering
\myincludegraphics{ff/XOR/4byte/original2.png}
\caption{PE-header}
\end{figure}

\clearpage

Here it is \q{encrypted}:

\begin{figure}[H]
\centering
\myincludegraphics{ff/XOR/4byte/encrypted2.png}
\caption{\q{Encrypted} PE-header}
\end{figure}

It's easy to spot that the key is the following 4 bytes: \TT{8C 61 D2 63}.

With this information, it's easy to decrypt the whole file.

So it is important to keep in mind these properties of PE-files:
1) PE-header has many zero-filled areas;
2) all PE-sections are padded with zeros at a page boundary (4096 bytes),
so long zero areas are usually present after each section.

Some other file formats may contain long zero areas.

It's typical for files used by scientific and engineering software.

For those who want to inspect these files on their own, they are downloadable here: \url{http://go.yurichev.com/17352}.

\subsubsection{\Exercise}

\begin{itemize}
	\item \url{http://challenges.re/50}
\end{itemize}

}\RU{\clearpage
\subsection{Простейшее четырехбайтное XOR-шифрование}

Если при XOR-шифровании применялся шаблон длиннее байта, например, 4-байтный, то его также легко увидеть.

Например, вот начало файла kernel32.dll (32-битная версия из Windows Server 2008):

\begin{figure}[H]
\centering
\myincludegraphics{ff/XOR/4byte/original1.png}
\caption{Оригинальный файл}
\end{figure}

\clearpage
Вот он же, но \q{зашифрованный} 4-байтным ключом:

\begin{figure}[H]
\centering
\myincludegraphics{ff/XOR/4byte/encrypted1.png}
\caption{\q{Зашифрованный} файл}
\end{figure}

Очень легко увидеть повторяющиеся 4 символа.

Ведь в заголовке PE-файла много длинных нулевых областей, из-за которых ключ становится видным.

\clearpage
Вот начало PE-заголовка в 16-ричном виде:

\begin{figure}[H]
\centering
\myincludegraphics{ff/XOR/4byte/original2.png}
\caption{PE-заголовок}
\end{figure}

\clearpage
И вот он же, \q{зашифрованный}:

\begin{figure}[H]
\centering
\myincludegraphics{ff/XOR/4byte/encrypted2.png}
\caption{\q{Зашифрованный} PE-заголовок}
\end{figure}

Легко увидеть визуально, что ключ это следующие 4 байта: \TT{8C 61 D2 63}.
Используя эту информацию, довольно легко расшифровать весь файл.

Таким образом, важно помнить эти свойства PE-файлов:
1) в PE-заголовке много нулевых областей;
2) все PE-секции дополняются нулями до границы страницы (4096 байт), 
так что после всех секций обычно имеются длинные нулевые области.

Некоторые другие форматы файлов могут также иметь длинные нулевые области.

Это очень типично для файлов, используемых научным и инженерным ПО.

Для тех, кто самостоятельно хочет изучить эти файлы, то их можно скачать здесь:

\url{http://go.yurichev.com/17352}.

\subsubsection{\Exercise}

\begin{itemize}
	\item \url{http://challenges.re/50}
\end{itemize}

}%
\FR{\clearpage
\subsection{Chiffrement le plus simple possible avec un XOR de 4-octets}

Si un pattern plus long était utilisé, comme un pattern de 4 octets, ça serait
facile à repérer.

Par exemple, voici le début du fichier kernel32.dll (version 32-bit de Windows Server
2008):

\begin{figure}[H]
\centering
\myincludegraphics{ff/XOR/4byte/original1.png}
\caption{Fichier original}
\end{figure}

\clearpage

Ici, il est \q{chiffré} avec une clef de 4-octet:

\begin{figure}[H]
\centering
\myincludegraphics{ff/XOR/4byte/encrypted1.png}
\caption{Fichier \q{chiffré}}
\end{figure}

Il est facile de repérer les 4 symboles récurrents.

En effet, l'entête d'un fichier PE comporte de longues zones de zéro, ce qui explique
que la clef devient visible.

\clearpage

Voici le début d'un entête PE au format hexadécimal:

\begin{figure}[H]
\centering
\myincludegraphics{ff/XOR/4byte/original2.png}
\caption{Entête PE}
\end{figure}

\clearpage

Le voici \q{chiffré}:

\begin{figure}[H]
\centering
\myincludegraphics{ff/XOR/4byte/encrypted2.png}
\caption{Entête PE \q{chiffré}}
\end{figure}

Il est facile de repérer que la clef est la séquence de 4 octets suivant: \TT{8C 61 D2 63}.

Avec cette information, c'est facile de déchiffrer le fichier entier.

Il est important de garder à l'esprit ces propriétés importantes des fichiers PE:
1) l'entête PE comporte de nombreuses zones remplies de zéro;
2) toutes les sections PE sont complétées avec des zéros jusqu'à une limite de page
(4096 octets),
donc il y a d'habitude de longues zones à zéro après chaque section.

Quelques autres formats de fichier contiennent de longues zones de zéro.

C'est typique des fichiers utilisés par les scientifiques et les ingénieurs logiciels.

Pour ceux qui veulent inspecter ces fichiers eux-même, ils sont téléchargeables ici:
\url{http://go.yurichev.com/17352}.

\subsubsection{\Exercise}

\begin{itemize}
	\item \url{http://challenges.re/50}
\end{itemize}

}
\EN{\subsection{Simple encryption using XOR mask}
\label{XOR_mask_1}

I've found an old interactive fiction game while diving deep into \IT{if-archive}\footnote{\url{http://www.ifarchive.org/}}:

\begin{lstlisting}
The New Castle v3.5 - Text/Adventure Game
in the style of the original Infocom (tm)
type games, Zork, Collosal Cave (Adventure),
etc.  Can you solve the mystery of the
abandoned castle?
Shareware from Software Customization.
Software Customization [ASP] Version 3.5 Feb. 2000
\end{lstlisting}

It's downloadable here: \url{http://yurichev.com/blog/XOR_mask_1/files/newcastle.tgz}.

There is a file inside (named \IT{castle.dbf}) which is clearly encrypted, but not by a real crypto algorithm, nor it's compressed, this is something rather simpler.
I wouldn't even measure entropy level (\myref{entropy}) of the file.
Here is how it looks like in Midnight Commander:

\begin{figure}[H]
\centering
\myincludegraphics{ff/XOR/mask_1/mc_encrypted.png}
\caption{Encrypted file in Midnight Commander}
\end{figure}

The encrypted file can be downloaded \href{http://yurichev.com/blog/XOR_mask_1/files/castle.dbf}{here}.

Will it be possible to decrypt it without accessing to the program, using just this file?

There is a clearly visible pattern of repeating string.
If a simple encryption by XOR mask was applied, such repeating strings is a prominent signature, because, probably, there were a long lacunas of zero bytes,
which, in turn, are present in many executable files as well as in binary data files.

\myindex{UNIX!xxd}
Here I'll dump the file's beginning using \IT{xxd} UNIX utility:

\lstinputlisting{ff/XOR/mask_1/xxd_result.txt}

Let's stick at visible repeating \q{iubgv} string.
By looking at this dump, we can clearly see that the period of the string occurrence is 0x51 or 81.
Probably, 81 is size of block?
Size of the file is 1658961, and it can be divided evenly by 81 (and there are 20481 blocks then).

Now I'll use Mathematica to analyze, are there repeating 81-byte blocks in the file?
I'll split input file by 81-byte blocks and then I'll use 
\IT{Tally[]}\footnote{\url{https://reference.wolfram.com/language/ref/Tally.html}}
function which just calculates, how many times some item has been occurred in the input list.
Tally's output is not sorted, so I also add \IT{Sort[]} function to sort it by number of occurrences in descending order.

\begin{lstlisting}[style=custommath]
input = BinaryReadList["/home/dennis/.../castle.dbf"];

blocks = Partition[input, 81];

stat = Sort[Tally[blocks], #1[[2]] > #2[[2]] &]
\end{lstlisting}

And here is output:

\begin{lstlisting}[style=custommath]
{{{80, 103, 2, 116, 113, 102, 118, 25, 99, 8, 19, 23, 116, 125, 107, 
   25, 99, 109, 114, 102, 14, 121, 115, 31, 9, 117, 113, 111, 5, 4, 
   127, 28, 122, 101, 8, 110, 14, 18, 124, 106, 16, 20, 104, 119, 8, 
   109, 26, 106, 9, 97, 13, 99, 15, 119, 20, 105, 117, 98, 103, 118, 
   1, 126, 29, 97, 122, 17, 15, 114, 110, 3, 5, 125, 125, 99, 126, 
   119, 102, 30, 122, 2, 117}, 1739}, 
{{80, 100, 2, 116, 113, 102, 118, 25, 99, 8, 19, 23, 116, 
   125, 107, 25, 99, 109, 114, 102, 14, 121, 115, 31, 9, 117, 113, 
   111, 5, 4, 127, 28, 122, 101, 8, 110, 14, 18, 124, 106, 16, 20, 
   104, 119, 8, 109, 26, 106, 9, 97, 13, 99, 15, 119, 20, 105, 117, 
   98, 103, 118, 1, 126, 29, 97, 122, 17, 15, 114, 110, 3, 5, 125, 
   125, 99, 126, 119, 102, 30, 122, 2, 117}, 1422}, 
{{80, 101, 2, 116, 113, 102, 118, 25, 99, 8, 19, 23, 116, 
   125, 107, 25, 99, 109, 114, 102, 14, 121, 115, 31, 9, 117, 113, 
   111, 5, 4, 127, 28, 122, 101, 8, 110, 14, 18, 124, 106, 16, 20, 
   104, 119, 8, 109, 26, 106, 9, 97, 13, 99, 15, 119, 20, 105, 117, 
   98, 103, 118, 1, 126, 29, 97, 122, 17, 15, 114, 110, 3, 5, 125, 
   125, 99, 126, 119, 102, 30, 122, 2, 117}, 1012},
{{80, 120, 2, 116, 113, 102, 118, 25, 99, 8, 19, 23, 116, 
   125, 107, 25, 99, 109, 114, 102, 14, 121, 115, 31, 9, 117, 113, 
   111, 5, 4, 127, 28, 122, 101, 8, 110, 14, 18, 124, 106, 16, 20, 
   104, 119, 8, 109, 26, 106, 9, 97, 13, 99, 15, 119, 20, 105, 117, 
   98, 103, 118, 1, 126, 29, 97, 122, 17, 15, 114, 110, 3, 5, 125, 
   125, 99, 126, 119, 102, 30, 122, 2, 117}, 377},

...

{{80, 2, 74, 49, 113, 21, 62, 88, 39, 71, 68, 23, 63, 51, 36, 78, 48, 
   108, 114, 102, 14, 121, 115, 31, 9, 117, 113, 111, 5, 4, 127, 28, 
   122, 101, 8, 110, 14, 18, 124, 106, 16, 20, 104, 119, 8, 109, 26, 
   106, 9, 97, 13, 99, 15, 119, 20, 105, 117, 98, 103, 118, 1, 126, 
   29, 97, 122, 17, 15, 114, 110, 3, 5, 125, 125, 99, 126, 119, 102, 
   30, 122, 2, 117}, 1},
{{80, 1, 74, 59, 113, 45, 56, 86, 52, 91, 19, 64, 60, 60, 63, 
   25, 38, 59, 59, 42, 14, 53, 38, 77, 66, 38, 113, 38, 75, 4, 43, 84,
    63, 101, 64, 43, 79, 64, 40, 57, 16, 91, 46, 119, 69, 40, 84, 117,
    9, 97, 13, 99, 15, 119, 20, 105, 117, 98, 103, 118, 1, 126, 29, 
   97, 122, 17, 15, 114, 110, 3, 5, 125, 125, 99, 126, 119, 102, 30, 
   122, 2, 117}, 1},
{{80, 2, 74, 49, 113, 49, 51, 92, 39, 8, 92, 81, 116, 62, 57, 
   80, 46, 40, 114, 36, 75, 56, 33, 76, 9, 55, 56, 59, 81, 65, 45, 28,
    60, 55, 93, 39, 90, 28, 124, 106, 16, 20, 104, 119, 8, 109, 26, 
   106, 9, 97, 13, 99, 15, 119, 20, 105, 117, 98, 103, 118, 1, 126, 
   29, 97, 122, 17, 15, 114, 110, 3, 5, 125, 125, 99, 126, 119, 102, 
   30, 122, 2, 117}, 1}}
\end{lstlisting}

Tally's output is list pairs, each pair has 81-byte block and number of times it has been occurred in the file.
We see that the most frequent block is the first, it has been occurred 1739 times.
The second one has been occurred 1422 times. There are others: 1012 times, 377 times, etc.
81-byte blocks which has been occurred just once are at the end of output.

Let's try to compare these blocks? The first and the second?
Is there a function in Mathematica which compares lists/arrays? Certainly is, but for educational purposes, I'll use XOR operation for comparison.
Indeed: if bytes in two input arrays are identical, XOR result is 0. If they are non-equal, result will be non-zero.

Let's compare first block (occurred 1739 times) and the second (occurred 1422 times):

\begin{lstlisting}[style=custommath]
In[]:= BitXor[stat[[1]][[1]], stat[[2]][[1]]]
Out[]= {0, 3, 0, 0, 0, 0, 0, 0, 0, 0, 0, 0, 0, 0, 0, 0, 0, 0, 0, \
0, 0, 0, 0, 0, 0, 0, 0, 0, 0, 0, 0, 0, 0, 0, 0, 0, 0, 0, 0, 0, 0, 0, \
0, 0, 0, 0, 0, 0, 0, 0, 0, 0, 0, 0, 0, 0, 0, 0, 0, 0, 0, 0, 0, 0, 0, \
0, 0, 0, 0, 0, 0, 0, 0, 0, 0, 0, 0, 0, 0, 0, 0}
\end{lstlisting}

They are differ only in the second byte.

Let's compare second (occurred 1422 times) and third (occurred 1012 times):

\begin{lstlisting}[style=custommath]
In[]:= BitXor[stat[[2]][[1]], stat[[3]][[1]]]
Out[]= {0, 1, 0, 0, 0, 0, 0, 0, 0, 0, 0, 0, 0, 0, 0, 0, 0, 0, 0, \
0, 0, 0, 0, 0, 0, 0, 0, 0, 0, 0, 0, 0, 0, 0, 0, 0, 0, 0, 0, 0, 0, 0, \
0, 0, 0, 0, 0, 0, 0, 0, 0, 0, 0, 0, 0, 0, 0, 0, 0, 0, 0, 0, 0, 0, 0, \
0, 0, 0, 0, 0, 0, 0, 0, 0, 0, 0, 0, 0, 0, 0, 0}
\end{lstlisting}

They are also differ only in the second byte.

Anyway, let's try to use the most occurred block as a XOR key and try to decrypt four first 81-byte blocks in the file:

\begin{lstlisting}[style=custommath]
In[]:= key = stat[[1]][[1]]
Out[]= {80, 103, 2, 116, 113, 102, 118, 25, 99, 8, 19, 23, 116, \
125, 107, 25, 99, 109, 114, 102, 14, 121, 115, 31, 9, 117, 113, 111, \
5, 4, 127, 28, 122, 101, 8, 110, 14, 18, 124, 106, 16, 20, 104, 119, \
8, 109, 26, 106, 9, 97, 13, 99, 15, 119, 20, 105, 117, 98, 103, 118, \
1, 126, 29, 97, 122, 17, 15, 114, 110, 3, 5, 125, 125, 99, 126, 119, \
102, 30, 122, 2, 117}

In[]:= ToASCII[val_] := If[val == 0, " ", FromCharacterCode[val, "PrintableASCII"]]

In[]:= DecryptBlockASCII[blk_] := Map[ToASCII[#] &, BitXor[key, blk]]

In[]:= DecryptBlockASCII[blocks[[1]]]
Out[]= {" ", " ", " ", " ", " ", " ", " ", " ", " ", " ", " ", " \
", " ", " ", " ", " ", " ", " ", " ", " ", " ", " ", " ", " ", " ", " \
", " ", " ", " ", " ", " ", " ", " ", " ", " ", " ", " ", " ", " ", " \
", " ", " ", " ", " ", " ", " ", " ", " ", " ", " ", " ", " ", " ", " \
", " ", " ", " ", " ", " ", " ", " ", " ", " ", " ", " ", " ", " ", " \
", " ", " ", " ", " ", " ", " ", " ", " ", " ", " ", " ", " ", " "}

In[]:= DecryptBlockASCII[blocks[[2]]]
Out[]= {" ", "e", "H", "E", " ", "W", "E", "E", "D", " ", "O", \
"F", " ", "C", "R", "I", "M", "E", " ", "B", "E", "A", "R", "S", " ", \
"B", "I", "T", "T", "E", "R", " ", "F", "R", "U", "I", "T", "?", \
" ", " ", " ", " ", " ", " ", " ", " ", " ", " ", " ", " ", " ", " ", \
" ", " ", " ", " ", " ", " ", " ", " ", " ", " ", " ", " ", " ", " ", \
" ", " ", " ", " ", " ", " ", " ", " ", " ", " ", " ", " ", " ", " ", \
" "}

In[]:= DecryptBlockASCII[blocks[[3]]]
Out[]= {" ", "?", " ", " ", " ", " ", " ", " ", " ", " ", " \
", " ", " ", " ", " ", " ", " ", " ", " ", " ", " ", " ", " ", " ", " \
", " ", " ", " ", " ", " ", " ", " ", " ", " ", " ", " ", " ", " ", " \
", " ", " ", " ", " ", " ", " ", " ", " ", " ", " ", " ", " ", " ", " \
", " ", " ", " ", " ", " ", " ", " ", " ", " ", " ", " ", " ", " ", " \
", " ", " ", " ", " ", " ", " ", " ", " ", " ", " ", " ", " ", " ", " \
"}

In[]:= DecryptBlockASCII[blocks[[4]]]
Out[]= {" ", "f", "H", "O", " ", "K", "N", "O", "W", "S", " ", \
"W", "H", "A", "T", " ", "E", "V", "I", "L", " ", "L", "U", "R", "K", \
"S", " ", "I", "N", " ", "T", "H", "E", " ", "H", "E", "A", "R", "T", \
"S", " ", "O", "F", " ", "M", "E", "N", "?", " ", " ", " ", " ", \
" ", " ", " ", " ", " ", " ", " ", " ", " ", " ", " ", " ", " ", " ", \
" ", " ", " ", " ", " ", " ", " ", " ", " ", " ", " ", " ", " ", " ", \
" "}
\end{lstlisting}

(I've replaced unprintable characters by \q{?}.)

So we see that first and third blocks are empty (or almost empty), but second and fourth has clearly visible English language words/phrases.
It seems that our assumption about key is correct (at least partially).
This means that the most occurred 81-block in the file can be found at places of lacunas of zero bytes or something like that.

Let's try to decrypt the whole file:

\begin{lstlisting}[style=custommath]
DecryptBlock[blk_] := BitXor[key, blk]

decrypted = Map[DecryptBlock[#] &, blocks];

BinaryWrite["/home/dennis/.../tmp", Flatten[decrypted]]

Close["/home/dennis/.../tmp"]
\end{lstlisting}

\begin{figure}[H]
\centering
\myincludegraphics{ff/XOR/mask_1/mc_decrypted1.png}
\caption{Decrypted file in Midnight Commander, 1st attempt}
\end{figure}

Looks like some kind of English phrases for some game, but something wrong.
First of all, cases are inverted: phrases and some words are started with lowercase characters, while other characters are in upper case.
Also, some phrases started with wrong letters.
Take a look at the very first phrase: \q{eHE WEED OF CRIME BEARS BITTER FRUIT}.
What is \q{eHE}? Isn't \q{tHE} have to be here?
Is it possible that our decryption key has wrong byte at this place?

Let's look again at the second block in the file, at key and at decryption result:

\begin{lstlisting}[style=custommath]
In[]:= blocks[[2]]
Out[]= {80, 2, 74, 49, 113, 49, 51, 92, 39, 8, 92, 81, 116, 62, \
57, 80, 46, 40, 114, 36, 75, 56, 33, 76, 9, 55, 56, 59, 81, 65, 45, \
28, 60, 55, 93, 39, 90, 28, 124, 106, 16, 20, 104, 119, 8, 109, 26, \
106, 9, 97, 13, 99, 15, 119, 20, 105, 117, 98, 103, 118, 1, 126, 29, \
97, 122, 17, 15, 114, 110, 3, 5, 125, 125, 99, 126, 119, 102, 30, \
122, 2, 117}

In[]:= key
Out[]= {80, 103, 2, 116, 113, 102, 118, 25, 99, 8, 19, 23, 116, \
125, 107, 25, 99, 109, 114, 102, 14, 121, 115, 31, 9, 117, 113, 111, \
5, 4, 127, 28, 122, 101, 8, 110, 14, 18, 124, 106, 16, 20, 104, 119, \
8, 109, 26, 106, 9, 97, 13, 99, 15, 119, 20, 105, 117, 98, 103, 118, \
1, 126, 29, 97, 122, 17, 15, 114, 110, 3, 5, 125, 125, 99, 126, 119, \
102, 30, 122, 2, 117}

In[]:= BitXor[key, blocks[[2]]]
Out[]= {0, 101, 72, 69, 0, 87, 69, 69, 68, 0, 79, 70, 0, 67, 82, \
73, 77, 69, 0, 66, 69, 65, 82, 83, 0, 66, 73, 84, 84, 69, 82, 0, 70, \
82, 85, 73, 84, 14, 0, 0, 0, 0, 0, 0, 0, 0, 0, 0, 0, 0, 0, 0, 0, 0, \
0, 0, 0, 0, 0, 0, 0, 0, 0, 0, 0, 0, 0, 0, 0, 0, 0, 0, 0, 0, 0, 0, 0, \
0, 0, 0, 0}
\end{lstlisting}

Encrypted byte is 2, byte from key is 103, $2 \oplus 103=101$ and 101 is ASCII code for \q{e} character.
What byte a key must be equal to, so the resulting ASCII code will be 116 (for \q{t} character)?
$2 \oplus 116=118$, let's put 118 in key at the second byte...

\begin{lstlisting}[style=custommath]
key = {80, 118, 2, 116, 113, 102, 118, 25, 99, 8, 19, 23, 116, 125, 
  107, 25, 99, 109, 114, 102, 14, 121, 115, 31, 9, 117, 113, 111, 5, 
  4, 127, 28, 122, 101, 8, 110, 14, 18, 124, 106, 16, 20, 104, 119, 8,
   109, 26, 106, 9, 97, 13, 99, 15, 119, 20, 105, 117, 98, 103, 118, 
  1, 126, 29, 97, 122, 17, 15, 114, 110, 3, 5, 125, 125, 99, 126, 119,
   102, 30, 122, 2, 117}
\end{lstlisting}

... and decrypt the whole file again.

\begin{figure}[H]
\centering
\myincludegraphics{ff/XOR/mask_1/mc_decrypted2.png}
\caption{Decrypted file in Midnight Commander, 2nd attempt}
\end{figure}

Wow, now the grammar is correct, all phrases started with correct letters.
But still, case inversion is suspicious.
Why would game's developer write them in such a manner?
Maybe our key is still incorrect?

% TODO ASCII table somewhere in the book
While observing ASCII table in Wikipedia article\footnote{\url{https://en.wikipedia.org/wiki/ASCII}}
we can notice that uppercase and lowercase letter's ASCII codes are differ in just one bit
(6th bit starting at 1st, 0b100000).
This bit in decimal form is 32... 32? But 32 is ASCII code for space!

Indeed, one can switch case just by XOR-ing ASCII character code with 32.

It is possible that the empty lacunas in the file are not zero bytes, but rather spaces?
Let's modify XOR key one more time (I'll XOR each byte of key by 32):

\begin{lstlisting}[style=custommath]
(* "32" is scalar and "key" is vector, but that's OK *)

In[]:= key3 = BitXor[32, key]
Out[]= {112, 86, 34, 84, 81, 70, 86, 57, 67, 40, 51, 55, 84, 93, 75, \
57, 67, 77, 82, 70, 46, 89, 83, 63, 41, 85, 81, 79, 37, 36, 95, 60, \
90, 69, 40, 78, 46, 50, 92, 74, 48, 52, 72, 87, 40, 77, 58, 74, 41, \
65, 45, 67, 47, 87, 52, 73, 85, 66, 71, 86, 33, 94, 61, 65, 90, 49, \
47, 82, 78, 35, 37, 93, 93, 67, 94, 87, 70, 62, 90, 34, 85}

In[]:= DecryptBlock[blk_] := BitXor[key3, blk]
\end{lstlisting}

Let's decrypt the input file again:

\begin{figure}[H]
\centering
\myincludegraphics{ff/XOR/mask_1/mc_decrypted.png}
\caption{Decrypted file in Midnight Commander, final attempt}
\end{figure}

(Decrypted file is available \href{http://yurichev.com/blog/XOR_mask_1/files/decrypted.dat}{here}.)
This is undoubtedly a correct source file.
Oh, and we see numbers at the start of each block. It has to be a source of our erroneous XOR key.
As it seems, the most occurred 81-byte block in the file is a block filled with spaces and containing \q{1} character at the place of second byte.
Indeed, somehow, many blocks here are interleaved with this one.
Maybe it's some kind of padding for short phrases/messages?
Other highly occurred 81-byte blocks are also space-filled blocks, but with different digits, hence, they are differ only at the second byte.

That's all! Now we can write utility to encrypt the file back, and maybe modify it before.

Mathematica notebook file is downloadable \href{http://yurichev.com/blog/XOR_mask_1/files/XOR_mask_1.nb}{here}.

Summary: XOR encryption like that is not robust at all. It has been intended by game's developer(s), probably, just to prevent gamer(s) to peek into internals of game, nothing else.
Still, encryption like that is extremely popular due to its simplicity and many reverse engineers are usually familiar with it.

}\RU{\subsection{Простое шифрование используя XOR-маску}
\label{XOR_mask_1}

Я нашел одну старую игру в стиле interactive fiction в архиве \IT{if-archive}\footnote{\url{http://www.ifarchive.org/}}:

\begin{lstlisting}
The New Castle v3.5 - Text/Adventure Game
in the style of the original Infocom (tm)
type games, Zork, Collosal Cave (Adventure),
etc.  Can you solve the mystery of the
abandoned castle?
Shareware from Software Customization.
Software Customization [ASP] Version 3.5 Feb. 2000
\end{lstlisting}

Можно скачать здесь: \url{https://github.com/dennis714/RE-for-beginners/blob/master/ff/XOR/mask_1/files/newcastle.tgz}.

Там внутри есть файл (с названием \IT{castle.dbf}), который явно зашифрован, но не настоящим криптоалгоритмом,
и оне сжат, это что-то куда проще.
Я бы даже не стал измерять уровень энтропии (\myref{entropy}) этого файла, потому что я итак уверен, что он низкий.
Вот как он выглядит в Midnight Commander:

\begin{figure}[H]
\centering
\myincludegraphics{ff/XOR/mask_1/mc_encrypted.png}
\caption{Зашифрованный файл в Midnight Commander}
\end{figure}

Зашифрованный файл можно скачать здесь:
\url{https://github.com/dennis714/RE-for-beginners/blob/master/ff/XOR/mask_1/files/castle.dbf.bz2}.

Можно ли расшифровать его без доступа к программе, используя просто этот файл?

Тут явно просматривается повторяющаяся строка. 
Если использовалось простое шифрование с XOR-маской, такие повторяющиеся строки это явное свидетельство,
потому что, вероятно, тут были длинные лакуны с нулевыми байтами, которые, в свою очередь, присутствуют
во мноигих исполняемых файлах, и в остальных бинарных файлах.

\myindex{UNIX!xxd}
Вот дам начала этого файла используя утилиту \IT{xxd} из UNIX:

\lstinputlisting{ff/XOR/mask_1/xxd_result.txt}

Давайте держаться за повторяющуюся строку \TT{iubgv}.
Глядя на этот дамп, мы можем легко увидеть, что период повторений этой строки это 0x51 или 81.
Вероятно, 81 это длина блока?
Длина файла 1658961, и она может быть поделена на 81 без остатка (и тогда там 20481 блоков).

Теперь я буду использовать Mathematica для анализа, есть ли тут повторяющиеся 81-байтные блоки в файле?
Я разделю входной файл на 81-байтные блоки и затем использую ф-цию
\IT{Tally[]}\footnote{\url{https://reference.wolfram.com/language/ref/Tally.html}}
которая просто считает, сколько раз каждый элемент встретился во входном списке.
Вывод Tally не отсортирован, так что я также добавлю ф-цию \IT{Sort[]} для сортировки его по кол-ву вхождений
в нисходящем порядке.

\begin{lstlisting}[style=custommath]
input = BinaryReadList["/home/dennis/.../castle.dbf"];

blocks = Partition[input, 81];

stat = Sort[Tally[blocks], #1[[2]] > #2[[2]] &]
\end{lstlisting}

И вот вывод:

\begin{lstlisting}[style=custommath]
{{{80, 103, 2, 116, 113, 102, 118, 25, 99, 8, 19, 23, 116, 125, 107, 
   25, 99, 109, 114, 102, 14, 121, 115, 31, 9, 117, 113, 111, 5, 4, 
   127, 28, 122, 101, 8, 110, 14, 18, 124, 106, 16, 20, 104, 119, 8, 
   109, 26, 106, 9, 97, 13, 99, 15, 119, 20, 105, 117, 98, 103, 118, 
   1, 126, 29, 97, 122, 17, 15, 114, 110, 3, 5, 125, 125, 99, 126, 
   119, 102, 30, 122, 2, 117}, 1739}, 
{{80, 100, 2, 116, 113, 102, 118, 25, 99, 8, 19, 23, 116, 
   125, 107, 25, 99, 109, 114, 102, 14, 121, 115, 31, 9, 117, 113, 
   111, 5, 4, 127, 28, 122, 101, 8, 110, 14, 18, 124, 106, 16, 20, 
   104, 119, 8, 109, 26, 106, 9, 97, 13, 99, 15, 119, 20, 105, 117, 
   98, 103, 118, 1, 126, 29, 97, 122, 17, 15, 114, 110, 3, 5, 125, 
   125, 99, 126, 119, 102, 30, 122, 2, 117}, 1422}, 
{{80, 101, 2, 116, 113, 102, 118, 25, 99, 8, 19, 23, 116, 
   125, 107, 25, 99, 109, 114, 102, 14, 121, 115, 31, 9, 117, 113, 
   111, 5, 4, 127, 28, 122, 101, 8, 110, 14, 18, 124, 106, 16, 20, 
   104, 119, 8, 109, 26, 106, 9, 97, 13, 99, 15, 119, 20, 105, 117, 
   98, 103, 118, 1, 126, 29, 97, 122, 17, 15, 114, 110, 3, 5, 125, 
   125, 99, 126, 119, 102, 30, 122, 2, 117}, 1012},
{{80, 120, 2, 116, 113, 102, 118, 25, 99, 8, 19, 23, 116, 
   125, 107, 25, 99, 109, 114, 102, 14, 121, 115, 31, 9, 117, 113, 
   111, 5, 4, 127, 28, 122, 101, 8, 110, 14, 18, 124, 106, 16, 20, 
   104, 119, 8, 109, 26, 106, 9, 97, 13, 99, 15, 119, 20, 105, 117, 
   98, 103, 118, 1, 126, 29, 97, 122, 17, 15, 114, 110, 3, 5, 125, 
   125, 99, 126, 119, 102, 30, 122, 2, 117}, 377},

...

{{80, 2, 74, 49, 113, 21, 62, 88, 39, 71, 68, 23, 63, 51, 36, 78, 48, 
   108, 114, 102, 14, 121, 115, 31, 9, 117, 113, 111, 5, 4, 127, 28, 
   122, 101, 8, 110, 14, 18, 124, 106, 16, 20, 104, 119, 8, 109, 26, 
   106, 9, 97, 13, 99, 15, 119, 20, 105, 117, 98, 103, 118, 1, 126, 
   29, 97, 122, 17, 15, 114, 110, 3, 5, 125, 125, 99, 126, 119, 102, 
   30, 122, 2, 117}, 1},
{{80, 1, 74, 59, 113, 45, 56, 86, 52, 91, 19, 64, 60, 60, 63, 
   25, 38, 59, 59, 42, 14, 53, 38, 77, 66, 38, 113, 38, 75, 4, 43, 84,
    63, 101, 64, 43, 79, 64, 40, 57, 16, 91, 46, 119, 69, 40, 84, 117,
    9, 97, 13, 99, 15, 119, 20, 105, 117, 98, 103, 118, 1, 126, 29, 
   97, 122, 17, 15, 114, 110, 3, 5, 125, 125, 99, 126, 119, 102, 30, 
   122, 2, 117}, 1},
{{80, 2, 74, 49, 113, 49, 51, 92, 39, 8, 92, 81, 116, 62, 57, 
   80, 46, 40, 114, 36, 75, 56, 33, 76, 9, 55, 56, 59, 81, 65, 45, 28,
    60, 55, 93, 39, 90, 28, 124, 106, 16, 20, 104, 119, 8, 109, 26, 
   106, 9, 97, 13, 99, 15, 119, 20, 105, 117, 98, 103, 118, 1, 126, 
   29, 97, 122, 17, 15, 114, 110, 3, 5, 125, 125, 99, 126, 119, 102, 
   30, 122, 2, 117}, 1}}
\end{lstlisting}

Вывод Tally это список пар, каждая пара это 81-байтный блок и количество раз, сколько он встретился в файле.
Мы видим, что наиболее частно встречающийся блок это первый, он встретился 1739 раз.
Второй встретился 1422 раза. Есть и другие: 1012 раза, 377 раз, итд.
81-байтные блоки, встреченные лишь один раз, находятся в конце вывода.

Попробуем сравнить эти блоки. Первый и второй.
Есть ли в Mathematica ф-ция для сравнения списков/массивов?
Наверняка есть, но в педагогических целях, я буду использоват операцию XOR для сравнения.
Действительно: если байты во входных массивах равны друг другу, результат операции XOR это 0.
Если не равны, результат будет ненулевой.

Сравним первый блок (встречается 1739 раз) и второй (встречается 1422 раз):

\begin{lstlisting}[style=custommath]
In[]:= BitXor[stat[[1]][[1]], stat[[2]][[1]]]
Out[]= {0, 3, 0, 0, 0, 0, 0, 0, 0, 0, 0, 0, 0, 0, 0, 0, 0, 0, 0, \
0, 0, 0, 0, 0, 0, 0, 0, 0, 0, 0, 0, 0, 0, 0, 0, 0, 0, 0, 0, 0, 0, 0, \
0, 0, 0, 0, 0, 0, 0, 0, 0, 0, 0, 0, 0, 0, 0, 0, 0, 0, 0, 0, 0, 0, 0, \
0, 0, 0, 0, 0, 0, 0, 0, 0, 0, 0, 0, 0, 0, 0, 0}
\end{lstlisting}

Они отличаются только вторым байтом.

Сравним второй блок (встречается 1422 раза) и третий (встречается 1012 раз):

\begin{lstlisting}[style=custommath]
In[]:= BitXor[stat[[2]][[1]], stat[[3]][[1]]]
Out[]= {0, 1, 0, 0, 0, 0, 0, 0, 0, 0, 0, 0, 0, 0, 0, 0, 0, 0, 0, \
0, 0, 0, 0, 0, 0, 0, 0, 0, 0, 0, 0, 0, 0, 0, 0, 0, 0, 0, 0, 0, 0, 0, \
0, 0, 0, 0, 0, 0, 0, 0, 0, 0, 0, 0, 0, 0, 0, 0, 0, 0, 0, 0, 0, 0, 0, \
0, 0, 0, 0, 0, 0, 0, 0, 0, 0, 0, 0, 0, 0, 0, 0}
\end{lstlisting}

Они тоже отличаются только вторым байтом.

Так или иначе, попробуем использовать самый встречающийся блок как XOR-ключ и попробуем расшифровать первые 4 81-байтных
блока в файле:

\begin{lstlisting}[style=custommath]
In[]:= key = stat[[1]][[1]]
Out[]= {80, 103, 2, 116, 113, 102, 118, 25, 99, 8, 19, 23, 116, \
125, 107, 25, 99, 109, 114, 102, 14, 121, 115, 31, 9, 117, 113, 111, \
5, 4, 127, 28, 122, 101, 8, 110, 14, 18, 124, 106, 16, 20, 104, 119, \
8, 109, 26, 106, 9, 97, 13, 99, 15, 119, 20, 105, 117, 98, 103, 118, \
1, 126, 29, 97, 122, 17, 15, 114, 110, 3, 5, 125, 125, 99, 126, 119, \
102, 30, 122, 2, 117}

In[]:= ToASCII[val_] := If[val == 0, " ", FromCharacterCode[val, "PrintableASCII"]]

In[]:= DecryptBlockASCII[blk_] := Map[ToASCII[#] &, BitXor[key, blk]]

In[]:= DecryptBlockASCII[blocks[[1]]]
Out[]= {" ", " ", " ", " ", " ", " ", " ", " ", " ", " ", " ", " \
", " ", " ", " ", " ", " ", " ", " ", " ", " ", " ", " ", " ", " ", " \
", " ", " ", " ", " ", " ", " ", " ", " ", " ", " ", " ", " ", " ", " \
", " ", " ", " ", " ", " ", " ", " ", " ", " ", " ", " ", " ", " ", " \
", " ", " ", " ", " ", " ", " ", " ", " ", " ", " ", " ", " ", " ", " \
", " ", " ", " ", " ", " ", " ", " ", " ", " ", " ", " ", " ", " "}

In[]:= DecryptBlockASCII[blocks[[2]]]
Out[]= {" ", "e", "H", "E", " ", "W", "E", "E", "D", " ", "O", \
"F", " ", "C", "R", "I", "M", "E", " ", "B", "E", "A", "R", "S", " ", \
"B", "I", "T", "T", "E", "R", " ", "F", "R", "U", "I", "T", "?", \
" ", " ", " ", " ", " ", " ", " ", " ", " ", " ", " ", " ", " ", " ", \
" ", " ", " ", " ", " ", " ", " ", " ", " ", " ", " ", " ", " ", " ", \
" ", " ", " ", " ", " ", " ", " ", " ", " ", " ", " ", " ", " ", " ", \
" "}

In[]:= DecryptBlockASCII[blocks[[3]]]
Out[]= {" ", "?", " ", " ", " ", " ", " ", " ", " ", " ", " \
", " ", " ", " ", " ", " ", " ", " ", " ", " ", " ", " ", " ", " ", " \
", " ", " ", " ", " ", " ", " ", " ", " ", " ", " ", " ", " ", " ", " \
", " ", " ", " ", " ", " ", " ", " ", " ", " ", " ", " ", " ", " ", " \
", " ", " ", " ", " ", " ", " ", " ", " ", " ", " ", " ", " ", " ", " \
", " ", " ", " ", " ", " ", " ", " ", " ", " ", " ", " ", " ", " ", " \
"}

In[]:= DecryptBlockASCII[blocks[[4]]]
Out[]= {" ", "f", "H", "O", " ", "K", "N", "O", "W", "S", " ", \
"W", "H", "A", "T", " ", "E", "V", "I", "L", " ", "L", "U", "R", "K", \
"S", " ", "I", "N", " ", "T", "H", "E", " ", "H", "E", "A", "R", "T", \
"S", " ", "O", "F", " ", "M", "E", "N", "?", " ", " ", " ", " ", \
" ", " ", " ", " ", " ", " ", " ", " ", " ", " ", " ", " ", " ", " ", \
" ", " ", " ", " ", " ", " ", " ", " ", " ", " ", " ", " ", " ", " ", \
" "}
\end{lstlisting}

(Я заменил непечатаемые символы на \q{?}.)

Мы видим что первый и третий блоки пустые (или почти пустые),
но второй и четвертый имеют ясно различимые английские слова/фразы.
Похоже что наше предположение насчет ключа верно (как минимум частично).
Это означает, что самый встречающийся 81-байтный блок в файле находится в местах лакун с нулевыми байтами
или что-то в этом роде.

Попробуем расшифровать весь файл:

\begin{lstlisting}[style=custommath]
DecryptBlock[blk_] := BitXor[key, blk]

decrypted = Map[DecryptBlock[#] &, blocks];

BinaryWrite["/home/dennis/.../tmp", Flatten[decrypted]]

Close["/home/dennis/.../tmp"]
\end{lstlisting}

\begin{figure}[H]
\centering
\myincludegraphics{ff/XOR/mask_1/mc_decrypted1.png}
\caption{Расшифрованный файл в Midnight Commander, первая попытка}
\end{figure}

Выглядит как английские фразы для какой-то игры, но что-то не так.
Прежде всего, регистр инвертирован: фразы и некоторые слова начинаются со строчных букв,
в то время как остальные буквы заглавные.
Также, некоторые фразы начинаются с не тех букв.
Посмотрите на самую первую фразу: \q{eHE WEED OF CRIME BEARS BITTER FRUIT}.
Что такое \q{eHE}? Разве не \q{tHE} тут должно быть?
Возможно ли что наш ключ для дешифрования имеет неверный байт в этом месте?

Посмотрим снова на второй блок в файле, на ключ и на результат дешифрования:

\begin{lstlisting}[style=custommath]
In[]:= blocks[[2]]
Out[]= {80, 2, 74, 49, 113, 49, 51, 92, 39, 8, 92, 81, 116, 62, \
57, 80, 46, 40, 114, 36, 75, 56, 33, 76, 9, 55, 56, 59, 81, 65, 45, \
28, 60, 55, 93, 39, 90, 28, 124, 106, 16, 20, 104, 119, 8, 109, 26, \
106, 9, 97, 13, 99, 15, 119, 20, 105, 117, 98, 103, 118, 1, 126, 29, \
97, 122, 17, 15, 114, 110, 3, 5, 125, 125, 99, 126, 119, 102, 30, \
122, 2, 117}

In[]:= key
Out[]= {80, 103, 2, 116, 113, 102, 118, 25, 99, 8, 19, 23, 116, \
125, 107, 25, 99, 109, 114, 102, 14, 121, 115, 31, 9, 117, 113, 111, \
5, 4, 127, 28, 122, 101, 8, 110, 14, 18, 124, 106, 16, 20, 104, 119, \
8, 109, 26, 106, 9, 97, 13, 99, 15, 119, 20, 105, 117, 98, 103, 118, \
1, 126, 29, 97, 122, 17, 15, 114, 110, 3, 5, 125, 125, 99, 126, 119, \
102, 30, 122, 2, 117}

In[]:= BitXor[key, blocks[[2]]]
Out[]= {0, 101, 72, 69, 0, 87, 69, 69, 68, 0, 79, 70, 0, 67, 82, \
73, 77, 69, 0, 66, 69, 65, 82, 83, 0, 66, 73, 84, 84, 69, 82, 0, 70, \
82, 85, 73, 84, 14, 0, 0, 0, 0, 0, 0, 0, 0, 0, 0, 0, 0, 0, 0, 0, 0, \
0, 0, 0, 0, 0, 0, 0, 0, 0, 0, 0, 0, 0, 0, 0, 0, 0, 0, 0, 0, 0, 0, 0, \
0, 0, 0, 0}
\end{lstlisting}

Зашифрованный байт это 2, байт из ключа это 103, $2 \oplus 103=101$ и 101 это ASCII-код символа \q{e}.
Чему должен равнятся этот байт ключа, чтобы ASCII-код был 116 (для символа  \q{t})?
$2 \oplus 116=118$, присвоим 118 второму байту в ключе \dots

\begin{lstlisting}[style=custommath]
key = {80, 118, 2, 116, 113, 102, 118, 25, 99, 8, 19, 23, 116, 125, 
  107, 25, 99, 109, 114, 102, 14, 121, 115, 31, 9, 117, 113, 111, 5, 
  4, 127, 28, 122, 101, 8, 110, 14, 18, 124, 106, 16, 20, 104, 119, 8,
   109, 26, 106, 9, 97, 13, 99, 15, 119, 20, 105, 117, 98, 103, 118, 
  1, 126, 29, 97, 122, 17, 15, 114, 110, 3, 5, 125, 125, 99, 126, 119,
   102, 30, 122, 2, 117}
\end{lstlisting}

\dots и снова дешифруем весь файл.

\begin{figure}[H]
\centering
\myincludegraphics{ff/XOR/mask_1/mc_decrypted2.png}
\caption{Дешифрованный файл в Midnight Commander, вторая попытка}
\end{figure}

Ух ты, теперь грамматика корректна, и все фразы начинаются с корректных букв.
Но все таки, регистр подозрителен.
С чего бы разработчику игры записывать их в такой манере?
Может быть наш ключ все еще неправилен?

% TODO ASCII table somewhere in the book
Изучая таблицу ASCII мы можем заметить что ASCII-коды для букв в верхнем и нижнем регистре отличаются только на один бит
(6-й бит, если считать с первого, 0b100000):

\begin{figure}[H]
\centering
\includegraphics[width=0.7\textwidth]{ascii.png}
\caption{7-битная таблица \ac{ASCII} в Emacs}
\end{figure}

В десятичном виде этот бит это 32 \dots 32?
Но 32 это ASCII-код пробела!

Действительно, можно менять регистр просто применяя XOR к ASCII-коду, с 32 (больше об этом: \myref{toupper_bit}).

Возможно ли, что пустые лакуны в файле это не нулевые байты, а скорее содержащие пробелы?
Еще раз модифицируем наш XOR-ключ (я про-XOR-ю каждый байт ключа с 32):

\begin{lstlisting}[style=custommath]
(* "32" это скаляр, и "key" это вектор, но это OK *)

In[]:= key3 = BitXor[32, key]
Out[]= {112, 86, 34, 84, 81, 70, 86, 57, 67, 40, 51, 55, 84, 93, 75, \
57, 67, 77, 82, 70, 46, 89, 83, 63, 41, 85, 81, 79, 37, 36, 95, 60, \
90, 69, 40, 78, 46, 50, 92, 74, 48, 52, 72, 87, 40, 77, 58, 74, 41, \
65, 45, 67, 47, 87, 52, 73, 85, 66, 71, 86, 33, 94, 61, 65, 90, 49, \
47, 82, 78, 35, 37, 93, 93, 67, 94, 87, 70, 62, 90, 34, 85}

In[]:= DecryptBlock[blk_] := BitXor[key3, blk]
\end{lstlisting}

И снова дешифруем входной файл:

\begin{figure}[H]
\centering
\myincludegraphics{ff/XOR/mask_1/mc_decrypted.png}
\caption{Дешифрованный файл в Midnight Commander, последняя попытка}
\end{figure}

(Расшифрованный файл доступен здесь:
\url{https://github.com/dennis714/RE-for-beginners/blob/master/ff/XOR/mask_1/files/decrypted.dat.bz2}.)

Несомненно, это корректный исходный файл.
Да, и мы видим числа в начале каждого блока. Должно быть это и есть источник некорректного XOR-ключа.
Как выходит, самый встречающийся 81-байтный блок в файле это блок заполненный пробелами и содержащий символ \q{1} на месте
второго байта.
Действительно, как-то так получилось что многие блоки здесь перемежаются с этим блоком.
Может быть это что-то вроде выравнивания (padding) для коротких фраз/сообщений?
Другой часто встречающийся 81-байтный блок также заполнен пробелами, но с другой цифрой, следовательно,
они отличаются только вторым байтом.

Вот и всё! Теперь мы можем написать утилиту для зашифрования файла назад, и, может быть, модифицировать его перед этим

Файл для Mathematica можно скачать здесь:
\url{https://github.com/dennis714/RE-for-beginners/blob/master/ff/XOR/mask_1/files/XOR_mask_1.nb}.

Итог: XOR-шифрование не надежно вообще. Вероятно, разработчик игры хотел просто скрыть внутренности игры от игрока,
ничего более серьезного.
Все же, шифрование вроде этого крайне популярно вследствии его простоты, так что многие реверс инженеры обычно хорошо
с этим знакомы.

}%
\FR{% TODO translate
\subsection{Chiffrement simple utilisant un masque XOR}
\label{XOR_mask_1}

J'ai trouvé un vieux jeu de fiction interactif en plongeant profondément dans \IT{if-archive}\footnote{\url{http://www.ifarchive.org/}}:

\begin{lstlisting}
The New Castle v3.5 - Text/Adventure Game
in the style of the original Infocom (tm)
type games, Zork, Collosal Cave (Adventure),
etc.  Can you solve the mystery of the
abandoned castle?
Shareware from Software Customization.
Software Customization [ASP] Version 3.5 Feb. 2000
\end{lstlisting}

Il est téléchargeable
\url{https://github.com/DennisYurichev/RE-for-beginners/blob/master/ff/XOR/mask_1/files/newcastle.tgz}{ici}.

Il y a un fichier à l'intérieur (appelé \IT{castle.dbf}) qui est visiblement chiffré,
mais pas avec un vrai algorithme de crypto, qui n'est pas non plus compressé, il
s'agit plutôt de quelque chose de plus simple.
Je ne vais même pas mesurer le niveau d'entropie (\myref{entropy}) du fichier, car
je suis sûr qu'il est bas.
Voici à quoi il ressemble dans Midnight Commander:

\begin{figure}[H]
\centering
\myincludegraphics{ff/XOR/mask_1/mc_encrypted.png}
\caption{Fichier chiffré dans Midnight Commander}
\end{figure}

Le fichier chiffré peut être téléchargé ici:
\url{https://github.com/DennisYurichev/RE-for-beginners/blob/master/ff/XOR/mask_1/files/castle.dbf.bz2}.

Sera-t-il possible de le décrypter sans accéder au programme, en utilisant juste ce
fichier?

Il y a clairement un pattern visible de chaînes répétées.
Si un simple chiffrement avec un masque XOR a été appliqué, une répétition de telles
chaînes en est une signature notable, car, il y avait probablement de longues
suites (lacunes\footnote{Comme dans \url{https://en.wikipedia.org/wiki/Lacuna_(manuscripts)}})
d'octets à zéro, qui, à tour de rôle, sont présentes dans de nombreux
fichiers exécutables, tout comme dans des fichiers de données binaires.

\myindex{UNIX!xxd}
Ici, je vais afficher le début du fichier en utilisant l'utilitaire UNIX \IT{xxd}:

\lstinputlisting{ff/XOR/mask_1/xxd_result.txt}

Concentrons-nous sur la chaîne visible \TT{iubgv} se répétant.
En regardant ce dump, nous voyons clairement que la période de l'occurrence de la chaîne
est 0x51 ou 81.
La taille du fichier est 1658961, et est divisible par 81 (et il y a donc 20481 blocs).

Maintenant, je vais utiliser Mathematica pour l'analyse, y a-t-il des blocs de 81
octets se répètant dans le fichier?
Je vais séparer le fichier d'entrée en blocs de 81 octets et ensuite utiliser la fonction
\IT{Tally[]}\footnote{\url{https://reference.wolfram.com/language/ref/Tally.html}}
qui compte simplement combien de fois un élément était présent dans la liste en entrée.
La sortie de Tally n'est pas triée, donc je vais ajouter la fonction \IT{Sort[]}
pour trier le nombre d'occurrences par ordre décroissant.

\begin{lstlisting}[style=custommath]
input = BinaryReadList["/home/dennis/.../castle.dbf"];

blocks = Partition[input, 81];

stat = Sort[Tally[blocks], #1[[2]] > #2[[2]] &]
\end{lstlisting}

Et voici la sortie:

\begin{lstlisting}[style=custommath]
{{{80, 103, 2, 116, 113, 102, 118, 25, 99, 8, 19, 23, 116, 125, 107, 
   25, 99, 109, 114, 102, 14, 121, 115, 31, 9, 117, 113, 111, 5, 4, 
   127, 28, 122, 101, 8, 110, 14, 18, 124, 106, 16, 20, 104, 119, 8, 
   109, 26, 106, 9, 97, 13, 99, 15, 119, 20, 105, 117, 98, 103, 118, 
   1, 126, 29, 97, 122, 17, 15, 114, 110, 3, 5, 125, 125, 99, 126, 
   119, 102, 30, 122, 2, 117}, 1739}, 
{{80, 100, 2, 116, 113, 102, 118, 25, 99, 8, 19, 23, 116, 
   125, 107, 25, 99, 109, 114, 102, 14, 121, 115, 31, 9, 117, 113, 
   111, 5, 4, 127, 28, 122, 101, 8, 110, 14, 18, 124, 106, 16, 20, 
   104, 119, 8, 109, 26, 106, 9, 97, 13, 99, 15, 119, 20, 105, 117, 
   98, 103, 118, 1, 126, 29, 97, 122, 17, 15, 114, 110, 3, 5, 125, 
   125, 99, 126, 119, 102, 30, 122, 2, 117}, 1422}, 
{{80, 101, 2, 116, 113, 102, 118, 25, 99, 8, 19, 23, 116, 
   125, 107, 25, 99, 109, 114, 102, 14, 121, 115, 31, 9, 117, 113, 
   111, 5, 4, 127, 28, 122, 101, 8, 110, 14, 18, 124, 106, 16, 20, 
   104, 119, 8, 109, 26, 106, 9, 97, 13, 99, 15, 119, 20, 105, 117, 
   98, 103, 118, 1, 126, 29, 97, 122, 17, 15, 114, 110, 3, 5, 125, 
   125, 99, 126, 119, 102, 30, 122, 2, 117}, 1012},
{{80, 120, 2, 116, 113, 102, 118, 25, 99, 8, 19, 23, 116, 
   125, 107, 25, 99, 109, 114, 102, 14, 121, 115, 31, 9, 117, 113, 
   111, 5, 4, 127, 28, 122, 101, 8, 110, 14, 18, 124, 106, 16, 20, 
   104, 119, 8, 109, 26, 106, 9, 97, 13, 99, 15, 119, 20, 105, 117, 
   98, 103, 118, 1, 126, 29, 97, 122, 17, 15, 114, 110, 3, 5, 125, 
   125, 99, 126, 119, 102, 30, 122, 2, 117}, 377},

...

{{80, 2, 74, 49, 113, 21, 62, 88, 39, 71, 68, 23, 63, 51, 36, 78, 48, 
   108, 114, 102, 14, 121, 115, 31, 9, 117, 113, 111, 5, 4, 127, 28, 
   122, 101, 8, 110, 14, 18, 124, 106, 16, 20, 104, 119, 8, 109, 26, 
   106, 9, 97, 13, 99, 15, 119, 20, 105, 117, 98, 103, 118, 1, 126, 
   29, 97, 122, 17, 15, 114, 110, 3, 5, 125, 125, 99, 126, 119, 102, 
   30, 122, 2, 117}, 1},
{{80, 1, 74, 59, 113, 45, 56, 86, 52, 91, 19, 64, 60, 60, 63, 
   25, 38, 59, 59, 42, 14, 53, 38, 77, 66, 38, 113, 38, 75, 4, 43, 84,
    63, 101, 64, 43, 79, 64, 40, 57, 16, 91, 46, 119, 69, 40, 84, 117,
    9, 97, 13, 99, 15, 119, 20, 105, 117, 98, 103, 118, 1, 126, 29, 
   97, 122, 17, 15, 114, 110, 3, 5, 125, 125, 99, 126, 119, 102, 30, 
   122, 2, 117}, 1},
{{80, 2, 74, 49, 113, 49, 51, 92, 39, 8, 92, 81, 116, 62, 57, 
   80, 46, 40, 114, 36, 75, 56, 33, 76, 9, 55, 56, 59, 81, 65, 45, 28,
    60, 55, 93, 39, 90, 28, 124, 106, 16, 20, 104, 119, 8, 109, 26, 
   106, 9, 97, 13, 99, 15, 119, 20, 105, 117, 98, 103, 118, 1, 126, 
   29, 97, 122, 17, 15, 114, 110, 3, 5, 125, 125, 99, 126, 119, 102, 
   30, 122, 2, 117}, 1}}
\end{lstlisting}

La sortie de Tally est une liste de paires, chaque paire a un bloc de 81 octets et
le nombre de fois qu'il apparaît dans le fichier.
Nous voyons que le bloc le plus fréquent est le premier, il est apparu 1739 fois.
Le second apparaît 1422 fois. Puis les autres: 1012 fois, 377 fois, etc.
Les blocs de 81 octets qui ne sont apparus qu'une fois sont à la fin de la sortie.

Essayons de comparer ces blocs. Le premier et le second.
Y a-t-il une fonction dans Mathematica qui compare les listes/tableaux?
Certainement qu'il y en a une, mais dans un but didactique, je vais utiliser
l'opération XOR pour la comparaison.
En effet: si les octets dans deux tableaux d'entrée sont identiques, le résultat
du XOR est 0. Si ils sont différents, le résultat sera différent de zéro.

Comparons le premier bloc (qui apparaît 1739 fois) et le second (qui apparaît 1422 fois):

\begin{lstlisting}[style=custommath]
In[]:= BitXor[stat[[1]][[1]], stat[[2]][[1]]]
Out[]= {0, 3, 0, 0, 0, 0, 0, 0, 0, 0, 0, 0, 0, 0, 0, 0, 0, 0, 0, \
0, 0, 0, 0, 0, 0, 0, 0, 0, 0, 0, 0, 0, 0, 0, 0, 0, 0, 0, 0, 0, 0, 0, \
0, 0, 0, 0, 0, 0, 0, 0, 0, 0, 0, 0, 0, 0, 0, 0, 0, 0, 0, 0, 0, 0, 0, \
0, 0, 0, 0, 0, 0, 0, 0, 0, 0, 0, 0, 0, 0, 0, 0}
\end{lstlisting}

Ils ne diffèrent que par le second octet.

Comparons le second bloc (qui apparaît 1422 fois) et le troisième (qui apparaît 1012 fois):

\begin{lstlisting}[style=custommath]
In[]:= BitXor[stat[[2]][[1]], stat[[3]][[1]]]
Out[]= {0, 1, 0, 0, 0, 0, 0, 0, 0, 0, 0, 0, 0, 0, 0, 0, 0, 0, 0, \
0, 0, 0, 0, 0, 0, 0, 0, 0, 0, 0, 0, 0, 0, 0, 0, 0, 0, 0, 0, 0, 0, 0, \
0, 0, 0, 0, 0, 0, 0, 0, 0, 0, 0, 0, 0, 0, 0, 0, 0, 0, 0, 0, 0, 0, 0, \
0, 0, 0, 0, 0, 0, 0, 0, 0, 0, 0, 0, 0, 0, 0, 0}
\end{lstlisting}

Ils ne diffèrent également que par le second octet.

Quoiqu'il en soit, essayons d'utiliser le bloc qui apparaît le plus comme une clef
XOR et essayons de déchiffrer les quatre premiers blocs de 81 octets dans le fichier:

\begin{lstlisting}[style=custommath]
In[]:= key = stat[[1]][[1]]
Out[]= {80, 103, 2, 116, 113, 102, 118, 25, 99, 8, 19, 23, 116, \
125, 107, 25, 99, 109, 114, 102, 14, 121, 115, 31, 9, 117, 113, 111, \
5, 4, 127, 28, 122, 101, 8, 110, 14, 18, 124, 106, 16, 20, 104, 119, \
8, 109, 26, 106, 9, 97, 13, 99, 15, 119, 20, 105, 117, 98, 103, 118, \
1, 126, 29, 97, 122, 17, 15, 114, 110, 3, 5, 125, 125, 99, 126, 119, \
102, 30, 122, 2, 117}

In[]:= ToASCII[val_] := If[val == 0, " ", FromCharacterCode[val, "PrintableASCII"]]

In[]:= DecryptBlockASCII[blk_] := Map[ToASCII[#] &, BitXor[key, blk]]

In[]:= DecryptBlockASCII[blocks[[1]]]
Out[]= {" ", " ", " ", " ", " ", " ", " ", " ", " ", " ", " ", " \
", " ", " ", " ", " ", " ", " ", " ", " ", " ", " ", " ", " ", " ", " \
", " ", " ", " ", " ", " ", " ", " ", " ", " ", " ", " ", " ", " ", " \
", " ", " ", " ", " ", " ", " ", " ", " ", " ", " ", " ", " ", " ", " \
", " ", " ", " ", " ", " ", " ", " ", " ", " ", " ", " ", " ", " ", " \
", " ", " ", " ", " ", " ", " ", " ", " ", " ", " ", " ", " ", " "}

In[]:= DecryptBlockASCII[blocks[[2]]]
Out[]= {" ", "e", "H", "E", " ", "W", "E", "E", "D", " ", "O", \
"F", " ", "C", "R", "I", "M", "E", " ", "B", "E", "A", "R", "S", " ", \
"B", "I", "T", "T", "E", "R", " ", "F", "R", "U", "I", "T", "?", \
" ", " ", " ", " ", " ", " ", " ", " ", " ", " ", " ", " ", " ", " ", \
" ", " ", " ", " ", " ", " ", " ", " ", " ", " ", " ", " ", " ", " ", \
" ", " ", " ", " ", " ", " ", " ", " ", " ", " ", " ", " ", " ", " ", \
" "}

In[]:= DecryptBlockASCII[blocks[[3]]]
Out[]= {" ", "?", " ", " ", " ", " ", " ", " ", " ", " ", " \
", " ", " ", " ", " ", " ", " ", " ", " ", " ", " ", " ", " ", " ", " \
", " ", " ", " ", " ", " ", " ", " ", " ", " ", " ", " ", " ", " ", " \
", " ", " ", " ", " ", " ", " ", " ", " ", " ", " ", " ", " ", " ", " \
", " ", " ", " ", " ", " ", " ", " ", " ", " ", " ", " ", " ", " ", " \
", " ", " ", " ", " ", " ", " ", " ", " ", " ", " ", " ", " ", " ", " \
"}

In[]:= DecryptBlockASCII[blocks[[4]]]
Out[]= {" ", "f", "H", "O", " ", "K", "N", "O", "W", "S", " ", \
"W", "H", "A", "T", " ", "E", "V", "I", "L", " ", "L", "U", "R", "K", \
"S", " ", "I", "N", " ", "T", "H", "E", " ", "H", "E", "A", "R", "T", \
"S", " ", "O", "F", " ", "M", "E", "N", "?", " ", " ", " ", " ", \
" ", " ", " ", " ", " ", " ", " ", " ", " ", " ", " ", " ", " ", " ", \
" ", " ", " ", " ", " ", " ", " ", " ", " ", " ", " ", " ", " ", " ", \
" "}
\end{lstlisting}

(J'ai remplacé les caractères non imprimables par \q{?}.)

Donc nous voyons que le premier et le troisième blocs sont vides (ou presque vide),
mais le second et le quatrième comportent clairement des mots/phrases en anglais.
Ils semble que notre hypothèse à propos de la clef soit correct (au moins en partie).
Cala signifie que le bloc de 81 octets qui apparaît le plus souvent dans le fichier
peut être trouvé à des endroits comportant des séries d'octets à zéro ou quelque
chose comme ça.

Essayons de déchiffrer le fichier entier:

\begin{lstlisting}[style=custommath]
DecryptBlock[blk_] := BitXor[key, blk]

decrypted = Map[DecryptBlock[#] &, blocks];

BinaryWrite["/home/dennis/.../tmp", Flatten[decrypted]]

Close["/home/dennis/.../tmp"]
\end{lstlisting}

\begin{figure}[H]
\centering
\myincludegraphics{ff/XOR/mask_1/mc_decrypted1.png}
\caption{Fichier déchiffré dans Midnight Commander, 1er essai}
\end{figure}

Ceci ressemble a des sortes de phrases en anglais d'un jeu, mais quelque chose ne
va pas.
Tout d'abord, la casse est inversée: les phrases et certains mots commence avec une
minuscule, tandis que d'autres caractères sont en majuscule.
De plus, certaines phrases commencent avec une mauvaise lettre.
Regardez la toute première phrase: \q{eHE WEED OF CRIME BEARS BITTER FRUIT}.
Que signifie \q{eHE}? Ne devrait-on pas avoir \q{tHE} ici?
Est-il possible que notre clef de déchiffrement ait un mauvais octet à cet endroit?

Regardons à nouveau le second bloc dans le fichier, la clef et le résultat décrypté:

\begin{lstlisting}[style=custommath]
In[]:= blocks[[2]]
Out[]= {80, 2, 74, 49, 113, 49, 51, 92, 39, 8, 92, 81, 116, 62, \
57, 80, 46, 40, 114, 36, 75, 56, 33, 76, 9, 55, 56, 59, 81, 65, 45, \
28, 60, 55, 93, 39, 90, 28, 124, 106, 16, 20, 104, 119, 8, 109, 26, \
106, 9, 97, 13, 99, 15, 119, 20, 105, 117, 98, 103, 118, 1, 126, 29, \
97, 122, 17, 15, 114, 110, 3, 5, 125, 125, 99, 126, 119, 102, 30, \
122, 2, 117}

In[]:= key
Out[]= {80, 103, 2, 116, 113, 102, 118, 25, 99, 8, 19, 23, 116, \
125, 107, 25, 99, 109, 114, 102, 14, 121, 115, 31, 9, 117, 113, 111, \
5, 4, 127, 28, 122, 101, 8, 110, 14, 18, 124, 106, 16, 20, 104, 119, \
8, 109, 26, 106, 9, 97, 13, 99, 15, 119, 20, 105, 117, 98, 103, 118, \
1, 126, 29, 97, 122, 17, 15, 114, 110, 3, 5, 125, 125, 99, 126, 119, \
102, 30, 122, 2, 117}

In[]:= BitXor[key, blocks[[2]]]
Out[]= {0, 101, 72, 69, 0, 87, 69, 69, 68, 0, 79, 70, 0, 67, 82, \
73, 77, 69, 0, 66, 69, 65, 82, 83, 0, 66, 73, 84, 84, 69, 82, 0, 70, \
82, 85, 73, 84, 14, 0, 0, 0, 0, 0, 0, 0, 0, 0, 0, 0, 0, 0, 0, 0, 0, \
0, 0, 0, 0, 0, 0, 0, 0, 0, 0, 0, 0, 0, 0, 0, 0, 0, 0, 0, 0, 0, 0, 0, \
0, 0, 0, 0}
\end{lstlisting}

L'octet chiffré est 2, l'octet de la clef est 103, $2 \oplus 103=101$ et 101 est
le code ASCII du caractère \q{e}.
A quoi devrait être égal l'octet de la clef, afin que le code ASCII résultant soit
116 (pour le caractère \q{t})?
$2 \oplus 116=118$, mettons 118 comme second octet de la clef \dots

\begin{lstlisting}[style=custommath]
key = {80, 118, 2, 116, 113, 102, 118, 25, 99, 8, 19, 23, 116, 125, 
  107, 25, 99, 109, 114, 102, 14, 121, 115, 31, 9, 117, 113, 111, 5, 
  4, 127, 28, 122, 101, 8, 110, 14, 18, 124, 106, 16, 20, 104, 119, 8,
   109, 26, 106, 9, 97, 13, 99, 15, 119, 20, 105, 117, 98, 103, 118, 
  1, 126, 29, 97, 122, 17, 15, 114, 110, 3, 5, 125, 125, 99, 126, 119,
   102, 30, 122, 2, 117}
\end{lstlisting}

\dots et déchiffrons le fichier à nouveau.

\begin{figure}[H]
\centering
\myincludegraphics{ff/XOR/mask_1/mc_decrypted2.png}
\caption{Fichier déchiffré dans Midnight Commander, 2nd essai}
\end{figure}

Ouah, maintenant, la grammaire est correcte, toutes les phrases commencent avec une lettre
correcte.
Mais encore, l'inversion de la casse est suspecte.
Pourquoi est-ce que le développeur les aurait écrites de cette façon?
Peut-être que notre clef est toujours incorrecte?

En regardant la table ASCII, nous pouvons remarquer que les codes des lettres majuscules
et des minuscules ne diffèrent que d'un bit (6ème bit en partant de 1, 0b100000):

\begin{figure}[H]
\centering
\includegraphics[width=0.7\textwidth]{ascii.png}
\caption{table \ac{ASCII} 7-bit dans Emacs}
\end{figure}

Cet octet avec seul le 6ème bit mis est 32 au format décimal.
Mais 32 est le code ASCII de l'espace!

En effet, on peut changer la casse juste en XOR-ant le code ASCII d'un caractère
avec 32 (plus à ce sujet: \myref{toupper_bit}).

Est-ce possible que les parties vides dans le fichier ne soient pas des octets à zéro,
mais plutôt des espaces?
Modifions notre clef XOR encore une fois (je vais appliquer Un XOR avec 32 à chaque
octet de la clef):

\begin{lstlisting}[style=custommath]
(* "32" is scalar and "key" is vector, but that's OK *)

In[]:= key3 = BitXor[32, key]
Out[]= {112, 86, 34, 84, 81, 70, 86, 57, 67, 40, 51, 55, 84, 93, 75, \
57, 67, 77, 82, 70, 46, 89, 83, 63, 41, 85, 81, 79, 37, 36, 95, 60, \
90, 69, 40, 78, 46, 50, 92, 74, 48, 52, 72, 87, 40, 77, 58, 74, 41, \
65, 45, 67, 47, 87, 52, 73, 85, 66, 71, 86, 33, 94, 61, 65, 90, 49, \
47, 82, 78, 35, 37, 93, 93, 67, 94, 87, 70, 62, 90, 34, 85}

In[]:= DecryptBlock[blk_] := BitXor[key3, blk]
\end{lstlisting}

Déchiffrons à nouveau le fichier d'entrée:

\begin{figure}[H]
\centering
\myincludegraphics{ff/XOR/mask_1/mc_decrypted.png}
\caption{Fichier déchiffré dans Midnight Commander, essai final}
\end{figure}

(Le fichier déchiffré est disponible
\url{https://github.com/DennisYurichev/RE-for-beginners/blob/master/ff/XOR/mask_1/files/decrypted.dat.bz2}{ici}.)

Ceci est indiscutablement un fichier source correct.
Oh, et nous voyons des nombres au début de chaque bloc. Ça doit être la source de
notre clef XOR erronée.
Il semble que le bloc de 81 octets le plus fréquent dans le fichier soit un bloc
rempli avec des espaces et contenant le caractère \q{1} à la place du second octet.
En effet, d'une façon ou d'une autre, de nombreux blocs sont entrelacés avec celui-ci.

Peut-être est-ce une sorte de remplissage pour les phrases/messages courts?
D'autres blocs de 81 octets sont aussi remplis avec des blocs d'espaces, mais avec
un chiffre différent, ainsi, ils ne diffèrent que du second octet.

C'est tout! Maintenant nous pouvons écrire un utilitaire pour chiffrer à nouveau le
fichier, et peut-être le modifier avant.

Le fichier notebook de Mathematica est téléchargeable
\url{https://github.com/DennisYurichev/RE-for-beginners/blob/master/ff/XOR/mask_1/files/XOR_mask_1.nb}{ici}.

Résumé: un tel chiffrement avec XOR n'est pas robuste du tout. Le développeur du jeu
escomptait, probablement, empêcher les joueurs de chercher des informations sur le
jeu, mais rien de plus sérieux.
Néanmoins, un tel chiffrement est très populaire du fait de sa simplicité et de nombreux
rétro-ingénieurs sont traditionnellement familier avec.

}
\EN{\subsection{Simple encryption using XOR mask, case II}
\label{XOR_mask_2}

I've got another encrypted file, which is clearly encrypted by something simple, like XOR-ing:

\begin{figure}[H]
\centering
\myincludegraphics{ff/XOR/mask_2/cipher.png}
\caption{Encrypted file in Midnight Commander}
\end{figure}

The encrypted file can be downloaded \href{https://github.com/dennis714/yurichev.com/blob/master/blog/XOR_mask_2/files/cipher.txt}{here}.

\IT{ent} Linux utility reports about ~7.5 bits per byte, and this is high level of entropy (\myref{entropy}),
close to compressed or properly encrypted file.
But still, we clearly see some pattern, there are some blocks with size of 17 bytes, hence, you see some kind of ladder, shifting by 1 byte at each 16-byte line.

It's also known that the plain text is just English language text.

Now let's assume that this piece of text is encrypted by simple XOR-ing with 17-byte key.

I tried to find some repeating 17-byte blocks in Mathematica, like I did before in my previous example (\myref{XOR_mask_1}):

\begin{lstlisting}[caption=Mathematica,style=custommath]
In[]:=input = BinaryReadList["/home/dennis/tmp/cipher.txt"];

In[]:=blocks = Partition[input, 17];

In[]:=Sort[Tally[blocks], #1[[2]] > #2[[2]] &]

Out[]:={{{248,128,88,63,58,175,159,154,232,226,161,50,97,127,3,217,80},1},
{{226,207,67,60,42,226,219,150,246,163,166,56,97,101,18,144,82},1},
{{228,128,79,49,59,250,137,154,165,236,169,118,53,122,31,217,65},1},
{{252,217,1,39,39,238,143,223,241,235,170,91,75,119,2,152,82},1},
{{244,204,88,112,59,234,151,147,165,238,170,118,49,126,27,144,95},1},
{{241,196,78,112,54,224,142,223,242,236,186,58,37,50,17,144,95},1},
{{176,201,71,112,56,230,143,151,234,246,187,118,44,125,8,156,17},1},
...
{{255,206,82,112,56,231,158,145,165,235,170,118,54,115,9,217,68},1},
{{249,206,71,34,42,254,142,154,235,247,239,57,34,113,27,138,88},1},
{{157,170,84,32,32,225,219,139,237,236,188,51,97,124,21,141,17},1},
{{248,197,1,61,32,253,149,150,235,228,188,122,97,97,27,143,84},1},
{{252,217,1,38,42,253,130,223,233,226,187,51,97,123,20,217,69},1},
{{245,211,13,112,56,231,148,223,242,226,188,118,52,97,15,152,93},1},
{{221,210,15,112,28,231,158,141,233,236,172,61,97,90,21,149,92},1}}
\end{lstlisting}

No luck, each 17-byte block is unique withing the file and occurred only once.
Perhaps, there are no 17-byte zero (or space) lacunas.
It is possible indeed: such long space indentation and padding may be absent in tightly typeset text.

The first idea is to try all possible 17-byte keys and find those, which will result in printable plain text after decryption.
Bruteforce is not an option, because there are $256^{17}$ possible keys ($\tilde{}10^{40}$), that's too much.
But there are good news: who said we have to test 17-byte key as a whole, why can't we test each byte of key separately?
It is possible indeed.

Now the algorithm is:

\begin{itemize}
\item try all 256 bytes for 1st byte of key;
\item decrypt 1st byte of each 17-byte blocks in the file;
\item are all decrypted bytes we got are printable? keep tabs on it;
\item do so for all 17 bytes of key.
\end{itemize}

I wrote with the following Python script to check this idea:

\begin{lstlisting}[caption=Python script,style=custompy]
each_Nth_byte=[""]*KEY_LEN

content=read_file(sys.argv[1])
# split input by 17-byte chunks:
all_chunks=chunks(content, KEY_LEN)
for c in all_chunks:
    for i in range(KEY_LEN):
        each_Nth_byte[i]=each_Nth_byte[i] + c[i]

# try each byte of key
for N in range(KEY_LEN):
    print "N=", N
    possible_keys=[]
    for i in range(256):
        tmp_key=chr(i)*len(each_Nth_byte[N])
        tmp=xor_strings(tmp_key,each_Nth_byte[N])
        # are all characters in tmp[] are printable?
        if is_string_printable(tmp)==False:
	    continue
	possible_keys.append(i)
    print possible_keys, "len=", len(possible_keys)
\end{lstlisting}

(Full version of the source code is \href{https://github.com/dennis714/yurichev.com/blob/master/blog/XOR_mask_2/files/decrypt2.py}{here}.)

Here is its output:

\begin{lstlisting}
N= 0
[144, 145, 151] len= 3
N= 1
[160, 161] len= 2
N= 2
[32, 33, 38] len= 3
N= 3
[80, 81, 87] len= 3
N= 4
[78, 79] len= 2
N= 5
[142, 143] len= 2
N= 6
[250, 251] len= 2
N= 7
[254, 255] len= 2
N= 8
[130, 132, 133] len= 3
N= 9
[130, 131] len= 2
N= 10
[206, 207] len= 2
N= 11
[81, 86, 87] len= 3
N= 12
[64, 65] len= 2
N= 13
[18, 19] len= 2
N= 14
[122, 123] len= 2
N= 15
[248, 249] len= 2
N= 16
[48, 49] len= 2
\end{lstlisting}

So there are 2 or 3 possible bytes for each byte of 17-byte key.
This is much better than 256 possible bytes for each byte, but still too much.
There are up to 1 million of possible keys:

\begin{lstlisting}[caption=Mathematica,style=custommath]
In[]:= 3*2*3*3*2*2*2*2*3*2*2*3*2*2*2*2*2
Out[]= 995328
\end{lstlisting}

It's possible to check all of them, but then we must check visually, if the decrypted text is looks like English language text.

Let's also take into consideration the fact that we deal with 1) natural language; 2) English language.
Natural languages has some prominent statistical features.
First of all, punctuation and word lengths.
What is average word length in English language?
Let's just count spaces in some well-known English language texts using Mathematica.

Here is \href{http://www.gutenberg.org/cache/epub/100/pg100.txt}{\q{The Complete Works of William Shakespeare}} text file from Gutenberg Library:

\begin{lstlisting}[caption=Mathematica,style=custommath]
In[]:= input = BinaryReadList["/home/dennis/tmp/pg100.txt"];

In[]:= Tally[input]
Out[]= {{239, 1}, {187, 1}, {191, 1}, {84, 39878}, {104, 
  218875}, {101, 406157}, {32, 1285884}, {80, 12038}, {114, 
  209907}, {111, 282560}, {106, 2788}, {99, 67194}, {116, 
  291243}, {71, 11261}, {117, 115225}, {110, 216805}, {98, 
  46768}, {103, 57328}, {69, 42703}, {66, 15450}, {107, 29345}, {102, 
  69103}, {67, 21526}, {109, 95890}, {112, 46849}, {108, 146532}, {87,
   16508}, {115, 215605}, {105, 199130}, {97, 245509}, {83, 
  34082}, {44, 83315}, {121, 85549}, {13, 124787}, {10, 124787}, {119,
   73155}, {100, 134216}, {118, 34077}, {46, 78216}, {89, 9128}, {45, 
  8150}, {76, 23919}, {42, 73}, {79, 33268}, {82, 29040}, {73, 
  55893}, {72, 18486}, {68, 15726}, {58, 1843}, {65, 44560}, {49, 
  982}, {50, 373}, {48, 325}, {91, 2076}, {35, 3}, {93, 2068}, {74, 
  2071}, {57, 966}, {52, 107}, {70, 11770}, {85, 14169}, {78, 
  27393}, {75, 6206}, {77, 15887}, {120, 4681}, {33, 8840}, {60, 
  468}, {86, 3587}, {51, 343}, {88, 608}, {40, 643}, {41, 644}, {62, 
  440}, {39, 31077}, {34, 488}, {59, 17199}, {126, 1}, {95, 71}, {113,
   2414}, {81, 1179}, {63, 10476}, {47, 48}, {55, 45}, {54, 73}, {64, 
  3}, {53, 94}, {56, 47}, {122, 1098}, {90, 532}, {124, 33}, {38, 
  21}, {96, 1}, {125, 2}, {37, 1}, {36, 2}}

In[]:= Length[input]/1285884 // N
Out[]= 4.34712
\end{lstlisting}

There are 1285884 spaces in the whole file, and the frequency of space occurrence is 1 space per ~4.3 characters.

Now here is \href{http://www.gutenberg.org/ebooks/11}{Alice's Adventures in Wonderland, by Lewis Carroll} from the same library:

\begin{lstlisting}[caption=Mathematica,style=custommath]
In[]:= input = BinaryReadList["/home/dennis/tmp/pg11.txt"];

In[]:= Tally[input]
Out[]= {{239, 1}, {187, 1}, {191, 1}, {80, 172}, {114, 6398}, {111, 
  9243}, {106, 222}, {101, 15082}, {99, 2815}, {116, 11629}, {32, 
  27964}, {71, 193}, {117, 3867}, {110, 7869}, {98, 1621}, {103, 
  2750}, {39, 2885}, {115, 6980}, {65, 721}, {108, 5053}, {105, 
  7802}, {100, 5227}, {118, 911}, {87, 256}, {97, 9081}, {44, 
  2566}, {121, 2442}, {76, 158}, {119, 2696}, {67, 185}, {13, 
  3735}, {10, 3735}, {84, 571}, {104, 7580}, {66, 125}, {107, 
  1202}, {102, 2248}, {109, 2245}, {46, 1206}, {89, 142}, {112, 
  1796}, {45, 744}, {58, 255}, {68, 242}, {74, 13}, {50, 12}, {53, 
  13}, {48, 22}, {56, 10}, {91, 4}, {69, 313}, {35, 1}, {49, 68}, {93,
   4}, {82, 212}, {77, 222}, {57, 11}, {52, 10}, {42, 88}, {83, 
  288}, {79, 234}, {70, 134}, {72, 309}, {73, 831}, {85, 111}, {78, 
  182}, {75, 88}, {86, 52}, {51, 13}, {63, 202}, {40, 76}, {41, 
  76}, {59, 194}, {33, 451}, {113, 135}, {120, 170}, {90, 1}, {122, 
  79}, {34, 135}, {95, 4}, {81, 85}, {88, 6}, {47, 24}, {55, 6}, {54, 
  7}, {37, 1}, {64, 2}, {36, 2}}

In[]:= Length[input]/27964 // N
Out[]= 5.99049
\end{lstlisting}

The result is different probably because of different formatting of these texts (maybe indentation and/or padding).

OK, so let's assume the average frequency of space in English language is 1 space per 4..7 characters.

Now the good news again: we can measure frequency of spaces while decrypting our file gradually.
Now I count spaces in each \IT{slice} and throw away 1-byte keys which produce buffers with too small number of spaces (or too large, but this is almost impossible given so short key):

\begin{lstlisting}[caption=Python script,style=custompy]
each_Nth_byte=[""]*KEY_LEN

content=read_file(sys.argv[1])
# split input by 17-byte chunks:
all_chunks=chunks(content, KEY_LEN)
for c in all_chunks:
    for i in range(KEY_LEN):
        each_Nth_byte[i]=each_Nth_byte[i] + c[i]

# try each byte of key
for N in range(KEY_LEN):
    print "N=", N
    possible_keys=[]
    for i in range(256):
        tmp_key=chr(i)*len(each_Nth_byte[N])
        tmp=xor_strings(tmp_key,each_Nth_byte[N])
        # are all characters in tmp[] are printable?
        if is_string_printable(tmp)==False:
	    continue
        # count spaces in decrypted buffer:
	spaces=tmp.count(' ')
	if spaces==0:
            continue
	spaces_ratio=len(tmp)/spaces
	if spaces_ratio<4:
	    continue
	if spaces_ratio>7:
	    continue
	possible_keys.append(i)
    print possible_keys, "len=", len(possible_keys)
\end{lstlisting}

(Full version of the source code is \href{https://github.com/dennis714/yurichev.com/blob/master/blog/XOR_mask_2/files/decrypt3.py}{here}.)

This reports just one single possible byte for each byte of key:

\begin{lstlisting}
N= 0
[144] len= 1
N= 1
[160] len= 1
N= 2
[33] len= 1
N= 3
[80] len= 1
N= 4
[79] len= 1
N= 5
[143] len= 1
N= 6
[251] len= 1
N= 7
[255] len= 1
N= 8
[133] len= 1
N= 9
[131] len= 1
N= 10
[207] len= 1
N= 11
[86] len= 1
N= 12
[65] len= 1
N= 13
[18] len= 1
N= 14
[122] len= 1
N= 15
[249] len= 1
N= 16
[49] len= 1
\end{lstlisting}

Let's check this key in Mathematica:

\begin{lstlisting}[caption=Mathematica,style=custommath]
In[]:= input = BinaryReadList["/home/dennis/tmp/cipher.txt"];

In[]:= blocks = Partition[input, 17];

In[]:= key = {144, 160, 33, 80, 79, 143, 251, 255, 133, 131, 207, 86, 65, 18, 122, 249, 49};

In[]:= EncryptBlock[blk_] := BitXor[key, blk]

In[]:= encrypted = Map[EncryptBlock[#] &, blocks];

In[]:= BinaryWrite["/home/dennis/tmp/plain2.txt", Flatten[encrypted]]

In[]:= Close["/home/dennis/tmp/plain2.txt"]
\end{lstlisting}

And the plain text is:

\begin{lstlisting}
Mr. Sherlock Holmes, who was usually very late in the mornings, save
upon those not infrequent occasions when he was up all night, was seated
at the breakfast table. I stood upon the hearth-rug and picked up the
stick which our visitor had left behind him the night before. It was a
fine, thick piece of wood, bulbous-headed, of the sort which is known as
a "Penang lawyer." Just under the head was a broad silver band nearly
an inch across. "To James Mortimer, M.R.C.S., from his friends of the
C.C.H.," was engraved upon it, with the date "1884." It was just such a
stick as the old-fashioned family practitioner used to carry--dignified,
solid, and reassuring.

"Well, Watson, what do you make of it?"

Holmes was sitting with his back to me, and I had given him no sign of
my occupation.

...
\end{lstlisting}

(Full version of the text is \href{https://github.com/dennis714/yurichev.com/blob/master/blog/XOR_mask_2/files/plain.txt}{here}.)

The text looks correct.
Yes, I made up this example and choose well-known text of Conan Doyle, but it's very close to what I had in my practice some time ago.

\subsubsection{Other ideas to consider}

If we would fail with space counting, there are other ideas to try:

\begin{itemize}

\item Take into consideration the fact that lowercase letters are much more frequent than uppercase ones.

\item Frequency analysis.

\item There is also a good technique to detect language of a text: trigrams.
Each language has some very frequent letter triplets, these may be \q{the} and \q{tha} for English.
Read more about it:
\href{http://odur.let.rug.nl/~vannoord/TextCat/textcat.pdf}{N-Gram-Based Text Categorization},
\url{http://code.activestate.com/recipes/326576/}.
Interestingly enough, trigrams detection can be used when you decrypt a ciphertext gradually, like in this example (you just have to test 3 adjacent decrypted characters).

For non-Latin writing systems encoded in UTF-8, things may be easier. For example, Russian text encoded in UTF-8 has each byte interleaved with 0xD0/0xD1 byte.
It is because Cyrillic characters are placed in 4th block of Unicode.
Other writing systems has their own blocks.

\end{itemize}
}\RU{% TODO translate
\subsection{Простое шифрование используя XOR-маску, второй случай}
\label{XOR_mask_2}

Нашел еще один зашифрованный файл, который явно зашифрован чем-то простым вроде XOR-шифрования:

\begin{figure}[H]
\centering
\myincludegraphics{ff/XOR/mask_2/cipher.png}
\caption{Зашифрованный файл в Midnight Commander}
\end{figure}

Зашифрованный файл можно скачать \href{https://github.com/dennis714/RE-for-beginners/blob/master/ff/XOR/mask_2/files/cipher.txt}{здесь}.

Утилита \IT{ent} в Linux сообщает о $\textasciitilde{}7.5$ бит на байт, и это высокий уровень энтропии (\myref{entropy}),
что близко к сжатому или правильно зашифрованному файлу.
Но все-таки, мы ясно видим некоторый шаблон, здесь есть блоки длиной в 17 байт, и вы можете увидеть что-то вроде лестницы,
сдвигающеся на 1 байт на каждой 16-байтной линии.

Также известно, что исходный текст это текст на английском языке.

Предположим что этот фрагмент текста зашифрован простым XOR-шифрованием с 17-байтным ключом.

Я попробовал поискать повторяющиеся 17-байтные блоки при помощи Mathematica, как я делал это в моем предыдущем примере
(\myref{XOR_mask_1}):

\begin{lstlisting}[caption=Mathematica,style=custommath]
In[]:=input = BinaryReadList["/home/dennis/tmp/cipher.txt"];

In[]:=blocks = Partition[input, 17];

In[]:=Sort[Tally[blocks], #1[[2]] > #2[[2]] &]

Out[]:={{{248,128,88,63,58,175,159,154,232,226,161,50,97,127,3,217,80},1},
{{226,207,67,60,42,226,219,150,246,163,166,56,97,101,18,144,82},1},
{{228,128,79,49,59,250,137,154,165,236,169,118,53,122,31,217,65},1},
{{252,217,1,39,39,238,143,223,241,235,170,91,75,119,2,152,82},1},
{{244,204,88,112,59,234,151,147,165,238,170,118,49,126,27,144,95},1},
{{241,196,78,112,54,224,142,223,242,236,186,58,37,50,17,144,95},1},
{{176,201,71,112,56,230,143,151,234,246,187,118,44,125,8,156,17},1},
...
{{255,206,82,112,56,231,158,145,165,235,170,118,54,115,9,217,68},1},
{{249,206,71,34,42,254,142,154,235,247,239,57,34,113,27,138,88},1},
{{157,170,84,32,32,225,219,139,237,236,188,51,97,124,21,141,17},1},
{{248,197,1,61,32,253,149,150,235,228,188,122,97,97,27,143,84},1},
{{252,217,1,38,42,253,130,223,233,226,187,51,97,123,20,217,69},1},
{{245,211,13,112,56,231,148,223,242,226,188,118,52,97,15,152,93},1},
{{221,210,15,112,28,231,158,141,233,236,172,61,97,90,21,149,92},1}}
\end{lstlisting}

Ничего не выходит, каждый 17-байтный блок уникален внутри файла и встречается только один раз.
Возможно, здесь нет 17-байтных нулевых лакун (или лакун содержащих пробелы).
Это действительно возможно: подобное выравнивание пробелами может и отсутствовать в плотно сверстаном тексте.

Первая идея это попробовать все возможные 17-байтные ключи и найти тот, который после дешифровки приведет к читаемому тексту.
Полный перебор брутфорсом это не вариант, потому что здесь $256^{17}$ возможных ключей ($\textasciitilde{}10^{40}$),
это слишком.
Но есть и хорошие новости: кто сказал что нужно тестировать 17-байтный ключ как что-то целое, почему мы не можем тестировать
каждый байт ключа отдельно?
Это действительно возможно.

И алгоритм такой:

\begin{itemize}
\item попробовать все 256 байт для первого байта ключа;
\item дешифровать первый байт каждого 17-байтного блока в файле;
\item все ли полученные дешифрованные байты печатаемы? вести учет;
\item делать это для всех 17 байт ключа.
\end{itemize}

Я написал такой скрипт на Питоне для проверки этой идеи:

\begin{lstlisting}[caption=Python script,style=custompy]
each_Nth_byte=[""]*KEY_LEN

content=read_file(sys.argv[1])
# split input by 17-byte chunks:
all_chunks=chunks(content, KEY_LEN)
for c in all_chunks:
    for i in range(KEY_LEN):
        each_Nth_byte[i]=each_Nth_byte[i] + c[i]

# try each byte of key
for N in range(KEY_LEN):
    print "N=", N
    possible_keys=[]
    for i in range(256):
        tmp_key=chr(i)*len(each_Nth_byte[N])
        tmp=xor_strings(tmp_key,each_Nth_byte[N])
        # are all characters in tmp[] are printable?
        if is_string_printable(tmp)==False:
	    continue
	possible_keys.append(i)
    print possible_keys, "len=", len(possible_keys)
\end{lstlisting}

(Полная версия исходного кода \href{https://github.com/dennis714/RE-for-beginners/blob/master/ff/XOR/mask_2/files/decrypt2.py}{здесь}.)

И вот вывод:

\begin{lstlisting}
N= 0
[144, 145, 151] len= 3
N= 1
[160, 161] len= 2
N= 2
[32, 33, 38] len= 3
N= 3
[80, 81, 87] len= 3
N= 4
[78, 79] len= 2
N= 5
[142, 143] len= 2
N= 6
[250, 251] len= 2
N= 7
[254, 255] len= 2
N= 8
[130, 132, 133] len= 3
N= 9
[130, 131] len= 2
N= 10
[206, 207] len= 2
N= 11
[81, 86, 87] len= 3
N= 12
[64, 65] len= 2
N= 13
[18, 19] len= 2
N= 14
[122, 123] len= 2
N= 15
[248, 249] len= 2
N= 16
[48, 49] len= 2
\end{lstlisting}

Так что есть 2 или 3 возможных байта для каждого байта 17-байтного ключа.
Это намного лучше чем 256 возможных байт для каждого ключа, но все равно слишком.
Тут вплоть до одного миллиона возможных ключей:

\begin{lstlisting}[caption=Mathematica,style=custommath]
In[]:= 3*2*3*3*2*2*2*2*3*2*2*3*2*2*2*2*2
Out[]= 995328
\end{lstlisting}

Можно проверить их все, но затем нам придется проверять визуально, похож ли дешифрованный текст на текст на английском языке.

Также будет учитывать те факты, что мы имеем дело с 1) человеческим языком; 2) английским языком.
Человеческие языки имеют выдающиеся статистические особенности.
Прежде всего, пунктуация и длины слов.
Какая средняя длина слова в английском языке?
Просто будем считать пробелы в некоторых хорошо известных текстах на английском используя Mathematica.

Вот текст is \href{http://www.gutenberg.org/cache/epub/100/pg100.txt}{\q{The Complete Works of William Shakespeare}}
из библиотеки Гутенберга:

\begin{lstlisting}[caption=Mathematica,style=custommath]
In[]:= input = BinaryReadList["/home/dennis/tmp/pg100.txt"];

In[]:= Tally[input]
Out[]= {{239, 1}, {187, 1}, {191, 1}, {84, 39878}, {104, 
  218875}, {101, 406157}, {32, 1285884}, {80, 12038}, {114, 
  209907}, {111, 282560}, {106, 2788}, {99, 67194}, {116, 
  291243}, {71, 11261}, {117, 115225}, {110, 216805}, {98, 
  46768}, {103, 57328}, {69, 42703}, {66, 15450}, {107, 29345}, {102, 
  69103}, {67, 21526}, {109, 95890}, {112, 46849}, {108, 146532}, {87,
   16508}, {115, 215605}, {105, 199130}, {97, 245509}, {83, 
  34082}, {44, 83315}, {121, 85549}, {13, 124787}, {10, 124787}, {119,
   73155}, {100, 134216}, {118, 34077}, {46, 78216}, {89, 9128}, {45, 
  8150}, {76, 23919}, {42, 73}, {79, 33268}, {82, 29040}, {73, 
  55893}, {72, 18486}, {68, 15726}, {58, 1843}, {65, 44560}, {49, 
  982}, {50, 373}, {48, 325}, {91, 2076}, {35, 3}, {93, 2068}, {74, 
  2071}, {57, 966}, {52, 107}, {70, 11770}, {85, 14169}, {78, 
  27393}, {75, 6206}, {77, 15887}, {120, 4681}, {33, 8840}, {60, 
  468}, {86, 3587}, {51, 343}, {88, 608}, {40, 643}, {41, 644}, {62, 
  440}, {39, 31077}, {34, 488}, {59, 17199}, {126, 1}, {95, 71}, {113,
   2414}, {81, 1179}, {63, 10476}, {47, 48}, {55, 45}, {54, 73}, {64, 
  3}, {53, 94}, {56, 47}, {122, 1098}, {90, 532}, {124, 33}, {38, 
  21}, {96, 1}, {125, 2}, {37, 1}, {36, 2}}

In[]:= Length[input]/1285884 // N
Out[]= 4.34712
\end{lstlisting}

Тут 1285884 пробела во всем файле, и распространение пробелов это один пробел на $\textasciitilde{}4.3$ символов.

Теперь вот \href{http://www.gutenberg.org/ebooks/11}{Alice's Adventures in Wonderland, by Lewis Carroll} из той же библиотеки:

\begin{lstlisting}[caption=Mathematica,style=custommath]
In[]:= input = BinaryReadList["/home/dennis/tmp/pg11.txt"];

In[]:= Tally[input]
Out[]= {{239, 1}, {187, 1}, {191, 1}, {80, 172}, {114, 6398}, {111, 
  9243}, {106, 222}, {101, 15082}, {99, 2815}, {116, 11629}, {32, 
  27964}, {71, 193}, {117, 3867}, {110, 7869}, {98, 1621}, {103, 
  2750}, {39, 2885}, {115, 6980}, {65, 721}, {108, 5053}, {105, 
  7802}, {100, 5227}, {118, 911}, {87, 256}, {97, 9081}, {44, 
  2566}, {121, 2442}, {76, 158}, {119, 2696}, {67, 185}, {13, 
  3735}, {10, 3735}, {84, 571}, {104, 7580}, {66, 125}, {107, 
  1202}, {102, 2248}, {109, 2245}, {46, 1206}, {89, 142}, {112, 
  1796}, {45, 744}, {58, 255}, {68, 242}, {74, 13}, {50, 12}, {53, 
  13}, {48, 22}, {56, 10}, {91, 4}, {69, 313}, {35, 1}, {49, 68}, {93,
   4}, {82, 212}, {77, 222}, {57, 11}, {52, 10}, {42, 88}, {83, 
  288}, {79, 234}, {70, 134}, {72, 309}, {73, 831}, {85, 111}, {78, 
  182}, {75, 88}, {86, 52}, {51, 13}, {63, 202}, {40, 76}, {41, 
  76}, {59, 194}, {33, 451}, {113, 135}, {120, 170}, {90, 1}, {122, 
  79}, {34, 135}, {95, 4}, {81, 85}, {88, 6}, {47, 24}, {55, 6}, {54, 
  7}, {37, 1}, {64, 2}, {36, 2}}

In[]:= Length[input]/27964 // N
Out[]= 5.99049
\end{lstlisting}

Результат другой, вероятно потому что используется разное форматирование этих текстов (может быть из-за выравнивания
и отступов).

ОК, будем считать что средняя частота появления пробела в английском тексте это 1 пробел на 4..7 символов.

И снова хорошие новости: мы можем измерять частоту пробелов во время постепенного дешифрования файла.
Теперь я считаю пробелы в каждом \IT{ломтике} и выкидываю 1-байтные ключи, которые приводят к результатам со слишком
малым количеством пробелов
(или слишком большим, но это почти невозможно учитывая такой короткий ключ):

\begin{lstlisting}[caption=Python script,style=custompy]
each_Nth_byte=[""]*KEY_LEN

content=read_file(sys.argv[1])
# split input by 17-byte chunks:
all_chunks=chunks(content, KEY_LEN)
for c in all_chunks:
    for i in range(KEY_LEN):
        each_Nth_byte[i]=each_Nth_byte[i] + c[i]

# try each byte of key
for N in range(KEY_LEN):
    print "N=", N
    possible_keys=[]
    for i in range(256):
        tmp_key=chr(i)*len(each_Nth_byte[N])
        tmp=xor_strings(tmp_key,each_Nth_byte[N])
        # are all characters in tmp[] are printable?
        if is_string_printable(tmp)==False:
	    continue
        # count spaces in decrypted buffer:
	spaces=tmp.count(' ')
	if spaces==0:
            continue
	spaces_ratio=len(tmp)/spaces
	if spaces_ratio<4:
	    continue
	if spaces_ratio>7:
	    continue
	possible_keys.append(i)
    print possible_keys, "len=", len(possible_keys)
\end{lstlisting}

(Полная версия исходного кода \href{https://github.com/dennis714/RE-for-beginners/blob/master/ff/XOR/mask_2/files/decrypt3.py}{здесь}.)

Это выдает всего один возможный байт для каждого байта ключа:

\begin{lstlisting}
N= 0
[144] len= 1
N= 1
[160] len= 1
N= 2
[33] len= 1
N= 3
[80] len= 1
N= 4
[79] len= 1
N= 5
[143] len= 1
N= 6
[251] len= 1
N= 7
[255] len= 1
N= 8
[133] len= 1
N= 9
[131] len= 1
N= 10
[207] len= 1
N= 11
[86] len= 1
N= 12
[65] len= 1
N= 13
[18] len= 1
N= 14
[122] len= 1
N= 15
[249] len= 1
N= 16
[49] len= 1
\end{lstlisting}

Проверим этот ключ в Mathematica:

\begin{lstlisting}[caption=Mathematica,style=custommath]
In[]:= input = BinaryReadList["/home/dennis/tmp/cipher.txt"];

In[]:= blocks = Partition[input, 17];

In[]:= key = {144, 160, 33, 80, 79, 143, 251, 255, 133, 131, 207, 86, 65, 18, 122, 249, 49};

In[]:= EncryptBlock[blk_] := BitXor[key, blk]

In[]:= encrypted = Map[EncryptBlock[#] &, blocks];

In[]:= BinaryWrite["/home/dennis/tmp/plain2.txt", Flatten[encrypted]]

In[]:= Close["/home/dennis/tmp/plain2.txt"]
\end{lstlisting}

И дешифрованный текст:

\begin{lstlisting}
Mr. Sherlock Holmes, who was usually very late in the mornings, save
upon those not infrequent occasions when he was up all night, was seated
at the breakfast table. I stood upon the hearth-rug and picked up the
stick which our visitor had left behind him the night before. It was a
fine, thick piece of wood, bulbous-headed, of the sort which is known as
a "Penang lawyer." Just under the head was a broad silver band nearly
an inch across. "To James Mortimer, M.R.C.S., from his friends of the
C.C.H.," was engraved upon it, with the date "1884." It was just such a
stick as the old-fashioned family practitioner used to carry--dignified,
solid, and reassuring.

"Well, Watson, what do you make of it?"

Holmes was sitting with his back to me, and I had given him no sign of
my occupation.

...
\end{lstlisting}

(Полная версия текста \href{https://github.com/dennis714/RE-for-beginners/blob/master/ff/XOR/mask_2/files/plain.txt}{здесь}.)

Текст выглядит правильным.
Да, я придумал этот пример и выбрал хорошо известный текст Конан Дойля, но это очень близко к тому,
что у меня недавно было на практике.

\subsubsection{Другие идеи}

Если бы не получилось с подсчетом пробелов, вот еще идеи, которые можно было бы попробовать:

\begin{itemize}

\item Учитывать тот факт что буквы в нижнем регистре встречаются намного чаще, чем в верхнем.

\item Частотный анализ.

\item Есть очень хорошая техника для определения языка текста: триграммы.
Каждый язык имеет часто встречающиеся тройки буквы, для английского это могут быть \q{the} и \q{tha}.
Больше об этом:
\href{http://odur.let.rug.nl/~vannoord/TextCat/textcat.pdf}{N-Gram-Based Text Categorization},
\url{http://code.activestate.com/recipes/326576/}.
Интересно знать, что выявление триграмм может быть использовано при постепенном дешифровании текста, как в этом примере
(нужно просто проверять 3 рядом стоящих дешифрованных символа).

Для систем письменности отличных от латинского алфавита, закодированных в UTF-8, все может быть еще проще.
Например, в тексте на русском, закодированном в UTF-8, каждый байт перемежается с байтом 0xD0 или 0xD1.
Это потому что символы кириллицы расположен в 4-м блоке в таблице Уникода.
Другие системы письменности имеют свои блоки.

\end{itemize}

}%
\FR{% TODO translate
\subsection{Chiffrement simple utilisant un masque XOR, cas II}
\label{XOR_mask_2}

J'ai un autre fichier chiffré, qui est clairement chiffré avec quelque chose de simple,
comme un XOR:

\begin{figure}[H]
\centering
\myincludegraphics{ff/XOR/mask_2/cipher.png}
\caption{Fichier chiffré dans Midnight Commander}
\end{figure}

Le fichier chiffré peut être téléchargé
\href{https://github.com/DennisYurichev/RE-for-beginners/blob/master/ff/XOR/mask_2/files/cipher.txt}{ici}.

L'utilitaire Linux \IT{ent} indique environ $\textasciitilde{}7.5$ bits par octet,
et ceci est un haut niveau d'entropie (\myref{entropy}), proche de celui de fichiers
compressés ou chiffrés correctement.
Mais encore, nous distinguons clairement quelques patterns, il y a quelques blocs
avec une taille de 17 octets, et nous pouvons voir des sortes d'échelles, se décalant
d'un octet à chaque ligne de 16 octets.

On sait aussi que le texte clair est en anglais.

Maintenant, supposons que ce morceau de texte est chiffré par un simple XOR avec une
clef de 17 octets.

J'ai essayé de repérer des blocs de 17 octets se répétant avec Mathematica, comme
je l'ai fait dans l'exemple précédant (\myref{XOR_mask_1}):

\begin{lstlisting}[caption=Mathematica,style=custommath]
In[]:=input = BinaryReadList["/home/dennis/tmp/cipher.txt"];

In[]:=blocks = Partition[input, 17];

In[]:=Sort[Tally[blocks], #1[[2]] > #2[[2]] &]

Out[]:={{{248,128,88,63,58,175,159,154,232,226,161,50,97,127,3,217,80},1},
{{226,207,67,60,42,226,219,150,246,163,166,56,97,101,18,144,82},1},
{{228,128,79,49,59,250,137,154,165,236,169,118,53,122,31,217,65},1},
{{252,217,1,39,39,238,143,223,241,235,170,91,75,119,2,152,82},1},
{{244,204,88,112,59,234,151,147,165,238,170,118,49,126,27,144,95},1},
{{241,196,78,112,54,224,142,223,242,236,186,58,37,50,17,144,95},1},
{{176,201,71,112,56,230,143,151,234,246,187,118,44,125,8,156,17},1},
...
{{255,206,82,112,56,231,158,145,165,235,170,118,54,115,9,217,68},1},
{{249,206,71,34,42,254,142,154,235,247,239,57,34,113,27,138,88},1},
{{157,170,84,32,32,225,219,139,237,236,188,51,97,124,21,141,17},1},
{{248,197,1,61,32,253,149,150,235,228,188,122,97,97,27,143,84},1},
{{252,217,1,38,42,253,130,223,233,226,187,51,97,123,20,217,69},1},
{{245,211,13,112,56,231,148,223,242,226,188,118,52,97,15,152,93},1},
{{221,210,15,112,28,231,158,141,233,236,172,61,97,90,21,149,92},1}}
\end{lstlisting}

Pas de chance, chaque bloc de 17 octets est unique dans le fichier, et n'apparaît
donc qu'une fois.
Peut-être n'y a-t-il pas de zone de 17 octets à zéro, ou de zone contenant seulement
des espaces.
C'est possible toutefois: de telles séries d'espace peuvent être absentes dans des
textes composés rigoureusement.

La première idée est d'essayer toutes les clefs de 17 octets possible et trouver
celles qui donnent un résultat lisible après déchiffrement.
La force brute n'est pas une option, car il y a $256^{17}$ clefs possible ($\textasciitilde{}10^{40}$),
c'est beaucoup trop.
Mais il y a une bonne nouvelle: qui a dit que nous devons tester la clef de 17 octets
en entier, pourquoi ne pas teste chaque octet séparémment?
C'est possible en effet.

Maintenant, l'algorithme est:

\begin{itemize}
\item essayer chacun des 25 octets pour le premier octet de la clef;
\item déchiffrer le 1er octet de chaque bloc de 17 octets du fichier;
\item est-ce que tous les octets déchiffrés sont imprimable? garder un oeil dessus;
\item faire de même pour chacun des 17 octets de la clef.
\end{itemize}

J'ai écrit le script Python suivant pour essayer cette idée:

\begin{lstlisting}[caption=Python script,style=custompy]
each_Nth_byte=[""]*KEY_LEN

content=read_file(sys.argv[1])
# split input by 17-byte chunks:
all_chunks=chunks(content, KEY_LEN)
for c in all_chunks:
    for i in range(KEY_LEN):
        each_Nth_byte[i]=each_Nth_byte[i] + c[i]

# try each byte of key
for N in range(KEY_LEN):
    print "N=", N
    possible_keys=[]
    for i in range(256):
        tmp_key=chr(i)*len(each_Nth_byte[N])
        tmp=xor_strings(tmp_key,each_Nth_byte[N])
        # are all characters in tmp[] are printable?
        if is_string_printable(tmp)==False:
	    continue
	possible_keys.append(i)
    print possible_keys, "len=", len(possible_keys)
\end{lstlisting}

(La version complète du code source est
 \href{https://github.com/DennisYurichev/RE-for-beginners/blob/master/ff/XOR/mask_2/files/decrypt2.py}{ici}.)

Voici sa sortie:

\begin{lstlisting}
N= 0
[144, 145, 151] len= 3
N= 1
[160, 161] len= 2
N= 2
[32, 33, 38] len= 3
N= 3
[80, 81, 87] len= 3
N= 4
[78, 79] len= 2
N= 5
[142, 143] len= 2
N= 6
[250, 251] len= 2
N= 7
[254, 255] len= 2
N= 8
[130, 132, 133] len= 3
N= 9
[130, 131] len= 2
N= 10
[206, 207] len= 2
N= 11
[81, 86, 87] len= 3
N= 12
[64, 65] len= 2
N= 13
[18, 19] len= 2
N= 14
[122, 123] len= 2
N= 15
[248, 249] len= 2
N= 16
[48, 49] len= 2
\end{lstlisting}

Donc, il y a 2 ou 3 octets possible pour chaque octet de l clef de 17 octets.
C'est mieux que 256 octets pour chaque octet, mais encore beaucoup trop.
Il y a environ 1 million de clefs possible:

\begin{lstlisting}[caption=Mathematica,style=custommath]
In[]:= 3*2*3*3*2*2*2*2*3*2*2*3*2*2*2*2*2
Out[]= 995328
\end{lstlisting}

Il est possible de les vérifier toutes, mais alors nous devons vérifier visuellement
si le texte déchiffré à l'air d'un texte en anglais.

Prenons en compte le fait que nous avons à faire avec 1) un langage naturel 2) de l'anglais.
Les langages naturels ont quelques caractéristiques statistiques importantes.
Tout d'abord, le ponctuation et la longueur des mots.
Quelle est la longueur moyenne des mots en anglais?
Comptons les espaces dans quelques textes bien connus en anglais avec Mathematica.

Voici le fichier texte de \href{http://www.gutenberg.org/cache/epub/100/pg100.txt}{\q{The Complete Works of William Shakespeare}}
provenant de la bibliothèque Gutenberg.

\begin{lstlisting}[caption=Mathematica,style=custommath]
In[]:= input = BinaryReadList["/home/dennis/tmp/pg100.txt"];

In[]:= Tally[input]
Out[]= {{239, 1}, {187, 1}, {191, 1}, {84, 39878}, {104, 
  218875}, {101, 406157}, {32, 1285884}, {80, 12038}, {114, 
  209907}, {111, 282560}, {106, 2788}, {99, 67194}, {116, 
  291243}, {71, 11261}, {117, 115225}, {110, 216805}, {98, 
  46768}, {103, 57328}, {69, 42703}, {66, 15450}, {107, 29345}, {102, 
  69103}, {67, 21526}, {109, 95890}, {112, 46849}, {108, 146532}, {87,
   16508}, {115, 215605}, {105, 199130}, {97, 245509}, {83, 
  34082}, {44, 83315}, {121, 85549}, {13, 124787}, {10, 124787}, {119,
   73155}, {100, 134216}, {118, 34077}, {46, 78216}, {89, 9128}, {45, 
  8150}, {76, 23919}, {42, 73}, {79, 33268}, {82, 29040}, {73, 
  55893}, {72, 18486}, {68, 15726}, {58, 1843}, {65, 44560}, {49, 
  982}, {50, 373}, {48, 325}, {91, 2076}, {35, 3}, {93, 2068}, {74, 
  2071}, {57, 966}, {52, 107}, {70, 11770}, {85, 14169}, {78, 
  27393}, {75, 6206}, {77, 15887}, {120, 4681}, {33, 8840}, {60, 
  468}, {86, 3587}, {51, 343}, {88, 608}, {40, 643}, {41, 644}, {62, 
  440}, {39, 31077}, {34, 488}, {59, 17199}, {126, 1}, {95, 71}, {113,
   2414}, {81, 1179}, {63, 10476}, {47, 48}, {55, 45}, {54, 73}, {64, 
  3}, {53, 94}, {56, 47}, {122, 1098}, {90, 532}, {124, 33}, {38, 
  21}, {96, 1}, {125, 2}, {37, 1}, {36, 2}}

In[]:= Length[input]/1285884 // N
Out[]= 4.34712
\end{lstlisting}

Il y a 1285884 espaces dans l'ensemble du fichier, et la fréquence de l'occurrence
des espaces est de 1 par $\textasciitilde{}4.3$ caractères.

Maintenant voici \href{http://www.gutenberg.org/ebooks/11}{Alice's Adventures in Wonderland, par Lewis Carroll}
de la même bibliothèque:

\begin{lstlisting}[caption=Mathematica,style=custommath]
In[]:= input = BinaryReadList["/home/dennis/tmp/pg11.txt"];

In[]:= Tally[input]
Out[]= {{239, 1}, {187, 1}, {191, 1}, {80, 172}, {114, 6398}, {111, 
  9243}, {106, 222}, {101, 15082}, {99, 2815}, {116, 11629}, {32, 
  27964}, {71, 193}, {117, 3867}, {110, 7869}, {98, 1621}, {103, 
  2750}, {39, 2885}, {115, 6980}, {65, 721}, {108, 5053}, {105, 
  7802}, {100, 5227}, {118, 911}, {87, 256}, {97, 9081}, {44, 
  2566}, {121, 2442}, {76, 158}, {119, 2696}, {67, 185}, {13, 
  3735}, {10, 3735}, {84, 571}, {104, 7580}, {66, 125}, {107, 
  1202}, {102, 2248}, {109, 2245}, {46, 1206}, {89, 142}, {112, 
  1796}, {45, 744}, {58, 255}, {68, 242}, {74, 13}, {50, 12}, {53, 
  13}, {48, 22}, {56, 10}, {91, 4}, {69, 313}, {35, 1}, {49, 68}, {93,
   4}, {82, 212}, {77, 222}, {57, 11}, {52, 10}, {42, 88}, {83, 
  288}, {79, 234}, {70, 134}, {72, 309}, {73, 831}, {85, 111}, {78, 
  182}, {75, 88}, {86, 52}, {51, 13}, {63, 202}, {40, 76}, {41, 
  76}, {59, 194}, {33, 451}, {113, 135}, {120, 170}, {90, 1}, {122, 
  79}, {34, 135}, {95, 4}, {81, 85}, {88, 6}, {47, 24}, {55, 6}, {54, 
  7}, {37, 1}, {64, 2}, {36, 2}}

In[]:= Length[input]/27964 // N
Out[]= 5.99049
\end{lstlisting}

Le résultat est différent, sans soute à cause d'un formatage des textes différents
(indentation ou remplissage).

Ok, donc supposons que la fréquence moyenne de l'espace en anglais est de 1 espace
tous les 4..7 caractères.

Maintenant, encore une bonne nouvelle: nous pouvons mesurer la fréquence des espaces
au fur et à mesure du déchiffrement de notre fichier.
Maintenant je compte les espaces dans chaque \IT{slice} et jette les clefs de 1 octets
qui produise un résultat avec un nombre d'espaces trop petit (ou trop grand, mais
c'est presque impossible avec une si petite clef):

\begin{lstlisting}[caption=Python script,style=custompy]
each_Nth_byte=[""]*KEY_LEN

content=read_file(sys.argv[1])
# split input by 17-byte chunks:
all_chunks=chunks(content, KEY_LEN)
for c in all_chunks:
    for i in range(KEY_LEN):
        each_Nth_byte[i]=each_Nth_byte[i] + c[i]

# try each byte of key
for N in range(KEY_LEN):
    print "N=", N
    possible_keys=[]
    for i in range(256):
        tmp_key=chr(i)*len(each_Nth_byte[N])
        tmp=xor_strings(tmp_key,each_Nth_byte[N])
        # are all characters in tmp[] are printable?
        if is_string_printable(tmp)==False:
	    continue
        # count spaces in decrypted buffer:
	spaces=tmp.count(' ')
	if spaces==0:
            continue
	spaces_ratio=len(tmp)/spaces
	if spaces_ratio<4:
	    continue
	if spaces_ratio>7:
	    continue
	possible_keys.append(i)
    print possible_keys, "len=", len(possible_keys)
\end{lstlisting}

(La version complète du code source se trouve
\href{https://github.com/DennisYurichev/RE-for-beginners/blob/master/ff/XOR/mask_2/files/decrypt3.py}{ici}.)

Ceci nous donne un seul octet possible pour chaque octet de la clef:

\begin{lstlisting}
N= 0
[144] len= 1
N= 1
[160] len= 1
N= 2
[33] len= 1
N= 3
[80] len= 1
N= 4
[79] len= 1
N= 5
[143] len= 1
N= 6
[251] len= 1
N= 7
[255] len= 1
N= 8
[133] len= 1
N= 9
[131] len= 1
N= 10
[207] len= 1
N= 11
[86] len= 1
N= 12
[65] len= 1
N= 13
[18] len= 1
N= 14
[122] len= 1
N= 15
[249] len= 1
N= 16
[49] len= 1
\end{lstlisting}

Vérifions cette clef dans Mathematica:

\begin{lstlisting}[caption=Mathematica,style=custommath]
In[]:= input = BinaryReadList["/home/dennis/tmp/cipher.txt"];

In[]:= blocks = Partition[input, 17];

In[]:= key = {144, 160, 33, 80, 79, 143, 251, 255, 133, 131, 207, 86, 65, 18, 122, 249, 49};

In[]:= EncryptBlock[blk_] := BitXor[key, blk]

In[]:= encrypted = Map[EncryptBlock[#] &, blocks];

In[]:= BinaryWrite["/home/dennis/tmp/plain2.txt", Flatten[encrypted]]

In[]:= Close["/home/dennis/tmp/plain2.txt"]
\end{lstlisting}

Et le texte brut est:

\begin{lstlisting}
Mr. Sherlock Holmes, who was usually very late in the mornings, save
upon those not infrequent occasions when he was up all night, was seated
at the breakfast table. I stood upon the hearth-rug and picked up the
stick which our visitor had left behind him the night before. It was a
fine, thick piece of wood, bulbous-headed, of the sort which is known as
a "Penang lawyer." Just under the head was a broad silver band nearly
an inch across. "To James Mortimer, M.R.C.S., from his friends of the
C.C.H.," was engraved upon it, with the date "1884." It was just such a
stick as the old-fashioned family practitioner used to carry--dignified,
solid, and reassuring.

"Well, Watson, what do you make of it?"

Holmes was sitting with his back to me, and I had given him no sign of
my occupation.

...
\end{lstlisting}

(La version complète de ce texte se trouve
\href{https://github.com/DennisYurichev/RE-for-beginners/blob/master/ff/XOR/mask_2/files/plain.txt}{ici}.)

Le texte semble correct.
Oui, j'ai créé cet exemple de toutes pièces et j'ai choisi un texte très connu de
Conan Doyle, mais c'est très proche de ce que j'ai eu à faire il y a quelques temps.

\subsubsection{Autres idées à envisager}

Si nous échouions avec le comptage des espaces, il y a d'autres idées à essayer:

\begin{itemize}

\item Prenons en considération le fait que les lettres minuscules sont plus fréquentes
que celles en majuscule.

\item Analyse des fréquences.

\item Il y a aussi une bonne technique pour détecter le langage d'un texte: les trigrammes.
Chaque langage possède des triplets de lettres fréquences, qui peuvent être \q{the}
et \q{tha} en anglais.
En lire plus à ce sujet:
\href{http://odur.let.rug.nl/~vannoord/TextCat/textcat.pdf}{N-Gram-Based Text Categorization},
\url{http://code.activestate.com/recipes/326576/}.
Fait suffisamment intéressant, la détection des trigrammes peut être utilisée lorsque
vous décryptez un texte chiffré progressivement, comme dans cet exemple (yous devez
juste tester les 3 caractères décryptez adjacents).

Pour les systèmes non-latin encodés en UTF-8, les choses peuvent être plus simples.
Par exemple, les textes en russe encodés en UTF-8 ont chaque octet intercalé avec
des octets 0xD0/0xD1.
C'est parce que les caractères cyrilliques sont situés dans le 4ème bloc de la table
Unicode.
D'autres systèmes d'écriture on leurs propres blocs.

\end{itemize}

}


\EN{% TODO png blur? too wide listings
% TODO separate section for Mathematica example
\section[Analyzing using information entropy]{Analyzing unknown binary files using information entropy}
\label{entropy}
\myindex{Entropy}

For the sake of simplification, I would say, information entropy is a measure, how tightly some piece of data can be compressed.
For example, it is usually not possible to compress already compressed archive file, so it has high entropy.
On the other hand, one megabyte of zero bytes can be compressed to a tiny output file.
Indeed, in plain English language, one million of zeros can be described just as ``resulting file is one million zero bytes''.
Compressed files are usually a list of instructions to decompressor, like this: ``put 1000 zeros, then 0x23 byte, then 0x45 byte, then put a block of size 10 bytes which we've seen 500 bytes back, etc.''

Texts written in natural languages are also can be compressed tightly, 
because natural languages has a lot of redundancy
(otherwise, a tiny typo will always lead to misunderstanding, 
like any toggled bit in compressed archive make decompression nearly impossible), 
some words are used very often, etc.
It's possible to drop some words and text will be still readable.

Code for CPUs is also can be compressed, because some ISA instructions are used much more often than others.
\myindex{x86!\Instructions!MOV}
\myindex{x86!\Instructions!PUSH}
\myindex{x86!\Instructions!CALL}
In x86, most used instructions are MOV/PUSH/CALL---indeed, most of the time, computer CPU is just shuffling data and switching between
levels of abstractions.
If to consider data shuffling as moving data between levels of abstractions, this is also part of switching.

Data compressors and encryptors tend to produce very high entropy results.
Good pseudorandom number generators also produce data which cannot be compressed 
(it is possible to measure their quality by this sign).

So, in other words, entropy is a measure which can help to probe unknown data block.

\subsection{Analyzing entropy in Mathematica}

(This part has been first appeared in my blog at 13-May-2015.
Some discussion: \url{https://news.ycombinator.com/item?id=9545276}.)

It is possible to slice some file by blocks, probe each and draw a graph a graph.
I did this in Wolfram Mathematica for demonstration and here is a source code (Mathematica 10):

\begin{lstlisting}[style=custommath]
(* loading the file *)
input=BinaryReadList["file.bin"];

(* setting block sizes *)
BlockSize=4096;BlockSizeToShow=256;

(* slice blocks by 4k *)
blocks=Partition[input,BlockSize];

(* how many blocks we've got? *)
Length[blocks]

(* calculate entropy for each block. 2 in Entropy[] (base) is set with the intention so Entropy[] 
function will produce the same results as Linux ent utility does *)
entropies=Map[N[Entropy[2,#]]&,blocks];

(* helper functions *)
fBlockToShow[input_,offset_]:=Take[input,{1+offset,1+offset+BlockSizeToShow}]
fToASCII[val_]:=FromCharacterCode[val,"PrintableASCII"]
fToHex[val_]:=IntegerString[val,16]
fPutASCIIWindow[data_]:=Framed[Grid[Partition[Map[fToASCII,data],16]]]
fPutHexWindow[data_]:=Framed[Grid[Partition[Map[fToHex,data],16],Alignment->Right]]

(* that will be the main knob here *)
{Slider[Dynamic[offset],{0,Length[input]-BlockSize,BlockSize}],Dynamic[BaseForm[offset,16]]}

(* main UI part *)
Dynamic[{ListLinePlot[entropies,GridLines->{{-1,offset/BlockSize,1}},Filling->Axis,AxesLabel->{"offset","entropy"}],
CurrentBlock=fBlockToShow[input,offset];
fPutHexWindow[CurrentBlock],
fPutASCIIWindow[CurrentBlock]}]
\end{lstlisting}

\subsubsection{GeoIP ISP database}

\myindex{GeoIP}
Let's start with the \href{https://www.maxmind.com/en/geoip-demo}{GeoIP} file (which assigns ISP to the block of IP addresses).
This binary file (\IT{GeoIPISP.dat}) has some tables (which are IP address ranges perhaps) plus some text blob at the end of the file
(containing ISP names).

When I load it to Mathematica, I see this:

\begin{figure}[H]
\centering
\myincludegraphics{ff/entropy/geoipisp1.png}
\end{figure}

There are two parts in graph: first is somewhat chaotic, second is more steady.

0 in horizontal axis in graph means lowest entropy (the data which can be compressed very tightly, \IT{ordered} in other words) 
and 8 is highest (cannot be compressed at all, \IT{chaotic} or \IT{random} in other words).
Why 0 and 8? 0 means 0 bits per byte (byte slot is not filled at all) 
and 8 means 8 bits per byte, i.e., the whole byte slot is filled with the information tightly.

So I put slider to point in the middle of the first block, and I clearly see some array of 32-bit integers.
Now I put slider in the middle of the second block and I see English text:

\begin{figure}[H]
\centering
\myincludegraphics{ff/entropy/geoipisp2.png}
\end{figure}

Indeed, this are names of ISPs.
So, entropy of English text is 4.5-5.5 bits per byte? Yes, something like this.
Wolfram Mathematica has some well-known English literature corpus embedded, and we can see entropy of Shakespeare's sonnets:

\begin{lstlisting}[style=custommath]
In[]:= Entropy[2,ExampleData[{"Text","ShakespearesSonnets"}]]//N
Out[]= 4.42366
\end{lstlisting}

4.4 is close to what we've got (4.7-5.3). 
Of course, classic English literature texts are somewhat different from ISP names and other English texts we can find in binary files 
(debugging/logging/error messages), but this value is close.

\subsubsection{TP-Link WR941 firmware}

Now advanced example. I've got firmware for TP-Link WR941 router:

\begin{figure}[H]
\centering
\includegraphics[width=0.6\textwidth]{ff/entropy/tplink.png}
\end{figure}

Wee see here 3 blocks with empty lacunas.
The first block (started at address 0) is small, second (address somewhere at 0x22000) is bigger and third (address 0x123000) is biggest.
I can't be sure about exact entropy of the first block, but 2nd and 3rd has very high entropy, meaning that these blocks are either
compressed and/or encrypted.

\myindex{Binwalk}
I tried \href{http://binwalk.org/}{binwalk} for this firmware file:

\begin{lstlisting}
DECIMAL       HEXADECIMAL     DESCRIPTION
--------------------------------------------------------------------------------
0             0x0             TP-Link firmware header, firmware version: 0.-15221.3, image version: "", product ID: 0x0, product version: 155254789, kernel load address: 0x0, kernel entry point: 0x-7FFFE000, kernel offset: 4063744, kernel length: 512, rootfs offset: 837431, rootfs length: 1048576, bootloader offset: 2883584, bootloader length: 0
14832         0x39F0          U-Boot version string, "U-Boot 1.1.4 (Jun 27 2014 - 14:56:49)"
14880         0x3A20          CRC32 polynomial table, big endian
16176         0x3F30          uImage header, header size: 64 bytes, header CRC: 0x3AC66E95, created: 2014-06-27 06:56:50, image size: 34587 bytes, Data Address: 0x80010000, Entry Point: 0x80010000, data CRC: 0xDF2DBA0B, OS: Linux, CPU: MIPS, image type: Firmware Image, compression type: lzma, image name: "u-boot image"
16240         0x3F70          LZMA compressed data, properties: 0x5D, dictionary size: 33554432 bytes, uncompressed size: 90000 bytes
131584        0x20200         TP-Link firmware header, firmware version: 0.0.3, image version: "", product ID: 0x0, product version: 155254789, kernel load address: 0x0, kernel entry point: 0x-7FFFE000, kernel offset: 3932160, kernel length: 512, rootfs offset: 837431, rootfs length: 1048576, bootloader offset: 2883584, bootloader length: 0
132096        0x20400         LZMA compressed data, properties: 0x5D, dictionary size: 33554432 bytes, uncompressed size: 2388212 bytes
1180160       0x120200        Squashfs filesystem, little endian, version 4.0, compression:lzma, size: 2548511 bytes, 536 inodes, blocksize: 131072 bytes, created: 2014-06-27 07:06:52
\end{lstlisting}

\myindex{LZMA}
Indeed: there are some stuff at the beginning, but two large LZMA compressed blocks are started at 0x20400 and 0x120200.
These are roughly addresses we have seen in Mathematica.
Oh, and by the way, binwalk can show entropy information as well (-E option):

\begin{lstlisting}
DECIMAL       HEXADECIMAL     ENTROPY
--------------------------------------------------------------------------------
0             0x0             Falling entropy edge (0.419187)
16384         0x4000          Rising entropy edge (0.988639)
51200         0xC800          Falling entropy edge (0.000000)
133120        0x20800         Rising entropy edge (0.987596)
968704        0xEC800         Falling entropy edge (0.508720)
1181696       0x120800        Rising entropy edge (0.989615)
3727360       0x38E000        Falling entropy edge (0.732390)
\end{lstlisting}

Rising edges are corresponding to rising edges of block on our graph.
Falling edges are the points where empty lacunas are started.

I wasn't able to force binwalk to generate PNG graphs (due to absence of some Python library), but here is an example how binwalk
can do them: \url{http://binwalk.org/wp-content/uploads/2013/12/lg_dtv.png}.
% FIXME inline picture

What can we say about lacunas? By looking in hex editor, we see that these are just filled with 0xFF bytes.
Why developers put them? Perhaps, because they weren't able to calculate precise compressed blocks sizes, so they allocated space
for them with some reserve.

\subsubsection{Notepad}

\myindex{Notepad}

Another example is notepad.exe I've picked in Windows 8.1:

\begin{figure}[H]
\centering
\myincludegraphics{ff/entropy/notepad1.png}
\end{figure}

There is cavity at ~0x19000 (absolute file offset).
I opened the executable file in hex editor and found imports table there (which has lower entropy than x86-64 code
in the first half of graph).

There are also high entropy block started at ~0x20000:

\begin{figure}[H]
\centering
\myincludegraphics{ff/entropy/notepad2.png}
\end{figure}

\myindex{PNG}
In hex editor I can see PNG file here, embedded in the PE file resource section (it is a large image of notepad icon).
PNG files are compressed, indeed.

\subsubsection{Unnamed dashcam}

Now the most advanced example in this article is the firmware of some unnamed dashcam I've received from friend:

\begin{figure}[H]
\centering
\myincludegraphics{ff/entropy/dashcam_text.png}
\end{figure}

The cavity at the very beginning is an English text: debugging messages.
\myindex{MIPS}
I checked various ISAs and I found that 
the first third of the whole file (with the text segment inside) is in fact MIPS (little-endian) code!

For instance, this is very distinctive MIPS function epilogue:

\begin{lstlisting}[style=customasmMIPS]
ROM:000013B0                 move    $sp, $fp
ROM:000013B4                 lw      $ra, 0x1C($sp)
ROM:000013B8                 lw      $fp, 0x18($sp)
ROM:000013BC                 lw      $s1, 0x14($sp)
ROM:000013C0                 lw      $s0, 0x10($sp)
ROM:000013C4                 jr      $ra
ROM:000013C8                 addiu   $sp, 0x20
\end{lstlisting}

From our graph we can see that MIPS code has entropy of 5-6 bits per byte.
Indeed, I once measured various ISAs entropy and I've got these values:

\begin{itemize}
\item x86: .text section of ntoskrnl.exe file from Windows 2003: 6.6
\item x64: .text section of ntoskrnl.exe file from Windows 7 x64: 6.5
\item ARM (thumb mode), Angry Birds Classic: 7.05
\item ARM (ARM mode) Linux Kernel 3.8.0: 6.03
\item MIPS (little endian), .text section of user32.dll from Windows NT 4: 6.09
\end{itemize}

So the entropy of executable code is higher than of English text, but still can be compressed.

Now the second third is started at 0xF5000. I don't know what this is. I tried different ISAs but without success.
The entropy of the block is looks even steadier than for executable one.
Maybe some kind of data?

\myindex{JPEG}
There is also a spike at ~0x213000. I checked it in hex editor and I found JPEG file there 
(which, of course, compressed)!
I also don't know what is at the end.
Let's try Binwalk for this file:

\begin{lstlisting}
dennis@ubuntu:~/P/entropy$ binwalk FW96650A.bin 

DECIMAL       HEXADECIMAL     DESCRIPTION
--------------------------------------------------------------------------------
167698        0x28F12         Unix path: /15/20/24/25/30/60/120/240fps can be served..
280286        0x446DE         Copyright string: "Copyright (c) 2012 Novatek Microelectronic Corp."
2169199       0x21196F        JPEG image data, JFIF standard 1.01
2300847       0x231BAF        MySQL MISAM compressed data file Version 3

dennis@ubuntu:~/P/entropy$ binwalk -E FW96650A.bin 

WARNING: pyqtgraph not found, visual entropy graphing will be disabled

DECIMAL       HEXADECIMAL     ENTROPY
--------------------------------------------------------------------------------
0             0x0             Falling entropy edge (0.579792)
2170880       0x212000        Rising entropy edge (0.967373)
2267136       0x229800        Falling entropy edge (0.802974)
2426880       0x250800        Falling entropy edge (0.846639)
2490368       0x260000        Falling entropy edge (0.849804)
2560000       0x271000        Rising entropy edge (0.974340)
2574336       0x274800        Rising entropy edge (0.970958)
2588672       0x278000        Falling entropy edge (0.763507)
2592768       0x279000        Rising entropy edge (0.951883)
2596864       0x27A000        Falling entropy edge (0.712814)
2600960       0x27B000        Rising entropy edge (0.968167)
2607104       0x27C800        Rising entropy edge (0.958582)
2609152       0x27D000        Falling entropy edge (0.760989)
2654208       0x288000        Rising entropy edge (0.954127)
2670592       0x28C000        Rising entropy edge (0.967883)
2676736       0x28D800        Rising entropy edge (0.975779)
2684928       0x28F800        Falling entropy edge (0.744369)
\end{lstlisting}

Yes, it found JPEG file and even MySQL data!
But I'm not sure if it's true---I didn't check it yet.

\myindex{clusterization}
It's also interesting to try clusterization in Mathematica:

\begin{figure}[H]
\centering
\myincludegraphics{ff/entropy/dashcam_clusters.png}
\end{figure}

Here is an example of how Mathematica grouped various entropy values into distinctive groups.
Indeed, there is something credible. Blue dots in range of 5.0-5.5 are supposedly related to English text.
Yellow dots in 5.5-6 are MIPS code. A lot of green dots in 6.0-6.5 is the unknown second third.
Orange dots close to 8.0 are related to compressed JPEG file.
Other orange dots are supposedly related to the end of the firmware (unknown to us data).

\subsubsection{Links}

Binary files used while writing: \url{http://yurichev.com/blog/entropy/files/}.
Wolfram Mathematica notebook file: \url{http://yurichev.com/blog/entropy/files/binary_file_entropy.nb}
(all cells must be evaluated to start things working).



\subsection{Conclusion}

Information entropy can be used as a quick-n-dirty method for inspecting unknown binary files.
In particular, it is a very quick way to find compressed/encrypted pieces of data.
Someone say it's possible to find RSA (and other asymmetric cryptographic algorithms) public/private keys 
in executable code (which has high entropy as well), but I didn't try this myself.

\subsection{Tools}

Handy Linux \IT{ent} utility to measure entropy of a file\footnote{\url{http://www.fourmilab.ch/random/}}.

There is a great online entropy visualizer made by Aldo Cortesi, 
which I tried to mimic using Mathematica: \url{http://binvis.io}.
His articles about entropy visualization are worth reading:
\url{http://corte.si/posts/visualisation/entropy/index.html},
\url{http://corte.si/posts/visualisation/malware/index.html},
\url{http://corte.si/posts/visualisation/binvis/index.html}.

\myindex{radare2}
radare2 framework has \IT{\#entropy} command for this.

A tool for IDA: IDAtropy\footnote{\url{https://github.com/danigargu/IDAtropy}}.

\subsection{A word about primitive encryption like XORing}

It's interesting that simple XOR encryption doesn't affect entropy of data.
I've shown this in \IT{Norton Guide} example in the book (\myref{norton_guide}).

Generalizing: encryption by substitution cipher also doesn't affect entropy of data (and XOR can be viewed as substitution cipher).
The reason of that is because entropy calculation algorithm view data on byte-level.
On the other hand, the data encrypted by 2 or 4-byte XOR pattern will result in another entropy.

Nevertheless, low entropy is usually a good sign of weak amateur cryptography
(which is also used in license keys, license files, etc.).

\subsection{More about entropy of executable code}

It is quickly noticeable that probably a biggest source of high-entropy in executable code are relative offsets encoded in opcodes.
For example, these two consequent instructions will produce different relative offsets in their opcodes, 
while they are in fact pointing to the same function:

\begin{lstlisting}[style=customasmx86]
function proc
...
function endp

...

CALL function
...
CALL function
\end{lstlisting}

Ideal executable code compressor would encode information like this:
\IT{there is a CALL to a ``function'' at address X and the same CALL at address Y} without necessity to encode
address of the \IT{function} twice.

\myindex{UPX}
To deal with this, executable compressors are sometimes able to reduce entropy here.
One example is UPX: \url{http://sourceforge.net/p/upx/code/ci/default/tree/doc/filter.txt}.

\subsection{Random number generators}

\myindex{GnuPG}
When I run GnuPG to generate new secret key, it asking for some entropy...

\begin{lstlisting}
We need to generate a lot of random bytes. It is a good idea to perform
some other action (type on the keyboard, move the mouse, utilize the
disks) during the prime generation; this gives the random number
generator a better chance to gain enough entropy.

Not enough random bytes available.  Please do some other work to give
the OS a chance to collect more entropy! (Need 169 more bytes)
\end{lstlisting}

This means that good a PRNG produces long high-entropy results, and this is what the secret asymetrical cryptographical key needs.
But \ac{CPRNG} is tricky (because computer is highly deterministic device itself),
so the GnuPG asking for some additional randomness from the user.

Here is a case where I made attempt to calculate entropy of some unknown blocks: \myref{encrypted_DB1}.

\subsection{Entropy of various files}

Entropy of random bytes is close to 8:

\begin{lstlisting}[basicstyle=\ttfamily, mathescape]
$\$$ dd bs=1M count=1 if=/dev/urandom | ent
Entropy = 7.999803 bits per byte.
\end{lstlisting}

This means, almost all available space inside of byte is filled with information.

256 bytes in range of 0..255 gives exact value of 8:

\begin{lstlisting}[basicstyle=\ttfamily, mathescape,style=custompy]
#!/usr/bin/env python
import sys

for i in range(256):
    sys.stdout.write(chr(i))
\end{lstlisting}

\begin{lstlisting}[basicstyle=\ttfamily, mathescape]
python 1.py | ent
Entropy = 8.000000 bits per byte.
\end{lstlisting}

Order of bytes doesn't matter.
This means, all available space inside of byte is filled.

Entropy of all zero bytes is 0:

\begin{lstlisting}[basicstyle=\ttfamily, mathescape]
$\$$ dd bs=1M count=1 if=/dev/zero | ent
Entropy = 0.000000 bits per byte.
\end{lstlisting}

Entropy of a string costisting of a single (any) byte is 0:

\begin{lstlisting}[basicstyle=\ttfamily, mathescape]
$\$$ echo -n "aaaaaaaaaaaaaaaaaaa" | ent
Entropy = 0.000000 bits per byte.
\end{lstlisting}

\myindex{base64}
Entropy of base64 string is the same as entropy of source data, but multiplied by $\frac{3}{4}$.
This is because base64 encoding uses 64 symbols instead of 256.

\begin{lstlisting}[basicstyle=\ttfamily, mathescape]
$\$$ dd bs=1M count=1 if=/dev/urandom | base64 | ent
Entropy = 6.022068 bits per byte.
\end{lstlisting}

Perhaps, 6.02 not that close to 6 because padding symbols (\TT{=}) spoils our statistics for a little.

\myindex{Uuencode}
Uuencode, also uses 64 symbols:

\begin{lstlisting}[basicstyle=\ttfamily, mathescape]
$\$$ dd bs=1M count=1 if=/dev/urandom | uuencode - | ent
Entropy = 6.013162 bits per byte.
\end{lstlisting}

This means, any base64 and Uuencode strings can be transmitted using 6-bit bytes or characters.

Any random information in hexadecimal form has entropy of 4 bits per byte:

\begin{lstlisting}[basicstyle=\ttfamily, mathescape]
$\$$ openssl rand -hex $\$$(( 2**16 )) | ent
Entropy = 4.000013 bits per byte.
\end{lstlisting}

Entropy of randomly picked English language text from Gutenberg library has entropy ~4.5.
The reason of this is because English texts uses mostly 26 symbols, and $log_2(26)=~4.7$, i.e., you would need
5-bit bytes to transmit uncompressed English texts, that would be enough (it was indeed so in teletype era).

Randomly chosen Russian language text from \url{http://lib.ru}
library is F.M.Dostoevsky ``Idiot''\footnote{\url{http://az.lib.ru/d/dostoewskij_f_m/text_0070.shtml}},
internally encoded in CP1251 encoding.

And this file has entropy of ~4.98.
Russian language has 33 characters, and $log_2(33)=~5.04$.
But it has unpopular and rare ``ё'' character
% FIXME YO letter isn't rendered in Eng version
\footnote{When I typed it here in text, I've needed to look down to my keyboard and find it.}.
And $log_2(32)=5$ (Russian alphabet without this rare character) --- now this close to what we've got.

In fact, the text we studying uses ``ё'' letter, but, probably, it's still rarely used there.

\myindex{UTF-8}
The very same file transcoded from CP1251 to UTF-8 gave entropy of ~4.23.
Each Cyrillic character encoded in UTF-8 is usually encoded as a pair,
and the first byte is always one of: 0xD0 or 0xD1.
Perhaps, this caused bias.

Let's generate random bits and output them as ``T'' or ``F'' characters:

\begin{lstlisting}[style=custompy]
#!/usr/bin/env python
import random, sys

rt=""
for i in range(102400):
    if random.randint(0,1)==1:
        rt=rt+"T"
    else:
        rt=rt+"F"
print rt
\end{lstlisting}

Sample: \TT{...TTTFTFTTTFFFTTTFTTTTTTFTTFFTTTFTFTTFTTFFFFFF...}.

Entropy is very close to 1 (i.e., 1 bit per byte).

Let's generate random decimal numbers:

\begin{lstlisting}[style=custompy]
#!/usr/bin/env python
import random, sys

rt=""
for i in range(102400):
    rt=rt+"%d" % random.randint(0,9)
print rt
\end{lstlisting}

Sample: \TT{...52203466119390328807552582367031963888032...}.

Entropy will be close to 3.32, indeed, this is $log_2(10)$.

\subsection{Making lower level of entropy}

The author of these lines once saw a software which stored each byte of encrypted data in 3 bytes:
each has roughly $\frac{byte}{3}$ value, so reconstructing encrypted byte back involving summing up 3 consecutive bytes.
Looks absurdly.

But some people say this was done in order to conceal the very fact
the data has something encrypted inside: measuring entropy of such block will show much lower level of it.

}
\EN{\mysection{Millenium game save file}
\label{Millenium_DOS_game}
\myindex{MS-DOS}

The \q{Millenium Return to Earth} 
is an ancient DOS game (1991), that allows you to mine resources, build ships,
equip them and send them on other planets, and so on\footnote{It can be downloaded for free
\href{http://go.yurichev.com/17316}{here}}.

Like many other games, it allows you to save all game state into a file.

Let's see if we can find something in it.

\clearpage
So there is a mine in the game.
Mines at some planets 
work faster, or slower on others. 
The set of resources is also different.

Here we can see what resources are mined at the time: 

\begin{figure}[H]
\centering
\myincludegraphics{ff/millenium/1.png}
\caption{Mine: state 1}
\label{fig:mill_1}
\end{figure}

Let's save a game state.
This is a file of size 9538 bytes.

Let's wait some \q{days} here in the game, and now we've got more resources from the mine:

\begin{figure}[H]
\centering
\myincludegraphics{ff/millenium/2.png}
\caption{Mine: state 2}
\label{fig:mill_2}
\end{figure}

Let's save game state again.

Now let's try to just do binary comparison of the save files using the simple DOS/Windows FC utility:

\lstinputlisting{ff/millenium/fc_result.txt}

The output is incomplete here, there are more differences, but we will cut result to show the most interesting.

In the first state, we have 14 \q{units} of hydrogen and 102 \q{units} of oxygen.

We have 22 and 155 \q{units} respectively in the second state.
If these values are saved into 
the save file, we would see this in the difference.
And indeed we do. 
There is 0x0E (14) at position 0xBDA and this value is 
0x16 (22) in the new version of the file.
This is probably hydrogen.
There is 0x66 (102) at position 0xBDC in the old 
version and 0x9B (155) in the new version of the file. 
This seems to be the oxygen.

Both files are available on the website for those who wants to inspect them (or experiment) more: 
\href{http://go.yurichev.com/17212}{beginners.re}.

\clearpage
Here is the new version of file opened in Hiew, we marked the values related to the resources mined in the game: 

\begin{figure}[H]
\centering
\myincludegraphics{ff/millenium/hiew3.png}
\caption{Hiew: state 1}
\label{fig:mill_hiew3}
\end{figure}

Let's check each of them.

These are clearly 16-bit values: not a strange thing for 16-bit DOS software where the \Tint type has 16-bit width.

\clearpage
Let's check our assumptions.
We will write the 1234 (0x4D2) value at the first position (this must be hydrogen):

\begin{figure}[H]
\centering
\myincludegraphics{ff/millenium/hiew4.png}
\caption{Hiew: let's write 1234 (0x4D2) there}
\label{fig:mill_hiew4}
\end{figure}

Then we will load the changed file in the game and took a look at mine statistics:

\begin{figure}[H]
\centering
\myincludegraphics{ff/millenium/5.png}
\caption{Let's check for hydrogen value}
\label{fig:mill_5}
\end{figure}

So yes, this is it.

\clearpage
Now let's try to 
finish the game as soon as possible, set the maximal values everywhere:

\begin{figure}[H]
\centering
\myincludegraphics{ff/millenium/hiew7.png}
\caption{Hiew: let's set maximal values}
\label{fig:mill_hiew7}
\end{figure}

0xFFFF is 65535, so yes, we now have a 
lot of resources:

\begin{figure}[H]
\centering
\myincludegraphics{ff/millenium/6.png}
\caption{All resources are 65535 (0xFFFF) indeed}
\label{fig:mill_6}
\end{figure}

\clearpage
Let's skip some \q{days} in the game and oops! 
We have a lower amount of some resources:

\begin{figure}[H]
\centering
\myincludegraphics{ff/millenium/8.png}
\caption{Resource variables overflow}
\label{fig:mill_8}
\end{figure}

That's just overflow. 

The game's developer supposedly didn't think about such high amounts of resources,
so there are probably no overflow checks, but the mine is \q{working} in the game, resources are added,
hence the overflows.
Apparently, it is a bad idea to be that greedy.

There are probably a lot of more values 
saved in this file.

So this is very simple method of cheating in games.
High score files often can be easily 
patched like that.

More about files and memory snapshots comparing: 
\myref{snapshots_comparing}.
}\RU{\mysection{Файл сохранения состояния в игре Millenium}
\label{Millenium_DOS_game}
\myindex{MS-DOS}

Игра \q{Millenium Return to Earth} под DOS довольно древняя (1991), позволяющая
добывать ресурсы, строить корабли, снаряжать их на другие планеты, итд.
\footnote{Её можно скачать бесплатно
\href{http://go.yurichev.com/17316}{здесь}}.

Как и многие другие игры, она позволяет сохранять состояние игры в файл.

Посмотрим, сможем ли мы найти что-нибудь в нем.

\clearpage
В игре есть шахта.
Шахты на некоторых планетах работают быстрее, на некоторых других --- медленнее. 
Набор ресурсов также разный.

Здесь видно, какие ресурсы добыты в этот момент: 

\begin{figure}[H]
\centering
\myincludegraphics{ff/millenium/1.png}
\caption{Шахта: первое состояние}
\label{fig:mill_1}
\end{figure}

Сохраним состояние игры.
Это файл размером 9538 байт.

Подождем несколько \q{дней} здесь в игре и теперь в шахте добыто больше ресурсов:

\begin{figure}[H]
\centering
\myincludegraphics{ff/millenium/2.png}
\caption{Шахта: второе состояние}
\label{fig:mill_2}
\end{figure}

Снова сохраним состояние игры.

Теперь просто попробуем сравнить оба файла побайтово используя простую утилиту FC под DOS/Windows:

\lstinputlisting{ff/millenium/fc_result.txt}

Вывод здесь неполный, там было больше отличий, но мы обрежем результат до самого интересного.

В первой версии у нас было 14 единиц водорода (hydrogen) и 102 --- кислорода (oxygen).

Во второй версии у нас 22 и 155 единиц соответственно.

Если эти значения сохраняются в файл, мы должны увидеть разницу.
И она действительно есть. 
Там 0x0E (14) на позиции 0xBDA и это значение 0x16 (22) в новой версии файла.
Это, наверное, водород.
Там также 0x66 (102) на позиции 0xBDC в старой версии и 0x9B (155) в новой версии файла. 
Это, наверное, кислород.

Обе версии файла доступны на сайте, для тех кто хочет их изучить (или поэкспериментировать): 
\href{http://go.yurichev.com/17212}{beginners.re}.

\clearpage
Новую версию файла откроем в Hiew и отметим значения, связанные с ресурсами, добытыми на шахте в игре: 

\begin{figure}[H]
\centering
\myincludegraphics{ff/millenium/hiew3.png}
\caption{Hiew: первое состояние}
\label{fig:mill_hiew3}
\end{figure}

Проверим каждое.
Это явно 16-битные значения: не удивительно для 16-битной программы под DOS, где \Tint имел длину в 16 бит.

\clearpage
Проверим наши предположения.
Запишем 1234 (0x4D2) на первой позиции (это должен быть водород):

\begin{figure}[H]
\centering
\myincludegraphics{ff/millenium/hiew4.png}
\caption{Hiew: запишем там (0x4D2)}
\label{fig:mill_hiew4}
\end{figure}

Затем загрузим измененный файл в игру и посмотрим на статистику в шахте:

\begin{figure}[H]
\centering
\myincludegraphics{ff/millenium/5.png}
\caption{Проверим значение водорода}
\label{fig:mill_5}
\end{figure}

Так что да, это оно.

\clearpage
Попробуем пройти игру как можно быстрее, установим максимальные значения везде:

\begin{figure}[H]
\centering
\myincludegraphics{ff/millenium/hiew7.png}
\caption{Hiew: установим максимальные значения}
\label{fig:mill_hiew7}
\end{figure}

0xFFFF это 65535, так что да, у нас много ресурсов теперь:

\begin{figure}[H]
\centering
\myincludegraphics{ff/millenium/6.png}
\caption{Все ресурсы теперь действительно 65535 (0xFFFF)}
\label{fig:mill_6}
\end{figure}

\clearpage
Пропустим еще несколько \q{дней} в игре и видим что-то неладное! 
Некоторых ресурсов стало меньше:

\begin{figure}[H]
\centering
\myincludegraphics{ff/millenium/8.png}
\caption{Переполнение переменных ресурсов}
\label{fig:mill_8}
\end{figure}

Это просто переполнение. 
Разработчик игры, должно быть, никогда не думал, что значения ресурсов будут такими большими,
так что, здесь, наверное, нет проверок на переполнение, но шахта в игре \q{работает}, ресурсы добавляются,
отсюда и переполнение.

Вероятно, не нужно было жадничать.

Здесь наверняка еще какие-то значения в этом файле.

Так что это очень простой способ читинга в играх.
Файл с таблицей очков также можно легко модифицировать.

Еще насчет сравнения файлов и снимков памяти: \myref{snapshots_comparing}.
}
\EN{\subsection{Fall-through}

Another popular usage of \TT{switch()} operator is so-called \q{fallthrough}.
Here is simple example\footnote{Copypasted from \url{https://github.com/azonalon/prgraas/blob/master/prog1lib/lecture_examples/is_whitespace.c}}:

\lstinputlisting[numbers=left,style=customc]{patterns/08_switch/4_fallthrough/fallthrough1.c}

Slightly harder, from Linux kernel\footnote{Copypasted from \url{https://github.com/torvalds/linux/blob/master/drivers/media/dvb-frontends/lgdt3306a.c}}:

\lstinputlisting[numbers=left,style=customc]{patterns/08_switch/4_fallthrough/fallthrough2.c}

\lstinputlisting[caption=Optimizing GCC 5.4.0 x86,numbers=left,style=customasmx86]{patterns/08_switch/4_fallthrough/fallthrough2.s}

We can get to \TT{.L5} label if there is number 3250 at function's input.
But we can get to this label from the other side:
we see that there are no jumps between \printf call and \TT{.L5} label.

Now we can understand why \IT{switch()} statement is sometimes a source of bugs:
one forgotten \IT{break} will transform your
\IT{switch()} statement into \IT{fallthrough} one, and several blocks will be executed instead of single one.

}
\EN{\section{\oracle: .SYM-files}
\myindex{\oracle}
\label{Oracle_SYM_files_example}

When an \oracle process experiences some kind of crash, it writes a lot of information into log files,
including stack trace, like this:

\begin{lstlisting}
----- Call Stack Trace -----
calling              call     entry                argument values in hex      
location             type     point                (? means dubious value)     
-------------------- -------- -------------------- ----------------------------
_kqvrow()                     00000000             
_opifch2()+2729      CALLptr  00000000             23D4B914 E47F264 1F19AE2
                                                   EB1C8A8 1
_kpoal8()+2832       CALLrel  _opifch2()           89 5 EB1CC74
_opiodr()+1248       CALLreg  00000000             5E 1C EB1F0A0
_ttcpip()+1051       CALLreg  00000000             5E 1C EB1F0A0 0
_opitsk()+1404       CALL???  00000000             C96C040 5E EB1F0A0 0 EB1ED30
                                                   EB1F1CC 53E52E 0 EB1F1F8
_opiino()+980        CALLrel  _opitsk()            0 0
_opiodr()+1248       CALLreg  00000000             3C 4 EB1FBF4
_opidrv()+1201       CALLrel  _opiodr()            3C 4 EB1FBF4 0
_sou2o()+55          CALLrel  _opidrv()            3C 4 EB1FBF4
_opimai_real()+124   CALLrel  _sou2o()             EB1FC04 3C 4 EB1FBF4
_opimai()+125        CALLrel  _opimai_real()       2 EB1FC2C
_OracleThreadStart@  CALLrel  _opimai()            2 EB1FF6C 7C88A7F4 EB1FC34 0
4()+830                                            EB1FD04
77E6481C             CALLreg  00000000             E41FF9C 0 0 E41FF9C 0 EB1FFC4
00000000             CALL???  00000000             
\end{lstlisting}

But of course, \oracle's executables must have some kind of debug information or map files with symbol
information included or something like that.

Windows NT \oracle has symbol information in files with .SYM extension, but the format is proprietary.
(Plain text files are good, but needs additional parsing, hence offer slower access.)

Let's see if we can understand its format.

We will pick the shortest \TT{orawtc8.sym} file that comes with the \TT{orawtc8.dll} file in Oracle 8.1.7
\footnote{We can chose an ancient \oracle version intentionally due to the smaller size of its modules}.

\clearpage
Here is the file opened in Hiew:

\begin{figure}[H]
\centering
\myincludegraphics{ff/Oracle_SYM/whole1.png}
\caption{The whole file in Hiew}
\label{fig:oracle_SYM_whole1}
\end{figure}

By comparing the file with other .SYM files, we can quickly see that \TT{OSYM} is always header (and footer),
so this is maybe the file's signature.

We also see that basically, the file format is: OSYM + some binary data + zero delimited text strings + OSYM.
The strings are, obviously, function and global variable names.

\clearpage
We will mark the OSYM signatures and strings here: 

\begin{figure}[H]
\centering
\myincludegraphics{ff/Oracle_SYM/whole2.png}
\caption{OSYM signature and text strings}
\label{fig:oracle_SYM_whole2}
\end{figure}

Well, let's see. 
In Hiew, we will mark the whole strings block (except the trailing OSYM signatures) and put it into a separate file.
Then we run UNIX \IT{strings} and \IT{wc} utilities to count the text strings:

\begin{lstlisting}
strings strings_block | wc -l
66
\end{lstlisting}

So there are 66 text strings.
Please note that number.

We can say, in general, as a rule, the number of \IT{anything} is often stored separately in binary files.

It's indeed so, we can find the 66 value (0x42) at the file's start, right after the OSYM signature:

\lstinputlisting{ff/Oracle_SYM/dump1.txt}

Of course, 0x42 here is not a byte, but most likely a 32-bit value packed as little-endian, hence we see
0x42 and then at least 3 zero bytes.

Why do we believe it's 32-bit?
Because, \oracle's symbol 
files may be pretty big.

The oracle.sym file for the main oracle.exe (version 10.2.0.4) executable contains \TT{0x3A38E} (238478) symbols.
A 16-bit value isn't enough here.

We can check other .SYM files like this and it proves our guess: the value after the 32-bit OSYM signature always
reflects the number of text strings in the file.

It's a general feature of almost all binary files: a header with a signature plus some other information 
about the file.

Now let's investigate closer what this binary block is.

Using Hiew again, we put the block starting at address 8 (i.e., after the 32-bit \IT{count} value) 
ending at the strings block, into a separate binary file.

\clearpage
Let's see the binary block in Hiew:

\begin{figure}[H]
\centering
\myincludegraphics{ff/Oracle_SYM/binary1.png}
\caption{Binary block}
\label{fig:oracle_SYM_binary1}
\end{figure}

There is a clear pattern in it. 

\clearpage
We will add red lines to divide the block: 

\begin{figure}[H]
\centering
\myincludegraphics{ff/Oracle_SYM/binary2.png}
\caption{Binary block patterns}
\label{fig:oracle_SYM_binary2}
\end{figure}

Hiew, like almost any other hexadecimal editor, shows 16 bytes per line.
So the pattern is clearly visible: 
there are 4 32-bit values per line.

The pattern is visually visible because some values here (till address \TT{0x104}) 
are always in \TT{0x1000xxxx} form, 
started with 0x10 and zero bytes.

Other values (starting at \TT{0x108}) are in \TT{0x0000xxxx} form, so always started with two zero bytes.

Let's dump the block as an array of 32-bit values:

\lstinputlisting[caption=first column is address]{ff/Oracle_SYM/dump2.txt}

There are 132 values, that's 66*2.
Probably, there are two 32-bit values for each symbol, but maybe there are two arrays? 
Let's see.

Values starting with \TT{0x1000} may be addresses.

This is a .SYM file for a DLL after all, and the default base address of
win32 DLLs is \TT{0x10000000}, and the code usually starts at \TT{0x10001000}.

When we open the orawtc8.dll file in \IDA, the base address is different, but nevertheless, the first function is:

\begin{lstlisting}[style=customasmx86]
.text:60351000 sub_60351000    proc near
.text:60351000
.text:60351000 arg_0    = dword ptr  8
.text:60351000 arg_4    = dword ptr  0Ch
.text:60351000 arg_8    = dword ptr  10h
.text:60351000
.text:60351000          push    ebp
.text:60351001          mov     ebp, esp
.text:60351003          mov     eax, dword_60353014
.text:60351008          cmp     eax, 0FFFFFFFFh
.text:6035100B          jnz     short loc_6035104F
.text:6035100D          mov     ecx, hModule
.text:60351013          xor     eax, eax
.text:60351015          cmp     ecx, 0FFFFFFFFh
.text:60351018          mov     dword_60353014, eax
.text:6035101D          jnz     short loc_60351031
.text:6035101F          call    sub_603510F0
.text:60351024          mov     ecx, eax
.text:60351026          mov     eax, dword_60353014
.text:6035102B          mov     hModule, ecx
.text:60351031
.text:60351031 loc_60351031:    ; CODE XREF: sub_60351000+1D
.text:60351031          test    ecx, ecx
.text:60351033          jbe     short loc_6035104F
.text:60351035          push    offset ProcName ; "ax_reg"
.text:6035103A          push    ecx             ; hModule
.text:6035103B          call    ds:GetProcAddress
...
\end{lstlisting}

Wow, \q{ax\_reg} string sounds familiar. 

It's indeed the first string in the strings block!
So the name of this function seems to be \q{ax\_reg}.

The second function is:

\begin{lstlisting}[style=customasmx86]
.text:60351080 sub_60351080    proc near
.text:60351080
.text:60351080 arg_0    = dword ptr  8
.text:60351080 arg_4    = dword ptr  0Ch
.text:60351080
.text:60351080          push    ebp
.text:60351081          mov     ebp, esp
.text:60351083          mov     eax, dword_60353018
.text:60351088          cmp     eax, 0FFFFFFFFh
.text:6035108B          jnz     short loc_603510CF
.text:6035108D          mov     ecx, hModule
.text:60351093          xor     eax, eax
.text:60351095          cmp     ecx, 0FFFFFFFFh
.text:60351098          mov     dword_60353018, eax
.text:6035109D          jnz     short loc_603510B1
.text:6035109F          call    sub_603510F0
.text:603510A4          mov     ecx, eax
.text:603510A6          mov     eax, dword_60353018
.text:603510AB          mov     hModule, ecx
.text:603510B1
.text:603510B1 loc_603510B1:    ; CODE XREF: sub_60351080+1D
.text:603510B1          test    ecx, ecx
.text:603510B3          jbe     short loc_603510CF
.text:603510B5          push    offset aAx_unreg ; "ax_unreg"
.text:603510BA          push    ecx             ; hModule
.text:603510BB          call    ds:GetProcAddress
...
\end{lstlisting}

The \q{ax\_unreg} string is also the second string in the strings block!

The starting address of the second function is \TT{0x60351080}, and the second value in the binary 
block is \TT{10001080}.
So this is the address, 
but for a DLL with the default base address.

We can quickly check and be sure that the first 66 values in the array (i.e., the first half of the array) 
are just function addresses in the DLL, including some labels, \etc{}.
Well, what's the other part of array then? 
The other 66 values that start with \TT{0x0000}? 
These seem to be in range \TT{[0...0x3F8]}. 
And they do not look like bitfields: 
the series of numbers is increasing.

The last hexadecimal digit seems to be random, so, it's unlikely the address of something 
(it would be divisible by 4 or maybe 8 or 0x10 otherwise).

Let's ask ourselves: what else \oracle's developers would save here, in this file?

Quick wild guess: it could be the address of the text string (function name).

It can be quickly checked, and yes, each number is just the position of the first character in the strings block.

This is it! All done.

\myindex{IDA}
We will write an utility to convert these .SYM files into \IDA script, 
so we can load the .idc script and it sets the function names:

\lstinputlisting[style=customc]{ff/Oracle_SYM/unpacker.c}

Here is an example of its work:

\begin{lstlisting}[style=customc]
#include <idc.idc>

static main() {
	MakeName(0x60351000, "_ax_reg");
	MakeName(0x60351080, "_ax_unreg");
	MakeName(0x603510F0, "_loaddll");
	MakeName(0x60351150, "_wtcsrin0");
	MakeName(0x60351160, "_wtcsrin");
	MakeName(0x603511C0, "_wtcsrfre");
	MakeName(0x603511D0, "_wtclkm");
	MakeName(0x60351370, "_wtcstu");
...
}
\end{lstlisting}

The example files were used in this example are here: 
\href{http://go.yurichev.com/17216}{beginners.re}.

\clearpage
Oh, let's also try \oracle for win64.
There has to be 64-bit addresses instead, right?

The 8-byte pattern is visible even easier here:

\begin{figure}[H]
\centering
\myincludegraphics{ff/Oracle_SYM/whole64.png}
\caption{.SYM-file example from \oracle for win64}
\label{fig:oracle_SYM_whole64}
\end{figure}

So yes, all tables now have 64-bit elements, even string offsets!

The signature is now \TT{OSYMAM64}, to distinguish the target platform, apparently.

This is it!

Here is also library which has functions to access \oracle .SYM-files:
\href{http://go.yurichev.com/17007}{GitHub}.
}\RU{\section{\oracle: .SYM-файлы}
\myindex{\oracle}
\label{Oracle_SYM_files_example}

Когда процесс в \oracle терпит серьезную ошибку (crash), он записывает массу информации в лог-файлы,
включая состояние стека, вроде:

\begin{lstlisting}
----- Call Stack Trace -----
calling              call     entry                argument values in hex      
location             type     point                (? means dubious value)     
-------------------- -------- -------------------- ----------------------------
_kqvrow()                     00000000             
_opifch2()+2729      CALLptr  00000000             23D4B914 E47F264 1F19AE2
                                                   EB1C8A8 1
_kpoal8()+2832       CALLrel  _opifch2()           89 5 EB1CC74
_opiodr()+1248       CALLreg  00000000             5E 1C EB1F0A0
_ttcpip()+1051       CALLreg  00000000             5E 1C EB1F0A0 0
_opitsk()+1404       CALL???  00000000             C96C040 5E EB1F0A0 0 EB1ED30
                                                   EB1F1CC 53E52E 0 EB1F1F8
_opiino()+980        CALLrel  _opitsk()            0 0
_opiodr()+1248       CALLreg  00000000             3C 4 EB1FBF4
_opidrv()+1201       CALLrel  _opiodr()            3C 4 EB1FBF4 0
_sou2o()+55          CALLrel  _opidrv()            3C 4 EB1FBF4
_opimai_real()+124   CALLrel  _sou2o()             EB1FC04 3C 4 EB1FBF4
_opimai()+125        CALLrel  _opimai_real()       2 EB1FC2C
_OracleThreadStart@  CALLrel  _opimai()            2 EB1FF6C 7C88A7F4 EB1FC34 0
4()+830                                            EB1FD04
77E6481C             CALLreg  00000000             E41FF9C 0 0 E41FF9C 0 EB1FFC4
00000000             CALL???  00000000             
\end{lstlisting}

Но конечно, для этого исполняемые файлы \oracle должны содержать некоторую отладочную информацию,
либо map-файлы с информацией о символах или что-то в этом роде.

\oracle для Windows NT содержит информацию о символах в файлах с расширением .SYM, но его формат закрыт.

(Простые текстовые файлы --- это хорошо, но они требуют дополнительной обработки (парсинга), и из-за этого доступ
к ним медленнее.)

Посмотрим, сможем ли мы разобрать его формат.
Выберем самый короткий файл \TT{orawtc8.sym}, поставляемый с файлом \TT{orawtc8.dll} в Oracle 8.1.7
\footnote{Будем использовать древнюю версию \oracle сознательно, из-за более короткого размера его модулей}.

\clearpage
Вот я открываю этот файл в Hiew:

\begin{figure}[H]
\centering
\myincludegraphics{ff/Oracle_SYM/whole1.png}
\caption{Весь файл в Hiew}
\label{fig:oracle_SYM_whole1}
\end{figure}

Сравнивая этот файл с другими .SYM-файлами, мы можем быстро заметить, что \TT{OSYM} всегда является
заголовком (и концом), так что это, наверное, сигнатура файла.

Мы также видим, что в общем-то, формат файла это: OSYM + какие-то бинарные данные + 
текстовые строки разделенные нулем + OSYM.

Строки --- это, очевидно, имена функций и глобальных переменных.

\clearpage
Отметим сигнатуры OSYM и строки здесь: 

\begin{figure}[H]
\centering
\myincludegraphics{ff/Oracle_SYM/whole2.png}
\caption{Сигнатура OSYM и текстовые строки}
\label{fig:oracle_SYM_whole2}
\end{figure}

Посмотрим. 
В Hiew отметим весь блок со строками (исключая оконечивающую сигнатуру OSYM) и сохраним его в отдельный
файл.

Затем запустим UNIX-утилиты \IT{strings} и \IT{wc} для подсчета текстовых строк:%

\begin{lstlisting}
strings strings_block | wc -l
66
\end{lstlisting}

Так что здесь 66 текстовых строк.  Запомните это число.

Можно сказать, что в общем, как правило, количество \IT{чего-либо} часто сохраняется в бинарном
файле отдельно.

Это действительно так, мы можем найти значение 66 (0x42) в самом начале файла, прямо после сигнатуры OSYM:

\lstinputlisting{ff/Oracle_SYM/dump1.txt}

Конечно, 0x42 здесь это не байт, но скорее всего, 32-битное значение, запакованное как little-endian,
поэтому мы видим 0x42 и затем как минимум 3 байта.

Почему мы полагаем, что оно 32-битное?
Потому что файлы с символами в \oracle могут быть очень большими.
oracle.sym для главного исполняемого файла oracle.exe (версия 10.2.0.4) содержит \TT{0x3A38E} (238478) 
символов.

16-битного значения тут недостаточно.

Проверим другие .SYM-файлы как этот и это подтвердит нашу догадку: значение после 32-битной сигнатуры OSYM
всегда отражает количество текстовых строк в файле.

Это общая особенность почти всех бинарных файлов: заголовок с сигнатурой плюс некоторая дополнительная
информация о файле.

Рассмотрим бинарный блок поближе.
Снова используя Hiew, сохраним блок начиная с адреса 8 (т.е. после 32-битного значения,
отражающего количество) до блока со строками, в отдельный файл.%

\clearpage
Посмотрим этот блок в Hiew:

\begin{figure}[H]
\centering
\myincludegraphics{ff/Oracle_SYM/binary1.png}
\caption{Бинарный блок}
\label{fig:oracle_SYM_binary1}
\end{figure}

Тут явно есть какая-то структура. 

\clearpage
Добавим красные линии, чтобы разделить блок: 

\begin{figure}[H]
\centering
\myincludegraphics{ff/Oracle_SYM/binary2.png}
\caption{Структура бинарного блока}
\label{fig:oracle_SYM_binary2}
\end{figure}

Hiew, как и многие другие шестнадцатеричные редакторы, показывает 16 байт на строку.

Так что структура явно видна: здесь 4 32-битных значения на строку.

Эта структура видна визуально потому что некоторые значения здесь (вплоть до адреса \TT{0x104}) 
всегда в виде \TT{0x1000xxxx}, так что начинаются с байт 0x10 и 0.

Другие значения (начинающиеся на \TT{0x108}) всегда в виде \TT{0x0000xxxx}, так что начинаются с двух
нулевых байт.

Посмотрим на этот блок как на массив 32-битных значений:

\lstinputlisting[caption=первый столбец --- это адрес]{ff/Oracle_SYM/dump2.txt}

Здесь 132 значения, а это 66*2.
Может быть здесь 2 32-битных значения на каждый символ, а может быть здесь два массива? 
Посмотрим.

Значения, начинающиеся с \TT{0x1000} могут быть адресами.
В конце концов, этот .SYM-файл для DLL, а базовый адрес для DLL в win32 это \TT{0x10000000}, и сам код
обычно начинается по адресу \TT{0x10001000}.

Когда открываем файл orawtc8.dll в \IDA, базовый адрес другой, но тем не менее, первая функция это:

\begin{lstlisting}[style=customasmx86]
.text:60351000 sub_60351000    proc near
.text:60351000
.text:60351000 arg_0    = dword ptr  8
.text:60351000 arg_4    = dword ptr  0Ch
.text:60351000 arg_8    = dword ptr  10h
.text:60351000
.text:60351000          push    ebp
.text:60351001          mov     ebp, esp
.text:60351003          mov     eax, dword_60353014
.text:60351008          cmp     eax, 0FFFFFFFFh
.text:6035100B          jnz     short loc_6035104F
.text:6035100D          mov     ecx, hModule
.text:60351013          xor     eax, eax
.text:60351015          cmp     ecx, 0FFFFFFFFh
.text:60351018          mov     dword_60353014, eax
.text:6035101D          jnz     short loc_60351031
.text:6035101F          call    sub_603510F0
.text:60351024          mov     ecx, eax
.text:60351026          mov     eax, dword_60353014
.text:6035102B          mov     hModule, ecx
.text:60351031
.text:60351031 loc_60351031:    ; CODE XREF: sub_60351000+1D
.text:60351031          test    ecx, ecx
.text:60351033          jbe     short loc_6035104F
.text:60351035          push    offset ProcName ; "ax_reg"
.text:6035103A          push    ecx             ; hModule
.text:6035103B          call    ds:GetProcAddress
...
\end{lstlisting}

Ух ты, \q{ax\_reg} звучит знакомо. 
Действительно, это самая первая строка в блоке строк!

Так что имя этой функции, похоже \q{ax\_reg}.

Вторая функция:

\begin{lstlisting}[style=customasmx86]
.text:60351080 sub_60351080    proc near
.text:60351080
.text:60351080 arg_0    = dword ptr  8
.text:60351080 arg_4    = dword ptr  0Ch
.text:60351080
.text:60351080          push    ebp
.text:60351081          mov     ebp, esp
.text:60351083          mov     eax, dword_60353018
.text:60351088          cmp     eax, 0FFFFFFFFh
.text:6035108B          jnz     short loc_603510CF
.text:6035108D          mov     ecx, hModule
.text:60351093          xor     eax, eax
.text:60351095          cmp     ecx, 0FFFFFFFFh
.text:60351098          mov     dword_60353018, eax
.text:6035109D          jnz     short loc_603510B1
.text:6035109F          call    sub_603510F0
.text:603510A4          mov     ecx, eax
.text:603510A6          mov     eax, dword_60353018
.text:603510AB          mov     hModule, ecx
.text:603510B1
.text:603510B1 loc_603510B1:    ; CODE XREF: sub_60351080+1D
.text:603510B1          test    ecx, ecx
.text:603510B3          jbe     short loc_603510CF
.text:603510B5          push    offset aAx_unreg ; "ax_unreg"
.text:603510BA          push    ecx             ; hModule
.text:603510BB          call    ds:GetProcAddress
...
\end{lstlisting}

Строка \q{ax\_unreg} также это вторая строка в строке блок!

Адрес начала второй функции это \TT{0x60351080}, а второе значение в бинарном блоке это \TT{10001080}.

Так что это адрес, но для DLL с базовым адресом по умолчанию.

Мы можем быстро проверить и убедиться, что первые 66 значений в массиве (т.е. первая половина)
это просто адреса функций в DLL, включая некоторые метки, \etc{}.

Хорошо, что же тогда остальная часть массива? 
Остальные 66 значений, начинающиеся с \TT{0x0000}? 
Они похоже в пределах \TT{[0...0x3F8]}. 
И не похоже, что это битовые поля: ряд чисел возрастает.
Последняя шестнадцатеричная цифра выглядит как случайная, так что, не похоже, что это
адрес чего-либо (в противном случае, он бы делился, может быть, на 4 или 8 или 0x10).

Спросим себя: что еще разработчики \oracle хранили бы здесь, в этом файле?

Случайная догадка: это может быть адрес текстовой строки (название функции).

Это можно легко проверить, и да, каждое число --- это просто позиция первого символа в блоке строк.

Вот и всё! Всё закончено.

\myindex{IDA}
Напишем утилиту для конвертирования .SYM-файлов в \IDA-скрипт, 
так что сможем загружать .idc-скрипт и он выставит имена функций:

\lstinputlisting[style=customc]{ff/Oracle_SYM/unpacker.c}

Пример его работы:

\begin{lstlisting}[style=customc]
#include <idc.idc>

static main() {
	MakeName(0x60351000, "_ax_reg");
	MakeName(0x60351080, "_ax_unreg");
	MakeName(0x603510F0, "_loaddll");
	MakeName(0x60351150, "_wtcsrin0");
	MakeName(0x60351160, "_wtcsrin");
	MakeName(0x603511C0, "_wtcsrfre");
	MakeName(0x603511D0, "_wtclkm");
	MakeName(0x60351370, "_wtcstu");
...
}
\end{lstlisting}

Файлы, использованные в этом примере, здесь: \href{http://go.yurichev.com/17216}{beginners.re}.

\clearpage
О, можно еще попробовать \oracle для win64.
Там ведь должны быть 64-битные адреса, верно?

8-байтная структура здесь видна даже еще лучше:

\begin{figure}[H]
\centering
\myincludegraphics{ff/Oracle_SYM/whole64.png}
\caption{пример .SYM-файла из \oracle для win64}
\label{fig:oracle_SYM_whole64}
\end{figure}

Так что да, все таблицы здесь имеют 64-битные элементы, даже смещения строк!

Сигнатура теперь \TT{OSYMAM64}, чтобы отличить целевую платформу, очевидно.

Вот и всё!
Вот также библиотека в которой есть функция для доступа к .SYM-файлам \oracle{}:
\href{http://go.yurichev.com/17007}{GitHub}.
}
\EN{\section{\oracle: .MSB-files\ESph{}\PTBRph{}\PLph{}\ITAph{}\DEph{}\NLph{}}
\myindex{\oracle}
\epigraph{When working toward the solution of a problem, it always helps if you know the answer.}{Murphy's Laws, Rule of Accuracy}

This is a binary file that contains error messages with their corresponding numbers.
Let's try to understand 
its format and find a way to unpack it.

There are \oracle error message files in text form, 
so we can compare the text and packed binary files
\footnote{Open-source text files don't exist in \oracle for every .MSB file, so that's why we will work on their file format}.

This is the beginning of the ORAUS.MSG text file with some irrelevant comments stripped:

\begin{lstlisting}[caption=Beginning of ORAUS.MSG file without comments]
00000, 00000, "normal, successful completion"
00001, 00000, "unique constraint (%s.%s) violated"
00017, 00000, "session requested to set trace event"
00018, 00000, "maximum number of sessions exceeded"
00019, 00000, "maximum number of session licenses exceeded"
00020, 00000, "maximum number of processes (%s) exceeded"
00021, 00000, "session attached to some other process; cannot switch session"
00022, 00000, "invalid session ID; access denied"
00023, 00000, "session references process private memory; cannot detach session"
00024, 00000, "logins from more than one process not allowed in single-process mode"
00025, 00000, "failed to allocate %s"
00026, 00000, "missing or invalid session ID"
00027, 00000, "cannot kill current session"
00028, 00000, "your session has been killed"
00029, 00000, "session is not a user session"
00030, 00000, "User session ID does not exist."
00031, 00000, "session marked for kill"
...
\end{lstlisting}

The first number is the error code.
The second is perhaps maybe some additional flags.

\clearpage
Now let's open the ORAUS.MSB 
binary file and find these text strings. 
And there are:

\begin{figure}[H]
\centering
\myincludegraphics{ff/Oracle_MSB/1.png}
\caption{Hiew: first block}
\label{fig:oracle_MSB_1}
\end{figure}

We see the text strings (including those from the beginning of the ORAUS.MSG file) 
interleaved with some binary values.
By quick investigation, we can see that main part of the binary file is divided by blocks of 
size 0x200 (512) bytes.

\clearpage
Let's see the contents of the first block:

\begin{figure}[H]
\centering
\myincludegraphics{ff/Oracle_MSB/2.png}
\caption{Hiew: first block}
\label{fig:oracle_MSB_2}
\end{figure}

Here we see the texts of the first messages errors.
What we also see is 
that there are no zero bytes between the error messages.
This implies that these are not null-terminated C strings.
As a consequence, 
the length of each error message must be encoded somehow.
Let's also try to find the error numbers.
The ORAUS.MSG files starts with these: 
0, 1, 17 (0x11), 18 (0x12), 19 (0x13), 20 (0x14), 21 (0x15), 22 (0x16), 23 (0x17), 24 (0x18)...
We will find these numbers at the beginning 
of the block and mark them with red lines.
The period between error codes is 6 bytes.

This implies that there are probably 6 bytes of information allocated for each error message.

The first 16-bit value (0xA here or 10) means the number of messages in each block: this can be checked by investigating other blocks.
Indeed: the error messages have arbitrary size. 
Some are longer, some are shorter. 
But block size is always fixed, hence,
you never know how many text messages can be packed in each block.

As we already noted, since these are not null-terminated C strings, their size must be encoded somewhere.
The size of the first string \q{normal, successful completion} is 
29 (0x1D) bytes.
The size of the second string \q{unique constraint (\%s.\%s) violated} 
is 34 (0x22) bytes.
We can't find these values (0x1D or/and 0x22) in the block.%

There is also another thing.
\oracle 
has to determine the position of the string it needs to load in the block, right?
The first string \q{normal, successful completion} starts 
at  position 0x1444 (if we count starting at the beginning of the file) or at 0x44 (from the block's start).
The second string \q{unique constraint (\%s.\%s) violated} 
starts at position 0x1461 (from the
file's start) or at 0x61 (from the at the block's start).
These numbers (0x44 and 0x61) are familiar somehow! 
We can clearly see them at the start of the block.

So, each 6-byte block is:

\begin{itemize}
\item 16-bit error number; 
\item 16-bit zero (maybe additional flags); 
\item 16-bit starting position of 
the text string within the current block.
\end{itemize}

We can quickly check the other values and be sure our guess is correct.
And there is also the last \q{dummy} 6-byte block 
with an error number of zero and starting position beyond the last error message's last character.
Probably that's how text message length is 
determined?
We just enumerate 6-byte blocks to find the error number
we need, then we get the text string's position, then we get the position of the text string by looking at the next
6-byte block!
This way we determine the string's boundaries!
This method allows to 
save some space by not saving the text string's size in the file!

It's not possible to say it saves a lot of space, but it's a clever trick.

\clearpage
Let's back to the header of .MSB-file:

\begin{figure}[H]
\centering
\myincludegraphics{ff/Oracle_MSB/3.png}
\caption{Hiew: file header}
\label{fig:oracle_MSB_3}
\end{figure}

Now we can quickly find the number of blocks in the file (marked by red).
We can checked other .MSB-files and we see that it's true for all of them.

There are a lot of other values, but we will not investigate them, since our job (an unpacking utility) is done.

If we have to write a .MSB file packer, we would probably have to understand the meaning of the other values.

\clearpage
There is also a table that came after the header which probably contains 16-bit values:

\begin{figure}[H]
\centering
\myincludegraphics{ff/Oracle_MSB/4.png}
\caption{Hiew: last\_errnos table}
\label{fig:oracle_MSB_4}
\end{figure}

Their size can be determined visually (red lines are drawn here).

While dumping these values, we have found that each 16-bit number is the last error code for each block.

So that's how \oracle quickly finds the error message:

\begin{itemize}
\item load a table we will call last\_errnos (that contains the last error number for each block);

\item find a block that contains the error code we need, assuming all error codes 
increase across each block and across the file as well;

\item load the specific block;

\item enumerate the 6-byte structures until the specific error number is found;

\item get the position of the first character from the current 6-byte block;

\item get the position of the last character from the next 6-byte block;

\item load all characters of the message in this range.
\end{itemize}

This is C program that we wrote which unpacks .MSB-files:
\href{http://go.yurichev.com/17213}{beginners.re}.

There are also the two files which were used in the example 
(\oracle 11.1.0.6):
\href{http://go.yurichev.com/17214}{beginners.re},
\href{http://go.yurichev.com/17215}{beginners.re}.

\subsection{Summary}

The method is probably too 
old-school for modern computers.
Supposedly, this file format was developed in the mid-80's by 
someone who also coded for \IT{big iron} with
memory/disk space economy in mind.
Nevertheless, it has been an interesting and yet easy task 
to understand a proprietary file format without looking into \oracle's code.
}\RU{\section{\oracle: .MSB-файлы\ESph{}\PTBRph{}\PLph{}\ITAph{}\DEph{}\NLph{}}
\myindex{\oracle}

\epigraph{Работая над решением задачи, всегда полезно знать ответ.}{Законы Мерфи, правило точности}

Это бинарный файл, содержащий сообщения об ошибках вместе с их номерами.

Давайте попробуем понять его формат и найти способ распаковать его.

В \oracle имеются файлы с сообщениями об ошибках в текстовом виде, так что мы можем сравнивать файлы:
текстовый и запакованный бинарный
\footnote{Текстовые файлы с открытым кодом в \oracle имеются не для каждого .MSB-файла, вот почему мы будем работать над его форматом}.

Это начало файла ORAUS.MSG без ненужных комментариев:

\begin{lstlisting}[caption=Начало файла ORAUS.MSG без комментариев]
00000, 00000, "normal, successful completion"
00001, 00000, "unique constraint (%s.%s) violated"
00017, 00000, "session requested to set trace event"
00018, 00000, "maximum number of sessions exceeded"
00019, 00000, "maximum number of session licenses exceeded"
00020, 00000, "maximum number of processes (%s) exceeded"
00021, 00000, "session attached to some other process; cannot switch session"
00022, 00000, "invalid session ID; access denied"
00023, 00000, "session references process private memory; cannot detach session"
00024, 00000, "logins from more than one process not allowed in single-process mode"
00025, 00000, "failed to allocate %s"
00026, 00000, "missing or invalid session ID"
00027, 00000, "cannot kill current session"
00028, 00000, "your session has been killed"
00029, 00000, "session is not a user session"
00030, 00000, "User session ID does not exist."
00031, 00000, "session marked for kill"
...
\end{lstlisting}

Первое число --- это код ошибки.
Второе это, вероятно, могут быть дополнительные флаги.

\clearpage
Давайте откроем бинарный файл ORAUS.MSB и найдем эти текстовые строки. 
И вот они:

\begin{figure}[H]
\centering
\myincludegraphics{ff/Oracle_MSB/1.png}
\caption{Hiew: первый блок}
\label{fig:oracle_MSB_1}
\end{figure}

Мы видим текстовые строки (включая те, с которых начинается файл ORAUS.MSG) перемежаемые с какими-то
бинарными значениями.
Мы можем довольно быстро обнаружить что главная часть бинарного файла поделена на блоки размером 0x200 (512) байт.

\clearpage
Посмотрим содержимое первого блока:

\begin{figure}[H]
\centering
\myincludegraphics{ff/Oracle_MSB/2.png}
\caption{Hiew: первый блок}
\label{fig:oracle_MSB_2}
\end{figure}

Мы видим тексты первых сообщений об ошибках.
Что мы видим еще, так это то, что здесь нет нулевых байтов между сообщениями.
Это значит, что это не оканчивающиеся нулем Си-строки.
Как следствие, длина каждого сообщения об ошибке должна быть как-то закодирована.
Попробуем также найти номера ошибок.
Файл ORAUS.MSG начинается с таких: 
0, 1, 17 (0x11), 18 (0x12), 19 (0x13), 20 (0x14), 21 (0x15), 22 (0x16), 23 (0x17), 24 (0x18)...
Найдем эти числа в начале блока и отметим их красными линиями.
Период между кодами ошибок 6 байт.
Это значит, здесь, наверное, 6 байт информации выделено для каждого сообщения об ошибке.

Первое 16-битное значение (здесь 0xA или 10) означает количество сообщений в блоке: это можно проверить глядя на другие блоки.

Действительно: сообщения об ошибках имеют произвольный размер. 
Некоторые длиннее, некоторые короче. 
Но длина блока всегда фиксирована, следовательно, никогда не знаешь, сколько сообщений можно запаковать
в каждый блок.

Как мы уже отметили, так как это не оканчивающиеся нулем Си-строки, длина строки должна быть закодирована где-то.%

Длина первой строки \q{normal, successful completion} это 
29 (0x1D) байт.
Длина второй строки \q{unique constraint (\%s.\%s) violated} 
это 34 (0x22) байт.

Мы не можем отыскать этих значений (0x1D или/и 0x22) в блоке.

А вот еще кое-что.
\oracle должен как-то определять позицию строки, которую он должен загрузить, верно?
Первая строка \q{normal, successful completion} начинается с позиции 0x1444 (если считать с начала бинарного файла) или с 0x44 (от начала блока).
Вторая строка \q{unique constraint (\%s.\%s) violated} 
начинается с позиции 0x1461 (от начала файла) или с 0x61 (считая с начала блока).
Эти числа (0x44 и 0x61) нам знакомы! 
Мы их можем легко отыскать в начале блока.

Так что, каждый 6-байтный блок это:

\begin{itemize}
\item 16-битный номер ошибки; 
\item 16-битный ноль (может быть, дополнительные флаги; 
\item 16-битная начальная позиция текстовой строки внутри текущего блока.
\end{itemize}

Мы можем быстро проверить остальные значения чтобы удостовериться в своей правоте.
И здесь еще последний \q{пустой} 6-байтный блок с нулевым номером ошибки и начальной позицией за последним
символом последнего сообщения об ошибке.
Может быть именно так и определяется длина сообщения?
Мы просто перебираем 6-байтные блоки в поисках нужного номера ошибки, затем
мы узнаем позицию текстовой строки, затем мы узнаем позицию следующей текстовой строки глядя на
следующий 6-байтный блок!
Так мы определяем границы строки!
Этот метод позволяет сэкономить место в файле не записывая длину строки!
Нельзя сказать, что экономия памяти большая, но это интересный трюк.

\clearpage
Вернемся к заголовку .MSB-файла:

\begin{figure}[H]
\centering
\myincludegraphics{ff/Oracle_MSB/3.png}
\caption{Hiew: заголовок файла}
\label{fig:oracle_MSB_3}
\end{figure}

Теперь мы можем быстро найти количество блоков (отмечено красным).
Проверяем другие .MSB-файлы и оказывается что это справедливо для всех.
Здесь есть много других значений, но мы не будем разбираться с ними, так как наша задача (утилита для распаковки) уже решена.

А если бы мы писали запаковщик .MSB-файлов, тогда нам наверное пришлось бы понять, зачем нужны остальные.

\clearpage
Тут еще есть таблица после заголовка, по всей видимости, содержащая 16-битные значения:

\begin{figure}[H]
\centering
\myincludegraphics{ff/Oracle_MSB/4.png}
\caption{Hiew: таблица last\_errnos}
\label{fig:oracle_MSB_4}
\end{figure}

Их длина может быть определена визуально (здесь нарисованы красные линии).

Когда мы сдампили эти значения, мы обнаружили, что каждое 16-битное число --- это последний код ошибки для каждого блока.%

Так вот как \oracle быстро находит сообщение об ошибке:

\begin{itemize}
\item загружает таблицу, которую мы назовем last\_errnos 
(содержащую последний номер ошибки для каждого блока);

\item находит блок содержащий код ошибки, полагая что все коды ошибок увеличиваются и внутри каждого блока
и также в файле;

\item загружает соответствующий блок;

\item перебирает 6-байтные структуры, пока не найдется соответствующий номер ошибки;

\item находит позицию первого символа из текущего 6-байтного блока;

\item находит позицию последнего символа из следующего 6-байтного блока;

\item загружает все символы сообщения в этих пределах.
\end{itemize}

Это программа на Си которую мы написали для распаковки .MSB-файлов:
\href{http://go.yurichev.com/17213}{beginners.re}.

И еще два файла которые были использованы в этом примере
 
(\oracle 11.1.0.6):
\href{http://go.yurichev.com/17214}{beginners.re},
\href{http://go.yurichev.com/17215}{beginners.re}.

\subsection{Вывод}

Этот метод, наверное, слишком олд-скульный для современных компьютеров.
Возможно, формат этого файла был разработан в середине 1980-х кем-то, кто программировал для мейнфреймов,
учитывая экономию памяти и места на дисках.
Тем не менее, это интересная (хотя и простая) задача на разбор проприетарного формата файла без
заглядывания в код \oracle.
}


% another place for `unsorted' sections

\chapter{\RU{Прочее}\EN{Other things}\DE{Weitere Themen}}

% sections:
\EN{\section{Executable files patching}

\subsection{Text strings}

The C strings are the thing that is the easiest to patch (unless they are encrypted) in any hex editor.
This technique is available even for those who are not aware of machine code and executable file formats.
The new string has not to be bigger than the old one, because there's a risk of overwriting another value or code
there.
\myindex{MS-DOS}

Using this method, a lot of software was \IT{localized} in the MS-DOS era, at least in the ex-USSR countries in 80's
and 90's.
It was the reason why some weird abbreviations were present in the \IT{localized} software: there was no room for
longer strings.

\myindex{Borland Delphi}

As for Delphi strings, the string's size must also be corrected, if needed.

\subsection{x86 code}
\label{x86_patching}

Frequent patching tasks are:

\myindex{x86!\Instructions!NOP}
\begin{itemize}

\item 
One of the most frequent jobs is to disable some instruction.
It is often done by filling it using byte 
\TT{0x90} (\ac{NOP}).

\item Conditional jumps, which have an opcode like \TT{74 xx} (\JZ), 
can be filled with two \ac{NOP}s.

It is also possible to disable a conditional jump by writing 0 at the second byte (\IT{jump offset}).

\myindex{x86!\Instructions!JMP}
\item 
Another frequent job is to make a conditional jump to always trigger: 
this can be done by writing \TT{0xEB} 
instead of the opcode, which stands for \JMP.

\myindex{x86!\Instructions!RET}
\myindex{stdcall}
\item A function's execution can be disabled by writing \RETN (0xC3) at its beginning.
This is true for all functions excluding \TT{stdcall} (\myref{sec:stdcall}).
While patching \TT{stdcall} functions, one has to determine the number of arguments (for example, 
by finding \RETN in this function), 
and use \RETN with a 16-bit argument (0xC2).

\myindex{x86!\Instructions!MOV}
\myindex{x86!\Instructions!XOR}
\myindex{x86!\Instructions!INC}
\item Sometimes, a disabled functions has to return 0 or 1.
This can be done by \TT{MOV EAX, 0} or \TT{MOV EAX, 1}, 
but it's slightly verbose.\\
A better way is \TT{XOR EAX, EAX} (2 bytes \TT{0x31 0xC0}) or \TT{XOR EAX, EAX / INC EAX} (3 bytes \TT{0x31 0xC0 0x40}).

\end{itemize}

A software may be protected against modifications.

This protection is often done by reading the executable code and calculating a checksum.
Therefore, 
the code must be read before protection is triggered.

This can be determined by setting a breakpoint on reading memory.

\myindex{tracer}
\tracer has the BPM option for this.

PE executable file relocs (\myref{subsec:relocs}) 
must not to be touched while patching, 
because the Windows loader may overwrite your new code.
\myindex{Hiew}
(They are grayed in Hiew, for example:
\figref{fig:scanf_ex3_hiew_1}).

As a last resort, it is possible to write jumps that circumvent the relocs, 
or you will have to edit the relocs table.

}
\RU{\section{Модификация исполняемых файлов}

\subsection{Текстовые строки}

Сишные строки проще всего модифицировать (если они не зашифрованы) в любом шестнадцатеричном редакторе.
Эта техника доступна даже для тех, кто вовсе не разбирается в машинном коде и форматах исполняемых
файлов.
Новая строка не должна быть длиннее старой, потому что имеется риск затереть какую-то другую переменную
или код.
\myindex{MS-DOS}
Используя этот метод, очень много ПО было \IT{локализовано} во времена MS-DOS, как минимум,
в странах бывшего СССР, в 80-х и 90-х.
Отсюда наличие очень странных аббревиатур и сокращений в \IT{локализованном} ПО: 
там просто не было места для более
длинных строк.

\myindex{Borland Delphi}
В строках в Delphi, длина строки также должна быть поправлена, если нужно.

\subsection{x86-код}
\label{x86_patching}

Часто необходимые задачи:

\myindex{x86!\Instructions!NOP}
\begin{itemize}

\item Часто нужно просто запретить исполнение какой-либо инструкции.
И чаще всего, это можно сделать, заполняя её байтом 
\TT{0x90} (\ac{NOP}).

\item Условные переходы, имеющие опкод вроде \TT{74 xx} (\JZ), 
так же могут быть заполнены двумя \ac{NOP}-ами.
Также возможно запретить исполнение условного перехода записав 0 во второй байт (\IT{jump offset}).

\myindex{x86!\Instructions!JMP}
\item Еще одна часто необходимая задача это сделать условный переход всегда срабатывающим: 
это возможно при помощи записи \TT{0xEB} 
вместо опкода, это значит \JMP.

\myindex{x86!\Instructions!RET}
\myindex{stdcall}
\item Исполнение функции может быть запрещено, если записать
\RETN (0xC3) в её начале.
Это справедливо для всех функций кроме \TT{stdcall} 
(\myref{sec:stdcall}).
При модификации функций \TT{stdcall}, нужно в начале определить количество аргументов 
(например, отыскав \RETN в этой функции),
и использовать \RETN с 16-битным аргументом (0xC2).

\myindex{x86!\Instructions!MOV}
\myindex{x86!\Instructions!XOR}
\myindex{x86!\Instructions!INC}
\item Иногда, запрещенная функция должна возвращать 0 или 1.
Это можно сделать при помощи \TT{MOV EAX, 0} или \TT{MOV EAX, 1}, 
но это слишком многословно.\\
Способ получше это \TT{XOR EAX, EAX} (2 байта \TT{0x31 0xC0}) или \TT{XOR EAX, EAX / INC EAX} (3 байта \TT{0x31 0xC0 0x40}).

\end{itemize}

ПО может быть защищено от модификаций.
Эта защита чаще всего реализуется путем чтения кода и вычисления контрольной суммы.
Следовательно, код должен быть прочитан перед тем как защита сработает.
Это можно определить установив точку останова на чтение памяти.

\myindex{tracer}
В \tracer имеется опция BPM для этого.

Релоки в исполняемых PE-файлах (\myref{subsec:relocs}) 
не должны быть тронуты, потому что загрузчик Windows перезапишет ваш новый код.

\myindex{Hiew}
(Они выделяются серым в Hiew, например: \figref{fig:scanf_ex3_hiew_1}).
В качестве последней меры, можно записать \JMP для обхода релока, либо же придется модифицировать таблицу
релоков.


}
\DE{\section{Patchen von ausführbaren Dateien}

\subsection{Zeichenketten}

Zeichenketten in C können sehr einfach mit einem Hex-Editor verändert werden,
sofern sie nicht verschlüsselt sind.
Diese Technik kann sogar von denen angewandt werden, die nicht viel von Maschinencode
und Formaten von ausführbaren Dateien verstehen.
Die neue Zeichenkette darf nicht größer sein als die alte, da sonst die Gefahr
groß ist, dass ein anderer Wert oder Code überschrieben wird.
\myindex{MS-DOS}

Mit dieser Methode wurde in der MS-DOS-Ära eine Vielzahl von Software \IT{übersetzt},
zumindest in den ehemaligen USSR-Staaten in den 80er- und 90er-Jahren.
Aus diesem Grund existieren einige seltsame Abkürzungen in den \IT{übersetzten}
Programmen: es gab keinen Platz für längere Zeichenketten.

\myindex{Borland Delphi}

In Delphi müssen die Längen der Zeichenketten falls nötig korrigiert werden.

\subsection{x86-Code}
\label{x86_patching}

Häufige Aufgaben beim Patchen sind:

\myindex{x86!\Instructions!NOP}
\begin{itemize}

\item 
Eine der häufigsten Aufgaben ist das Deaktivieren bestimmter Anweisungen. Oft
wird dies durch Austauschen des Bytes durch \TT{0x90} (\ac{NOP}).

\item
Bedingte Sprünge, die den Opcode wie \TT{74 xx} (\JZ) haben, können durch
\ac{NOP}s ersetzt werden.

Es ist möglich alle bedingten Sprünge zu deaktivieren, in dem eine 0 in das
zweite Byte geschrieben wird (\IT{Sprung-Offset}).

\myindex{x86!\Instructions!JMP}
\item 
Eine weitere häufige Aufgabe ist es einen bedingten Sprung immer ausführen zu
lassen: dies kann durch Schreiben von \TT{0xEB}, was für \JMP steht, anstatt des
Opcodes erreicht werden.

\myindex{x86!\Instructions!RET}
\myindex{stdcall}
\item Die Ausführung einer Funktion kann deaktiviert werden, wenn \RETN (0xC3) an
den Anfang geschrieben wird. Dies gilt für alle Funktionen außer \TT{stdcall}
(\myref{sec:stdcall}).
Um \TT{stdcall}-Funktionen zu patchen muss die Anzahl der Argumente bekannt sein
(zum Beispiel durch Finden der \RETN-Anweisung in der Funktion) und die \RETN-Anweisung
mit einem 16-Bit-Argument (0xC2) angewendet werden.

\myindex{x86!\Instructions!MOV}
\myindex{x86!\Instructions!XOR}
\myindex{x86!\Instructions!INC}
\item Manchmal muss eine deaktivierte Funktion den Wert 0 oder 1 zurückgeben.
Dies kann durch \TT{MOV EAX, 0} oder \TT{MOV EAX, 1} erreicht werden, was aber relativ
ausführlich ist.
Ein besserer Weg ist \TT{XOR EAX, EAX} (2 Byte \TT{0x31 0xC0}) oder \TT{XOR EAX, EAX / INC EAX}
(3 Byte \TT{0x31 0xC0 0x40}).

\end{itemize}

Eine Software kann gegen Manipulation geschützt sein.

Dieser Schutz ist häufig realisiert indem der ausführbare Code gelesen und ein
passende Checksumme errechnet wird.
Aus diesem Grund muss der Code gelesen werden bevor die Schutzfunktion aktiviert
wird. Die Stelle kann durch setzen eines Breakpoints beim Lesen von Speicher
herausgefunden werden.

\myindex{tracer}
\tracer hat für diesen Zweck die BPM-Option.

Die Relocs (\myref{subsec:relocs}) in ausführbaren PE-Dateien sollten nicht
verändert werden, da der Windows-Laser den neuen, veränderten Code möglicherweise
überschreibt.
\myindex{Hiew}
(In Hiew sind die Stellen grau markiert, zum Beispiel: \figref{fig:scanf_ex3_hiew_1}).

Eine Möglichkeit ist es Sprünge zu schreiben, welche die Relocs umgehen oder die
Reloc-Tabelle muss editiert werden.
}

\EN{\section{Function arguments number statistics}
\label{args_stat}

I've always been interesting in what is average number of function arguments.

\index{UNIX!grep}
I've analyzed many Windows 7 32-bit DLLs \\
(crypt32.dll, mfc71.dll, msvcr100.dll, shell32.dll, 
user32.dll, d3d11.dll, mshtml.dll, msxml6.dll, sqlncli11.dll, wininet.dll, mfc120.dll, msvbvm60.dll, ole32.dll, themeui.dll, wmp.dll) 
(because they use \IT{stdcall} convention, and so it is easy to \IT{grep} disassembly output just by \INS{RETN X}).

\begin{itemize}
\item no arguments: $\approx 29\%$
\item 1 argument: $\approx 23\%$
\item 2 arguments: $\approx 20\%$
\item 3 arguments: $\approx 11\%$
\item 4 arguments: $\approx 7\%$
\item 5 arguments: $\approx 3\%$
\item 6 arguments: $\approx 2\%$
\item 7 arguments: $\approx 1\%$
\end{itemize}

\begin{figure}[H]
\centering
\includegraphics[width=0.5\textwidth]{other/args_stat.png}
\caption{Function arguments number statistics}
\end{figure}

This is heavily dependent on programming style and may be very different for other software products.

}
\DE{\mysection{Statistiken von Funktionsargumenten}
\label{args_stat}

Ich war immer sehr daran interessiert welches die durchschnittliche Anzahl von
Argumenten der einzelnen Funktionen ist.

\index{UNIX!grep}
Dazu wurden viele Windows 7 32-Bit-DLLs analysiert
(crypt32.dll, mfc71.dll, msvcr100.dll, shell32.dll, user32.dll, d3d11.dll, mshtml.dll,
msxml6.dll, sqlncli11.dll, wininet.dll, mfc120.dll, msvbvm60.dll, ole32.dll, themeui.dll,
wmp.dll), da diese die stdcall-Konvention nutzen, was es einfach macht das Ergebnis des
Disassemblers mit grep nach \INS{RETN X} zu durchsuchen.

\begin{itemize}
\item keine Argumente: $\approx 29\%$
\item 1 Argument: $\approx 23\%$
\item 2 Argumente: $\approx 20\%$
\item 3 Argumente: $\approx 11\%$
\item 4 Argumente: $\approx 7\%$
\item 5 Argumente: $\approx 3\%$
\item 6 Argumente: $\approx 2\%$
\item 7 Argumente: $\approx 1\%$
\end{itemize}

\begin{figure}[H]
\centering
\includegraphics[width=0.5\textwidth]{other/args_stat.png}
\caption{Statistiken von Funktionsargumenten}
\end{figure}

Das Ergebnis ist stark vom Programmierstil abhängig und kann bei anderen Programmen
deutlich anders ausfallen.
}

\EN{\section{Compiler intrinsic}
\myindex{Compiler intrinsic}
\label{sec:compiler_intrinsic}

\myindex{x86!\Instructions!ROL}
\myindex{x86!\Instructions!ROR}

A function specific to a compiler which is not an usual library function.
The compiler generates a specific machine code instead of a call to it.
It is often a pseudofunction for specific \ac{CPU} instruction. \\
\\
For example, there are no cyclic shift operations in \CCpp languages, but they are present in most \ac{CPU}s.
For programmer's convenience, at least MSVC has pseudofunctions
\IT{\_rotl()} and \IT{\_rotr()}\FNMSDNROTxURL{}
which are translated by the compiler directly to the ROL/ROR x86 instructions. \\
\\
Another example are functions to generate SSE-instructions right in the code.

Full list of MSVC intrinsics: \href{http://go.yurichev.com/17254}{MSDN}.

}
\RU{\section{Compiler intrinsic}
\myindex{Compiler intrinsic}
\label{sec:compiler_intrinsic}

\myindex{x86!\Instructions!ROL}
\myindex{x86!\Instructions!ROR}
Специфичная для компилятора функция не являющаяся обычной библиотечной функцией.
Компилятор вместо её вызова генерирует определенный машинный код.
Нередко, это псевдофункции для определенной инструкции \ac{CPU}. \\
\\
Например, в языках \CCpp нет операции циклического сдвига, а во многих \ac{CPU} она есть.
Чтобы программисту были доступны эти инструкции, в MSVC есть псевдофункции 
\IT{\_rotl()} and \IT{\_rotr()}\FNMSDNROTxURL{},
которые компилятором напрямую транслируются в x86-инструкции \TT{ROL}/\TT{ROR}. \\
\\
Еще один пример это функции позволяющие генерировать SSE-инструкции прямо в коде.

Полный список intrinsics от MSVC: \href{http://go.yurichev.com/17254}{MSDN}.

}
\DE{\section{Intrinsische Compiler-Funktionen}
\myindex{Compiler intrinsic}
\label{sec:compiler_intrinsic}

\myindex{x86!\Instructions!ROL}
\myindex{x86!\Instructions!ROR}

Dabei handelt es sich um spezielle Funktionen eines Compilers, die nicht in der
Standard-Bibliothek enthalten sind.
Der Compiler generiert einen spezifischen Maschinencode anstatt ihn aufzurufen.
Dies ist häufig eine Pseudofunktion für eine spezielle \ac{CPU}-Anweisung.

Beispielsweise gibt es keine zyklische Schiebe-Anweisungen in \CCpp -Sprachen,
in den meisten \ac{CPU}s sind sie jedoch vorhanden.
Um dem Programmierer das Leben einfacher zu machen hat zumindest MSVC die
Pseudofunktionen \IT{\_rotl()} und \IT{\_rotr()}\FNMSDNROTxURL{} welche vom
Compiler direkt in die ROL/ROR x86-Anweisungen übersetzt werden.

Ein anderes Beispiel sind Funktionen die SSE-Anweisungen direkt im Code umwandeln.

Eine vollständige Liste von intrinsischen Funktionen in MSVC ist hier zu finden:
\href{http://go.yurichev.com/17254}{MSDN}.
}

\EN{\section{Compiler's anomalies}
\label{anomaly:Intel}
\myindex{\CompilerAnomaly}

\subsection{\oracle 11.2 and Intel C++ 10.1}

\myindex{Intel C++}
\myindex{\oracle}
\myindex{x86!\Instructions!JZ}

Intel C++ 10.1, which was used for \oracle 11.2 Linux86 compilation, may emit two \JZ in row,
and there are no references to the second \JZ. The second \JZ is thus meaningless.

\begin{lstlisting}[caption=kdli.o from libserver11.a,style=customasmx86]
.text:08114CF1                   loc_8114CF1: ; CODE XREF: __PGOSF539_kdlimemSer+89A
.text:08114CF1                                ; __PGOSF539_kdlimemSer+3994
.text:08114CF1 8B 45 08              mov     eax, [ebp+arg_0]
.text:08114CF4 0F B6 50 14           movzx   edx, byte ptr [eax+14h]
.text:08114CF8 F6 C2 01              test    dl, 1
.text:08114CFB 0F 85 17 08 00 00     jnz     loc_8115518
.text:08114D01 85 C9                 test    ecx, ecx
.text:08114D03 0F 84 8A 00 00 00     jz      loc_8114D93
.text:08114D09 0F 84 09 08 00 00     jz      loc_8115518
.text:08114D0F 8B 53 08              mov     edx, [ebx+8]
.text:08114D12 89 55 FC              mov     [ebp+var_4], edx
.text:08114D15 31 C0                 xor     eax, eax
.text:08114D17 89 45 F4              mov     [ebp+var_C], eax
.text:08114D1A 50                    push    eax
.text:08114D1B 52                    push    edx
.text:08114D1C E8 03 54 00 00        call    len2nbytes
.text:08114D21 83 C4 08              add     esp, 8
\end{lstlisting}

\begin{lstlisting}[caption=from the same code,style=customasmx86]
.text:0811A2A5                   loc_811A2A5: ; CODE XREF: kdliSerLengths+11C
.text:0811A2A5                                ; kdliSerLengths+1C1
.text:0811A2A5 8B 7D 08              mov     edi, [ebp+arg_0]
.text:0811A2A8 8B 7F 10              mov     edi, [edi+10h]
.text:0811A2AB 0F B6 57 14           movzx   edx, byte ptr [edi+14h]
.text:0811A2AF F6 C2 01              test    dl, 1
.text:0811A2B2 75 3E                 jnz     short loc_811A2F2
.text:0811A2B4 83 E0 01              and     eax, 1
.text:0811A2B7 74 1F                 jz      short loc_811A2D8
.text:0811A2B9 74 37                 jz      short loc_811A2F2
.text:0811A2BB 6A 00                 push    0
.text:0811A2BD FF 71 08              push    dword ptr [ecx+8]
.text:0811A2C0 E8 5F FE FF FF        call    len2nbytes
\end{lstlisting}

It is supposedly a code generator bug that was not found by tests, because 
resulting code works correctly anyway.

% TODO to be translated ...
\subsection{MSVC 6.0}

Just found in some old code:

\begin{lstlisting}[style=customasmx86]
                 fabs
                 fild    [esp+50h+var_34]
                 fabs
                 fxch    st(1) ; first instruction
                 fxch    st(1) ; second instruction
                 faddp   st(1), st
                 fcomp   [esp+50h+var_3C]
                 fnstsw  ax
                 test    ah, 41h
                 jz      short loc_100040B7
\end{lstlisting}

\myindex{x86!\Instructions!FXCH}
The first \INS{FXCH} instruction swaps \TT{ST(0)} and \TT{ST(1)}, the second do the same, so both do nothing.
This is a program uses MFC42.dll, so it could be MSVC 6.0, 5.0 or maybe even MSVC 4.2 from 1990s.

This pair do nothing, so it probably wasn't caught by MSVC compiler tests.
Or maybe I wrong?



\subsection{Summary}

Other compiler anomalies here in this book: 
\myref{anomaly:LLVM}, \myref{loops_iterators_loop_anomaly}, \myref{Keil_anomaly},
\myref{MSVC2013_anomaly},
\myref{MSVC_double_JMP_anomaly},
\myref{MSVC2012_anomaly}.

Such cases are demonstrated here in this book, to show that such compilers errors are possible and sometimes
one should not to rack one's brain while thinking why did the compiler generate such strange code.

}
\RU{\input{other/compiler_anomalies_RU}}
\DE{\section{Compiler Anomalien}
\label{anomaly:Intel}
\myindex{\CompilerAnomaly}

\subsection{\oracle 11.2 und Intel C++ 10.1}

\myindex{Intel C++}
\myindex{\oracle}
\myindex{x86!\Instructions!JZ}

Der Intel C++ 10.1-Compiler, der für \oracle 11.2 für Linux 86 genutzt wurde, kann
zwei \JZ in einer Reihe ausgeben. Es gibt keine Referenz zum zweiten \JZ. Das zweite
ist also ohne Bedeutung.

\begin{lstlisting}[caption=kdli.o from libserver11.a,style=customasmx86]
.text:08114CF1                   loc_8114CF1: ; CODE XREF: __PGOSF539_kdlimemSer+89A
.text:08114CF1                                ; __PGOSF539_kdlimemSer+3994
.text:08114CF1 8B 45 08              mov     eax, [ebp+arg_0]
.text:08114CF4 0F B6 50 14           movzx   edx, byte ptr [eax+14h]
.text:08114CF8 F6 C2 01              test    dl, 1
.text:08114CFB 0F 85 17 08 00 00     jnz     loc_8115518
.text:08114D01 85 C9                 test    ecx, ecx
.text:08114D03 0F 84 8A 00 00 00     jz      loc_8114D93
.text:08114D09 0F 84 09 08 00 00     jz      loc_8115518
.text:08114D0F 8B 53 08              mov     edx, [ebx+8]
.text:08114D12 89 55 FC              mov     [ebp+var_4], edx
.text:08114D15 31 C0                 xor     eax, eax
.text:08114D17 89 45 F4              mov     [ebp+var_C], eax
.text:08114D1A 50                    push    eax
.text:08114D1B 52                    push    edx
.text:08114D1C E8 03 54 00 00        call    len2nbytes
.text:08114D21 83 C4 08              add     esp, 8
\end{lstlisting}

\begin{lstlisting}[caption=from the same code,style=customasmx86]
.text:0811A2A5                   loc_811A2A5: ; CODE XREF: kdliSerLengths+11C
.text:0811A2A5                                ; kdliSerLengths+1C1
.text:0811A2A5 8B 7D 08              mov     edi, [ebp+arg_0]
.text:0811A2A8 8B 7F 10              mov     edi, [edi+10h]
.text:0811A2AB 0F B6 57 14           movzx   edx, byte ptr [edi+14h]
.text:0811A2AF F6 C2 01              test    dl, 1
.text:0811A2B2 75 3E                 jnz     short loc_811A2F2
.text:0811A2B4 83 E0 01              and     eax, 1
.text:0811A2B7 74 1F                 jz      short loc_811A2D8
.text:0811A2B9 74 37                 jz      short loc_811A2F2
.text:0811A2BB 6A 00                 push    0
.text:0811A2BD FF 71 08              push    dword ptr [ecx+8]
.text:0811A2C0 E8 5F FE FF FF        call    len2nbytes
\end{lstlisting}

Dies ist vermutlich ein Fehler im Codegenerator der während der Tests nicht
gefunden wurde. Der resultierende Code funktioniert trotzdem.

% TODO to be translated ...
\subsection{MSVC 6.0}

Gerade in einem altem Code gefunden:

\begin{lstlisting}[style=customasmx86]
                 fabs
                 fild    [esp+50h+var_34]
                 fabs
                 fxch    st(1) ; erste Anweisung
                 fxch    st(1) ; zweite Anweisung
                 faddp   st(1), st
                 fcomp   [esp+50h+var_3C]
                 fnstsw  ax
                 test    ah, 41h
                 jz      short loc_100040B7
\end{lstlisting}

\myindex{x86!\Instructions!FXCH}
Die erste \INS{FXCH}-Anweisung tauscht \TT{ST(0)} und \TT{ST(1)}, die zweite tu das
gleiche, also haben beide zusammen keine Wirkung.
Das Programm nutzt MFC42.dll, also könnte es sich bei dem Compiler im MSVC 6.0, 5.0
oder eventuell MSVC 4.2 aus den 1990ern handeln.


\subsection{Zusammenfassung}

Andere Compiler-Anomalien in diesem Buch:
\myref{anomaly:LLVM}, \myref{loops_iterators_loop_anomaly}, \myref{Keil_anomaly},
\myref{MSVC2013_anomaly},
\myref{MSVC_double_JMP_anomaly},
\myref{MSVC2012_anomaly}.

Diese Beispiele werden in diesem Buch gezeigt, um zu verdeutlichen, das solche Fehler
in den Compilern möglich sind und es gelegentlich keinen Sinn ergibt sich den Kopf
darüber zu zerbrechen warum der Compiler diesen \q{seltsamen} Code erzeugte.
}

\EN{\section{Itanium}
\label{itanium}
\myindex{Itanium}

Although almost failed, Intel Itanium (\ac{IA64}) is a very interesting architecture.

While \ac{OOE} CPUs decides how to rearrange their instructions and execute them in parallel,
\ac{EPIC} was an attempt to shift these decisions to the compiler:
to let it group the instructions at the compile stage.

This resulted in notoriously complex compilers.

Here is one sample of \ac{IA64} code: simple cryptographic algorithm from the Linux kernel:

\lstinputlisting[caption=Linux kernel 3.2.0.4]{other/itanium/tea_from_linux.c}

Here is how it was compiled:

\lstinputlisting[caption=Linux Kernel 3.2.0.4 for Itanium 2 (McKinley)]{other/itanium/ia64_linux_3.2.0.4_mckinley.lst}

First of all, all \ac{IA64} instructions are grouped into 3-instruction bundles.

Each bundle has a size of 16 bytes (128 bits) and consists of template code (5 bits) + 3 instructions (41 bits for each).

\IDA shows the bundles as 6+6+4 bytes~---you can easily spot the pattern.

All 3 instructions from each bundle usually executes simultaneously, unless one of instructions has a \q{stop bit}.

Supposedly, Intel and HP engineers gathered statistics on most frequent instruction patterns and decided to bring
bundle types (\ac{AKA} \q{templates}): a bundle code defines the instruction types in the bundle.
There are 12 of them.

For example, the zeroth bundle type is \TT{MII}, which implies 
the first instruction is Memory (load or store), the second and third ones are I (integer instructions).

Another example is the bundle of type 0x1d: \TT{MFB}:
the first instruction is Memory (load or store), the second one is Float 
(\ac{FPU} instruction), and the third is Branch (branch instruction).

If the compiler cannot pick a suitable instruction for the relevant bundle slot, it may insert a \ac{NOP}:
you can see here the
\TT{nop.i} instructions (\ac{NOP} at the place where the integer instruction might be) or \TT{nop.m} 
(a memory instruction might be at this slot).

\ac{NOP}s are inserted automatically when one uses assembly language manually.

And that is not all. Bundles are also grouped.

Each bundle may have a \q{stop bit},
so all the consecutive bundles with a terminating bundle which has the \q{stop bit} 
can be executed simultaneously.

In practice, Itanium 2 can execute 2 bundles at once, resulting in the execution of 6 instructions at once.

So all instructions inside a bundle and a bundle group cannot interfere with each other 
(i.e., must not have data hazards).

If they do, the results are to be undefined.

Each stop bit is marked in assembly language as two semicolons (\TT{;;}) after the instruction.

So, the instructions at [90-ac] may be executed simultaneously:
they do not interfere. The next group is [b0-cc].

We also see a stop bit at 10c.
The next instruction at 110 has a stop bit too.

This implies that these instructions must be executed isolated from all others (as in \ac{CISC}).

Indeed: the next instruction at 110 uses the result from the previous one (the value in register r26),
so they cannot be executed at the same time.

Apparently, the compiler was not able to find a better way to parallelize the instructions,
in other words, to load \ac{CPU} as much as possible, hence too much stop bits and \ac{NOP}s.

Manual assembly programming is a tedious job as well: the programmer has to group the instructions manually.

The programmer is still able to add stop bits to each instructions, but this will degrade
the performance that Itanium was made for.

An interesting examples of manual \ac{IA64} assembly code can be found in the Linux kernel's sources:

\url{http://go.yurichev.com/17322}.

Another introductory paper on Itanium assembly:
[Mike Burrell, \IT{Writing Efficient Itanium 2 Assembly Code} (2010)]\footnote{\AlsoAvailableAs \url{http://yurichev.com/mirrors/RE/itanium.pdf}},
[papasutra of haquebright, \IT{WRITING SHELLCODE FOR IA-64} (2001)]\footnote{\AlsoAvailableAs \url{http://phrack.org/issues/57/5.html}}.

Another very interesting Itanium feature is the \IT{speculative execution} and the NaT (\q{not a thing}) bit,
somewhat resembling \gls{NaN} numbers: \\
\href{http://go.yurichev.com/17323}{MSDN}.

}
\RU{\section{Itanium}
\label{itanium}
\myindex{Itanium}
Еще одна очень интересная архитектура (хотя и почти провальная) это Intel Itanium (\ac{IA64}).
Другие \ac{OOE}-процессоры сами решают, как переставлять инструкции и исполнять их параллельно,
\ac{EPIC} это была попытка сдвинуть эти решения на компилятор: дать ему возможность самому 
группировать инструкции во время компиляции.

Это вылилось в очень сложные компиляторы.

Вот один пример \ac{IA64}-кода: простой криптоалгоритм из ядра Linux:

\lstinputlisting[caption=Linux kernel 3.2.0.4,style=customc]{other/itanium/tea_from_linux.c}

И вот как он был скомпилирован:

\lstinputlisting[caption=Linux Kernel 3.2.0.4 для Itanium 2 (McKinley)]{other/itanium/ia64_linux_3.2.0.4_mckinley.lst}

Прежде всего, все инструкции \ac{IA64} сгруппированы в пачки (bundle) из трех инструкций.
Каждая пачка имеет размер 16 байт (128 бит) и состоит из template-кода (5 бит) и трех инструкций (41 бит на каждую).

\IDA показывает пачки как 6+6+4 байт --- вы можете легко заметить эту повторяющуюся структуру.

Все 3 инструкции каждой пачки обычно исполняются одновременно, если только у какой-то инструкции
нет \q{стоп-бита}.

Может быть, инженеры Intel и HP собрали статистику наиболее встречающихся шаблонных сочетаний
инструкций и решили ввести типы пачек (\ac{AKA} \q{templates}): код пачки определяет типы инструкций
в пачке.
Их всего 12.
Например, нулевой тип это \TT{MII}, что означает: первая инструкция это Memory (загрузка
или запись в память), вторая и третья это I (инструкция, работающая с целочисленными значениями).

Еще один пример, тип 0x1d: \TT{MFB}: первая инструкция это Memory (загрузка или запись
в память), вторая это Float (инструкция, работающая с \ac{FPU}), третья это Branch (инструкция
перехода).

Если компилятор не может подобрать подходящую инструкцию в соответствующее место пачки,
он может вставить \ac{NOP}:
вы можете здесь увидеть инструкции \TT{nop.i} (\ac{NOP} на том месте где должна была бы находиться
целочисленная инструкция) или \TT{nop.m} (инструкция обращения к памяти должна была находиться
здесь).

Если вручную писать на ассемблере, \ac{NOP}-ы могут вставляться автоматически.

И это еще не все. Пачки тоже могут быть объединены в группы.
Каждая пачка может иметь \q{стоп-бит}, так что все следующие друг за другом пачки вплоть до той,
что имеет стоп-бит, могут быть исполнены одновременно.
На практике, Itanium 2 может исполнять 2 пачки одновременно, таким образом, исполнять
6 инструкций одновременно.

Так что все инструкции внутри пачки и группы не могут мешать друг другу (т.е. не должны
иметь data hazard-ов).
А если это так, то результаты будут непредсказуемые.

На ассемблере, каждый стоп-бит маркируется как две точки с запятой (\TT{;;}) после инструкции.

Так, инструкции на [90-ac] могут быть исполнены одновременно: они не мешают друг другу. Следующая группа: [b0-cc].

Мы также видим стоп-бит на 10c.
Следующая инструкция на 110 также имеет стоп-бит.
Это значит, что эти инструкции должны исполняться изолированно от всех остальных (как в \ac{CISC}).
Действительно: следующая инструкция на 110 использует результат, полученный от предыдущей (значение
в регистре r26), так что они не могут исполняться одновременно.
Должно быть, компилятор не смог найти лучший способ распараллелить инструкции, или, иными
словами, загрузить \ac{CPU} насколько это возможно, отсюда так много стоп-битов и \ac{NOP}-ов.
Писать на ассемблере вручную это также очень трудная задача: программист должен группировать
инструкции вручную.

У программиста остается возможность добавлять стоп-биты к каждой инструкции, но это
сведет на нет всю мощность Itanium, ради которой он создавался.

Интересные примеры написания \ac{IA64}-кода вручную можно найти в исходниках ядра Linux:

\url{http://go.yurichev.com/17322}.

Еще пара вводных статей об ассемблере Itanium:
[Mike Burrell, \IT{Writing Efficient Itanium 2 Assembly Code} (2010)]\footnote{\AlsoAvailableAs \url{http://yurichev.com/mirrors/RE/itanium.pdf}},
[papasutra of haquebright, \IT{WRITING SHELLCODE FOR IA-64} (2001)]\footnote{\AlsoAvailableAs \url{http://phrack.org/issues/57/5.html}}.

Еще одна интересная особенность Itanium это \IT{speculative execution} (исполнение инструкций
заранее, когда еще не известно, нужно ли это) и бит NaT (\q{not a thing}), отдаленно напоминающий
\gls{NaN}-числа: \\
\href{http://go.yurichev.com/17323}{MSDN}.

}
\DE{\section{Itanium}
\label{itanium}
\myindex{Itanium}

Auch wenn fast gescheitert, ist der Intel Itanium (\ac{IA64}) eine sehr interessante
Architektur.

Während \ac{OOE}-CPUs entscheiden wie die Anweisungen neu organisiert werden und
diese parallel ausführen, war \ac{EPIC} ein Versuch diese Entscheidung dem Compiler
zu überlassen: das Gruppieren der Anweisungen soll während des Kompilierens erfolgen.

Dies führte zu einer berüchtigten Komplexität der Compiler.

Hier ist ein Beispiel von \ac{IA64}-Code, ein einfacher kryptografischer Algorithmus
aus dem Linux-Kernel:

\lstinputlisting[caption=Linux kernel 3.2.0.4,style=customc]{other/itanium/tea_from_linux.c}

Nachfolgend das Ergebnis des Compilers:

\lstinputlisting[caption=Linux Kernel 3.2.0.4 for Itanium 2 (McKinley)]{other/itanium/ia64_linux_3.2.0.4_mckinley.lst}

Zunächst sind alle \ac{IA64}-Anweisungen in Pakete von 3 Anweisungen zusammengefasst.

Jedes Paket hat eine Größe von 16 Byte (128 Bit) und besteht aus Template-Code
(5 Bit) und drei Anweisungen (je 41 Bit).

\IDA zeigt die Pakete als 6+6+4 Byte, das Muster ist leicht zu erkennen.

Alle drei Anweisungen von jedem Paket wird in der Regel gleichzeitig ausgeführt,
außer eine der Anweisungen enthält ein \q{Stop-Bit}.

Vermutlich haben die Intel- und HP-Ingenieure Statistiken über die am meisten verwendeten
Anweisungsmuster erhoben und entschieden die Pakettypen zu erstellen (\ac{AKA} \q{Templates}): 
ein Paket-Code definiert den Anweisungstyp im Paket.
Es existieren 12 von ihnen.

Beispielsweise ist der nullte Pakettyp \TT{MII}, was impliziert, dass die erste
Anweisung Speicher (Lesen oder Schreiben) ist und die zweite und dritte jeweils
eine Integer-Anweisung ist.

Ein weiteres Beispiel ist das Paket vom Typ 0x1d: \TT{MFB}:
die erste Anweisung ist betrifft wieder den Speicher (Lesen oder Schreiben), die
zweite eine Fließkomma (\ac{FPU} Anweisung) und die dritte ein Springbefehl.

Wenn der Compiler keine passende Anweisung für den entsprechenden Paketplatz finden
kann, ist es möglich, dass er ein \ac{NOP} einfügt: man kann hier die \TT{nop.i}-Anweisung
(\ac{NOP} anstelle einer Integer-Anweisung) oder \TT{nop.m} (anstelle einer Speicheroperation)
sehen .

\ac{NOP}s werden automatisch eingefügt wenn mit Assembler gearbeitet wird.

Dies ist nicht alles: Pakete können ebenfalls grupiert werden.

Jedes Paket kann ein  \q{Stop-Bit} enthalten, so dass alle aufeinander folgenden
Pakete mit einem terminierenden Paket (mit \q{Stop-Bit}) gleichzeitig verarbeitet
werden können.

In der Praxis kann Itanium 2 gleichzeitig zwei Pakete ausführen, was zu sechs
Anweisungen führt.

Also kann keine der Anweisungen innerhalb einer Paket-Gruppe mit einer anderen
interagieren (es kann also nicht zu Datenkonflikten kommen).

Falls sie auftreten können die Ergebnisse undefiniert sein.

Jedes Stop-Bit ist in Assembler mit zwei Semikolons (\TT{;;}) nach de Anweisung markiert.

Die Anweisungen bei [90-ac] können also simultan ausgeführt werden: sie beeinflussen
sich gegenseitig nicht. Die nächste Gruppe ist [b0-cc].

Hier ist auch das Stop-Bit bei 10c zu sehen
Die nächste Anweisung bei 110 hat ebenfalls ein Stop-Bit.

Dies impliziert dass diese Anweisungen von allen anderen getrennt ausgeführt werden
müssen (wie in \ac{CISC}).

Außerdem ist zu sehen, dass die Anweisung nach 110 das Ergebnis der vorangehenden
benutzt (Den Wert im Register r26), dementsprechend können sie nicht gleichzeitig
ausgeführt werden.

Anscheinend war der Compiler nicht in der Lage einen besseren Weg zum Parallelisieren
der Anweisungen zu finden, also die \ac{CPU} so weit wie möglich auszulasten. Daher
die vielen Stop-Bits und \ac{NOP}-Anweisungen.

Manuelle Assembler-Programmierung ist ein mühsamer Job: der Programmierer muss
die Anweisungen selber in Gruppen einteilen.

Der Programmierer ist immer noch in der Lage Stop-Bits zu jeder Anweisung hinzuzufügen,
doch dies wird die Geschwindigkeit heruntersetzen für die Itanium gemacht wurde.

Ein interessantes Beispiel von manuellem Assembler-Code in \ac{IA64} kann im Code
des Linux-Kernels gefunden werden:

\url{http://go.yurichev.com/17322}.

Eine weitere Einführung für den Itanium-Assembler:
[Mike Burrell, \IT{Writing Efficient Itanium 2 Assembly Code} (2010)]\footnote{\AlsoAvailableAs \url{http://yurichev.com/mirrors/RE/itanium.pdf}},
[papasutra of haquebright, \IT{WRITING SHELLCODE FOR IA-64} (2001)]\footnote{\AlsoAvailableAs \url{http://phrack.org/issues/57/5.html}}.

Weitere sehr interessante Itanium-Features sind \IT{speculative execution} und das
NaT (\q{not a thing})-Bit, was in gewisser Weise \gls{NaN}-Zahlen ähnelt:

\href{http://go.yurichev.com/17323}{MSDN}
}

\EN{\section{8086 memory model}
\myindex{Intel!8086!Memory model}
\myindex{MS-DOS}
\label{8086_memory_model}

When dealing with 16-bit programs for MS-DOS or Win16
(\myref{dongle_16bit_dos} or \myref{win16_near_far_pointers}),
we can see that the pointers consist of two 16-bit values.
What do they mean? Oh yes, that is another weird MS-DOS and 8086 artifact.

8086/8088 was a 16-bit CPU, but was able to address 20-bit address in RAM 
(thus being able to access 1MB of external memory).

The external memory address space was divided between \ac{RAM} (640KB max),
\ac{ROM}, windows for video memory, EMS cards, \etc{}.

Let's also recall that 8086/8088 was in fact an inheritor of the 8-bit 8080 CPU.

The 8080 has a 16-bit memory space, i.e., it was able to address only 64KB.

And probably because of reason of old software porting\footnote{The author is not 100\% sure here},
8086 can support many 64KB windows simultaneously, placed
within the 1MB address space.

This is some kind of a toy-level virtualization.
\myindex{x86!\Registers!CS}
\myindex{x86!\Registers!DS}
\myindex{x86!\Registers!ES}
\myindex{x86!\Registers!SS}

All 8086 registers are 16-bit, so to address more, special segment registers (CS, DS, ES, SS) were
introduced.

Each 20-bit pointer is calculated using the values from a segment register and 
an address register pair (e.g. DS:BX) as follows:

\begin{center}
$real\_address = (segment\_register \ll 4) + address\_register$
\end{center}

For example, the graphics (\ac{EGA}, \ac{VGA}) video \ac{RAM} window on old IBM PC-compatibles 
has a size of 64KB.

To access it, a value of 0xA000 has to be stored in one of the segment registers, e.g. into DS.

Then DS:0 will address the first byte of video \ac{RAM} and DS:0xFFFF ~--- the last byte of RAM.

The real address on the 20-bit address bus, however, will range from 0xA0000 to 0xAFFFF.

The program may contain hard-coded addresses like 0x1234, but the \ac{OS} may need to load the program at arbitrary
addresses, so it recalculates the segment register values in a way that the program does not have to care 
where it's placed in the RAM.

So, any pointer in the old MS-DOS environment in fact consisted of the segment address and the address inside
segment, i.e., two 16-bit values. 20-bit was enough for that, though, but we needed to recalculate
the addresses very often: passing more information on the stack seemed a better space/convenience balance.

By the way, because of all this it was not possible to allocate a memory block larger than 64KB.

\myindex{Intel!80286}
\myindex{Intel!80386}

The segment registers were reused at 80286 as selectors, serving a different function.

\myindex{MS-DOS!DOS extenders}

When the 80386 CPU and computers with bigger \ac{RAM} were introduced,
MS-DOS was still popular, so the DOS extenders emerged: these were in fact
a step toward a \q{serious} \ac{OS},
switching the CPU in protected mode and providing much better memory \ac{API}s for the programs 
which still needed to run under MS-DOS.

Widely popular examples include DOS/4GW (the DOOM video game was compiled for it), Phar Lap, PMODE.
\par
\myindex{Windows!Windows 3.x}
\myindex{Windows!Win32}

By the way, the same way of addressing memory was used in the 16-bit line of Windows 3.x, before Win32.

}
\RU{\section{Модель памяти в 8086}
\myindex{Intel!8086!Модель памяти}
\myindex{MS-DOS}
\label{8086_memory_model}

Разбирая 16-битные программы для MS-DOS или Win16
(\myref{dongle_16bit_dos} или \myref{win16_near_far_pointers}),
мы можем увидеть, что указатель состоит из двух 16-битных значений.
Что это означает? О да, еще один дивный артефакт MS-DOS и 8086.

8086/8088 был 16-битным процессором, но мог адресовать 20-битное адресное
пространство (таким образом мог адресовать 1MB внешней памяти).
Внешняя адресное пространство было разделено между \ac{RAM} (максимум 640KB),
\ac{ROM}, окна для видеопамяти, EMS-карт, \etc{}.

Припомним также что 8086/8088 был на самом деле наследником 8-битного процессора 8080.
Процессор 8080 имел 16-битное адресное пространство, т.е. мог адресовать только 64KB.
И возможно в расчете на портирование старого ПО\footnote{Автор не уверен на 100\% здесь},
8086 может поддерживать 64-килобайтные
окна, одновременно много таких, расположенных внутри одномегабайтного адресного пространства.
Это, в каком-то смысле, игрушечная виртуализация.
\myindex{x86!\Registers!CS}
\myindex{x86!\Registers!DS}
\myindex{x86!\Registers!ES}
\myindex{x86!\Registers!SS}
Все регистры 8086 16-битные, так что, чтобы адресовать больше, специальные сегментные
регистры (CS, DS, ES, SS) были введены.
Каждый 20-битный указатель вычисляется, используя значения из пары состоящей из сегментного регистра
и адресного регистра (например DS:BX) вот так:

\begin{center}
$real\_address = (segment\_register \ll 4) + address\_register$
\end{center}

Например, окно памяти для графики (\ac{EGA}, \ac{VGA}) на старых IBM PC-совместимых компьютерах
имело размер 64KB.

Для доступа к нему, значение 0xA000 должно быть записано в один из сегментных регистров,
например, в DS.

Тогда DS:0 будет адресовать самый первый байт видеопамяти, а DS:0xFFFF ~--- самый последний байт.

А реальный адрес на 20-битной адресной шине, на самом деле будет от 0xA0000 до 0xAFFFF.

Программа может содержать жесткопривязанные адреса вроде 0x1234, но \ac{OS} может иметь необходимость
загрузить программу по другим адресам, так что она пересчитает значения для сегментных регистров так,
что программа будет нормально работать, не обращая внимания на то,
в каком месте памяти она была расположена.

Так что, любой указатель в окружении старой MS-DOS на самом деле состоял из адреса сегмента
и адреса внутри сегмента, т.е. из двух 16-битных значений. 20-битного значения было бы достаточно для
этого, хотя, тогда пришлось бы вычислять адреса слишком часто: так что передача большего количества
информации в стеке ~--- это более хороший баланс между экономией места и удобством.

Кстати, из-за всего этого, не было возможным выделить блок памяти больше чем 64KB.

\myindex{Intel!80286}
\myindex{Intel!80386}
В 80286 сегментные регистры получили новую роль селекторов, имеющих немного другую функцию.

\myindex{MS-DOS!DOS extenders}
Когда появился процессор 80386 и компьютеры с большей памятью,
MS-DOS была всё еще популярна, так что появились DOS-экстендеры: на самом деле это уже был шаг
к \q{серьезным} \ac{OS}, они переключали \ac{CPU} в защищенный режим и предлагали куда лучшее \ac{API} для
программ, которые всё еще предполагалось запускать в MS-DOS.

Широко известные примеры это DOS/4GW (игра DOOM была скомпилирована под него), Phar Lap, PMODE.

\par
\myindex{Windows!Windows 3.x}
\myindex{Windows!Win32}
Кстати, точно такой же способ адресации памяти был и в 16-битной линейке Windows 3.x, перед Win32.

}
\DE{\section{8086-Speichermodell}
\myindex{Intel!8086!Memory model}
\myindex{MS-DOS}
\label{8086_memory_model}

Wenn es um 16-Bit-Programme für MS-DOS oder Win16 geht (\myref{dongle_16bit_dos}
oder \myref{win16_near_far_pointers}), kann man sehen, dass die Zeiger aus zwei
16-Bit-Werten bestehen.
Was bedeutet das? Ja, das ist wieder ein weiteres sonderbares Artefakt von MS-DOS
und 8086.

8086/8088 war eine 16-Bit-CPU, war aber in der Lage 20-Bit-Adressen im RAM anzusprechen
(und somit externen Speicher bis 1MB zu adressieren).

Der Adressbereich für externen Speicher ist aufgeteilt zwischen \ac{RAM} (maximal 640KB),
\ac{ROM}, Fenster für Videospeicher, EMS-Karten, \etc{}.

Erinnern wir uns auch nochmal daran, dass der 8086/8088 der Nachfolger der 8-Bit-CPU
8080 war.

Der 8080 hat einen 16-Bit-Adressspeicher, kann also lediglich 64KB Speicher adressieren.

Möglicherweise aus Gründen der Portierung alter Software\footnote{Der Autor ist sich hier jedoch nicht 100\% sicher.},
kann der 8086 viele 64KB-Fenster gleichzeitig unterstützen, die sich im 1MB-Adressbereich
befinden.

Dies ist eine Art Top-Level-Virtualisierung.
\myindex{x86!\Registers!CS}
\myindex{x86!\Registers!DS}
\myindex{x86!\Registers!ES}
\myindex{x86!\Registers!SS}

Alle 8086-Register sind 16-Bit breit. Um einen größeren Bereich adressieren zu können,
wurden spezielle Segment-Register (CS, DS, ES, SS)  eingeführt.

Jeder 20-Bit-Zeiger wird aus den Werten eines Segment-Registers und einem
Adressregister-Paar (z.B. DS:BX) berechnet.

\begin{center}
$reale\_adresse = (segment\_register \ll 4) + adress\_register$
\end{center}

Zum Beispiel: das Grafik-Video-Speicher-Fenster (\ac{EGA}, \ac{VGA}) auf alten zu
IBM PC kompatiblen Rechnern hat eine Größe von 64KB.

Um darauf zuzugreifen muss der Wert 0xA000 in eines der Segment-Register geschrieben
werden, zum Beispiel in DS.

Anschließend wird DS:0 das erste Byte des Video-RAM und DS:0xFFFF das letzte
Byte adressieren.

Die echte Adresse auf dem 20-Bit-Adressbus ist in dem Bereich zwischen 0xA0000
und 0xAFFFF.

Das Programm kann hart-kodierte Adressen wie 0x1234 beinhalten, das \ac{OS} lädt
das Programm aber bei Bedarf an eine beliebige Adresse. Dazu werden die Segment-Registerwerte
derart neu berechnet, dass das Programm sich nicht darum kümmern muss an welcher
Stelle im RAM es sich befindet.

Jeder Zeiger in der alten MS-DOS-Umgebung besteht aus der Segmentadresse und der
Adresse innerhalb des Segment, also zwei 16-Bit-Werten. 20 Bit sind hierfür genug,
allerdings muss die Adresse recht oft neu berechnet werden. Mehr Informationen auf
dem Stack zu übergeben schien eine bessere Speicher- / Komfort-Balance zu haben.

Übrigens: aufgrund all der vorherigen Überlegungen war es nicht mögliche Speicherblöcke
zu allozieren die größer 64KB waren.

\myindex{Intel!80286}
\myindex{Intel!80386}

Die Segmentregister wurde beim 80286 als \q{Selektoren} wieder genutzt, jedoch mit
einer anderen Funktion.

\myindex{MS-DOS!DOS extenders}

Als die 80386-CPU mit größerem \ac{RAM} eingeführt wurde, war MS-DOS immer noch
weit verbreite, so das die DOS-Extender auftraten. Diese waren eigentlich ein
Schritt zu einem \q{seriösen} \ac{OS} indem die CPU in den Protected Mode geschaltet
wurde und sehr viel bessere Speicher-\ac{API}s für die Programme angeboten wurden,
die noch unter MS-DOS liefen.

Sehr populäre Beispiele waren DOS/4GW (das Spiel DOOM wurde hierfür kompiliert),
Phar Lap und PMODE.
\par
\myindex{Windows!Windows 3.x}
\myindex{Windows!Win32}

Übrigens wurde das gleiche Adressierungsmodel für Speicher in der 16-Bit-Reihe von
Windows 3.x genutzt, bevor Win32 aufkam.
}

\EN{\section{Basic blocks reordering}

% TODO __builtin_expect in GCC?

\subsection{Profile-guided optimization}
\label{PGO}

\myindex{\oracle}
\myindex{Intel C++}

This optimization method can move some \gls{basic block}s to another section of the executable binary file.

Obviously, there are parts of a function which are executed more frequently (e.g., loop bodies)
and less often (e.g., error reporting code, exception handlers).

The compiler adds instrumentation code into the executable, then the developer runs it with
a lot of tests to collect statistics.

Then the compiler, with the help of the statistics gathered,
prepares final the executable file with all infrequently executed code moved into another section.

As a result, all frequently executed function code is compacted, and that is very important
for execution speed and cache usage.

An example from \oracle code, which was compiled with Intel C++:

\begin{lstlisting}[caption=orageneric11.dll (win32),style=customasmx86]
                public _skgfsync
_skgfsync       proc near

; address 0x6030D86A

                db      66h
                nop
                push    ebp
                mov     ebp, esp
                mov     edx, [ebp+0Ch]
                test    edx, edx
                jz      short loc_6030D884
                mov     eax, [edx+30h]
                test    eax, 400h
                jnz     __VInfreq__skgfsync  ; write to log
continue:
                mov     eax, [ebp+8]
                mov     edx, [ebp+10h]
                mov     dword ptr [eax], 0
                lea     eax, [edx+0Fh]
                and     eax, 0FFFFFFFCh
                mov     ecx, [eax]
                cmp     ecx, 45726963h
                jnz     error                ; exit with error
                mov     esp, ebp
                pop     ebp
                retn
_skgfsync       endp

...

; address 0x60B953F0

__VInfreq__skgfsync:
                mov     eax, [edx]
                test    eax, eax
                jz      continue
                mov     ecx, [ebp+10h]
                push    ecx
                mov     ecx, [ebp+8]
                push    edx
                push    ecx
                push    offset ... ; "skgfsync(se=0x%x, ctx=0x%x, iov=0x%x)\n"
                push    dword ptr [edx+4]
                call    dword ptr [eax] ; write to log
                add     esp, 14h
                jmp     continue

error:
                mov     edx, [ebp+8]
                mov     dword ptr [edx], 69AAh ; 27050 "function called with invalid FIB/IOV structure"
                mov     eax, [eax]
                mov     [edx+4], eax
                mov     dword ptr [edx+8], 0FA4h ; 4004
                mov     esp, ebp
                pop     ebp
                retn
; END OF FUNCTION CHUNK FOR _skgfsync
\end{lstlisting}

The distance of addresses between these two code fragments is almost 9 MB.

All infrequently executed code was placed at the end of the code section of the DLL file,
among all function parts.

This part of the function was marked by the Intel C++ compiler with the \TT{VInfreq} prefix.

Here we see that a part of the function that writes to a log file (presumably in case of error or warning
or something like that) which was probably not executed very often when Oracle's developers gathered 
statistics (if it was executed at all).

The writing to log basic block eventually returns the control flow to the \q{hot} part of the function.

Another \q{infrequent} part is the \gls{basic block} returning error code 27050.

In Linux ELF files, all infrequently executed code is moved by Intel C++ into the separate 
\TT{text.unlikely} section, leaving all \q{hot} code in the \TT{text.hot} section.

From a reverse engineer's perspective, this information may help to split the function
into its core and error handling parts.
}
\RU{\section{Перестановка basic block-ов}

% TODO __builtin_expect in GCC?

\subsection{Profile-guided optimization}
\label{PGO}

\myindex{\oracle}
\myindex{Intel C++}

Этот метод оптимизации кода может перемещать некоторые \gls{basic block}-и в другую секцию
исполняемого бинарного файла.

Очевидно, в функции есть места которые исполняются чаще всего (например, тела циклов)
и реже всего (например, код обработки ошибок, обработчики исключений).

Компилятор добавляет дополнительный (instrumentation) код в исполняемый файл,
затем разработчик запускает его с тестами для сбора статистики.

Затем компилятор, при помощи собранной статистики, приготавливает итоговый исполняемый
файл где весь редко исполняемый код перемещен в другую секцию.

В результате, весь часто исполняемый код функции становится компактным, что очень важно для скорости
исполнения и кэш-памяти.

Пример из \oracle, который скомпилирован при помощи Intel C++:

\begin{lstlisting}[caption=orageneric11.dll (win32),style=customasmx86]
                public _skgfsync
_skgfsync       proc near

; address 0x6030D86A

                db      66h
                nop
                push    ebp
                mov     ebp, esp
                mov     edx, [ebp+0Ch]
                test    edx, edx
                jz      short loc_6030D884
                mov     eax, [edx+30h]
                test    eax, 400h
                jnz     __VInfreq__skgfsync  ; write to log
continue:
                mov     eax, [ebp+8]
                mov     edx, [ebp+10h]
                mov     dword ptr [eax], 0
                lea     eax, [edx+0Fh]
                and     eax, 0FFFFFFFCh
                mov     ecx, [eax]
                cmp     ecx, 45726963h
                jnz     error                ; exit with error
                mov     esp, ebp
                pop     ebp
                retn
_skgfsync       endp

...

; address 0x60B953F0

__VInfreq__skgfsync:
                mov     eax, [edx]
                test    eax, eax
                jz      continue
                mov     ecx, [ebp+10h]
                push    ecx
                mov     ecx, [ebp+8]
                push    edx
                push    ecx
                push    offset ... ; "skgfsync(se=0x%x, ctx=0x%x, iov=0x%x)\n"
                push    dword ptr [edx+4]
                call    dword ptr [eax] ; write to log
                add     esp, 14h
                jmp     continue

error:
                mov     edx, [ebp+8]
                mov     dword ptr [edx], 69AAh ; 27050 "function called with invalid FIB/IOV structure"
                mov     eax, [eax]
                mov     [edx+4], eax
                mov     dword ptr [edx+8], 0FA4h ; 4004
                mov     esp, ebp
                pop     ebp
                retn
; END OF FUNCTION CHUNK FOR _skgfsync
\end{lstlisting}

Расстояние между двумя адресами приведенных фрагментов кода почти 9 МБ.

Весь редко исполняемый код помещен в конце секции кода DLL-файла, среди редко
исполняемых частей прочих функций.
Эта часть функции была отмечена компилятором Intel C++ префиксом \TT{VInfreg}.
Мы видим часть функции которая записывает в лог-файл (вероятно, в случае ошибки или предупреждения,
или чего-то в этом роде) которая, наверное, не исполнялась слишком часто, когда разработчики Oracle
собирали статистику (если вообще исполнялась).

Basic block записывающий в лог-файл, в конце концов возвращает управление в \q{горячую} часть
функции.

Другая \q{редкая} часть --- это \gls{basic block} возвращающий код ошибки 27050.

В ELF-файлах для Linux весь редко исполняемый код перемещается компилятором Intel C++
в другую секцию (\TT{text.unlikely}) оставляя весь \q{горячий} код в секции \TT{text.hot}.

С точки зрения reverse engineer-а, эта информация может помочь разделить функцию на её основу
и части, отвечающие за обработку ошибок.
}
\DE{\section{Basic Block Reordering}

% TODO __builtin_expect in GCC?

\subsection{Profile-guided Optimization}
\label{PGO}

\myindex{\oracle}
\myindex{Intel C++}

Diese Optimierungsmethode kann einige \gls{basic block}s zu anderen Sektionen der
ausführbaren Datei verschieben.

Offensichtlich gibt es Teile einer Funktion die öfter ausgeführt werden als andere
(zum Beispiel Schleifen-Rümpfe) und welche, die weniger oft ausgeführt werden
(beispielsweise Fehlerberichte oder Ausnahmebehandlungen).

Der Compiler fügt Messcode in die ausführbare Datei ein. Anschließend führt der
Programmierer diesen mit vielen Tests aus um Statistiken zu erstellen.

Der Compiler präpariert die ausführbare Datei mithilfe der erstellten Statistiken
insofern, dass alle weniger häufige Codeteile in eine andere Sektion der Datei
verschoben werden.

Als Ergebnis ist der häufig ausgeführte Funktionscode zusammengefasst, was sehr
wichtig für die Ausführgeschwindigkeit und die Cachebenutzung ist.

Ein Beispiel vom \oracle-Code, der mit dem Intel C++-Compiler übersetzt wurde:

\begin{lstlisting}[caption=orageneric11.dll (win32),style=customasmx86]
                public _skgfsync
_skgfsync       proc near

; address 0x6030D86A

                db      66h
                nop
                push    ebp
                mov     ebp, esp
                mov     edx, [ebp+0Ch]
                test    edx, edx
                jz      short loc_6030D884
                mov     eax, [edx+30h]
                test    eax, 400h
                jnz     __VInfreq__skgfsync  ; write to log
continue:
                mov     eax, [ebp+8]
                mov     edx, [ebp+10h]
                mov     dword ptr [eax], 0
                lea     eax, [edx+0Fh]
                and     eax, 0FFFFFFFCh
                mov     ecx, [eax]
                cmp     ecx, 45726963h
                jnz     error                ; exit with error
                mov     esp, ebp
                pop     ebp
                retn
_skgfsync       endp

...

; address 0x60B953F0

__VInfreq__skgfsync:
                mov     eax, [edx]
                test    eax, eax
                jz      continue
                mov     ecx, [ebp+10h]
                push    ecx
                mov     ecx, [ebp+8]
                push    edx
                push    ecx
                push    offset ... ; "skgfsync(se=0x%x, ctx=0x%x, iov=0x%x)\n"
                push    dword ptr [edx+4]
                call    dword ptr [eax] ; write to log
                add     esp, 14h
                jmp     continue

error:
                mov     edx, [ebp+8]
                mov     dword ptr [edx], 69AAh ; 27050 "function called with invalid FIB/IOV structure"
                mov     eax, [eax]
                mov     [edx+4], eax
                mov     dword ptr [edx+8], 0FA4h ; 4004
                mov     esp, ebp
                pop     ebp
                retn
; END OF FUNCTION CHUNK FOR _skgfsync
\end{lstlisting}

Der Abstand der Adressen zwischen diesen beiden Code-Fragmenten beträgt fast 9 MB.

Alle weniger oft ausgeführten Codeteile wurden an das Ende der Code-Sektion der
DLL-Datei verschoben.

Dieser Teil der Funktion wurde vom Intel C++-Compiler mit dem \TT{VInfreq}-Präfix
markiert.

Man kann hier sehen, dass der Teil der Funktion der in die Logdatei schreibt (zum
Beispiel im Falle eines Fehlers oder einer Warnung) vermutlich selten oder vielleicht
gar nicht ausgeführt wurde als der Entwickler von Oracle die Statistiken erstellt hat.

Das Schreiben in log basic block gibt die Ausführkontrolle letztendlich wieder zurück
an den \q{heißen} Teil der Funktion.

Ein weiterer \q{seltener} Teil ist der \gls{basic block}, welcher den Fehlercode
27050 zurück gibt.

In Linux ELF-Dateien wird der selten ausgeführte Code vom Intel C++-Compiler in die
separate \TT{text.unlikely}-Sektion verschoben und der \q{heiße} Code in die Sektion
\TT{text.hot}.

Aus Sicht eines Reverse-Engineers kann diese Information helfen um die Funktion in
den Hauptteil und den Fehlerbehandlungsteil zu unterteilen.
}

\chapter{\IFRU{Что стоит почитать}{Books/blogs worth reading}}

\section{\IFRU{Книги}{Books}}

\subsection{Windows}

\cite{Russinovich}.

\subsection{\CCpp}

\begin{itemize}
\item
\IFRU{Стандарт языка Си++}{C++ language standard}: ISO/IEC 14882:2003\footnote{\url{http://www.iso.org/iso/catalogue_detail.htm?csnumber=38110}}
\end{itemize}

\subsection{x86 / x86-64}

\begin{itemize}
\item
\IFRU{Документация от Intel}{Intel manuals}: \url{http://www.intel.com/products/processor/manuals/}
\item
\IFRU{Документация от AMD}{AMD manuals}: \url{http://developer.amd.com/documentation/guides/Pages/default.aspx#manuals}
\end{itemize}

\subsection{ARM}

\IFRU{Документация от ARM}{ARM manuals}: \url{http://infocenter.arm.com/help/index.jsp?topic=/com.arm.doc.subset.architecture.reference/index.html}

\section{\IFRU{Блоги}{Blogs}}

\subsection{Windows}

\begin{itemize}
\item
\href{http://blogs.msdn.com/oldnewthing/}{Microsoft: Raymond Chen}
\item
\url{http://www.nynaeve.net/}
\end{itemize}


\part{\IFRU{Задачи}{Exercises}}

\IFRU{Почти для всех задач, если не указано иное, два вопроса:}
{There are two questions almost for every exercise, if otherwise is not specified:}

1) \IFRU{Что делает эта функция? Ответ должен состоять из одной фразы.}
{What this function does? Answer in one-sentence form.}

2) \IFRU{Перепишите эту функцию на \CCpp}{Rewrite this function into \CCpp}.

\IFRU{Пользоваться Google-м, для поиска
каких-либо зацепок, разрешается}{It is allowed to use Google to search for any leads}.
\IFRU{Впрочем, для усложения своей задачи, вы можете попробовать и без Google}
{However, if you like to make your task harder, you may try to solve it without Google}.

\IFRU{Подсказки и ответы собраны в приложении к этой книге.}{Hints and solutions are in the appendix of
this book.}

\chapter{\IFRU{Уровень}{Level} 1}

\IFRU{Задачи первого уровня, это те, которые можно решать голове}{Level 1 exercises are ones you
may try to solve in mind}.

\section{\Exercise 1.1}
% max()

\subsection{MSVC 2012 x64 + \Ox}

\index{x86!\Instructions!CMOVcc}
\begin{lstlisting}
a$ = 8
b$ = 16
f	PROC
	cmp	ecx, edx
	cmovg	edx, ecx
	mov	eax, edx
	ret	0
f	ENDP
\end{lstlisting}

\subsection{Keil (ARM)}

\begin{lstlisting}
        CMP      r0,r1
        MOVLE    r0,r1
        BX       lr
\end{lstlisting}

\subsection{Keil (thumb)}
        
\begin{lstlisting}
	CMP      r0,r1
        BGT      |L0.6|
        MOVS     r0,r1
|L0.6|
        BX       lr
\end{lstlisting}

\section{\Exercise 1.2}

\index{x86!\Instructions!LOOP}
\IFRU{Почему инструкция}{Why} \LOOP \IFRU{больше не используется компиляторами}{instruction is 
not used by compilers anymore}?

\section{\Exercise 1.3}

\IFRU{Возьмите пример из секции}{Take an loop example from} ``\Loops''\EN{ section} (\ref{sec:loops}), 
\IFRU{скомпилируйте его в вашей любимой}{compile it in your favorite} \ac{OS}
\IFRU{и компиляторе, и модифицируйте исполняемый файл так, чтобы цикл был в пределах}{and compiler 
and modify (patch) executable file, so the loop range will be} [6..20].

\section{\Exercise 1.4}

\IFRU{Эта программа запрашивает пароль}{This program requires password}. \IFRU{Найдите его}{Find it}.

\IFRU{Как дополнительное упражнение, попробуйте изменить пароль, модифицируя исполняемый файл,
в т.ч., на более короткий или более длинный.}
{As an additional exercise, try to change the password by patching executable file. 
It may also has a different length.}

\IFRU{Попытайтесь также вызвать аварийное завершение программы только при помощи ввода строки}
{Try also to crash the program using only string input}.

\begin{itemize}
\item win32: \url{http://yurichev.com/RE-exercises/1/4/password1.exe}
\item Linux x86: \url{http://yurichev.com/RE-exercises/1/4/password1_Linux_x86.tar}
\item \MacOSX: \url{http://yurichev.com/RE-exercises/1/4/password1_MacOSX64.tar}
\end{itemize}


\chapter{\RU{Уровень}\EN{Level} 2}

\RU{Для решения задач второго уровня, вам вероятно понадобится текстовый редактор или тетрадка с ручкой}
\EN{For solving exercises of level 2, you probably will need text editor or paper with pencil}.

% 2.1
% 2.2
% 2.3

\section{\Exercise 2.4}
% strstr()

\RU{Это стандартная функция из библиотек Си. Исходник взят из MSVC 2010.}
\EN{This is standard C library function. Source code taken from MSVC 2010.}

\subsection{\Optimizing MSVC 2010}

\lstinputlisting{exercises/1_4_msvc.asm}

\subsection{GCC 4.4.1}

\lstinputlisting{exercises/1_4_gcc.asm}

\subsection{\Optimizing Keil (\ARMMode)}

\lstinputlisting{exercises/1_4_ARM.s}

\subsection{\Optimizing Keil (\ThumbMode)}

\lstinputlisting{exercises/1_4_thumb.s}

\section{\Exercise 2.5}
% Pentium FDIV bug

\RU{Задача, скорее, на эрудицию, нежели на чтение кода.}
\EN{This exercise is rather on knowledge than on reading code.}

\index{OpenWatcom}
\RU{Функция взята из OpenWatcom}.
\EN{The function is taken from OpenWatcom}.

\subsection{\Optimizing MSVC 2010}

\lstinputlisting{exercises/1_5_msvc.asm}

\section{\Exercise 2.6}
% TEA

\subsection{\Optimizing MSVC 2010}

\lstinputlisting{exercises/1_6_msvc.asm}

\subsection{\Optimizing Keil (\ARMMode)}

\lstinputlisting{exercises/1_6_ARM.s}

\subsection{\Optimizing Keil (\ThumbMode)}

\lstinputlisting{exercises/1_6_thumb.s}

% 2.7
% 2.8
% 2.9
% 2.10
% 2.11
% 2.12

\section{\Exercise 2.13}
% LFSR

\RU{Это довольно известный криптоалгоритм прошлого}\EN{This is a well-known cryptoalgorithm of the past}.
\RU{Как он называется}\EN{How it's called}?

\subsection{\Optimizing MSVC 2012}

\begin{lstlisting}
_in$ = 8						; size = 2
_f	PROC
	movzx	ecx, WORD PTR _in$[esp-4]
	lea	eax, DWORD PTR [ecx*4]
	xor	eax, ecx
	add	eax, eax
	xor	eax, ecx
	shl	eax, 2
	xor	eax, ecx
	and	eax, 32					; 00000020H
	shl	eax, 10					; 0000000aH
	shr	ecx, 1
	or	eax, ecx
	ret	0
_f	ENDP
\end{lstlisting}

\subsection{Keil (\ARMMode)}

\begin{lstlisting}
f PROC
        EOR      r1,r0,r0,LSR #2
        EOR      r1,r1,r0,LSR #3
        EOR      r1,r1,r0,LSR #5
        AND      r1,r1,#1
        LSR      r0,r0,#1
        ORR      r0,r0,r1,LSL #15
        BX       lr
        ENDP
\end{lstlisting}

\subsection{Keil (\ThumbMode)}

\begin{lstlisting}
f PROC
        LSRS     r1,r0,#2
        EORS     r1,r1,r0
        LSRS     r2,r0,#3
        EORS     r1,r1,r2
        LSRS     r2,r0,#5
        EORS     r1,r1,r2
        LSLS     r1,r1,#31
        LSRS     r0,r0,#1
        LSRS     r1,r1,#16
        ORRS     r0,r0,r1
        BX       lr
        ENDP
\end{lstlisting}

\section{\Exercise 2.14}
% GCD

\RU{Еще один хорошо известный алгоритм. Ф-ция берет на вход 2 значения и возвращает одно.}
\EN{Another well-known algorithm. The function takes two variables and returning one.}

\subsection{MSVC 2012}

\index{ARM!\Instructions!CLZ}
\lstinputlisting{exercises/2/GCD_MSVC_2012_Ox.asm}

\subsection{Keil (\ARMMode)}

\index{ARM!\Instructions!CLZ}
\lstinputlisting{exercises/2/GCD_Keil_ARM_O3.s}

\subsection{GCC 4.6.3 for Raspberry Pi (\ARMMode)}

\index{x86!\Instructions!BSF}
\lstinputlisting{exercises/2/GCD_ARM_pi_GCC_4.6.3_O3.s}

\section{\Exercise 2.15}
% Monte Carlo

\RU{И снова известный алгоритм. Что он делает?}\EN{Well-known algorithm again. What it does?}

\RU{Обратите внимание, что код для x86 использует FPU, а для x64 --- SIMD-инструкции. Это нормально}
\EN{Take also notice that the code for x86 uses FPU, but SIMD-instructions are used instead in x64 code.
That's OK}: \ref{floating_SIMD}.

\subsection{\Optimizing MSVC 2012 x64}

\lstinputlisting{exercises/2/monte_MSVC_2012_Ox_x64.asm}

\subsection{\Optimizing GCC 4.4.6 x64}

\lstinputlisting{exercises/2/monte_GCC_4.4.6_O3_x64.s}

\subsection{\Optimizing GCC 4.8.1 x86}

\lstinputlisting{exercises/2/monte_GCC_4.8.1_O3_x86.s}

\subsection{Keil (\ARMMode): \RU{для процессора Cortex-R4F}\EN{Cortex-R4F CPU as target}}

\lstinputlisting{exercises/2/monte_Keil_ARM_Cortex.s}

\section{\Exercise 2.16}
% Ackermann function

\RU{Известная функция. Что она вычисляет? Почему стек переполняется если на вход подать
числа 4 и 2? Есть ли здесь какая-то ошибка?}\EN{Well-known function. What it computes? 
Why stack overflows if 4 and 2 are supplied at input? Are there any error?}

\subsection{\Optimizing MSVC 2012 x64}

\lstinputlisting{exercises/2/ack_MSVC_Ox_x64.asm}

\subsection{\Optimizing Keil (\ARMMode)}

\lstinputlisting{exercises/2/ack_ARM_O3.s}

\subsection{\Optimizing Keil (\ThumbMode)}

\lstinputlisting{exercises/2/ack_thumb_O3.s}

\section{\Exercise 2.17}
% Rule 110

\RU{Эта программа выдает в \gls{stdout} какую-то информацию, каждый раз --- разную}\EN{This program
prints some information to \gls{stdout}, each time different}.
\RU{Что это}\EN{What is it}?

\RU{Скомпилированные бинарные файлы}\EN{Compiled binaries}:

\begin{itemize}
\item Linux x64: \url{http://beginners.re/exercises/2/17/17_Linux_x64.tar}
\item \MacOSX: \url{http://beginners.re/exercises/2/17/17_MacOSX_x64.tar}
\item Win32: \url{http://beginners.re/exercises/2/17/17_win32.exe}
\item Win64: \url{http://beginners.re/exercises/2/17/17_win64.exe}
\end{itemize}

\RU{Для версий под Windows, возможно, нужно будет установить}
\EN{As of Windows versions, you may need to install} 
\href{http://www.microsoft.com/en-us/download/details.aspx?id=30679}{MSVC 2012 redist}.

\section{\Exercise 2.18}

\RU{Эта программа запрашивает пароль}\EN{This program requires password}.
\RU{Найдите его}\EN{Find it}.

\RU{Кстати, не только один пароль может подойти}\EN{By the way, multiple passwords may work}. 
\RU{Попробуйте найти еще}\EN{Try to find more}.

\RU{Как дополнительное упражнение, попробуйте изменить пароль модифицируя исполняемый файл}
\EN{As an additional exercise, try to change the password by patching executable file}.

\begin{itemize}
\item Win32: \url{http://beginners.re/exercises/2/18/password2.exe}
\item Linux x86: \url{http://beginners.re/exercises/2/18/password2_Linux_x86.tar}
\item \MacOSX: \url{http://beginners.re/exercises/2/18/password2_MacOSX64.tar}
\end{itemize}

\section{\Exercise 2.19}

\RU{То же что и в упражнении}\EN{The same as in exercise} 2.18.

\begin{itemize}
\item Win32: \url{http://beginners.re/exercises/2/19/password3.exe}
\item Linux x86: \url{http://beginners.re/exercises/2/19/password3_Linux_x86.tar}
\item \MacOSX: \url{http://beginners.re/exercises/2/19/password3_MacOSX64.tar}
\end{itemize}


\chapter{\IFRU{Уровень}{Level} 3}

\IFRU{Для решения задач третьего уровня вам придется потратить какое-то ощутимое время, 
вплоть до одного дня}
{For solving level 3 tasks, you'll probably need considerable ammount of time, maybe up to one day}.

\section{\Exercise 3.1}

\IFRU{Довольно известный алгоритм, так же включен в стандартную библиотеку Си. Исходник взят из glibc 2.11.1. 
Скомпилирован в GCC 4.4.1 с ключом \TT{-Os} (оптимизация по размеру кода). 
Листинг сделан дизассемблером IDA 4.9 из ELF-файла созданным GCC и линкером.}
{Well-known algorithm, also included in standard C library. Source code was taken from glibc 2.11.1.
Compiled in GCC 4.4.1 with \TT{-Os} option (code size optimization).
Listing was done by IDA 4.9 disassembler from ELF-file generated by GCC and linker.}

\IFRU{Для тех кто хочет использовать IDA в процессе изучения, вот здесь лежат .elf и .idb файлы, 
.idb можно открыть при помощи бесплатной IDA 4.9:}
{For those who wants use IDA while learning, here you may find .elf and .idb files,
.idb can be opened with freeware IDA 4.9:}

\url{http://yurichev.com/RE-exercises/3/1/}

\lstinputlisting{exercises/2_1_gcc.asm}

\section{\Exercise 3.2}

\IFRU{Имеется небольшой исполняемый файл, внутри которого находится довольно известная криптосистема}
{There is a small executable file with a well-known cryptosystem inside}.
\IFRU{Попробуйте её идентифицировать}{Try to identify it}.

\begin{itemize}
\item
\href{http://yurichev.com/RE-exercises/3/2/unknown_cryptosystem.exe}{Windows x86}

\item
\href{http://yurichev.com/RE-exercises/3/2/unknown_encryption_linux86.tar}{Linux x86}

\item
\href{http://yurichev.com/RE-exercises/3/2/unknown_encryption_MacOSX.tar}{MacOSX (x64)}
\end{itemize}

\section{\Exercise 3.3}

\IFRU{Имеется небольшой исполняемый файл, некая утилита}
{There is a small executable file, some utility}.
\IFRU{Она открывает другой файл, читает его, что-то вычисляет и показывает число с плавающей точкой}
{It opens another file, reads it, calculate something and prints a float number}.
\IFRU{Попробуйте разобраться, что она делает}{Try to understand what it do}.

\begin{itemize}
\item
\href{http://yurichev.com/RE-exercises/3/3/unknown_utility_2_3.exe}{Windows x86}

\item
\href{http://yurichev.com/RE-exercises/3/3/unknown_utility_2_3_Linux86.tar}{Linux x86}

\item
\href{http://yurichev.com/RE-exercises/3/3/unknown_utility_2_3_MacOSX.tar}{MacOSX (x64)}
\end{itemize}

\section{\Exercise 3.4}

\IFRU{Утилита, шифрующая и дешифрующая файлы, по паролю}
{There is an utility which encrypts/decrypts files, by password}.
\IFRU{Есть зашифрованный текстовый файл, пароль неизвестен}{There is an encrypted text file,
password is unknown}.
\IFRU{Зашифрованный файл ~--- это текст на английском языке}{Encrypted file is a text in English language}.
\IFRU{Утилита использует сравнительно мощный алгоритм шифрования, тем не менее,
он был применен с очень грубой ошибкой. И из-за ошибки расшифровать файл вполне возможно с минимумом затрат}
{The utility uses relatively strong cryptosystem, nevertheless, it was implemented with a serious blunder.
Since the mistake present, it is possible to decrypt the file with a little effort.}.

\IFRU{Попробуйте найти ошибку и расшифровать файл}{Try to find the mistake and decrypt the file}.

\begin{itemize}
\item
\href{http://yurichev.com/RE-exercises/3/4/amateur_cryptor.exe}{Windows x86}

\item
\href{http://yurichev.com/RE-exercises/3/4/text_encrypted}{\IFRU{Текстовый файл}{Text file}}
\end{itemize}

\section{\Exercise 3.5}

\IFRU{Это имитация защиты от копирования использующей ключевой файл}
{This is software copy protection imitation, which uses key file}.
\IFRU{В ключевом файле имя пользователя и серийный номер}
{The key file contain user (or customer) name and serial number}.

\IFRU{Задачи две}{There are two tasks}:

\begin{itemize}
\item
\IFRU{(Простая) при помощи \tracer либо иного отладчика, 
заставьте эту программу принимать измененный ключевой файл}{(Easy) with the help of \tracer
or any other debugger, force the program to accept changed key file}.

\item
\IFRU{(Средняя) ваша задача заключается в том, чтобы изменить в файле имя пользователя на другое, 
но при этом, модифицировать саму программу нельзя}
{(Medium) your goal is to modify user name to another, however, it is not allowed to patch the program}.
\end{itemize}

\begin{itemize}
\item
\href{http://yurichev.com/RE-exercises/3/5/super_mega_protection.exe}{Windows x86}

\item
\href{http://yurichev.com/RE-exercises/3/5/super_mega_protection.tar}{Linux x86}

\item
\href{http://yurichev.com/RE-exercises/3/5/super_mega_protection_MacOSX.tar}{MacOSX (x64)}

\item
\href{http://yurichev.com/RE-exercises/3/5/sample.key}{\IFRU{Ключевой файл}{Key file}}
\end{itemize}

\section{\Exercise 3.6}

\IFRU{Это очень примитивный игрушечный веб-сервер, поддерживающий только статические файлы, без \ac{CGI}, и т.д}
{Here is a very primitive toy web-server, supporting only static files, without \ac{CGI}, etc}.
\IFRU{В нем сознательно оставлено по крайней мере 4 уязвимости}
{At least 4 vulnerabilities are leaved here intentionally}.
\IFRU{Постарайтесь найти их все и использовать для взлома удаленной машины}
{Try to find them all and exploit them in order for breaking into a remote host}.

\begin{itemize}
\item
\href{http://yurichev.com/RE-exercises/3/6/webserv_win32.rar}{Windows x86}

\item
\href{http://yurichev.com/RE-exercises/3/6/webserv_Linux_x86.tar}{Linux x86}

\item
\href{http://yurichev.com/RE-exercises/3/6/webserv_MacOSX_x64.tar}{MacOSX (x64)}
\end{itemize}

\section{\Exercise 3.7}

\IFRU{При помощи \tracer или любого другого win32-отладчика, найдите скрытые мины во время игры,
в стандартной игре Windows MineSweeper}
{With the help of \tracer or any other win32 debugger, reveal hidden mines in the MineSweeper standard Widnows game
during play}.

\IFRU{Подсказка: в}{Hint:} \cite{trew} \IFRU{имеются некоторые описания внутренностей игры MineSweeper}
{have some insights about MineSweeper's internals}.


\chapter{crackme / keygenme}

\IFRU{Несколько моих \gls{keygenme}:}
{Couple of my \glspl{keygenme}:}

\url{http://crackmes.de/users/yonkie/}


\fi
\part*{\RU{Послесловие}\EN{Afterword}}
\addcontentsline{toc}{part}{\RU{Послесловие}\EN{Afterword}}

\chapter{\RU{Вопросы?}\EN{Questions?}}

\RU{Совершенно по любым вопросам, вы можете не раздумывая писать автору}
\EN{Do not hesitate to mail any questions to the author}: \TT{<\EMAIL>}

\EN{There is also a support forum, you can ask any questions there}
\RU{Есть также форум поддержки, вы можете задавать там абсолютно любые вопросы}:\\
\begin{center}
\url{http://go.yurichev.com/17010}
\end{center}
 
\RU{Пожалуйста, присылайте мне информацию о замеченных ошибках 
(включая грамматические), и т.д.}
\EN{Please, also do not hesitate to send me any corrections 
(including grammar ones (you see how horrible my English is?)), etc.}\\
\\
\RU{Я много работаю над книгой, поэтому номера страниц, листингов, и т.д., очень часто меняются.}
\EN{I'm working on the book a lot, so the page, listings numbers, etc, are changing very rapidly.}
\RU{Пожалуйста, в своих письмах мне не ссылайтесь на номера страниц и листингов.}
\EN{Please, do not refer to page/listing numbers in your emails to me.}
\RU{Есть метод проще: сделайте скриншот страницы, затем в графическом редакторе подчеркните место, где вы видите
ошибку, и отправьте мне. Так я исправлю её намного быстрее.}
\EN{There is much simpler method: just make a screenshot of the page, then underline the place where you see the error in a graphics editor,
and send  me it. I'll fix it much faster in this manner.}
\RU{Ну а если вы знакомы с git и \LaTeX\, вы можете исправить ошибку прямо в исходных текстах:}\EN{And if you familiar with git and \LaTeX\, you can fix the error right in the source code:}\\
\href{http://go.yurichev.com/17089}{GitHub}.

\part*{\RU{Приложение}\EN{Appendix}\DE{Anhang}}
\appendix
\addcontentsline{toc}{part}{\RU{Приложение}\EN{Appendix}\DE{Anhang}}

% chapters
\EN{\section{x86}

\subsection{Terminology}

Common for 16-bit (8086/80286), 32-bit (80386, etc.), 64-bit.

\myindex{IEEE 754}
\myindex{MS-DOS}
\begin{description}
	\item[byte] 8-bit.
		The DB assembly directive is used for defining variables and arrays of bytes.
		Bytes are passed in the 8-bit part of registers: \TT{AL/BL/CL/DL/AH/BH/CH/DH/SIL/DIL/R*L}.
	\item[word] 16-bit. 
		DW assembly directive \dittoclosing.
		Words are passed in the 16-bit part of the registers:\\
			\TT{AX/BX/CX/DX/SI/DI/R*W}.
	\item[double word] (\q{dword}) 32-bit.
		DD assembly directive \dittoclosing.
		Double words are passed in registers (x86) or in the 32-bit part of registers (x64). 
		In 16-bit code, double words are passed in 16-bit register pairs.
	\item[quad word] (\q{qword}) 64-bit.
		DQ assembly directive \dittoclosing.
		In 32-bit environment, quad words are passed in 32-bit register pairs.
	\item[tbyte] (10 bytes) 80-bit or 10 bytes (used for IEEE 754 FPU registers).
	\item[paragraph] (16 bytes)---term was popular in MS-DOS environment.
\end{description}

\myindex{Windows!API}

Data types of the same width (BYTE, WORD, DWORD) are also the same in Windows \ac{API}.

\subsection{\IFRU{Регистры общего пользования}{General purpose registers}}

\IFRU{Ко многим регистрам можно обращаться как к частям размером в байт или 16-битное слово}
{It is possible to access many registers by byte or 16-bit word parts}.
\IFRU{Это всё\EMDASH{}наследие от более старых процессоров Intel (вплоть до 8-битного 8080),
все еще поддерживаемое для обратной совместимости}{It is all inheritance from older Intel CPUs (up to 8-bit 8080) 
still supported for backward compatibility}.
\IFRU{Например, в \ac{RISC} процессорах, такой возможности, как правило, нет}{For example, this feature
is usually not present in \ac{RISC} CPUs}.

\index{x86-64}
\IFRU{Регистры, имеющие префикс R- появились только в x86-64, а префикс E- ~--- в 80386}
{Registers prefixed with R- appeared in x86-84, and those prefixed with E-~---in 80386}.
\IFRU{Таким образом, R-регистры 64-битные, а E-регистры ~--- 32-битные}
{Thus, R-registers are 64-bit, and E-registers~---32-bit}.

\IFRU{В x86-64 добавили еще 8 \ac{GPR}: R8-R15}
{8 more \ac{GPR}'s were added in x86-86: R8-R15}.

N.B.: \IFRU{В документации от Intel, для обращения к самому младшему байту к имени регистра
нужно добавлять суффикс \IT{L}: \IT{R8L}, но \ac{IDA} называет эти регистры добавляя суффикс \IT{B}: \IT{R8B}}
{In the Intel manuals byte parts of these registers are prefixed by \IT{L}, e.g.: \IT{R8L}, but \ac{IDA}
names these registers by adding \IT{B} suffix, e.g.: \IT{R8B}}.

\subsubsection{RAX/EAX/AX/AL}
\RegTableOne{RAX}{EAX}{AX}{AH}{AL}

\ac{AKA} \IFRU{аккумулятор}{accumulator}.
\IFRU{Результат ф-ции обычно возвращается через этот регистр}
{The result of function if usually returned via this register}.

\subsubsection{RBX/EBX/BX/BL}
\RegTableOne{RBX}{EBX}{BX}{BH}{BL}

\subsubsection{RCX/ECX/CX/CL}
\RegTableOne{RCX}{ECX}{CX}{CH}{CL}

\ac{AKA} \IFRU{счетчик}{counter}: 
\IFRU{используется в этой роли в инструкциях с префиксом REP и в инструкциях сдвига}
{in this role it is used in REP prefixed instructions and also in shift instructions}
(SHL/SHR/RxL/RxR).

\subsubsection{RDX/EDX/DX/DL}
\RegTableOne{RDX}{EDX}{DX}{DH}{DL}

\subsubsection{RSI/ESI/SI/SIL}
\RegTableTwo{RSI}{ESI}{SI}{SIL}

\ac{AKA} ``source''. \IFRU{Используется как источник в инструкциях}{Used as source in the instructions} 
REP MOVSx, REP CMPSx.

\subsubsection{RDI/EDI/DI/DIL}
\RegTableTwo{RDI}{EDI}{DI}{DIL}

\ac{AKA} ``destination''. \IFRU{Используется как указатель на место назначения в инструкции}
{Used as a pointer to destination place in the instructions} REP MOVSx, REP STOSx.

\subsubsection{R8/R8D/R8W/R8L}
\RegTableThree{R8}{R8D}{R8W}{R8L}

\subsubsection{R9/R9D/R9W/R9L}
\RegTableThree{R9}{R9D}{R9W}{R9L}

\subsubsection{R10/R10D/R10W/R10L}
\RegTableThree{R10}{R10D}{R10W}{R10L}

\subsubsection{R11/R11D/R11W/R11L}
\RegTableThree{R11}{R11D}{R11W}{R11L}

\subsubsection{R12/R12D/R12W/R12L}
\RegTableThree{R12}{R12D}{R12W}{R12L}

\subsubsection{R13/R13D/R13W/R13L}
\RegTableThree{R13}{R13D}{R13W}{R13L}

\subsubsection{R14/R14D/R14W/R14L}
\RegTableThree{R14}{R14D}{R14W}{R14L}

\subsubsection{R15/R15D/R15W/R15L}
\RegTableThree{R15}{R15D}{R15W}{R15L}

\subsubsection{RSP/ESP/SP/SPL}
\RegTableTwo{RSP}{ESP}{SP}{SPL}

\ac{AKA} \gls{stack pointer}. \IFRU{Обычно всегда указывает на текущий стек, кроме тех случаев,
когда он не инициализирован}{Usually points to the current stack except those cases when it is not yet initialized}.

\subsubsection{RBP/EBP/BP/BPL}
\RegTableTwo{RBP}{EBP}{BP}{BPL}

\ac{AKA} frame pointer. \IFRU{Обычно используется для доступа к локальным переменным ф-ции и аргументам,
Больше о нем}
{Usually used for local variables and arguments of function accessing. More about it}: (\ref{stack_frame}).

\subsubsection{RIP/EIP/IP}

\begin{center}
\begin{tabular}{ | l | l | l | l | l | l | l | l | l |}
\hline
\RegHeader \\
\hline
\multicolumn{8}{ | c | }{RIP\textsuperscript{x64}} \\
\hline
\multicolumn{4}{ | c | }{} & \multicolumn{4}{ c | }{EIP} \\
\hline
\multicolumn{6}{ | c | }{} & \multicolumn{2}{ c | }{IP} \\
\hline
\end{tabular}
\end{center}

\ac{AKA} ``instruction pointer''
\footnote{\IFRU{Иногда называется так же}{Sometimes also called} ``program counter''}.
\IFRU{Обычно всегда указывает на исполняющуюся инструкцию}{Usually always points to the current instruction}.
\IFRU{Напрямую модифицировать
регистр нельзя, хотя можно делать так (что равноценно)}
{Cannot be modified, however, it is possible to do (which is equivalent to)}:

\begin{lstlisting}
mov eax...
jmp eax
\end{lstlisting}

\IFRU{Либо}{Or}:

\begin{lstlisting}
push val
ret
\end{lstlisting}

\subsubsection{CS/DS/ES/SS/FS/GS}

\IFRU{16-битные регистры, содержащие селектор кода}{16-bit registers containing code selector} (CS), 
\IFRU{данных}{data selector} (DS), \IFRU{стека}{stack selector} (SS).\\
\\
\index{TLS}
\index{Windows!TIB}
FS \InENRU win32 \IFRU{указывает на}{points to} \ac{TLS}, \IFRU{а в Linux на эту роль был выбран GS}
{GS took this role in Linux}.
\IFRU{Это сделано для более быстрого доступа к \ac{TLS} и прочим структурам там вроде \ac{TIB}}
{It is done for faster access to the \ac{TLS} and other structures like \ac{TIB}}.
\\
\IFRU{В прошлом эти регистры использовались как сегментные регистры}
{In the past, these registers were used as segment registers} (\ref{8086_memory_model}).

\subsubsection{\IFRU{Регистр флагов}{Flags register}}

\label{EFLAGS}
\ac{AKA} EFLAGS.

\begin{center}
\begin{tabular}{ | l | l | l | }
\hline
\headercolor{} \IFRU{Бит}{Bit} (\IFRU{маска}{mask}) &
\headercolor{} \IFRU{Аббревиатура}{Abbreviation} (\IFRU{значение}{meaning}) &
\headercolor{} \IFRU{Описание}{Description} \\
\hline
0 (1) & CF (Carry) & \RU{Флаг переноса.} \\
      &            & \IFRU{Инструкции}{The} CLC/STC/CMC \IFRU{используются}{instructions are used} \\
      &            & \IFRU{для установки/сброса/инвертирования этого флага}{for setting/resetting/toggling this flag} \\
\hline
2 (4) & PF (Parity) & \RU{Флаг четности }(\ref{parity_flag}). \\
\hline
4 (0x10) & AF (Adjust) & \\
\hline
6 (0x40) & ZF (Zero) & \IFRU{Выставляется в}{Setting to} 0 \\
         &           & \IFRU{если результат последней операции был}{if the last operation's result was} 0. \\
\hline
7 (0x80) & SF (Sign) & \RU{Флаг знака.} \\
\hline
8 (0x100) & TF (Trap) & \IFRU{Применяется при отладке}{Used for debugging}. \\
&         &             \IFRU{Если включен, то после исполнения каждой инструкции}{If turned on, an exception will be} \\
&         &             \IFRU{будет сгенерировано исключение}{generated after each instruction execution}. \\
\hline
9 (0x200) & IF (Interrupt enable) & \IFRU{Разрешены ли прерывания}{Are interrupts enabled}. \\
          &                       & \IFRU{Инструкции}{The} CLI/STI \IFRU{используются}{instructions are used} \\
	  &                       & \IFRU{для установки/сброса этого флага}{for the flag setting/resetting} \\
\hline
10 (0x400) & DF (Direction) & \IFRU{Задается направление для инструкций}{A directions is set for the} \\
           &                & REP MOVSx, REP CMPSx, REP LODSx, REP SCASx\EN{ instructions}.\\
           &                & \IFRU{Инструкции}{The} CLD/STD \IFRU{используются}{instructions are used} \\
	   &                & \IFRU{для установки/сброса этого флага}{for the flag setting/resetting} \\
\hline
11 (0x800) & OF (Overflow) & \RU{Переполнение.} \\
\hline
12, 13 (0x3000) & IOPL (I/O privilege level)\textsuperscript{80286} & \\
\hline
14 (0x4000) & NT (Nested task)\textsuperscript{80286} & \\
\hline
16 (0x10000) & RF (Resume)\textsuperscript{80386} & \IFRU{Применяется при отладке}{Used for debugging}. \\
             &                  & \IFRU{Если включить,}{CPU will ignore hardware breakpoint in DRx} \\
	     &                  & \IFRU{CPU проигнорирует хардварную точку останова в DRx}{if the flag is set}. \\
\hline
17 (0x20000) & VM (Virtual 8086 mode)\textsuperscript{80386} & \\
\hline
18 (0x40000) & AC (Alignment check)\textsuperscript{80486} & \\
\hline
19 (0x80000) & VIF (Virtual interrupt)\textsuperscript{Pentium} & \\
\hline
20 (0x100000) & VIP (Virtual interrupt pending)\textsuperscript{Pentium} & \\
\hline
21 (0x200000) & ID (Identification)\textsuperscript{Pentium} & \\
\hline
\end{tabular}
\end{center}

\IFRU{Остальные флаги зарезервированы}{All the rest flags are reserved}.

\subsection{FPU-\IFRU{регистры}{registers}}

\index{x86!FPU}
8 80-\IFRU{битных регистров работающих как стек}{bit registers working as a stack}: ST(0)-ST(7).
N.B.: \ac{IDA} \IFRU{называет}{calls} ST(0) \IFRU{просто}{as just} ST.
\IFRU{Числа хранятся в формате}{Numbers are stored in the} IEEE 754\EN{ format}.

\IFRU{Формат значения \IT{long double}}{\IT{long double} value format}:

\bigskip
% a hack used here! http://tex.stackexchange.com/questions/73524/bytefield-package
\begin{center}
\begingroup
\makeatletter
\let\saved@bf@bitformatting\bf@bitformatting
\renewcommand*{\bf@bitformatting}{%
	\ifnum\value{header@val}=21 %
	\value{header@val}=62 %
	\else\ifnum\value{header@val}=22 %
	\value{header@val}=63 %
	\else\ifnum\value{header@val}=23 %
	\value{header@val}=64 %
	\else\ifnum\value{header@val}=30 %
	\value{header@val}=78 %
	\else\ifnum\value{header@val}=31 %
	\value{header@val}=79 %
	\fi\fi\fi\fi\fi
	\saved@bf@bitformatting
}%
\begin{bytefield}{32}
	\bitheader[endianness=big]{0,21,22,23,30,31} \\
	\bitbox{1}{S} &
	\bitbox{8}{\IFRU{экспонента}{exponent}} &
	\bitbox{1}{I} &
	\bitbox{22}{\IFRU{мантисса}{mantissa or fraction}}
\end{bytefield}
\endgroup
\end{center}

\begin{center}
( S\EMDASH{}\IFRU{знак}{sign}, I\EMDASH{}\IFRU{целочисленная часть}{integer part} )
\end{center}

\label{FPU_control_word}
\subsubsection{\IFRU{Регистр управления}{Control Word}}

\IFRU{Регистр, при помощи которого можно задавать поведение}{Register controlling behaviour of the}
\ac{FPU}.

\begin{center}
\begin{tabular}{ | l | l | l | }
\hline
\IFRU{Бит}{Bit} &
\IFRU{Аббревиатура (значение)}{Abbreviation (meaning)} &
\IFRU{Описание}{Description} \\
\hline
0   & IM (Invalid operation Mask) & \\
\hline
1   & DM (Denormalized operand Mask) & \\
\hline
2   & ZM (Zero divide Mask) & \\
\hline
3   & OM (Overflow Mask) & \\
\hline
4   & UM (Underflow Mask) & \\
\hline
5   & PM (Precision Mask) & \\
\hline
7   & IEM (Interrupt Enable Mask) & \IFRU{Разрешение исключений, по умолчанию 1 (запрещено)}
{Exceptions enabling, 1 by default (disabled)} \\
\hline
8, 9 & PC (Precision Control) & \RU{Управление точностью} \\
     &                        & 00 ~--- 24 \IFRU{бита}{bits} (REAL4) \\
     &                        & 10 ~--- 53 \IFRU{бита}{bits} (REAL8) \\
     &                        & 11 ~--- 64 \IFRU{бита}{bits} (REAL10) \\
\hline
10, 11 & RC (Rounding Control) & \RU{Управление округлением} \\
       &                       & 00 ~--- \IFRU{(по умолчанию) округлять к ближайшему}{(by default) round to nearest} \\
       &                       & 01 ~--- \IFRU{округлять к}{round toward} $-\infty$ \\
       &                       & 10 ~--- \IFRU{округлять к}{round toward} $+\infty$ \\
       &                       & 11 ~--- \IFRU{округлять к}{round toward} $0$ \\
\hline
12 & IC (Infinity Control) & 0 ~--- (\IFRU{по умолчанию}{by default}) \IFRU{считать}{treat} $+\infty$ \AndENRU $-\infty$ \IFRU{за беззнаковое}{as unsigned} \\
   &                       & 1 ~--- \IFRU{учитывать и}{respect both} $+\infty$ \AndENRU $-\infty$ \\
\hline
\end{tabular}
\end{center}

\IFRU{Флагами}{The} PM, UM, OM, ZM, DM, IM 
\IFRU{задается, генерировать ли исключения в случае соответствующих ошибок}
{flags are defining if to generate exception in case of corresponding errors}.

\subsubsection{\IFRU{Регистр статуса}{Status Word}}

\label{FPU_status_word}
\IFRU{Регистр только для чтения}{Read-only register}.

\begin{center}
\begin{tabular}{ | l | l | l | }
\hline
\IFRU{Бит}{Bit} &
\IFRU{Аббревиатура (значение)}{Abbreviation (meaning)} &
\IFRU{Описание}{Description} \\
\hline
15   & B (Busy) & \IFRU{Работает ли сейчас FPU}{Is FPU do something} (1)
\IFRU{или закончил и результаты готовы}{or results are ready} (0) \\
\hline
14   & C3 & \\
\hline
13, 12, 11 & TOP & \IFRU{указывает, какой сейчас регистр является нулевым}
{points to the currently zeroth register} \\
\hline
10 & C2 & \\
\hline
9  & C1 & \\
\hline
8  & C0 & \\
\hline
7  & IR (Interrupt Request) & \\
\hline
6  & SF (Stack Fault) & \\
\hline
5  & P (Precision) & \\
\hline
4  & U (Underflow) & \\
\hline
3  & O (Overflow) & \\
\hline
2  & Z (Zero) & \\
\hline
1  & D (Denormalized) & \\
\hline
0  & I (Invalid operation) & \\
\hline
\end{tabular}
\end{center}

\IFRU{Биты}{The} SF, P, U, O, Z, D, I \IFRU{сигнализируют об исключениях}
{bits are signaling about exceptions}.

\IFRU{О}{About the} C3, C2, C1, C0 \IFRU{читайте больше}{read more}: (\ref{Czero_etc}).

N.B.: \IFRU{когда используется регистр ST(x), FPU прибавляет $x$ к TOP по модулю 8 и получается номер
внутреннего регистра}{When ST(x) is used, FPU adds $x$ to TOP (by modulo 8) and that is how it gets 
internal register's number}.

\subsubsection{Tag Word}

\IFRU{Этот регистр отражает текущее содержимое регистров чисел}
{The register has current information about number's registers usage}.

\begin{center}
\begin{tabular}{ | l | l | l | }
\hline
\IFRU{Бит}{Bit} & \IFRU{Аббревиатура (значение)}{Abbreviation (meaning)} \\
\hline
15, 14 & Tag(7) \\
\hline
13, 12 & Tag(6) \\
\hline
11, 10 & Tag(5) \\
\hline
9, 8 & Tag(4) \\
\hline
7, 6 & Tag(3) \\
\hline
5, 4 & Tag(2) \\
\hline
3, 2 & Tag(1) \\
\hline
1, 0 & Tag(0) \\
\hline
\end{tabular}
\end{center}

\IFRU{Для каждого тэга}{For each tag}:

\begin{itemize}
\item
00 ~--- \IFRU{Регистр содержит ненулевое значение}{The register contains a non-zero value}
\item
01 ~--- \IFRU{Регистр содержит 0}{The register contains 0}
\item
10 ~--- \IFRU{Регистр содержит специальное число}{The register contains a special value} 
(\ac{NAN}, $\infty$, \OrENRU \IFRU{денормализованное число}{denormal})
\item
11 ~--- \IFRU{Регистр пуст}{The register is empty}
\end{itemize}

\subsection{SIMD-\IFRU{регистры}{registers}}

\subsubsection{MMX-\IFRU{регистры}{registers}}

8 64-\IFRU{битных регистров}{bit registers}: MM0..MM7.

\subsubsection{SSE \AndENRU AVX-\IFRU{регистры}{registers}}

\index{x86-64}
SSE: 8 128-\IFRU{битных регистров}{bit registers}: XMM0..XMM7.
\IFRU{В}{In the} x86-64 \IFRU{добавлено еще 8 регистров}{8 more registers were added}: XMM8..XMM15.

AVX \IFRU{это расширение всех регистры до 256 бит}{is the extension of all these registers to 256 bits}.

\input{appendix/x86/DRx}

% TODO: control registers
 % subsection
\subsection{\IFRU{Инструкции}{Instructions}}

% ... instructions marked as (A)

\subsubsection{\IFRU{Наиболее часто используемые}{Most frequently used}}

\begin{description}
\index{x86!\Instructions!ADC}
  \item[ADC] (\IT{add with carry}) \IFRU{сложить два значения, \glslink{increment}{инкремент} 
  если выставлен флаг CF.
  часто используется для складывания больших значений, например, складывания двух 64-битных
  значений в 32-битной среде используя две инструкции ADD и ADC, например:}
  {add values, \gls{increment} result if CF flag is set.
  often used for addition of large values, for example, 
  to add two 64-bit values in 32-bit environment using two ADD and ADC instructions, for example:}

\lstinputlisting{appendix/x86/ADC_example_\IFRU{ru}{en}.lst}

\index{x86!\Instructions!ADD}
  \item[ADD] \IFRU{сложить два значения}{add two values}
\index{x86!\Instructions!AND}
  \item[AND] \IFRU{логическое ``И''}{logical ``and''}
\index{x86!\Instructions!CALL}
  \item[CALL] \IFRU{вызвать другую ф-цию}{call another function}: \TT{PUSH address\_after\_CALL\_instruction; JMP label}
\index{x86!\Instructions!CMP}
  \item[CMP] \IFRU{сравнение значений и установка флагов, то же что и \TT{SUB}, но только без записи результата}
\index{x86!\Instructions!DEC}
  \item[DEC] \gls{decrement}
\index{x86!\Instructions!IMUL} 
  \item[IMUL] \IFRU{умножение с учетом знаковых значений}{signed multiply}
\index{x86!\Instructions!INC} 
  \item[INC] \gls{increment}
\index{x86!\Instructions!JAE}
  \item[JAE] \IFRU{переход если больше или равно (беззнаковый)}{jump if above or equal (unsigned)}
\index{x86!\Instructions!JA}
  \item[JA] AKA JNBE: \IFRU{переход если больше (беззнаковый)}{jump if greater (unsigned)}
\index{x86!\Instructions!JBE}
  \item[JBE] \IFRU{переход если меньше или равно (беззнаковый)}{jump if lesser or equal (unsigned)}
\index{x86!\Instructions!JB}
  \item[JB] \IFRU{переход если меньше (беззнаковый)}{jump if below (unsigned)}
\index{x86!\Instructions!JE}
  \item[JE] \ac{AKA} JZ: \IFRU{переход если равно или ноль}{jump if equal or zero}
\index{x86!\Instructions!JGE}
  \item[JGE] \IFRU{переход если больше или равно (знаковый)}{jump if greater or equal (signed)}
\index{x86!\Instructions!JG}
  \item[JG] \IFRU{переход если больше (знаковый)}{jump if greater (signed)}
\index{x86!\Instructions!JLE}
  \item[JLE] \IFRU{переход если меньше или равно (знаковый)}{jump if lesser or equal (signed)}
\index{x86!\Instructions!JL}
  \item[JL] \IFRU{переход если меньше (знаковый)}{jump if lesser (signed)}
\index{x86!\Instructions!JMP}
  \item[JMP] \IFRU{перейти на другой адрес}{jump to another address}
\index{x86!\Instructions!JNE}
  \item[JNE] \ac{AKA} JNZ: \IFRU{переход если не равно или не ноль}{jump if not equal or not zero}
\index{x86!\Instructions!JP}
  \item[JP] \IFRU{переход если выставлен флаг PF}{jump if PF flag is set}
\index{x86!\Instructions!LEA}
  \item[LEA] \IFRU{сформировать адрес}{form address} \IFRU{см.также}{see also}: \ref{sec:LEA}
\index{\CStandardLibrary!memcpy()}
\index{x86!\Instructions!MOVSB}
\index{x86!\Instructions!MOVSW}
\index{x86!\Instructions!MOVSD}
\index{x86!\Instructions!MOVSQ}
  \item[MOVSB/MOVSW/MOVSD/MOVSQ] \IFRU{скопировать}{copy} CX/ECX/RCX \IFRU{байт}{bytes}/16-\IFRU{битных слов}{bit words}/32-\IFRU{битных слов}{bit words}/64-\IFRU{битных слов}{bit words} \IFRU{из}{from} SI/ESI/RSI \IFRU{в}{into} DI/EDI/RDI, \IFRU{работает как}{works like} memcpy() \IFRU{в Си}{in C}
\index{x86!\Instructions!MOVSX}
  \item[MOVSX] \IFRU{загрузить с расширением знака}{load with sign extension} \IFRU{см.также}{see also}: (\ref{MOVSX})
\index{x86!\Instructions!MOVZX}
  \item[MOVZX] \IFRU{загрузить и очистить все остальные биты}{load and clear all the rest bits} \IFRU{см.также}{see also}: (\ref{movzx})
\index{x86!\Instructions!MOV}
  \item[MOV] \IFRU{загрузить значение}{load value}. \IFRU{эта инструкция была названа неудачно, что является результатом путанницы: в других архитектурах эта же инструкция называется LOAD или что-то в этом роде}
  {this instruction was named awry resulting confusion: in other architectures the same instructions is usually named LOAD or something like that}.
\index{x86!\Instructions!MUL}
  \item[MUL] \IFRU{умножение с учетом беззнаковых значений}{unsigned multiply}
\index{x86!\Instructions!NOP}
  \item[NOP] \ac{NOP}
\index{x86!\Instructions!NOT}
  \item[NOT] \IFRU{логическое ``НЕ''}{logical inversion}
\index{x86!\Instructions!OR}
  \item[OR] \IFRU{логическое ``ИЛИ''}{logical ``or''}
\index{x86!\Instructions!POP}
  \item[POP] \IFRU{взять значение из стека}{get value from the stack}: \TT{value=SS:[ESP]; ESP=ESP+4 (or 8)}
\index{x86!\Instructions!PUSH}
  \item[PUSH] \IFRU{записать значение в стек}{push value to stack}: \TT{ESP=ESP-4 (or 8); SS:[ESP]=value}
\index{x86!\Instructions!RET}
  \item[RET] \ac{AKA} RETN: \IFRU{возврат из процедуры}{return from subroutine}: \TT{POP tmp; JMP tmp}
\index{x86!\Instructions!SAHF}
  \item[SAHF] \IFRU{скопировать биты из AH в флаги, см.также}{copy bits from AH to flags, see also}: \ref{SAHF}
\index{x86!\Instructions!SBB}
  \item[SBB] (\IT{subtraction with borrow}) 
  \IFRU{вычесть одно значение из другого, \glslink{decrement}{декремент} результата если флаг CF выставлен.
  часто используется для вычитания больших значений, например, для вычитания двух 64-битных
  значений в 32-битной среде используя инструкции SUB и SBB, например:}
  {subtract values, \gls{decrement} result if CF flag is set.
  often used for subtraction of large values, for example,
  to subtract two 64-bit values in 32-bit environment using two SUB and SBB instructions, for example:}

\lstinputlisting{appendix/x86/ADC_example_\IFRU{ru}{en}.lst}

\index{x86!\Instructions!SHL}
  \item[SHL] \IFRU{сдвинуть значение влево на один бит}{shift value left by one bit}
\index{x86!\Instructions!SHR}
  \item[SHR] \IFRU{сдвинуть значение вправо на один бит}{shift value right by one bit}
\index{x86!\Instructions!SUB}
  \item[SUB] \IFRU{вычесть одно значение из другого. часто встречающийся вариант \TT{SUB reg,reg} означает обнуление reg.}{subtract values. frequenly occured pattern \TT{SUB reg,reg} meaning write 0 to reg.}
\index{x86!\Instructions!TEST}
  \item[TEST] \IFRU{то же что и AND, но без записи результатов, см.также}{same as AND but without results saving, see also}: \ref{sec:bitfields}
\index{x86!\Instructions!XOR}
  \item[XOR] \ac{XOR} \IFRU{значений}{values}. \IFRU{часто встречающийся вариант}{frequenly occured pattern} \TT{XOR reg,reg} \IFRU{означает обнуление reg}{meaning write 0 to reg}.
  {compare values and set flags, the same as \TT{SUB} but no results writing}
\end{description}

\subsubsection{\IFRU{Реже используемые}{Less frequently used}}

\begin{description}
\index{x86!\Instructions!BSF}
  \item[BSF] \IT{bit scan forward}, \IFRU{см.также}{see also}: \ref{instruction_BSF}
\index{x86!\Instructions!CLC}
  \item[CLC] \IFRU{сбросить флаг CF}{clear CF flag}
\index{x86!\Instructions!CLD}
  \item[CLD] \IFRU{сбросить флаг DF}{clear DF flag}
\index{x86!\Instructions!CLI}
  \item[CLI] \IFRU{сбросить флаг IF}{clear IF flag}
\index{x86!\Instructions!CMC}
  \item[CMC] \IFRU{инвертировать флаг CF}{toggle CF flag}
\index{x86!\Instructions!CMOVcc}
  \item[CMOVcc] \IFRU{условный}{conditional} MOV: \IFRU{загрузить значение если условие верно}{load if condition is true}
  %\item[CMPSB/CMPSW/CMPSD/CMPSQ]
  %\item[CPUID]
  %\item[ENTER]
  %\item[IN]
  %\item[LEAVE]
  %\item[LES]
  %\item[LOOP]
  %\item[OUT]
  %\item[RCL]
  %\item[RCR]
  %\item[ROL]
  %\item[ROR]
\index{x86!\Instructions!SAL}
  \item[SAL] \IFRU{синонимично}{synonymous to} \TT{SHL}
  %\item[SAR]
  %\item[SETcc]
\index{x86!\Instructions!STC}
  \item[STC] \IFRU{установить флаг CF}{set CF flag}
\index{x86!\Instructions!STD}
  \item[STD] \IFRU{установить флаг DF}{set DF flag}
\index{x86!\Instructions!STI}
  \item[STI] \IFRU{установить флаг IF}{set IF flag}
  %\item[STOSx]
\index{x86!\Instructions!SYSENTER}
  \item[SYSENTER] \IFRU{вызов сисколла}{call syscall} (\ref{syscalls})
\end{description}

\iffalse
\subsubsection{\IFRU{Инструкции FPU}{FPU instructions}}

\begin{description}
  \item[FADDP]
  \item[FCOM]
  \item[FCOMP]
  \item[FDIV]
  \item[FDIVP]
  \item[FDIVR]
  \item[FLD]
  \item[FLD]
  \item[FMUL]
  \item[FNSTSW]
  \item[FSTP]
  \item[FUCOM]
  \item[FUCOMPP]
\end{description}

\subsubsection{\IFRU{SIMD-инструкции}{SIMD-instructions}}

\begin{description}
  \item[DIVSD]
  \item[MOVDQA]
  \item[MOVDQU]
  \item[PADDD]
  \item[PCMPEQB]
  \item[PLMULHW]
  \item[PLMULLD]
  \item[PMOVMSKB]
  \item[PXOR]
\end{description}
\fi

 % subsection
\subsection{npad}
\label{sec:npad}

\RU{Это макрос в ассемблере, для выравнивания некоторой метки по некоторой границе.}
\EN{It is an assembly language macro for aligning labels on a specific boundary.}

\RU{Это нужно для тех \IT{нагруженных} меток, куда чаще всего передается управление, например, 
начало тела цикла. 
Для того чтобы процессор мог эффективнее вытягивать данные или код из памяти, через шину с памятью, 
кэширование, итд.}
\EN{That's often needed for the busy labels to where the control flow is often passed, e.g., loop body starts.
So the CPU can load the data or code from the memory effectively, through the memory bus, cache lines, etc.}

\RU{Взято из}\EN{Taken from} \TT{listing.inc} (MSVC):

\myindex{x86!\Instructions!NOP}
\RU{Это, кстати, любопытный пример различных вариантов \NOP{}-ов. 
Все эти инструкции не дают никакого эффекта, но отличаются разной длиной.}
\EN{By the way, it is a curious example of the different \NOP variations.
All these instructions have no effects whatsoever, but have a different size.}

\RU{Цель в том, чтобы была только одна инструкция, а не набор NOP-ов, 
считается что так лучше для производительности CPU.}
\EN{Having a single idle instruction instead of couple of NOP-s,
is accepted to be better for CPU performance.}

\begin{lstlisting}[style=customasmx86]
;; LISTING.INC
;;
;; This file contains assembler macros and is included by the files created
;; with the -FA compiler switch to be assembled by MASM (Microsoft Macro
;; Assembler).
;;
;; Copyright (c) 1993-2003, Microsoft Corporation. All rights reserved.

;; non destructive nops
npad macro size
if size eq 1
  nop
else
 if size eq 2
   mov edi, edi
 else
  if size eq 3
    ; lea ecx, [ecx+00]
    DB 8DH, 49H, 00H
  else
   if size eq 4
     ; lea esp, [esp+00]
     DB 8DH, 64H, 24H, 00H
   else
    if size eq 5
      add eax, DWORD PTR 0
    else
     if size eq 6
       ; lea ebx, [ebx+00000000]
       DB 8DH, 9BH, 00H, 00H, 00H, 00H
     else
      if size eq 7
	; lea esp, [esp+00000000]
	DB 8DH, 0A4H, 24H, 00H, 00H, 00H, 00H 
      else
       if size eq 8
        ; jmp .+8; .npad 6
	DB 0EBH, 06H, 8DH, 9BH, 00H, 00H, 00H, 00H
       else
        if size eq 9
         ; jmp .+9; .npad 7
         DB 0EBH, 07H, 8DH, 0A4H, 24H, 00H, 00H, 00H, 00H
        else
         if size eq 10
          ; jmp .+A; .npad 7; .npad 1
          DB 0EBH, 08H, 8DH, 0A4H, 24H, 00H, 00H, 00H, 00H, 90H
         else
          if size eq 11
           ; jmp .+B; .npad 7; .npad 2
           DB 0EBH, 09H, 8DH, 0A4H, 24H, 00H, 00H, 00H, 00H, 8BH, 0FFH
          else
           if size eq 12
            ; jmp .+C; .npad 7; .npad 3
            DB 0EBH, 0AH, 8DH, 0A4H, 24H, 00H, 00H, 00H, 00H, 8DH, 49H, 00H
           else
            if size eq 13
             ; jmp .+D; .npad 7; .npad 4
             DB 0EBH, 0BH, 8DH, 0A4H, 24H, 00H, 00H, 00H, 00H, 8DH, 64H, 24H, 00H
            else
             if size eq 14
              ; jmp .+E; .npad 7; .npad 5
              DB 0EBH, 0CH, 8DH, 0A4H, 24H, 00H, 00H, 00H, 00H, 05H, 00H, 00H, 00H, 00H
             else
              if size eq 15
               ; jmp .+F; .npad 7; .npad 6
               DB 0EBH, 0DH, 8DH, 0A4H, 24H, 00H, 00H, 00H, 00H, 8DH, 9BH, 00H, 00H, 00H, 00H
              else
	       %out error: unsupported npad size
               .err
              endif
             endif
            endif
           endif
          endif
         endif
        endif
       endif
      endif
     endif
    endif
   endif
  endif
 endif
endif
endm
\end{lstlisting}
 % subsection

}\RU{\mysection{x86}

\subsection{Терминология}

Общее для 16-bit (8086/80286), 32-bit (80386, итд), 64-bit.

\myindex{IEEE 754}
\myindex{MS-DOS}
\begin{description}
	\item[byte] 8-бит. 
		Для определения переменных и массива байт используется директива ассемблера DB.
		Байты передаются в 8-битных частях регистров: \TT{AL/BL/CL/DL/AH/BH/CH/DH/SIL/DIL/R*L}.
	\item[word] 16-бит. \dittoclosing директива ассемблера DW.
		Слова передаются в 16-битных частях регистров: \\
			\TT{AX/BX/CX/DX/SI/DI/R*W}.
	\item[double word] (\q{dword}) 32-бит. \dittoclosing директива ассемблера DD.
		Двойные слова передаются в регистрах (x86) или в 32-битных частях регистров (x64). 
		В 16-битном коде, двойные слова передаются в парах 16-битных регистров.
	\item[quad word] (\q{qword}) 64-бит. \dittoclosing директива ассемблера DQ.
		В 32-битной среде, учетверенные слова передаются в парах 32-битных регистров.
	\item[tbyte] (10 байт) 80-бит или 10 байт (используется для регистров IEEE 754 FPU).
	\item[paragraph] (16 байт) --- термин был популярен в среде MS-DOS.
\end{description}

\myindex{Windows!API}
Типы данных с той же шириной (BYTE, WORD, DWORD) точно такие же и в Windows \ac{API}.

\subsection{\IFRU{Регистры общего пользования}{General purpose registers}}

\IFRU{Ко многим регистрам можно обращаться как к частям размером в байт или 16-битное слово}
{It is possible to access many registers by byte or 16-bit word parts}.
\IFRU{Это всё\EMDASH{}наследие от более старых процессоров Intel (вплоть до 8-битного 8080),
все еще поддерживаемое для обратной совместимости}{It is all inheritance from older Intel CPUs (up to 8-bit 8080) 
still supported for backward compatibility}.
\IFRU{Например, в \ac{RISC} процессорах, такой возможности, как правило, нет}{For example, this feature
is usually not present in \ac{RISC} CPUs}.

\index{x86-64}
\IFRU{Регистры, имеющие префикс R- появились только в x86-64, а префикс E- ~--- в 80386}
{Registers prefixed with R- appeared in x86-84, and those prefixed with E-~---in 80386}.
\IFRU{Таким образом, R-регистры 64-битные, а E-регистры ~--- 32-битные}
{Thus, R-registers are 64-bit, and E-registers~---32-bit}.

\IFRU{В x86-64 добавили еще 8 \ac{GPR}: R8-R15}
{8 more \ac{GPR}'s were added in x86-86: R8-R15}.

N.B.: \IFRU{В документации от Intel, для обращения к самому младшему байту к имени регистра
нужно добавлять суффикс \IT{L}: \IT{R8L}, но \ac{IDA} называет эти регистры добавляя суффикс \IT{B}: \IT{R8B}}
{In the Intel manuals byte parts of these registers are prefixed by \IT{L}, e.g.: \IT{R8L}, but \ac{IDA}
names these registers by adding \IT{B} suffix, e.g.: \IT{R8B}}.

\subsubsection{RAX/EAX/AX/AL}
\RegTableOne{RAX}{EAX}{AX}{AH}{AL}

\ac{AKA} \IFRU{аккумулятор}{accumulator}.
\IFRU{Результат ф-ции обычно возвращается через этот регистр}
{The result of function if usually returned via this register}.

\subsubsection{RBX/EBX/BX/BL}
\RegTableOne{RBX}{EBX}{BX}{BH}{BL}

\subsubsection{RCX/ECX/CX/CL}
\RegTableOne{RCX}{ECX}{CX}{CH}{CL}

\ac{AKA} \IFRU{счетчик}{counter}: 
\IFRU{используется в этой роли в инструкциях с префиксом REP и в инструкциях сдвига}
{in this role it is used in REP prefixed instructions and also in shift instructions}
(SHL/SHR/RxL/RxR).

\subsubsection{RDX/EDX/DX/DL}
\RegTableOne{RDX}{EDX}{DX}{DH}{DL}

\subsubsection{RSI/ESI/SI/SIL}
\RegTableTwo{RSI}{ESI}{SI}{SIL}

\ac{AKA} ``source''. \IFRU{Используется как источник в инструкциях}{Used as source in the instructions} 
REP MOVSx, REP CMPSx.

\subsubsection{RDI/EDI/DI/DIL}
\RegTableTwo{RDI}{EDI}{DI}{DIL}

\ac{AKA} ``destination''. \IFRU{Используется как указатель на место назначения в инструкции}
{Used as a pointer to destination place in the instructions} REP MOVSx, REP STOSx.

\subsubsection{R8/R8D/R8W/R8L}
\RegTableThree{R8}{R8D}{R8W}{R8L}

\subsubsection{R9/R9D/R9W/R9L}
\RegTableThree{R9}{R9D}{R9W}{R9L}

\subsubsection{R10/R10D/R10W/R10L}
\RegTableThree{R10}{R10D}{R10W}{R10L}

\subsubsection{R11/R11D/R11W/R11L}
\RegTableThree{R11}{R11D}{R11W}{R11L}

\subsubsection{R12/R12D/R12W/R12L}
\RegTableThree{R12}{R12D}{R12W}{R12L}

\subsubsection{R13/R13D/R13W/R13L}
\RegTableThree{R13}{R13D}{R13W}{R13L}

\subsubsection{R14/R14D/R14W/R14L}
\RegTableThree{R14}{R14D}{R14W}{R14L}

\subsubsection{R15/R15D/R15W/R15L}
\RegTableThree{R15}{R15D}{R15W}{R15L}

\subsubsection{RSP/ESP/SP/SPL}
\RegTableTwo{RSP}{ESP}{SP}{SPL}

\ac{AKA} \gls{stack pointer}. \IFRU{Обычно всегда указывает на текущий стек, кроме тех случаев,
когда он не инициализирован}{Usually points to the current stack except those cases when it is not yet initialized}.

\subsubsection{RBP/EBP/BP/BPL}
\RegTableTwo{RBP}{EBP}{BP}{BPL}

\ac{AKA} frame pointer. \IFRU{Обычно используется для доступа к локальным переменным ф-ции и аргументам,
Больше о нем}
{Usually used for local variables and arguments of function accessing. More about it}: (\ref{stack_frame}).

\subsubsection{RIP/EIP/IP}

\begin{center}
\begin{tabular}{ | l | l | l | l | l | l | l | l | l |}
\hline
\RegHeader \\
\hline
\multicolumn{8}{ | c | }{RIP\textsuperscript{x64}} \\
\hline
\multicolumn{4}{ | c | }{} & \multicolumn{4}{ c | }{EIP} \\
\hline
\multicolumn{6}{ | c | }{} & \multicolumn{2}{ c | }{IP} \\
\hline
\end{tabular}
\end{center}

\ac{AKA} ``instruction pointer''
\footnote{\IFRU{Иногда называется так же}{Sometimes also called} ``program counter''}.
\IFRU{Обычно всегда указывает на исполняющуюся инструкцию}{Usually always points to the current instruction}.
\IFRU{Напрямую модифицировать
регистр нельзя, хотя можно делать так (что равноценно)}
{Cannot be modified, however, it is possible to do (which is equivalent to)}:

\begin{lstlisting}
mov eax...
jmp eax
\end{lstlisting}

\IFRU{Либо}{Or}:

\begin{lstlisting}
push val
ret
\end{lstlisting}

\subsubsection{CS/DS/ES/SS/FS/GS}

\IFRU{16-битные регистры, содержащие селектор кода}{16-bit registers containing code selector} (CS), 
\IFRU{данных}{data selector} (DS), \IFRU{стека}{stack selector} (SS).\\
\\
\index{TLS}
\index{Windows!TIB}
FS \InENRU win32 \IFRU{указывает на}{points to} \ac{TLS}, \IFRU{а в Linux на эту роль был выбран GS}
{GS took this role in Linux}.
\IFRU{Это сделано для более быстрого доступа к \ac{TLS} и прочим структурам там вроде \ac{TIB}}
{It is done for faster access to the \ac{TLS} and other structures like \ac{TIB}}.
\\
\IFRU{В прошлом эти регистры использовались как сегментные регистры}
{In the past, these registers were used as segment registers} (\ref{8086_memory_model}).

\subsubsection{\IFRU{Регистр флагов}{Flags register}}

\label{EFLAGS}
\ac{AKA} EFLAGS.

\begin{center}
\begin{tabular}{ | l | l | l | }
\hline
\headercolor{} \IFRU{Бит}{Bit} (\IFRU{маска}{mask}) &
\headercolor{} \IFRU{Аббревиатура}{Abbreviation} (\IFRU{значение}{meaning}) &
\headercolor{} \IFRU{Описание}{Description} \\
\hline
0 (1) & CF (Carry) & \RU{Флаг переноса.} \\
      &            & \IFRU{Инструкции}{The} CLC/STC/CMC \IFRU{используются}{instructions are used} \\
      &            & \IFRU{для установки/сброса/инвертирования этого флага}{for setting/resetting/toggling this flag} \\
\hline
2 (4) & PF (Parity) & \RU{Флаг четности }(\ref{parity_flag}). \\
\hline
4 (0x10) & AF (Adjust) & \\
\hline
6 (0x40) & ZF (Zero) & \IFRU{Выставляется в}{Setting to} 0 \\
         &           & \IFRU{если результат последней операции был}{if the last operation's result was} 0. \\
\hline
7 (0x80) & SF (Sign) & \RU{Флаг знака.} \\
\hline
8 (0x100) & TF (Trap) & \IFRU{Применяется при отладке}{Used for debugging}. \\
&         &             \IFRU{Если включен, то после исполнения каждой инструкции}{If turned on, an exception will be} \\
&         &             \IFRU{будет сгенерировано исключение}{generated after each instruction execution}. \\
\hline
9 (0x200) & IF (Interrupt enable) & \IFRU{Разрешены ли прерывания}{Are interrupts enabled}. \\
          &                       & \IFRU{Инструкции}{The} CLI/STI \IFRU{используются}{instructions are used} \\
	  &                       & \IFRU{для установки/сброса этого флага}{for the flag setting/resetting} \\
\hline
10 (0x400) & DF (Direction) & \IFRU{Задается направление для инструкций}{A directions is set for the} \\
           &                & REP MOVSx, REP CMPSx, REP LODSx, REP SCASx\EN{ instructions}.\\
           &                & \IFRU{Инструкции}{The} CLD/STD \IFRU{используются}{instructions are used} \\
	   &                & \IFRU{для установки/сброса этого флага}{for the flag setting/resetting} \\
\hline
11 (0x800) & OF (Overflow) & \RU{Переполнение.} \\
\hline
12, 13 (0x3000) & IOPL (I/O privilege level)\textsuperscript{80286} & \\
\hline
14 (0x4000) & NT (Nested task)\textsuperscript{80286} & \\
\hline
16 (0x10000) & RF (Resume)\textsuperscript{80386} & \IFRU{Применяется при отладке}{Used for debugging}. \\
             &                  & \IFRU{Если включить,}{CPU will ignore hardware breakpoint in DRx} \\
	     &                  & \IFRU{CPU проигнорирует хардварную точку останова в DRx}{if the flag is set}. \\
\hline
17 (0x20000) & VM (Virtual 8086 mode)\textsuperscript{80386} & \\
\hline
18 (0x40000) & AC (Alignment check)\textsuperscript{80486} & \\
\hline
19 (0x80000) & VIF (Virtual interrupt)\textsuperscript{Pentium} & \\
\hline
20 (0x100000) & VIP (Virtual interrupt pending)\textsuperscript{Pentium} & \\
\hline
21 (0x200000) & ID (Identification)\textsuperscript{Pentium} & \\
\hline
\end{tabular}
\end{center}

\IFRU{Остальные флаги зарезервированы}{All the rest flags are reserved}.

\subsection{FPU-\IFRU{регистры}{registers}}

\index{x86!FPU}
8 80-\IFRU{битных регистров работающих как стек}{bit registers working as a stack}: ST(0)-ST(7).
N.B.: \ac{IDA} \IFRU{называет}{calls} ST(0) \IFRU{просто}{as just} ST.
\IFRU{Числа хранятся в формате}{Numbers are stored in the} IEEE 754\EN{ format}.

\IFRU{Формат значения \IT{long double}}{\IT{long double} value format}:

\bigskip
% a hack used here! http://tex.stackexchange.com/questions/73524/bytefield-package
\begin{center}
\begingroup
\makeatletter
\let\saved@bf@bitformatting\bf@bitformatting
\renewcommand*{\bf@bitformatting}{%
	\ifnum\value{header@val}=21 %
	\value{header@val}=62 %
	\else\ifnum\value{header@val}=22 %
	\value{header@val}=63 %
	\else\ifnum\value{header@val}=23 %
	\value{header@val}=64 %
	\else\ifnum\value{header@val}=30 %
	\value{header@val}=78 %
	\else\ifnum\value{header@val}=31 %
	\value{header@val}=79 %
	\fi\fi\fi\fi\fi
	\saved@bf@bitformatting
}%
\begin{bytefield}{32}
	\bitheader[endianness=big]{0,21,22,23,30,31} \\
	\bitbox{1}{S} &
	\bitbox{8}{\IFRU{экспонента}{exponent}} &
	\bitbox{1}{I} &
	\bitbox{22}{\IFRU{мантисса}{mantissa or fraction}}
\end{bytefield}
\endgroup
\end{center}

\begin{center}
( S\EMDASH{}\IFRU{знак}{sign}, I\EMDASH{}\IFRU{целочисленная часть}{integer part} )
\end{center}

\label{FPU_control_word}
\subsubsection{\IFRU{Регистр управления}{Control Word}}

\IFRU{Регистр, при помощи которого можно задавать поведение}{Register controlling behaviour of the}
\ac{FPU}.

\begin{center}
\begin{tabular}{ | l | l | l | }
\hline
\IFRU{Бит}{Bit} &
\IFRU{Аббревиатура (значение)}{Abbreviation (meaning)} &
\IFRU{Описание}{Description} \\
\hline
0   & IM (Invalid operation Mask) & \\
\hline
1   & DM (Denormalized operand Mask) & \\
\hline
2   & ZM (Zero divide Mask) & \\
\hline
3   & OM (Overflow Mask) & \\
\hline
4   & UM (Underflow Mask) & \\
\hline
5   & PM (Precision Mask) & \\
\hline
7   & IEM (Interrupt Enable Mask) & \IFRU{Разрешение исключений, по умолчанию 1 (запрещено)}
{Exceptions enabling, 1 by default (disabled)} \\
\hline
8, 9 & PC (Precision Control) & \RU{Управление точностью} \\
     &                        & 00 ~--- 24 \IFRU{бита}{bits} (REAL4) \\
     &                        & 10 ~--- 53 \IFRU{бита}{bits} (REAL8) \\
     &                        & 11 ~--- 64 \IFRU{бита}{bits} (REAL10) \\
\hline
10, 11 & RC (Rounding Control) & \RU{Управление округлением} \\
       &                       & 00 ~--- \IFRU{(по умолчанию) округлять к ближайшему}{(by default) round to nearest} \\
       &                       & 01 ~--- \IFRU{округлять к}{round toward} $-\infty$ \\
       &                       & 10 ~--- \IFRU{округлять к}{round toward} $+\infty$ \\
       &                       & 11 ~--- \IFRU{округлять к}{round toward} $0$ \\
\hline
12 & IC (Infinity Control) & 0 ~--- (\IFRU{по умолчанию}{by default}) \IFRU{считать}{treat} $+\infty$ \AndENRU $-\infty$ \IFRU{за беззнаковое}{as unsigned} \\
   &                       & 1 ~--- \IFRU{учитывать и}{respect both} $+\infty$ \AndENRU $-\infty$ \\
\hline
\end{tabular}
\end{center}

\IFRU{Флагами}{The} PM, UM, OM, ZM, DM, IM 
\IFRU{задается, генерировать ли исключения в случае соответствующих ошибок}
{flags are defining if to generate exception in case of corresponding errors}.

\subsubsection{\IFRU{Регистр статуса}{Status Word}}

\label{FPU_status_word}
\IFRU{Регистр только для чтения}{Read-only register}.

\begin{center}
\begin{tabular}{ | l | l | l | }
\hline
\IFRU{Бит}{Bit} &
\IFRU{Аббревиатура (значение)}{Abbreviation (meaning)} &
\IFRU{Описание}{Description} \\
\hline
15   & B (Busy) & \IFRU{Работает ли сейчас FPU}{Is FPU do something} (1)
\IFRU{или закончил и результаты готовы}{or results are ready} (0) \\
\hline
14   & C3 & \\
\hline
13, 12, 11 & TOP & \IFRU{указывает, какой сейчас регистр является нулевым}
{points to the currently zeroth register} \\
\hline
10 & C2 & \\
\hline
9  & C1 & \\
\hline
8  & C0 & \\
\hline
7  & IR (Interrupt Request) & \\
\hline
6  & SF (Stack Fault) & \\
\hline
5  & P (Precision) & \\
\hline
4  & U (Underflow) & \\
\hline
3  & O (Overflow) & \\
\hline
2  & Z (Zero) & \\
\hline
1  & D (Denormalized) & \\
\hline
0  & I (Invalid operation) & \\
\hline
\end{tabular}
\end{center}

\IFRU{Биты}{The} SF, P, U, O, Z, D, I \IFRU{сигнализируют об исключениях}
{bits are signaling about exceptions}.

\IFRU{О}{About the} C3, C2, C1, C0 \IFRU{читайте больше}{read more}: (\ref{Czero_etc}).

N.B.: \IFRU{когда используется регистр ST(x), FPU прибавляет $x$ к TOP по модулю 8 и получается номер
внутреннего регистра}{When ST(x) is used, FPU adds $x$ to TOP (by modulo 8) and that is how it gets 
internal register's number}.

\subsubsection{Tag Word}

\IFRU{Этот регистр отражает текущее содержимое регистров чисел}
{The register has current information about number's registers usage}.

\begin{center}
\begin{tabular}{ | l | l | l | }
\hline
\IFRU{Бит}{Bit} & \IFRU{Аббревиатура (значение)}{Abbreviation (meaning)} \\
\hline
15, 14 & Tag(7) \\
\hline
13, 12 & Tag(6) \\
\hline
11, 10 & Tag(5) \\
\hline
9, 8 & Tag(4) \\
\hline
7, 6 & Tag(3) \\
\hline
5, 4 & Tag(2) \\
\hline
3, 2 & Tag(1) \\
\hline
1, 0 & Tag(0) \\
\hline
\end{tabular}
\end{center}

\IFRU{Для каждого тэга}{For each tag}:

\begin{itemize}
\item
00 ~--- \IFRU{Регистр содержит ненулевое значение}{The register contains a non-zero value}
\item
01 ~--- \IFRU{Регистр содержит 0}{The register contains 0}
\item
10 ~--- \IFRU{Регистр содержит специальное число}{The register contains a special value} 
(\ac{NAN}, $\infty$, \OrENRU \IFRU{денормализованное число}{denormal})
\item
11 ~--- \IFRU{Регистр пуст}{The register is empty}
\end{itemize}

\subsection{SIMD-\IFRU{регистры}{registers}}

\subsubsection{MMX-\IFRU{регистры}{registers}}

8 64-\IFRU{битных регистров}{bit registers}: MM0..MM7.

\subsubsection{SSE \AndENRU AVX-\IFRU{регистры}{registers}}

\index{x86-64}
SSE: 8 128-\IFRU{битных регистров}{bit registers}: XMM0..XMM7.
\IFRU{В}{In the} x86-64 \IFRU{добавлено еще 8 регистров}{8 more registers were added}: XMM8..XMM15.

AVX \IFRU{это расширение всех регистры до 256 бит}{is the extension of all these registers to 256 bits}.

\input{appendix/x86/DRx}

% TODO: control registers
 % subsection
\subsection{\IFRU{Инструкции}{Instructions}}

% ... instructions marked as (A)

\subsubsection{\IFRU{Наиболее часто используемые}{Most frequently used}}

\begin{description}
\index{x86!\Instructions!ADC}
  \item[ADC] (\IT{add with carry}) \IFRU{сложить два значения, \glslink{increment}{инкремент} 
  если выставлен флаг CF.
  часто используется для складывания больших значений, например, складывания двух 64-битных
  значений в 32-битной среде используя две инструкции ADD и ADC, например:}
  {add values, \gls{increment} result if CF flag is set.
  often used for addition of large values, for example, 
  to add two 64-bit values in 32-bit environment using two ADD and ADC instructions, for example:}

\lstinputlisting{appendix/x86/ADC_example_\IFRU{ru}{en}.lst}

\index{x86!\Instructions!ADD}
  \item[ADD] \IFRU{сложить два значения}{add two values}
\index{x86!\Instructions!AND}
  \item[AND] \IFRU{логическое ``И''}{logical ``and''}
\index{x86!\Instructions!CALL}
  \item[CALL] \IFRU{вызвать другую ф-цию}{call another function}: \TT{PUSH address\_after\_CALL\_instruction; JMP label}
\index{x86!\Instructions!CMP}
  \item[CMP] \IFRU{сравнение значений и установка флагов, то же что и \TT{SUB}, но только без записи результата}
\index{x86!\Instructions!DEC}
  \item[DEC] \gls{decrement}
\index{x86!\Instructions!IMUL} 
  \item[IMUL] \IFRU{умножение с учетом знаковых значений}{signed multiply}
\index{x86!\Instructions!INC} 
  \item[INC] \gls{increment}
\index{x86!\Instructions!JAE}
  \item[JAE] \IFRU{переход если больше или равно (беззнаковый)}{jump if above or equal (unsigned)}
\index{x86!\Instructions!JA}
  \item[JA] AKA JNBE: \IFRU{переход если больше (беззнаковый)}{jump if greater (unsigned)}
\index{x86!\Instructions!JBE}
  \item[JBE] \IFRU{переход если меньше или равно (беззнаковый)}{jump if lesser or equal (unsigned)}
\index{x86!\Instructions!JB}
  \item[JB] \IFRU{переход если меньше (беззнаковый)}{jump if below (unsigned)}
\index{x86!\Instructions!JE}
  \item[JE] \ac{AKA} JZ: \IFRU{переход если равно или ноль}{jump if equal or zero}
\index{x86!\Instructions!JGE}
  \item[JGE] \IFRU{переход если больше или равно (знаковый)}{jump if greater or equal (signed)}
\index{x86!\Instructions!JG}
  \item[JG] \IFRU{переход если больше (знаковый)}{jump if greater (signed)}
\index{x86!\Instructions!JLE}
  \item[JLE] \IFRU{переход если меньше или равно (знаковый)}{jump if lesser or equal (signed)}
\index{x86!\Instructions!JL}
  \item[JL] \IFRU{переход если меньше (знаковый)}{jump if lesser (signed)}
\index{x86!\Instructions!JMP}
  \item[JMP] \IFRU{перейти на другой адрес}{jump to another address}
\index{x86!\Instructions!JNE}
  \item[JNE] \ac{AKA} JNZ: \IFRU{переход если не равно или не ноль}{jump if not equal or not zero}
\index{x86!\Instructions!JP}
  \item[JP] \IFRU{переход если выставлен флаг PF}{jump if PF flag is set}
\index{x86!\Instructions!LEA}
  \item[LEA] \IFRU{сформировать адрес}{form address} \IFRU{см.также}{see also}: \ref{sec:LEA}
\index{\CStandardLibrary!memcpy()}
\index{x86!\Instructions!MOVSB}
\index{x86!\Instructions!MOVSW}
\index{x86!\Instructions!MOVSD}
\index{x86!\Instructions!MOVSQ}
  \item[MOVSB/MOVSW/MOVSD/MOVSQ] \IFRU{скопировать}{copy} CX/ECX/RCX \IFRU{байт}{bytes}/16-\IFRU{битных слов}{bit words}/32-\IFRU{битных слов}{bit words}/64-\IFRU{битных слов}{bit words} \IFRU{из}{from} SI/ESI/RSI \IFRU{в}{into} DI/EDI/RDI, \IFRU{работает как}{works like} memcpy() \IFRU{в Си}{in C}
\index{x86!\Instructions!MOVSX}
  \item[MOVSX] \IFRU{загрузить с расширением знака}{load with sign extension} \IFRU{см.также}{see also}: (\ref{MOVSX})
\index{x86!\Instructions!MOVZX}
  \item[MOVZX] \IFRU{загрузить и очистить все остальные биты}{load and clear all the rest bits} \IFRU{см.также}{see also}: (\ref{movzx})
\index{x86!\Instructions!MOV}
  \item[MOV] \IFRU{загрузить значение}{load value}. \IFRU{эта инструкция была названа неудачно, что является результатом путанницы: в других архитектурах эта же инструкция называется LOAD или что-то в этом роде}
  {this instruction was named awry resulting confusion: in other architectures the same instructions is usually named LOAD or something like that}.
\index{x86!\Instructions!MUL}
  \item[MUL] \IFRU{умножение с учетом беззнаковых значений}{unsigned multiply}
\index{x86!\Instructions!NOP}
  \item[NOP] \ac{NOP}
\index{x86!\Instructions!NOT}
  \item[NOT] \IFRU{логическое ``НЕ''}{logical inversion}
\index{x86!\Instructions!OR}
  \item[OR] \IFRU{логическое ``ИЛИ''}{logical ``or''}
\index{x86!\Instructions!POP}
  \item[POP] \IFRU{взять значение из стека}{get value from the stack}: \TT{value=SS:[ESP]; ESP=ESP+4 (or 8)}
\index{x86!\Instructions!PUSH}
  \item[PUSH] \IFRU{записать значение в стек}{push value to stack}: \TT{ESP=ESP-4 (or 8); SS:[ESP]=value}
\index{x86!\Instructions!RET}
  \item[RET] \ac{AKA} RETN: \IFRU{возврат из процедуры}{return from subroutine}: \TT{POP tmp; JMP tmp}
\index{x86!\Instructions!SAHF}
  \item[SAHF] \IFRU{скопировать биты из AH в флаги, см.также}{copy bits from AH to flags, see also}: \ref{SAHF}
\index{x86!\Instructions!SBB}
  \item[SBB] (\IT{subtraction with borrow}) 
  \IFRU{вычесть одно значение из другого, \glslink{decrement}{декремент} результата если флаг CF выставлен.
  часто используется для вычитания больших значений, например, для вычитания двух 64-битных
  значений в 32-битной среде используя инструкции SUB и SBB, например:}
  {subtract values, \gls{decrement} result if CF flag is set.
  often used for subtraction of large values, for example,
  to subtract two 64-bit values in 32-bit environment using two SUB and SBB instructions, for example:}

\lstinputlisting{appendix/x86/ADC_example_\IFRU{ru}{en}.lst}

\index{x86!\Instructions!SHL}
  \item[SHL] \IFRU{сдвинуть значение влево на один бит}{shift value left by one bit}
\index{x86!\Instructions!SHR}
  \item[SHR] \IFRU{сдвинуть значение вправо на один бит}{shift value right by one bit}
\index{x86!\Instructions!SUB}
  \item[SUB] \IFRU{вычесть одно значение из другого. часто встречающийся вариант \TT{SUB reg,reg} означает обнуление reg.}{subtract values. frequenly occured pattern \TT{SUB reg,reg} meaning write 0 to reg.}
\index{x86!\Instructions!TEST}
  \item[TEST] \IFRU{то же что и AND, но без записи результатов, см.также}{same as AND but without results saving, see also}: \ref{sec:bitfields}
\index{x86!\Instructions!XOR}
  \item[XOR] \ac{XOR} \IFRU{значений}{values}. \IFRU{часто встречающийся вариант}{frequenly occured pattern} \TT{XOR reg,reg} \IFRU{означает обнуление reg}{meaning write 0 to reg}.
  {compare values and set flags, the same as \TT{SUB} but no results writing}
\end{description}

\subsubsection{\IFRU{Реже используемые}{Less frequently used}}

\begin{description}
\index{x86!\Instructions!BSF}
  \item[BSF] \IT{bit scan forward}, \IFRU{см.также}{see also}: \ref{instruction_BSF}
\index{x86!\Instructions!CLC}
  \item[CLC] \IFRU{сбросить флаг CF}{clear CF flag}
\index{x86!\Instructions!CLD}
  \item[CLD] \IFRU{сбросить флаг DF}{clear DF flag}
\index{x86!\Instructions!CLI}
  \item[CLI] \IFRU{сбросить флаг IF}{clear IF flag}
\index{x86!\Instructions!CMC}
  \item[CMC] \IFRU{инвертировать флаг CF}{toggle CF flag}
\index{x86!\Instructions!CMOVcc}
  \item[CMOVcc] \IFRU{условный}{conditional} MOV: \IFRU{загрузить значение если условие верно}{load if condition is true}
  %\item[CMPSB/CMPSW/CMPSD/CMPSQ]
  %\item[CPUID]
  %\item[ENTER]
  %\item[IN]
  %\item[LEAVE]
  %\item[LES]
  %\item[LOOP]
  %\item[OUT]
  %\item[RCL]
  %\item[RCR]
  %\item[ROL]
  %\item[ROR]
\index{x86!\Instructions!SAL}
  \item[SAL] \IFRU{синонимично}{synonymous to} \TT{SHL}
  %\item[SAR]
  %\item[SETcc]
\index{x86!\Instructions!STC}
  \item[STC] \IFRU{установить флаг CF}{set CF flag}
\index{x86!\Instructions!STD}
  \item[STD] \IFRU{установить флаг DF}{set DF flag}
\index{x86!\Instructions!STI}
  \item[STI] \IFRU{установить флаг IF}{set IF flag}
  %\item[STOSx]
\index{x86!\Instructions!SYSENTER}
  \item[SYSENTER] \IFRU{вызов сисколла}{call syscall} (\ref{syscalls})
\end{description}

\iffalse
\subsubsection{\IFRU{Инструкции FPU}{FPU instructions}}

\begin{description}
  \item[FADDP]
  \item[FCOM]
  \item[FCOMP]
  \item[FDIV]
  \item[FDIVP]
  \item[FDIVR]
  \item[FLD]
  \item[FLD]
  \item[FMUL]
  \item[FNSTSW]
  \item[FSTP]
  \item[FUCOM]
  \item[FUCOMPP]
\end{description}

\subsubsection{\IFRU{SIMD-инструкции}{SIMD-instructions}}

\begin{description}
  \item[DIVSD]
  \item[MOVDQA]
  \item[MOVDQU]
  \item[PADDD]
  \item[PCMPEQB]
  \item[PLMULHW]
  \item[PLMULLD]
  \item[PMOVMSKB]
  \item[PXOR]
\end{description}
\fi

 % subsection
\subsection{npad}
\label{sec:npad}

\RU{Это макрос в ассемблере, для выравнивания некоторой метки по некоторой границе.}
\EN{It is an assembly language macro for aligning labels on a specific boundary.}

\RU{Это нужно для тех \IT{нагруженных} меток, куда чаще всего передается управление, например, 
начало тела цикла. 
Для того чтобы процессор мог эффективнее вытягивать данные или код из памяти, через шину с памятью, 
кэширование, итд.}
\EN{That's often needed for the busy labels to where the control flow is often passed, e.g., loop body starts.
So the CPU can load the data or code from the memory effectively, through the memory bus, cache lines, etc.}

\RU{Взято из}\EN{Taken from} \TT{listing.inc} (MSVC):

\myindex{x86!\Instructions!NOP}
\RU{Это, кстати, любопытный пример различных вариантов \NOP{}-ов. 
Все эти инструкции не дают никакого эффекта, но отличаются разной длиной.}
\EN{By the way, it is a curious example of the different \NOP variations.
All these instructions have no effects whatsoever, but have a different size.}

\RU{Цель в том, чтобы была только одна инструкция, а не набор NOP-ов, 
считается что так лучше для производительности CPU.}
\EN{Having a single idle instruction instead of couple of NOP-s,
is accepted to be better for CPU performance.}

\begin{lstlisting}[style=customasmx86]
;; LISTING.INC
;;
;; This file contains assembler macros and is included by the files created
;; with the -FA compiler switch to be assembled by MASM (Microsoft Macro
;; Assembler).
;;
;; Copyright (c) 1993-2003, Microsoft Corporation. All rights reserved.

;; non destructive nops
npad macro size
if size eq 1
  nop
else
 if size eq 2
   mov edi, edi
 else
  if size eq 3
    ; lea ecx, [ecx+00]
    DB 8DH, 49H, 00H
  else
   if size eq 4
     ; lea esp, [esp+00]
     DB 8DH, 64H, 24H, 00H
   else
    if size eq 5
      add eax, DWORD PTR 0
    else
     if size eq 6
       ; lea ebx, [ebx+00000000]
       DB 8DH, 9BH, 00H, 00H, 00H, 00H
     else
      if size eq 7
	; lea esp, [esp+00000000]
	DB 8DH, 0A4H, 24H, 00H, 00H, 00H, 00H 
      else
       if size eq 8
        ; jmp .+8; .npad 6
	DB 0EBH, 06H, 8DH, 9BH, 00H, 00H, 00H, 00H
       else
        if size eq 9
         ; jmp .+9; .npad 7
         DB 0EBH, 07H, 8DH, 0A4H, 24H, 00H, 00H, 00H, 00H
        else
         if size eq 10
          ; jmp .+A; .npad 7; .npad 1
          DB 0EBH, 08H, 8DH, 0A4H, 24H, 00H, 00H, 00H, 00H, 90H
         else
          if size eq 11
           ; jmp .+B; .npad 7; .npad 2
           DB 0EBH, 09H, 8DH, 0A4H, 24H, 00H, 00H, 00H, 00H, 8BH, 0FFH
          else
           if size eq 12
            ; jmp .+C; .npad 7; .npad 3
            DB 0EBH, 0AH, 8DH, 0A4H, 24H, 00H, 00H, 00H, 00H, 8DH, 49H, 00H
           else
            if size eq 13
             ; jmp .+D; .npad 7; .npad 4
             DB 0EBH, 0BH, 8DH, 0A4H, 24H, 00H, 00H, 00H, 00H, 8DH, 64H, 24H, 00H
            else
             if size eq 14
              ; jmp .+E; .npad 7; .npad 5
              DB 0EBH, 0CH, 8DH, 0A4H, 24H, 00H, 00H, 00H, 00H, 05H, 00H, 00H, 00H, 00H
             else
              if size eq 15
               ; jmp .+F; .npad 7; .npad 6
               DB 0EBH, 0DH, 8DH, 0A4H, 24H, 00H, 00H, 00H, 00H, 8DH, 9BH, 00H, 00H, 00H, 00H
              else
	       %out error: unsupported npad size
               .err
              endif
             endif
            endif
           endif
          endif
         endif
        endif
       endif
      endif
     endif
    endif
   endif
  endif
 endif
endif
endm
\end{lstlisting}
 % subsection

}
\chapter{ARM}

\section{32-\RU{битный}\EN{bit} ARM (AArch32)}

\subsection{\RU{Регистры общего пользования}\EN{General purpose registers}}

% FIXME: make it tidy
\begin{itemize}
\index{ARM!\Registers!R0}
	\item R0 --- \RU{результат ф-ции обычно возвращается через R0}
		\EN{function result is usually returned using R0}
	\item R1
	\item R2
	\item R3
	\item R4
	\item R5
	\item R6
	\item R7
	\item R8
	\item R9
	\item R10
	\item R11
	\item R12
	\item R13 --- \ac{AKA} SP (\gls{stack pointer})
\index{ARM!\Registers!Link Register}
	\item R14 --- \ac{AKA} LR (\gls{link register})
	\item R15 --- \ac{AKA} PC (program counter)
\end{itemize}

\index{ARM!\Registers!scratch registers}
\Reg{0}-\Reg{3} \RU{называются также ``scratch registers'': аргументы ф-ции обычно передаются через них,
и эти значения не обязательно восстанавливать перед выходом из ф-ции}
\EN{are also called ``scratch registers'': function arguments are usually passed in them,
and values in them are not necessary to restore upon function exit}.

\subsection{Current Program Status Register (CPSR)}

\begin{center}
\begin{tabular}{ | l | l | }
\hline
\headercolor\ \RU{Бит}\EN{Bit} &
\headercolor\ \RU{Описание}\EN{Description} \\
\hline
0..4           & M --- processor mode \\
\hline
5              & T --- Thumb state \\
\hline
6              & F --- FIQ disable \\
\hline
7              & I --- IRQ disable \\
\hline
8              & A --- imprecise data abort disable \\
\hline
9              & E --- data endianness \\
\hline
10..15, 25, 26 & IT --- if-then state \\
\hline
16..19         & GE --- greater-than-or-equal-to \\
\hline
20..23         & DNM --- do not modify \\
\hline
24             & J --- Java state \\
\hline
27             & Q --- sticky overflow \\
\hline
28             & V --- overflow \\
\hline
29             & C --- carry/borrow/extend \\
\hline
\index{ARM!\Registers!Z}
30             & Z --- zero bit \\
\hline
31             & N --- negative/less than \\
\hline
\end{tabular}
\end{center}

% TODO
% \index{ARM!\Registers!APSR}
% \subsection{Application Program Status Register (APSR)}

% TODO
% \index{ARM!\Registers!FPSCR}
% \subsection{Floating-Point Status and Control Register (FPPSR)}
% http://infocenter.arm.com/help/index.jsp?topic=/com.arm.doc.ddi0344b/Chdfafia.html

\subsection{\RU{Регистры VPF (для чисел с плавающей точкой) и NEON}
\EN{VFP (floating point) and NEON registers}}

% http://infocenter.arm.com/help/index.jsp?topic=/com.arm.doc.dht0002a/ch01s03s02.html

\index{ARM!D-\RU{регистры}\EN{registers}}
\index{ARM!S-\RU{регистры}\EN{registers}}
\begin{center}
\begin{tabular}{ | l | l | l | l | }
\hline
0..31\textsuperscript{bits} & 32..64 & 65..96 & 97..127 \\
\hline
\multicolumn{4}{ | c | }{Q0\textsuperscript{128 bits}} \\
\hline
\multicolumn{2}{ | c | }{D0\textsuperscript{64 bits}} & \multicolumn{2}{ c | }{D1} \\
\hline
S0\textsuperscript{32 bits} & S1 & S2 & S3 \\
\hline
\end{tabular}
\end{center}

\RU{S-регистры 32-битные, используются для хранения чисел с одинарной точностью}
\EN{S-registers are 32-bit ones, used for single precision numbers storage}.

\RU{D-регистры 64-битные, используются для хранения чисел с двойной точностью}
\EN{D-registers are 64-bit ones, used for double precision numbers storage}.

\RU{D- и S-регистры занимают одно и то же место в памяти CPU --- 
можно обращаться к D-регистрам через S-регистры (хотя это и бессмысленно)}
\EN{D- and S-registers share the same physical space in CPU---it is possible to access 
D-register via S-registers (it is senseless though)}.

\RU{Точно также, \gls{NEON} Q-регистры имеют размер 128 бит и занимают то же физическое место 
в памяти CPU что и остальные регистры, предназначенные для чисел с плавающей точкой}
\EN{Likewise, \gls{NEON} Q-registers are 128-bit ones and share the same physical space in CPU 
with other floating point registers}.

\RU{В VFP присутствует 32 S-регистров: S0..S31}
\EN{In VFP 32 S-registers are present: S0..S31}.

\RU{В VPFv2 были добавлены 16 D-регистров, которые занимают то же место что и S0..S31}
\EN{In VFPv2 there are 16 D-registers added, which are, in fact, occupy the same space as S0..S31}.

\RU{В}\EN{In} VFPv3 (\gls{NEON} \OrENRU ``Advanced SIMD'') 
\RU{добавили еще 16 D-регистров, в итоге это D0..D31, но регистры D16..D31 не делят место
с другими S-регистрами}
\EN{there are 16 more D-registers added, resulting D0..D31, but D16..D31 registers are not 
sharing a space with other S-registers}.

\RU{В}\EN{In} \gls{NEON} \OrENRU ``Advanced SIMD'' \RU{были добавлены также 16 128-битных Q-регистров,
делящих место с регистрами D0..D31}
\EN{there are also 16 128-bit Q-registers added, which share the same space as D0..D31}.

\section{64-\RU{битный}\EN{bit} ARM (AArch64)}

\subsection{\RU{Регистры общего пользования}\EN{General purpose registers}}
\label{ARM64_GPRs}

\RU{Количество регистров было удвоено со времен}\EN{Register count was doubled since} AArch32.

\begin{itemize}
\index{ARM!\Registers!X0}
	\item X0\EMDASH{}\RU{результат ф-ции обычно возвращается через X0}
		\EN{function result is usually returned using X0}
        \item X0...X7\EMDASH{}\RU{Здесь передаются аргументы ф-ции}\EN{Function arguments are passed here}.
	\item X8
	\item X9...X15\EMDASH{}\RU{временные регистры, вызываемая ф-ция может их использовать и не восстанавливать 
их}\EN{are temporary registers, callee function may use it and not restore}.
	\item X16
	\item X17
	\item X18
	\item X19...X29\EMDASH{}\RU{вызываемая ф-ция может их использовать, но должна восстанавливать их по 
завершению}\EN{callee function may use, but should restore them upon exit}.
	\item X29\EMDASH{}\EN{used as}\RU{используется как} \ac{FP} (\EN{at least}\RU{как минимум в} GCC)
	\item X30\EMDASH{}``Procedure Link Register'' \ac{AKA} \ac{LR} (\gls{link register}).
	\item X31\EMDASH{}\EN{register always containing zero}\RU{регистр, всегда содержащий ноль}
\ac{AKA} XZR \OrENRU ``Zero Register''. \RU{Его 32-битная часть называется}\EN{It's 32-bit part called} WZR.
	\item \ac{SP}, \RU{больше не регистр общего пользования}\EN{not general register anymore}.
\end{itemize}

\RU{См.также}\EN{See also}: \cite{ARM64_PCS}.

\EN{32-bit part of each X-register is also accessible via W-registers}\RU{32-битная часть 
каждого X-регистра также доступна как W-регистр} (W0, W1, \RU{и т.д.}\EN{etc}).

\input{ARM_X0_register}
 % subsection
% assembly directives: DCB, DCW, DCD
%\subsection{\IFRU{Инструкции}{Instructions}}
% ADD
% ADDAL
% ADDCC
% ADDS
% ADR
% ADREQ
% ADRGT
% ADRHI
% ADRNE
% ASRS
% B
% BCS
% BEQ
% BGE
% BIC
% BL
% BLE
% BLEQ
% BLGT
% BLHI
% BLS
% BLT
% BLX
% BNE
% BX
% CMP
% IDIV
% IT
% LDMCSFD
% LDMEA
% LDMED
% LDMFA
% LDMFD
% LDMGEFD
% LDR.W
% LDR
% LDRB.W
% LDRB
% LDRSB
% LSL.W
% LSL
% LSLS
% MLA
% MOV
% MOVT.W
% MOVT
% MOVW
% MULS
% MVNS
% ORR
% POP
% PUSH
% RSB
% SMMUL
% STMEA
% STMED
% STMFA
% STMFD
% STMIA
% STMIB
% STR
% SUB
% SUBEQ
% SXTB
% TEST
% TST
% VADD
% VDIV
% VLDR
% VMOV
% VMOVGT
% VMRS
% VMUL
%\index{ARM!Optional operators!ASR
%\index{ARM!Optional operators!LSL
%\index{ARM!Optional operators!LSR
%\index{ARM!Optional operators!ROR
%\index{ARM!Optional operators!RRX

 % subsection

\ifx\RUSSIAN\undefined
\chapter{MIPS}

\section{Registers}
\label{MIPS_registers_ref}

\index{MIPS!O32}
( O32 Calling Convention )

\begin{center}
\begin{tabular}{ | l | l | l | }
\hline
\cellcolor{blue!25} Name & \cellcolor{blue!25} Pseudoname & \cellcolor{blue!25} Description \\
\hline
\$0             & \$zero          & Always zero. Writing to this register is effectively idle instruction (\ac{NOP}). \\
\hline
\$1             & \$at            & Used as a temporary register for assembly macros. \\
\hline
\$2 \dots \$3   & \$v0 \dots \$v1 & Function results returned here. \\
\hline
\$4 \dots \$7   & \$a0 \dots \$a3 & Function arguments. \\
\hline
\$8 \dots \$15  & \$t0 \dots \$t7 & Used for temporary data. \\
\hline
\$16 \dots \$23 & \$s0 \dots \$s7 & Used for temporary data\AsteriskOne{}. \\
\hline
\$24 \dots \$25 & \$t8 \dots \$t9 & Used for temporary data. \\
\hline
\$26 \dots \$27 & \$k0 \dots \$k1 & reserved for OS kernel. \\
\hline
\$28            & \$gp            & Global Pointer\AsteriskTwo{}. \\
\hline
\$29            & \$sp            & Stack Pointer\AsteriskOne{}. \\
\hline
\$30            & \$fp            & Frame Pointer\AsteriskOne{}. \\
\hline
\$31            & \$ra            & Return Address. \\
\hline
\end{tabular}
\end{center}

\AsteriskOne{} --- \Gls{callee} must preserve.\\
\AsteriskTwo{} --- \Gls{callee} must preserve (except \ac{PIC} code).\\

\iffalse
PC
HI/LO - holds results of mult/multu/div/divu

FPU:
$f0..$f30
\fi

\fi

\section{\IFRU{Некоторые библиотечные функции GCC}{Some GCC library functions}}
\index{GCC}
\label{sec:GCC_library_func}

%__ashldi3
%__ashrdi3
%__floatundidf
%__floatdisf
%__floatdixf
%__floatundidf
%__floatundisf
%__floatundixf
%__lshrdi3
%__muldi3

\begin{center}
\begin{tabular}{ | l | l | }
\hline
\cellcolor{blue!25} \IFRU{имя}{name} & \cellcolor{blue!25} \IFRU{значение}{meaning} \\
\hline \TT{\_\_divdi3} & \IFRU{знаковое деление}{signed division} \\
\hline \TT{\_\_moddi3} & \IFRU{остаток от знакового деления}{getting remainder (modulo) of signed division} \\
\hline \TT{\_\_udivdi3} & \IFRU{беззнаковое деление}{unsigned division} \\
\hline \TT{\_\_umoddi3} & \IFRU{остаток от беззнакового деления}{getting remainder (modulo) of unsigned division} \\
\hline
\end{tabular}
\end{center}



\mysection{\RU{Некоторые библиотечные функции MSVC}\EN{Some MSVC library functions}\DE{Einige MSVC-Bibliotheks-Funktionen}}
\myindex{MSVC}
\label{sec:MSVC_library_func}

\TT{ll} \RU{в имени функции означает}\EN{in function name stands for}\DE{in Funktionsnamen steht für} \q{long long}, \RU{т.е. 64-битный тип данных}
\EN{e.g., a 64-bit data type}\DE{z.B. einen 64-Bit-Datentyp}.

\begin{center}
\begin{tabular}{ | l | l | }
\hline
\HeaderColor \RU{имя}\EN{name}\DE{Name} & \HeaderColor \RU{значение}\EN{meaning}\DE{Bedeutung} \\
\hline \TT{\_\_alldiv} & \RU{знаковое деление}\EN{signed division}\DE{vorzeichenbehaftete Division} \\
\hline \TT{\_\_allmul} & \RU{умножение}\EN{multiplication}\DE{Multiplikation} \\
\hline \TT{\_\_allrem} & \RU{остаток от знакового деления}\EN{remainder of signed division}\DE{Rest einer vorzeichenbehafteten Division} \\
\hline \TT{\_\_allshl} & \RU{сдвиг влево}\EN{shift left}\DE{Schiebe links} \\
\hline \TT{\_\_allshr} & \RU{знаковый сдвиг вправо}\EN{signed shift right}\DE{Schiebe links, vorzeichenbehaftet} \\
\hline \TT{\_\_aulldiv} & \RU{беззнаковое деление}\EN{unsigned division}\DE{vorzeichenlose Division} \\
\hline \TT{\_\_aullrem} & \RU{остаток от беззнакового деления}\EN{remainder of unsigned division}\DE{Rest (Modulo) einer vorzeichenlosen Division} \\
\hline \TT{\_\_aullshr} & \RU{беззнаковый сдвиг вправо}\EN{unsigned shift right}\DE{Schiebe rechts, vorzeichenlos} \\
\hline
\end{tabular}
\end{center}

\RU{Процедуры умножения и сдвига влево, одни и те же и для знаковых чисел, и для беззнаковых,
поэтому здесь только одна функция для каждой операции}
\EN{Multiplication and shift left procedures are the same for both signed and unsigned numbers, hence there is only one function 
for each operation here}.
\DE{Multiplikation und Links-Schiebebefehle sind sowohl für vorzeichenbehaftete als auch vorzeichenlose Zahlen,
da hier für jede Operation nur ein Befehl existiert}. \\
\\
\RU{Исходные коды этих функций можно найти в установленной \ac{MSVS}, в}\EN{The source code of these function
can be found in the installed \ac{MSVS}, in} \TT{VC/crt/src/intel/*.asm}
\DE{Der Quellcode dieser Funktionen kann im Pfad des installierten \ac{MSVS}, gefunden werden: } \TT{VC/crt/src/intel/*.asm}.


\chapter{Cheatsheets}

% sections
\subsection{IDA}
\myindex{IDA}
\label{sec:IDA_cheatsheet}

\ShortHotKeyCheatsheet:

\begin{center}
\begin{tabular}{ | l | l | }
\hline
\HeaderColor \RU{клавиша}\EN{key}\DE{Taste} & \HeaderColor \RU{значение}\EN{meaning}\DE{Bedeutung} \\
\hline
Space 	& \RU{переключать между листингом и просмотром кода в виде графа}
            \EN{switch listing and graph view}
            \DE{Zwischen Quellcode und grafischer Ansicht wechseln}\\
C 	& \RU{конвертировать в код}\EN{convert to code}\DE{zu Code konvertieren} \\
D 	& \RU{конвертировать в данные}\EN{convert to data}\DE{zu Daten konvertieren} \\
A 	& \RU{конвертировать в строку}\EN{convert to string}\DE{zu Zeichenkette konvertieren} \\
* 	& \RU{конвертировать в массив}\EN{convert to array}\DE{zu Array konvertieren} \\
U 	& \RU{сделать неопределенным}\EN{undefine}\DE{undefinieren}\\
O 	& \RU{сделать смещение из операнда}\EN{make offset of operand}\DE{Offset von Operanden}\\
H 	& \RU{сделать десятичное число}\EN{make decimal number}\DE{Dezimalzahl erstellen} \\
R 	& \RU{сделать символ}\EN{make char}\DE{Zeichen erstellen} \\
B 	& \RU{сделать двоичное число}\EN{make binary number}\DE{Binärzahl erstellen} \\
Q 	& \RU{сделать шестнадцатеричное число}\EN{make hexadecimal number}\DE{Hexadezimalzahl erstellen} \\
N 	& \RU{переименовать идентификатор}\EN{rename identifier}\DE{Identifikator umbenennen} \\
? 	& \RU{калькулятор}\EN{calculator}\DE{Rechner} \\
G 	& \RU{переход на адрес}\EN{jump to address}\DE{zu Adresse springen} \\
: 	& \RU{добавить комментарий}\EN{add comment}\DE{Kommentar einfügen} \\
Ctrl-X 	& \RU{показать ссылки на текущую функцию, метку, переменную}
		\EN{show references to the current function, label, variable }
        \DE{Referenz zu aktueller Funktion, Variable, ... zeigen}\\
	& \RU{(в т.ч., в стеке)}\EN{(incl. in local stack)}\DE{(inkl. lokalem Stack)} \\
X 	& \RU{показать ссылки на функцию, метку, переменную, итд}\EN{show references to the function, label, variable, etc.}
        \DE{Referenz zu Funktion, Variable, ... zeigen}\\
Alt-I 	& \RU{искать константу}\EN{search for constant}\DE{Konstante suchen} \\
Ctrl-I 	& \RU{искать следующее вхождение константы}\EN{search for the next occurrence of constant}\DE{Nächstes Auftreten der Konstante suchen} \\
Alt-B 	& \RU{искать последовательность байт}\EN{search for byte sequence}\DE{Byte-Sequenz suchen} \\
Ctrl-B 	& \RU{искать следующее вхождение последовательности байт}
		\EN{search for the next occurrence of byte sequence}
        \DE{Nächstes Auftreten der Byte-Sequenz suchen} \\
Alt-T 	& \RU{искать текст (включая инструкции, итд.)}\EN{search for text (including instructions, etc.)}\EN{Text suchen (inkl. Anweisungen, usw.)} \\
Ctrl-T 	& \RU{искать следующее вхождение текста}\EN{search for the next occurrence of text}\DE{nächstes Aufreten des Textes suchen} \\
Alt-P 	& \RU{редактировать текущую функцию}\EN{edit current function}\DE{akutelle Funktion editieren} \\
		Enter 	& \RU{перейти к функции, переменной, итд.}\EN{jump to function, variable, etc.}\DE{zu Funktion, Variable, ... springen} \\
Esc 	& \RU{вернуться назад}\EN{get back}\DE{zurückgehen} \\
Num -   & \RU{свернуть функцию или отмеченную область}\EN{fold function or selected area}\DE{Funktion oder markierten Bereich einklappen} \\
Num + 	& \RU{снова показать функцию или область}\EN{unhide function or area}\DE{Funktion oder Bereich anzeigen}\\
\hline
\end{tabular}
\end{center}

\RU{Сворачивание функции или области может быть удобно чтобы прятать те части функции,
чья функция вам стала уже ясна}
\EN{Function/area folding may be useful for hiding function parts when you realize what they do}.
\DE{Das Einklappen ist nützlich um Teile von Funktionen zu verstecken, wenn bekannt ist was sie tun}.
\RU{это используется в моем скрипте\footnote{\href{\YurichevIDAIDCScripts}{GitHub}}}\EN{this is used in my}\DE{dies wird genutzt im}
\RU{для сворачивания некоторых очень часто используемых фрагментов inline-кода}
\EN{script\footnote{\href{\YurichevIDAIDCScripts}{GitHub}} for hiding some often used patterns of inline code}.
\DE{Script\footnote{\href{\YurichevIDAIDCScripts}{GitHub}} um häufig genutzte Inline-Code-Stellen zu verstecken}.


\ifdefined\IncludeOlly
\subsection{\olly}
\myindex{\olly}
\label{sec:Olly_cheatsheet}

\ShortHotKeyCheatsheet:

\begin{center}
\begin{tabular}{ | l | l | }
\hline
\HeaderColor \RU{хот-кей}\EN{hot-key}\DE{Tastenkürzel} & 
\HeaderColor \RU{значение}\EN{meaning}\DE{Bedeutung} \\
\hline
F7	& \RU{трассировать внутрь}\EN{trace into}\DE{Schritt}\\
F8	& \stepover\\
F9	& \RU{запуск}\EN{run}\DE{starten}\\
Ctrl-F2	& \RU{перезапуск}\EN{restart}\DE{Neustart}\\
\hline
\end{tabular}
\end{center}

\fi
\section{MSVC}
\index{MSVC}
\label{sec:MSVC_options}

\IFRU{Некоторые полезные опции, которые я использовал в книге}
{Some useful options I used through this book}.

\begin{center}
\begin{tabular}{ | l | l | }
\hline
\cellcolor{blue!25} \IFRU{опция}{option} & 
\cellcolor{blue!25} \IFRU{значение}{meaning} \\
\hline
/O1		& \IFRU{оптимизация по размеру кода}{minimize space}\\
/Ob0		& \IFRU{не заменять вызовы inline-ф-ций их кодом}{no inline expansion}\\
/Ox		& \IFRU{максимальная оптимизация}{maximum optimizations}\\
/GS-		& \IFRU{отключить проверки переполнений буфера}
		{disable security checks (buffer overflows)}\\
/Fa(file)	& \IFRU{генерировать листинг на ассемблере}{generate assembly listing}\\
/Zi		& \IFRU{генерировать отладочную информацию}{enable debugging information}\\
/Zp(n)		& \IFRU{паковать структуры по границе в $n$ байт}{pack structs on $n$-byte boundary}\\
/MD		& \IFRU{выходной исполняемый файл будет использовать}
			{produced executable will use} \TT{MSVCR*.DLL}\\
\hline
\end{tabular}
\end{center}


\subsection{GCC}
\myindex{GCC}

\RU{Некоторые полезные опции, которые были использованы в книге.}%
\EN{Some useful options which were used through this book.}
\DE{Einige nützliche Optionen die in diesem Buch genutzt werden.}

\begin{center}
\begin{tabular}{ | l | l | }
\hline
\HeaderColor \RU{опция}\EN{option}\DE{Option} & 
\HeaderColor \RU{значение}\EN{meaning}\DE{Bedeutung} \\
\hline
-Os		& \RU{оптимизация по размеру кода}\EN{code size optimization}\DE{Optimierung der Code-Größe} \\
-O3		& \RU{максимальная оптимизация}\EN{maximum optimization}\DE{maximale Optimierung} \\
-regparm=	& \RU{как много аргументов будет передаваться через регистры}
			\EN{how many arguments are to be passed in registers}\DE{Anzahl der in Registern übergebenen Argumente} \\
-o file		& \RU{задать имя выходного файла}\EN{set name of output file}\DE{Name der Ausgabedatei} \\
-g		& \RU{генерировать отладочную информацию в итоговом исполняемом файле}
			\EN{produce debugging information in resulting executable}
            \DE{Debug-Informationen in der ausführbaren Datei erzeugen} \\
-S		& \RU{генерировать листинг на ассемблере}
			\EN{generate assembly listing file}
            \DE{Assembler-Quellcode erstellen}\\
-masm=intel	& \RU{генерировать листинг в синтаксисе Intel}\EN{produce listing in Intel syntax}\DE{Quellcode im Intel-Syntax erstellen} \\
-fno-inline	& \RU{не вставлять тело функции там, где она вызывается}\EN{do not inline functions}\DE{keine Inline-Funktionen verwenden} \\
\hline
\end{tabular}
\end{center}



\ifdefined\IncludeGDB
\subsection{GDB}
\myindex{GDB}
\label{sec:GDB_cheatsheet}

\RU{Некоторые команды, которые были использованы в книге}\EN{Some of commands we used in this book}\DE{Einige nützliche Optionen die in diesem Buch genutzt werden}:

\small
\begin{center}
\begin{tabular}{ | l | l | }
\hline
\HeaderColor \RU{опция}\EN{option}\DE{Option} & 
\HeaderColor \RU{значение}\EN{meaning}\DE{Bedeutung} \\
\hline
break filename.c:number		& \RU{установить точку останова на номере строки в исходном файле}
					\EN{set a breakpoint on line number in source code}
                    \DE{Setzen eines Breakpoints in der angegebenen Zeile}\\
break function			& \RU{установить точку останова на функции}\EN{set a breakpoint on function}\DE{Setzen eines Breakpoints in der Funktion} \\
break *address			& \RU{установить точку останова на адресе}\EN{set a breakpoint on address}\DE{Setzen eines Breakpoints auf Adresse} \\
b				& \dittoclosing \\
p variable			& \RU{вывести значение переменной}\EN{print value of variable}\DE{Ausgabe eines Variablenwerts} \\
run				& \RU{запустить}\EN{run}\DE{Starten} \\
r				& \dittoclosing \\
cont				& \RU{продолжить исполнение}\EN{continue execution}\DE{Ausführung fortfahren} \\
c				& \dittoclosing \\
bt				& \RU{вывести стек}\EN{print stack}\DE{Stack ausgeben} \\
set disassembly-flavor intel	& \RU{установить Intel-синтаксис}\EN{set Intel syntax}\DE{Intel-Syntax nutzen} \\
disas				& disassemble current function \\
disas function			& \RU{дизассемблировать функцию}\EN{disassemble function}\DE{Funktion disassemblieren} \\
disas function,+50		& disassemble portion \\
disas \$eip,+0x10		& \dittoclosing \\
disas/r				& \EN{disassemble with opcodes}\RU{дизассемблировать с опкодами}\DE{mit OpCodes disassemblieren} \\
info registers			& \RU{вывести все регистры}\EN{print all registers}\DE{Ausgabe aller Register} \\
info float			& \RU{вывести FPU-регистры}\EN{print FPU-registers}\DE{Ausgabe der FPU-Register} \\
info locals			& \RU{вывести локальные переменные (если известны)}\EN{dump local variables (if known)}\DE{(bekannte) lokale Variablen ausgeben} \\
x/w ...				& \RU{вывести память как 32-битные слова}\EN{dump memory as 32-bit word}\DE{Speicher als 32-Bit-Wort ausgeben} \\
x/w \$rdi			& \RU{вывести память как 32-битные слова}\EN{dump memory as 32-bit word}\DE{Speicher als 32-Bit-Wort ausgeben} \\
				& \RU{по адресу в \TT{RDI}}\EN{at address in \TT{RDI}}\DE{an Adresse in \TT{RDI}} \\

x/10w ...			& \RU{вывести 10 слов памяти}\EN{dump 10 memory words}\DE{10 Speicherworte ausgeben} \\
x/s ...				& \RU{вывести строку из памяти}\EN{dump memory as string}\DE{Speicher als Zeichenkette ausgeben} \\
x/i ...				& \RU{трактовать память как код}\EN{dump memory as code}\DE{Speicher als Code ausgeben} \\
x/10c ...			& \RU{вывести 10 символов}\EN{dump 10 characters}\DE{10 Zeichen ausgeben} \\
x/b ...				& \RU{вывести байты}\EN{dump bytes}\DE{Bytes ausgeben} \\
x/h ...				& \RU{вывести 16-битные полуслова}\EN{dump 16-bit halfwords}\DE{16-Bit-Halbworte ausgeben} \\
x/g ...				& \RU{вывести 64-битные слова}\EN{dump giant (64-bit) words}\DE{große (64-Bit-) Worte ausgeben} \\
finish				& \RU{исполнять до конца функции}\EN{execute till the end of function}\DE{bis Funktionsende fortfahren} \\
next				& \RU{следующая инструкция (не заходить в функции)}
					\EN{next instruction (don't dive into functions)}
                    \DE{Nächste Anweisung (nicht in Funktion springen)} \\
step				& \RU{следующая инструкция (заходить в функции)}
					\EN{next instruction (dive into functions)}
                    \DE{Nächste Anweisung (in Funktion springen)} \\
set step-mode on		& \RU{не использовать информацию о номерах строк при использовании команды step}
					\EN{do not use line number information while stepping}
                    \DE{Beim schrittweisen Ausführen keine Zeilennummerninfos nutzen} \\
frame n				& \RU{переключить фрейм стека}\EN{switch stack frame}\DE{Stack-Frame tauschen} \\
info break			& \RU{список точек останова}\EN{list of breakpoints} \\
del n				& \RU{удалить точку останова}\EN{delete breakpoint}\DE{Breakpoints löschen} \\
set args ...			& \RU{установить аргументы командной строки}\EN{set command-line arguments}\DE{Aufrufparameter setzen} \\
\hline
\end{tabular}
\end{center}
\normalsize


\fi


\part*{%
	\RU{Список принятых сокращений}%
	\EN{Acronyms used}%
	\NL{Gebruikte afkortingen}%
	\ES{Acr\'onimos utilizados}%
	\PTBRph{}%
	\DE{Verwendete Abkürzungen}%
	\PLph{}%
	\ITAph{}%
	\FRph{}
}
\addcontentsline{toc}{part}{%
	\RU{Список принятых сокращений}%
	\EN{Acronyms used}%
	\ES{Acr\'onimos utilizados}%
	\NL{Gebruikte afkortingen}%
	\PTBRph{}%
	\DE{Verwendete Abkürzungen}%
	\PLph{}%
	\ITAph{}%
	\FRph{Accronymes utilisés}
}
\begin{acronym}
\RU{
	\acro{OS}[ОС]{Операционная Система}
	\acro{FAQ}[ЧаВО]{Часто задаваемые вопросы}
	\acro{OOP}[ООП]{Объектно-Ориентированное Программирование}
	\acro{PL}[ЯП]{Язык Программирования}
	\acro{PRNG}[ГПСЧ]{Генератор псевдослучайных чисел}
	\acro{ROM}[ПЗУ]{Постоянное запоминающее устройство}
	\acro{ALU}[АЛУ]{Арифметико-логическое устройство}
	\acro{PID}{ID программы/процесса}
	\acro{LF}{Line feed (подача строки) (10 или '\textbackslash{}n' в \CCpp)}
	\acro{CR}{Carriage return (возврат каретки) (13 или '\textbackslash{}r' в \CCpp)}
	\acro{LIFO}{Last In First Out (последним вошел, первым вышел)}
}%
\EN{
	\acro{OS}{Operating System}
	\acro{FAQ}{Frequently Asked Questions}
	\acro{OOP}{Object-Oriented Programming}
	\acro{PL}{Programming language}
	\acro{PRNG}{Pseudorandom number generator}
	\acro{ROM}{Read-only memory}
	\acro{ALU}{Arithmetic logic unit}
	\acro{PID}{Program/process ID}
	\acro{LF}{Line feed (10 or '\textbackslash{}n' in \CCpp)}
	\acro{CR}{Carriage return (13 or '\textbackslash{}r' in \CCpp)}
	\acro{LIFO}{Last In First Out}
}%
\ES{
	\acro{OS}[SO]{\ES{Sistema Operativo}}
	\acro{FAQ}{\ES{Preguntas Frecuentes}}
	\acro{OOP}[POO]{\ES{Programaci\'on Orientada a Objetos}}
	\acro{PL}[LP]{\ES{Lenguaje de Programaci\'on}}
	\acro{PRNG}[GPAN]{\ES{Generador Pseudo-Aleatorio de N\'umeros}}
	\acro{ROM}{\ES{Memoria de Solo Lectura}}
	\acro{ALU}{\ES{Unidad Aritm\'etica L\'ogica}}
}%
\PTBR{
	\acro{OS}{\PTBRph{}}
	\acro{FAQ}{\PTBRph{}}
	\acro{OOP}{\PTBRph{}}
	\acro{PL}{\PTBRph{}}
	\acro{PRNG}{\PTBRph{}}
	\acro{ROM}{\PTBRph{}}
	\acro{ALU}{\PTBRph{}}
}%
\PL{
	\acro{OS}{\PLph{}}
	\acro{FAQ}{\PLph{}}
	\acro{OOP}{\PLph{}}
	\acro{PL}{\PLph{}}
	\acro{PRNG}{\PLph{}}
	\acro{ROM}{\PLph{}}
	\acro{ALU}{\PLph{}}
}%
\DE{
	\acro{OS}[BS]{Betriebssystem}
	\acro{FAQ}{\DEph{}}
	\acro{OOP}{\DEph{}}
	\acro{PL}{\DEph{}}
	\acro{PRNG}{\DEph{}}
	\acro{ROM}{\DEph{}}
	\acro{ALU}{\DEph{}}
}%
\ITA{
	\acro{OS}{\ITAph{}}
	\acro{FAQ}{\ITAph{}}
	\acro{OOP}{\ITAph{}}
	\acro{PL}{\ITAph{}}
	\acro{PRNG}{\ITAph{}}
	\acro{ROM}{\ITAph{}}
	\acro{ALU}{\ITAph{}}
}%
\THA{
	\acro{OS}{\THAph{}}
	\acro{FAQ}{\THAph{}}
	\acro{OOP}{\THAph{}}
	\acro{PL}{\THAph{}}
	\acro{PRNG}{\THAph{}}
	\acro{ROM}{\THAph{}}
	\acro{ALU}{\THAph{}}
}%
\NL{
	\acro{OS}{\NL{Operating System}}
	\acro{FAQ}{\NL{Veelvoorkomende vragen}}
	\acro{OOP}{\NL{Object-Oriented Programmeren}}
	\acro{PL}[PT]{\NL{Programmeertaal}}
	\acro{PRNG}{\NL{Pseudorandom number generator}}
	\acro{ROM}{\NL{Read-only memory}}
	\acro{ALU}{\NL{Arithmetic logic unit}}
}%
\FR{
	\acro{OS}[SE]{Système d'exploitation}
	\acro{FAQ}{Foire Aux Questions}
	\acro{OOP}[POO]{Programmation orientée objet}
	\acro{PL}[LP]{Language de programmation}
	\acro{PRNG}{Nombre généré pseudo-aléatoirement}
	\acro{ROM}{Mémoire morte}
	\acro{ALU}[UAL]{Unité arithmétique et logique}
}%
\acro{RA}{\ReturnAddress}
\acro{PE}{Portable Executable}
\acro{SP}{\gls{stack pointer}. SP/ESP/RSP \InENRU x86/x64. SP \InENRU ARM.}
\acro{DLL}{Dynamic-link library}
\acro{PC}{Program Counter. IP/EIP/RIP \InENRU x86/64. PC \InENRU ARM.}
\acro{LR}{Link Register}
\acro{IDA}{
	\RU{Интерактивный дизассемблер и отладчик, разработан \href{https://hex-rays.com/}{Hex-Rays}}%
	\EN{Interactive Disassembler and debugger developed by \href{https://hex-rays.com/}{Hex-Rays}}%
	\ES{Desensamblador Interactivo y depurador desarrollado por \href{https://hex-rays.com/}{Hex-Rays}}%
	\NL{Interactive Disassembler en debugger ontwikkeld door \href{https://hex-rays.com}{Hex-Rays}}
	\PTBRph{}%
	\PLph{}%
	\DEph{}%
	\ITAph{}%
	\THAph{}%
	\FRph{Désassembleur interactif et débuggueur développé par \href{https://hex-rays.com/}{Hex-Rays}}%
}
\acro{IAT}{Import Address Table}
\acro{INT}{Import Name Table}
\acro{RVA}{Relative Virtual Address}
\acro{VA}{Virtual Address}
\acro{OEP}{Original Entry Point}
\acro{MSVC}{Microsoft Visual C++}
\acro{MSVS}{Microsoft Visual Studio}
\acro{ASLR}{Address Space Layout Randomization}
\acro{MFC}{Microsoft Foundation Classes}
\acro{TLS}{Thread Local Storage}
\acro{AKA}{Also Known As%
	\RU{ - (Также известный как)}%
	\ES{ - (Tambi\'en Conocido Como)}%
	\NL{ - (Ook gekend als)}%
	\PTBRph{}%
	\PLph{}%
	\DEph{}%
	\ITAph{}%
	\THAph{}%
	\FRph{Aussi connu sous le nom}
}
\acro{CRT}{C runtime library}
\acro{CPU}{Central processing unit}
\acro{FPU}{Floating-point unit}
\acro{CISC}{Complex instruction set computing}
\acro{RISC}{Reduced instruction set computing}
\acro{GUI}{Graphical user interface}
\acro{RTTI}{Run-time type information}
\acro{BSS}{Block Started by Symbol}
\acro{SIMD}{Single instruction, multiple data}
\acro{BSOD}{Blue Screen of Death}
\acro{DBMS}{Database management systems}
\acro{ISA}{Instruction Set Architecture\RU{ (Архитектура набора команд)}}
\acro{CGI}{Common Gateway Interface}
\acro{HPC}{High-Performance Computing}
\acro{SOC}{System on Chip}
\acro{SEH}{Structured Exception Handling}
\acro{ELF}{\RU{Формат исполняемых файлов, использующийся в Linux и некоторых других *NIX}
\EN{Executable file format widely used in *NIX systems including Linux}\ESph{}\PTBRph{}\PLph{}\ITAph{}\DEph{}\NLph{}}
\acro{TIB}{Thread Information Block}
\acro{TEA}{Tiny Encryption Algorithm}
\acro{PIC}{Position Independent Code: \myref{sec:PIC}}
\acro{NAN}{Not a Number}
\acro{NOP}{No OPeration}
\acro{BEQ}{(PowerPC, ARM) Branch if Equal}
\acro{BNE}{(PowerPC, ARM) Branch if Not Equal}
\acro{BLR}{(PowerPC) Branch to Link Register}
\acro{XOR}{eXclusive OR\RU{ (исключающее \q{ИЛИ})}}
\acro{MCU}{Microcontroller unit}
\acro{RAM}{Random-access memory}
\acro{GCC}{GNU Compiler Collection}
\acro{EGA}{Enhanced Graphics Adapter}
\acro{VGA}{Video Graphics Array}
\acro{API}{Application programming interface}
\acro{ASCII}{American Standard Code for Information Interchange}
\acro{ASCIIZ}{ASCII Zero (\RU{ASCII-строка заканчивающаяся нулем}\EN{null-terminated ASCII string}\PTBRph{})}
\acro{IA64}{Intel Architecture 64 (Itanium): \myref{itanium}}
\acro{EPIC}{Explicitly parallel instruction computing}
\acro{OOE}{Out-of-order execution}
\acro{MSDN}{Microsoft Developer Network}
\acro{MSB}{Most significant bit/byte\RU{ (самый старший бит/байт)}}
\acro{LSB}{Least significant bit/byte\RU{ (самый младший бит/байт)}}
\acro{STL}{(\Cpp) Standard Template Library: \myref{sec:STL}}
\acro{PODT}{(\Cpp) Plain Old Data Type}
\acro{HDD}{Hard disk drive}
\acro{VM}{Virtual Memory\RU{ (виртуальная память)}}
\acro{WRK}{Windows Research Kernel}
\acro{GPR}{General Purpose Registers\RU{ (регистры общего пользования)}}
\acro{SSDT}{System Service Dispatch Table}
\acro{RE}{Reverse Engineering}
\acro{RAID}{Redundant Array of Independent Disks}
\acro{SSE}{Streaming SIMD Extensions}
\acro{BCD}{Binary-coded decimal}
\acro{BOM}{Byte order mark}
\acro{GDB}{GNU debugger}
\acro{FP}{Frame Pointer}
\acro{MBR}{Master Boot Record}
\acro{JPE}{Jump Parity Even (\RU{инструкция x86}\EN{x86 instruction})}
\acro{CIDR}{Classless Inter-Domain Routing}
\acro{STMFD}{Store Multiple Full Descending (\RU{инструкция ARM}\EN{ARM instruction})}
\acro{LDMFD}{Load Multiple Full Descending (\RU{инструкция ARM}\EN{ARM instruction})}
\acro{STMED}{Store Multiple Empty Descending (\RU{инструкция ARM}\EN{ARM instruction})}
\acro{LDMED}{Load Multiple Empty Descending (\RU{инструкция ARM}\EN{ARM instruction})}
\acro{STMFA}{Store Multiple Full Ascending (\RU{инструкция ARM}\EN{ARM instruction})}
\acro{LDMFA}{Load Multiple Full Ascending (\RU{инструкция ARM}\EN{ARM instruction})}
\acro{STMEA}{Store Multiple Empty Ascending (\RU{инструкция ARM}\EN{ARM instruction})}
\acro{LDMEA}{Load Multiple Empty Ascending (\RU{инструкция ARM}\EN{ARM instruction})}
\acro{APSR}{(ARM) Application Program Status Register}
\acro{FPSCR}{(ARM) Floating-Point Status and Control Register}
\acro{RFC}{Request for Comments}
\acro{TOS}{Top Of Stack\RU{ (вершина стека)}}
\acro{LVA}{(Java) Local Variable Array\RU{ (массив локальных переменных)}}
\acro{JVM}{Java virtual machine}
\acro{JIT}{Just-in-time compilation}
\acro{CDFS}{Compact Disc File System}
\acro{CD}{Compact Disc}
\acro{ADC}{Analog-to-digital converter}
\acro{EOF}{End of file\RU{ (конец файла)}}
\acro{TBT}{To be translated} % temporary...
\acro{DIY}{Do It Yourself}
\acro{MMU}{Memory management unit}
\acro{CPRNG}{Cryptographically secure Pseudorandom Number Generator}
\acro{DES}{Data Encryption Standard}
\acro{MIME}{Multipurpose Internet Mail Extensions}
\acro{XML}{Extensible Markup Language}
\acro{JSON}{JavaScript Object Notation}
\end{acronym}


\bookmarksetup{startatroot}

\clearpage
\phantomsection
\addcontentsline{toc}{chapter}{\RU{Глоссарий}\EN{Glossary}}
\printglossaries

\clearpage
\phantomsection
\printindex

\clearpage
\phantomsection
\addcontentsline{toc}{chapter}{\RU{Библиография}\EN{Bibliography}}
\printbibliography

\end{document}
