\documentclass[11pt,a4paper,oneside]{book}

\usepackage{cmap}

\ifdefined\RUSSIAN
\usepackage[english,russian]{babel}
\usepackage[T2A]{fontenc}
\usepackage{paratype}
\renewcommand*\familydefault{\sfdefault}
% http://www.emerson.emory.edu/services/latex/latex_169.html
\newcommand{\lstlistingsize}{\scriptsize}
\else
\usepackage[russian,english]{babel}
\usepackage[T2A]{fontenc}
\usepackage[default]{sourcesanspro}
\newcommand{\lstlistingsize}{\footnotesize}
\fi

\usepackage[utf8]{inputenc}
\usepackage{listings}
\usepackage{ulem}
\usepackage{url}
\usepackage{graphicx}
\usepackage{listingsutf8}
\usepackage{makeidx}
\usepackage{cite}
\usepackage[cm]{fullpage}
\usepackage{color}
\usepackage{fancyvrb}
\usepackage{xspace}
\usepackage{framed}
\usepackage{ccicons}
\usepackage[nottoc]{tocbibind}
\usepackage{amsmath}
\usepackage[footnote,printonlyused,withpage]{acronym}
\usepackage[table]{xcolor}% http://ctan.org/pkg/xcolor
\usepackage[]{hyperref} % should be last

\definecolor{lstbgcolor}{rgb}{0.94,0.94,0.94}
\makeindex

\newcommand*{\TT}[1]{\texttt{#1}}
\newcommand*{\IT}[1]{\textit{#1}}
\newcommand*{\EN}[1]{\iflanguage{english}{#1}{}}
\newcommand*{\RU}[1]{\iflanguage{english}{}{#1}}
\newcommand{\LANG}{\RU{ru}\EN{en}}
\newcommand*{\dittoclosing}{---''---}
\newcommand*{\EMDASH}{\RU{ --- }\EN{---}}
\newcommand*{\AsteriskOne}{${}^{*}$}
\newcommand*{\AsteriskTwo}{${}^{**}$}
\newcommand*{\AsteriskThree}{${}^{***}$}

\newcommand{\ttf}{\TT{f()}\xspace}
\newcommand{\ttfone}{\TT{f1()}\xspace}

% http://tex.stackexchange.com/questions/32160/new-line-after-paragraph
\newcommand{\myparagraph}[1]{\paragraph{#1}\mbox{}\\} 

\newcommand{\figname}{\RU{илл}\EN{fig}.\xspace}
\newcommand{\figref}[1]{\figname{}\ref{#1}\xspace}
\newcommand{\listingname}{\RU{листинг}\EN{listing}.\xspace}
\newcommand{\lstref}[1]{\listingname{}\ref{#1}\xspace}
\newcommand{\bitENRU}{\RU{бит}\EN{bit}\xspace}
\newcommand{\bitsENRU}{\RU{бита}\EN{bits}\xspace}
\newcommand{\Sourcecode}{\RU{Исходный код}\EN{Source code}\xspace}
\newcommand{\Seealso}{\RU{См. также}\EN{See also}\xspace}
\newcommand{\MacOSX}{Mac OS X\xspace}

% FIXME TODO non-overlapping color!
% \newcommand{\headercolor}{\cellcolor{blue!25}}
\newcommand{\headercolor}{}

\newcommand{\tableheader}{\headercolor{} \RU{смещение}\EN{offset} & \headercolor{} \RU{описание}\EN{description}}

\newcommand{\IDA}{\ac{IDA}\xspace}

\newcommand{\tracer}{\protect\gls{tracer}\xspace}

\newcommand{\Tchar}{\IT{char}\xspace} 
\newcommand{\Tint}{\IT{int}\xspace}
\newcommand{\Tbool}{\IT{bool}\xspace}
\newcommand{\Tfloat}{\IT{float}\xspace}
\newcommand{\Tdouble}{\IT{double}\xspace}
\newcommand{\Tvoid}{\IT{void}\xspace}
\newcommand{\ITthis}{\IT{this}\xspace}

\newcommand{\Ox}{\TT{/Ox}\xspace}
\newcommand{\Obzero}{\TT{/Ob0}\xspace}
\newcommand{\Othree}{\TT{-O3}\xspace}

\newcommand{\oracle}{Oracle RDBMS\xspace}

\newcommand{\idevices}{iPod/iPhone/iPad\xspace}
\newcommand{\olly}{OllyDbg\xspace}

% common C functions
\newcommand{\printf}{\TT{printf()}\xspace} 
\newcommand{\puts}{\TT{puts()}\xspace} 
\newcommand{\main}{\TT{main()}\xspace} 
\newcommand{\qsort}{\TT{qsort()}\xspace} 
\newcommand{\strlen}{\TT{strlen()}\xspace} 
\newcommand{\scanf}{\TT{scanf()}\xspace} 
\newcommand{\rand}{\TT{rand()}\xspace} 

% x86 instructions
\newcommand{\ADD}{\TT{ADD}\xspace} 
\newcommand{\ANDIns}{\TT{AND}\xspace} 
\newcommand{\CALL}{\TT{CALL}\xspace} 
\newcommand{\CPUID}{\TT{CPUID}\xspace} 
\newcommand{\CMP}{\TT{CMP}\xspace} 
\newcommand{\DEC}{\TT{DEC}\xspace} 
\newcommand{\FADDP}{\TT{FADDP}\xspace} 
\newcommand{\FCOM}{\TT{FCOM}\xspace}
\newcommand{\FCOMP}{\TT{FCOMP}\xspace}
\newcommand{\FCOMI}{\TT{FCOMI}\xspace}
\newcommand{\FCOMIP}{\TT{FCOMIP}\xspace}
\newcommand{\FUCOM}{\TT{FUCOM}\xspace}
\newcommand{\FUCOMI}{\TT{FUCOMI}\xspace}
\newcommand{\FUCOMIP}{\TT{FUCOMIP}\xspace}
\newcommand{\FUCOMPP}{\TT{FUCOMPP}\xspace}
\newcommand{\FDIVR}{\TT{FDIVR}\xspace} 
\newcommand{\FDIV}{\TT{FDIV}\xspace} 
\newcommand{\FLD}{\TT{FLD}\xspace} 
\newcommand{\FMUL}{\TT{FMUL}\xspace} 
\newcommand{\FSTP}{\TT{FSTP}\xspace} 
\newcommand{\FDIVP}{\TT{FDIVP}\xspace}
\newcommand{\IDIV}{\TT{IDIV}\xspace} 
\newcommand{\IMUL}{\TT{IMUL}\xspace} 
\newcommand{\INC}{\TT{INC}\xspace} 
\newcommand{\JAE}{\TT{JAE}\xspace} 
\newcommand{\JA}{\TT{JA}\xspace} 
\newcommand{\JBE}{\TT{JBE}\xspace} 
\newcommand{\JB}{\TT{JBE}\xspace} 
\newcommand{\JE}{\TT{JE}\xspace} 
\newcommand{\JGE}{\TT{JGE}\xspace} 
\newcommand{\JG}{\TT{JG}\xspace} 
\newcommand{\JLE}{\TT{JLE}\xspace} 
\newcommand{\JL}{\TT{JL}\xspace} 
\newcommand{\JMP}{\TT{JMP}\xspace} 
\newcommand{\JNE}{\TT{JNE}\xspace} 
\newcommand{\JNZ}{\TT{JNZ}\xspace} 
\newcommand{\JNA}{\TT{JNA}\xspace} 
\newcommand{\JNAE}{\TT{JNAE}\xspace} 
\newcommand{\JNB}{\TT{JNB}\xspace} 
\newcommand{\JNBE}{\TT{JNBE}\xspace} 
\newcommand{\JZ}{\TT{JZ}\xspace} 
\newcommand{\JP}{\TT{JP}\xspace} 
\newcommand{\Jcc}{\TT{Jcc}\xspace} 
\newcommand{\SETcc}{\TT{SETcc}\xspace} 
\newcommand{\LEA}{\TT{LEA}\xspace} 
\newcommand{\LOOP}{\TT{LOOP}\xspace}
\newcommand{\MOVSX}{\TT{MOVSX}\xspace} 
\newcommand{\MOVZX}{\TT{MOVZX}\xspace} 
\newcommand{\MOV}{\TT{MOV}\xspace} 
\newcommand{\NOP}{\TT{NOP}\xspace} 
\newcommand{\POP}{\TT{POP}\xspace} 
\newcommand{\PUSH}{\TT{PUSH}\xspace} 
\newcommand{\NOT}{\TT{NOT}\xspace} 
\newcommand{\RET}{\TT{RET}\xspace} 
\newcommand{\RETN}{\TT{RETN}\xspace} 
\newcommand{\SETNZ}{\TT{SETNZ}\xspace} 
\newcommand{\SETBE}{\TT{SETBE}\xspace} 
\newcommand{\SETNBE}{\TT{SETNBE}\xspace} 
\newcommand{\SUB}{\TT{SUB}\xspace} 
\newcommand{\TEST}{\TT{TEST}\xspace} 
\newcommand{\FNSTSW}{\TT{FNSTSW}\xspace}
\newcommand{\SAHF}{\TT{SAHF}\xspace}
\newcommand{\XOR}{\TT{XOR}\xspace} 
\newcommand{\OR}{\TT{OR}\xspace} 
\newcommand{\SHL}{\TT{SHL}\xspace} 
\newcommand{\SHR}{\TT{SHR}\xspace} 
\newcommand{\LEAVE}{\TT{LEAVE}\xspace} 
\newcommand{\MOVDQA}{\TT{MOVDQA}\xspace} 
\newcommand{\MOVDQU}{\TT{MOVDQU}\xspace} 
\newcommand{\PADDD}{\TT{PADDD}\xspace} 
\newcommand{\PCMPEQB}{\TT{PCMPEQB}\xspace} 

% x86 flags

\newcommand{\ZF}{\TT{ZF}\xspace} 
\newcommand{\CF}{\TT{CF}\xspace} 
\newcommand{\PF}{\TT{PF}\xspace} 

% x86 registers

\newcommand{\AL}{\TT{AL}\xspace} 
\newcommand{\AH}{\TT{AH}\xspace} 
\newcommand{\AX}{\TT{AX}\xspace} 
\newcommand{\EAX}{\TT{EAX}\xspace} 
\newcommand{\EBX}{\TT{EBX}\xspace} 
\newcommand{\ECX}{\TT{ECX}\xspace} 
\newcommand{\EDX}{\TT{EDX}\xspace} 
\newcommand{\DL}{\TT{DL}\xspace} 
\newcommand{\ESI}{\TT{ESI}\xspace} 
\newcommand{\EDI}{\TT{EDI}\xspace} 
\newcommand{\EBP}{\TT{EBP}\xspace} 
\newcommand{\ESP}{\TT{ESP}\xspace} 
\newcommand{\RSP}{\TT{RSP}\xspace} 
\newcommand{\EIP}{\TT{EIP}\xspace} 
\newcommand{\RIP}{\TT{RIP}\xspace} 
\newcommand{\RAX}{\TT{RAX}\xspace} 
\newcommand{\RBX}{\TT{RBX}\xspace} 
\newcommand{\RCX}{\TT{RCX}\xspace} 
\newcommand{\RDX}{\TT{RDX}\xspace} 
\newcommand{\RBP}{\TT{RBP}\xspace} 
\newcommand{\RSI}{\TT{RSI}\xspace} 
\newcommand{\RDI}{\TT{RDI}\xspace} 
\newcommand*{\ST}[1]{\TT{ST(#1)}\xspace}
\newcommand*{\XMM}[1]{\TT{XMM#1}\xspace}

% ARM
\newcommand*{\Reg}[1]{\TT{R#1}\xspace}
\newcommand*{\RegX}[1]{\TT{X#1}\xspace}
\newcommand*{\RegW}[1]{\TT{W#1}\xspace}
\newcommand*{\RegD}[1]{\TT{D#1}\xspace}
\newcommand{\ADREQ}{\TT{ADREQ}\xspace}
\newcommand{\ADRNE}{\TT{ADRNE}\xspace}
\newcommand{\BEQ}{\TT{BEQ}\xspace}

% instructions descriptions
\newcommand{\ASRdesc}{\RU{арифметический сдвиг вправо}\EN{arithmetic shift right}}

% x86 registers tables
% TODO: non-overlapping color!
\newcommand{\RegHeader}{
\RU{
 7 \textsuperscript{(номер байта)} &
 6 &
 5 &
 4 &
 3 &
 2 &
 1 &
 0 }
\EN{
 7th \textsuperscript{(byte number)} &
 6th &
 5th &
 4th &
 3rd &
 2nd &
 1st &
 0th}
}

% FIXME навести порядок тут...
\newcommand{\RegTableThree}[5]{
\begin{center}
\begin{tabular}{ | l | l | l | l | l | l | l | l | l |}
\hline
\RegHeader \\
\hline
\multicolumn{8}{ | c | }{#1} \\
\hline
\multicolumn{4}{ | c | }{} & \multicolumn{4}{ c | }{#2} \\
\hline
\multicolumn{6}{ | c | }{} & \multicolumn{2}{ c | }{#3} \\
\hline
\multicolumn{6}{ | c | }{} & #4 & #5 \\
\hline
\end{tabular}
\end{center}
}

\newcommand{\RegTableOne}[5]{\RegTableThree{#1\textsuperscript{x64}}{#2}{#3}{#4}{#5}}

\newcommand{\RegTableTwo}[4]{
\begin{center}
\begin{tabular}{ | l | l | l | l | l | l | l | l | l |}
\hline
\RegHeader \\
\hline
\multicolumn{8}{ | c | }{#1\textsuperscript{x64}} \\
\hline
\multicolumn{4}{ | c | }{} & \multicolumn{4}{ c | }{#2} \\
\hline
\multicolumn{6}{ | c | }{} & \multicolumn{2}{ c | }{#3} \\
\hline
\multicolumn{7}{ | c | }{} & #4\textsuperscript{x64} \\
\hline
\end{tabular}
\end{center}
}

\newcommand{\RegTableFour}[4]{
\begin{center}
\begin{tabular}{ | l | l | l | l | l | l | l | l | l |}
\hline
\RegHeader \\
\hline
\multicolumn{8}{ | c | }{#1} \\
\hline
\multicolumn{4}{ | c | }{} & \multicolumn{4}{ c | }{#2} \\
\hline
\multicolumn{6}{ | c | }{} & \multicolumn{2}{ c | }{#3} \\
\hline
\multicolumn{7}{ | c | }{} & #4 \\
\hline
\end{tabular}
\end{center}
}


\newcommand{\URLWPDA}
{\RU
 {
  \href{http://go.yurichev.com/17012}{Wikipedia: Выравнивание данных}
 }
 \EN{
  \href{http://go.yurichev.com/17013}{Wikipedia: Data structure alignment}
 }
}

\newcommand{\OracleTablesName}{oracle tables\xspace}
\newcommand{\oracletables}{\OracleTablesName\footnote{\url{http://go.yurichev.com/17014}}\xspace}

\newcommand{\WPMAO}
{\RU
{
    \href{http://go.yurichev.com/17015}{wikipedia: Умножение-сложение}
}
\EN{
    \href{http://go.yurichev.com/17016}{wikipedia: Multiply–accumulate operation}
}
}

\newcommand{\BGREPURL}{\url{http://go.yurichev.com/17017}}
\newcommand{\FNMSDNROTxURL}{\footnote{\url{http://go.yurichev.com/17018}}}

\newcommand{\YurichevIDAIDCScripts}{http://go.yurichev.com/17019}

% for index
\newcommand{\GrepUsage}{\IFRU{Использование grep}{grep usage}}
\newcommand{\SyntacticSugar}{\IFRU{Синтаксический сахар}{Syntactic Sugar}}
\newcommand{\CompilerAnomaly}{\IFRU{Аномалии компиляторов}{Compiler's anomalies}}
\newcommand{\CLanguageElements}{\IFRU{Элементы языка Си}{C language elements}}
\newcommand{\CStandardLibrary}{\IFRU{Стандартная библиотека Си}{C standard library}}
\newcommand{\Flags}{\IFRU{Флаги}{Flags}}
\newcommand{\Registers}{\IFRU{Регистры}{Registers}}
\newcommand{\Stack}{\IFRU{Стек}{Stack}}
\newcommand{\Recursion}{\IFRU{Рекурсия}{Recursion}}
\newcommand{\RAM}{\IFRU{ОЗУ}{RAM}}
\newcommand{\ROM}{\IFRU{ПЗУ}{ROM}}
\newcommand{\Pointers}{\IFRU{Указатели}{Pointers}}
\newcommand{\BufferOverflow}{\IFRU{Переполнение буфера}{Buffer Overflow}}

\newcommand{\Exercise}{\IFRU{Задача}{Exercise}\xspace}
\newcommand{\Cpp}{\IFRU{Си++}{C++}\xspace}
\newcommand{\CCpp}{\IFRU{Си/Си++}{C/C++}\xspace}
\newcommand{\NonOptimizing}{\IFRU{Неоптимизирующий}{Non-optimizing}\xspace}
\newcommand{\Optimizing}{\IFRU{Оптимизирующий}{Optimizing}\xspace}
\newcommand{\NonOptimizingKeil}{\NonOptimizing Keil\xspace}
\newcommand{\OptimizingKeil}{\Optimizing Keil\xspace}
\newcommand{\NonOptimizingXcode}{\NonOptimizing Xcode (LLVM)\xspace}
\newcommand{\OptimizingXcode}{\Optimizing Xcode (LLVM)\xspace}
\newcommand{\ARMMode}{\IFRU{Режим ARM}{ARM mode}\xspace}
\newcommand{\ThumbMode}{\IFRU{Режим thumb}{thumb mode}\xspace}
\newcommand{\ThumbTwoMode}{\IFRU{Режим thumb-2}{thumb-2 mode}\xspace}
\newcommand{\AndENRU}{\IFRU{и}{and}\xspace}
\newcommand{\OrENRU}{\IFRU{или}{or}\xspace}
\newcommand{\InENRU}{\IFRU{в}{in}\xspace}
\newcommand{\ForENRU}{\IFRU{для}{for}\xspace}

\newcommand{\FNQUOTIENT}{\footnote{\IFRU{результат деления}{result of division}}}
\newcommand{\FNPRODUCT}{\footnote{\IFRU{результат умножения}{result of multiplication}}}
\newcommand{\FNSUM}{\footnote{\IFRU{результат сложения}{result of addition}}}

\newcommand{\DataProcessingInstructionsFootNote}{\IFRU{Эти инструкции также называются}
{These instructions are also called} ``data processing instructions''}

\newcommand{\Instructions}{\IFRU{Инструкции}{Instructions}}

% for .bib files
\newcommand{\AlsoAvailableAs}{\IFRU{Также доступно здесь:}{Also available as}\xspace}

% section names
\newcommand{\ShiftsSectionName}{\IFRU{Сдвиги}{Shifts}}
\newcommand{\SignedNumbersSectionName}{\IFRU{Представление знака в числах}{Signed number representations}}
\newcommand{\HelloWorldSectionName}{Hello, world!}
\newcommand{\SwitchCaseDefaultSectionName}{switch()/case/default}
\newcommand{\PrintfSeveralArgumentsSectionName}{\printf \IFRU{с несколькими аргументами}{with several arguments}}
\newcommand{\DivisionByNineSectionName}{\IFRU{Деление на 9}{Division by 9}}
\newcommand{\WorkingWithFloatAsWithStructSubSubSectionName}{\IFRU
{Работа с типом float как со структурой}{Working with the float type as with a structure}}

\newcommand{\StructurePackingSectionName}{\IFRU{Упаковка полей в структуре}{Fields packing in structure}}

\newcommand{\PICcode}{\IFRU{адресно-независимый код}{position-independent code}}
\newcommand{\CapitalPICcode}{\IFRU{Адресно-независимый код}{Position-independent code}}
\newcommand{\Loops}{\IFRU{Циклы}{Loops}}

% C
\newcommand{\PostIncrement}{\IFRU{Пост-инкремент}{Post-increment}}
\newcommand{\PostDecrement}{\IFRU{Пост-декремент}{Post-decrement}}
\newcommand{\PreIncrement}{\IFRU{Пре-инкремент}{Pre-increment}}
\newcommand{\PreDecrement}{\IFRU{Пре-декремент}{Pre-decrement}}

% other
\newcommand{\IntelSyntax}{\IFRU{Синтаксис Intel}{Intel syntax}}
\newcommand{\ATTSyntax}{\IFRU{Синтаксис AT\&T}{AT\&T syntax}}
\newcommand{\randomNoise}{\IFRU{случайный шум}{random noise}}
\newcommand{\Example}{\IFRU{Пример}{Example}}
\newcommand{\argument}{\IFRU{аргумент}{argument}}
\newcommand{\MarkedInIDAAs}{\IFRU{маркируется в \IDA как}{marked in \IDA as}}
\newcommand{\HERMIT}{\IFRU{Андрей}{Andrey} ``herm1t'' \IFRU{Баранович}{Baranovich}}
\newcommand{\stepover}{\IFRU{сделать шаг не входя в ф-цию}{step over}}
\newcommand{\ShortHotKeyCheatsheet}{\IFRU{Краткий справочник хот-кеев}{Short hot-keys cheatsheet}}



\newcommand{\TITLE}{\IFRU{Краткое введение в reverse engineering для начинающих}
{Quick introduction to reverse engineering for beginners}}
\newcommand{\AUTHOR}{\IFRU{Денис Юричев}{Dennis Yurichev}}
\newcommand{\EMAIL}{dennis@yurichev.com}

\hypersetup{
    pdftex,
    colorlinks=true,
    allcolors=blue,
    pdfauthor={\AUTHOR},
    pdftitle={\TITLE}
    }

\selectlanguage{english}

\lstset{
    backgroundcolor=\color{lstbgcolor},
    basicstyle=\ttfamily\lstlistingsize, 
    breaklines=true,
    frame=single,
    inputencoding=cp1251,
    columns=fullflexible,keepspaces,
}

\begin{document}

\VerbatimFootnotes

\frontmatter

\begin{titlepage}
\begin{center}
\vspace*{\fill}
\LARGE \TITLE

\vspace*{\fill}

\large \AUTHOR

\large \TT{<\EMAIL>}
\vspace*{\fill}
\vfill

\ccbyncnd

\textcopyright 2013, \AUTHOR. 

\IFRU{Это произведение доступно по лицензии Creative Commons «Attribution-NonCommercial-NoDerivs» 
(«Атрибуция — Некоммерческое использование — Без производных произведений») 3.0 Непортированная. 
Чтобы увидеть копию этой лицензии, посетите}
{This work is licensed under the Creative Commons Attribution-NonCommercial-NoDerivs 3.0 Unported License. 
To view a copy of this license, visit} \url{http://creativecommons.org/licenses/by-nc-nd/3.0/}.

\IFRU{Версия этого текста}{Text version} ({\large \today}).

\IFRU{Возможно, более новая версии текста, а так же англоязычная версия, также доступна по ссылке}
{Probably, newer version of this text, and also Russian language version is also accessible at} \url{http://yurichev.com/RE-book.html}

\IFRU{Вы также можете подписаться на мой twitter для получения информации о новых версиях этого текста, итд:
\href{https://twitter.com/yurichev_ru}{@yurichev\_ru}}
{You may also subscribe to my twitter, to get information about updates of this text, etc: 
\href{https://twitter.com/yurichev}{@yurichev}}
\end{center}
\end{titlepage}

\tableofcontents
\cleardoublepage

\begin{center}
\vspace*{\fill}

\Huge\IFRU{Пожалуйста жертвуйте}
{Please donate}!
\normalsize

\bigskip
\bigskip
\bigskip

\Large\IFRU{Я писал эту книгу не менее года, здесь более 600 страниц, и она бесплатная.
Книги такого же уровня стоят от \$20 до \$50.}
{I worked more than year on this book, here are more than 600 pages, and it's free.
Same level books has price tag from \$20 to \$50.}
\normalsize

\bigskip
\bigskip
\bigskip

\IFRU{Больше об этом}{More about it}: \ref{sec:donate}.

\vspace*{\fill}
\vfill
\end{center}


\cleardoublepage
\EN{\section*{Preface}

There are several popular meanings of the term \q{\gls{reverse engineering}}:
1) The reverse engineering of software: researching compiled programs;
2) The scanning of 3D structures and the subsequent digital manipulation required in order to duplicate them;
3) Recreating \ac{DBMS} structure.
This book is about the first meaning.

\subsection*{Topics discussed in-depth}

x86/x64, ARM/ARM64, MIPS, Java/JVM.

\subsection*{Topics touched upon}

\oracle (\myref{oracle}),
Itanium (\myref{itanium}),
copy-protection dongles (\myref{dongles}), 
LD\_PRELOAD (\myref{ld_preload}),
stack overflow,
\ac{ELF},
win32 PE file format (\myref{win32_pe}),
x86-64 (\myref{x86-64}),
critical sections (\myref{critical_sections}),
syscalls (\myref{syscalls}), 
\ac{TLS},
position-independent code (\ac{PIC}) (\myref{sec:PIC}), 
profile-guided optimization (\myref{PGO}),
C++ STL (\myref{cpp_STL}),
OpenMP (\myref{openmp}),
SEH (\myref{sec:SEH}).

\subsection*{Prerequisites}

Basic C \ac{PL} knowledge.
Recommended reading: \myref{CCppBooks}.

\subsection*{Exercises and tasks}

\dots 
are all moved to the separate website: \url{http://challenges.re}.

\subsection*{About the author}
\begin{tabularx}{\textwidth}{ l X }

\raisebox{-\totalheight}{
\includegraphics[scale=0.60]{Dennis_Yurichev.jpg}
}

&
Dennis Yurichev is an experienced reverse engineer and programmer.
He can be contacted by email: \textbf{\EMAIL{}}.

% FIXME: no link. \tablefootnote doesn't work
\end{tabularx}

% subsections:
\subsection*{%
	\RU{Отзывы о книге}%
	\EN{Praise for}%
	\ES{Elogios para}%
	\PTBRph{}%
	\DEph{}\PLph{}%
	\ITAph{}
	\IT{\TITLE}%
}

\begin{itemize}
% expanded URLs to make it more robust for printouts. In electronic editions people will click anyway, so tracking will keep working
\item \q{It's very well done .. and for free .. amazing.}\footnote{\href{http://go.yurichev.com/17095}{twitter.com/daniel\_bilar/status/436578617221742593}} Daniel Bilar, Siege Technologies, LLC.

\item \q{... excellent and free}\footnote{\href{http://go.yurichev.com/17096}{twitter.com/petefinnigan/status/400551705797869568}} Pete Finnigan,%
	\RU{гуру по безопасности}%
	\ES{gur\'u de seguridad en}%
	\PTBRph{}%
	\DEph{}\PLph{}%
	\ITAph{}
\oracle
	\EN{security guru}.

\item \q{... book is interesting, great job!} Michael Sikorski,
	\RU{автор книги}%
	\EN{author of}%
	\ES{autor de}%
	\PTBRph{}%
	\DEph{}\PLph{}%
	\ITAph{}
\IT{Practical Malware Analysis: The Hands-On Guide to Dissecting Malicious Software}.

\item \q{... my compliments for the very nice tutorial!} Herbert Bos,
	\RU{профессор университета}%
	\EN{full professor at the}%
	\ES{catedr\'atico de tiempo completo en la}%
	\PTBRph{}%
	\DEph{}\PLph{}%
	\ITAph{}
Vrije Universiteit Amsterdam,
	\RU{соавтор}%
	\EN{co-author of}%
	\ES{coautor de}%
	\PTBRph{}%
	\DEph{}\PLph{}%
	\ITAph{}
\IT{Modern Operating Systems (4th Edition)}.

\item \q{... It is amazing and unbelievable.} Luis Rocha, CISSP / ISSAP, Technical Manager, Network \& Information Security at Verizon Business.

\item \q{Thanks for the great work and your book.} Joris van de Vis,
	\RU{специалист по}%
	\ES{especialista en}%
	\PTBRph{}%
	\DEph{}\PLph{}%
	\ITAph{}
SAP Netweaver \& Security
	\EN{specialist}.

\item \q{... reasonable intro to some of the techniques.}\footnote{\href{http://go.yurichev.com/17099}{reddit}} Mike Stay,
	\RU{преподаватель в}%
	\EN{teacher at the}%
	\ES{profesor en el}%
	\PTBRph{}%
	\DEph{}\PLph{}%
	\ITAph{}
Federal Law Enforcement Training Center, Georgia, US.

\item \q{I love this book! I have several students reading it at the moment, plan to use it in graduate course.}\footnote{\href{http://go.yurichev.com/17097}{twitter.com/sergeybratus/status/505590326560833536}}
	\RU{Сергей Братусь}%
	\EN{Sergey Bratus}%
	\ES{Sergey Bratus}%
	\PTBRph{}%
	\DEph{}\PLph{}%
	\ITAph{},
Research Assistant Professor
	\RU{в отделе Computer Science в}%
	\EN{at the Computer Science Department at}%
	\ES{en el Departamento de Ciencias de la Computaci\'on en}%
	\PTBRph{}%
	\DEph{}\PLph{}%
	\ITAph{}
Dartmouth College

\item \q{Dennis @Yurichev has published an impressive (and free!) book on reverse engineering}\footnote{\href{http://go.yurichev.com/17098}{twitter.com/TanelPoder/status/524668104065159169}} Tanel Poder,
	\RU{эксперт по настройке производительности Oracle RDBMS}%
	\EN{Oracle RDBMS performance tuning expert}%
	\ES{experto en afinaci\'on de rendimiento de Oracle RDBMS}%
	\PTBRph{}%
	\DEph{}\PLph{}
	\ITAph{}.

\item \q{This book is some kind of Wikipedia to beginners...} Archer, Chinese Translator, IT Security Researcher.

\RU{\item \q{Прочел Вашу книгу~--- отличная работа, рекомендую на своих курсах студентам
в качестве учебного пособия}. Николай Ильин, преподаватель в ФТИ НТУУ \q{КПИ} и DefCon-UA}
\end{itemize}

\ifdefined\RUSSIAN
\newcommand{\PeopleMistakesInaccuracies}{Станислав \q{Beaver} Бобрицкий, Александр Лысенко, Shell Rocket, Zhu Ruijin, Changmin Heo, Александр \q{Solar Designer} Песляк, Vitor Vidal, Марк Уилсон.}
\else
\newcommand{\PeopleMistakesInaccuracies}{Stanislav \q{Beaver} Bobrytskyy, Alexander Lysenko, Shell Rocket, Zhu Ruijin, Changmin Heo, Alexander \q{Solar Designer} Peslyak, Vitor Vidal, Mark Wilson.}
\fi

\EN{\input{thanks_EN}}
\ES{\input{thanks_ES}}
\NL{\input{thanks_NL}}
\RU{\input{thanks_RU}}


\subsection*{mini-FAQ}

\par Q: What are prerequisites for reading this book?
\par A: Basic understanding of C/C++ is desirable.

\par Q: Can I buy Russian/English hardcopy/paper book?
\par A: Unfortunately no, no publisher got interested in publishing Russian or English version so far.
Meanwhile, you can ask your favorite copy shop to print/bind it.

\par Q: Is there epub/mobi version?
\par A: The book is highly dependent on TeX/LaTeX-specific hacks, so converting to HTML (epub/mobi is a set of HTMLs)
will not be easy.

\par Q: Why should one learn assembly language these days?
\par A: Unless you are an \ac{OS} developer, you probably don't need to code in assembly\textemdash{}latest compilers (2010s) are much better at performing optimizations than humans \footnote{A very good text about this topic: \InSqBrackets{\AgnerFog}}.

Also, latest \ac{CPU}s are very complex devices and assembly knowledge doesn't really help one to understand their internals.

That being said, there are at least two areas where a good understanding of assembly can be helpful: 
First and foremost, security/malware research. It is also a good way to gain a better understanding of your compiled code whilst debugging.
This book is therefore intended for those who want to understand assembly language rather 
than to code in it, which is why there are many examples of compiler output contained within.

\par Q: I clicked on a hyperlink inside a PDF-document, how do I go back?
\par A: In Adobe Acrobat Reader click Alt+LeftArrow. In Evince click ``<'' button.

\par Q: May I print this book / use it for teaching?
\par A: Of course! That's why the book is licensed under the Creative Commons license (CC BY-SA 4.0).

\par Q: Why is this book free? You've done great job. This is suspicious, as many other free things.
\par A: In my own experience, authors of technical literature do this mostly for self-advertisement purposes. It's not possible to get any decent money from such work.

\par Q: How does one get a job in reverse engineering?
\par A: There are hiring threads that appear from time to time on reddit, devoted to RE\FNURLREDDIT{}
(\RedditHiringThread{}).
Try looking there.

A somewhat related hiring thread can be found in the \q{netsec} subreddit: \NetsecHiringThread{}.

\par Q: I have a question...
\par A: Send it to me by email (\EMAIL).



\subsection*{About the Korean translation}

In January 2015, the Acorn publishing company (\href{http://www.acornpub.co.kr}{www.acornpub.co.kr}) in South Korea did a huge amount of work in translating and publishing 
my book (as it was in August 2014) into Korean.

It's now available at \href{http://go.yurichev.com/17343}{their website}.

\iffalse
\begin{figure}[H]
\centering
\includegraphics[scale=0.3]{acorn_cover.jpg}
\end{figure}
\fi

The translator is Byungho Min (\href{http://go.yurichev.com/17344}{twitter/tais9}).
The cover art was done by my artistic friend, Andy Nechaevsky:
\href{http://go.yurichev.com/17023}{facebook/andydinka}.
They also hold the copyright to the Korean translation.

So, if you want to have a \IT{real} book on your shelf in Korean and 
want to support my work, it is now available for purchase.

\subsection*{About the Persian/Farsi translation}

In 2016 the book has been translated by Mohsen Mostafa Jokar (who is also known to Iranian community by his translation of Radare manual\footnote{\url{http://rada.re/get/radare2book-persian.pdf}}).
It is available on the publisher’s website\footnote{\url{http://goo.gl/2Tzx0H}} (Pendare Pars).

40 page excerpt: \url{https://beginners.re/farsi.pdf}.

Registration of the book in National Library of Iran: \url{http://opac.nlai.ir/opac-prod/bibliographic/4473995}.

\subsection*{About the Chinese translation}

In April 2017, translation to Chinese has been finished by Chinese PTPress publisher. They are also the Chinese translation copyright holder. 

 It's available for order here: \url{http://www.epubit.com.cn/book/details/4174}. Some kind of review and history behind the translation: \url{http://www.cptoday.cn/news/detail/3155}.

Principal translator is Archer, to whom I owe so much. He was extremely meticulous (in good sense) and reported most of known mistakes and bugs, which is very important to literature like this book.
I'll recommend his services to any other author!

Guys from \href{http://www.antiy.net/}{Antiy Labs} has also helped with translation. \href{http://www.epubit.com.cn/book/onlinechapter/51413}{Here is preface} written by them.


}
\RU{\section*{Предисловие}

У термина \q{\gls{reverse engineering}} несколько популярных значений:
1) исследование скомпилированных
программ; 2) сканирование трехмерной модели для последующего копирования;
3) восстановление структуры СУБД. Настоящий сборник заметок
связан с первым значением.

\subsection*{Желательные знания перед началом чтения}

Очень желательно базовое знание \ac{PL} Си.
Рекомендуемые материалы: \myref{CCppBooks}.

\subsection*{Упражнения и задачи}

\dots 
все перемещены на отдельный сайт: \url{http://challenges.re}.

\subsection*{Об авторе}
\begin{tabularx}{\textwidth}{ l X }

\raisebox{-\totalheight}{
\includegraphics[scale=0.60]{Dennis_Yurichev.jpg}
}

&
Денис Юричев~--- опытный reverse engineer и программист.
С ним можно контактировать по емейлу: \textbf{\EMAIL{}} или по Skype: \textbf{dennis.yurichev}.

% FIXME: no link. \tablefootnote doesn't work
\end{tabularx}

% subsections:
\subsection*{%
	\RU{Отзывы о книге}%
	\EN{Praise for}%
	\ES{Elogios para}%
	\PTBRph{}%
	\DEph{}\PLph{}%
	\ITAph{}
	\IT{\TITLE}%
}

\begin{itemize}
% expanded URLs to make it more robust for printouts. In electronic editions people will click anyway, so tracking will keep working
\item \q{It's very well done .. and for free .. amazing.}\footnote{\href{http://go.yurichev.com/17095}{twitter.com/daniel\_bilar/status/436578617221742593}} Daniel Bilar, Siege Technologies, LLC.

\item \q{... excellent and free}\footnote{\href{http://go.yurichev.com/17096}{twitter.com/petefinnigan/status/400551705797869568}} Pete Finnigan,%
	\RU{гуру по безопасности}%
	\ES{gur\'u de seguridad en}%
	\PTBRph{}%
	\DEph{}\PLph{}%
	\ITAph{}
\oracle
	\EN{security guru}.

\item \q{... book is interesting, great job!} Michael Sikorski,
	\RU{автор книги}%
	\EN{author of}%
	\ES{autor de}%
	\PTBRph{}%
	\DEph{}\PLph{}%
	\ITAph{}
\IT{Practical Malware Analysis: The Hands-On Guide to Dissecting Malicious Software}.

\item \q{... my compliments for the very nice tutorial!} Herbert Bos,
	\RU{профессор университета}%
	\EN{full professor at the}%
	\ES{catedr\'atico de tiempo completo en la}%
	\PTBRph{}%
	\DEph{}\PLph{}%
	\ITAph{}
Vrije Universiteit Amsterdam,
	\RU{соавтор}%
	\EN{co-author of}%
	\ES{coautor de}%
	\PTBRph{}%
	\DEph{}\PLph{}%
	\ITAph{}
\IT{Modern Operating Systems (4th Edition)}.

\item \q{... It is amazing and unbelievable.} Luis Rocha, CISSP / ISSAP, Technical Manager, Network \& Information Security at Verizon Business.

\item \q{Thanks for the great work and your book.} Joris van de Vis,
	\RU{специалист по}%
	\ES{especialista en}%
	\PTBRph{}%
	\DEph{}\PLph{}%
	\ITAph{}
SAP Netweaver \& Security
	\EN{specialist}.

\item \q{... reasonable intro to some of the techniques.}\footnote{\href{http://go.yurichev.com/17099}{reddit}} Mike Stay,
	\RU{преподаватель в}%
	\EN{teacher at the}%
	\ES{profesor en el}%
	\PTBRph{}%
	\DEph{}\PLph{}%
	\ITAph{}
Federal Law Enforcement Training Center, Georgia, US.

\item \q{I love this book! I have several students reading it at the moment, plan to use it in graduate course.}\footnote{\href{http://go.yurichev.com/17097}{twitter.com/sergeybratus/status/505590326560833536}}
	\RU{Сергей Братусь}%
	\EN{Sergey Bratus}%
	\ES{Sergey Bratus}%
	\PTBRph{}%
	\DEph{}\PLph{}%
	\ITAph{},
Research Assistant Professor
	\RU{в отделе Computer Science в}%
	\EN{at the Computer Science Department at}%
	\ES{en el Departamento de Ciencias de la Computaci\'on en}%
	\PTBRph{}%
	\DEph{}\PLph{}%
	\ITAph{}
Dartmouth College

\item \q{Dennis @Yurichev has published an impressive (and free!) book on reverse engineering}\footnote{\href{http://go.yurichev.com/17098}{twitter.com/TanelPoder/status/524668104065159169}} Tanel Poder,
	\RU{эксперт по настройке производительности Oracle RDBMS}%
	\EN{Oracle RDBMS performance tuning expert}%
	\ES{experto en afinaci\'on de rendimiento de Oracle RDBMS}%
	\PTBRph{}%
	\DEph{}\PLph{}
	\ITAph{}.

\item \q{This book is some kind of Wikipedia to beginners...} Archer, Chinese Translator, IT Security Researcher.

\RU{\item \q{Прочел Вашу книгу~--- отличная работа, рекомендую на своих курсах студентам
в качестве учебного пособия}. Николай Ильин, преподаватель в ФТИ НТУУ \q{КПИ} и DefCon-UA}
\end{itemize}

\ifdefined\RUSSIAN
\newcommand{\PeopleMistakesInaccuracies}{Станислав \q{Beaver} Бобрицкий, Александр Лысенко, Shell Rocket, Zhu Ruijin, Changmin Heo, Александр \q{Solar Designer} Песляк, Vitor Vidal, Марк Уилсон.}
\else
\newcommand{\PeopleMistakesInaccuracies}{Stanislav \q{Beaver} Bobrytskyy, Alexander Lysenko, Shell Rocket, Zhu Ruijin, Changmin Heo, Alexander \q{Solar Designer} Peslyak, Vitor Vidal, Mark Wilson.}
\fi

\EN{\input{thanks_EN}}
\ES{\input{thanks_ES}}
\NL{\input{thanks_NL}}
\RU{\input{thanks_RU}}


\subsection*{mini-ЧаВО}

\par Q: Что необходимо знать перед чтением книги?
\par A: Желательно иметь базовое понимание Си/Си++.

\par Q: Возможно ли купить русскую/английскую бумажную книгу?
\par A: К сожалению нет, пока ни один издатель не заинтересовался в издании русской или английской версии.
А пока вы можете распечатать/переплести её в вашем любимом копи-шопе или копи-центре.

\par Q: Существует ли версия epub/mobi?
\par A: Книга очень сильно завязана на специфические для TeX/LaTeX хаки, поэтому преобразование в HTML (epub/mobi это набор HTML)
легким не будет.

\par Q: Зачем в наше время нужно изучать язык ассемблера?
\par A: Если вы не разработчик \ac{OS}, вам наверное не нужно писать на ассемблере: современные компиляторы (2010-ые) оптимизируют код намного лучше человека
\footnote{Очень хороший текст на эту тему: \InSqBrackets{\AgnerFog}}.

К тому же, современные \ac{CPU} это крайне сложные устройства и знание ассемблера вряд ли
поможет узнать их внутренности.

Но все-таки остается по крайней мере две области, где знание ассемблера может хорошо помочь:
1) исследование malware (\IT{зловредов}) с целью анализа; 2) лучшее понимание
вашего скомпилированного кода в процессе отладки.
Таким образом, эта книга предназначена для тех, кто хочет скорее понимать ассемблер,
нежели писать на нем, и вот почему здесь масса примеров, связанных с результатами
работы компиляторов.

\par Q: Я кликнул на ссылку внутри PDF-документа, как теперь вернуться назад?
\par A: В Adobe Acrobat Reader нажмите сочетание Alt+LeftArrow. В Evince кликните на ``<''.

\par Q: Могу ли я распечатать эту книгу? Использовать её для обучения?
\par A: Конечно, поэтому книга и лицензирована под лицензией Creative Commons (CC BY-SA 4.0).

\par Q: Почему эта книга бесплатная? Вы проделали большую работу. Это подозрительно, как и многие другие бесплатные вещи.
\par A: По моему опыту, авторы технической литературы делают это, в основном ради само-рекламы. Такой работой заработать приличные деньги невозможно.

\par Q: Как можно найти работу reverse engineer-а?
\par A: На reddit, посвященному RE\FNURLREDDIT, время от времени бывают hiring thread (\RedditHiringThread{}).
Посмотрите там.

В смежном субреддите \q{netsec} имеется похожий тред: \NetsecHiringThread{}.

\par Q: Куда пойти учиться в Украине?
\par A: \href{http://go.yurichev.com/17336}{НТУУ \q{КПИ}: \q{Аналіз програмного коду та бінарних вразливостей}};
\href{http://go.yurichev.com/17337}{факультативы}.

\par Q: У меня есть вопрос...
\par A: Напишите мне его емейлом (\EMAIL).


\subsection*{О переводе на корейский язык}

В январе 2015, издательство Acorn в Южной Корее сделало много работы в переводе 
и издании моей книги (по состоянию на август 2014) на корейский язык.
Она теперь доступна на \href{http://go.yurichev.com/17343}{их сайте}.

\iffalse
\begin{figure}[H]
\centering
\includegraphics[scale=0.3]{acorn_cover.jpg}
\end{figure}
\fi

Переводил Byungho Min (\href{http://go.yurichev.com/17344}{twitter/tais9}).
Обложку нарисовал мой хороший знакомый художник Андрей Нечаевский
\href{http://go.yurichev.com/17023}{facebook/andydinka}.
Они также имеют права на издание книги на корейском языке.
Так что если вы хотите иметь \IT{настоящую} книгу на полке на корейском языке и
хотите поддержать мою работу, вы можете купить её.

\subsection*{О переводе на персидский язык (фарси)}

В 2016 году книга была переведена Mohsen Mostafa Jokar (который также известен иранскому сообществу по переводу руководства Radare\footnote{\url{http://rada.re/get/radare2book-persian.pdf}}).
Книга доступна на сайте издательства\footnote{\url{http://goo.gl/2Tzx0H}} (Pendare Pars).

Первые 40 страниц: \url{https://beginners.re/farsi.pdf}.

Регистрация книги в Национальной Библиотеке Ирана: \url{http://opac.nlai.ir/opac-prod/bibliographic/4473995}.

\subsection*{О переводе на китайский язык}

В апреле 2017, перевод на китайский был закончен китайским издательством PTPress. Они также имеют права на издание книги на китайском языке.

Она доступна для заказа здесь: \url{http://www.epubit.com.cn/book/details/4174}. Что-то вроде рецензии и история о переводе: \url{http://www.cptoday.cn/news/detail/3155}.

Основным переводчиком был Archer, перед которым я теперь в долгу.
Он был крайне дотошным (в хорошем смысле) и сообщил о большинстве известных ошибок и баг, что крайне важно для литературы вроде этой книги.
Я буду рекомендовать его услуги всем остальным авторам!

Ребята из \href{http://www.antiy.net/}{Antiy Labs} также помогли с переводом. \href{http://www.epubit.com.cn/book/onlinechapter/51413}{Здесь предисловие} написанное ими.

}
\ES{% TODO to be synced with EN version
\section*{Pr\'ologo}

Existen muchos significados populares para el t\'ermino \q{\gls{reverse engineering}}:
1) La ingenier\'ia inversa de software: la investigaci\'on de programas compilados;
2) El escaneo de estructuras 3D y la manipulaci\'on digital subsecuente requerida para duplicarlas;
3) La recreaci\'on de la estructura de un \ac{DBMS}.
Este libro es acerca del primer significado.

\subsection*{T\'opicos discutidos a profundidad}

x86/x64, ARM/ARM64, MIPS, Java/JVM.

\subsection*{T\'opicos tocados}

\oracle (\myref{oracle}),
Itanium (\myref{itanium}),
dongles para protecci\'on de copias (\myref{dongles}), 
LD\_PRELOAD (\myref{ld_preload}),
desbordamiento de pila,
\ac{ELF},
formato de archivo win32 PE
(\myref{win32_pe}),
x86-64 (\myref{x86-64}),
secciones cr\'iticas
(\myref{critical_sections}),
llamadas al sistema
(\myref{syscalls}), 
\ac{TLS},
c\'odigo de posici\'on independiente
(\ac{PIC}) (\myref{sec:PIC}), 
profile-guided optimization (\myref{PGO}),
C++ STL (\myref{cpp_STL}),
OpenMP (\myref{openmp}),
SEH (\myref{sec:SEH}).

\subsection*{Ejercicios y tareas}

\dots 
fueron movidos al sitio web: \url{http://challenges.re}.

\subsection*{Sobre el autor}
\begin{tabularx}{\textwidth}{ l X }

\raisebox{-\totalheight}{
\includegraphics[scale=0.60]{Dennis_Yurichev.jpg}
}

&
Dennis Yurichev es un reverser y programador experimentado.
Puede ser contactado por email: \textbf{\EMAIL{}}.

% FIXME: no link. \tablefootnote doesn't work
\end{tabularx}

% subsections:
\subsection*{%
	\RU{Отзывы о книге}%
	\EN{Praise for}%
	\ES{Elogios para}%
	\PTBRph{}%
	\DEph{}\PLph{}%
	\ITAph{}
	\IT{\TITLE}%
}

\begin{itemize}
% expanded URLs to make it more robust for printouts. In electronic editions people will click anyway, so tracking will keep working
\item \q{It's very well done .. and for free .. amazing.}\footnote{\href{http://go.yurichev.com/17095}{twitter.com/daniel\_bilar/status/436578617221742593}} Daniel Bilar, Siege Technologies, LLC.

\item \q{... excellent and free}\footnote{\href{http://go.yurichev.com/17096}{twitter.com/petefinnigan/status/400551705797869568}} Pete Finnigan,%
	\RU{гуру по безопасности}%
	\ES{gur\'u de seguridad en}%
	\PTBRph{}%
	\DEph{}\PLph{}%
	\ITAph{}
\oracle
	\EN{security guru}.

\item \q{... book is interesting, great job!} Michael Sikorski,
	\RU{автор книги}%
	\EN{author of}%
	\ES{autor de}%
	\PTBRph{}%
	\DEph{}\PLph{}%
	\ITAph{}
\IT{Practical Malware Analysis: The Hands-On Guide to Dissecting Malicious Software}.

\item \q{... my compliments for the very nice tutorial!} Herbert Bos,
	\RU{профессор университета}%
	\EN{full professor at the}%
	\ES{catedr\'atico de tiempo completo en la}%
	\PTBRph{}%
	\DEph{}\PLph{}%
	\ITAph{}
Vrije Universiteit Amsterdam,
	\RU{соавтор}%
	\EN{co-author of}%
	\ES{coautor de}%
	\PTBRph{}%
	\DEph{}\PLph{}%
	\ITAph{}
\IT{Modern Operating Systems (4th Edition)}.

\item \q{... It is amazing and unbelievable.} Luis Rocha, CISSP / ISSAP, Technical Manager, Network \& Information Security at Verizon Business.

\item \q{Thanks for the great work and your book.} Joris van de Vis,
	\RU{специалист по}%
	\ES{especialista en}%
	\PTBRph{}%
	\DEph{}\PLph{}%
	\ITAph{}
SAP Netweaver \& Security
	\EN{specialist}.

\item \q{... reasonable intro to some of the techniques.}\footnote{\href{http://go.yurichev.com/17099}{reddit}} Mike Stay,
	\RU{преподаватель в}%
	\EN{teacher at the}%
	\ES{profesor en el}%
	\PTBRph{}%
	\DEph{}\PLph{}%
	\ITAph{}
Federal Law Enforcement Training Center, Georgia, US.

\item \q{I love this book! I have several students reading it at the moment, plan to use it in graduate course.}\footnote{\href{http://go.yurichev.com/17097}{twitter.com/sergeybratus/status/505590326560833536}}
	\RU{Сергей Братусь}%
	\EN{Sergey Bratus}%
	\ES{Sergey Bratus}%
	\PTBRph{}%
	\DEph{}\PLph{}%
	\ITAph{},
Research Assistant Professor
	\RU{в отделе Computer Science в}%
	\EN{at the Computer Science Department at}%
	\ES{en el Departamento de Ciencias de la Computaci\'on en}%
	\PTBRph{}%
	\DEph{}\PLph{}%
	\ITAph{}
Dartmouth College

\item \q{Dennis @Yurichev has published an impressive (and free!) book on reverse engineering}\footnote{\href{http://go.yurichev.com/17098}{twitter.com/TanelPoder/status/524668104065159169}} Tanel Poder,
	\RU{эксперт по настройке производительности Oracle RDBMS}%
	\EN{Oracle RDBMS performance tuning expert}%
	\ES{experto en afinaci\'on de rendimiento de Oracle RDBMS}%
	\PTBRph{}%
	\DEph{}\PLph{}
	\ITAph{}.

\item \q{This book is some kind of Wikipedia to beginners...} Archer, Chinese Translator, IT Security Researcher.

\RU{\item \q{Прочел Вашу книгу~--- отличная работа, рекомендую на своих курсах студентам
в качестве учебного пособия}. Николай Ильин, преподаватель в ФТИ НТУУ \q{КПИ} и DefCon-UA}
\end{itemize}

\ifdefined\RUSSIAN
\newcommand{\PeopleMistakesInaccuracies}{Станислав \q{Beaver} Бобрицкий, Александр Лысенко, Shell Rocket, Zhu Ruijin, Changmin Heo, Александр \q{Solar Designer} Песляк, Vitor Vidal, Марк Уилсон.}
\else
\newcommand{\PeopleMistakesInaccuracies}{Stanislav \q{Beaver} Bobrytskyy, Alexander Lysenko, Shell Rocket, Zhu Ruijin, Changmin Heo, Alexander \q{Solar Designer} Peslyak, Vitor Vidal, Mark Wilson.}
\fi

\EN{\input{thanks_EN}}
\ES{\input{thanks_ES}}
\NL{\input{thanks_NL}}
\RU{\input{thanks_RU}}


\input{FAQ_ES}

\subsection*{Acerca de la traducci\'on al Coreano}

En enero del 2015, la editorial Acorn (\href{http://www.acornpub.co.kr}{www.acornpub.co.kr}) en Corea del Sur realiz\'o una enorme cantidad de trabajo
traduciendo y publicando mi libro (como era en agosto del 2014) en Coreano.
Ahora se encuentra disponible en
\href{http://go.yurichev.com/17343}{su sitio web}.

\iffalse
\begin{figure}[H]
\centering
\includegraphics[scale=0.3]{acorn_cover.jpg}
\end{figure}
\fi

El traductor es Byungho Min (\href{http://go.yurichev.com/17344}{twitter/tais9}).
El arte de la portada fue hecho por mi art\'istico amigo, Andy Nechaevsky
\href{http://go.yurichev.com/17023}{facebook/andydinka}.
Ellos tambi\'en poseen los derechos de autor de la traducci\'on al coreano.
As\'i que, si quieren tener un libro \IT{real} en coreano en su estante
y quieren apoyar mi trabajo, ya se encuentra disponible a la venta.

%\subsection*{About the Persian/Farsi translation}
%TBT

}
\NL{% TODO to be synced with EN version
\section*{Voorwoord}

Er zijn verschillende populaire betekenissen voor de term \q{\gls{reverse engineering}}:
1) Reverse engineeren van software: gecompileerde programma\'s onderzoeken;
2) Scannen van 3D structuren en de onderliggende digitale bewerkingen om deze te kunnen dupliceren;
3) Het nabootsen van een \ac{DBMS} structuur.
Dit boek gaat over de eerste betekenis.

\subsection*{Oefeningen en opdrachten}

\dots 
zijn allen verplaatst naar de website: \url{http://challenges.re}.

\subsection*{Over de auteur}
\begin{tabularx}{\textwidth}{ l X }

\raisebox{-\totalheight}{
\includegraphics[scale=0.60]{Dennis_Yurichev.jpg}
}

&
Dennis Yurichev is een ervaren reverse engineer en programmeur.
Je kan hem contacteren via email: \textbf{\EMAIL{}}.

% FIXME: no link. \tablefootnote doesn't work
\end{tabularx}

% subsections:
\subsection*{%
	\RU{Отзывы о книге}%
	\EN{Praise for}%
	\ES{Elogios para}%
	\PTBRph{}%
	\DEph{}\PLph{}%
	\ITAph{}
	\IT{\TITLE}%
}

\begin{itemize}
% expanded URLs to make it more robust for printouts. In electronic editions people will click anyway, so tracking will keep working
\item \q{It's very well done .. and for free .. amazing.}\footnote{\href{http://go.yurichev.com/17095}{twitter.com/daniel\_bilar/status/436578617221742593}} Daniel Bilar, Siege Technologies, LLC.

\item \q{... excellent and free}\footnote{\href{http://go.yurichev.com/17096}{twitter.com/petefinnigan/status/400551705797869568}} Pete Finnigan,%
	\RU{гуру по безопасности}%
	\ES{gur\'u de seguridad en}%
	\PTBRph{}%
	\DEph{}\PLph{}%
	\ITAph{}
\oracle
	\EN{security guru}.

\item \q{... book is interesting, great job!} Michael Sikorski,
	\RU{автор книги}%
	\EN{author of}%
	\ES{autor de}%
	\PTBRph{}%
	\DEph{}\PLph{}%
	\ITAph{}
\IT{Practical Malware Analysis: The Hands-On Guide to Dissecting Malicious Software}.

\item \q{... my compliments for the very nice tutorial!} Herbert Bos,
	\RU{профессор университета}%
	\EN{full professor at the}%
	\ES{catedr\'atico de tiempo completo en la}%
	\PTBRph{}%
	\DEph{}\PLph{}%
	\ITAph{}
Vrije Universiteit Amsterdam,
	\RU{соавтор}%
	\EN{co-author of}%
	\ES{coautor de}%
	\PTBRph{}%
	\DEph{}\PLph{}%
	\ITAph{}
\IT{Modern Operating Systems (4th Edition)}.

\item \q{... It is amazing and unbelievable.} Luis Rocha, CISSP / ISSAP, Technical Manager, Network \& Information Security at Verizon Business.

\item \q{Thanks for the great work and your book.} Joris van de Vis,
	\RU{специалист по}%
	\ES{especialista en}%
	\PTBRph{}%
	\DEph{}\PLph{}%
	\ITAph{}
SAP Netweaver \& Security
	\EN{specialist}.

\item \q{... reasonable intro to some of the techniques.}\footnote{\href{http://go.yurichev.com/17099}{reddit}} Mike Stay,
	\RU{преподаватель в}%
	\EN{teacher at the}%
	\ES{profesor en el}%
	\PTBRph{}%
	\DEph{}\PLph{}%
	\ITAph{}
Federal Law Enforcement Training Center, Georgia, US.

\item \q{I love this book! I have several students reading it at the moment, plan to use it in graduate course.}\footnote{\href{http://go.yurichev.com/17097}{twitter.com/sergeybratus/status/505590326560833536}}
	\RU{Сергей Братусь}%
	\EN{Sergey Bratus}%
	\ES{Sergey Bratus}%
	\PTBRph{}%
	\DEph{}\PLph{}%
	\ITAph{},
Research Assistant Professor
	\RU{в отделе Computer Science в}%
	\EN{at the Computer Science Department at}%
	\ES{en el Departamento de Ciencias de la Computaci\'on en}%
	\PTBRph{}%
	\DEph{}\PLph{}%
	\ITAph{}
Dartmouth College

\item \q{Dennis @Yurichev has published an impressive (and free!) book on reverse engineering}\footnote{\href{http://go.yurichev.com/17098}{twitter.com/TanelPoder/status/524668104065159169}} Tanel Poder,
	\RU{эксперт по настройке производительности Oracle RDBMS}%
	\EN{Oracle RDBMS performance tuning expert}%
	\ES{experto en afinaci\'on de rendimiento de Oracle RDBMS}%
	\PTBRph{}%
	\DEph{}\PLph{}
	\ITAph{}.

\item \q{This book is some kind of Wikipedia to beginners...} Archer, Chinese Translator, IT Security Researcher.

\RU{\item \q{Прочел Вашу книгу~--- отличная работа, рекомендую на своих курсах студентам
в качестве учебного пособия}. Николай Ильин, преподаватель в ФТИ НТУУ \q{КПИ} и DefCon-UA}
\end{itemize}

\ifdefined\RUSSIAN
\newcommand{\PeopleMistakesInaccuracies}{Станислав \q{Beaver} Бобрицкий, Александр Лысенко, Shell Rocket, Zhu Ruijin, Changmin Heo, Александр \q{Solar Designer} Песляк, Vitor Vidal, Марк Уилсон.}
\else
\newcommand{\PeopleMistakesInaccuracies}{Stanislav \q{Beaver} Bobrytskyy, Alexander Lysenko, Shell Rocket, Zhu Ruijin, Changmin Heo, Alexander \q{Solar Designer} Peslyak, Vitor Vidal, Mark Wilson.}
\fi

\EN{\input{thanks_EN}}
\ES{\input{thanks_ES}}
\NL{\input{thanks_NL}}
\RU{\input{thanks_RU}}


%\input{FAQ_NL} % to be translated

% {\RU{Целевая аудитория}\EN{Target audience}}

\subsection*{Over de Koreaanse vertaling}

In Januari 2015 heeft de Acorn uitgeverij (\href{http://www.acornpub.co.kr}{www.acornpub.co.kr}) in Zuid Korea een enorme hoeveelheid werk verricht in het vertalen en uitgeven
van mijn boek (zoals het was in augustus 2014) in het Koreaans.

Het is nu beschikbaar op
\href{http://go.yurichev.com/17343}{hun website}

\iffalse
\begin{figure}[H]
\centering
\includegraphics[scale=0.3]{acorn_cover.jpg}
\end{figure}
\fi

De vertaler is Byungho Min (\href{http://go.yurichev.com/17344}{twitter/tais9}).
De cover art is verzorgd door mijn artistieke vriend, Andy Nechaevsky
\href{http://go.yurichev.com/17023}{facebook/andydinka}.
Zij bezitten ook de auteursrechten voor de Koreaanse vertaling.
Dus, als je een \IT{echt} boek op je kast wil in het Koreaans en je
wil mijn werk steunen, is het nu beschikbaar voor verkoop.

%\subsection*{About the Persian/Farsi translation}
%TBT

}
\IT{\input{preface_IT}}



\chapter{\IFRU{Об авторе}{About author}}

\IFRU{Денис Юричев ~--- опытный reverse engineer, свободный для найма как reverse engineer, консультант, тренер. 
С его резюме можно ознакомиться \href{http://yurichev.com/Dennis_Yurichev.pdf}{здесь}.}
{Dennis Yurichev is experienced reverse engineering, available for hire as reverse engineer, consultant, trainer. 
His CV is available \href{http://yurichev.com/Dennis_Yurichev.pdf}{here}.}

\chapter{\IFRU{Благодарности}{Thanks}}

\IFRU{Андрей ''herm1t'' Баранович, Слава ''Avid'' Казаков, Станислав ''Beaver'' Бобрицкий, Александр Лысенко, 
Александр ''Lstar'' Черненький, Андрей Зубинский}
{Andrey ''herm1t'' Baranovich, Slava ''Avid'' Kazakov, Stanislav ''Beaver'' Bobrytskyy, Alexander Lysenko, 
Alexander ''Lstar'' Chernenkiy, Andrew Zubinski}, Arnaud Patard (rtp \IFRU{на}{on} \#debian-arm IRC), 
\IFRU{и всем тем на github.com кто присылал замечания и коррективы}{and all folks on github.com
for notes and correctives}.

\mainmatter

% only chapters here!
\chapter{\IFRU{Паттерны компиляторов}{Compiler's patterns}}

\IFRU
{Когда я учил Си, а затем Си++, я просто писал небольшие фрагменты кода, компилировал и смотрел что 
получилось на ассемблере, так мне понять было намного проще. Я делал это такое количество раз, 
что связь между кодом на \CCpp и тем что генерирует компилятор вбилась мне в подсознание достаточно 
глубоко, поэтому я могу глядя на код на ассемблере сразу понимать, в общих чертах, что там было написано 
на Си. Возможно это поможет кому-то еще, попробую описать некоторые примеры.}
{When I first learn C and then C++, I was just writing small pieces of code, compiling it and see what 
is producing in assembly language, that was an easy way to me. I did it a lot times and relation 
between \CCpp code and what compiler produce was imprinted in my mind so deep so that 
I can quickly understand what was in C code when I look at produced x86 code. 
Perhaps, this method may be helpful for anyone else, so I'll try to describe here some examples.}

\section{\HelloWorldSectionName}
\label{sec:helloworld}

\IFRU{Начнем с знаменитого примера из книги}{Let's start with that famous example from the book}
``The C programming Language''\cite{Kernighan:1988:CPL:576122}:

\lstinputlisting{01_helloworld/1_1.c}

\subsection{x86}

\subsubsection{MSVC}

\IFRU{Компилируем в}{Let's compile it in} MSVC 2010: \TT{cl 1.cpp /Fa1.asm}

\IFRU
{(Ключ /Fa означает сгенерировать листинг на ассемблере)}
{(/Fa option mean generate assembly listing file)}

\begin{lstlisting}[caption=MSVC 2010]
CONST	SEGMENT
$SG3830	DB	'hello, world', 00H
CONST	ENDS
PUBLIC	_main
EXTRN	_printf:PROC
; Function compile flags: /Odtp
_TEXT	SEGMENT
_main	PROC
	push	ebp
	mov	ebp, esp
	push	OFFSET $SG3830
	call	_printf
	add	esp, 4
	xor	eax, eax
	pop	ebp
	ret	0
_main	ENDP
_TEXT	ENDS
\end{lstlisting}

\IFRU{MSVC выдает листинки в Intel-овском синтаксисе.}{MSVC produces assembly listings in Intel-syntax.} 
\IFRU{Разница между Intel-синтаксисом и AT\&T будет рассмотрена немного позже.}{The difference between 
Intel-syntax and AT\&T-syntax will be discussed below.}

\IFRU{Компилятор сгенерировал файл \TT{1.obj}, который впоследствии будет слинкован линкером в \TT{1.exe}.} 
{Compiler generated \TT{1.obj} file which will be linked into \TT{1.exe}.}

\IFRU{В нашем случае, этот файл состоит из двух сегментов: \TT{CONST} (для данных-констант) и \TT{\_TEXT} (для кода).}
{In our case, the file contain two segments: \TT{CONST} (for data constants) and \TT{\_TEXT} (for code).} 

\index{\CLanguageElements!const}
\IFRU{Строка \TT{``hello, world''} в \CCpp имеет тип \TT{const char*}, однако не имеет имени.}
{The string \TT{``hello, world''} in \CCpp has type \TT{const char*}, however hasn't its own name.}

\IFRU{Но компилятору нужно как-то с ней работать, так что он дает ей внутреннее имя \TT{\$SG3830}.}
{But compiler need to work with the string somehow, so it define internal name \TT{\$SG3830} for it.}

\IFRU{Как видно, строка заканчивается нулевым байтом ~--- это требования стандарта \CCpp для строк.}
{As we can see, the string is terminated by zero byte ~--- it's \CCpp standard for strings.}

\IFRU{В сегменте кода \TT{\_TEXT} находится пока только одна функция ~--- \main.}
{In the code segment \TT{\_TEXT} there are only one function so far ~--- \main.}

\IFRU{Функция \main, как и практически все функции, начинается с пролога и заканчивается эпилогом.}
{Function \main starting with prologue code and ending with epilogue code, like almost any function.}

\IFRU{Об этом смотрите подробнее в разделе о прологе и эпилоге функции}
{Read more about it in section about function prolog and epilog}
~\ref{sec:prologepilog}.

\index{x86!\Instructions!CALL}
\IFRU{Далее следует вызов функции \printf}
{After function prologue we see a function \printf call}: \TT{CALL \_printf}. 

\index{x86!\Instructions!PUSH}
\IFRU
{Перед этим вызовом, адрес строки (или указатель на нее) с нашим приветствием при помощи инструкции \PUSH помещается в стек.}
{Before the call, string address (or pointer to it) containing our greeting is placed into stack with help of \PUSH instruction.}

\IFRU{После того как функция \printf возвращает управление в функцию \main, адрес строки (или указатель на нее) все еще лежит в стеке.}
{When \printf function returning control flow to \main function, string address (or pointer to it) is still in stack.}

\IFRU{Так как он больше не нужен, то указатель стека (регистр \ESP) корректируется.} 
{Because we do not need it anymore, stack pointer (\ESP register) is to be corrected.}

\index{x86!\Instructions!ADD}
\TT{ADD ESP, 4} \IFRU{означает прибавить 4 к значению в регистре \ESP.}
{mean add 4 to the value in \ESP register.}

\IFRU
{Почему 4? Так как, это 32-битный код, для передачи адреса нужно аккурат 4 байта. В x64-коде это 8 байт.}
{Why 4? Since it is 32-bit code, we need exactly 4 bytes for address passing through the stack. 
It's 8 bytes in x64-code}

\TT{``ADD ESP, 4''} \IFRU{эквивалентно \TT{``POP регистр''}, но без использования какого-либо регистра\footnote{Флаги
процессора, впрочем, модифицируются}.}
{is equivalent to \TT{``POP register''} but without any register usage\footnote{CPU flags, however, modified}.}

\index{Intel C++}
\index{Oracle RDBMS}
\index{x86!\Instructions!POP}
\IFRU{Некоторые компиляторы, например Intel C++ Compiler, в этой же ситуации, могут вместо 
\ADD сгенерировать \TT{POP ECX} (подобное можно встретить например в коде \oracle{}, им скомпилированном), 
что почти то же самое, только портится значение в регистре \ECX.}
{Some compilers like Intel C++ Compiler, at the same point, could emit \TT{POP ECX} 
instead of \ADD (for example, such pattern can be observed in \oracle{} code, compiled by Intel C++ compiler), 
and this instruction has almost the same effect, but \ECX register contents will be rewritten.}

\IFRU
{Возможно, компилятор применяет \TT{POP ECX} потому что эта инструкция короче (1 байт против 3).}
{Probably, Intel C++ compiler using \TT{POP ECX} because this instruction's opcode is shorter then 
\TT{ADD ESP, x} (1 byte against 3).}

\IFRU{О стеке можно прочитать в соответствующем разделе}{Read more about stack in relevant section}~\ref{sec:stack}.

\index{\CLanguageElements!return}
\IFRU{После вызова \printf, в оригинальном коде на \CCpp указано \TT{return 0} ~--- вернуть 0 
в качестве результата функции \main.} 
{After \printf call, in original \CCpp code was \TT{return 0} ~--- return zero as a \main function result.} 

\index{x86!\Instructions!XOR}
\IFRU{В сгенерированном коде это обеспечивается инструкцией}
{In the generated code this is implemented by instruction} \TT{XOR EAX, EAX} 

\index{x86!\Instructions!MOV}
\IFRU{\XOR, на самом деле, как легко догадаться, ``исключающее ИЛИ''}
{\XOR, in fact, just ``eXclusive OR''}
\footnote{\url{http://en.wikipedia.org/wiki/Exclusive_or}}, 
\IFRU{но компиляторы часто используют его вместо простого}
{but compilers using it often instead of}
\TT{MOV EAX, 0} ~--- 
\IFRU
{потому что снова опкод короче (2 байта против 5).}
{slightly shorter opcode again (2 bytes against 5).}

\index{x86!\Instructions!SUB}
\IFRU{Бывает так, что некоторые компиляторы генерируют}{Some compilers emitting} 
\TT{SUB EAX, EAX}, 
\IFRU
{что значит, \IT{отнять значение \EAX от \EAX}, в любом случае это даст 0 в результате.}
{which mean \IT{SUBtract \EAX value from \EAX}, which is in any case will result zero.}

\index{x86!\Instructions!RET}
\IFRU{Самая последняя инструкция \RET возвращает управление в вызывающую функцию.
Обычно, это код \CCpp CRT\footnote{C Run-Time Code}, который, в свою очередь, 
вернет управление операционной системе.}
{Last instruction \RET returning control flow to calling function.
Usually, it's \CCpp CRT\footnote{C Run-Time Code} code, which, in turn, 
return control to operation system.}

\subsubsection{GCC}

\IFRU{Теперь скомпилируем то же самое компилятором GCC 4.4.1 в Linux}
{Now let's try to compile the same \CCpp code in GCC 4.4.1 compiler in Linux}: \TT{gcc 1.c -o 1}

\IFRU{Затем при помощи \IDA. посмотрим как создалась функция \main.}
{After, with the \IDA disassembler assistance, let's see how \main function was created.} 

(\IDA, \IFRU{как и MSVC, показывает код в Intel-синтаксисе}{as MSVC, showing code in Intel-syntax}).

\IFRU{Замечание: мы также можем заставить GCC генерировать листинги в этом формате при помощи ключа}
{Note: we could also switch GCC to produce assembly listings in Intel-syntact by applying option} 
\TT{-S -masm=intel}

\begin{lstlisting}[caption=GCC]
main            proc near

var_10          = dword ptr -10h

                push    ebp
                mov     ebp, esp
                and     esp, 0FFFFFFF0h
                sub     esp, 10h
                mov     eax, offset aHelloWorld ; "hello, world"
                mov     [esp+10h+var_10], eax
                call    _printf
                mov     eax, 0
                leave
                retn
main            endp
\end{lstlisting}

\index{Function prologue}
\index{x86!\Instructions!AND}
\IFRU{Почти то же самое. 
Адрес строки ``hello, world'' лежащей в сегменте данных, в начале сохраняется в \EAX, затем записывается в стек.
А еще в прологе функции мы видим \TT{AND ESP, 0FFFFFFF0h} ~--- 
эта инструкция выравнивает значение в \ESP по 16-байтной границе, делая все значения 
в стеке также выровненными по этой границе (процессор более эффективно работает с переменными расположенными
в памяти по адресам кратным 4 или 16)\footnote{\URLWPDA}.}
{Almost the same.
Address of ``hello world'' string (stored in data segment) is saved in \EAX register first, then it stored into stack.
Also, in function prologue we see \TT{AND ESP, 0FFFFFFF0h} ~--- 
this instruction aligning \ESP value on 16-byte border, resulting all values in stack aligned too
(CPU performing better if values it working with are located in memory at addresses aligned by 
4 or 16 byte border)\footnote{\URLWPDA}.}

\index{x86!\Instructions!SUB}
\TT{SUB ESP, 10h} \IFRU{выделяет в стеке 16 байт, хотя, как будет видно далее, здесь достаточно только 4.}
{allocate 16 bytes in stack, although, as we could see below, only 4 need here.} 

\IFRU{Это происходит потому что количество выделяемого места в локальном стеке тоже выровнено по 
16-байтной границе.}{This is because the size of allocated stack is also aligned on 16-byte border.}

% TODO: rewrite.
\index{x86!\Instructions!PUSH}
\IFRU{Адрес строки (или указатель на строку) затем записывается прямо в стек без помощи инструкции \PUSH.
\IT{var\_10} по совместительству ~--- и локальная переменная и одновременно аргумент для \printf{}. Подробнее об этом будет ниже.}
{String address (or pointer to string) is then writing directly into stack space without \PUSH instruction use.
\IT{var\_10} ~--- is local variable, but also argument for \printf{}. Read below about it.}

\IFRU{Затем вызывается \printf.}{Then \printf function is called.}

\IFRU{В отличие от MSVC, GCC в компиляции без включенной оптимизации генерирует \TT{MOV EAX, 0} вместо 
более короткого опкода.}{Unlike MSVC, GCC while compiling without optimization turned on, 
emitting \TT{MOV EAX, 0} instead of shorter opcode.}

\index{x86!\Instructions!LEAVE}
\IFRU{Последняя инструкция \LEAVE ~--- это аналог команд \TT{MOV ESP, EBP} и \TT{POP EBP} ~--- 
то есть возврат указателя стека и регистра \EBP в первоначальное состояние.} 
{The last instruction \LEAVE ~--- is \TT{MOV ESP, EBP} and \TT{POP EBP} instructions pair equivalent ~--- 
in other words, this instruction setting back stack pointer (\ESP) and \EBP register to its initial state.} 

\IFRU{Это необходимо, т.к., в начале функции мы модифицировали регистры \ESP и \EBP (при помощи}
{This is necessary because we modified these register values (\ESP and \EBP) at the function start (executing}
\TT{\MOV EBP, ESP} / \TT{AND ESP, ...}).

\subsubsection{GCC: \ATTSyntax}

\IFRU{Попробуем посмотреть, как выглядит то же самое в AT\&T-синтаксисе языка ассемблера.}
{Let's see how this can be represented in AT\&T syntax of assembly language.}
\IFRU{Этот синтаксис больше распространен в UNIX-мире.}
{This syntax is much more popular in UNIX-world.}

\begin{lstlisting}[caption=\IFRU{компилируем в}{let's compile in} GCC 4.7.3]
gcc -S 1_1.c
\end{lstlisting}

\IFRU{Получим такой файл:}{We got this:}

\lstinputlisting[caption=GCC 4.7.3]{01_helloworld/1_1.s}

\IFRU{Здесь много макросов (начинающихся с точки), которые пока нам не интересны.}
{There are a lot of macros (started with dot), which are not very interesting to us so far.}
\IFRU{Пока что, ради упрощения, мы можем
их игнорировать и впредь (кроме макроса \IT{.string}, при помощи которого кодируется последовательность символов 
оканчивающихся нулем, такие же строки как в Си) и тогда получится следующее}
{For now, for the sake of simplification, we can ignore them (except \IT{.string} macro, which
encode null-terminated characters sequence, just like C-strings) and then we'll see this}
\footnote{\IFRU{Кстати, для уменьшения генерации ``лишних'' макросов, можно использовать такой ключ GCC}
{By the way, for eliminating ``unnecessary'' macros, this GCC option can be used}: 
\IT{-fno-asynchronous-unwind-tables}}:

\lstinputlisting[caption=GCC 4.7.3]{01_helloworld/1_1_refined.s}

\index{\ATTSyntax}
\index{\IntelSyntax}
\IFRU{Основные отличия синтаксиса Intel и AT\&T следующие:}{Major differences between Intel and AT\&T syntax are:}

\begin{itemize}

\item
\IFRU{Операнды записываются наоборот.}{Operands are written backwards.}

\IFRU{В Intel-синтаксисе: <инструкция> <операнд назначения> <операнд-источник>.}
{In Intel-syntax: <instruction> <destination operand> <source operand>.}

\IFRU{В AT\&T-синтаксисе: <инструкция> <операнд-источник> <операнд назначения>.}
{In AT\&T syntax: <instruction> <source operand> <destination operand>.}

\IFRU{Чтобы легче понимать разницу, можно запомнить следующее}
{Here is a thing can be memorized for easier difference understanding}: \IFRU{когда вы работаете с Intel-синтаксисом, можете в уме ставить знак равенства ($=$) между операндами}
{when you work with Intel-syntax, you can put equality sign ($=$) in your mind between operands}, 
\IFRU{а когда с AT\&T-синтаксисом, мысленно ставьте стрелку направо}{and when with AT\&T-syntax, put right arrow} 
($\rightarrow$)
\footnote{
\index{\CStandardLibrary!memcpy()}
\index{\CStandardLibrary!strcpy()}
\IFRU{Кстати, в некоторые стандартных функциях библиотеки Си (например, memcpy(), strcpy()) также применяется 
расстановка аргументов как в Intel-синтаксисе: в начале указатель в памяти на блок назначения, 
затем указатель на блок-источник.}{By the way, in some C standard functions (e.g., memcpy(), strcpy()), arguments
are listed in the same way as in Intel-syntax: pointer to destination memory block at the beginning and then
pointer to source memory block.}}.

\item
\IFRU{Перед именами регистров ставится знак процента (\%), а перед числами знак доллара (\$).}
{Before registers names, percent sign should be written (\%), and dollar sign (\$) before numbers.}
\IFRU{Вместо квадратных скобок применяются круглые.}{Parentheses are used instead of brackets.}

\item
\IFRU{К каждой инструкции добавляется специальный символ, определяющий тип данных:}
{A special symbol is to be added to each instruction, defining type of data:}

\begin{itemize}
\item l --- long (32 \IFRU{бита}{bits})
\item w --- word (16 \IFRU{бит}{bits})
\item b --- byte (8 \IFRU{бит}{bits})
\end{itemize}

\end{itemize}

\IFRU{Возвращаясь к результату компиляции: он идентичен тому, который мы посмотрели в \IDA.}
{Let's return back to compilation result: it is identical to which we saw in \IDA.}
\IFRU{Одна мелочь}{One small difference}: \TT{0FFFFFFF0h} \IFRU{записывается как}{is written as} \TT{\$-16}.
\IFRU{Это тоже самое}{It is the same}: \TT{16} \IFRU{в десятичной системе это}{in decimal system is} \TT{0x10} 
\IFRU{в шестнадцатеричной}{in hexadecimal}. 
\TT{-0x10} \IFRU{будет как раз}{is exactly} \TT{0xFFFFFFF0} 
(\IFRU{в рамках 32-битных чисел}{within 32-bit data type}).




\subsection{ARM}
\label{sec:hw_ARM}

\index{\idevices}
\index{Xcode}
\index{LLVM}
\index{Keil}
\IFRU{Для экспериментов с процессором ARM, я выбрал два компилятора}{For my experiments with ARM CPU I choose two compilers}: \IFRU{популярный в embedded-среде}{popular in embedded area} Keil Release 6/2013 
\IFRU{и среду разработки}{and} Apple Xcode 4.6.3 \IFRU{}{IDE} (\IFRU{с компилятором}{with} LLVM-GCC 4.2 \IFRU{}{compiler}), \IFRU{генерирующую код для ARM-совместимых процессоров и}{producing code for ARM-compatible processors and} \IFRU{SoC}{SoCs}\footnote{system on chip} \IFRU{в}{in} \idevices, 
\IFRU{планшетных компьютеров для Windows 8 и Windows RT}{Windows 8 and Window RT tables}\footnote{\url{http://en.wikipedia.org/wiki/List_of_Windows_8_and_RT_tablet_devices}} 
\IFRU{и таких устройствах как}{and also such devices as} Raspberry Pi.

\subsubsection{\NonOptimizingKeil + \ARMMode}

\IFRU{Для начала, скомпилируем наш пример в Keil}{Let's start by compiling our example in Keil}:

\begin{lstlisting}
armcc.exe --arm --c90 -O0 1.c 
\end{lstlisting}

\IFRU{Компилятор \IT{armcc} генерирует листинг на ассемблере}{\IT{armcc} compiler producing assembly listing}, 
\IFRU{но он содержит некоторые высокоуровневые макросы связанные с ARM}{but it has some high-level ARM-processor related macros}\footnote{
\IFRU{например, он показывает инструкции \PUSH/\POP отсутствующие в режиме ARM}{for example, ARM mode lacks 
\PUSH/\POP instructions}}, 
\IFRU{а нам важнее увидеть инструкции ``как есть'', так что посмотрим скомпилированный результат в \IDA}{but it's more important for us to see instructions ``as is'', so let's see compiled results in \IDA}.

\begin{lstlisting}[caption=\NonOptimizingKeil + \ARMMode + \IDA]
.text:00000000             main
.text:00000000 10 40 2D E9                 STMFD   SP!, {R4,LR}
.text:00000004 1E 0E 8F E2                 ADR     R0, aHelloWorld ; "hello, world"
.text:00000008 15 19 00 EB                 BL      __2printf
.text:0000000C 00 00 A0 E3                 MOV     R0, #0
.text:00000010 10 80 BD E8                 LDMFD   SP!, {R4,PC}

.text:000001EC 68 65 6C 6C+aHelloWorld     DCB "hello, world",0    ; DATA XREF: main+4
\end{lstlisting}

\index{ARM!\ARMMode}
\index{ARM!\ThumbMode}
\index{ARM!\ThumbTwoMode}
\IFRU{Вот чуть-чуть фактов о процессоре ARM, которые желательно знать}{Here is couple of ARM-related facts we should know in order to proceed}.
\IFRU{Процессор ARM имеет по крайней мере два основных режима: режим ARM и thumb}{ARM processor has at least two major modes: ARM mode and thumb}. 
\IFRU{В первом (ARM) режиме доступны все инструкции и каждая имеет размер 32 бита (или 4 байта)}{In first (ARM) mode all instructions are enabled and each has 32-bit (4 bytes) size}. 
\IFRU{Во втором режиме (thumb) каждая инструкция имеет размер 16 бит (или 2 байта)}{In second (thumb) mode each instruction has 16-bit (or 2 bytes) size}\footnote{\IFRU{Кстати, инструкции фиксированного размера удобны тем, что всегда можно легко узнать адрес предыдущей инструкции, или следующей}{NOTTRANSLATED}}. 
\IFRU{Режим thumb может выглядеть привлекательнее тем, что программа на нем может быть 1) компактнее; 2) эффективнее исполняться на микроконтроллере с 16-битной шиной данных}{Thumb mode may look attractive because program in it may be 1) compact; 2) executing faster on microcontroller having 16-bit memory datapath}. 
\IFRU{Но за всё нужно платить: в режиме thumb куда меньше возможностей процессора, например, возможен доступ только к 8-и регистрам процессора, и чтобы совершить некоторые действия, выполнимые в режиме ARM одной инструкцией, нужны несколько thumb-инструкций}{Nothing come for free of charge, so, in thumb mode there are reduced instruction set, only 8 registers are accessible and one need several thumb instructions for doing some operations when in ARM mode you'll need just one}.
\IFRU{Начиная с ARMv7, имеется также поддержка инструкций thumb-2, это thumb расширенный до поддержки куда большего числа инструкций}{Starting at ARMv7, there are also thumb-2 instructions set present, this is a thumb extended to support much bigger instructions set}.
\IFRU{Распространено заблуждение что thumb-2 это смесь ARM и thumb. Это не верно. Просто thumb-2 был дополен до
более полной поддержки возможностей процессора, что теперь может легко конкурировать с режимом ARM.}
{There is a common misconception that thumb-2 is a mix of ARM and thumb. It's not correct. 
But rather thumb-2 was extended to support processor features so fully,
so now it can compete with ARM mode.}
\IFRU{Программа для процессора ARM может представлять смесь процедур скомпилированных для обоих режимов}{A program for ARM processor may be mix of procedures compiled for both modes}.
\IFRU{Основное количество приложений для \idevices скомпилировано для набора инструкций thumb-2, потому что Xcode
делает так по умолчанию}{Majority of \idevices applications are compiled for thumb-2 instructions set, because Xcode do this by default}.

\IFRU{В вышеприведененном примере можно легко увидеть что каждая инструкция имеет размер 4 байта}{In example we see here we can easily see that each instruction has size of 4 bytes}.
\IFRU{Действительно, ведь мы же компилировали наш код для режима ARM а не thumb}{Indeed, we compiled our code for ARM mode, but for thumb}.

\index{ARM!\Instructions!STMFD}
\index{ARM!\Instructions!POP}
\IFRU{Самая первая инструкция}{The very first instruction} \TT{''STMFD SP!, \{R4,LR\}''}\footnote{Store Multiple Full Descending} \IFRU{работает как инструкция}{works here as} \PUSH \IFRU{в}{in} x86, \IFRU{записывает значения двух регистров}{instruction, writing values of two} (\TT{R4} \IFRU{и}{and} \LR) \IFRU{в стек}{registers into stack}. 
\IFRU{Действительно, в выдаваемом листинге на ассемблере, компилятор \IT{armcc}, для упрощения, указывает здесь инструкцию}{Indeed, in output listing, \IT{armcc} compiler, for the sake of simplification, showing here} \TT{''PUSH \{r4,lr\}''}\IFRU{}{instruction}.
\IFRU{Но это не совсем точно, инструкция \PUSH доступна только в режиме thumb, поэтому, во избежания путанницы, я предложил работать в \IDA}{But it's not quite correct, \PUSH instruction available only in thumb mode, so, to make things less messy, I offered to work in \IDA}.

\IFRU{Итак, эта инструкция записывает значения регистров \TT{R4} и \LR по адресу в памяти, на который указывает регистр \SPwithfootnote, затем уменьшает \TT{SP}, чтобы он указывал на место в стеке, доступное для новых записей}{So this instruction writes values of \TT{R4} and \LR registers at the address in memory to which \SPwithfootnote pointing, then decrements \SP so it will points to a place in stack free for new entries}.

\IFRU{Эта инструкция, как и инструкция \PUSH в режиме thumb, может сохранить в стеке одновременно несколько значений регистров, что может быть очень удобно}{This instruction, like \PUSH instruction in thumb mode, is able save several register values at once and this may be useful}. 
\IFRU{Кстати, такого в x86 нет}{By the way, there is no such thing in x86}. 
\IFRU{Так же следует заметить, что \TT{STMFD} ~--- генерализация инструкции \PUSH (то есть, расширяет её возможности), потому что может работать с любым регистром а не только с \SP, это тоже может быть очень удобно}{It's also can be noted that \TT{STMFD} ~--- generalization of \PUSH instruction (extending its features), because it can work with any register, not just with \SP and this can be very useful}.

\index{\PICcode}
\index{ARM!\Instructions!ADR}
\IFRU{Инструкция}{} \TT{''ADR R0, aHelloWorld''} \IFRU{прибавляет значение регистра \PC к смещению, где хранится строка}{instruction adding \PC register value to the offset, where the} \IT{``hello, world''} \IFRU{}{string is located}. 
\IFRU{Причем здесь \PC, можно спросить}{How \TT{PC} register used here, one might ask}?
\IFRU{Притом, что это так называемый ``\PICcode''}{This is so called ``\PICcode''}
\footnote{\IFRU{Читайте больше об этом в соответствующем разделе}{Read more about it in relevant section}~\ref{sec:PIC}}, 
\IFRU{он предназначен для исполнения будучи не привязанным к каким-либо адресам в памяти}{it is intended to be executed not to be fixed to any addresses in memory}.
\IFRU{В опкоде инструкции \TT{ADR} указывается разница между адресом этой инструкции и местом, где хранится строка}{In the opcode of \TT{ADR} instruction, here is encoded a difference between address of this instruction and the place where the string is located}.
\IFRU{Эта разница всегда будет постоянной, вне зависимости от того, куда был загружен операционной системой наш код}{Difference will always be constant, without any dependence to the address where that code being loaded, by operation system, presumably}. 
\IFRU{Поэтому всё что нужно это прибавить адрес текущей инструкции (из \PC) чтобы получить текущий абсолютный адрес нашей Си-строки}{That's why all we need is to add address of current instruction (from \PC) in order to get absolute address of our C-string in memory}.

\index{ARM!\Registers!Link Register}
\index{ARM!\Instructions!BL}
\IFRU{Инструкция}{} \TT{''BL \_\_2printf''}\footnote{Branch with Link} \IFRU{вызывает функцию \printf}{instruction calling \printf function}. 
\IFRU{Работа этой инструкции состоит из двух фаз}{That's how this instruction works}: 
\begin{itemize}
\item
\IFRU{записать адрес после инструкции \TT{BL} ($0xC$) в регистр \LRwithfootnote}
{write address after \TT{BL} instruction ($0xC$) into \LRwithfootnote register};
\item
\IFRU{затем собственно передать управление в \printf, записав адрес этой функции в регистр \PCwithfootnote}
{then pass control flow into \printf by writing its address into \PCwithfootnote register}.
\end{itemize}

\IFRU{Ведь, когда функция \printf закончит работу, нужно знать, куда вернуть управление, поэтому закончив работу, всякая функция передает управление по адресу записанному в регистре \LR}
{Because, when \printf finishes its work, it should have information, where it should return control, that's why each function passes control to the address stored in \LR register}.

\IFRU{В этом разница между ``чистыми'' RISC-процессорами вроде ARM и x86, где адрес возврата записывается в стек}{That is the difference between ``pure'' RISC-processors like ARM and x86, where address of return is stored in stack}\footnote{\IFRU{Подробнее об этом будет описано в следующей главе}{Read more about this in next section}~\ref{sec:stack}}.

\IFRU{Кстати, 32-битный абсолютный адрес, либо же смещение, невозможно закодировать в 32-битной инструкции \TT{BL}, в ней есть место только для 24-х бит}
{By the way, absolute 32-bit address or offset cannot be encoded in 32-bit \TT{BL} instruction, because it has space only for 24 bits}.
\IFRU{Так же следует отметить, что из-за того что все инструкции в режиме ARM имеют длину 4 байта (32 бита), и инструкции могут находится только по адресам кратным 4, то последние 2 бита (всегда нулевых) можно не кодировать.}
{It's also worth to note that all ARM mode instructions has size 4 bytes (32 bits), hence they all can be located only on 4-byte boundary addresses. This mean, last 2 bits of instruction address (always zero bits) may be omitted.}
\IFRU{В итоге имеем 26 бит, при помощи которых можно закодировать смещение}
{In summary, we have 26 bit for offset encoding, this is enough to represent offset} $\pm{}\approx{}32M$.

\index{ARM!\Instructions!MOV}
\IFRU{Следующая инструкция}{Next} \TT{''MOV R0, \#0''}\footnote{MOVe} \IFRU{просто записывает $0$ в регистр \Rzero}{instruction just writes $0$ into \Rzero register}.
\IFRU{Ведь наша Си-функция возвращает $0$ а возвращаемое значение всякая функция оставляет в \Rzero}{That's because our C-function returning $0$ and returning value is to be placed in \Rzero}.

\index{ARM!\Registers!Link Register}
\index{ARM!\Instructions!LDMFD}
\index{ARM!\Instructions!POP}
\IFRU{Последняя инструкция}{The last instruction} \TT{''LDMFD SP!, {R4,PC}''}\footnote{\LDMFDDESC} \IFRU{это инструкция обратная от}{is an inversive instruction of} \TT{STMFD}, \IFRU{она загружает из стека значения для сохранения их в \TT{R4} и \PC, увеличивая указатель стека \SP}{it loads values from stack for saving them into \TT{R4} and \PC, incremeting stack pointer \SP}.
\IFRU{Это, в каком-то смысле, аналог \POP}{It can be said, it is similar to \POP}. 
\IFRU{Обратите внимание: самая первая инструкция \TT{STMFD} сохранила в стеке \TT{R4} и \LR, а \IT{восстанавливаются} \TT{R4} и \PC}{Note: the very first instruction \TT{STMFD} saved \TT{R4} and \LR into stack, but \TT{R4} and \PC are \IT{restored}}.
\IFRU{Как я уже описывал, в регистре \LRwithfootnote обычно сохраняется адрес места, куда нужно всякой функции вернуть управление}{As I wrote before, in \LRwithfootnote register address of place saved, to where each function should return control}.
\IFRU{Самая первая инструкция сохраняет это значение в стеке, потому что наша функция \main позже будет сама пользоваться этим регистром, в момент вызова \printf}{The very first function saving its value in stack because our \main function will use that register in order to call \printf}.
\IFRU{А затем, в конце функции, это значение можно сразу записать в \PC, таким образом, передав управление туда, откуда была вызвана наша функция}{And then, in the function end this value can be written to \PC, thus, by passing control to where our function was called}.
\IFRU{Так как функция \main обычно самая главная в \CCpp, вероятно, управление будет возвращено в загрузчик операционной системы, либо куда-то в runtime функции Си, или что-то в этом роде}
{Since our \main function is usually primary function in \CCpp, apparently, control will be returned to operation system loader or to some place in runtime C functions, or something like that}.

\index{ARM!DCB}
\TT{DCB} ~--- \IFRU{директива ассемблера, описывающая массивы байт или ASCII-строк, аналог директивы DB в 
x86-ассемблере}
{assembly language directive, defining array of bytes or ASCII-strings, similar to DB directive 
in x86-assembly language}.

\subsubsection{\NonOptimizingKeil: \ThumbMode}

\IFRU{Скомпилируем тот же пример в Keil для режима thumb}{Let's compile the same example in Keil in thumb mode}:

\begin{lstlisting}
armcc.exe --thumb --c90 -O0 1.c 
\end{lstlisting}

\IFRU{Получим (в \IDA)}{We will get (in \IDA)}:

\begin{lstlisting}[caption=\NonOptimizingKeil + \ThumbMode + \IDA]
.text:00000000             main
.text:00000000 10 B5                       PUSH    {R4,LR}
.text:00000002 C0 A0                       ADR     R0, aHelloWorld ; "hello, world"
.text:00000004 06 F0 2E F9                 BL      __2printf
.text:00000008 00 20                       MOVS    R0, #0
.text:0000000A 10 BD                       POP     {R4,PC}

.text:00000304 68 65 6C 6C+aHelloWorld     DCB "hello, world",0    ; DATA XREF: main+2
\end{lstlisting}

\IFRU{Сразу бросаются в глаза двухбайтные (16-битные) опкоды, это, как я уже упоминал, thumb}{We can easily spot 2-byte (16-bit) opcodes, this is, as I mentioned, thumb}.
\index{ARM!\Instructions!BL}
\IFRU{Кроме инструкции \TT{BL}}{Except \TT{BL} instruction}.
\IFRU{Но на самом деле, она состоит из двух 16-битных инструкций}{In fact, it consisted in two 16-bit instructions}.
\IFRU{Это потому что загрузить в \PC смещение, по которому находится функция \printf, используя так мало места в одном 16-битном опкоде, очевидно, нельзя}{That's because it's not possible to load offset to \printf function into \PC when using so small space in one 16-bit opcode, obviously}.
\IFRU{Поэтому первая 16-битная инструкция загружает старшие 10 бит смещения, а вторая ~--- младшие 11 бит смещения}{That's why first 16-bit instruction loads higher 10 bits of offset and second ~--- loads 11 lower bits of offset}.
\IFRU{Как я уже упоминал, все инструкции в thumb-режиме имеют длину 2 байта или 16 бит}{As I mentioned, all instructions in thumb mode has size of 2 bytes or 16 bits}.
\IFRU{Поэтому невозможна такая ситуация, когда thumb-инструкция начинается по нечетному адресу}
{This mean, it's not possible for thumb-instruction to be on odd address whatsoever}.
\IFRU{Следовательно, последний бит адреса можно не кодировать}{Considering this, last address bit may be omitted while instruction encoding}.
\IFRU{Таким образом, в итоге, в thumb-инструкции \TT{BL} кодируется смещение}{Summarizing, in \TT{BL} thumb-instruction,} $\pm{}\approx{}2M$ \IFRU{от текущего адреса}{can be encoded as offset from current address}.

\IFRU{Остальные инструкции в функции: \PUSH и \POP работают почти так же как и описанные \TT{STMFD}/\TT{LDMFD}, только регистр \SP здесь не указывается явно}{Other instructions in functions are: \PUSH and \POP works just like described \TT{STMFD}/\TT{LDMFD}, but \SP register not mentioned explicitely here}.
\TT{ADR} \IFRU{работает также как и в предыдущем примере}{works just like in previous example}.
\TT{MOVS} \IFRU{записывает $0$ в регистр \Rzero для возврата нуля}{writes $0$ in \Rzero register to zero returning}.

\subsubsection{\OptimizingXcode + \ARMMode}

Xcode 4.6.3 \IFRU{без включенной оптимизации выдает слишком много лишнего кода, поэтому остановимся на той версии, где как можно меньше инструкций}{without optimization turned on, produces a lot of redundant code, so we'll study that version where instruction count as small as possible}: \Othree.

\begin{lstlisting}[caption=\OptimizingXcode + \ARMMode]
__text:000028C4             _hello_world
__text:000028C4 80 40 2D E9                 STMFD           SP!, {R7,LR}
__text:000028C8 86 06 01 E3                 MOV             R0, #0x1686
__text:000028CC 0D 70 A0 E1                 MOV             R7, SP
__text:000028D0 00 00 40 E3                 MOVT            R0, #0
__text:000028D4 00 00 8F E0                 ADD             R0, PC, R0
__text:000028D8 C3 05 00 EB                 BL              _puts
__text:000028DC 00 00 A0 E3                 MOV             R0, #0
__text:000028E0 80 80 BD E8                 LDMFD           SP!, {R7,PC}

__cstring:00003F62 48 65 6C 6C+aHelloWorld_0   DCB "Hello world!",0
\end{lstlisting}

\IFRU{Инструкции}{Instructions} \TT{STMFD} \IFRU{и}{and} \TT{LDMFD} \IFRU{нам уже знакомы}{are familiar to us}.

\IFRU{Инструкция \MOV просто записывает число $0x1686$ в регистр \Rzero, это смещение указывающее на строку ``Hello world!''}{\MOV instruction just writes $0x1686$ number into \Rzero register, this is offset pointing to the ``Hello world!'' string}.

\IFRU{Регистр \TT{R7}, по стандарту принятому в}{\TT{R7} register, as it is standardized in}\cite{IOSABI}
\IFRU{это}{is} frame pointer, \IFRU{о нем будет рассказано позже}{more on it below}.

\index{ARM!\Instructions!MOVT}
\IFRU{Инструкция}{} \TT{MOVT R0, \#0} \IFRU{записывает 0 в старшие 16 бит регистра}{instruction writes 0 into higher 16 bit of register}.
\IFRU{Дело в том, что обычная инструкция \MOV в режиме ARM может записывать какое-либо значение только в младшие 16 бит регистра, ведь, больше нельзя закодировать в ней}{The issue is here in that generic \MOV instruction in ARM mode may writes only lower 16 bit of register}.
\IFRU{Помните, что в режиме ARM опкоды всех инструкций ограничены длиной в 32 бита. Конечно, это ограничение не касается перемещений между регистрами.}{Remember, all instruction's opcodes in ARM mode are limited in size to 32 bits. Of course, this limitation is not related to moving between registers.}
\IFRU{Поэтому для записи в старшие биты (от 16-го по 31-го включительно) существует дополнительная команда \TT{MOVT}}{So that's why additional instruction \TT{MOVT} exist for writing into higher bits (from 16 to 31 inclusive)}.
\IFRU{Впрочем, здесь её использование избыточно, потому что инструкция \TT{''MOV R0, \#0x1686''} выше итак обнулила старшую часть регистра}{However, its usage here is redundant, because \TT{''MOV R0, \#0x1686''} instruction above cleared higher part of register}. \IFRU{Возможно, это недочет компилятора}{Probably, it's compiler's shortcoming}.

\index{ARM!\Instructions!ADD}
\IFRU{Инструкция} \TT{''ADD R0, PC, R0''} \IFRU{прибавляет \PC к \Rzero, для вычисления действительного адреса строки ``Hello world!'', как нам уже известно, это ``\PICcode'', поэтому такая корректива необходима}
{instruction adding \PC to \Rzero, for calculating absolute address of ``Hello world!'' string, 
and as we already know that, it's ``\PICcode'', so this corrective is essential here}.

\IFRU{Инструкция \TT{BL} вызывает \puts вместо \printf}{\TT{BL} instruction calling \puts instead of \printf}.

\label{puts}
\index{puts() \IFRU{вместо}{instead of} printf()}
\IFRU{Компилятор заменил вызов \printf на \puts. 
Действительно, \printf с одним агрументом это почти аналог \puts.}
{GCC replaced first \printf call to \puts. 
Indeed: \printf with sole argument is almost analogous to \puts.} 

\IFRU{\IT{Почти}, если принять условие что в строке не будет управляющих символов \printf 
начинающихся со знака процента. Тогда эффект от работы этих двух функций будет разным.}
{\IT{Almost}, because we need to be sure that this string will not contain printf-control 
statements starting with \IT{\%}: then effect of these two functions will be different.}

\IFRU{Зачем компилятор заменил один вызов на другой? Потому что \puts() работает быстрее}
{Why compiler replaced \printf to \puts? Because \puts() work faster}
\footnote{\url{http://www.ciselant.de/projects/gcc_printf/gcc_printf.html}}. 

\IFRU{Видимо потому, что \puts проталкивает символы в stdout не сравнивая каждый со знаком процента.}
{\puts working faster because it just passes characters to stdout not comparing each with \IT{\%} symbol.}

\IFRU{Далее уже знакомая инструкция}{Next, we see familiar to us} \TT{''MOV R0, \#0''}\IFRU{, служащая для установки в 0 возвращаемого значения функции}{instruction, intended to set 0 to \Rzero register}.

\subsubsection{\OptimizingXcode + \ThumbTwoMode}

\IFRU{По умолчанию}{By default}, Xcode 4.6.3 \IFRU{генерирует код для режима thumb-2, примерно в такой манере}{generating code for thumb-2 in such manner}:

\begin{lstlisting}[caption=\OptimizingXcode + \ThumbTwoMode]
__text:00002B6C                   _hello_world
__text:00002B6C 80 B5                             PUSH            {R7,LR}
__text:00002B6E 41 F2 D8 30                       MOVW            R0, #0x13D8
__text:00002B72 6F 46                             MOV             R7, SP
__text:00002B74 C0 F2 00 00                       MOVT.W          R0, #0
__text:00002B78 78 44                             ADD             R0, PC
__text:00002B7A 01 F0 38 EA                       BLX             _puts
__text:00002B7E 00 20                             MOVS            R0, #0
__text:00002B80 80 BD                             POP             {R7,PC}

...

__cstring:00003E70 48 65 6C 6C 6F 20+aHelloWorld     DCB "Hello world!",0xA,0
\end{lstlisting}

\index{ThumbTwoMode}
\IFRU{Инструкции \TT{BL} и \TT{BLX} в thumb, как мы помним, кодируются как пара 16-битных инструкций, 
а в thumb-2 эти \IT{суррогатные} опкоды расширены так, что новые инструкции кодируются здесь как 
32-битные инструкции}{\TT{BL} and \TT{BLX} instructions in thumb mode, as we remember, encoded as pair
of 16-bit instructions and in thumb-2, these \IT{surrogate} opcodes extended in such way so that new instruction
may be encoded here as 32-bit instructions}.
\IFRU{Это можно заметить по тому что опкоды thumb-2 инструкций всегда начинаются с $0xFx$ либо с $0xEx$}{That's
easily observable ~--- opcodes of thumb-2 instructions are also beginning with $0xFx$ or $0xEx$}.
\IFRU{Но в листинге \IDA, первый байт опкода стоит вторым, это из-за того что в ARM инструкции кодируются так:
в начале последний байт, потом первый (для thumb и thumb-2 режима), либо, 
(для инструкций в режиме ARM) в начале четвертый байт, затем третий, второй и первый}{But in \IDA listings,
first byte of opcode is at the place of second, that's because instructions here encoded as follows: last byte and then first one (for thumb and thumb-2 modes), or, (for instructions in ARM mode): fourth byte, then third, then second and first}.
\index{ARM!\Instructions!MOVW}
\index{ARM!\Instructions!MOVT.W}
\index{ARM!\Instructions!BLX}
\IFRU{Так что мы видим здесь что инструкции \TT{MOVW}, \TT{MOVT.W} и \TT{BLX} начинаются с}{So as we see, \TT{MOVW}, \TT{MOVT.W} and \TT{BLX} instructions are beginning with} $0xFx$.

\IFRU{Одна из thumb-2 инструкций это}{One of thumb-2 instructions is} \TT{``MOVW R0, \#0x13D8''} ~--- \IFRU{она записывает 16-битное число в младшую часть регистра \Rzero}{it writes 16-bit value into lower part of \Rzero register}.

\IFRU{Еще}{Also} \TT{``MOVT.W R0, \#0''} ~--- \IFRU{эта инструкция работает так же как и}{this instruction works just like} 
\TT{MOVT} \IFRU{из предыдущего примера, но она работает в}{from previous example, but it works in} thumb-2.

\index{ARM!\IFRU{переключение режимов}{mode switching}}
\index{ARM!\Instructions!BLX}
\IFRU{Помимо прочих отличий, здесь используется инструкция}{Among other differences, here is} \TT{BLX} \IFRU{вместо}{instruction used instead of} \TT{BL}.
\IFRU{Отличие в том, что помимо сохранения адреса возврата в регистре \LR и передаче управления в функцию \puts, происходит смена режима процессора с thumb на ARM, либо наоборот}{Difference in that way that beside saving of return address in \LR register and passing control to \puts function, processor is switching from thumb mode to ARM or back}.
\IFRU{Здесь это нужно потому что инструкция, куда ведет переход, выглядит так (она закодирована в режиме ARM)}{This instruction in place here because the instruction to which control is passed looks like (it's encoded in ARM mode)}:

\begin{lstlisting}
__symbolstub1:00003FEC _puts           ; CODE XREF: _hello_world+E
__symbolstub1:00003FEC 44 F0 9F E5     LDR  PC, =__imp__puts
\end{lstlisting}

\IFRU{Итак, внимательный читатель может задать справделивый вопрос: почему бы не вызывать \puts сразу в 
том же месте кода, где он нужен?}
{So, observant reader may ask: why not to call \puts right at the place of code where it needed?}

\IFRU{Но это не очень выгодно (в плане экономия места) и вот почему}{But that's not very space-efficient, and that's why}.

\index{\IFRU{Динамически подгружаемые библиотеки}{Dynamically loaded libraries}}
\IFRU{Практически любая программа использует внешние динамические библиотеки, будь то DLL в Windows, .so в *NIX 
либо .dylib в Mac OS X}{Almost any program uses external dynamic libraries, like DLL in Windows, .so in *NIX or .dylib in Mac OS X}. 
\IFRU{В динамических библиотеках находятся часто используемые библиотечные функции, в том числе стандартная функция Си \puts}
{Often used library functions are stored in dynamic libraries, including standard C-function \puts}.

\index{Relocation}
\IFRU{В исполняемом бинарном файле}{In executable binary file} (Windows PE .exe, ELF \IFRU{либо}{or} Mach-O) \IFRU{имеется секция импортов, список символов (функций либо глобальных переменных) импортируемых из внешних модулей, а также названия самих модулей}{a section of imports is present, that is list of symbols (functions or global variables) being imported from external modules and also names of these modules}.

\IFRU{Загрузчик операционной системы загружает необходимые модули и, перебирая импортируемые символы в основном модуле, проставляет правильные адреса каждого символа}{Operation system loader loads all modules need and, while enumerating importing symbols in primary module, sets correct addresses of each symbol}.

\IFRU{В нашем случае}{In our case}, \IT{\_\_imp\_\_puts} \IFRU{это 32-битная переменная, куда загрузчик ОС запишет правильный адрес этой же функции во внешней библиотеке}{is 32-bit variable where OS loader will write correct address of that function in external library}. 
\IFRU{Так что инструкция \TT{LDR} просто берет 32-битное значение из этой переменной и, записывая его в регистр \PC, просто передает туда управление}{So that \TT{LDR} instruction just takes 32-bit value from this variable and, writing it into \PC register, just passing control to it}.

\IFRU{Чтобы уменьшить время работы загрузчика ОС, нужно чтобы ему пришлось записать адрес каждого символа только один раз, в соответствующее для них место}{So to readuce a time OS loader needs for doing this procedure, it's good idea for it to write address of each symbol only once, to special place for it}.

\index{thunk-\IFRU{функции}{functions}}
\IFRU{К тому же, как мы уже убедились, нельзя одной инструкцией загрузить в регистр 32-битное число без обращений к памяти}{Besides, as we already figured out, it's not possible to load 32-bit value into register 
using only one instruction, without memory access}.
\IFRU{Так что, наиболее оптимально, выделить отдельную функцию, работающую в режиме ARM, 
чья единственная цель ~--- передавать управление дальше, в динамическую библиотеку}
{So, it is optimal to allocate separate function working in ARM mode with only one goal ~--- 
to pass control to dynamic library}. 
\IFRU{И затем ссылаться на эту короткую функцию из одной инструкции (так называемую thunk-функцию) из thumb-кода}{And then to jump to this short one-instruction function (so called thunk-function) from thumb-code}.

\index{ARM!\Instructions!BL}
\IFRU{Кстати, в предыдущем примере (скомпилированном для режима ARM), переход при помощи инструкции \TT{BL} ведет 
на такую же thunk-функцию, однако режим процессора не переключается (отсюда, отсутствие ``X'' в мнемонике инструкции)}{By the way, in previous example (compiled for ARM mode) control passing by \TT{BL} instruction is going to the same thunk-function, however, processor mode is not switched (hence, absence of ``X'' in instruction mnemonic)}.



\section{\Stack}
\label{sec:stack}
\index{\Stack}

\IFRU{Стек в компьютерных науках ~--- это одна из наиболее фундаментальных вещей}
{Stack ~--- is one of the most fundamental things in computer science.}\footnote{\url{http://en.wikipedia.org/wiki/Call_stack}}.

\IFRU{Технически, это просто блок памяти в памяти процесса + регистр \ESP или \RSP в x86, либо \SP в ARM, который указывает где-то в пределах этого блока.}
{Technically, it's just a memory block in process memory + \ESP or \RSP register in x86, or \SP register in ARM, as a pointer within this block.}

\index{ARM!\Instructions!PUSH}
\index{ARM!\Instructions!POP}
\index{x86!\Instructions!PUSH}
\index{x86!\Instructions!POP}
\IFRU{Часто используемые инструкции для работы со стеком это \PUSH и \POP (в x86 и thumb-режиме ARM). 
\PUSH уменьшает \ESP/\RSP/\SP на $4$, затем записывает по адресу на который указывает \ESP/\RSP/\SP содержимое своего единственного операнда.}
{Most frequently used stack access instructions are \PUSH and \POP (both in x86 and ARM thumb-mode). 
\PUSH subtracting \ESP/\RSP/\SP by $4$ and then writing contents of its sole operand to the memory address pointing by \ESP/\RSP/\SP.} 

\IFRU{\POP это обратная операция ~--- сначала достает из \ESP/\RSP/\SP значение и помещает его в операнд 
(который очень часто является регистром) и затем увеличивает \ESP/\RSP/\SP на $4$. 
Конечно, это для 32-битной среды. В x64-среде это будет $8$ а не $4$.}
{\POP is reverse operation: get a data from memory pointing by \ESP/\RSP/\SP, put it to operand
(often register) and then add $4$ to \ESP/\RSP/\SP. 
Of course, this is for 32-bit environment. $8$ will be here instead of $4$ in x64 environment.}

\IFRU{В самом начале, регистр-указатель указывает на конец стека.}{After stack allocation, stack pointer pointing to the end of stack.}
\IFRU{\PUSH уменьшает регистр-указатель, а \POP ~--- увеличивает.}{\PUSH increasing stack pointer, and \POP decreasing.}
\IFRU{Конец стека находится в начале блока памяти выделенного под стек. Это странно, но это так.}
{The end of stack is actually at the beginning of allocated for stack memory block. 
It seems strange, but it is so.}

\IFRU{В процессоре ARM, тем не менее, есть поддержка стеков растущих как в сторону уменьшения, так и в
сторону увеличения}{Nevertheless, ARM has instructions supporting ascending stacks, but also descending stacks}. 
\index{ARM!\Instructions!STMFD}
\index{ARM!\Instructions!LDMFD}
\index{ARM!\Instructions!STMED}
\index{ARM!\Instructions!LDMED}
\index{ARM!\Instructions!STMFA}
\index{ARM!\Instructions!LDMFA}
\index{ARM!\Instructions!STMEA}
\index{ARM!\Instructions!LDMEA}
\IFRU{Например, инструкции}{For example,} 
STMFD\footnote{\STMFDdesc}/LDMFD\footnote{\LDMFDDESC}, 
STMED\footnote{\STMEDdesc}/LDMED\footnote{\LDMEDdesc} 
\IFRU{предназначены для descending-стека, т.е., уменьшающегося}{instructions are intended for work with 
descending stack}.
\IFRU{Инструкции}{}
STMFA\footnote{\STMFAdesc}/LMDFA\footnote{\LDMFAdesc}, 
STMEA\footnote{\STMEAdesc}/LDMEA\footnote{\LDMEAdesc} 
\IFRU{предназначены для ascending-стека, т.е., увеличивающегося}{instructions are intended for work with 
ascending stack}.

\IFRU{Для чего используется стек?}{What stack is used for?}

\subsubsection{\IFRU{Сохранение адреса куда должно вернуться управление после вызова функции}
{Save the return address where a function should return control after execution}}

\paragraph{x86}

\index{x86!\Instructions!CALL}
\IFRU{При вызове другой функции через \CALL, сначала в стек записывается адрес указывающий на место аккурат после 
инструкции \CALL, затем делается безусловный переход (почти как \TT{JMP}) на адрес указанный в операнде.} 
{While calling another function with a \CALL instruction the address of the point exactly after the \CALL instruction is saved 
to the stack and then an unconditional jump to the address in the CALL operand is executed.} 

\index{x86!\Instructions!PUSH}
\index{x86!\Instructions!JMP}
\IFRU{\CALL это аналог пары инструкций \TT{PUSH address\_after\_call / JMP}.}
{The \CALL instruction is equivalent to a \TT{PUSH address\_after\_call / JMP operand} instruction pair}.

\index{x86!\Instructions!RET}
\index{x86!\Instructions!POP}
\IFRU{\RET вытаскивает из стека значение и передает управление по этому адресу ~--- 
это аналог пары инструкций \TT{POP tmp / JMP tmp}.}
{\RET fetches a value from the stack and jumps to it ~--- it is equivalent to a \TT{POP tmp / JMP tmp} instruction pair.}

\index{\Stack!\IFRU{Переполнение стека}{Stack overflow}}
\index{\Recursion}
\IFRU{Крайне легко устроить переполнение стека запустив бесконечную рекурсию:}
{Overflow the stack is simple. Just run eternal recursion:}

\begin{lstlisting}
void f()
{
	f();
};
\end{lstlisting}

\IFRU{MSVC 2008 предупреждает о проблеме:}{MSVC 2008 reports the problem:}

\begin{lstlisting}
c:\tmp6>cl ss.cpp /Fass.asm
Microsoft (R) 32-bit C/C++ Optimizing Compiler Version 15.00.21022.08 for 80x86
Copyright (C) Microsoft Corporation.  All rights reserved.

ss.cpp
c:\tmp6\ss.cpp(4) : warning C4717: 'f' : recursive on all control paths, function will cause runtime stack overflow
\end{lstlisting}

\dots \IFRU{но тем не менее создает нужный код}{but generates the right code anyway}:

\begin{lstlisting}
?f@@YAXXZ PROC						; f
; File c:\tmp6\ss.cpp
; Line 2
	push	ebp
	mov	ebp, esp
; Line 3
	call	?f@@YAXXZ				; f
; Line 4
	pop	ebp
	ret	0
?f@@YAXXZ ENDP						; f
\end{lstlisting}

\dots \IFRU
{причем, если включить оптимизацию (\Ox), то будет даже интереснее, без переполнения стека, 
но работать будет \IT{корректно}\footnote{здесь ирония}:}
{Also if we turn on optimization (\Ox option) the optimized code will not overflow the stack 
but will work \IT{correctly}\footnote{irony here}:}

\begin{lstlisting}
?f@@YAXXZ PROC						; f
; File c:\tmp6\ss.cpp
; Line 2
$LL3@f:
; Line 3
	jmp	SHORT $LL3@f
?f@@YAXXZ ENDP						; f
\end{lstlisting}

\IFRU{GCC 4.4.1 генерирует точно такой же код в обоих случаях, хотя и не предупреждает о проблеме.}
{GCC 4.4.1 generating the same code in both cases, although not warning about problem.}

\paragraph{ARM}

\index{ARM!\Registers!Link Register}
\IFRU{Программы для ARM также используют стек для сохранения \ac{RA}, куда нужно вернуться, но несколько иначе}{ARM
programs also use the stack for saving return addresses, but differently}.
\IFRU{Как уже упоминалось в секции}{As it was mentioned in} ``\HelloWorldSectionName''~\ref{sec:hw_ARM}, 
\IFRU{\ac{RA} записывается в регистр}{the \ac{RA} is saved to the} \LR (\IT{link register}).
\IFRU{Но если есть необходимость вызывать какую-то другую функцию, и использовать регистр \LR еще
раз, его значение желательно сохранить}
{However, if one needs to call another function and use the \LR register
one more time its value should be saved}.
\index{Function prologue}
\IFRU{Обычно, это происходит в прологе функции, часто мы видим там инструкцию вроде}
{Usually it is saved in the function prologue. Often, we see instructions like}
\index{ARM!\Instructions!PUSH}
\index{ARM!\Instructions!POP}
\TT{``PUSH {R4-R7,LR}''} \IFRU{, а в эпилоге}{along with this instruction in epilogue} \TT{``POP {R4-R7,PC}''} ~--- 
\IFRU{так сохраняются регистры, которые будут использоваться в текущей функции, в том числе}
{thus register values
to be used in the function are saved in the stack, including} \LR.

\index{ARM!Leaf function}
\IFRU{Тем не менее, если некая функция не вызывает никаких более функций, в терминологии ARM она называется}
{Nevertheless, if a function never calls any other function, in ARM terminology it is called}
\IT{leaf function}\footnote{\url{http://infocenter.arm.com/help/index.jsp?topic=/com.arm.doc.faqs/ka13785.html}}. 
\IFRU{Как следствие, ``leaf''-функция не использует регистр \LR}
{As a consequence ``leaf'' functions do not use the \LR register}.
\IFRU{А если эта функция небольшая, использует мало регистров, она может не использовать стек вообще}
{And if this function is small and it uses a small number of registers it may not use stack at all}.
\IFRU{Таким образом, в ARM возможен вызов небольших ``leaf'' функций не используя стек}
{Thus, it is possible to call ``leaf'' functions without using stack}.
\IFRU{Это может быть быстрее чем в x86, ведь внешняя память для стека не используется}
{This can be faster than on x86 because external RAM is not used for the stack}
\footnote{\IFRU{Когда-то очень давно, на PDP-11 и VAX, на инструкцию CALL (вызов других функций) могло тратиться
вплоть до 50\% времени, возможно из-за работы с памятью, 
поэтому считалось что много небольших функций это анти-паттерн}
{Some time ago, on PDP-11 and VAX, CALL instruction (calling other functions) was expensive, up to 50\%
of execution time might be spent on it, so it was common sense that big number of small function is anti-pattern}\cite[Chapter 4, Part II]{Raymond:2003:AUP:829549}.}.
\IFRU{Либо, это может быть полезным для тех ситуаций, когда память для стека еще не выделена либо недоступна}
{It can be useful for such situations when memory for the stack is not yet allocated or not available}.


\subsection{\IFRU{Передача параметров для функции}{Function arguments passing}}

\begin{lstlisting}
push arg3
push arg2
push arg1
call f
add esp, 4*3
\end{lstlisting}

\IFRU{Вызываемая функция получает свои параметры также через указатель стека.}
{Callee{\footnote{Function being called}} function get its arguments via stack ponter.}

\IFRU{См.также в соответствующем разделе о способах передачи аргументов через стек}
{See also section about calling conventions}~\ref{sec:callingconventions}.

\IFRU{Важно отметить, что, в общем, никто не заставляет программистов передавать параметры именно через стек,
это не является требованием к исполняемому коду.}
{It is important to note that no one oblige programmers to pass arguments through stack, it is not prerequisite.}

\IFRU{Вы можете делать это совершенно иначе, не используя стек.}
{One could implement any other method not using stack.}

\IFRU{К примеру, можно выделять в куче\footnote{heap в англоязычной литературе} место для аргументов, 
заполнять их и передавать в функцию указатель на это место через \EAX. И это вполне будет работать}
{For example, it is possible to allocate a place for arguments in heap, fill it and pass to a function 
via pointer to this pack in \EAX register. And this will work}
\footnote{\IFRU{Например, в книге Дональда Кнута ``Искусство программирования'', в разделе 1.4.1 
посвященном подпрограммам\cite[раздел 1.4.1]{Knuth:1998:ACP:521463}, 
мы можем прочитать о возможности располагать параметры для вызываемой подпрограммы после инструкции \JMP
передающей управление подпрограмме. Кнут описывает что это было особенно удобно для компьютеров System/360.}
{For example, in ``The Art of Computer Programming'' book by Donald Knuth, 
in section 1.4.1 dedicated to subroutines\cite[section 1.4.1]{Knuth:1998:ACP:521463},
we can read about one way to supply arguments to subroutine is simply to list them after the \JMP instruction
passing control to subroutine. Knuth writes that this method was particularly convenient on System/360.}}.

\IFRU{Однако, так традиционно сложилось, что в x86 и ARM передача аргументов происходит именно через стек.}
{However, it is convenient tradition in x86 and ARM to use stack for this.}


\subsubsection{\IFRU{Хранение локальных переменных}{Local variable storage}}

\IFRU{Функция может выделить для себя некоторое место в стеке для локальных переменных просто отодвинув 
\glslink{stack pointer}{указатель стека} глубже к концу стека.}
{A function could allocate a space in the stack for its local variables just by shifting 
the \gls{stack pointer} towards stack bottom.}

\IFRU{Это снова не является необходимым требованием. Вы можете хранить локальные переменные где угодно. 
Но по традиции всё сложилось так.}
{It is also not a requirement. You could store local variables wherever you like. 
But traditionally it is so.}


\subsection{x86: \IFRU{Функция alloca()}{alloca() function}}
\label{alloca}
\index{\CStandardLibrary!alloca()}
\IFRU{Интересен случай с функцией \TT{alloca()}}
{It is worth noting \TT{alloca()} function.}\footnote{
\IFRU
{В MSVC, реализацию функции можно посмотреть в файлах}
{As of MSVC, function implementation can be found in} 
  \TT{alloca16.asm} 
  \IFRU{и}{and} 
  \TT{chkstk.asm} 
  \IFRU{в}{in} 
  \TT{C:\textbackslash{}Program Files (x86)\textbackslash{}Microsoft Visual Studio 10.0\textbackslash{}VC\textbackslash{}crt\textbackslash{}src\textbackslash{}intel}}. 

\IFRU{Эта функция работает как \TT{malloc()}, но выделяет память прямо в стеке.} 
{This function works like \TT{malloc()} but allocates memory just in stack.}

\IFRU{Память освобождать через \TT{free()} не нужно, так как эпилог функции~\ref{sec:prologepilog} 
вернет \ESP назад в изначальное состояние и выделенная память просто аyнулируется.}
{Allocated memory chunk is not needed to be freed via \TT{free()} function call since 
function epilogue~\ref{sec:prologepilog} shall return value of the \ESP back to initial state and 
allocated memory will be just annuled.} 

\IFRU{Интересна реализация функции \TT{alloca()}.}
{It is worth noting how \TT{alloca()} implemented.}

\IFRU{Эта функция, если упрощенно, просто сдвигает \ESP вглубь стека 
на столько байт сколько вам нужно и возвращает \ESP в качестве указателя на выделенный блок.}
{This function, if to simplify, just shifting \ESP deeply to stack bottom so much bytes you 
need and set \ESP as a pointer to that \IT{allocated} block.}
\IFRU{Попробуем:}{Let's try:}

\lstinputlisting{02_stack/2_1.c}

\IFRU{(Функция \TT{\_snprintf()} работает так же как и \printf, только вместо выдачи результата в 
stdout (т.е., на терминал или в консоль),
записывает его в буфер \TT{buf}. \puts выдает содержимое буфера \TT{buf} в stdout. Конечно, можно было бы
заменить оба этих вызова на один \printf, но мне нужно проиллюстрировать использование небольшого буфера.)}
{(\TT{\_snprintf()} function works just like \printf, but instead dumping result into stdout (e.g., to terminal or 
console), write it to the \TT{buf} buffer. \puts copies \TT{buf} contents to stdout. Of course, these two
function calls might be replaced by one \printf call, but I would like to illustrate small buffer usage.)}

\subsubsection{MSVC}

\IFRU{Компилируем}{Let's compile} (MSVC 2010):

\lstinputlisting[caption=MSVC 2010]{02_stack/2_2_msvc.asm}

\index{Compiler intrinsic}
\IFRU {Единственный параметр в \TT{alloca()} передается через \EAX, а не как обычно через стек}
{The sole \TT{alloca()} argument passed via \EAX (instead of pushing into stack)}
\footnote{\IFRU{Это потому что alloca() это не сколько функция, сколько т.е. compiler intrinsic}{It's because
alloca() is rather compiler intrinsic than usual function}}.
\IFRU{После вызова \TT{alloca()}, \ESP теперь указывает на блок в 600 байт который 
мы можем использовать под \TT{buf}.}
{After \TT{alloca()} call, \ESP is now pointing to the block of 600 bytes and we can 
use it as memory for \TT{buf} array.}

\subsubsection{GCC + \IntelSyntax}

\IFRU{А GCC 4.4.1 обходится без вызова других функций:}
{GCC 4.4.1 can do the same without calling external functions:}

\lstinputlisting[caption=GCC 4.7.3]{\IFRU{02_stack/2_1_gcc_intel_O3_ru.asm}{02_stack/2_1_gcc_intel_O3_en.asm}}

\subsubsection{GCC + \ATTSyntax}

\IFRU{Посмотрим на тот же код, только в синтаксисе AT\&T}{Let's see the same code, but in AT\&T syntax}:

\lstinputlisting[caption=GCC 4.7.3]{02_stack/2_1_gcc_ATT_O3.s}

\index{\ATTSyntax}
\IFRU{Всё то же самое что и в прошлом листинге.}{The same code as in previos listing.}

\IFRU{Обратите внимание что, например}{Please note that, for example}, \TT{movl \$3, 20(\%esp)} 
\IFRU{это аналог}{is analogous to} \TT{mov DWORD PTR [esp+20], 3} \IFRU{в Intel-синтаксисе}{in Intel-syntax} ~--- 
\IFRU{при адресации памяти в виде}{when addressing memory in form} \IT{\IFRU{регистр+смещение}{register+offset}}, 
\IFRU{это записывается в AT\&T синтаксисе как}{it's written in AT\&T syntax as} 
\TT{\IFRU{смещение}{offset}(\%\IFRU{регистр}{register})}.



\subsection{(Windows) SEH}
\index{Windows!Structured Exception Handling}

\IFRU{В стеке хранятся записи SEH (\IT{Structured Exception Handling}) для функции (если имеются)}
{SEH (\IT{Structured Exception Handling}) records are also stored in stack (if needed).}
\footnote{
\IFRU{О SEH: классическая статья Мэтта Питрека}{Classic Matt Pietrek article about SEH}: 
\url{http://www.microsoft.com/msj/0197/Exception/Exception.aspx}}.


\subsection{\RU{Защита от переполнений буфера}\EN{Buffer overflow protection}\PTBR{Proteção contra estouro de buffer}}

\RU{Здесь больше об этом}\EN{More about it here}\PTBR{Mais sobre aqui}~(\myref{subsec:bufferoverflow}).



\input{03_printf/printf}
\section{scanf()}

\IFRU{Теперь попробуем использовать scanf().}{Now let's use scanf().}

\begin{lstlisting}
int main() 
{
	int x;
	printf ("Enter X:\n");

	scanf ("%d", &x);

	printf ("You entered %d...\n", x);

	return 0;
};
\end{lstlisting}

\IFRU
{Да, согласен, использовать \scanf в наши времена для того чтобы спросить у юзера что-то: не самая хорошая идея.
Но я хотел проиллюстрировать передачу указателя на \Tint.}
{OK, I agree, it is not clever to use \scanf today. But I wanted to illustrate passing pointer to \Tint.}

\subsection{x86}

\IFRU{Что получаем на ассемблере компилируя MSVC 2010:}
{What we got after compiling in MSVC 2010:}

\lstinputlisting{04_scanf/4_1_msvc.asm}

\IFRU{Переменная \TT{x} является локальной.}{Variable \TT{x} is local.} 

\IFRU{По стандарту \CCpp она доступна только из этой же функции и ниоткуда более. 
Так получилось, что локальные переменные располагаются в стеке. 
Может быть, можно было бы использовать и другие варианты, но в x86 это традиционно так.}
{\CCpp standard tell us it must be visible only in this function and not from any other point. 
Traditionally, local variables are placed in the stack. 
Probably, there could be other ways, but in x86 it is so.}

\index{x86!\Instructions!PUSH}
\IFRU{Следующая после пролога инструкция \TT{PUSH ECX} не ставит своей целью сохранить 
значение регистра \ECX. 
(Заметьте отсутствие сооветствующей инструкции \TT{POP ECX} в конце функции)}
{Next instruction after function prologue, \TT{PUSH ECX}, hasn't goal to save \ECX state 
(notice absence of corresponding \TT{POP ECX} at the function end).}

\IFRU{Она на самом деле выделяет в стеке 4 байта для хранения \TT{x} в будущем.} 
{In fact, this instruction just allocates 4 bytes on the stack for \TT{x} variable storage.} 

\index{\Stack!\IFRU{Стековый фрейм}{Stack frame}}
\index{x86!\Registers!EBP}
\IFRU{Доступ к \TT{x} будет осуществляться при помощи объявленного макроса \TT{\_x\$} 
(он равен -4) и регистра \EBP указывающего на текущий фрейм.}
{\TT{x} will be accessed with the assistance of the \TT{\_x\$} macro 
(it equals to -4) and the \EBP register pointing to current frame.}

\IFRU{Вообще, во все время исполнения функции, \EBP указывает на текущий фрейм и через \TT{EBP+смещение}
можно иметь доступ как к локальным переменным функции, так и аргументам функции.} 
{Over a span of function execution, \EBP is pointing to current stack frame and it is possible 
to have an access to local variables and function arguments via \TT{EBP+offset}.}

\index{x86!\Registers!ESP}
\IFRU
{Можно было бы использовать \ESP, но он во время исполнения функции постоянно меняется. 
Так что можно сказать что \EBP это \IT{замороженное состояние} \ESP на момент начала исполнения функции.}
{It is also possible to use \ESP, but it's often changing and not very convenient.
So it can be said, the value of the \EBP is \IT{frozen state} of the value of the \ESP at the moment of function execution start.}

\IFRU
{У функции \scanf в нашем примере два аргумента.}{Function \scanf in our example has two arguments.}

\IFRU
{Первый ~--- указатель на строку содержащую \TT{``\%d''} и второй ~--- адрес переменной \TT{x}.} 
{First is pointer to the string containing \TT{``\%d''} and second ~--- address of variable \TT{x}.} 

\index{x86!\Instructions!LEA}
\IFRU{Вначале адрес \TT{x} помещается в регистр \EAX при помощи инструкции \TT{lea eax, DWORD PTR \_x\$[ebp]}.}
{First of all, address of the \TT{x} variable is placed into the \EAX register by \TT{lea eax, DWORD PTR \_x\$[ebp]} instruction}

\IFRU{Инструкция \LEA означает \IT{load effective address}, но со временем она изменила свою функцию}
{\LEA meaning \IT{load effective address} but over a time it changed its primary application}
~\ref{sec:LEA}.

\IFRU{Можно сказать что в данном случае \LEA просто помещает в \EAX результат суммы значения в регистре 
\EBP и макроса \TT{\_x\$}.}
{It can be said, \LEA here just stores sum of the value in the \EBP register and \TT{\_x\$} macro to the \EAX register.}

\IFRU{Это тоже что и}{It is the same as} \TT{lea eax, [ebp-4]}.

\IFRU{Итак, от значения \EBP отнимается $4$ и помещается в \EAX.
Далее значение \EAX заталкивается в стек и вызывается \scanf.}
{So, $4$ subtracting from value in the \EBP register and result is placed to the \EAX register.
And then value in the \EAX register is pushing into stack and \scanf is called.}

\IFRU{После этого вызывается \printf. Первый аргумент вызова которого, строка:} 
{After that, \printf is called. First argument is pointer to string:} \TT{``You entered \%d...\textbackslash{}n''}.

\IFRU{Второй аргумент: \TT{mov ecx, [ebp-4]}, эта инструкция помещает в \ECX не адрес переменной \TT{x}, 
а его значение, что там сейчас находится.}
{Second argument is prepared as: \TT{mov ecx, [ebp-4]},
this instruction places to the \ECX not address of the \TT{x} variable, but its contents.}

\IFRU{Далее значение \ECX заталкивается в стек и вызывается последний \printf.}
{After, value in the \ECX is placed on the stack and the last \printf called.}

\IFRU{Попробуем тоже самое скомпилировать в Linux при помощи GCC 4.4.1:}
{Let's try to compile this code in GCC 4.4.1 under Linux:}

\lstinputlisting{04_scanf/4_1_gcc.asm}

\index{puts() \IFRU{вместо}{instead of} printf()}
\IFRU{GCC заменил первый вызов \printf на \puts, почему это было сделано, 
уже было описано раннее~\ref{puts}.}
{GCC replaced first the \printf call to the \puts, it was already described~\ref{puts} 
why it was done.}

% TODO: rewrite
%\IFRU
%{Почему \scanf переименовали в \TT{\_\_\_isoc99\_scanf}, я честно говоря, пока не знаю.}
%{Why \scanf is renamed to \TT{\_\_\_isoc99\_scanf}, I do not know yet.}

\IFRU{Далее все как и прежде ~--- параметры заталкиваются через стек при помощи \MOV.}
{As before ~--- arguments are placed on the stack by \MOV instruction.}


\subsection{ARM}

\subsubsection{\OptimizingKeil + \ThumbMode}

\begin{lstlisting}
.text:00000042             scanf_main
.text:00000042
.text:00000042             var_8           = -8
.text:00000042
.text:00000042 08 B5                       PUSH    {R3,LR}
.text:00000044 A9 A0                       ADR     R0, aEnterX     ; "Enter X:\n"
.text:00000046 06 F0 D3 F8                 BL      __2printf
.text:0000004A 69 46                       MOV     R1, SP
.text:0000004C AA A0                       ADR     R0, aD          ; "%d"
.text:0000004E 06 F0 CD F8                 BL      __0scanf
.text:00000052 00 99                       LDR     R1, [SP,#8+var_8]
.text:00000054 A9 A0                       ADR     R0, aYouEnteredD___ ; "You entered %d...\n"
.text:00000056 06 F0 CB F8                 BL      __2printf
.text:0000005A 00 20                       MOVS    R0, #0
.text:0000005C 08 BD                       POP     {R3,PC}
\end{lstlisting}

Чтобы \scanf мог вернуть значение, нужно передать ему указатель на переменную типа \Tint. \Tint ~--- 32-битное 
значение, для его хранения нужно только 4 байта и оно помещается в регистр.
Локальная переменная \TT{x} выделяется в стеке, \IDA наименовала её \IT{var\_8}, место для нее выделять
не обязательно, т.к., указатель стека \SP уже указывает на место, свободное для использования.
Так что указатель \SP копируется в регистр \TT{R1} и вместе с format-строкой, передается в \scanf.
Позже, при помощи инструкции \TT{LDR}, это значение перемещается из стека в регистр R1, чтобы быть переданным
в \printf.

Варианты скомпилированные для ARM-режима процессора, а также варианты скомпилированные при помощи Xcode,
не очень отличаются от этого, так что, мы можем пропустить их здесь.



\subsection{\IFRU{Глобальные переменные}{Global variables}}
\index{\IFRU{Глобальные переменные}{Global variables}}
\subsubsection{x86}

\IFRU
{А что если переменная \TT{x} из предыдущего примера будет глобальной переменной а не локальной? 
Тогда к ней смогут обращаться из любого другого места, а не только из тела функции. 
Это снова не очень хорошая практика программирования, но ради примера мы можем себе это позволить.}
{What if \TT{x} variable from previous example will not be local but global variable? 
Then it will be accessible from any point, not only from function body. 
It is not very good programming practice, but for the sake of experiment we could do this.}

\lstinputlisting{04_scanf/4_2_msvc.asm}

\IFRU
{Ничего особенного, в целом. Теперь \TT{x} объявлена в сегменте \TT{\_DATA}. 
Память для нее в стеке более не выделяется. Все обращения к ней происходит не через стек, а уже напрямую. 
Её значение неопределено. 
Это означает, что память под нее будет выделена, но ни компилятор, ни \ac{ОС} не будет заботиться о том, 
что там будет лежать на момент старта функции \main.
В качестве домашнего задания, попробуйте объявить большой неопределенный массив и посмотреть 
что там будет лежать после загрузки.}
{Now \TT{x} variable is defined in the \TT{\_DATA} segment. 
Memory in local stack is not allocated anymore. 
All accesses to it are not via stack but directly to process memory. 
Its value is not defined. 
This means that memory will be allocated by \ac{OS}, but not compiler, 
neither \ac{OS} will not take care about its initial value at the moment of 
the \main function start.
As experiment, try to declare large array and see what will it contain after 
program loading.}

\IFRU{Попробуем изменить объявление этой переменной:}
{Now let's assign value to variable explicitly:}

\begin{lstlisting}
int x=10; // default value
\end{lstlisting}

\IFRU{Выйдет в итоге:}{We got:}

\begin{lstlisting}
_DATA	SEGMENT
_x	DD	0aH

...
\end{lstlisting}

\IFRU{Здесь уже по месту этой переменной записано \TT{0xA} с типом DD (dword = 32 бита).}
{Here we see value \TT{0xA} of DWORD type (DD meaning DWORD = 32 bit).}

\IFRU{Если вы откроете скомпилированный .exe-файл в \IDA, то увидите что \IT{x} 
находится аккурат в начале сегмента \TT{\_DATA}, после этой переменной будут текстовые строки.}
{If you will open compiled .exe in \IDA, you will see the \IT{x} variable placed at the beginning of 
the \TT{\_DATA} segment, and after you'll see text strings.}

\IFRU{А вот если вы откроете в \IDA, .exe скомплированный в прошлом примере, 
где значение \IT{x} неопределено, то в IDA вы увидите:}
{If you will open compiled .exe in \IDA from previous example where \IT{x} value is not defined, 
you'll see something like this:}

\begin{lstlisting}
.data:0040FA80 _x              dd ?                    ; DATA XREF: _main+10
.data:0040FA80                                         ; _main+22
.data:0040FA84 dword_40FA84    dd ?                    ; DATA XREF: _memset+1E
.data:0040FA84                                         ; unknown_libname_1+28
.data:0040FA88 dword_40FA88    dd ?                    ; DATA XREF: ___sbh_find_block+5
.data:0040FA88                                         ; ___sbh_free_block+2BC
.data:0040FA8C ; LPVOID lpMem
.data:0040FA8C lpMem           dd ?                    ; DATA XREF: ___sbh_find_block+B
.data:0040FA8C                                         ; ___sbh_free_block+2CA
.data:0040FA90 dword_40FA90    dd ?                    ; DATA XREF: _V6_HeapAlloc+13
.data:0040FA90                                         ; __calloc_impl+72
.data:0040FA94 dword_40FA94    dd ?                    ; DATA XREF: ___sbh_free_block+2FE
\end{lstlisting}

\IFRU{\TT{\_x} обозначен как \TT{?}, наряду с другими переменными не требующими инициализции. 
Это означает, что при загрузке .exe в память, место под все это выделено будет. 
Но в самом .exe ничего этого нет. Неинициализированные переменные не занимают места в исполняемых файлах. Удобно для больших массивов, например.}
{\TT{\_x} marked as \TT{?} among another variables not required to be initialized. 
This means that after loading .exe to memory, a space for all these variables will be 
allocated and some random garbage will be here. 
But in an .exe file these not initialized variables are not occupy anything. 
It is suitable for large arrays, for example.}

\index{ELF}
\IFRU{В Linux все также почти. За исключением того что если значение \TT{x} не определено, 
то эта переменная будет находится в сегменте \TT{\_bss}. В ELF\footnote{Формат исполняемых файлов, использующийся в Linux и некоторых других *NIX} этот сегмент имеет такие аттрибуты:}
{It is almost the same in Linux, except segment names and properties: 
not initialized variables are located in the \TT{\_bss} segment. 
In ELF\footnote{Executable file format widely used in *NIX system including Linux} 
file format this segment has such attributes:}

\begin{lstlisting}
; Segment type: Uninitialized
; Segment permissions: Read/Write
\end{lstlisting}

\IFRU{Ну а если сделать присвоение этой переменной значения $10$, то она будет находится 
в сегменте \TT{\_data},
это сегмент с такими аттрибутами:}
{If to assign some value to variable, e.g. $10$, it will be placed in the \TT{\_data} segment, 
this is segment with such attributes:}

\begin{lstlisting}
; Segment type: Pure data
; Segment permissions: Read/Write
\end{lstlisting}

\input{04_scanf/global_vars_ARM}



\subsection{\IFRU{Проверка результата scanf()}{scanf() result checking}}

\subsubsection{x86}

\IFRU {Как я уже упоминал, использовать \scanf в наше время это слегка старомодно. 
Но если уж жизнь заставила этим заниматься, нужно хотя бы проверять, сработал ли \scanf 
правильно или пользователь ввел вместо числа что-то другое, что \scanf не смог трактовать как число.}
{As I noticed before, it is slightly old-fashioned to use \scanf today. 
But if we have to, we need at least check if \scanf finished correctly without error.}

\lstinputlisting{04_scanf/retval_check.c}

\IFRU{По стандарту}{By standard}, \scanf\footnote{\href{http://msdn.microsoft.com/en-us/library/9y6s16x1(VS.71).aspx}{MSDN: scanf, wscanf}} 
\IFRU{возвращает количество успешно полученных значений.}{function returns number of fields it successfully read.}

\IFRU{В нашем случае, если все успешно и пользователь ввел таки некое число, \scanf вернет 1. 
А если нет, то 0 или EOF.} 
{In our case, if everything went fine and user entered a number, 
\scanf will return 1 or 0 or EOF in case of error.}

\IFRU{Я добавил код проверяющий результат \scanf и в случае ошибки, он сообщает пользователю что-то другое.}
{I added C code for \scanf result checking and printing error message in case of error.}

\IFRU{Вот, что выходит на ассемблере}{What we got in assembly language} (MSVC 2010):

\lstinputlisting{04_scanf/retval_check_MSVC.asm}

\index{x86!\Registers!EAX}
\IFRU{Для того чтобы вызывающая функция имела доступ к результату вызываемой функции, 
вызываемая функция (в нашем случае \scanf) оставляет это значение в регистре \EAX.}
{Caller function (\main) must have access to the result of callee function (\scanf), 
so callee leaves this value in the \EAX register.}

\index{x86!\Instructions!CMP}
\IFRU{Мы проверяем его инструкцией \TT{CMP EAX, 1} (\IT{CoMPare}), то есть, 
сравниваем значение в \EAX с 1.}
{After, we check it with the help of instruction \TT{CMP EAX, 1} (\IT{CoMPare}),
in other words, we compare value in the \EAX register with $1$.} 

\index{x86!\Instructions!JNE}
\IFRU{Следующий за инструкцией \CMP: условный переход \JNE. 
Это означает \IT{Jump if Not Equal}, то есть, условный переход \IT{если не равно}.}
{\JNE conditional jump follows \CMP instruction. \JNE means \IT{Jump if Not Equal}.}

\IFRU{Итак, если \EAX не равен 1, то \JNE заставит перейти процессор 
по адресу указанном в операнде \JNE, у нас это \TT{\$LN2@main}.}
{So, if value in the \EAX register not equals to $1$, then the processor will pass execution to the 
address mentioned in operand of \JNE, in our case it is \TT{\$LN2@main}.}
\IFRU
{Передав управление по этому адресу, процессор как раз начнет исполнять вызов \printf с 
аргументом \TT{``What you entered? Huh?''}.}
{Passing control to this address, microprocesor will execute function \printf 
with argument \TT{``What you entered? Huh?''}.}
\IFRU
{Но если все нормально, перехода не случится, и исполнится другой \printf с двумя аргументами: 
\TT{'You entered \%d...'} и значением переменной \TT{x}.}
{But if everything is fine, conditional jump will not be taken, and another \printf call 
will be executed, with two arguments: \TT{'You entered \%d...'} and value of variable \TT{x}. }

\index{x86!\Instructions!XOR}
\index{\CLanguageElements!return}
\IFRU {А для того чтобы после этого вызова не исполнился сразу второй вызов \printf, 
после него имеется инструкция \JMP, безусловный переход, он отправит процессор на место аккурат 
после второго \printf и перед инструкцией \TT{XOR EAX, EAX}, которая собственно \TT{return 0}.}
{Since second subsequent \printf not needed to be executed, there is \JMP after (unconditional jump),
it will pass control to the point after second \printf and before \TT{XOR EAX, EAX} instruction, 
which implement \TT{return 0}.}

\index{x86!\Registers!\Flags}
\IFRU{Итак, можно сказать, что в подавляющих случаях сравнение какой либо переменной с чем-то другим 
происходит при помощи пары инструкций \CMP и \Jcc, где \IT{cc} это \IT{condition code}.}
{So, it can be said that comparing a value with another is \IT{usually} implemented
by \CMP/\Jcc instructions pair, where \IT{cc} is \IT{condition code}.}
\IFRU{\CMP сравнивает два значения и выставляет 
флаги процессора\footnote{См.также о флагах x86-процессора: \url{http://en.wikipedia.org/wiki/FLAGS_register_(computing)}.}.}
{\CMP comparing two values and set 
processor flags\footnote{About x86 flags, see also: \url{http://en.wikipedia.org/wiki/FLAGS_register_(computing)}.}.}
\IFRU
{\Jcc проверяет нужные ему флаги и выполняет переход по указанному адресу (или не выполняет).}
{\Jcc check flags needed to be checked and pass control to mentioned address (or not pass).}

\index{x86!\Instructions!CMP}
\index{x86!\Instructions!SUB}
\label{CMPandSUB}
\IFRU{Но на самом деле, как это не парадоксально поначалу звучит, \CMP это почти то же самое что и 
инструкция \SUB, которая отнимает числа одно от другого.}
{But in fact, this could be perceived paradoxical, but \CMP instruction is in fact \SUB (subtract).}
\IFRU{Все арифметические инструкции также выставляют флаги в соответствии с результатом, не только \CMP.}
{All arithmetic instructions set processor flags too, not only \CMP.}
\IFRU{Если мы сравним 1 и 1, от единицы отнимется единица, получится $0$, и выставится флаг 
\ZF (\IT{zero flag}), означающий что последний полученный результат был $0$.}
{If we compare 1 and 1, $1-1$ will be $0$ in result, \ZF flag will be set (meaning the last result was $0$).}
\IFRU{Ни при каких других значениях \EAX, флаг \ZF выставлен не будет, кроме тех, когда операнды равны друг другу.}
{There is no any other circumstance when it is possible except when operands are equal.}
\index{x86!\Instructions!JNE}
\index{x86!\Registers!ZF}
\IFRU{Инструкция \JNE проверяет только флаг \ZF, и совершает переход только если флаг не поднят. 
Фактически, \JNE это синоним инструкции \JNZ (\IT{Jump if Not Zero}).}
{\JNE checks only \ZF flag and jumping only if it is not set. 
\JNE is in fact a synonym of \JNZ (\IT{Jump if Not Zero}) instruction.}
\IFRU{Ассемблер транслирует обе инструкции в один и тот же опкод.}
{Assembler translating both \JNE and \JNZ instructions into one single opcode.}
\IFRU
{Таким образом, можно \CMP заменить на \SUB и все будет работать также, но разница в том что \SUB 
все-таки испортит значение в первом операнде. \CMP это \IT{SUB без сохранения результата}.}
{So, \CMP instruction can be replaced to \SUB instruction and almost everything will be fine,
but the difference is in 
the \SUB alter the value of the first operand.
\CMP is \IT{``SUB without saving result''}.}

\IFRU
{Код созданный при помощи GCC 4.4.1 в Linux практически такой же, если не считать мелких отличий, 
которые мы уже рассмотрели раннее.}
{Code generated by GCC 4.4.1 in Linux is almost the same, except differences we already considered.}

\input{04_scanf/checking_retval_ARM}


\section{\IFRU{Передача параметров через стек}{Passing arguments via stack}}

\IFRU{Как мы уже успели заметить, вызывающая функция передает аргументы для вызываемой через стек. 
А как вызываемая функция имеет к ним доступ?}
{Now we figured out that caller function passing arguments to callee via stack. 
But how callee\footnote{function being called} access them?}

\lstinputlisting{05_passing_arguments/ex.c}

\subsection{x86: \IFRU{3 аргумента}{3 arguments}}

\subsubsection{MSVC}

\IFRU{Компилируем при помощи MSVC 2010 Express, и в итоге получим:}
{Let's compile it by MSVC 2010 Express and we got:}

\begin{lstlisting}
$SG3830	DB	'a=%d; b=%d; c=%d', 00H

...

	push	3
	push	2
	push	1
	push	OFFSET $SG3830
	call	_printf
	add	esp, 16					; 00000010H
\end{lstlisting}

\IFRU{Все почти то же, за исключением того, что теперь видно, что аргументы для \printf заталкиваются в стек в обратном порядке: самый первый аргумент заталкивается последним.}
{Almost the same, but now we can see the \printf arguments are pushing into stack in reverse order: and the first argument is pushing in as the last one.}

\IFRU{Кстати, вспомним что переменные типа \Tint в 32-битной системе, как известно, имеет ширину 32 бита, это 4 байта}
{By the way, variables of \Tint type in 32-bit environment has 32-bit width that is 4 bytes}.

\IFRU{Итак, у нас всего 4 аргумента. $4*4 = 16$ ~--- именно 16 байт занимают в стеке указатель на строку плюс еще 3 числа типа \Tint.}
{So, we got here 4 arguments. $4*4 = 16$~---they occupy exactly 16 bytes in the stack: 32-bit pointer to string and 3 number of \Tint type.}

\index{x86!\Instructions!ADD}
\index{x86!\Registers!ESP}
\index{cdecl}
\IFRU{Когда при помощи инструкции \TT{``ADD ESP, X''} корректируется \glslink{stack pointer}{указатель стека} \ESP 
после вызова какой-либо функции, зачастую можно сделать вывод о том, сколько аргументов 
у вызываемой функции было, разделив X на 4.}
{When \gls{stack pointer} (the \ESP register) is corrected by \TT{``ADD ESP, X''}
instruction after a function 
call, often, the number of function arguments could be deduced here: just divide X by 4.}

\IFRU{Конечно, это относится только к cdecl-методу передачи аргументов через стек.}
{Of course, this is related only to \IT{cdecl} calling convention.}

\IFRU{См. также в соответствующем разделе о способах передачи аргументов через стек}
{See also section about calling conventions}~(\ref{sec:callingconventions}).

\IFRU{Иногда бывает так, что подряд идут несколько вызовов разных функций, 
но стек корректируется только один раз, после последнего вызова:}
{It is also possible for compiler to merge several \TT{``ADD ESP, X''} instructions into one, after last call:}

\begin{lstlisting}
push a1
push a2
call ...
...
push a1
call ...
...
push a1
push a2
push a3
call ...
add esp, 24
\end{lstlisting}

\subsubsection{MSVC \AndENRU \olly}
\index{\olly}

\IFRU{Попробуем этот же пример в}{Now let's try to load this example in} \olly.
\IFRU{Это один из наиболее популярных win32-отладчиков user-режима}{It is one of the most 
popular user-land win32 debugger}.
\IFRU{Мы можем компилировать наш пример в}{We can try to compile our example in} MSVC 2012 
\IFRU{с опцией}{with} \TT{/MD} \IFRU{что означает, линковать с библиотекой}{option, meaning, to link 
against} \TT{MSVCR*.DLL},
\IFRU{чтобы импортируемые ф-ции были хорошо видны в отладчике}{so we will able to see imported 
functions clearly in debugger}.

\IFRU{Затем загружаем исполняемый файл в}{Then load executable in} \olly.
\IFRU{Самый первый брякпойнт в}{The very first breakpoint is in} \TT{ntdll.dll}, \IFRU{нажмите}{press} 
F9 (\IFRU{запустить}{run}).
\IFRU{Второй брякпойнт в}{The second breakpoint is in} \ac{CRT}-\IFRU{коде}{code}.
\IFRU{Теперь мы должны найти ф-цию}{Now we should find} \main\EN{ function}.

\IFRU{Найдите этот код скроллируя окно кода до самого верха (MSVC располагает ф-цию \main в самом начале
секции кода)}{Find this code by scrolling the code to the very bottom (MSVC allocates \main function at
the very beginning of the code section)}: 
\figname \ref{fig:printf3_olly_1}.

\IFRU{Кликните на инструкции}{Click on} \TT{PUSH EBP}\IFRU{, нажмите}{ instruction, press} F2 
(\IFRU{установка брякпойнта}{set breakpoint}) \IFRU{и нажмите}{and press} F9 (\IFRU{запустить}{run}).
\IFRU{Нам нужно произвести все эти манипуляции, чтобы пропустить \ac{CRT}-код, потому что нам он пока
не интересен}{We need to do these manupulations in order to skip \ac{CRT}-code, because, we don't really 
interesting in it yet}.

\IFRU{Нажмите}{Press} F8 (\stepover) 6 \IFRU{раз, т.е., пропустить
6 инструкций}{times, i.e., skip 6 instructions}: \figname \ref{fig:printf3_olly_2}.

\IFRU{Теперь}{Now the} \PC \IFRU{указывает на инструкцию}{points to the}
\TT{CALL printf}\EN{ instruction}.
\olly, \IFRU{как и другие отладчики, подсвечивает регистры со значениями, которые изменились}
{like other debuggers, highlights value of registers which were changed}.
\IFRU{Так что, каждый раз, когда мы нажимаем}{So each time you press F8}, \EIP 
\IFRU{изменяется и его значение подсвечивается красным}{is changing and its value looking red}.
\ESP \IFRU{также меняется, потому что значения заталкиваются в стек}{is changing as well, 
because values are pushed into the stack}.

\IFRU{Где находятся эти значения в стеке}{Where are the values in the stack}?
\IFRU{Посмотрите на правое/нижнее окно в отладчике}{Take a look into right/bottom window of debugger}:

\begin{figure}[H]
\centering
\includegraphics[scale=0.66]{patterns/03_printf/olly3_stack.png}
\caption{\olly: \IFRU{стек, после того как значения там сохранены}{stack after values pushed}
(\IFRU{я сделал здесь округлую красную пометку в графическом редакторе}{I made round red mark 
here in graphics editor})}
\end{figure}

\IFRU{Так что здесь видно 3 столбца: адрес в стеке, значение в стеке и еще дополнительный комментарий
от \olly}{So we can see there 3 columns: address in the stack, 
value in the stack and some additional \olly comments}. 
\olly \IFRU{понимает}{understands} \printf\IFRU{-строки}{-like strings}, 
\IFRU{так что он показывает здесь и строку и 3 значения \IT{привязанных} к ней}{so it reports the 
string here and 3 values \IT{attached} to it}.

\IFRU{Нажмите}{Press} F8 (\stepover).

\IFRU{В коносил мы видим вывод}{In the console we'll see the output}:

\begin{figure}[H]
\centering
\includegraphics[scale=0.66]{patterns/03_printf/olly3_console.png}
\caption{\RU{Ф-ция }\printf \IFRU{исполнилась}{function executed}}
\end{figure}

\IFRU{Посмотрим, как изменились регистры и состояние стека}{Let's see how registers and stack state 
are changed}: \figname \ref{fig:printf3_olly_3}.

\RU{Регистр }\EAX \IFRU{теперь содержит}{register now contains} \TT{0xD} (13).
That's correct, \printf returns number of characters printed.
\RU{Значение }\EIP \IFRU{изменилось: действительно, теперь здесь адрес инструкции после}
{value is changed: indeed, now there is address of the instruction after} \TT{CALL printf}.
\RU{Значения регистров }\ECX \AndENRU \EDX \IFRU{также изменились}{values are changed as well}.
\IFRU{Очевидно, внутренности ф-ции \printf используют их для каких-то своих нужд}{Apparently, 
\printf function's hidden machinery used them for its own needs}.

\IFRU{Очень важный момент в том что значение \ESP не изменилось. И состояние стека также!}
{A very important thing is that \ESP value is not changed. And stack state too!}
\IFRU{Мы ясно видим здесь и строку формата и соответствующие ей 3 значения, они все еще здесь}
{We clearly see that format string and corresponding 3 values are still there}.
\IFRU{Действительно, по соглашению вызовов \IT{cdecl}, вызывающая ф-ция не очищает аргументы из стека}
{Indeed, that's \IT{cdecl} calling convention, calling function doesn't clear arguments in stack}.
\IFRU{Это должна делать вызывающая ф-ция}{It's caller's duty to do so}.

\IFRU{Нажмите}{Press} F8 \IFRU{снова, чтобы исполнилась инструкция}{again to execute} 
\TT{ADD ESP, 10}\EN{ instruction}: \figname \ref{fig:printf3_olly_4}.

\ESP \IFRU{изменился, но значения все еще в стеке}{is changed, but values are still in the stack}!
\IFRU{Конечно, никому не нужно заполнять эти значения нулями или что-то в этом роде}{Yes, 
of course, no one needs to fill these values by zero or something like that}.
\IFRU{Потому что всё что выше указателя стека}{Because, everything above stack pointer} (\SP) 
\IFRU{это}{is} \IT{\IFRU{шум}{noise}} \OrENRU \IT{\IFRU{мусор}{garbage}}, \IFRU{это всё не имеет
особой ценности}{it has no value at all}.
\IFRU{Было бы очень затратно по времени очищать ненужные элементы стека, к тому же, никому это и не 
нужно}{It would be time consuming to clear unused stack entries, besides, no one really needs to}.

\begin{figure}[H]
\centering
\includegraphics[scale=0.66]{patterns/03_printf/olly3_1.png}
\caption{\olly: \IFRU{самое начало ф-ции}{the very start of the} \main\EN{ function}}
\label{fig:printf3_olly_1}
\end{figure}

\begin{figure}[H]
\centering
\includegraphics[scale=0.66]{patterns/03_printf/olly3_2.png}
\caption{\olly: \IFRU{перед исполнением}{before} \printf\EN{ execution}}
\label{fig:printf3_olly_2}
\end{figure}

\begin{figure}[H]
\centering
\includegraphics[scale=0.66]{patterns/03_printf/olly3_3.png}
\caption{\olly: \IFRU{после исполнения}{after} \printf\EN{ execution}}
\label{fig:printf3_olly_3}
\end{figure}

\begin{figure}[H]
\centering
\includegraphics[scale=0.66]{patterns/03_printf/olly3_4.png}
\caption{\olly: \IFRU{после исполнения инструкции}{after} \TT{ADD ESP, 10}\EN{ instruction execution}}
\label{fig:printf3_olly_4}
\end{figure}

\subsubsection{GCC}

\IFRU{Скомпилируем то же самое в Linux при помощи GCC 4.4.1 и посмотрим в \IDA что вышло:}
{Now let's compile the same in Linux by GCC 4.4.1 and take a look in \IDA what we got:}

\begin{lstlisting}
main            proc near

var_10          = dword ptr -10h
var_C           = dword ptr -0Ch
var_8           = dword ptr -8
var_4           = dword ptr -4

                push    ebp
                mov     ebp, esp
                and     esp, 0FFFFFFF0h
                sub     esp, 10h
                mov     eax, offset aADBDCD ; "a=%d; b=%d; c=%d"
                mov     [esp+10h+var_4], 3
                mov     [esp+10h+var_8], 2
                mov     [esp+10h+var_C], 1
                mov     [esp+10h+var_10], eax
                call    _printf
                mov     eax, 0
                leave
                retn
main            endp
\end{lstlisting}

\IFRU{Можно сказать, что этот короткий код, созданный GCC, отличается от кода MSVC только способом помещения 
значений в стек.
Здесь GCC снова работает со стеком напрямую без \PUSH/\POP.}
{It can be said, the difference between code by MSVC and GCC is only in method of placing arguments on the stack.
Here GCC working directly with stack without \PUSH/\POP.}


\subsection{ARM}

\subsubsection{\NonOptimizingKeil + \ARMMode}

\begin{lstlisting}
.text:000000A4 00 30 A0 E1                 MOV     R3, R0
.text:000000A8 93 21 20 E0                 MLA     R0, R3, R1, R2
.text:000000AC 1E FF 2F E1                 BX      LR
...
.text:000000B0             main
.text:000000B0 10 40 2D E9                 STMFD   SP!, {R4,LR}
.text:000000B4 03 20 A0 E3                 MOV     R2, #3
.text:000000B8 02 10 A0 E3                 MOV     R1, #2
.text:000000BC 01 00 A0 E3                 MOV     R0, #1
.text:000000C0 F7 FF FF EB                 BL      f
.text:000000C4 00 40 A0 E1                 MOV     R4, R0
.text:000000C8 04 10 A0 E1                 MOV     R1, R4
.text:000000CC 5A 0F 8F E2                 ADR     R0, aD_0        ; "%d\n"
.text:000000D0 E3 18 00 EB                 BL      __2printf
.text:000000D4 00 00 A0 E3                 MOV     R0, #0
.text:000000D8 10 80 BD E8                 LDMFD   SP!, {R4,PC}
\end{lstlisting}

\IFRU{В функции \main просто вызываются две функции, в первую (\TT{f}) передается три значения.}
{In \main function, two other functions are simply called, and three values are passed to the 
first one (\TT{f}).}

\IFRU{Как я уже упоминал, первые 4 значения, в ARM обычно передаются в первых 4-х регистрах}
{As I mentioned before, in ARM, first 4 values are usually passed in first 4 registers} (\Rzero-\Rthree).

\IFRU{Функция }{}\TT{f}\IFRU{, как видно, использует три первых регистра (\Rzero-\Rtwo) как аргументы.}
{function, as it seems, use first 3 registers (\Rzero-\Rtwo) as arguments.}

\IFRU{Инструкция }{}\TT{MLA} (\IT{Multiply Accumulate}) \IFRU{перемножает два первых операнда (\Rthree и \Rone), 
прибавляет к произведению
третий операнд (\Rtwo) и помещает результат в нулевой операнд (\Rzero), через который, по стандарту, 
возвращаются значения функций.}
{instruction multiplicates two first operands (\Rthree and \Rone), adds third operand (\Rtwo) to product and places
result into zeroth operand (\Rzero), via which, by standard, values are returned from functions.}

\IFRU{Умножение и сложение одновременно}{Multiplication and addition at once}\footnote{\WPMAO} 
(\IT{Fused multiply–add}) \IFRU{это много где применяемая операция, кстати, аналогичной
инструкции в x86 нет}{is very useful operation, by the way, there are no such instruction in x86}, 
\IFRU{если не считать новых FMA-инструкций}{if not to count new FMA-instruction}\footnote{\url{https://en.wikipedia.org/wiki/FMA_instruction_set}} \IFRU{в}{in} SIMD.

\IFRU{Самая первая инструкция}{The very first} \TT{MOV R3, R0}, \IFRU{по видимому, избыточна (можно было бы обойтись только одной инструкцией \TT{MLA})}
{instruction, as it seems, redundant (single \TT{MLA} instruction could be used here instead)}, 
\IFRU{компилятор не оптимизировал её, ведь, это компиляция без оптимизации}{compiler wasn't optimized it,
because, this is non-optimizing compilation}.

\IFRU{Инструкция \TT{BX} возвращает управление по адресу записанному в \LR и, если нужно, 
переключает режимы процессора с thumb на ARM или наоборот.}
{\TT{BX} instruction returns control to the address stored in \LR and, if need, switches processor mode from
thumb to ARM or vice versa.}
\IFRU{Это может быть необходимым потому, что, как мы видим, 
функции \TT{f} неизвестно, из какого кода она будет вызываться, из ARM или thumb.}
{This can be necessary because, as we can see, \TT{f} function is not aware, from which code it may be
called, from ARM or thumb.}
\IFRU{Поэтому, если она будет вызываться из кода thumb, \TT{BX} не только вернет
управление в вызывающую функцию, но также переключит процессор в режим thumb.}
{This, if it will be called from thumb code, \TT{BX} will not only return control to the calling function,
but also will switch processor mode to thumb mode.}
\IFRU{Либо не переключит, если функция вызывалась из кода для режима ARM.}
{Or not switch, if the function was called from ARM code.}

\subsubsection{\OptimizingKeil + \ARMMode}

\begin{lstlisting}
.text:00000098             f
.text:00000098 91 20 20 E0                 MLA     R0, R1, R0, R2
.text:0000009C 1E FF 2F E1                 BX      LR
\end{lstlisting}

\IFRU{А вот и функция \TT{f} скомпилированная компилятором Keil в режиме полной оптимизации}
{And here is \TT{f} function compiled by Keil compiler in full optimization mode} (\Othree).
\IFRU{Инструкция \MOV была соптимизирована и теперь \TT{MLA} использует все входящие регистры 
и помещает результат в \Rzero, как раз, где вызываемая функция будет его читать и использовать.}
{\MOV instruction was optimized and now \TT{MLA} uses all input registers and place result into \Rzero, 
exactly where calling function will read it and use.}

\subsubsection{\OptimizingKeil + \ThumbMode}

\begin{lstlisting}
.text:0000005E 48 43                       MULS    R0, R1
.text:00000060 80 18                       ADDS    R0, R0, R2
.text:00000062 70 47                       BX      LR
\end{lstlisting}

\IFRU{В режиме thumb, инструкция \TT{MLA} недоступна, так что компилятору пришлось сгенерировать код, делающий
обе операции по отдельности.}
{\TT{MLA} instruction is not available in thumb mode, so, compiler generates the code doing these two operations
separately.}
\IFRU{Первая инструкция \TT{MULS} умножает \Rzero на \Rone оставляя результат в \Rone.}
{First \TT{MULS} instruction multiply \Rzero by \Rone leaving result in \Rone.}
\IFRU{Вторая (\TT{ADDS}) складывает результат и \Rtwo, оставляя результат в \Rzero.}
{Second (\TT{ADDS}) instruction adds result and \Rtwo leaving result in \Rzero.}



\section{\IFRU{И еще немного о возвращаемых результатах}{One more word about results returning.}}

\newcommand{\MSDNURL}{\href{http://msdn.microsoft.com/en-us/library/7572ztz4.aspx}{MSDN: Return Values (C++)}}

\IFRU{Резльутат выполнения функции в x86 обычно возвращается\footnote{См.также: \MSDNURL} через регистр \EAX, 
а если результат имеет тип байт или символ (\IT{char}), 
то в самой младшей части \EAX ~--- \AL. Если функция возвращает число с плавающей запятой, 
то регистр FPU \STZERO будет использован.
В ARM обычно результат возвращается в регистре R0.}
{As of x86, function execution result is usually returned\footnote{See also: \MSDNURL} in 
\EAX register. 
If it's byte type or character (\IT{char}) ~--- then in lowest register \EAX part ~--- \AL. 
If function returning \Tfloat number, FPU register 
\STZERO will be used instead.
In ARM, result is usually returned in R0 register.}

\IFRU{Вот почему старые компиляторы Си не способны создавать функции возвращающие нечто большее нежели помещается 
в один регистр (обычно, тип \Tint), а когда нужно, приходится возвращать через указатели, указываемые 
в аргументах.}
{That is why old C compilers can't create functions capable of returning something not fitting in one 
register (usually type \Tint), but if one need it, one should return information via pointers passed 
in function arguments.}
\IFRU{Хотя, позже и стало возможным, вернуть, скажем, целую структуру, но этот метод до сих пор не очень популярен. 
Если функция должна вернуть структуру, вызывающая функция должна сама, скрыто и прозрачно для программиста, 
выделить место и передать указатель на него в качестве первого аргумента. Это почти то же самое 
что и сделать это вручную, но компилятор прячет это.

Небольшой пример:}
{Now it is possible, to return, let's say, whole structure, but its still not very popular. 
If function should return a large structure, caller must allocate it and pass pointer to it via first argument, 
hiddenly and transparently for programmer. 
That is almost the same as to pass pointer in first argument manually, but compiler hide this.

Small example:}

\lstinputlisting{06_return_results/6_1.c}

\dots \IFRU{получим}{what we got} (MSVC 2010 \Ox):

\lstinputlisting{06_return_results/6_1.asm}

\IFRU{Имя внутреннего макроса для передачи указателя на структуру здесь это \TT{\$T3853}.}
{Macro name for internal variable passing pointer to structure is \TT{\$T3853} here.}


\input{061_pointers/ptrs_and_refs}
\section{\IFRU{Условные переходы}{Conditional jumps}}
\label{sec:Jcc}

\IFRU{Об условных переходах.}{Now about conditional jumps.}

\lstinputlisting{07_jcc/7_1.c}

\subsection{x86: \IFRU{3 аргумента}{3 arguments}}

\subsubsection{MSVC}

\IFRU{Компилируем при помощи MSVC 2010 Express, и в итоге получим:}
{Let's compile it by MSVC 2010 Express and we got:}

\begin{lstlisting}
$SG3830	DB	'a=%d; b=%d; c=%d', 00H

...

	push	3
	push	2
	push	1
	push	OFFSET $SG3830
	call	_printf
	add	esp, 16					; 00000010H
\end{lstlisting}

\IFRU{Все почти то же, за исключением того, что теперь видно, что аргументы для \printf заталкиваются в стек в обратном порядке: самый первый аргумент заталкивается последним.}
{Almost the same, but now we can see the \printf arguments are pushing into stack in reverse order: and the first argument is pushing in as the last one.}

\IFRU{Кстати, вспомним что переменные типа \Tint в 32-битной системе, как известно, имеет ширину 32 бита, это 4 байта}
{By the way, variables of \Tint type in 32-bit environment has 32-bit width that is 4 bytes}.

\IFRU{Итак, у нас всего 4 аргумента. $4*4 = 16$ ~--- именно 16 байт занимают в стеке указатель на строку плюс еще 3 числа типа \Tint.}
{So, we got here 4 arguments. $4*4 = 16$~---they occupy exactly 16 bytes in the stack: 32-bit pointer to string and 3 number of \Tint type.}

\index{x86!\Instructions!ADD}
\index{x86!\Registers!ESP}
\index{cdecl}
\IFRU{Когда при помощи инструкции \TT{``ADD ESP, X''} корректируется \glslink{stack pointer}{указатель стека} \ESP 
после вызова какой-либо функции, зачастую можно сделать вывод о том, сколько аргументов 
у вызываемой функции было, разделив X на 4.}
{When \gls{stack pointer} (the \ESP register) is corrected by \TT{``ADD ESP, X''}
instruction after a function 
call, often, the number of function arguments could be deduced here: just divide X by 4.}

\IFRU{Конечно, это относится только к cdecl-методу передачи аргументов через стек.}
{Of course, this is related only to \IT{cdecl} calling convention.}

\IFRU{См. также в соответствующем разделе о способах передачи аргументов через стек}
{See also section about calling conventions}~(\ref{sec:callingconventions}).

\IFRU{Иногда бывает так, что подряд идут несколько вызовов разных функций, 
но стек корректируется только один раз, после последнего вызова:}
{It is also possible for compiler to merge several \TT{``ADD ESP, X''} instructions into one, after last call:}

\begin{lstlisting}
push a1
push a2
call ...
...
push a1
call ...
...
push a1
push a2
push a3
call ...
add esp, 24
\end{lstlisting}

\subsubsection{MSVC \AndENRU \olly}
\index{\olly}

\IFRU{Попробуем этот же пример в}{Now let's try to load this example in} \olly.
\IFRU{Это один из наиболее популярных win32-отладчиков user-режима}{It is one of the most 
popular user-land win32 debugger}.
\IFRU{Мы можем компилировать наш пример в}{We can try to compile our example in} MSVC 2012 
\IFRU{с опцией}{with} \TT{/MD} \IFRU{что означает, линковать с библиотекой}{option, meaning, to link 
against} \TT{MSVCR*.DLL},
\IFRU{чтобы импортируемые ф-ции были хорошо видны в отладчике}{so we will able to see imported 
functions clearly in debugger}.

\IFRU{Затем загружаем исполняемый файл в}{Then load executable in} \olly.
\IFRU{Самый первый брякпойнт в}{The very first breakpoint is in} \TT{ntdll.dll}, \IFRU{нажмите}{press} 
F9 (\IFRU{запустить}{run}).
\IFRU{Второй брякпойнт в}{The second breakpoint is in} \ac{CRT}-\IFRU{коде}{code}.
\IFRU{Теперь мы должны найти ф-цию}{Now we should find} \main\EN{ function}.

\IFRU{Найдите этот код скроллируя окно кода до самого верха (MSVC располагает ф-цию \main в самом начале
секции кода)}{Find this code by scrolling the code to the very bottom (MSVC allocates \main function at
the very beginning of the code section)}: 
\figname \ref{fig:printf3_olly_1}.

\IFRU{Кликните на инструкции}{Click on} \TT{PUSH EBP}\IFRU{, нажмите}{ instruction, press} F2 
(\IFRU{установка брякпойнта}{set breakpoint}) \IFRU{и нажмите}{and press} F9 (\IFRU{запустить}{run}).
\IFRU{Нам нужно произвести все эти манипуляции, чтобы пропустить \ac{CRT}-код, потому что нам он пока
не интересен}{We need to do these manupulations in order to skip \ac{CRT}-code, because, we don't really 
interesting in it yet}.

\IFRU{Нажмите}{Press} F8 (\stepover) 6 \IFRU{раз, т.е., пропустить
6 инструкций}{times, i.e., skip 6 instructions}: \figname \ref{fig:printf3_olly_2}.

\IFRU{Теперь}{Now the} \PC \IFRU{указывает на инструкцию}{points to the}
\TT{CALL printf}\EN{ instruction}.
\olly, \IFRU{как и другие отладчики, подсвечивает регистры со значениями, которые изменились}
{like other debuggers, highlights value of registers which were changed}.
\IFRU{Так что, каждый раз, когда мы нажимаем}{So each time you press F8}, \EIP 
\IFRU{изменяется и его значение подсвечивается красным}{is changing and its value looking red}.
\ESP \IFRU{также меняется, потому что значения заталкиваются в стек}{is changing as well, 
because values are pushed into the stack}.

\IFRU{Где находятся эти значения в стеке}{Where are the values in the stack}?
\IFRU{Посмотрите на правое/нижнее окно в отладчике}{Take a look into right/bottom window of debugger}:

\begin{figure}[H]
\centering
\includegraphics[scale=0.66]{patterns/03_printf/olly3_stack.png}
\caption{\olly: \IFRU{стек, после того как значения там сохранены}{stack after values pushed}
(\IFRU{я сделал здесь округлую красную пометку в графическом редакторе}{I made round red mark 
here in graphics editor})}
\end{figure}

\IFRU{Так что здесь видно 3 столбца: адрес в стеке, значение в стеке и еще дополнительный комментарий
от \olly}{So we can see there 3 columns: address in the stack, 
value in the stack and some additional \olly comments}. 
\olly \IFRU{понимает}{understands} \printf\IFRU{-строки}{-like strings}, 
\IFRU{так что он показывает здесь и строку и 3 значения \IT{привязанных} к ней}{so it reports the 
string here and 3 values \IT{attached} to it}.

\IFRU{Нажмите}{Press} F8 (\stepover).

\IFRU{В коносил мы видим вывод}{In the console we'll see the output}:

\begin{figure}[H]
\centering
\includegraphics[scale=0.66]{patterns/03_printf/olly3_console.png}
\caption{\RU{Ф-ция }\printf \IFRU{исполнилась}{function executed}}
\end{figure}

\IFRU{Посмотрим, как изменились регистры и состояние стека}{Let's see how registers and stack state 
are changed}: \figname \ref{fig:printf3_olly_3}.

\RU{Регистр }\EAX \IFRU{теперь содержит}{register now contains} \TT{0xD} (13).
That's correct, \printf returns number of characters printed.
\RU{Значение }\EIP \IFRU{изменилось: действительно, теперь здесь адрес инструкции после}
{value is changed: indeed, now there is address of the instruction after} \TT{CALL printf}.
\RU{Значения регистров }\ECX \AndENRU \EDX \IFRU{также изменились}{values are changed as well}.
\IFRU{Очевидно, внутренности ф-ции \printf используют их для каких-то своих нужд}{Apparently, 
\printf function's hidden machinery used them for its own needs}.

\IFRU{Очень важный момент в том что значение \ESP не изменилось. И состояние стека также!}
{A very important thing is that \ESP value is not changed. And stack state too!}
\IFRU{Мы ясно видим здесь и строку формата и соответствующие ей 3 значения, они все еще здесь}
{We clearly see that format string and corresponding 3 values are still there}.
\IFRU{Действительно, по соглашению вызовов \IT{cdecl}, вызывающая ф-ция не очищает аргументы из стека}
{Indeed, that's \IT{cdecl} calling convention, calling function doesn't clear arguments in stack}.
\IFRU{Это должна делать вызывающая ф-ция}{It's caller's duty to do so}.

\IFRU{Нажмите}{Press} F8 \IFRU{снова, чтобы исполнилась инструкция}{again to execute} 
\TT{ADD ESP, 10}\EN{ instruction}: \figname \ref{fig:printf3_olly_4}.

\ESP \IFRU{изменился, но значения все еще в стеке}{is changed, but values are still in the stack}!
\IFRU{Конечно, никому не нужно заполнять эти значения нулями или что-то в этом роде}{Yes, 
of course, no one needs to fill these values by zero or something like that}.
\IFRU{Потому что всё что выше указателя стека}{Because, everything above stack pointer} (\SP) 
\IFRU{это}{is} \IT{\IFRU{шум}{noise}} \OrENRU \IT{\IFRU{мусор}{garbage}}, \IFRU{это всё не имеет
особой ценности}{it has no value at all}.
\IFRU{Было бы очень затратно по времени очищать ненужные элементы стека, к тому же, никому это и не 
нужно}{It would be time consuming to clear unused stack entries, besides, no one really needs to}.

\begin{figure}[H]
\centering
\includegraphics[scale=0.66]{patterns/03_printf/olly3_1.png}
\caption{\olly: \IFRU{самое начало ф-ции}{the very start of the} \main\EN{ function}}
\label{fig:printf3_olly_1}
\end{figure}

\begin{figure}[H]
\centering
\includegraphics[scale=0.66]{patterns/03_printf/olly3_2.png}
\caption{\olly: \IFRU{перед исполнением}{before} \printf\EN{ execution}}
\label{fig:printf3_olly_2}
\end{figure}

\begin{figure}[H]
\centering
\includegraphics[scale=0.66]{patterns/03_printf/olly3_3.png}
\caption{\olly: \IFRU{после исполнения}{after} \printf\EN{ execution}}
\label{fig:printf3_olly_3}
\end{figure}

\begin{figure}[H]
\centering
\includegraphics[scale=0.66]{patterns/03_printf/olly3_4.png}
\caption{\olly: \IFRU{после исполнения инструкции}{after} \TT{ADD ESP, 10}\EN{ instruction execution}}
\label{fig:printf3_olly_4}
\end{figure}

\subsubsection{GCC}

\IFRU{Скомпилируем то же самое в Linux при помощи GCC 4.4.1 и посмотрим в \IDA что вышло:}
{Now let's compile the same in Linux by GCC 4.4.1 and take a look in \IDA what we got:}

\begin{lstlisting}
main            proc near

var_10          = dword ptr -10h
var_C           = dword ptr -0Ch
var_8           = dword ptr -8
var_4           = dword ptr -4

                push    ebp
                mov     ebp, esp
                and     esp, 0FFFFFFF0h
                sub     esp, 10h
                mov     eax, offset aADBDCD ; "a=%d; b=%d; c=%d"
                mov     [esp+10h+var_4], 3
                mov     [esp+10h+var_8], 2
                mov     [esp+10h+var_C], 1
                mov     [esp+10h+var_10], eax
                call    _printf
                mov     eax, 0
                leave
                retn
main            endp
\end{lstlisting}

\IFRU{Можно сказать, что этот короткий код, созданный GCC, отличается от кода MSVC только способом помещения 
значений в стек.
Здесь GCC снова работает со стеком напрямую без \PUSH/\POP.}
{It can be said, the difference between code by MSVC and GCC is only in method of placing arguments on the stack.
Here GCC working directly with stack without \PUSH/\POP.}


\subsection{ARM}

\subsubsection{\OptimizingKeil + \ARMMode}

\lstinputlisting[caption=\OptimizingKeil + \ARMMode]{07_jcc/ARM_O3_signed.asm}

\index{ARM!Condition codes}
\IFRU{Многие инструкции в режиме ARM могут быть исполнены только при некоторых выставленных флагах.}
{A lot of instructions in ARM mode can be executed only when specific flags are set.}
\IFRU{Это нередко используется для сравнения чисел, например.}
{This is often used while numbers comparing, for example.}

\index{ARM!\Instructions!ADD}
\index{ARM!\Instructions!ADDAL}
\IFRU{К примеру, инструкция \ADD на самом деле может быть представлена как \TT{ADDAL}, \TT{AL} означает 
\IT{Always}, то есть, исполнять всегда.}
{For instance, \ADD instruction is \TT{ADDAL} internally in fact, where \TT{AL} meaning
\IT{Always}, i.e., execute always.}
\IFRU{Предикаты кодируются в 4-х старших битах инструкции 32-битных ARM-инструкций}
{Predicates are encoded in 4 high bits of 32-bit ARM instructions} (\IT{condition field}).
\index{ARM!\Instructions!B}
\IFRU{Инструкция безусловного перехода \TT{B}, на самом деле условная и кодируется так же 
как и прочие инструкции условных переходов, но имеет \TT{AL} в \IT{condition field}, 
то есть, исполняется всегда, игнорируя флаги.}
{\TT{B} instruction of unconditional jump is in fact conditional and encoded just like any other
conditional jumps, but has \TT{AL} in the \IT{condition field}, and what it means, executing always, ignoring flags.}

\index{ARM!\Instructions!ADR}
\index{ARM!\Instructions!ADRGT}
\index{ARM!\Instructions!CMP}
\IFRU{Инструкция \TT{ADRGT} работает так же как и \TT{ADR}, но исполнится только в случае 
если предыдущая инструкция \CMP,
сравнивая два числа, обнаружила что одно из них больше второго}
{\TT{ADRGT} instructions works just like \TT{ADR} but will execute only in the case when previous \CMP
instruction, while comparing two numbers, found one number greater than another}
(\IT{Greater Than}).

\index{ARM!\Instructions!BL}
\index{ARM!\Instructions!BLGT}
\IFRU{Следующая инструкция \TT{BLGT} ведет себя так же как и \TT{BL} и сработает только если 
результат сравнения был такой же}{The next \TT{BLGT} instruction behaves exactly as \TT{BL} and will be
triggered only if result of comparison was the same} (\IT{Greater Than}). 
\TT{ADRGT} \IFRU{записывает в \Rzero указатель на строку}{writes a pointer to the string} 
\TT{``a>b\textbackslash{}n''}, 
\IFRU{а \TT{BLGT} вызывает}{into \Rzero and \TT{BLGT} calls} \printf.
\IFRU{Следовательно, эти инструкции с суффиксом \TT{-GT}, исполнятся только в том случае, если значение
в \Rzero (там $a$) было больше чем значение в \Rfour (там $b$).}
{Consequently, these instructions with \TT{-GT} suffix, will be executed only in the case when
value in the \Rzero ($a$ is there) was bigger than value in the \Rfour ($b$ is there).}

\index{ARM!\Instructions!ADREQ}
\index{ARM!\Instructions!BLEQ}
\IFRU{Далее мы увидим инструкции \TT{ADREQ} и \TT{BLEQ}.}
{Then we see \TT{ADREQ} and \TT{BLEQ} instructions.}
\IFRU{Они работают так же как и \TT{ADR} и \TT{BL}, но исполнятся только в случае если значения при сравнении были равны.}
{They behave just like \TT{ADR} and \TT{BL} but is to be executed only in the case when operands were equal to each
other while comparison.}
\IFRU{Перед ними еще один \CMP (ведь вызов \printf мог испортить состояние флагов).}
{Another \CMP is before them (since \printf call may tamper state of flags).}

\index{ARM!\Instructions!LDMGEFD}
\index{ARM!\Instructions!LDMFD}
\IFRU{Далее мы увидим \TT{LDMGEFD}, эта инструкция работает так же как и \TT{LDMFD}\footnote{\LDMFDDESC}, 
но сработает только в случае если в результате сравнения одно из значений было больше 
или равно второму}
{Then we see \TT{LDMGEFD}, this instruction works just like \TT{LDMFD}\footnote{\LDMFDDESC},
but will be triggered only in the case when one value was greater or equal to another while comparison}
(\IT{Greater or Equal}).

\IFRU{Смысл инструкции}{The sense of} \TT{``LDMGEFD SP!, \{R4-R6,PC\}''} 
\IFRU{в том, что это как бы эпилог функции, но он сработает только если $a>=b$, только тогда работа 
функции закончится.}
{instruction is that is like function epilogue, but it will be triggered only if $a>=b$, only then function 
execution will be finished.}
\index{Function epilogue}
\IFRU{Но если это не так, то есть $a<b$, то исполнение дойдет до следующей инструкции 
\TT{``LDMFD SP!, \{R4-R6,LR\}''}, это еще один эпилог функции, эта инструкция восстанавливает состояние регистров
\TT{R4-R6}, но и \LR вместо \PC, таким образом, пока что не делая возврата из функции.}
{But if it is not true, i.e., $a<b$, then control flow come to next \TT{``LDMFD SP!, \{R4-R6,LR\}''} instruction,
this is one more function epilogue, this instruction restores \TT{R4-R6} registers state, 
but also \LR instead of \PC, thus, it does not returns from function.}
\IFRU{Последние две инструкции вызывают}{Last two instructions calls} \printf 
\IFRU{со строкой}{with the string} <<a<b\textbackslash{}n>> \IFRU{в качестве единственного аргумента}{as 
sole argument}.
\IFRU{Безусловный переход на \printf вместо возврата из функции, это то что мы уже рассматривали в 
секции}{Unconditional jump to the \printf function instead of function return, is what we already examined in} <<\PrintfSeveralArgumentsSectionName>>\IFRU{, здесь}{ section, here}~(\ref{ARM_B_to_printf}).

\index{ARM!\Instructions!ADRHI}
\index{ARM!\Instructions!BLHI}
\index{ARM!\Instructions!LDMCSFD}
\IFRU{Функция }{}\TT{f\_unsigned} \IFRU{точно такая же, но там используются инструкции}{is likewise,
but } \TT{ADRHI}, \TT{BLHI}, \AndENRU \TT{LDMCSFD} \IFRU{эти предикаты}{instructions are
used there, these predicates}
(\IT{HI = Unsigned higher, CS = Carry Set (greater than or equal)}) 
\IFRU{аналогичны рассмотренным, но служат для работы с беззнаковыми значениями.}
{are analogical to those examined before, but serving for unsigned values.}

\IFRU{В функции \main ничего для нас нового нет:}
{There is not much new in the \main function for us:}

\lstinputlisting[caption=\main]{07_jcc/ARM_O3_main.asm}

\IFRU{Так, в режиме ARM можно обойтись без условных переходов.}
{That's how to get rid of conditional jumps in ARM mode.}

\index{\IFRU{Конвеер RISC}{RISC pipeline}}
\IFRU{Почему это хорошо?}{Why it is so good?}
\IFRU{Потому что ARM это RISC-процессор имеющий конвеер (pipeline) для исполнения инструкций.}
{Since ARM is RISC-processor with pipeline for instructions executing.}
\IFRU{Если говорить 
коротко, то процессору с конвеером тяжело даются переходы вообще, поэтому есть спрос на возможность 
предсказывания переходов.}
{In short, pipelined processor is not very good on jumps at all,
so that is why branch predictor units are
critical here.}
\IFRU{Очень хорошо если программа имеют как можно меньшее переходов, как условных, 
так и безусловных, поэтому, 
инструкции с добавленными предикатами, указывающими,
исполнять инструкцию или нет, могут избавить от некоторого количества условных переходов.}
{It is very good if the program has as few jumps as possible, conditional and unconditional, so that is why,
predicated instructions can help in reducing conditional jumps count.}

\index{x86!\Instructions!CMOVcc}
\IFRU{В x86 нет аналогичной возможности, если не считать инструкцию \TT{CMOVcc}, это то же что и \MOV, 
но она срабатывает
только при определенных выставленных флагах, обычно, выставленных при помощи \CMP во время сравнения.}
{There is no such feature in x86, if not to count \TT{CMOVcc} instruction, it is the same as \MOV,
but triggered only when specific flags are set, usually set while value comparison by \CMP.}

\subsubsection{\OptimizingKeil + \ThumbMode}

\lstinputlisting[caption=\OptimizingKeil + \ThumbMode]{07_jcc/ARM_thumb_signed.asm}

\index{ARM!\Instructions!BLE}
\index{ARM!\Instructions!BNE}
\index{ARM!\Instructions!BGE}
\index{ARM!\Instructions!BLS}
\index{ARM!\Instructions!BCS}
\index{ARM!\Instructions!B}
\index{ARM!\ThumbMode}
\IFRU{В режиме thumb, только инструкции \TT{B} могут быть дополнены условием исполнения (\IT{condition code}), 
так что, код для режима thumb выглядит привычнее.}
{Only \TT{B} instructions in thumb mode may be supplemented by \IT{condition codes}, so the thumb code 
looks more ordinary.}

\TT{BLE} \IFRU{это обычный переход с условием}{is usual conditional jump} \IT{Less than or Equal}, 
\TT{BNE} ~--- \IT{Not Equal}, 
\TT{BGE} ~--- \IT{Greater than or Equal}.

\IFRU{Функция }{}\TT{f\_unsigned} \IFRU{точно такая же, но для работы с беззнаковыми величинами, 
там используются 
инструкции}
{function is just likewise, but other instructions are used while working with unsigned values:}\TT{BLS} 
(\IT{Unsigned lower or same}) \AndENRU \TT{BCS} (\IT{Carry Set (Greater than or equal)}).



\section{\SwitchCaseDefaultSectionName}

\subsection{\IFRU{Если вариантов мало}{Few number of cases}}

\section{\RU{Если вариантов мало}\EN{Few number of cases}}

\lstinputlisting{patterns/08_switch/few.c}

\input{patterns/08_switch/few_x86}
\input{patterns/08_switch/few_ARM}


\subsection{\IFRU{И если много}{A lot of cases}}

\section{\RU{И если много}\EN{A lot of cases}}

\RU{А если ветвлений слишком много, то конечно генерировать слишком длинный код с многочисленными \JE/\JNE 
уже не так удобно.}
\EN{If \TT{switch()} statement contain a lot of case's, it is not very convenient for compiler to emit too large code
with a lot \JE/\JNE instructions.}

\lstinputlisting{patterns/08_switch/lot.c}

\input{patterns/08_switch/lot_x86}
\input{patterns/08_switch/lot_ARM}


\section{\IFRU{Циклы}{Loops}}

\input{loops/loops_x86}

\subsection{ARM}

\subsubsection{\NonOptimizingKeil + режим ARM}

\lstinputlisting{loops/Keil_ARM_O0.asm}

Счетчик итераций \TT{i} будет храниться в регистре \TT{R4}.

Инструкция \TT{``MOV R4, \#2''} просто инициализирует \IT{i}.

Инструкции \TT{``MOV R0, R4''} и \TT{``BL f''} составляют тело цикла, первая инструкция готовит аргумент для
функции f и вторая собственно вызывает её.

Инструкция \TT{``ADD R4, R4, \#1''} прибавляет единицу к \IT{i} при каждой итерации.

\TT{``CMP R4, \#0xA''} сравнивает \TT{i} с $0xA$ ($10$). Следующая за ней инструкция \TT{BLT} 
(\IT{Branch Less Than}) совершит переход, если \IT{i} меньше чем $10$.

В противном случае, в \TT{R0} запишется $0$ (потому что наша функция возвращает 0) и произойдет выход из функции.

\subsubsection{\OptimizingKeil + режим thumb}

\lstinputlisting{loops/Keil_thumb_O3.asm}

Практически, всё то же самое.

\subsubsection{\OptimizingXcode + режим thumb}

\lstinputlisting{loops/xcode_thumb_O3.asm}

На самом деле, в моей функции f было такое:

\begin{lstlisting}
void f(int i)
{
    // do something here
    printf ("%d\n", i);
};
\end{lstlisting}

Так что, LLVM не только \IT{развернул} цикл, но также и представил мою очень простую функцию \TT{f} как inline-вую,
и вставил её тело вместо цикла 8 раз. Это возможно когда функция очень простая, как та что у меня, и когда
она вызывается не очень много раз, как здесь.



\subsection{\IFRU{Еще кое-что}{One more thing}}

По генерируемому коду мы видим следующее: после инициализации \IT{i}, тело цикла не исполняется, а исполняется сразу
проверка условия \IT{i}, а лишь затем исполняется тело цикла. Это правильно. Потому что если условие в самом начале
не выполняется, тело цикла исполнять нельзя. Так может быть, например, в таком случае:

\lstinputlisting{loops/loops_3_ru.c}

Если \IT{total\_entries\_to\_process} равно нулю, тело цикла не должно исполниться ни разу. Поэтому проверка
условия происходит перед тем как исполнить само тело.

Впрочем, оптимизирующий компилятор может переставить проверку условия и тело цикла местами, если компилятор уверен,
что описанная здесь ситуация невозможна, как в случае с нашим примером и компиляторами Keil, MSVC, GCC в режиме
оптимизации.

\section{strlen()}
\index{\CStandardLibrary!strlen()}
\index{\CLanguageElements!while}

\IFRU{Еще немного о циклах. Часто, функция \TT{strlen()}\footnote{подсчет длины строки в Си} 
реализуется при помощи \TT{while()}.}
{Now let's talk about loops one more time. Often, \TT{strlen()} 
function\footnote{counting characters in string in C language} is implemented using \TT{while()} 
statement.}
\IFRU{Например, как это сделано в стандартных библиотеках MSVC:}
{Here is how it's done in MSVC standard libraries:}

\begin{lstlisting}
int strlen (const char * str)
{
        const char *eos = str;

        while( *eos++ ) ;

        return( eos - str - 1 );
}
\end{lstlisting}

\input{10_strlen/strlen_x86}

\subsection{ARM}

\subsubsection{\NonOptimizingXcode + \ARMMode}

\lstinputlisting[caption=\NonOptimizingXcode + \ARMMode]{10_strlen/xcode_ARM_O0_en.asm}

\IFRU{Неоптимизирующий LLVM генерирует слишком много кода, зато на этом примере можно посмотреть, 
как функции работают с локальными переменными в стеке.}
{Non-optimizing LLVM generates too much code, however, here we can see how function works with local variables
in stack.}
\IFRU{В нашей функции только локальных переменных две, это два указателя}{There are only two
local variables in our function}, \IT{eos} \IFRU{и}{and} \IT{str}.

\IFRU{В этом листинге}{In this listing}, \IFRU{сгенерированном при помощи}{generated by} \IDA, 
\IFRU{я переименовал}{I renamed} \IT{var\_8} \IFRU{и}{and} \IT{var\_4} \IFRU{в}{into} \IT{eos} 
\IFRU{и}{and} \IT{str} \IFRU{вручную}{manually}.

\IFRU{Итак, первые несколько инструкций просто сохраняют входное значение в переменных}{So, 
first instructions are just saves input value in} \IT{str} \IFRU{и}{and} \IT{eos}.

\IFRU{Начиная с метки}{Loop body is beginning at} \IT{loc\_2CB8}\IFRU{, начинается тело цикла}{ label}.

\IFRU{Первые три инструкции в теле цикла}{First three instruction in loop body} (\TT{LDR}, \ADD, \TT{STR}) 
\IFRU{загружают значение}{loads} \IT{eos} \IFRU{в}{value into} \Rzero, 
\IFRU{затем происходит инкремент значения и оно сохраняется назад в локальной переменной \IT{eos} расположенной 
в стеке.}{then value is incremented and it's saving back into \IT{eos} local variable located in stack.}

\IFRU{Следующая инструкция}{The next} \TT{``LDRSB R0, [R0]''} (\IT{Load Register Signed Byte}) 
\IFRU{загружает байт из памяти по адресу \Rzero, расширяет его до 32-бит считая его знаковым (signed) 
и сохраняет в \Rzero}{instruction loading byte from memory at \Rzero address and sign-extends it to 32-bit}.
\IFRU{Это немного похоже на инструкцию}{This is similar to} \MOVSX \IFRU{в}{instruction in} x86.
\IFRU{Компилятор считает этот байт знаковым (signed), потому что тип \Tchar по стандарту Си ~--- знаковый.}
{The compiler treating this byte as signed because \Tchar type in C standard is signed.}
\IFRU{Об это я уже немного писал}{I already wrote about it}~\ref{MOVSX} \IFRU{в этой же секции, 
но посвященной x86}{in this section, but related to x86}.

\IFRU{Следует также заметить, что, в ARM нет возможности использовать 8-битную или 16-битную часть регистра, 
как это возможно в x86.}
{It's should be noted, there are to way to use 8-bit part or 16-bit part of 32-bit register in ARM, as it's
possible in x86.}
\IFRU{Вероятно, это связано с тем что за x86 тянется длинный шлейф совместимости со своими предками, такими как
16-битный 8086 и даже 8-битный 8080, а ARM разрабатывался с чистого листа как 32-битный RISC-процессор.}
{Apparently, it's because x86 has a huge history of compatibility with its ancestors like 16-bit 8086 
and even 8-bit 8080,
but ARM was developed from scratch as 32-bit RISC-processor.}
\IFRU{Следовательно, чтобы работать с отдельными байтами на ARM, так или иначе, придется использовать 
32-битные регистры.}
{Consequently, in order to process separate bytes in ARM, one have to use 32-bit registers anyway.}

\IFRU{Итак}{So}, \TT{LDRSB} \IFRU{загружает символ из строки в \Rzero, по одному.}{loads symbol from string
into \Rzero, one by one.}
\IFRU{Следующие инструкции}{Next} \CMP \IFRU{и}{and} \TT{BEQ} \IFRU{проверяют, является ли этот символ нулем.}
{instructions checks, if loaded symbol is zero.}
\IFRU{Если не ноль, то происходит переход на начало тела цикла.}{If not zero, control passing to loop body
begin.}
\IFRU{А если ноль, выходим из цикла.}{And if zero, loop is finishing.}

\IFRU{В конце функции вычисляется разница между}{At the end of function, a difference between} 
\IT{eos} \IFRU{и}{and} \IT{str}\IFRU{, вычитается еще единица и вычисленное 
значение возвращается через \Rzero.}{ is calculated, 1 is also subtracting, and resulting value is returned
via \Rzero.}

\IFRU{Кстати, обратите внимание, в этой функции не сохранялись регистры.}{By the way, please note, registers
wasn't saved in this function.}
\IFRU{Это потому что, по стандарту, регистры \Rzero-\Rthree называются также ``scratch registers'',
они предназначены для передачи аргументов, 
их значения не нужно восстанавливать при выходе из функции, потому что они больше не нужны в вызывающей функции.
Таким образом, их можно использовать как захочется}
{That's because by ARM calling convention, \Rzero-\Rthree registers are ``scratch registers'', 
they are intended for arguments passing,
its values may not be restored upon function exit, because calling function will not use them anymore.
Consequently, they may be used for anything we want.}
\IFRU{А так как никакие больше регистры не используются, то и сохранять нечего.}
{Other registers are not used here, so that's why we have nothing to save in stack.}
\IFRU{Поэтому, управление можно вернуть назад вызывающей функции 
простым переходом (\TT{BX}), по адресу в регистре \LR.}
{Thus, control may be returned back to calling function by simple jump (\TT{BX}), to address in \LR register.}

%\subsubsection{\NonOptimizingXcode + режим thumb}
%Практически, точно такой же код.

\subsubsection{\OptimizingXcode + \ThumbMode}

\lstinputlisting[caption=\OptimizingXcode + \ThumbMode]{10_strlen/xcode_thumb_O3.asm}

\IFRU{Оптимизирующий LLVM решил что под переменные \IT{eos} и \IT{str} выделять место в стеке не обязательно}
{As optimizing LLVM concludes, place in stack for \IT{eos} and \IT{str} may not be allocated},
\IFRU{и эти переменные можно хранить прямо в регистрах.}
{and these variables may always be stored right in registers.}
\IFRU{Перед началом тела цикла}{Before loop body beginning}, \IT{str} \IFRU{будет находиться в}{will always be in} 
\Rzero, \IFRU{а}{and} \IT{eos} ~--- \IFRU{в}{in} \Rone.

\IFRU{Инструкция }{}\TT{``LDRB.W R2, [R1],\#1''} \IFRU{загружает в \Rtwo байт из памяти по адресу \Rone, 
расширяя его как знаковый (signed), до 32-битного
значения, но не только это.}
{instruction loads byte from memory at the address \Rone into \Rtwo, sign-extending it to 32-bit value, but not
only that.}
\TT{\#1} \IFRU{в конце инструкции называется}{at the instruction's end calling} ``Post-indexed addressing'', 
\IFRU{это значит что после загрузки байта, к \Rone добавится единица.}{this mean, $1$ is to be added
to \Rone after byte load.}
\IFRU{Это очень удобно для работы с массивами.}{That's handy when accessing arrays.}

\IFRU{Такого режима адресации в x86 нет, но он есть в некоторых других процессорах, даже на PDP-11.}
{There are no such addressing mode in x86, but it's present in some other processors, even on PDP-11.}
\IFRU{Существует байка, что режимы пре-инкремента, пост-инкремента, 
пре-декремента и пост-декремента адреса в PDP-11}
{There is a legend that pre-increment, post-increment, pre-decrement and post-decrement modes in PDP-11},
\IFRU{были ``виновны'' в появлении таких конструктов языка Си (который разрабатывался на PDP-11) как}
{were ``guilty'' in appearance such C language (which developed on PDP-11) constructs as}
*ptr++, *++ptr, *ptr-{}-, *-{}-ptr. 
\IFRU{Кстати, это является труднозапоминаемой особенностью в Си.}
{By the way, this is one of hard to memorize C feature.}
\IFRU{Дела обстоят так:}{This is how it is:}

\begin{center}
\begin{tabular}{ | l | l | l | l | }
\hline                        
\cellcolor{blue!25} \IFRU{термин в Си}{C term} & 
\cellcolor{blue!25} \IFRU{термин в ARM}{ARM term} & 
\cellcolor{blue!25} \IFRU{выражение Си}{C statement} & 
\cellcolor{blue!25}\IFRU{как это работает}{how it works} \\
\hline                        
\IFRU{Пост-инкремент}{Post-increment} & 
post-indexed addressing & 
\TT{*ptr++} & 
\IFRU{использовать значение \TT{*ptr}}{use \TT{*ptr} value}, \\
& & & \IFRU{затем икремент указателя \TT{ptr}}{then increment \TT{ptr} pointer} \\
\hline                        
\IFRU{Пост-декремент}{Post-decrement} & 
post-indexed addressing & 
\TT{*ptr-{}-} & 
\IFRU{использовать значение \TT{*ptr}}{use \TT{*ptr} value}, \\
& & & \IFRU{затем декремент указателя \TT{ptr}}{then decrement \TT{ptr} pointer} \\
\hline                        
\IFRU{Пре-инкремент}{Pre-increment} & 
pre-indexed addressing & 
\TT{*++ptr} & 
\IFRU{инкремент указателя \TT{ptr}}{increment \TT{ptr} pointer}, \\
& & & \IFRU{затем использовать значение \TT{*ptr}}{then use \TT{*ptr} value} \\
\hline                        
\IFRU{Пре-декремент}{Pre-decrement} & 
post-indexed addressing & 
\TT{*-{}-ptr} & 
\IFRU{декремент указателя \TT{ptr}}{decrement \TT{ptr} pointer}, \\
& & & \IFRU{затем использовать значение \TT{*ptr}}{then use \TT{*ptr} value} \\
\hline  
\end{tabular}
\end{center}

\IFRU{Деннис Ритчи (один из создателей ЯП Си) указывал, что, это, вероятно, придумал Кен Томпсон 
(еще один создатель Си),
потому что подобная возможность процессора имелась еще в PDP-7}
{Dennis Ritchie (one of C language creators) mentioned that it's, probably, was invented by Ken Tompson
(another C creator) because this processor feature was present in PDP-7}
\cite{Ritchie:1986}\cite{Ritchie:1993:DCL:155360.155580}.
\IFRU{Таким образом, компиляторы с ЯП Си на тот процессор, где это есть, могут использовать это.}
{Thus, C language compilers may use it, if it's present in targer processor.}

Итак, далее в теле цикла \CMP и \TT{BNE} продолжают работу цикла, до тех пор, пока не будет встречен $0$.

После конца цикла \TT{MVNS}\footnote{MoVe Not} (инвертирование всех бит в значении, аналог \NOT на x86) 
и \ADD вычисляют $eos - str - 1$. 
На самом деле, эти две инструкции вычисляют $R0 = ~str + eos$, что эквивалентно тому, что было в исходном коде, 
а почему это так, я уже описывал чуть раньше, здесь~\ref{strlen_NOT_ADD}. 
Вероятно, LLVM, как и GCC, посчитал что так будет короче, или быстрее.

%\subsubsection{\OptimizingXcode + \ARMMode}
%Практически, точно такой же код.

\subsubsection{\OptimizingKeil{} + \ARMMode}

\lstinputlisting[caption=\OptimizingKeil + \ARMMode]{10_strlen/Keil_ARM_O3.asm}

Практически то же самое что мы уже видели, за тем исключением что выражение $str - eos - 1$ может быть вычислено
не в самом конце функции, а прямо в теле цикла. 
Суффикс \TT{-EQ}, как мы помним, означает что инструкция будет выполнена только
если операнды в исполненной перед этим инструкции \CMP были равны. 
Таким образом, если в \Rzero будет $0$, обе инструкции \TT{SUBEQ} исполнятся и результат останется в \Rzero.



\section{\DivisionByNineSectionName}
\label{sec:divisionbynine}

\IFRU{Простая функция:}{Very simple function:}

\begin{lstlisting}
int f(int a)
{
	return a/9;
};
\end{lstlisting}

\IFRU{Компилируется вполне предсказуемо:}{Is compiled in a very predictable way:}

\lstinputlisting{\IFRU{11_division_by_9/11_1_msvc_ru.asm}{11_division_by_9/11_1_msvc_en.asm}}

\IFRU{\IDIV делит 64-битное число хранящееся в паре регистров \TT{EDX:EAX} на значение в \ECX. 
В результате, \EAX будет содержать частное\footnote{результат деления}, а \EDX ~--- остаток от деления. 
Результат возвращается из функции через \EAX, так что после операции деления, 
это значение не перекладывается больше никуда, 
оно уже там где надо.}
{\IDIV divides 64-bit number stored in \TT{EDX:EAX} register pair by value in \ECX register. 
As a result, \EAX will contain quotient\footnote{result of division}, and \EDX ~--- remainder.
Result is returning from f() function in \EAX register, so, that value is not moved anymore after division 
operation, it is in right place already.}
\IFRU
{Из-за того что \IDIV требует пару регистров \TT{EDX:EAX}, то перед этим инструкция \TT{CDQ} 
расширяет \EAX до 64-битного значения учитывая знак, также как это делает \MOVSX.}
{Because \IDIV require value in \TT{EDX:EAX} register pair, \TT{CDQ} instruction (before \IDIV) extending 
\EAX value to 64-bit value taking value sign into account, just as \MOVSX does.}
\IFRU{Со включенной оптимизацией (\Ox) получается:}
{If we turn optimization on (\Ox), we got:}

\lstinputlisting{11_division_by_9/11_1_msvc_Ox.asm}

\newcommand{\URLMSDN}{\href{http://blogs.msdn.com/b/devdev/archive/2005/12/12/502980.aspx}
{MSDN: Integer division by constants}}
\newcommand{\URLN}{http://www.nynaeve.net/?p=115}

\IFRU{Это ~--- деление через умножение. Умножение конечно быстрее работает. 
Поэтому можно используя этот трюк
\footnote{Читайте подробнее о делении через умножение в книге 
``Генри Уоррен, мл. ~--- Алгоритмические трюки для программистов'' 
(глава 10 ~--- ``Целое деление на константы''): \URLMSDN, \url{\URLN}} 
создать код эквивалентный тому что мы хотим и работающий быстрее.}
{This is ~--- division using multiplication. Multiplication operation working much faster. 
And it is possible to use that trick
\footnote{Read more about division by multiplication in 
``Henry S. Warren Jr. ~--- Hacker's Delight'' book (chapter 10 ~--- ``Integer Division By Constants'') 
and: \URLMSDN, \url{\URLN}} 
to produce a code which is equivalent and faster.}
\IFRU
{GCC 4.4.1 даже без включенной оптимизации генерит примерно такой же код как и MSVC с оптимизацией:}
{GCC 4.4.1 even without optimization turned on, generate almost the same code as MSVC with optimization turned on:}

\lstinputlisting{11_division_by_9/11_2_gcc.asm}

\subsection{ARM}

\IFRU{В процессоре ARM, как и во многих других ``чистых'' (pure) RISC-процессорах нет инструкции деления,
да и инструкции умножения на 32-битную константу также нет.}{In ARM processor, just like in any other ''pure'' 
RISC-processors, there are no division instruction, instruction for multiplication by 32-bit constant 
is absent too.}
\IFRU{При некотором желании, можно обойтись только тремя действиями: сложением, вычитанием и 
битовыми сдвигами}{With some effort, it's possible to do division using only three instructions: addition,
subtraction and bit shifts}~\ref{sec:bitfields}.

\IFRU{Пример деления 32-битного числа на 10 из книги}{Here is an example of 32-bit number division by 10 from the} 
ARM Cookbook (1994)\footnote{\href{http://yurichev.com/ref/ARM\%20Cookbook\%20(1994)/cook3.txt}{ARM Cookbook (1994)}}. 
\IFRU{На выходе и частное и остаток}{Quotient and remainder on output}.

\begin{lstlisting}
; takes argument in a1
; returns quotient in a1, remainder in a2
; cycles could be saved if only divide or remainder is required
    SUB    a2, a1, #10             ; keep (x-10) for later
    SUB    a1, a1, a1, lsr #2
    ADD    a1, a1, a1, lsr #4
    ADD    a1, a1, a1, lsr #8
    ADD    a1, a1, a1, lsr #16
    MOV    a1, a1, lsr #3
    ADD    a3, a1, a1, asl #2
    SUBS   a2, a2, a3, asl #1      ; calc (x-10) - (x/10)*10
    ADDPL  a1, a1, #1              ; fix-up quotient
    ADDMI  a2, a2, #10             ; fix-up remainder
    MOV    pc, lr
\end{lstlisting}

\subsubsection{\OptimizingXcode + \ARMMode}

\begin{lstlisting}
__text:00002C58                         _f
__text:00002C58 39 1E 08 E3 E3 18 43 E3                 MOV             R1, 0x38E38E39
__text:00002C60 10 F1 50 E7                             SMMUL           R0, R0, R1
__text:00002C64 C0 10 A0 E1                             MOV             R1, R0,ASR#1
__text:00002C68 A0 0F 81 E0                             ADD             R0, R1, R0,LSR#31
__text:00002C6C 1E FF 2F E1                             BX              LR
\end{lstlisting}

Этот код почти тот же, что сгенерирован MSVC и GCC в режиме оптимизации. Должно быть, LLVM использует тот же
алгоритм для поиска констант.

Наблюдательный читатель может спросить, как \MOV записала в регистр сразу 32-битное число, ведь это невозможно.
Действительно невозможно, но как мы видим, здесь на инструкцию 8 байт вместо стандартных 4-х.
Первая инструкция загружает в младшие 16 бит регистра значение $0x8E39$, а вторая инструкция, 
на самом деле \TT{MOVT},
загружающая в старшие 16 бит регистра значение $0x383E$. \IDA распознала эту последовательность и для удобства
выдала только одну инструкцию.

Инструкция \TT{SMMUL} (\IT{Signed Most Significant Word Multiply}) умножает числа считая их знаковыми (signed)
и оставляет в R0 старшие 32 бита результата, не сохраняя младшие 32 бита.

Инструкция \TT{``MOV R1, R0,ASR\#1''} это арифметический сдвиг право на один бит.

\TT{``ADD R0, R1, R0,LSR\#31''} это $R0=R1 + R0>>31$

Дело в том что в режиме ARM нет отдельных инструкций для битовых сдвигов. 
Вместо этого, некоторые инструкции (\MOV, \ADD,
\SUB, \TT{RSB}) могут быть
дополнеты пометкой, сдвигать ли второй операнд и если да, то на сколько и как. 
\TT{ASR} означает \IT{Arithmetic Shift Right}, \TT{LSR} ~--- \IT{Logican Shift Right}.

\subsubsection{\OptimizingXcode + режим thumb}

\begin{lstlisting}
MOV             R1, 0x38E38E39
SMMUL.W         R0, R0, R1
ASRS            R1, R0, #1
ADD.W           R0, R1, R0,LSR#31
BX              LR
\end{lstlisting}

В режиме thumb отдельные инструкции для битовых сдвигов есть, и здесь применяется одна из них ~--- ASRS 
(арифметический сдвиг вправо).

\subsubsection{Неоптимизирующие Xcode (LLVM) и Keil}

Неоптимизирующий LLVM не занимается генерацией подобного кода а вместо этого просто вставляет вызов
библиотечной функции \IT{\_\_\_divsi3}. А Keil во всех случаях вставляет вызов функции \IT{\_\_aeabi\_idivmod}.


\input{12_FPU/FPU}
\section{\IFRU{Массивы}{Arrays}}
\label{arrays}

\IFRU{Массив это просто набор переменных в памяти, обязательно лежащих рядом, и обязательно одного типа.}
{Array is just a set of variables in memory, always lying next to each other, always has same type.}

\subsection{\IFRU{Простой пример}{Simple example}}

\lstinputlisting{arrays/simple.c}

\input{arrays/simple_x86}

\input{arrays/simple_ARM}


\section{\RU{Переполнение буфера}\EN{Buffer overflow}}
\label{subsec:bufferoverflow}
\index{\BufferOverflow}

\RU{Итак, индексация массива ~--- это просто \IT{массив\lbrack{}индекс\rbrack}.  % TODO как-то плохо отображаются []
Если вы присмотритесь к коду, в цикле печати значений массива через \printf вы 
не увидите проверок индекса, \IT{меньше ли он двадцати?} 
А что будет если он будет больше двадцати? 
Эта одна из особенностей \CCpp, за которую их, собственно, и ругают.}
\EN{So, array indexing is just \IT{array\lbrack{}index\rbrack}.
If you study generated code closely, you'll probably note missing index bounds checking,
which could check index, \IT{if it is less than 20}.
What if index will be greater than 20?
That's the one \CCpp feature it is often blamed for.}

\RU{Вот код который и компилируется и работает:}
\EN{Here is a code successfully compiling and working:}

\begin{lstlisting}
#include <stdio.h>

int main() 
{
	int a[20];
	int i;

	for (i=0; i<20; i++)
		a[i]=i*2;

	printf ("a[100]=%d\n", a[100]);

	return 0;
};
\end{lstlisting}

\RU{Вот в это}\EN{Compilation results} (MSVC 2010):

\lstinputlisting{patterns/13_arrays/BO2_msvc.asm}

\RU{У меня оно при запуске выдало вот это:}\EN{I'm running it, and I got:}

\begin{lstlisting}
a[100]=760826203
\end{lstlisting}

\RU{Это просто \IT{что-то}, что волею случая лежало в стеке рядом с массивом, 
через 400 байт от его первого элемента.}
\EN{It is just \IT{something}, occasionally lying in the stack near to array, 400 bytes from its first element.}

\RU{Действительно, а как могло бы быть иначе? Компилятор мог бы встроить какой-то код, 
каждый раз проверяющий индекс на соответствие пределам массива, как в языках программирования 
более высокого уровня\footnote{Java, Python, и т.д.}, что делало бы запускаемый код медленнее.}
\EN{Indeed, how it could be done differently?
Compiler may generate some additional code for checking index value to be always
in array's bound (like in higher-level programming languages\footnote{Java, Python, etc})
but this makes running code slower.}

\RU{Итак, мы прочитали какое-то число из стека явно \IT{нелегально}, а что если мы запишем?}
\EN{OK, we read some values from the stack \IT{illegally} but what if we could write something to it?}

\RU{Вот что мы пишем:}\EN{Here is what we will write:}

\begin{lstlisting}
#include <stdio.h>

int main() 
{
	int a[20];
	int i;

	for (i=0; i<30; i++)
		a[i]=i;

	return 0;
};
\end{lstlisting}

\RU{И вот что имеем на ассемблере:}\EN{And what we've got:}

\lstinputlisting{patterns/13_arrays/BO_\LANG.asm}

\RU{Запускаете скомпилированную программу, и она падает. Немудрено. Но давайте теперь узнаем, где именно.}
\EN{Run compiled program and its crashing. No wonder. Let's see, where exactly it is crashing.}

\index{tracer}
\RU{Отладчик я уже давно не использую, так как надоело для всяких мелких задач вроде подсмотреть состояние 
регистров, запускать что-то, двигать мышью, и т.д. 
Поэтому я написал очень минималистическую утилиту для себя, \tracer, коей обхожусь.}
\EN{I'm not using debugger anymore since I tried to run it each time, move mouse, etc, when I need just to
spot a register's state at the specific point.
That's why I wrote very minimalistic tool for myself, \tracer, which is enough for my tasks.}

\RU{Помимо всего прочего, я могу использовать мою утилиту просто чтобы посмотреть 
где и какое исключение произошло. 
Итак, пробую:}
\EN{I can also use it just to see, where \gls{debuggee} is crashed.
So let's see:}

\begin{lstlisting}
generic tracer 0.4 (WIN32), http://conus.info/gt

New process: C:\PRJ\...\1.exe, PID=7988
EXCEPTION_ACCESS_VIOLATION: 0x15 (<symbol (0x15) is in unknown module>), ExceptionInformation[0]=8
EAX=0x00000000 EBX=0x7EFDE000 ECX=0x0000001D EDX=0x0000001D
ESI=0x00000000 EDI=0x00000000 EBP=0x00000014 ESP=0x0018FF48
EIP=0x00000015
FLAGS=PF ZF IF RF
PID=7988|Process exit, return code -1073740791
\end{lstlisting}

\RU{Итак, следите внимательно за регистрами.}
\EN{Now please keep your eyes on registers.}

\RU{Исключение произошло по адресу 0x15. Это явно нелегальный адрес для кода ~--- по крайней мере, win32-кода! 
Мы там как-то очутились, причем, сами того не хотели. Интересен также тот факт, что в \EBP хранится 0x14, 
а в \ECX и \EDX ~--- 0x1D.}
\EN{Exception occurred at address 0x15. It is not legal address for code~---at least for win32 code!
We trapped there somehow against our will.
It is also interesting fact the \EBP register contain 0x14,
\ECX and \EDX{}~---0x1D.}

\RU{И еще немного изучим разметку стека.}\EN{Let's study stack layout more.}

\RU{После того как управление передалось в \main, в стек было сохранено значение \EBP. 
Затем, для массива + переменной \IT{i} было выделено $84$ байта. Это \TT{(20+1)*sizeof(int)}. 
\ESP сейчас указывает на переменную \TT{\_i} в локальном стеке и при исполнении следующего \TT{PUSH что-либо}, 
\IT{что-либо} появится рядом с \TT{\_i}.}
\EN{After control flow was passed into \TT{\main}, the value in the \EBP register was saved on the stack.
Then, $84$ bytes was allocated for array and \IT{i} variable.
That's \TT{(20+1)*sizeof(int)}.
The \ESP pointing now to the \TT{\_i} variable in the local stack and after execution of next \TT{PUSH something},
\IT{something} will be appeared next to \TT{\_i}.}

\RU{Вот так выглядит разметка стека пока управление находится внутри}
\EN{That's stack layout while control is inside} \main:

\begin{center}
\begin{tabular}{ | l | l | }
\hline
  \TT{ESP}    & \RU{4 байта для \IT{i}}\EN{4 bytes for \IT{i}} \\
  \TT{ESP+4}  & \RU{80 байт для массива \TT{a[20]}}\EN{80 bytes for \TT{a[20]} array} \\
  \TT{ESP+84} & \RU{сохраненное значение \EBP}\EN{saved \EBP value} \\
  \TT{ESP+88} & \RU{адрес возврата}\EN{returning address} \\
\hline
\end{tabular}
\end{center}

\RU{Команда \TT{a[19]=чего\_нибудь} записывает последний \Tint в пределах массива (пока что в пределах!)}
\EN{Instruction \TT{a[19]=something} writes last \Tint in array bounds (in bounds so far!)}

\RU{Команда \TT{a[20]=чего\_нибудь} записывает \IT{чего\_нибудь} на место где сохранено значение \EBP.}
\EN{Instruction \TT{a[20]=something} writes \IT{something} to the place where value from the \EBP is saved.}

\RU{Обратите внимание на состояние регистров на момент падения процесса. В нашем случае, 
в 20-й элемент записалось значение 20. 
И вот все дело в том, что заканчиваясь, эпилог функции восстанавливал значение \EBP. 
(20 в десятичной системе это как раз 0x14 в шестнадцатеричной). 
Далее выполнилась инструкция \RET, которая на самом деле эквивалентна \TT{POP EIP}.}
\EN{Please take a look at registers state at the crash moment. In our case,
number 20 was written to 20th element. 
By the function ending, function epilogue restores original \EBP value.
(20 in decimal system is 0x14 in hexadecimal).
Then, \RET instruction was executed, which is effectively equivalent to \TT{POP EIP} instruction.}

\RU{Инструкция \RET вытащила из стека адрес возврата (это адрес где-то внутри \ac{CRT}), 
которая вызвала \main), 
а там было записано 21 в десятичной системе, то есть 0x15 в шестнадцатеричной. 
И вот процессор оказался по адресу 0x15, но исполняемого кода там нет, так что случилось исключение.}
\EN{\RET instruction taking returning address from the stack (that is the address inside of \ac{CRT}),
which was called \main),
and 21 was stored there (0x15 in hexadecimal).
The CPU trapped at the address 0x15,
but there is no executable code, so exception was raised.}

\index{\RU{Переполнение буфера}\EN{Buffer overflow}}
\RU{Добро пожаловать! Это называется}
\EN{Welcome! It is called} \IT{buffer overflow}\footnote{\url{http://en.wikipedia.org/wiki/Stack_buffer_overflow}}.

\RU{Замените массив \Tint на строку (массив \Tchar), нарочно создайте слишком длинную строку, 
просуньте её в ту программу, 
в ту функцию, которая не проверяя длину строки скопирует её в слишком короткий буфер, 
и вы сможете указать программе, по какому именно адресу перейти. 
Не все так просто в реальности, конечно, но началось все с этого
\footnote{Классическая статья об этом: \cite{Phrack4914}}.}
\EN{Replace \Tint array by string (\Tchar array), create a long string deliberately,
and pass it to the program, to the function which is not checking string length and copies it to short buffer,
and you'll able to point to a program an address to which it must jump.
Not that simple in reality, but that is how it was emerged
\footnote{Classic article about it: \cite{Phrack4914}.}}

\RU{Попробуем то же самое в GCC 4.4.1. У нас выходит такое:}\EN{Let's try the same code in GCC 4.4.1. We got:}

\lstinputlisting{patterns/13_arrays/BO2_gcc.asm}

\RU{Запуск этого в Linux выдаст:}\EN{Running this in Linux will produce:} \TT{Segmentation fault}.

\index{GDB}
\RU{Если запустить полученное в отладчике GDB, получим:}
\EN{If we run this in GDB debugger, we getting this:}

\begin{lstlisting}
(gdb) r
Starting program: /home/dennis/RE/1 

Program received signal SIGSEGV, Segmentation fault.
0x00000016 in ?? ()
(gdb) info registers
eax            0x0	0
ecx            0xd2f96388	-755407992
edx            0x1d	29
ebx            0x26eff4	2551796
esp            0xbffff4b0	0xbffff4b0
ebp            0x15	0x15
esi            0x0	0
edi            0x0	0
eip            0x16	0x16
eflags         0x10202	[ IF RF ]
cs             0x73	115
ss             0x7b	123
ds             0x7b	123
es             0x7b	123
fs             0x0	0
gs             0x33	51
(gdb) 
\end{lstlisting}

\RU{Значения регистров немного другие чем в примере win32, это потому что разметка стека чуть другая.}
\EN{Register values are slightly different then in win32 example
since stack layout is slightly different too.}

\subsection{\IFRU{Защита от переполнения буфера}{Buffer overflow protection methods}}

\newcommand{\URLWPB}{\href{http://en.wikipedia.org/wiki/Buffer_overflow_protection}
{Wikipedia: \IFRU{описания защит, которые компилятор может вставлять в код}
{compiler-side buffer overflow protection methods}}}

\IFRU{В наше время пытаются бороться с этой напастью, не взирая на халатность программистов на \CCpp. 
В MSVC есть опции вроде\footnote{\URLWPB}:}
{There are several methods to protect against it, regardless of \CCpp programmers' negligence.
MSVC has options like\footnote{\URLWPB}:}

\begin{verbatim}
 /RTCs Stack Frame runtime checking
 /GZ Enable stack checks (/RTCs)
\end{verbatim}

\IFRU{Один из методов, это в прологе функции вставлять в область локальных переменных 
некоторое случайное значение 
и в эпилоге функции, перед выходом, это число проверять. 
И если проверка не прошла, то не выполнять инструкцию \RET а остановиться (или зависнуть). 
Процесс зависнет, но это лучше чем удаленная атака на ваш хост.}
{One of the methods is to write random value among local variables to stack at function prologue 
and to check it in function epilogue before function exiting.
And if value is not the same, do not execute last instruction \RET, but halt (or hang).
Process will hang, but that's much better then remote attack to your host.}

\subsection{\IFRU{Еще немного о массивах}{One more word about arrays}}

\IFRU{Теперь понятно, почему нельзя написать в исходном коде на \CCpp что-то вроде:
\footnote{GCC способен это сделать выделяя место под массив динамически в стеке (как alloca()~\ref{alloca}), 
но это расширение не является частью стандарта}}
{Now we understand, why it's not possible to write something like that in \CCpp code
\footnote{GCC can actually do this by allocating array dynammically in stack (like alloca()~\ref{alloca}), 
but it's not standard langauge extension}:}

\begin{lstlisting}
void f(int size)
{
    int a[size];
...
};
\end{lstlisting}

\IFRU{Все просто потому, чтобы выделять место под массив в локальном стеке или же сегменте данных 
(если массив глобальный), компилятору нужно знать его размер, чего он, на стадии компиляции, 
разумеется знать не может.}
{That's just because compiler should know exact array size to allocate place for it in local stack layout or
in data segment (in case of global variable) on compiling stage.}

\IFRU{Если вам нужен массив произвольной длины, то выделите столько, сколько нужно, через \TT{malloc()}, 
затем обращайтесь к выделенному блоку байт как к массиву того типа, который вам нужен.}
{If you need array of arbitrary size, allocate it by \TT{malloc()}, then access allocated memory block
as array of variables of type you need.}

\subsection{\IFRU{Многомерные массивы}{Multidimensional arrays}}

\IFRU{Многомерный массив выглядит внутри так же как и линейный.}
{Internally, multidimensional array is essentially the same thing as linear array.}

\IFRU{Попробуем:}{Let's see:}

\begin{lstlisting}
#include <stdio.h>

int a[10][20][30];

void insert(int x, int y, int z, int value)
{
	a[x][y][z]=value;
};
\end{lstlisting}

\IFRU{В итоге}{We got} (MSVC 2010):

\lstinputlisting{arrays/13_5_msvc.asm}

\IFRU{В принципе, ничего удивительного. В \TT{insert()} для индексирования нужного элемента массива, 
три входных аргумента перемножаются так, чтобы представить массив трехмерным.}
{Nothing special. For index calculation, three input arguments are multiplying 
in such way to represent array as multidimensional.}

GCC 4.4.1:

\lstinputlisting{arrays/13_5_gcc.asm}


\subsection{\IFRU{Работа с битовыми полями в структуре}{Bit fields in structure}}

\subsubsection{\IFRU{Пример CPUID}{CPUID example}}

\IFRU{Язык \CCpp позволяет указывать, сколько именно бит отвести для каждого поля структуры. 
Это удобно если нужно экономить место в памяти. К примеру, для переменной типа \Tbool достаточно одного бита.
Но, это не очень удобно, если нужна скорость.}
{\CCpp language allow to define exact number of bits for each structure fields.
It is very useful if one needs to save memory space. 
For example, one bit is enough for variable of \Tbool type.
But of course, it is not rational if speed is important.}

\newcommand{\FNCPUID}{\footnote{\url{http://en.wikipedia.org/wiki/CPUID}}}

\index{x86!\Instructions!CPUID}
\label{cpuid}
\IFRU{Рассмотрим пример с инструкцией \CPUID\FNCPUID. 
Эта инструкция возвращает информацию о том, какой процессор имеется в наличии и какие возможности он имеет.}
{Let's consider \CPUID\FNCPUID instruction example.
This instruction returning information about current CPU and its features.}

\IFRU{Если перед исполнением инструкции в \EAX будет 1, 
то \CPUID вернет упакованную в \EAX такую информацию о процессоре:}
{If the \EAX is set to 1 before instruction execution, 
\CPUID will return this information packed into the \EAX register:}

\begin{center}
\begin{tabular}{ | l | l | }
\hline
3:0 & Stepping \\
7:4 & Model \\
11:8 & Family \\
13:12 & Processor Type \\
19:16 & Extended Model \\
27:20 & Extended Family \\
\hline
\end{tabular}
\end{center}

\newcommand{\FNGCCAS}{\footnote{\href{http://www.ibiblio.org/gferg/ldp/GCC-Inline-Assembly-HOWTO.html}
{\IFRU{Подробнее о встроенном ассемблере GCC}{More about internal GCC assembler}}}}

\IFRU{MSVC 2010 имеет макрос для \CPUID, а GCC 4.4.1 ~--- нет. 
Поэтому для GCC сделаем эту функцию сами, используя его встроенный ассемблер\FNGCCAS.}
{MSVC 2010 has \CPUID macro, but GCC 4.4.1~---has not.
So let's make this function by yourself for GCC with the help of its built-in assembler\FNGCCAS.}

\lstinputlisting{patterns/15_structs/CPUID.c}

\IFRU{После того как \CPUID заполнит \EAX/\EBX/\ECX/\EDX, у нас они отразятся в массиве \TT{b[]}. 
Затем, мы имеем указатель на структуру \TT{CPUID\_1\_EAX}, и мы указываем его на значение 
\EAX из массива \TT{b[]}.}
{After \CPUID will fill \EAX/\EBX/\ECX/\EDX, these registers will be reflected in the \TT{b[]} array.
Then, we have a pointer to the \TT{CPUID\_1\_EAX} structure and we point it to the value in the \EAX from \TT{b[]} array.}

\IFRU{Иными словами, мы трактуем 32-битный \Tint как структуру.}
{In other words, we treat 32-bit \Tint value as a structure.}

\IFRU{Затем мы читаем из структуры.}{Then we read from the stucture.}

\IFRU{Компилируем в MSVC 2008 с опцией \Ox}{Let's compile it in MSVC 2008 with \Ox option}:

\lstinputlisting[caption=\Optimizing MSVC 2008]{patterns/15_structs/CPUID_msvc_Ox.asm}

\index{x86!\Instructions!SHR}
\IFRU{Инструкция \TT{SHR} сдвигает значение из \EAX на то количество бит, 
которое нужно \IT{пропустить}, то есть, мы игнорируем некоторые биты \IT{справа}.}
{\TT{SHR} instruction shifting value in the \EAX register by number of bits must be
\IT{skipped}, e.g., we ignore a bits \IT{at right}.}

\index{x86!\Instructions!AND}
\IFRU{А инструкция \ANDIns очищает биты \IT{слева} которые нам не нужны, или же, говоря иначе, 
она оставляет по маске только те биты в \EAX, которые нам сейчас нужны.}
{\ANDIns instruction clears bits not needed \IT{at left}, or, in other words, 
leaves only those bits in the \EAX register we need now.}

\IFRU{Попробуем GCC 4.4.1 с опцией \Othree.}{Let's try GCC 4.4.1 with \Othree option.}

\lstinputlisting[caption=\Optimizing GCC 4.4.1]{patterns/15_structs/CPUID_gcc_O3.asm}

\IFRU{Практически, то же самое. Единственное что стоит отметить это то, что GCC решил зачем-то объединить 
вычисление \TT{extended\_model\_id} и \TT{extended\_family\_id} в один блок, 
вместо того чтобы вычислять их перед соответствующим вызовом \printf.}
{Almost the same.
The only thing worth noting is the GCC somehow united calculation of
\TT{extended\_model\_id} and \TT{extended\_family\_id} into one block,
instead of calculating them separately, before corresponding each \printf call.}

\subsubsection{\WorkingWithFloatAsWithStructSubSubSectionName}
\label{sec:floatasstruct}

\IFRU{Как уже раннее указывалось в секции о FPU~(\ref{sec:FPU}), и \Tfloat и \Tdouble содержат в себе знак, 
мантиссу и экспоненту. 
Однако, можем ли мы работать с этими полями напрямую? Попробуем с \Tfloat.}
{As it was already noted in section about FPU~(\ref{sec:FPU}), both \Tfloat and \Tdouble types consisted of sign,
significand (or fraction) and exponent.
But will we able to work with these fields directly? Let's try with \Tfloat.}

\bigskip
% a hack used here! http://tex.stackexchange.com/questions/73524/bytefield-package
\begin{center}
\begin{bytefield}{32}
	\bitheader[endianness=big]{0,22,23,30,31} \\
	\bitbox{1}{S} & 
	\bitbox{8}{\IFRU{экспонента}{exponent}} & 
	\bitbox{23}{\IFRU{мантисса}{mantissa or fraction}}
\end{bytefield}
\end{center}

\begin{center}
( S\EMDASH{}\IFRU{знак}{sign} )
\end{center}

\lstinputlisting{patterns/15_structs/float_en.c}

\IFRU{Структура \TT{float\_as\_struct} занимает в памяти столько же места сколько и \Tfloat, 
то есть 4 байта или 32 бита.}
{\TT{float\_as\_struct} structure occupies as much space is memory as \Tfloat, e.g., 4 bytes or 32 bits.}

\IFRU{Далее мы выставляем во входящем значении отрицательный знак, 
а также прибавляя двойку к экспоненте, мы тем 
самым умножаем всё значение на \TT{$2^2$}, то есть на 4.}
{Now we setting negative sign in input value and also by adding 2 to exponent we thereby multiplicating
the whole number by \TT{$2^2$}, e.g., by 4.}

\IFRU{Компилируем в MSVC 2008 без оптимизации:}{Let's compile in MSVC 2008 without optimization:}

\lstinputlisting[caption=\NonOptimizing MSVC 2008]{patterns/15_structs/float_msvc_\LANG.asm}

\IFRU{Слегка избыточно. В версии скомпилированной с флагом \Ox нет вызовов \TT{memcpy()}, 
там работа происходит сразу с переменной f. Но по неоптимизированной версии будет проще понять.}
{Redundant for a bit.
If it is compiled with \Ox flag there is no \TT{memcpy()} call,
\TT{f} variable is used directly.
But it is easier to understand it all considering unoptimized version.}

\IFRU{А что сделает GCC 4.4.1 с опцией \Othree?}{What GCC 4.4.1 with \Othree will do?}

\lstinputlisting[caption=\Optimizing GCC 4.4.1]{patterns/15_structs/float_gcc_O3_\LANG.asm}

\IFRU{Да, функция \TT{f()} в целом понятна. Однако, что интересно, еще при компиляции, 
не взирая на мешанину с полями структуры, GCC умудрился вычислить результат функции \TT{f(1.234)} и 
сразу подставить его в аргумент для \printf{}!}
{The \TT{f()} function is almost understandable. However, what is interesting, GCC was able to calculate
\TT{f(1.234)} result during compilation stage despite all this hodge-podge with structure fields
and prepared this argument to the \printf{} as precalculated!}



\section{\IFRU{Структуры}{Structures}}

\IFRU{В принципе, структура в \CCpp это, с некоторыми допущениями, просто всегда лежащий рядом, 
и в той же последовательности, набор переменных, не обязательно одного типа.}
{It can be defined that \CCpp structure, with some assumptions, just a set of variables, always stored
in memory together, not necessary of the same type.}

\subsection{\IFRU{Пример SYSTEMTIME}{SYSTEMTIME example}}

\newcommand{\FNSYSTEMTIME}{\footnote{\href{http://msdn.microsoft.com/en-us/library/ms724950(VS.85).aspx}{MSDN: SYSTEMTIME structure}}}

\IFRU{Возьмем, к примеру, структуру SYSTEMTIME\FNSYSTEMTIME{} из win32 описывающую время.}
{Let's take SYSTEMTIME\FNSYSTEMTIME{} win32 structure describing time.}

\IFRU{Она объявлена так:}{That's how it's defined:}

\begin{lstlisting}[caption=WinBase.h]
typedef struct _SYSTEMTIME {
  WORD wYear;
  WORD wMonth;
  WORD wDayOfWeek;
  WORD wDay;
  WORD wHour;
  WORD wMinute;
  WORD wSecond;
  WORD wMilliseconds;
} SYSTEMTIME, *PSYSTEMTIME;
\end{lstlisting}

\IFRU{Пишем на Си функцию для получения текущего системного времени:}
{Let's write a C function to get current time:}

\lstinputlisting{15_structs/systemtime.c}

\IFRU{Что в итоге}{We got} (MSVC 2010):

\lstinputlisting[caption=MSVC 2010]{15_structs/systemtime.asm}

\IFRU{Под структуру в стеке выделено 16 байт ~--- именно столько будет \TT{sizeof(WORD)*8}
(в структуре 8 переменных с типом WORD).}
{16 bytes are allocated for this structure in local stack ~--- that's exactly \TT{sizeof(WORD)*8}
(there are 8 WORD variables in the structure).}

\newcommand{\FNMSDNGST}{\footnote{\href{http://msdn.microsoft.com/en-us/library/ms724390(VS.85).aspx}{MSDN: GetSystemTime function}}}

\IFRU{Обратите внимание на тот факт что структура начинается с поля \TT{wYear}. 
Можно сказать что в качестве аргумента для \TT{GetSystemTime()}\FNMSDNGST передается указатель на структуру 
SYSTEMTIME, но можно также сказать, что передается указатель на поле \TT{wYear}, 
что одно и тоже! 
\TT{GetSystemTime()} пишет текущий год в тот WORD на который указывает переданный указатель, 
затем сдвигается на 2 байта вправо, пишет текущий месяц, итд, итд.}
{Pay attention to the fact the structure beginning with \TT{wYear} field.
It can be said, an pointer to SYSTEMTIME structure is passed to \TT{GetSystemTime()}\FNSYSTEMTIME,
but it's also can be said, pointer to \TT{wYear} field is passed, and that's the same!
\TT{GetSystemTime()} writting current year to the WORD pointer pointing to, then shifting 2 bytes
ahead, then writting current month, etc, etc.}

Тот факт что поля структуры это просто переменные расположенные рядом, 
я могу проиллюстрировать следующим образом.
Глядя на описание структуры \TT{SYSTEMTIME}, мы можем переписать наш простой пример так:

\lstinputlisting{15_structs/systemtime2.c}

Компилятор немного поворчит:

\begin{lstlisting}
systemtime2.c(7) : warning C4133: 'function' : incompatible types - from 'WORD [8]' to 'LPSYSTEMTIME'
\end{lstlisting}

Тем не менее, выдаст такой код:

\lstinputlisting[caption=MSVC 2010]{15_structs/systemtime2.asm}

И это работает так же!

Любопытно что результат на ассемблере неотличим от предыдущего. Таким образом, глядя на этот код, 
никогда нельзя сказать с уверенностью, была ли там объявлена структура, либо просто набор переменных.

Тем не менее, никто в здравом уме делать так не будет. 
Потому что это неудобно. К тому же, иногда, поля в структуре могут меняться, переставляться местами, итд.




\subsection{\IFRU{Выделяем место для структуры через malloc()}{Let's allocate place for structure using malloc()}}

\IFRU{Однако, бывает и так, что проще хранить структуры не в стеке а в куче\footnote{heap}:}
{However, sometimes it's simpler to place structures not in local stack, but in heap:}

\lstinputlisting{15_structs/systemtime_malloc.c}

\IFRU{Скомпилируем на этот раз с оптимизацией (\Ox) чтобы было проще увидеть то, что нам нужно.}
{Let's compile it now with optimization (\Ox) so to easily see what we need.}

\lstinputlisting[caption=\Optimizing MSVC]{15_structs/systemtime_malloc.asm}

\index{\CLanguageElements!malloc()}
\IFRU{Итак, \TT{sizeof(SYSTEMTIME) = 16}, именно столько байт выделяется при помощи \TT{malloc()}. 
Она возвращает указатель на только что выделенный блок памяти в \EAX, который копируется в \ESI. 
Win32 функция \TT{GetSystemTime()} обязуется сохранить состояние \ESI, 
поэтому здесь оно нигде не сохраняется и продолжает использоваться после вызова \TT{GetSystemTime()}.}
{So, \TT{sizeof(SYSTEMTIME) = 16}, that's exact number of bytes to be allocated by \TT{malloc()}.
It return the pointer to freshly allocated memory block in \EAX, which is then moved into \ESI.
\TT{GetSystemTime()} win32 function undertake to save \ESI value, 
and that's why it is not saved here and continue to be used after \TT{GetSystemTime()} call.}

\index{x86!\Instructions!MOVZX}
\IFRU{
Новая инструкция ~--- \MOVZX (\IT{Move with Zero eXtent}). 
Она нужна почти там же где и \MOVSX, 
только всегда очищает остальные биты в $0$. Дело в том что \printf требует 32-битный тип \Tint, 
а в структуре лежит WORD ~--- это 16-битный беззнаковый тип. Поэтому копируя значение из WORD в \Tint, 
нужно очистить биты от 16 до 31, иначе там будет просто случайный мусор, оставшийся от предыдущих действий 
с регистрами.}
{New instruction ~--- \MOVZX (\IT{Move with Zero eXtent}).
It may be used almost in those cases as \MOVSX, but, it clearing other bits to $0$.
That's because \printf require 32-bit \Tint, but we got WORD in structure ~--- that's 16-bit unsigned type.
That's why by copying value from WORD into \Tint{}, bits from 16 to 31 should be cleared, 
because there will be random noise otherwise, leaved from previous operations on registers.}

\IFRU{В этом примере я тоже могу представить структуру как массив WORD-ов}{In this example, I can represent
structure as array of WORD-s}:

\lstinputlisting{15_structs/systemtime_malloc2.c}

\IFRU{Получим такое}{We got}:

\lstinputlisting[caption=\Optimizing MSVC]{15_structs/systemtime_malloc2.asm}

\IFRU{И снова мы получаем идетичный код, неотличимый от предыдущего}{Again, we got a code that cannot be distinguished
from previous}.
\IFRU{Но и снова я должен отметить, что в реальности так лучше не делать}{And again I should note, one shouldn't do
this in practice}.



\subsection{Linux}

\IFRU{В Линуксе, для примера, возьем структуру \TT{tm} из \TT{time.h}:}
{As of Linux, let's take \TT{tm} structure from \TT{time.h} for example:}

\lstinputlisting{15_structs/GCC_tm.c}

\IFRU{Компилируем при помощи}{Let's compile it in} GCC 4.4.1:

\IFRU{\lstinputlisting[caption=GCC 4.4.1]{15_structs/GCC_tm_ru.asm}}{\lstinputlisting{15_structs/GCC_tm_en.asm}}

\IFRU{К сожалению, по какой-то причине, \IDA не сформировала названия локальных переменных в стеке. 
Но так как мы уже опытные реверсеры :-) то можем обойтись и без этого в таком простом примере.}
{Somehow, \IDA didn't created local variables names in local stack.
But since we already experienced reverse engineers :-) we may do it without this information in 
this simple example.}

\IFRU{Обратите внимание на \TT{lea edx, [eax+76Ch]} ~--- эта инструкция прибавляет $0x76C$ к \EAX, 
но не модифицирует флаги. См. также соответствующий раздел об инструкции \LEA{}~\ref{sec:LEA}.}
{Please also pay attention to \TT{lea edx, [eax+76Ch]} ~--- this instruction just adding $0x76C$ to \EAX,
but not modify any flags. See also relevant section about \LEA{}~\ref{sec:LEA}.}

Чтобы проиллюстрировать то что структура это просто набор переменных лежащих в одном месте, переделаем немного
пример, заглянув предварительно в файл time.h:

\begin{lstlisting}[caption=time.h]
struct tm
{
  int	tm_sec;
  int	tm_min;
  int	tm_hour;
  int	tm_mday;
  int	tm_mon;
  int	tm_year;
  int	tm_wday;
  int	tm_yday;
  int	tm_isdst;
};
\end{lstlisting}

\lstinputlisting{15_structs/GCC_tm2.c}

Обратите внимание на то что в \TT{localtime\_r} передается указатель именно на \TT{tm\_sec}, 
т.е., на первый элемент ``структуры''.

В итоге:

\lstinputlisting[caption=GCC 4.7.3]{15_structs/GCC_tm2.asm}

Этот код почти идентичен уже рассмотренному, и нельзя сказать, была ли структура
в оригинальном исходном коде либо набор переменных.

И это работает. Однако, в реальности так лучше не делать. Обычно, компилятор располагает переменные в локальном
стеке в том же порядке, в котором они объявляются в функции. Тем не менее, никакой гарантии нет.

Я выбрал именно этот пример для иллюстрации, потому что члены структуры имеют тип \Tint, а члены структуры
\TT{SYSTEMTIME} ~--- 16-битные \TT{WORD}, и если их объявлять так же, то они будут выровнены по 32-битной границе 
и ничего не выйдет (потому что \TT{GetSystemTime()} заполнит их неверно). Читайте об этом в следующей секции
``\StructurePackingSectionName''.

Так что, структура это просто набор переменных лежащих в одном месте, рядом. Я мог бы сказать что структура
это такой синтаксический сахар, заставляющий компилятор удерживать их в одном месте. Впрочем, я не специалист
по языкам программирования, так что, скорее всего, ошибаюсь с этим термином.



\subsection{\StructurePackingSectionName}

\IFRU{Достаточно немаловажный момент, это упаковка полей в структурах\footnote{См.также: \URLWPDA}.}
{One important thing is fields packing in structures\footnote{See also: \URLWPDA}.}

\IFRU{Возьмем простой пример:}{Let's take a simple example:}

\lstinputlisting{15_structs/15_5.c}

\IFRU{Как видно, мы имеем два поля \Tchar (занимающий один байт) и еще два ~--- \Tint (по 4 байта).}
{As we see, we have two \Tchar fields (each is exactly one byte) and two more ~--- \Tint (each - 4 bytes).}

\IFRU{Компилируется это все в:}{That's all compiling into:}

\lstinputlisting{15_structs/15_5.asm}

\IFRU{Мы видим здесь что адрес каждого поля в структуре выравнивается по 4-байтной границе. 
Так что каждый \Tchar здесь занимает те же 4 байта что и \Tint. Зачем? 
Затем что процессору удобнее обращаться по таким адресам и кешировать данные из памяти.}
{As we can see, each field's address is aligned by 4-bytes border.
That's why each \Tchar using 4 bytes here, like \Tint. Why?
Thus it's easier for CPU to access memory at aligned addresses and to cache data from it.}

\IFRU{Но это не экономично по размеру данных.}{However, it's not very economical in size sense.}

\IFRU{Попробуем скомпилировать тот же исходник с опцией}{Let's try to compile it with option} (\TT{/Zp1}) 
(\IT{/Zp[n] pack structs on n-byte boundary}).

\lstinputlisting[caption=MSVC /Zp1]{15_structs/15_5_msvc_Zp1.asm}

\IFRU{Теперь структура занимает 10 байт и все \Tchar занимают по байту. Что это дает? 
Экономию места. Недостаток ~--- процессор будет обращаться к этим полям не так эффективно 
по скорости как мог бы.}
{Now the structure takes only 10 bytes and each \Tchar value takes 1 byte. What it give to us?
Size economy. And as drawback ~--- CPU will access these fields without maximal performance it can.}

\IFRU{Как нетрудно догадаться, если структура используется много в каких исходниках и объектных файлах, 
все они должны быть откомпилированы с одним и тем же соглашением об упаковке структур.}
{As it can be easily guessed, if the structure is used in many source and object files,
all these should be compiled with the same convention about structures packing.}

\newcommand{\FNURLMSDNZP}{\footnote{\href{http://msdn.microsoft.com/en-us/library/ms253935.aspx}
{MSDN: Working with Packing Structures}}}
\newcommand{\FNURLGCCPC}{\footnote{\href{http://gcc.gnu.org/onlinedocs/gcc/Structure_002dPacking-Pragmas.html}
{Structure-Packing Pragmas}}}

\IFRU{Помимо ключа MSVC \TT{/Zp}, указывающего, по какой границе упаковывать поля структур, есть также 
опция компилятора \TT{\#pragma pack}, её можно указывать прямо в исходнике. 
Это справедливо и для MSVC\FNURLMSDNZP и GCC\FNURLGCCPC{}.}
{Aside from MSVC \TT{/Zp} option which set how to align each structure field, here is also
\TT{\#pragma pack} compiler option, it can be defined right in source code.
It's available in both MSVC\FNURLMSDNZP and GCC\FNURLGCCPC{}.}

\IFRU{Давайте теперь вернемся к \TT{SYSTEMTIME}, которая состоит из 16-битных полей. 
Откуда наш компилятор знает что их надо паковать по однобайтной границе?}
{Let's back to \TT{SYSTEMTIME} structure consisting in 16-bit fields.
How our compiler know to pack them on 1-byte alignment method?}

\IFRU{В файле \TT{WinNT.h} попадается такое:}{\TT{WinNT.h} file has this:}

\begin{lstlisting}[caption=WinNT.h]
#include "pshpack1.h"
\end{lstlisting}

\IFRU{И такое:}{And this:}

\begin{lstlisting}[caption=WinNT.h]
#include "pshpack4.h"                   // 4 byte packing is the default
\end{lstlisting}

\IFRU{Сам файл PshPack1.h выглядит так:}{The file PshPack1.h looks like:}

\begin{lstlisting}[caption=PshPack1.h]
#if ! (defined(lint) || defined(RC_INVOKED))
#if ( _MSC_VER >= 800 && !defined(_M_I86)) || defined(_PUSHPOP_SUPPORTED)
#pragma warning(disable:4103)
#if !(defined( MIDL_PASS )) || defined( __midl )
#pragma pack(push,1)
#else
#pragma pack(1)
#endif
#else
#pragma pack(1)
#endif
#endif /* ! (defined(lint) || defined(RC_INVOKED)) */
\end{lstlisting}

\IFRU{Собственно, так и задается компилятору, как паковать объявленные после \TT{\#pragma pack} структуры.}
{That's how compiler will pack structures defined after \TT{\#pragma pack}.}

\subsection{\IFRU{Вложенные структуры}{Nested structures}}

\IFRU{Теперь, как насчет ситуаций, когда одна структура определяет внутри себя еще одну структуру?}
{Now what about situations when one structure define another structure inside?}

\lstinputlisting{15_structs/15_6.c}

\dots \IFRU{в этом случае, оба поля \TT{inner\_struct} просто будут располагаться между полями a,b и d,e в 
\TT{outer\_struct}.}
{in this case, both \TT{inner\_struct} fields will be placed between a,b and d,e fields of
\TT{outer\_struct}.}

\IFRU{Компилируем}{Let's compile} (MSVC 2010):

\lstinputlisting[caption=MSVC 2010]{15_structs/15_6_msvc.asm}

\IFRU{Очень любопытный момент в том, что глядя на этот код на ассемблере, мы даже не видим, 
что была использована какая-то еще другая структура внутри этой!
Так что, пожалуй, можно сказать, что все вложенные структуры в итоге разворачиваются в одну, \IT{линейную} 
или \IT{одномерную} структуру.}
{One curious point here is that by looking onto this assembly code, we do not even see that
another structure was used inside of it!
Thus, we would say, nested structures are finally unfolds into \IT{linear} or \IT{one-dimensional} structure.}

\IFRU{Конечно, если заменить объявление \TT{struct inner\_struct c;} на \TT{struct inner\_struct *c;} 
(объявляя таким образом указатель), ситауция будет совсем иная.}
{Of course, if to replace \TT{struct inner\_struct c;} declaration to \TT{struct inner\_struct *c;} 
(thus making a pointer here) situation will be significally different.}



\subsection{\IFRU{Работа с битовыми полями в структуре}{Bit fields in structure}}

\subsubsection{\IFRU{Пример CPUID}{CPUID example}}

\IFRU{Язык \CCpp позволяет укзывать, сколько именно бит отвести для каждого поля структуры. 
Это удобно если нужно экономить место в памяти. К примеру, для переменной типа \Tbool достаточно одного бита.
Но, это не очень удобно, если нужна скорость.}
{\CCpp language allow to define exact number of bits for each structure fields.
It's very useful if one needs to save memory space. 
For example, one bit is enough for variable of \Tbool type.
But of course, it's not rational if speed is important.}

\newcommand{\FNCPUID}{\footnote{\url{http://en.wikipedia.org/wiki/CPUID}}}

\index{x86!\Instructions!CPUID}
\IFRU{Рассмотрим пример с инструкцией \CPUID\FNCPUID. 
Эта инструкция возвращает информацию о том, какой процессор имеется в наличии и какие фичи он имеет.}
{Let's consider \CPUID\FNCPUID instruction example.
This instruction returning information about current CPU and its features.}

\IFRU{Если перед исполнением инструкции в \EAX будет 1, 
то \CPUID вернет упакованную в \EAX такую информацию о процессоре:}
{If the \EAX is set to 1 before instruction execution, 
\CPUID will return this information packed into the \EAX register:}

\begin{center}
\begin{tabular}{ | l | l | }
\hline
3:0 & Stepping \\
7:4 & Model \\
11:8 & Family \\
13:12 & Processor Type \\
19:16 & Extended Model \\
27:20 & Extended Family \\
\hline
\end{tabular}
\end{center}

\newcommand{\FNGCCAS}{\footnote{\href{http://www.ibiblio.org/gferg/ldp/GCC-Inline-Assembly-HOWTO.html}
{\IFRU{Подробнее о встроенном ассемблере GCC}{More about internal GCC assembler}}}}

\IFRU{MSVC 2010 имеет макрос для \CPUID, а GCC 4.4.1 ~--- нет. 
Поэтому для GCC сделаем эту функцию сами, используя его встроенный ассемблер\FNGCCAS.}
{MSVC 2010 has \CPUID macro, but GCC 4.4.1 ~--- hasn't.
So let's make this function by yourself for GCC with the help of its built-in assembler\FNGCCAS.}

\lstinputlisting{15_structs/CPUID.c}

\IFRU{После того как \CPUID заполнит \EAX/\EBX/\ECX/\EDX, у нас они отразятся в массиве \TT{b[]}. 
Затем, мы имеем указатель на структуру \TT{CPUID\_1\_EAX}, и мы указываем его на значение 
\EAX из массива \TT{b[]}.}
{After \CPUID will fill \EAX/\EBX/\ECX/\EDX, these registers will be reflected in the \TT{b[]} array.
Then, we have a pointer to the \TT{CPUID\_1\_EAX} structure and we point it to the value in the \EAX from \TT{b[]} array.}

\IFRU{Иными словами, мы трактуем 32-битный \Tint как структуру.}
{In other words, we treat 32-bit \Tint value as a structure.}

\IFRU{Затем мы читаем из структуры.}{Then we read from the stucture.}

\IFRU{Компилируем в MSVC 2008 с опцией \Ox}{Let's compile it in MSVC 2008 with \Ox option}:

\lstinputlisting[caption=\Optimizing MSVC 2008]{15_structs/CPUID_msvc_Ox.asm}

\index{x86!\Instructions!SHR}
\IFRU{Инструкция \TT{SHR} сдвигает значение из \EAX на то количество бит, 
которое нужно \IT{пропустить}, то есть, мы игнорируем некоторые биты \IT{справа}.}
{\TT{SHR} instruction shifting value in the \EAX register by number of bits should be
\IT{skipped}, e.g., we ignore some bits \IT{at right}.}

\index{x86!\Instructions!AND}
\IFRU{А инструкция \AND очищает биты \IT{слева} которые нам не нужны, или же, говоря иначе, 
она оставляет по маске только те биты в \EAX, которые нам сейчас нужны.}
{\AND instruction clears bits not needed \IT{at left}, or, in other words, 
leaves only those bits in the \EAX register we need now.}

\IFRU{Попробуем GCC 4.4.1 с опцией \Othree.}{Let's try GCC 4.4.1 with \Othree option.}

\lstinputlisting[caption=\Optimizing GCC 4.4.1]{15_structs/CPUID_gcc_O3.asm}

\IFRU{Практически, то же самое. Единственное что стоит отметить это то, что GCC решил зачем-то объеденить 
вычисление \TT{extended\_model\_id} и \TT{extended\_family\_id} в один блок, 
вместо того чтобы вычислять их перед соответствующим вызовом \printf.}
{Almost the same. The only thing worth noting is that GCC somehow united calculation of
\TT{extended\_model\_id} and \TT{extended\_family\_id} into one block,
instead of calculating them separately, before corresponding each \printf call.}

\subsubsection{\WorkingWithFloatAsWithStructSubSubSectionName}
\label{sec:floatasstruct}

\IFRU{Как уже раннее указывалось в секции о FPU~\ref{sec:FPU}, и \Tfloat и \Tdouble содержат в себе знак, 
мантиссу и экспоненту. 
Однако, можем ли мы работать с этими полями напрямую? Попробуем с \Tfloat.}
{As it was already noted in section about FPU~\ref{sec:FPU}, both \Tfloat and \Tdouble types consisted of sign,
significand (or fraction) and exponent.
But will we able to work with these fields directly? Let's try with \Tfloat.}

\index{IEEE 754}
\index{float}
\begin{figure}[ht!]
\centering
\includegraphics[scale=0.66]{15_structs/500px-Float_example.png}
\caption{\IFRU{Формат значения float\protect\footnotemark}
{float value format\protect\footnotemark}}
\end{figure}

\footnotetext{\IFRU{иллюстрация взята из}{illustration taken from} wikipedia}

\lstinputlisting{15_structs/float_en.c}

\IFRU{Структура \TT{float\_as\_struct} занимает в памяти столько же места сколько и \Tfloat, 
то есть 4 байта или 32 бита.}
{\TT{float\_as\_struct} structure occupies as much space is memory as \Tfloat, e.g., 4 bytes or 32 bits.}

\IFRU{Далее мы выставляем во входящем значении отрицательный знак, 
а также прибавляя двойку к экспоненте, мы тем 
самым умножаем всё значение на \TT{$2^2$}, то есть на 4.}
{Now we setting negative sign in input value and also by addding 2 to exponent we thereby multiplicating
the whole number by \TT{$2^2$}, e.g., by 4.}

\IFRU{Компилируем в MSVC 2008 без оптимизации:}{Let's compile in MSVC 2008 without optimization:}

\lstinputlisting[caption=\NonOptimizing MSVC 2008]
{\IFRU{15_structs/float_msvc_ru.asm}{15_structs/float_msvc_en.asm}}

\IFRU{Слекга избыточно. В версии скомпилированной с флагом \Ox нет вызовов \TT{memcpy()}, 
там работа происходит сразу с переменной f. Но по неоптимизированной версии будет проще понять.}
{Redundant for a bit. If it compiled with \Ox flag there are no \TT{memcpy()} call,
f variable is used directly. But it's easier to understand it all considering unoptimized version.}

\IFRU{А что сделает GCC 4.4.1 с опцией \TT{-O3}?}{What GCC 4.4.1 with \TT{-O3} will do?}

\lstinputlisting[caption=\Optimizing GCC 4.4.1]
{\IFRU{15_structs/float_gcc_O3_ru.asm}{15_structs/float_gcc_O3_en.asm}}

\IFRU{Да, функция \TT{f()} в целом понятна. Однако, что интересно, еще при компиляции, 
не взирая на мешанину с полями структуры, GCC умудрился вычислить результат функции \TT{f(1.234)} и 
сразу подставить его в аргумент для \printf{}!}
{The \TT{f()} function is almost understandable. However, what is interesting, GCC was able to calculate
\TT{f(1.234)} result during compilation stage despite all this hodge-podge with structure fields
and prepared this argument to the \printf{} as precalculated!}





\input{16_classes/classes}
\subsection{\IFRU{Наследование классов в C++}{Class inheritance in C++}}

\IFRU{О наследованных классах можно сказать что это та же простая структура которую мы уже рассмотрели, 
только расширяемая в наследуемых классах.}
{It can be said about inherited classes that it's simple structure we already considered, but extending 
in inherited classes.}

\IFRU{Возьмем очень простой пример}{Let's take simple example}:

\lstinputlisting{16_classes/classes1_inheritance.cpp}

\IFRU{Исследуя сгенерированный код для функций/методов \TT{dump()}, а также \TT{object::print\_color()},
посмотрим какая будет разметка памяти для структур-объектов (для 32-битного кода).}
{Let's investigate generated code of \TT{dump()} functions/methods and also \TT{object::print\_color()},
let's see memory layout for structures-objects (as of 32-bit code).}

\IFRU{Итак, методы \TT{dump()} разных классов сгенерированные MSVC 2008 с опциями \Ox и \Obzero}
{So, \TT{dump()} methods for several classes, generated by MSVC 2008 with \Ox and \Obzero options}
\footnote{\IFRU{опция \Obzero означает отмену inline expansion, 
ведь вставка компилятором тела функции/метода прямо в код где он вызывается только затруднит наши эксперименты}{
\Obzero options mean inline expansion disabling, because, function inlining right into the code where the function
is called will make our experiment harder}}

\lstinputlisting[caption=\Optimizing MSVC 2008 /Ob0]{16_classes/classes1_1.asm}

\lstinputlisting[caption=\Optimizing MSVC 2008 /Ob0]{16_classes/classes1_2.asm}

\lstinputlisting[caption=\Optimizing MSVC 2008 /Ob0]{16_classes/classes1_3.asm}

\IFRU{Итак, разметка полей получается следующая}{So, here is memory layout}:

\IFRU{(базовый класс \IT{object})}{(base class \IT{object})}

\begin{center}
\begin{tabular}{ | l | l | }
\hline
  \tableheader{} \\
  +0x0 & int color \\
\hline
\end{tabular}
\end{center}

\IFRU{(унаследованные классы)}{(inherited classes)}

\IT{box}:

\begin{center}
\begin{tabular}{ | l | l | }
\hline
  \tableheader{} \\
  +0x0 & int color \\
  +0x4 & int width \\
  +0x8 & int height \\
  +0xC & int depth \\
\hline
\end{tabular}
\end{center}

\IT{sphere}:

\begin{center}
\begin{tabular}{ | l | l | }
\hline
  \tableheader{} \\
  +0x0 & int color \\
  +0x4 & int radius \\
\hline
\end{tabular}
\end{center}

\IFRU{Посмотрим тело \main}{Let's see \main function body}:

\lstinputlisting[caption=\Optimizing MSVC 2008 /Ob0]{16_classes/classes1_4.asm}

\IFRU{Наследованные классы всегда должны добавлять свои поля после полей базового класса для того, чтобы методы
базового класса могли продолжать работать со своими полями.}
{Inherited classes should always add their fields after base classes' fields, so to make possible for base 
class methods to work with their fields.}

\IFRU{Когда метод \TT{object::print\_color()} вызывается, ему в качестве \TT{this} передается указатель и на объект типа \IT{box} 
и на объект типа \IT{sphere}, так как он может легко работать с классами \IT{box} и \IT{sphere}, потому что поле \IT{color} в этих
классах всегда стоит по тому же адресу (по смещению \IT{0x0}).}
{When \TT{object::print\_color()} method is called, a pointers to both \IT{box} object and \IT{sphere} object are passed as \TT{this},
it can work with these objects easily because \IT{color} field in these objects is always at the pinned address (at \IT{+0x0} offset).}

\IFRU{Можно также сказать что методу \TT{object::print\_color()} даже не нужно знать,
с каким классом он работает, до тех пор пока будет соблюдаться условие /IT{закрепления} полей по тем же адресам,
а это условие соблюдается всегда.}
{It can be said, \TT{object::print\_color()} method is agnostic in relation to input object type as long as fields will be \IT{pinned}
at the same addresses, and this condition is always true.}

\IFRU{А если вы создадите класс-наследник класса \IT{box}, например, 
то компилятор будет добавлять новые поля уже за полем \IT{depth}, оставляя уже имеющиеся поля класса \IT{box} по тем же адресам.}
{And if you create inherited class of \IT{box} class, for example, compiler will add new fields after \IT{depth} field,
leaving \IT{box} class fields at the pinned addresses.}

\IFRU{Так, метод \TT{box::dump()} будет нормально работать обращаясь к полям \IT{color}/\IT{width}/\IT{height}/\IT{depth} всегда находящимся по известным адресам.}
{Thus, \TT{box::dump()} method will work fine accessing \IT{color}/\IT{width}/\IT{height}/\IT{depths} fields always pinned on known addresses.}

\IFRU{Код на GCC практически точно такой же, за исключением способа передачи \TT{this} (он, как уже было указано, 
передается в первом аргументе, вместо регистра \ECX).}
{GCC-generated code is almost the same, with the sole exception of \TT{this} pointer passing (as it was described above,
it passing as first argument instead of \ECX registers.}


\input{16_classes/classes_2_encapsulation}
\input{16_classes/classes_3_mutiple}
\input{16_classes/classes_4_virtual}
\section{\IFRU{Объединения (union)}{Unions}}

\subsection{\IFRU{Пример генератора случайных чисел}{Pseudo-random number generator example}}

\IFRU{Если нам нужны случайные значения с плавающей запятой в интервале от 0 до 1, самое простое это взять
\ac{PRNG} вроде Mersenne twister выдающий случайные 32-битные числа в виде DWORD, преобразовать
это число в \Tfloat и затем разделить на \TT{RAND\_MAX} (\TT{0xFFFFFFFF} в данном случае) ~--- 
полученное число будет в интервале от 0 до 1.}
{If we need float random numbers from 0 to 1, the most simplest thing is to use \ac{PRNG} like
Mersenne twister produces random 32-bit values in DWORD form, transform this value to \Tfloat and then
dividing it by \TT{RAND\_MAX} (\TT{0xFFFFFFFF} in our case)~---value we got will be in 0..1 interval.}

\IFRU{Но как известно, операция деления ~--- это медленная операция. 
Сможем ли мы избежать её, как в случае с делением через умножение?}
{But as we know, division operation is slow.
Will it be possible to get rid of it, as in case of division by multiplication?}
~(\ref{sec:divisionbynine})

\index{IEEE 754}
\IFRU{Вспомним состав числа с плавающей запятой: это бит знака, биты мантиссы и биты экспоненты. 
Для получения случайного числа, нам нужно просто заполнить случайными битами все биты мантиссы!}
{Let's recall what float number consisted of: sign bit, significand bits and exponent bits.
We need just to store random bits to all significand bits for getting random float number!}

\IFRU{Экспонента не может быть нулевой (иначе число будет денормализованным), 
так что в эти биты мы запишем \TT{01111111} ~--- 
это будет означать что экспонента равна единице. Далее заполняем мантиссу случайными битами, 
знак оставляем в виде 0 (что значит наше число положительное), и вуаля. 
Генерируемые числа будут в интервале от 1 до 2, так что нам еще нужно будет отнять единицу.}
{Exponent cannot be zero (number will be denormalized in this case), so we will store \TT{01111111} 
to exponent~---this means exponent will be 1. Then fill significand with random bits, set sign bit to
0 (which means positive number) and voilà.
Generated numbers will be in 1 to 2 interval, so we also must subtract 1 from it.}

\newcommand{\URLXOR}{\url{http://xor0110.wordpress.com/2010/09/24/how-to-generate-floating-point-random-numbers-efficiently}}

\IFRU{В моем примере\footnote{идея взята здесь: \URLXOR} 
применяется очень простой линейный конгруэнтный генератор случайных чисел, выдающий 32-битные числа.
Генератор инициализируется текущим временем в стиле UNIX.}
{Very simple linear congruential random numbers generator is used in my 
example\footnote{idea was taken from: \URLXOR}, produces 32-bit numbers. 
The PRNG initializing by current time in UNIX-style.}

\IFRU{Далее, тип \Tfloat представляется в виде \IT{union} ~--- это конструкция \CCpp позволяющая 
интерпретировать часть памти по-разному. В нашем случае, мы можем создать переменную типа \TT{union} 
и затем обращаться к ней как к \Tfloat или как к \IT{uint32\_t}. Можно сказать, что это хак, причем грязный.}
{Then, \Tfloat type represented as \IT{union}~---it is the \CCpp construction enabling us
to interpret piece of memory as differently typed.
In our case, we are able to create a variable
of \TT{union} type and then access to it as it is \Tfloat or as it is \IT{uint32\_t}. 
It can be said, it is just a hack. A dirty one.}

\lstinputlisting{patterns/17_unions/FPU_PRNG.cpp}

\lstinputlisting[caption=MSVC 2010 (\Ox)]{patterns/17_unions/FPU_PRNG_msvc_2010_Ox_\LANG.asm}

\IFRU{А результат GCC будет почти таким же.}{GCC produces very similar code.}



\input{18_pointers_to_functions/pointers_to_functions}
\input{19_SIMD/SIMD}
\subsection{x64}

\index{x86-64}
\RU{Всё то же самое, только используются регистры вместо стека для передачи аргументов функций}%
\EN{The picture here is similar with the difference that the registers, rather than the stack, are used for arguments passing}.

\subsubsection{MSVC}

\lstinputlisting[caption=MSVC 2012 x64]{patterns/04_scanf/1_simple/ex1_MSVC_x64.asm.\LANG}

\ifdefined\IncludeGCC
\subsubsection{GCC}

\lstinputlisting[caption=\Optimizing GCC 4.4.6 x64]{patterns/04_scanf/1_simple/ex1_GCC_x64.s.\LANG}
\fi

\section{C99 restrict}

А вот причина из-за которой программы на FORTRAN, в некоторых случаях, работают быстрее чем на Си.

\begin{lstlisting}
void f1 (int* x, int* y, int* sum, int* product, int* sum_product, int* update_me, size_t s)
{
	for (int i=0; i<s; i++)
	{
		sum[i]=x[i]+y[i];
		product[i]=x[i]*y[i];
		update_me[i]=i*123; // some dummy value
		sum_product[i]=sum[i]+product[i];	
	};
};
\end{lstlisting}

Это очень простой пример, в котором есть одна особенность: указатель на массив \TT{update\_me} может быть указателем 
на массив
\TT{sum}, \TT{product}, или даже \TT{sum\_product} ~--- 
ведь нет ничего криминального в том чтобы аргументам функции быть такими.

Компилятор знает об этом, поэтому генерирует код, где в теле цикла будет 4 основных стадии:
\begin{itemize}
\item вычислить следующий \TT{sum[i]}
\item вычислить следующий \TT{product[i]}
\item вычислить следующий \TT{update\_me[i]}
\item вычислить следующий \TT{sum\_product[i]} ~--- на этой стадии придется снова загружать из 
      памяти подсчитанные \TT{sum[i]} и \TT{product[i]}
\end{itemize}

Возможно ли соптимизировать последнюю стадию? Ведь подсчитанные \TT{sum[i]} и \TT{product[i]} не обязательно 
снова загружать из памяти,
ведь мы их только что подсчитали. Можно, но компилятор не уверен, что на третьей стадии ничего не затерлось! Это называется
``pointer aliasing'', ситуация, когда компилятор не может быть уверен что память на которую указывает какой-то указатель, 
не изменилась.

\IT{restrict} в стандарте Си C99\cite[6.7.3.1]{C99TC3} это обещание, даваемое компилятору программистом, 
что аргументы функции отмеченные этим 
ключевым словом,
всегда будут указывать на разные места в памяти и пересекаться не будут.

Если быть более точным, и описывать это формально, \IT{restrict} показывает, что только данный указатель будет
использоваться для доступа к этому объекту, с которым мы работаем через этот указатель, больше никакой указатель для
этого использоваться не будет. Можно даже сказать, что к всякому объекту, доступ будет осуществляться только через
один единственный указатель, если он отмечен как \IT{restrict}.

Добавим это ключевое слово к каждому аргументу-указателю:

\begin{lstlisting}
void f2 (int* restrict x, int* restrict y, int* restrict sum, int* restrict product, int* restrict sum_product, 
	int* restrict update_me, size_t s)
{
	for (int i=0; i<s; i++)
	{
		sum[i]=x[i]+y[i];
		product[i]=x[i]*y[i];
		update_me[i]=i*123; // some dummy value
		sum_product[i]=sum[i]+product[i];	
	};
};
\end{lstlisting}

Посмотрим результат:

\lstinputlisting[caption=GCC x64: f1()]{21_C99_restrict/f1.asm}

\lstinputlisting[caption=GCC x64: f2()]{21_C99_restrict/f2.asm}

Разница между скомпилированной функцией \TT{f1()} и \TT{f2()} такая: 
в \TT{f1()} \TT{sum[i]} и \TT{product[i]} загружаются снова посреди тела цикла,
а в \TT{f2()} этого нет, используются уже подсчитанные значения, ведь мы ``пообещали'' компилятору, 
что никто и ничто не изменит
значения в \TT{sum[i]} и \TT{product[i]} во время исполнения тела цикла, поэтому он ``уверен'', что так можно делать. 
Очевидно, второй вариант будет работать быстрее.

Но что будет если указатели в аргументах функций все же будут пересекаться? Это останется на совести программиста, 
а результаты вычислений будут неверными.

Вернемся к FORTRAN. Компиляторы с этого ЯП, по умолчанию, все указатели считают таковыми, поэтому, когда в Си не было
возможности указать \IT{restrict}, FORTRAN в этих случаях мог генерировать более быстрый код.

Насколько это практично? Там где функция работает с несколькими большими блоками в памяти. 
Такого очень много в линейной алгебре, например. Очень много линейной алгебры используется на суперкомпьютерах/HPC,
возможно, поэтому, традиционно, там часто используется FORTRAN, до сих пор\cite{Loh:2010:IHP:1810226.1820518}.

Ну а когда итераций цикла не очень много, конечно, тогда прирост скорости не будет ощутимым.


\section{\IFRU{Inline-функции}{Inline functions}}
\index{Inline code}

\IFRU{Inline-код это когда компилятор, вместо того чтобы генерировать инструкцию вызова небольшой функции,
просто вставляет её тело прямо в это место.}
{Inlined code is when compiler, instead of placing call instruction to small or tiny function,
just placing its body right in-place.}

\lstinputlisting[caption=\IFRU{Простой пример}{Simple example}]{22_inline_function/1.c}

\IFRU{... это компилируется вполне предсказуемо, хотя, если включить оптимизации GCC (-O3), мы увидим:}
{... is compiled in very predictable way, however, if to turn on GCC optimization (-O3), we'll see:}

\lstinputlisting[caption=GCC 4.8.1 -O3]{22_inline_function/1.s}

(\IFRU{Здесь деление заменено умножением}{Here division is done by multiplication}\ref{sec:divisionbynine}.)

\IFRU{Да, наша маленькая ф-ция была помещена прямо перед вызовом \printf.}
{Yes, our small function was just placed befor \printf call.}
\IFRU{Почему? Это может быть быстрее чем исполнять код самой ф-ции плюс затраты на вызов и возврат.}
{Why? It may be faster than executing this function's code plus calling/returning overhead.}

\IFRU{В прошлом, такие ф-ции нужно было маркировать ключевым словом ``inline'' в определении ф-ции, хотя,
в наше время, такие ф-ции выбираются компилятором автоматически.}
{In past, such function should be marked with ``inline'' keyword in function's declaration, however,
in modern times, these functions are chosen automatically by compiler.}

\IFRU{Другая очень частая оптимизация это вставка кода строковых ф-ций таких как}
{Another very common automatic optimization is inlining of string functions like}
\IT{strcpy()}, \IT{strcmp()}, \IFRU{итд}{etc}.

\lstinputlisting[caption=\IFRU{Еще один простой пример}{Another simple example}]{22_inline_function/2.c}

\lstinputlisting[caption=GCC 4.8.1 -O3]{22_inline_function/2.s}

\IFRU{Вот пример очень часто попадающегося фрагмента кода strcmp() генерируемого MSVC:}
{Here is an example of very frequently seen piece of strcmp() code generated by MSVC:}

\lstinputlisting[caption=MSVC]{22_inline_function/strcmp.lst}

\IFRU{Я написал небольшой скрипт для \IDA для поиска и сворачивания таких очень часто 
попадающихся inline-функций:}
{I wrote small \IDA script for searching and folding such very frequently seen pieces of inline code:}
\url{https://github.com/yurichev/IDA_scripts}.


\chapter{\IFRU{Еще кое-что}{Couple things to add}}

\index{x86!\Instructions!LEA}
  \item[LEA] \IFRU{сформировать адрес}{form address} \IFRU{см.также}{see also}: \ref{sec:LEA}

\section{\IFRU{Пролог и эпилог в функции}{Function prologue and epilogue}}
\label{sec:prologepilog}
\index{Function epilogue}
\index{Function prologue}

\IFRU{Пролог функции это инструкции в самом начале функции. Как правило это что-то вроде такого
фрагмента кода:}
{Function prologue is instructions at function start. It is often something like the following
code fragment:}

\begin{lstlisting}
    push    ebp
    mov     ebp, esp
    sub     esp, X
\end{lstlisting}

\IFRU
{Эти инструкции делают следующее: сохраняют значение регистра \EBP на будущее, выставляют \EBP равным \ESP, 
затем подготавливают место в стеке для хранения локальных переменных.}
{What these instruction do: saves the value in the \EBP register,
set value of the \EBP register to the value of the \ESP and then allocates space on the stack 
for local variables.}

\IFRU{\EBP сохраняет свое значение на протяжении всей функции, он будет использоваться здесь для доступа 
к локальным переменным и аргументам. Можно было бы использовать и \ESP, но он постоянно меняется и 
это не очень удобно.}
{Value in the \EBP is fixed over a period of function execution and it is to be used for local variables and 
arguments access. 
One can use \ESP, but it changing over time and it is not convenient.}

\IFRU{Эпилог функции аннулирует выделенное место в стеке, возвращает значение \EBP на то что было и возвращает 
управление в вызывающую функцию:}
{Function epilogue annuled allocated space in stack, returns value in the \EBP register back to initial state 
and returns flow control to callee:}

\begin{lstlisting}
    mov    esp, ebp
    pop    ebp
    ret    0
\end{lstlisting}

\index{\Recursion}
\IFRU{Наличие эпилога и пролога может несколько ухудшить эффективность рекурсии.

Например, однажды я написал функцию для поиска нужного узла в двоичном дереве. 
Рекурсивно она выглядела очень красиво, но из-за того что при каждом вызове тратилось время на эпилог и пролог, 
все это работало в несколько раз медленнее чем та же функция но без рекурсии.}
{Epilogue and prologue can make recursion performance worse.

For example, once upon a time I wrote a function to seek right node in binary tree. 
As a recursive function it would look stylish but since some time is to be spend at each function call
for prologue/epilogue, it was working couple of times slower than iterative (recursion-free)
implementation.}

\index{\Recursion!Tail recursion}
\newcommand{\URLT}{\url{http://en.wikipedia.org/wiki/Tail_call}}
\IFRU
{Кстати, поэтому есть такая вещь как хвостовая рекурсия\footnote{\URLT}: 
когда компилятор или интерпретатор превращает рекурсию (с которой возможно это проделать: 
\IT{хвостовую}) в итерацию для эффективности.}
{By the way, that is the reason of tail call\footnote{\URLT} existence: when compiler (or interpreter) 
transforms recursion (with which it is possible: \IT{tail recursion}) into iteration for efficiency.}

\subsection{npad}
\label{sec:npad}

\RU{Это макрос в ассемблере, для выравнивания некоторой метки по некоторой границе.}
\EN{It is an assembly language macro for aligning labels on a specific boundary.}

\RU{Это нужно для тех \IT{нагруженных} меток, куда чаще всего передается управление, например, 
начало тела цикла. 
Для того чтобы процессор мог эффективнее вытягивать данные или код из памяти, через шину с памятью, 
кэширование, итд.}
\EN{That's often needed for the busy labels to where the control flow is often passed, e.g., loop body starts.
So the CPU can load the data or code from the memory effectively, through the memory bus, cache lines, etc.}

\RU{Взято из}\EN{Taken from} \TT{listing.inc} (MSVC):

\myindex{x86!\Instructions!NOP}
\RU{Это, кстати, любопытный пример различных вариантов \NOP{}-ов. 
Все эти инструкции не дают никакого эффекта, но отличаются разной длиной.}
\EN{By the way, it is a curious example of the different \NOP variations.
All these instructions have no effects whatsoever, but have a different size.}

\RU{Цель в том, чтобы была только одна инструкция, а не набор NOP-ов, 
считается что так лучше для производительности CPU.}
\EN{Having a single idle instruction instead of couple of NOP-s,
is accepted to be better for CPU performance.}

\begin{lstlisting}[style=customasmx86]
;; LISTING.INC
;;
;; This file contains assembler macros and is included by the files created
;; with the -FA compiler switch to be assembled by MASM (Microsoft Macro
;; Assembler).
;;
;; Copyright (c) 1993-2003, Microsoft Corporation. All rights reserved.

;; non destructive nops
npad macro size
if size eq 1
  nop
else
 if size eq 2
   mov edi, edi
 else
  if size eq 3
    ; lea ecx, [ecx+00]
    DB 8DH, 49H, 00H
  else
   if size eq 4
     ; lea esp, [esp+00]
     DB 8DH, 64H, 24H, 00H
   else
    if size eq 5
      add eax, DWORD PTR 0
    else
     if size eq 6
       ; lea ebx, [ebx+00000000]
       DB 8DH, 9BH, 00H, 00H, 00H, 00H
     else
      if size eq 7
	; lea esp, [esp+00000000]
	DB 8DH, 0A4H, 24H, 00H, 00H, 00H, 00H 
      else
       if size eq 8
        ; jmp .+8; .npad 6
	DB 0EBH, 06H, 8DH, 9BH, 00H, 00H, 00H, 00H
       else
        if size eq 9
         ; jmp .+9; .npad 7
         DB 0EBH, 07H, 8DH, 0A4H, 24H, 00H, 00H, 00H, 00H
        else
         if size eq 10
          ; jmp .+A; .npad 7; .npad 1
          DB 0EBH, 08H, 8DH, 0A4H, 24H, 00H, 00H, 00H, 00H, 90H
         else
          if size eq 11
           ; jmp .+B; .npad 7; .npad 2
           DB 0EBH, 09H, 8DH, 0A4H, 24H, 00H, 00H, 00H, 00H, 8BH, 0FFH
          else
           if size eq 12
            ; jmp .+C; .npad 7; .npad 3
            DB 0EBH, 0AH, 8DH, 0A4H, 24H, 00H, 00H, 00H, 00H, 8DH, 49H, 00H
           else
            if size eq 13
             ; jmp .+D; .npad 7; .npad 4
             DB 0EBH, 0BH, 8DH, 0A4H, 24H, 00H, 00H, 00H, 00H, 8DH, 64H, 24H, 00H
            else
             if size eq 14
              ; jmp .+E; .npad 7; .npad 5
              DB 0EBH, 0CH, 8DH, 0A4H, 24H, 00H, 00H, 00H, 00H, 05H, 00H, 00H, 00H, 00H
             else
              if size eq 15
               ; jmp .+F; .npad 7; .npad 6
               DB 0EBH, 0DH, 8DH, 0A4H, 24H, 00H, 00H, 00H, 00H, 8DH, 9BH, 00H, 00H, 00H, 00H
              else
	       %out error: unsupported npad size
               .err
              endif
             endif
            endif
           endif
          endif
         endif
        endif
       endif
      endif
     endif
    endif
   endif
  endif
 endif
endif
endm
\end{lstlisting}

\section{\SignedNumbersSectionName}
\label{sec:signednumbers}
\index{Signed numbers}

\newcommand{\URLS}{\url{http://en.wikipedia.org/wiki/Signed_number_representations}}

\IFRU
{Методов представления чисел с знаком ``плюс'' или ``минус'' несколько\footnote{\URLS}, 
а в x86 применяется метод ``дополнительный код'' или ``two's complement''.}
{There are several methods of representing signed numbers\footnote{\URLS}, 
but in x86 architecture used ``two's complement''.}

\begin{center}
\begin{tabular}{ | l | l | l | l | }
\hline
\cellcolor{blue!25} \IFRU{двоичное}{binary} & 
\cellcolor{blue!25} \IFRU{шестнадцатеричное}{hexadecimal} & 
\cellcolor{blue!25} \IFRU{беззнаковое}{unsigned} &
\cellcolor{blue!25} \IFRU{знаковое}{signed} (\IFRU{дополнительный код}{2's complement}) \\
\hline
01111111 & 0x7f & 127 & 127 \\
\hline
01111110 & 0x7e & 126 & 126 \\
\hline
\multicolumn{4}{ |c| }{...} \\
\hline
00000010 & 0x2 & 2 & 2 \\
\hline
00000001 & 0x1 & 1 & 1 \\
\hline
00000000 & 0x0 & 0 & 0 \\
\hline
11111111 & 0xff & 255 & -1 \\
\hline
11111110 & 0xfe & 254 & -2 \\
\hline
\multicolumn{4}{ |c| }{...} \\
\hline
10000010 & 0x82 & 130 & -126 \\
\hline
10000001 & 0x81 & 129 & -127 \\
\hline
10000000 & 0x80 & 128 & -128 \\
\hline
\end{tabular}
\end{center}

\index{x86!\Instructions!JA}
\index{x86!\Instructions!JB}
\index{x86!\Instructions!JL}
\index{x86!\Instructions!JG}
\IFRU{Разница в подходе к знаковым/беззнаковым числам, собственно, нужна потому что, например, 
если представить \TT{0xFFFFFFFE} и \TT{0x0000002} как беззнаковое, то первое число ($4294967294$) больше второго ($2$). 
Если их оба представить как знаковые, то первое будет $-2$, которое, разумеется, меньше чем второе ($2$).
Вот почему инструкции для условных переходов~(\ref{sec:Jcc}) представлены в обоих версиях ~--- 
и для знаковых сравнений (например \JG, \JL) и для беззнаковых (\JA, \JB).}
{The difference between signed and unsigned numbers is that if we represent \TT{0xFFFFFFFE} and \TT{0x0000002} 
as unsigned, then first number ($4294967294$) is bigger than second ($2$). 
If to represent them both as signed, first will be $-2$, and it is lesser than second ($2$). 
That is the reason why conditional jumps~(\ref{sec:Jcc}) are present both for signed (e.g. \JG, \JL) 
and unsigned (\JA, \JB) operations.} \\
\\
\IFRU{Для простоты, вот что нужно знать}{For the sake of simplicity, that is what one need to know}:
\begin{itemize}
\item \IFRU{Числа бывают знаковые и беззнаковые}{Number can be signed or unsigned}.
\item \IFRU{Знаковые типы в \CCpp}{\CCpp signed types}:
	\TT{int} (-2147483646..2147483647 \OrENRU \TT{0x80000000..0x7FFFFFFF}),
	\TT{char} (-127..128 \OrENRU \TT{0x7F..0x80}).
	\IFRU{Беззнаковые}{Unsigned}: \TT{unsigned int} (0..4294967295 \OrENRU \TT{0..0xFFFFFFFF}),
	\TT{unsigned char} (0..255 \OrENRU \TT{0..0xFF}),
	\TT{size\_t}.
\item \IFRU{У знаковых чисел знак определяется самым старшим битом: 1 означает ``минус'', 0 означает ``плюс''}
	{Signed types has sign in the most significant bit: 1 mean ``minus'', 0 mean ``plus''}.
\item \IFRU{Инструкции сложения и вычитания работают одинаково хорошо и для знаковых и для беззнаковых значений}
	{Addition and subtraction operations are working well for both signed and unsigned values}.
	\IFRU{Но для операций умножения и деления, в x86 имеются разные инструкции}
	{But for multiplication and division operations, x86 has different instructions}:
	\TT{IDIV}/\TT{IMUL} \IFRU{для знаковых}{for signed}
	\AndENRU \TT{DIV}/\TT{MUL} \IFRU{для беззнаковых}{for unsigned}.
\item \IFRU{Еще инструкции работающие с знаковыми числами}{More instructions working with signed numbers}:
	\TT{CBW/CWD/CWDE/CDQ/CDQE} (\ref{ins:CBW_CWD_etc}), \TT{MOVSX} (\ref{MOVSX}), \TT{SAR} (\ref{ins:SAR}).
\end{itemize}

\subsection{\IFRU{Переполнение integer}{Integer overflow}}

\IFRU{Бывает так, что ошибки представления знаковых/беззнаковых могут привести к уязвимости 
\IT{переполнение integer}.}
{It is worth noting that incorrect representation of number can lead integer overflow vulnerability.}

\IFRU{Например, есть некий сервис, который принимает по сети некие пакеты. 
В пакете есть заголовок где указана длина пакета. Это 32-битное значение. 
В процессе приема пакета, 
сервис проверяет это значение и сверяет, больше ли оно чем максимальный размер пакета, скажем, константа
\TT{MAX\_PACKET\_SIZE} (например, 10 килобайт), и если да, то пакет отвергается как некорректный. 
Сравнение знаковое. Злоумышленник подставляет значение \TT{0xFFFFFFFF}. Это число трактуется как знаковое $-1$ 
и оно меньше чем $10000$. Проверка проходит. Продолжаем дальше и копируем этот пакет куда-нибудь себе 
в сегмент данных\dots вызов функции \TT{memcpy (dst, src, 0xFFFFFFFF)} скорее всего, 
затрет много чего внутри процесса.}
{For example, we have a network service, it receives network packets. 
In the packets there is also a field where subpacket length is coded. 
It is 32-bit value. 
After network packet received, service checking the field, and if it is larger than, 
e.g. some \TT{MAX\_PACKET\_SIZE} (let's say, 10 kilobytes), the packet is rejected as incorrect.
Comparison is signed. Intruder set this value to the \TT{0xFFFFFFFF}.
While comparison, this number is considered as signed $-1$ and it is lesser than 10 kilobytes. 
No error here. 
Service would like to copy the subpacket to another place in memory and call 
\TT{memcpy (dst, src, 0xFFFFFFFF)} function: this operation, rapidly garbling a lot of 
inside of process memory.}

\IFRU{Немного подробнее}{More about it}: \cite{Phrack3C0A}.


\input{calling_conventions}
\section{\CapitalPICcode}
\index{\PICcode}
\label{sec:PIC}

\IFRU{Во время анализа динамических библиотек (.so) в Linux, часто можно заметить такой шаблонный код}{While analyzing Linux shared (.so) libraries, one may frequently spot such code pattern}:

\begin{lstlisting}[caption=libc-2.17.so x86]
.text:0012D5E3 __x86_get_pc_thunk_bx proc near         ; CODE XREF: sub_17350+3
.text:0012D5E3                                         ; sub_173CC+4 ...
.text:0012D5E3                 mov     ebx, [esp+0]
.text:0012D5E6                 retn
.text:0012D5E6 __x86_get_pc_thunk_bx endp

...

.text:000576C0 sub_576C0       proc near               ; CODE XREF: tmpfile+73

...

.text:000576C0                 push    ebp
.text:000576C1                 mov     ecx, large gs:0
.text:000576C8                 push    edi
.text:000576C9                 push    esi
.text:000576CA                 push    ebx
.text:000576CB                 call    __x86_get_pc_thunk_bx
.text:000576D0                 add     ebx, 157930h
.text:000576D6                 sub     esp, 9Ch

...

.text:000579F0                 lea     eax, (a__gen_tempname - 1AF000h)[ebx] ; "__gen_tempname"
.text:000579F6                 mov     [esp+0ACh+var_A0], eax
.text:000579FA                 lea     eax, (a__SysdepsPosix - 1AF000h)[ebx] ; "../sysdeps/posix/tempname.c"
.text:00057A00                 mov     [esp+0ACh+var_A8], eax
.text:00057A04                 lea     eax, (aInvalidKindIn_ - 1AF000h)[ebx] ; "! \"invalid KIND in __gen_tempname\""
.text:00057A0A                 mov     [esp+0ACh+var_A4], 14Ah
.text:00057A12                 mov     [esp+0ACh+var_AC], eax
.text:00057A15                 call    __assert_fail
\end{lstlisting}

\IFRU{Все указатели на строки корректируются при помощи некоторой константы из регистра \EBX, которая вычисляется в начале каждой функции.}
{All pointers to strings are corrected by a constant and by value in the \EBX,
which calculated at the beginning of each function.}
\IFRU{Это так называемый адресно-независимый код (\ac{PIC}), он предназначен для исполнения будучи расположенным по любому адресу в памяти, вот почему он не содержит никаких абсолютных адресов в памяти}
{This is so called \ac{PIC}, it is intended to execute placed at any random point of memory, that is why it cannot contain any absolute memory addresses}.

\IFRU{\ac{PIC} был очень важен в ранних компьютерных системах и важен сейчас во встраиваемых\footnote{embedded}, не имеющих поддержки виртуальной памяти (все процессы расположены в одном непрерывном блоке памяти)}
{\ac{PIC} was 
crucial in early computer systems and crucial now in embedded systems without 
virtual memory support (where processes are all placed in single continous memory block)}.
\IFRU{Он до сих пор используется в *NIX системах для динамических библиотек, потому что динамическая библиотека может использоваться одновременно в нескольких процессах, будучи загружена в память только один раз}
{It is also still used in *NIX systems for shared libraries since shared libraries 
are shared across many processes while loaded in memory only once}.
\IFRU{Но все эти процессы могут загрузить одну и ту же динамическую библиотеку по разным адресам, вот почему динамическая библиотека должна работать корректно не привыязываясь к абсолютным адресам}{But all these processes may 
map the same shared library on different addresses, so that is why
shared library should be working correctly without fixing on any absolute address}.

\IFRU{Простой эксперимент}{Let's do a simple experiment}:

\begin{lstlisting}
#include <stdio.h>

int global_variable=123;

int f1(int var)
{
    int rt=global_variable+var;
    printf ("returning %d\n", rt);
    return rt;
};
\end{lstlisting}

\IFRU{Скомпилируем в GCC 4.7.3 и посмотрим итоговый файл .so в}{Let's compile it in GCC 4.7.3 and see resulting .so file in} \IDA:

\begin{lstlisting}
gcc -fPIC -shared -O3 -o 1.so 1.c
\end{lstlisting}

\begin{lstlisting}[caption=GCC 4.7.3]
.text:00000440                 public __x86_get_pc_thunk_bx
.text:00000440 __x86_get_pc_thunk_bx proc near         ; CODE XREF: _init_proc+4
.text:00000440                                         ; deregister_tm_clones+4 ...
.text:00000440                 mov     ebx, [esp+0]
.text:00000443                 retn
.text:00000443 __x86_get_pc_thunk_bx endp

.text:00000570                 public f1
.text:00000570 f1              proc near
.text:00000570
.text:00000570 var_1C          = dword ptr -1Ch
.text:00000570 var_18          = dword ptr -18h
.text:00000570 var_14          = dword ptr -14h
.text:00000570 var_8           = dword ptr -8
.text:00000570 var_4           = dword ptr -4
.text:00000570 arg_0           = dword ptr  4
.text:00000570
.text:00000570                 sub     esp, 1Ch
.text:00000573                 mov     [esp+1Ch+var_8], ebx
.text:00000577                 call    __x86_get_pc_thunk_bx
.text:0000057C                 add     ebx, 1A84h
.text:00000582                 mov     [esp+1Ch+var_4], esi
.text:00000586                 mov     eax, ds:(global_variable_ptr - 2000h)[ebx]
.text:0000058C                 mov     esi, [eax]
.text:0000058E                 lea     eax, (aReturningD - 2000h)[ebx] ; "returning %d\n"
.text:00000594                 add     esi, [esp+1Ch+arg_0]
.text:00000598                 mov     [esp+1Ch+var_18], eax
.text:0000059C                 mov     [esp+1Ch+var_1C], 1
.text:000005A3                 mov     [esp+1Ch+var_14], esi
.text:000005A7                 call    ___printf_chk
.text:000005AC                 mov     eax, esi
.text:000005AE                 mov     ebx, [esp+1Ch+var_8]
.text:000005B2                 mov     esi, [esp+1Ch+var_4]
.text:000005B6                 add     esp, 1Ch
.text:000005B9                 retn
.text:000005B9 f1              endp
\end{lstlisting}

\newcommand{\retstring}{\IT{<<returning \%d\textbackslash{}n>>}}
\newcommand{\globvar}{\IT{global\_variable}}

\IFRU{Так и есть: указатели на строку \retstring{} и переменную \globvar{} корректируются при каждом исполнении функции}
{That's it: pointers to \retstring{} string and \globvar{} are to be corrected at each function execution.}
\IFRU{Функция}{The} \IT{\_\_x86\_get\_pc\_thunk\_bx()} \IFRU{возвращает адрес точки после вызова самой себя (здесь: \TT{0x57C}) в}{function return address of the point after call to itself (\TT{0x57C} here) in the} \EBX.
\IFRU{Это очень простой способ получить значение указателя на текущую инструкцию (\EIP) в произвольном месте}
{That's the simple way to get value of program counter (\EIP) at some point}.
\IFRU{Константа}{The} \IT{0x1A84} \IFRU{связана с разницей между началом этой функции и так называемой}{constant is related to the difference between this function begin and so called}
\IT{Global Offset Table Procedure Linkage Table} (GOT PLT), \IFRU{секцией, сразу же за}{the section right after} \IT{Global Offset Table} (GOT), \IFRU{где находится указатель на \globvar{}}{where pointer to \globvar{} is}.
\IDA \IFRU{показыавет смещения уже обработанными, чтобы их было проще понимать, но на самом деле код такой}{shows these offset processed, so to understand them easily, but in fact the code is}:

\begin{lstlisting}
.text:00000577                 call    __x86_get_pc_thunk_bx
.text:0000057C                 add     ebx, 1A84h
.text:00000582                 mov     [esp+1Ch+var_4], esi
.text:00000586                 mov     eax, [ebx-0Ch]
.text:0000058C                 mov     esi, [eax]
.text:0000058E                 lea     eax, [ebx-1A30h]
\end{lstlisting}

\IFRU{Так что, \EBX указывает на секцию \TT{GOT PLT} и для вычисления указателя на \globvar{}, которая хранится в \TT{GOT}, нужно вычесть 0xC}{So, \EBX pointing to the \TT{GOT PLT} section and to calculate pointer to \globvar{} which stored in 
the \TT{GOT}, \TT{0xC} must be subtracted}.
\IFRU{А чтобы вычислить указатель на \retstring{}, нужно вычесть \TT{0x1A30}}
{To calculate pointer to the \retstring{} string, \TT{0x1A30} must be subtracted}.

\index{x86-64}
\index{x86!\Registers!RIP}
\IFRU{Кстати, вот зачем в AMD64 появилась поддержка адресации относительно RIP\footnote{указатель инструкций в AMD64}, просто для упрощения PIC-кода}
{By the way, that is the reason why AMD64 instruction set supports RIP\footnote{program counter in AMD64}-relative addressing, just to simplify PIC-code}.

\IFRU{Скомпилируем тот же код на Си при помощи той же версии GCC, но для x64}{Let's compile the same C code in the same GCC version, but for x64}.

\index{objdump}
\IDA \IFRU{упростит код на выходе убирая упоминания RIP, так что я буду использовать \IT{objdump} вместо}
{would simplify output code but suppressing RIP-relative addressing details, so I will run \IT{objdump} instead to see the details}:

\begin{lstlisting}
0000000000000720 <f1>:
 720:	48 8b 05 b9 08 20 00 	mov    rax,QWORD PTR [rip+0x2008b9]        # 200fe0 <_DYNAMIC+0x1d0>
 727:	53                   	push   rbx
 728:	89 fb                	mov    ebx,edi
 72a:	48 8d 35 20 00 00 00 	lea    rsi,[rip+0x20]        # 751 <_fini+0x9>
 731:	bf 01 00 00 00       	mov    edi,0x1
 736:	03 18                	add    ebx,DWORD PTR [rax]
 738:	31 c0                	xor    eax,eax
 73a:	89 da                	mov    edx,ebx
 73c:	e8 df fe ff ff       	call   620 <__printf_chk@plt>
 741:	89 d8                	mov    eax,ebx
 743:	5b                   	pop    rbx
 744:	c3                   	ret    
\end{lstlisting}

\TT{0x2008b9} \IFRU{это разница между адресом инструкции по \TT{0x720} и \globvar{}, 
а \TT{0x20} это разница между инструкцией по \TT{0x72A} и строкой \retstring{}}
{is the difference between address of instruction at \TT{0x720} and \globvar{} and 
\TT{0x20} is the difference between tha address of the instruction at 
\TT{0x72A} and the \retstring{} string}.

\IFRU{Такой механизм не используется в Windows DLL. Если загрузчику в Windows приходится загружать DLL 
в другое место, он ``патчит'' DLL прямо в памяти (на местах \IT{FIXUP}-ов) чтобы скорректировать 
все адреса.}{The PIC mechanism is not used in Windows DLLs. If Windows loader needs to load DLL 
on another base address, it ``patches'' DLL in memory (at the \IT{FIXUP} places) in order to correct 
all addresses.}
\IFRU{Это приводит к тому что загруженную один раз DLL нельзя использовать одновременно в разных 
процессах, желающих расположить её по разным адресам ~--- потому что каждый загруженный в память 
экземпляр DLL \IT{доводится} до того чтобы работать только по этим адресам.}{This means, several 
Windows processes cannot share once loaded DLL on different addresses in different process' memory 
blocks ~--- since each loaded into memory DLL instance \IT{fixed} to be work only at these addresses..}


\chapter{Thread Local Storage}
\label{TLS}
\index{TLS}

\RU{Это область данных, отдельная для каждого треда. Каждый тред может хранить там то, что ему нужно}
\EN{It is a data area, specific to each thread. Every thread can store there what it needs}.
\RU{Один из известных примеров, это стандартная глобальная переменная в Си}\EN{One famous example
is C standard global variable} \IT{errno}. 
\RU{Несколько тредов одновременно могут вызывать функции
возвращающие код ошибки в \IT{errno}, поэтому глобальная переменная здесь не будет работать корректно, 
для мультитредовых программ \IT{errno} нужно хранить в в \ac{TLS}.}
\EN{Multiple threads may simultaneously call a functions
which returns error code in the \IT{errno}, so global variable will not work correctly here, for multi-thread programs,
\IT{errno} must be stored in the \ac{TLS}.} \\
\\
\index{\Cpp!C++11}
\RU{В}\EN{In the} C++11 \RU{ввели модификатор}\EN{standard, a new} \IT{thread\_local} 
\RU{, показывающий что каждый тред будет иметь свою версию этой переменной}
\EN{modifier was added, showing that each thread will have its own version of the variable},
\RU{и её можно инициализировать, и она расположена в}\EN{it can be initialized, and it is located in the} \ac{TLS}
\footnote{
\index{C11}
\RU{В C11 также есть поддержка тредов, хотя и опциональная}
\EN{C11 also has thread support, optional though}}:

\begin{lstlisting}[caption=C++11]
#include <iostream>
#include <thread>

thread_local int tmp=3;

int main()
{
	std::cout << tmp << std::endl;
};
\end{lstlisting}
\footnote{\RU{Компилируется в}\EN{Compiled in} MinGW GCC 4.8.1, \RU{но не в}\EN{but not in} MSVC 2012}

\RU{Если говорить о PE-файлах, то в исполняемом файле значение}
\EN{If to say about PE-files, in the resulting executable file, the} \IT{tmp} 
\RU{будет именно в секции отведенной}\EN{variable will be stored in the section devoted to}
\ac{TLS}.

\section{\RU{Трюк с }\IT{LD\_PRELOAD}\EN{ hack} \InENRU Linux}

\index{LD\_PRELOAD}
\label{ld_preload}

\IFRU{Это позволяет загружать свои динамические библиотеки перед другими, даже перед системными,
такими как}
{This allows us to load our own dynamic libraries before others, even before system ones, like} libc.so.6.

\IFRU{Что в свою очередь, позволяет ``подставлять'' написанные нами ф-ции перед оригинальными из системных библиотек.}
{What, in turn, allows to ``substitute'' our written functions before original ones in system libraries.}
\IFRU{Например, легко перехватывать все вызовы к}{For example, it is easy to intercept all calls to the} 
time(), read(), write(), \IFRU{и т.д}{etc}. \\
\\
\index{uptime}
\IFRU{Попробуем узнать, сможем ли мы обмануть утилиту \IFRU{uptime}}{Let's see, if we are able to fool
\IT{uptime} utility}.
\IFRU{Как известно, она сообщает, как долго компьютер работает}{As we know, it tells how long the computer
is working}.
\index{strace}
\IFRU{При помощи}{With the help of} strace(\ref{strace}), \IFRU{можно увидеть, что эту информацию утилита получает из файла}
{it is possible to see that this information the utility takes from the} \TT{/proc/uptime}
\EN{ file}:

\begin{lstlisting}
$ strace uptime 
...
open("/proc/uptime", O_RDONLY)          = 3
lseek(3, 0, SEEK_SET)                   = 0
read(3, "416166.86 414629.38\n", 2047)  = 20
...
\end{lstlisting}

\IFRU{Это не реальный файл на диске, это виртуальный файл,
содержимое которого генерируется на лету в ядре Linux.}
{It is not a real file on disk, it is a virtual one, its contents is generated on fly in Linux kernel.}
\IFRU{Там просто два числа}
{There are just two numbers}:

\begin{lstlisting}
$ cat /proc/uptime
416690.91 415152.03
\end{lstlisting}

\IFRU{Из wikipedia, можно узнать}{What we can learn from wikipedia}:

\begin{framed}
\begin{quotation}
The first number is the total number of seconds the system has been up.
The second number is how much of that time the machine has spent idle, in seconds.
\end{quotation}
\end{framed}\footnote{\url{https://en.wikipedia.org/wiki/Uptime}}

\index{\CStandardLibrary!open()}
\index{\CStandardLibrary!read()}
\index{\CStandardLibrary!close()}
\IFRU{Попробуем написать свою динамическую библиотеку, в которой будет}
{Let's try to write our own dynamic library with the} open(), read(), close() 
\IFRU{с нужной нам функциональностью}{functions working as we need}.

\IFRU{Во-первых, наш open() будет сравнивать имя открываемого файла с тем что нам нужно, и если да, 
то будет запоминать дескриптор открытого файла.}
{At first, our open() will compare name of file to be opened with what we need and if it is so,
it will write down the descriptor of the file opened.}
\IFRU{Во-вторых, read(), если будет вызываться для этого дескриптора, будет подменять вывод,
а в остальных случаях, будет вызывать настоящий}
{At second, read(), if it will be called for this file descriptor, will substitute output,
and in other cases, will call original} read() \IFRU{из}{from} libc.so.6.
\IFRU{А также}{And also} close(), \IFRU{будет следить, закрывается ли файл за которым мы следим.}
{will note, if the file we are currently follow is to be closed.}

\index{dlopen()}
\index{dlsym()}
\IFRU{Для того чтобы найти адреса настоящих ф-ций в libc.so.6, используем dlopen() и dlsym().}
{We will use the dlopen() and dlsym() functions to determine original addresses of functions in libc.so.6.}

\IFRU{Нам это нужно, потому что нам нужно передавать управление ``настоящим'' ф-циями.}
{We need them because we must pass control to ``real'' functions.}

\index{\CStandardLibrary!strcmp()}
\IFRU{С другой стороны, если бы мы перехватывали, скажем, strcmp(),
и следили бы за всеми сравнениями строк в программе, 
то, наверное, strcmp() можно было бы и самому реализовать, не
пользуясь настоящей ф-цией}
{On the other hand, if we could intercept e.g. strcmp(), and follow each string
comparisons in program, then strcmp() could be implemented easily on one's own, while not
using original function}
\footnote{\IFRU{Например, посмотрите как обеспечивается простейший перехват strcmp()}
{For example, here is how simple strcmp() interception is works} \InENRU
\href{http://yurichev.com/mirrors/LD\_PRELOAD/Yong\%20Huang\%20LD\_PRELOAD.txt}
{\IFRU{статье}{article}} \IFRU{от}{from} Yong Huang}.

\lstinputlisting{LD_PRELOAD/fool_uptime.c}
% FIXME: add URL to github source

\IFRU{Компилируем как динамическую библиотеку}{Let's compile it as common dynamic library}:

\begin{lstlisting}
gcc -fpic -shared -Wall -o fool_uptime.so fool_uptime.c -ldl
\end{lstlisting}

\IFRU{Запускаем \IT{uptime}, подгружая нашу библиотеку перед остальными}{Let's run \IT{uptime}
while loading our library before others}:

\begin{lstlisting}
LD_PRELOAD=`pwd`/fool_uptime.so uptime
\end{lstlisting}

\IFRU{Видим такое}{And we see}:

\begin{lstlisting}
 01:23:02 up 24855 days,  3:14,  3 users,  load average: 0.00, 0.01, 0.05
\end{lstlisting}

\IFRU{Если переменная окружения}{If the} \IT{LD\_PRELOAD} 
\IFRU{будет всегда указывать на путь и имя файла нашей библиотеки, то она будет
загружаться для всех запускаемых программ.}
{environment variable will always points to filename and path of our library, it will be loaded
for all starting programs.} \\
\\
\IFRU{Еще примеры}{More examples}:

\begin{itemize}
\IFRU{\item
\IFRU{Перехват}{Intercepting} time() \InENRU Sun Solaris \url{http://yurichev.com/mirrors/LD_PRELOAD/sun_hack.txt}
}{}

\item
\IFRU{Очень простой перехват}{Very simple interception of the} strcmp() (Yong Huang) 
\url{http://yurichev.com/mirrors/LD\_PRELOAD/Yong\%20Huang\%20LD\_PRELOAD.txt}

\item
Kevin Pulo --- Fun with LD\_PRELOAD. \IFRU{Много примеров и идей}{A lot of examples and ideas}.
\url{http://yurichev.com/mirrors/LD_PRELOAD/lca2009.pdf}

\item
\IFRU{Перехват ф-ций работы с файлами для компрессии и декомпрессии файлов на лету}
{File functions interception for compression/decompression files on fly} (zlibc). \url{ftp://metalab.unc.edu/pub/Linux/libs/compression}

\end{itemize}


\chapter{\IFRU{Поиск в коде того что нужно}{Finding important/interesting stuff in the code}}

\IFRU{Современное ПО, в общем-то, минимализмом не отличается.}{Minimalism it's not a significant feature
of modern software.}

\IFRU{Но не потому, что программисты слишком много пишут, 
а потому что к исполняемым файлам обыкновенно прикомпилируют все подряд библиотеки. 
Если бы все вспомогательные библиотеки всегда выносили во внешние DLL, мир был бы иным.
(Еще одна причина для Си++ ~--- STL и прочие библиотеки шаблонов.)}
{But not because programmers writting a lot, but because all libraries are usually linked statically
to executable files.
If all external libraries were shifted into external DLL files, the world would be different.
(Another reason for C++ ~--- STL and other template libraries.)}

\newcommand{\FOOTNOTEBOOST}{\footnote{\url{http://www.boost.org/}}}
\newcommand{\FOOTNOTELIBPNG}{\footnote{\url{http://www.libpng.org/pub/png/libpng.html}}}

\IFRU{Таким образом, очень полезно сразу понимать, какая функция из стандартной библиотеки или 
более-менее известной (как Boost\FOOTNOTEBOOST, libpng\FOOTNOTELIBPNG), 
а какая ~--- имеет отношение к тому что мы пытаемся найти в коде.}
{Thus, it's very important to determine origin of some function, if it's from standard library or 
well-known library (like Boost\FOOTNOTEBOOST, libpng\FOOTNOTELIBPNG),
and which one ~--- is related to what we are trying to find in the code.}

\IFRU{Переписывать весь код на \CCpp, чтобы разобраться в нем, безусловно, не имеет никакого смысла.}
{It's just absurdly to rewrite all code to \CCpp to find what we looking for.}

\IFRU{Одна из важных задач reverse engineer-а это быстрый поиск в коде того что собственно его интересует.}
{One of the primary reverse engineer's task is to find quickly in the code what is needed.}

\index{\GrepUsage}
\IFRU{Дизассемблер \IDA позволяет делать поиск как минимум строк, последовательностей байт, констант.
Можно даже сделать экспорт кода в текстовый файл .lst или .asm и затем натравить на него \TT{grep}, \TT{awk}, итд.}
{\IDA disassembler allow us search among text strings, byte sequences, constants.
It's even possible to export the code into .lst or .asm text file and then use \TT{grep}, \TT{awk}, etc.}

\IFRU{Когда вы пытаетесь понять, что делает тот или иной код, это запросто может быть какая-то 
опенсорсная библиотека вроде libpng. Поэтому когда находите константы, или текстовые строки которые 
выглядят явно знакомыми, всегда полезно их погуглить.
А если вы найдете искомый опенсорсный проект где это используется, 
то тогда будет достаточно будет просто сравнить вашу функцию с ней. 
Это решит часть проблем.}
{When you try to understand what some code is doing, this easily could be some open-source library like libpng.
So when you see some constants or text strings looks familiar, it's always worth to google it.
And if you find the opensource project where it's used, 
then it will be enough just to compare the functions. It may solve some part of problem.}

\IFRU{К примеру, если программа использует какие-то XML-файлы, первым шагом может быть
установление, какая именно XML-библиотека для этого используется, ведь часто используется какая-то
стандартная (или очень известная) вместо самодельной.}
{For example, if program use some XML files, the first step may be determining, which
XML-library is used for processing, because, standard (or well-known) library is often used
instead of self-made one.}

\index{SAP}
\index{PDB}
\IFRU{К примеру, однажды я пытался разобраться как происходит компрессия/декомпрессия сетевых пакетов в SAP 6.0. 
Это очень большая программа, но к ней идет подробный .PDB-файл с отладочной информацией, и это очень удобно. 
Я в конце концов пришел к тому что одна из функций декомпрессирующая пакеты называется CsDecomprLZC(). 
Не сильно раздумывая, я решил погуглить и оказалось что функция с таким же названием имеется в MaxDB
(это опен-сорсный проект SAP)\footnote{Больше об этом в соответствующей секции~\ref{sec:SAPGUI}}.}
{For example, once upon a time I tried to understand how SAP 6.0 network packets compression/decompression 
is working.
It's a huge software, but a detailed .PDB with debugging information is present, and that's cosily.
I finally came to idea that one of the functions doing decompressing of network packet called CsDecomprLZC().
Immediately I tried to google its name and I quickly found that the function named as the same is used in MaxDB
(it's open-source SAP project)\footnote{More about it in releval section~\ref{sec:SAPGUI}}.}

\url{http://www.google.com/search?q=CsDecomprLZC}

\IFRU{Каково же было мое удивление, когда оказалось, что в MaxDB используется точно такой же алгоритм, 
скорее всего, с таким же исходником.}
{Astoundingly, MaxDB and SAP 6.0 software shared the same code for network packets compression/decompression.}

\section{\IFRU{Связь с внешним миром}{Communication with the outer world}}

\IFRU{Первое на что нужно обратить внимание, это какие функции из API операционной 
системы и какие функции стандартных библиотек используются.}
{First what to look on is which functions from operation system API and standard libraries are used.}

\IFRU{Если программа поделена на главный исполняемый файл и группу DLL-файлов, 
то имена функций в этих DLL, бывает так, могут помочь.}
{If the program is divided into main executable file and a group of DLL-files, sometimes,
these function's names may be helpful.}

\IFRU{Если нас интересует, что именно приводит к вызову \TT{MessageBox()} с определенным текстом, 
то первое что можно попробовать сделать: найти в сегменте данных этот текст, найти ссылки на него, и найти, 
откуда может передаться управление к интересующему нас вызову \TT{MessageBox()}.}
{If we are interesting, what exactly may lead to \TT{MessageBox()} call with specific text,
first what we can try to do: find this text in data segment, find references to it and find the points
from which a control may be passed to \TT{MessageBox()} call we're interesting in.}

\IFRU{Если речь идет об игре, и нам интересно какие события в ней более-менее случайны, 
мы можем найти функцию \rand или её заменитель (как алгоритм Mersenne twister), и посмотреть, 
из каких мест эта функция вызывается и что самое главное: как используется результат этой функции.}
{If we are talking about some game and we're interesting, which events are more or less random in it,
we may try to find \rand function or its replacement (like Mersenne twister algorithm) and find a places
from which this function called and most important: how the results are used.}

\IFRU{Но если это не игра, а \rand используется, то также весьма любопытно, зачем. 
Бывают неожиданные случаи вроде использования \rand в алгоритме для сжатия данных (для имитации шифрования):}
{But if it's not a game, but \rand is used, it's also interesing, why.
There are cases of unexpected \rand usage in data compression algorithm (for encryption imitation):}
\url{http://blog.yurichev.com/node/44}.


\section{\IFRU{Строки}{Strings}}
\label{sec:digging_strings}

\IFRU{Очень сильно помогают отладочные сообщения, если они имеются. В некотором смысле, отладочные сообщения, 
это отчет о том, что сейчас происходит в программе. Зачастую, это \printf-подобные функции, 
которые пишут куда-нибудь в лог, а бывает так что и не пишут ничего, но вызовы остались, так как эта сборка ~--- 
не отладочная, а release.}
{Debugging messages are often very helpful if present.
In some sense, debugging messages are reporting
about what's going on in program right now. Often these are \printf-like functions,
which writes to log-files, and sometimes, not writing anything but calls are still present 
since this build is not a debug build but release one.}
\index{\oracle}
\IFRU{Если в отладочных сообщениях дампятся значения некоторых локальных или глобальных переменных, 
это тоже может помочь, как минимум, узнать их имена. 
Например, в \oracle одна из таких функций: \TT{ksdwrt()}.}
{If local or global variables are dumped in debugging messages, it might be helpful as well 
since it is possible to get variable names at least.
For example, one of such functions in \oracle is \TT{ksdwrt()}.}

\newcommand{\CONUSONE}{http://blog.yurichev.com/node/32}
\newcommand{\CONUSTWO}{http://blog.yurichev.com/node/43}

\IFRU{Осмысленные текстовые строки вообще очень сильно могут помочь. 
Дизассемблер \IDA может сразу указать, из какой функции и из какого её места используется эта строка. 
Встречаются и \href{\CONUSONE}{смешные случаи}.}
{Meaningful text strings are often helpful.
\IDA disassembler may show from which function and from which point this specific string is used.
Funny cases \href{\CONUSONE}{sometimes happen}.}

\IFRU{Сообщения об ошибках также могут помочь найти то что нужно. 
В \oracle сигнализация об ошибках проходит при помощи вызова некоторой группы функций. 
\href{\CONUSTWO}{Тут еще немного об этом}.}
{Error messages may help us as well.
In \oracle, errors are reporting using group of functions.
\href{\CONUSTWO}{More about it}.}

\index{Error messages}
\IFRU{Можно довольно быстро найти, какие функции сообщают о каких ошибках, и при каких условиях.}
{It is possible to find very quickly, which functions reporting about errors and in which conditions.}
\IFRU{Это, кстати, одна из причин, почему в защите софта от копирования, 
бывает так, что сообщение об ошибке заменяется 
невнятным кодом или номером ошибки. Мало кому приятно, если взломщик быстро поймет, 
из за чего именно срабатывает защита от копирования, просто по сообщению об ошибке.}
{By the way, it is often a reason why copy-protection systems has inarticulate cryptic error messages 
or just error numbers. No one happy when software cracker quickly understand why copy-protection
is triggered just by error message.}

% TODO software protection... set ref to section about dongle for SCO UNIX...


\chapter{\RU{Вызовы assert()}\EN{Calls to assert()}}
\index{\CStandardLibrary!assert()}
\RU{Может также помочь наличие \TT{assert()} в коде: обычно этот макрос оставляет название файла-исходника, 
номер строки, и условие.}
\EN{Sometimes \TT{assert()} macro presence is useful too: 
commonly this macro leaves source file name, line number and condition in code.}

\RU{Наиболее полезная информация содержится в assert-условии, по нему можно судить по именам переменных
или именам полей структур. Другая полезная информация ~--- это имена файлов, по их именам можно попытаться
предположить, что там за код. Также, по именам файлов можно опознать какую-либо очень известную опен-сорсную
библиотеку.}
\EN{Most useful information is contained in assert-condition, we can deduce variable names, or structure field
names from it. Another useful piece of information is file names~---we can try to deduce what type of
code is here.
Also by file names it is possible to recognize a well-known open-source libraries.}

\lstinputlisting[caption=\RU{Пример информативных вызовов assert()}
\EN{Example of informative assert() calls}]{digging_into_code/assert_examples.lst}

\RU{Полезно ``гуглить'' и условия и имена файлов, это может вывести вас к опен-сорсной бибилотеке.
Например, если ``погуглить'' ``sp->lzw\_nbits <= BITS\_MAX'', 
это вполне предсказуемо выводит на опенсорсный код, что-то связанное с LZW-компрессией.}
\EN{It is advisable to ``google'' both conditions and file names, that may lead us to open-source library.
For example, if to ``google'' ``sp->lzw\_nbits <= BITS\_MAX'', this predictably 
give us some open-source code, something related to LZW-compression.}


\chapter{\RU{Константы}\EN{Constants}}

\RU{Люди, включая программистов, часто используют круглые числа вроде}
\EN{Humans, including programmers, often use round numbers like} 10, 100, 1000, 
\RU{в т.ч. и в коде}\EN{in real life as well as in the code}.

\RU{Практикующие реверсеры, обычно, хорошо знают их в шестнадцатеричном представлении}
\EN{The practicing reverse engineer usually know them well in hexadecimal representation}:
10=0xA, 100=0x64, 1000=0x3E8, 10000=0x2710.

\RU{Иногда попадаются константы}\EN{The constants} \TT{0xAAAAAAAA} (10101010101010101010101010101010) \AndENRU \\
\TT{0x55555555} (01010101010101010101010101010101) \RU{\EMDASH{}это чередующиеся биты}\EN{ are also popular\EMDASH{}those
are composed of alternating bits}.
\RU{Это помогает отличить некоторый сигнал от сигнала где все биты включены (1111 \dots) или выключены (0000 \dots).}
\EN{That may help to distinguish some signal from the signal where all bits are turned on (1111 \dots) or off (0000 \dots).}
\RU{Например, константа}\EN{For example, the} \TT{0x55AA} \RU{используется как минимум в бут-секторе}\EN{constant
is used at least in the boot sector}, \ac{MBR}, 
\AndENRU \InENRU \EN{the }\ac{ROM} \RU{плат-расширений IBM-компьютеров}\EN{of IBM-compatible extension cards}.

\RU{Некоторые алгоритмы, особенно криптографические, используют хорошо различимые константы, 
которые при помощи \IDA легко находить в коде.}
\EN{Some algorithms, especially cryptographical ones use distinct constants, which are easy to find
in code using \IDA.}

\index{MD5}
\newcommand{\URLMD}{\RU{http://go.yurichev.com/17110}\EN{http://go.yurichev.com/17111}}

\RU{Например, алгоритм MD5\footnote{\href{\URLMD}{wikipedia}} инициализирует свои внутренние переменные так:}
\EN{For example, the MD5\footnote{\href{\URLMD}{wikipedia}} algorithm initializes its own internal variables like this:}

\begin{verbatim}
var int h0 := 0x67452301
var int h1 := 0xEFCDAB89
var int h2 := 0x98BADCFE
var int h3 := 0x10325476
\end{verbatim}

\RU{Если в коде найти использование этих четырех констант подряд\EMDASH{} очень высокая вероятность что эта функция имеет отношение к MD5.}
\EN{If you find these four constants used in the code in a row, it is very highly probable that this function is related to MD5.}\PTBRph{}\ESph{}\PLph{}\ITAph{} \\
\\
\RU{Еще такой пример это алгоритмы CRC16/CRC32, часто, алгоритмы вычисления контрольной суммы по CRC 
используют заранее заполненные таблицы, вроде}\EN{Another example are the CRC16/CRC32 algorithms, 
whose calculation algorithms often use precomputed tables like this one}:

\begin{lstlisting}[caption=linux/lib/crc16.c]
/** CRC table for the CRC-16. The poly is 0x8005 (x^16 + x^15 + x^2 + 1) */
u16 const crc16_table[256] = {
	0x0000, 0xC0C1, 0xC181, 0x0140, 0xC301, 0x03C0, 0x0280, 0xC241,
	0xC601, 0x06C0, 0x0780, 0xC741, 0x0500, 0xC5C1, 0xC481, 0x0440,
	0xCC01, 0x0CC0, 0x0D80, 0xCD41, 0x0F00, 0xCFC1, 0xCE81, 0x0E40,
	...
\end{lstlisting}

\ifx\LITE\undefined
\RU{См. также таблицу CRC32}\EN{See also the precomputed table for CRC32}: \myref{sec:CRC32}.
\fi

\section{Magic numbers}

\newcommand{\FNURLMAGIC}{\footnote{\href{http://go.yurichev.com/17112}{wikipedia}}}

\RU{Немало форматов файлов определяет стандартный заголовок файла где используются \IT{magic number}\FNURLMAGIC{}, один или даже несколько.}
\EN{A lot of file formats define a standard file header where a \IT{magic number(s)}\FNURLMAGIC{} is used, single one or even several.}

\index{MS-DOS}
\RU{Скажем, все исполняемые файлы для Win32 и MS-DOS начинаются с двух символов}
\EN{For example, all Win32 and MS-DOS executables start with the two characters} \q{MZ}\footnote{\href{http://go.yurichev.com/17113}{wikipedia}}.

\index{MIDI}
\RU{В начале MIDI-файла должно быть \q{MThd}. Если у нас есть использующая для чего-нибудь MIDI-файлы программа
очень вероятно, что она будет проверять MIDI-файлы на правильность хотя бы проверяя первые 4 байта.}
\EN{At the beginning of a MIDI file the \q{MThd} signature must be present. 
If we have a program which uses MIDI files for something,
it's very likely that it must check the file for validity by checking at least the first 4 bytes.}

\RU{Это можно сделать при помощи:}\EN{This could be done like this:}

\RU{(\IT{buf} указывает на начало загруженного в память файла)}
\EN{(\IT{buf} points to the beginning of the loaded file in memory)}

\begin{lstlisting}
cmp [buf], 0x6468544D ; "MThd"
jnz _error_not_a_MIDI_file
\end{lstlisting}

\index{\CStandardLibrary!memcmp()}
\index{x86!\Instructions!CMPSB}
\RU{\dots либо вызвав функцию сравнения блоков памяти \TT{memcmp()} или любой аналогичный код, 
вплоть до инструкции \TT{CMPSB} 
\ifx\LITE\undefined
(\myref{REPE_CMPSx})
\fi
.}
\EN{\dots or by calling a function for comparing memory blocks like \TT{memcmp()} or any other equivalent code
up to a \TT{CMPSB} 
\ifx\LITE\undefined
(\myref{REPE_CMPSx}) 
\fi
instruction.}

\RU{Найдя такое место мы получаем как минимум информацию о том, где начинается загрузка MIDI-файла, во-вторых, 
мы можем увидеть где располагается буфер с содержимым файла, и что еще оттуда берется, и как используется.}
\EN{When you find such point you already can say where the loading of the MIDI file starts,
also, we could see the location
of the buffer with the contents of the MIDI file, what is used from the buffer, and how.}

\subsection{DHCP}

\RU{Это касается также и сетевых протоколов. 
Например, сетевые пакеты протокола DHCP содержат так называемую \IT{magic cookie}: \TT{0x63538263}. 
Какой-либо код, генерирующий пакеты по протоколу DHCP где-то и как-то должен внедрять в пакет также и эту константу. 
Найдя её в коде мы сможем найти место где происходит это и не только это. 
Любая программа, получающая DHCP-пакеты, должна где-то как-то проверять \IT{magic cookie}, 
сравнивая это поле пакета с константой.}
\EN{This applies to network protocols as well.
For example, the DHCP protocol's network packets contains the so-called \IT{magic cookie}: \TT{0x63538263}.
Any code that generates DHCP packets somewhere must embed this constant into the packet.
If we find it in the code we may find where this happens and, not only that.
Any program which can receive DHCP packet must verify the \IT{magic cookie}, comparing it with the constant.}

\RU{Например, берем файл dhcpcore.dll из Windows 7 x64 и ищем эту константу. 
И находим, два раза: оказывается, эта константа используется в функциях с красноречивыми 
названиями}
\EN{For example, let's take the dhcpcore.dll file from Windows 7 x64 and search for the constant.
And we can find it, twice:
it seems that the constant is used in two functions with descriptive names 
like} \TT{DhcpExtractOptionsForValidation()} \AndENRU \TT{DhcpExtractFullOptions()}:

\begin{lstlisting}[caption=dhcpcore.dll (Windows 7 x64)]
.rdata:000007FF6483CBE8 dword_7FF6483CBE8 dd 63538263h          ; DATA XREF: DhcpExtractOptionsForValidation+79
.rdata:000007FF6483CBEC dword_7FF6483CBEC dd 63538263h          ; DATA XREF: DhcpExtractFullOptions+97
\end{lstlisting}

\RU{А вот те места в функциях где происходит обращение к константам:}
\EN{And here are the places where these constants are accessed:}

\begin{lstlisting}[caption=dhcpcore.dll (Windows 7 x64)]
.text:000007FF6480875F  mov     eax, [rsi]
.text:000007FF64808761  cmp     eax, cs:dword_7FF6483CBE8
.text:000007FF64808767  jnz     loc_7FF64817179
\end{lstlisting}

\RU{И:}\EN{And:}

\begin{lstlisting}[caption=dhcpcore.dll (Windows 7 x64)]
.text:000007FF648082C7  mov     eax, [r12]
.text:000007FF648082CB  cmp     eax, cs:dword_7FF6483CBEC
.text:000007FF648082D1  jnz     loc_7FF648173AF
\end{lstlisting}

\section{\RU{Поиск констант}\EN{Searching for constants}}

\RU{В \IDA это очень просто, Alt-B или Alt-I.}
\EN{It is easy in \IDA: Alt-B or Alt-I.}
\index{binary grep}
\RU{А для поиска константы в большом количестве файлов, либо для поиска их в неисполняемых файлах, 
имеется небольшая утилита}%
\EN{And for searching for a constant in a big pile of files, or for searching in non-executable files,
there is a small utility called}
\IT{binary grep}\footnote{\BGREPURL}.


\chapter{\RU{Поиск нужных инструкций}\EN{Finding the right instructions}}

\RU{Если программа использует инструкции сопроцессора, и их не очень много, 
то можно попробовать вручную проверить отладчиком какую-то из них.}
\EN{If the program is utilizing FPU instructions and there are very few of them in the code,
one can try to check each one manually with a debugger.}\PTBRph{}\ESph{}\PLph{}\ITAph{}\\
\\
\RU{К примеру, нас может заинтересовать, при помощи чего Microsoft Excel считает 
результаты формул, введенных пользователем. Например, операция деления.}
\EN{For example, we may be interested how Microsoft Excel calculates the formulae entered by user.
For example, the division operation.}

\index{\GrepUsage}
\index{x86!\Instructions!FDIV}
\RU{Если загрузить excel.exe (из Office 2010) версии 14.0.4756.1000 в \IDA, затем сделать полный листинг 
и найти все инструкции \FDIV (но кроме тех, которые в качестве второго операнда используют константы\EMDASH{}они, 
очевидно, не подходят нам):}
\EN{If we load excel.exe (from Office 2010) version 14.0.4756.1000 into \IDA, make a full listing
and to find every \FDIV instruction (except the ones which use constants as a second 
operand\EMDASH{}obviously, they do not suit us):}\PTBRph{}\ESph{}\PLph{}\ITAph{}\\

\begin{lstlisting}
cat EXCEL.lst | grep fdiv | grep -v dbl_ > EXCEL.fdiv
\end{lstlisting}

\RU{\dots то окажется, что их всего 144.}\EN{\dots then we see that there are 144 of them.}\PTBRph{}\ESph{}\PLph{}\ITAph{}\\
\\
\RU{Мы можем вводить в Excel строку вроде \TT{=(1/3)} и проверить все эти инструкции.}
\EN{We can enter a string like \TT{=(1/3)} in Excel and check each instruction.}\PTBRph{}\ESph{}\PLph{}\ITAph{}\\
\\
\index{tracer}
\RU{Проверяя каждую инструкцию в отладчике или \tracer 
(проверять эти инструкции можно по 4 за раз), 
окажется, что нам везет и срабатывает всего лишь 14-я по счету:}
\EN{By checking each instruction in a debugger or \tracer
(one may check 4 instruction at a time),
we get lucky and the sought-for instruction is just the 14th:}

\begin{lstlisting}
.text:3011E919 DC 33                                fdiv    qword ptr [ebx]
\end{lstlisting}

\begin{lstlisting}
PID=13944|TID=28744|(0) 0x2f64e919 (Excel.exe!BASE+0x11e919)
EAX=0x02088006 EBX=0x02088018 ECX=0x00000001 EDX=0x00000001
ESI=0x02088000 EDI=0x00544804 EBP=0x0274FA3C ESP=0x0274F9F8
EIP=0x2F64E919
FLAGS=PF IF
FPU ControlWord=IC RC=NEAR PC=64bits PM UM OM ZM DM IM 
FPU StatusWord=
FPU ST(0): 1.000000
\end{lstlisting}

\RU{В \ST{0} содержится первый аргумент (1), второй содержится в}
\EN{\ST{0} holds the first argument (1) and second one is in} \TT{[EBX]}.\\
\\
\index{x86!\Instructions!FDIV}
\RU{Следующая за \FDIV инструкция (\TT{FSTP}) записывает результат в память:}
\EN{The instruction after \FDIV (\TT{FSTP}) writes the result in memory:}\\

\begin{lstlisting}
.text:3011E91B DD 1E                                fstp    qword ptr [esi]
\end{lstlisting}

\RU{Если поставить breakpoint на ней, то мы можем видеть результат:}
\EN{If we set a breakpoint on it, we can see the result:}

\begin{lstlisting}
PID=32852|TID=36488|(0) 0x2f40e91b (Excel.exe!BASE+0x11e91b)
EAX=0x00598006 EBX=0x00598018 ECX=0x00000001 EDX=0x00000001
ESI=0x00598000 EDI=0x00294804 EBP=0x026CF93C ESP=0x026CF8F8
EIP=0x2F40E91B
FLAGS=PF IF
FPU ControlWord=IC RC=NEAR PC=64bits PM UM OM ZM DM IM 
FPU StatusWord=C1 P 
FPU ST(0): 0.333333
\end{lstlisting}

\RU{А также, в рамках пранка\footnote{practical joke}, модифицировать его на лету:}
\EN{Also as a practical joke, we can modify it on the fly:}\PTBRph{}\ESph{}\PLph{}\ITAph{}\\

\begin{lstlisting}
tracer -l:excel.exe bpx=excel.exe!BASE+0x11E91B,set(st0,666)
\end{lstlisting}

\begin{lstlisting}
PID=36540|TID=24056|(0) 0x2f40e91b (Excel.exe!BASE+0x11e91b)
EAX=0x00680006 EBX=0x00680018 ECX=0x00000001 EDX=0x00000001
ESI=0x00680000 EDI=0x00395404 EBP=0x0290FD9C ESP=0x0290FD58
EIP=0x2F40E91B
FLAGS=PF IF
FPU ControlWord=IC RC=NEAR PC=64bits PM UM OM ZM DM IM 
FPU StatusWord=C1 P 
FPU ST(0): 0.333333
Set ST0 register to 666.000000
\end{lstlisting}

\RU{Excel показывает в этой ячейке 666, что окончательно убеждает нас в том, что мы нашли нужное место.}
\EN{Excel shows 666 in the cell, finally convincing us that we have found the right point.}

\begin{figure}[H]
\centering
\includegraphics[scale=\NormalScale]{digging_into_code/Excel_prank.png}
\caption{\RU{Пранк сработал}\EN{The practical joke worked}}
\end{figure}

\RU{Если попробовать ту же версию Excel, только x64, то окажется что там инструкций \FDIV всего 12, 
причем нужная нам\EMDASH{}третья по счету.}
\EN{If we try the same Excel version, but in x64,
we will find only 12 \FDIV instructions there,
and the one we looking for is the third one.}

\begin{lstlisting}
tracer.exe -l:excel.exe bpx=excel.exe!BASE+0x1B7FCC,set(st0,666)
\end{lstlisting}

\index{x86!\Instructions!DIVSD}
\RU{Видимо, все дело в том, что много операций деления переменных типов \Tfloat и \Tdouble 
компилятор заменил на SSE-инструкции вроде \TT{DIVSD}, 
коих здесь теперь действительно много (\TT{DIVSD} присутствует в количестве 268 инструкций).}
\EN{It seems that a lot of division operations of \Tfloat and \Tdouble types, were replaced by the compiler with SSE instructions
like \TT{DIVSD} (\TT{DIVSD} is present 268 times in total).}


\chapter{\RU{Подозрительные паттерны кода}\EN{Suspicious code patterns}}

\section{\RU{Инструкции XOR}\EN{XOR instructions}}
\index{x86!\Instructions!XOR}

\RU{Инструкции вроде}\EN{Instructions like} \TT{XOR op, op} (\RU{например}\EN{for example}, \TT{XOR EAX, EAX}) 
\RU{обычно используются для обнуления регистра,
однако, если операнды разные, то применяется операция именно}\EN{are usually used for setting the register value
to zero, but if the operands are different, the} \q{\RU{исключающего или}\EN{exclusive or}}\EN{ operation
is executed}.
\RU{Эта операция очень редко применяется в обычном программировании, но применяется очень часто в криптографии,
включая любительскую.}
\EN{This operation is rare in common programming, but widespread in cryptography,
including amateur one.}
\RU{Особенно подозрительно, если второй операнд\EMDASH{}это большое число}\EN{It's especially suspicious if the
second operand is a big number}.
\RU{Это может указывать на шифрование, вычисление контрольной суммы,}
\EN{This may point to encrypting/decrypting, checksum computing,}\etc{}.\\
\\
\ifx\LITE\undefined
\RU{Одно из исключений из этого наблюдения о котором стоит сказать, то, что генерация и проверка значения \q{канарейки}
(\myref{subsec:BO_protection}) часто происходит, используя инструкцию \XOR.}
\EN{One exception to this observation worth noting is the \q{canary} (\myref{subsec:BO_protection}). 
Its generation and checking are often done using the \XOR instruction.} \\
\\
\fi
\index{AWK}
\RU{Этот AWK-скрипт можно использовать для обработки листингов (.lst) созданных \IDA{}}
\EN{This AWK script can be used for processing \IDA{} listing (.lst) files}:

\begin{lstlisting}
gawk -e '$2=="xor" { tmp=substr($3, 0, length($3)-1); if (tmp!=$4) if($4!="esp") if ($4!="ebp") { print $1, $2, tmp, ",", $4 } }' filename.lst
\end{lstlisting}

\ifx\LITE\undefined
\RU{Нельзя также забывать,
что если использовать подобный скрипт, то, возможно, он захватит и неверно дизассемблированный
код}\EN{It is also worth noting that this kind of script can also match incorrectly disassembled code} 
(\myref{sec:incorrectly_disasmed_code}).
\fi

\section{\RU{Вручную написанный код на ассемблере}\EN{Hand-written assembly code}}

\index{Function prologue}
\index{Function epilogue}
\index{x86!\Instructions!LOOP}
\index{x86!\Instructions!RCL}
\RU{Современные компиляторы не генерируют инструкции \TT{LOOP} и \TT{RCL}. 
С другой стороны, эти инструкции хорошо знакомы кодерам, предпочитающим писать прямо на ассемблере. 
\ifx\LITE\undefined
Подобные инструкции отмечены как (M) в списке инструкций в приложении: 
\myref{sec:x86_instructions}.
\fi
Если такие инструкции встретились, можно сказать с какой-то вероятностью, что этот фрагмент кода написан вручную.}
\EN{Modern compilers do not emit the \TT{LOOP} and \TT{RCL} instructions.
On the other hand, these instructions are well-known to coders who like to code directly in assembly language.
If you spot these, it can be said that there is a high probability that this fragment of code was hand-written.
\ifx\LITE\undefined
Such instructions are marked as (M) in the instructions list in this appendix: 
\myref{sec:x86_instructions}.
\fi
}\PTBRph{}\ESph{}\PLph{}\ITAph{}\\
\\
\RU{Также, пролог/эпилог функции обычно не встречается в ассемблерном коде, написанном вручную.}
\EN{Also the function prologue/epilogue are not commonly present in hand-written assembly.}\\
\\
\RU{Как правило, в вручную написанном коде, нет никакого четкого метода передачи аргументов в 
функцию}
\EN{Commonly there is no fixed system for passing arguments to functions in the hand-written
code}.\\
\\
\RU{Пример из ядра}\EN{Example from the} Windows 2003\EN{ kernel} 
(\RU{файл }ntoskrnl.exe\EN{ file}):

\begin{lstlisting}
MultiplyTest proc near               ; CODE XREF: Get386Stepping
             xor     cx, cx
loc_620555:                          ; CODE XREF: MultiplyTest+E
             push    cx
             call    Multiply
             pop     cx
             jb      short locret_620563
             loop    loc_620555
             clc
locret_620563:                       ; CODE XREF: MultiplyTest+C
             retn
MultiplyTest endp

Multiply     proc near               ; CODE XREF: MultiplyTest+5
             mov     ecx, 81h
             mov     eax, 417A000h
             mul     ecx
             cmp     edx, 2
             stc
             jnz     short locret_62057F
             cmp     eax, 0FE7A000h
             stc
             jnz     short locret_62057F
             clc
locret_62057F:                       ; CODE XREF: Multiply+10
                                     ; Multiply+18
             retn
Multiply     endp
\end{lstlisting}

\RU{Действительно, если заглянуть в исходные коды}\EN{Indeed, if we look in the} 
\ac{WRK} v1.2\RU{, данный код можно найти в файле}\EN{ source code, this code
can be found easily in file} 
\IT{WRK-v1.2\textbackslash{}base\textbackslash{}ntos\textbackslash{}ke\textbackslash{}i386\textbackslash{}cpu.asm}.


\chapter{\RU{Использование magic numbers для трассировки}\EN{Using magic numbers while tracing}}

\RU{Нередко бывает нужно узнать, как используется то или иное значение, прочитанное из файла либо взятое из пакета,
принятого по сети. Часто, ручное слежение за нужной переменной это трудный процесс. Один из простых методов (хотя и не
полностью надежный на 100\%) это использование вашей собственной \IT{magic number}.}
\EN{Often, our main goal is to understand how the program uses a value that was either read from file or received via network. 
The manual tracing of a value is often a very labour-intensive task. One of the simplest techniques for this (although not 100\% reliable) 
is to use your own \IT{magic number}.}

\RU{Это чем-то напоминает компьютерную томографию: пациенту перед сканированием вводят в кровь 
рентгеноконтрастный препарат, хорошо отсвечивающий в рентгеновских лучах.
Известно, как кровь нормального человека
расходится, например, по почкам, и если в этой крови будет препарат, то при томографии будет хорошо видно,
достаточно ли хорошо кровь расходится по почкам и нет ли там камней, например, и прочих образований.}
\EN{This resembles X-ray computed tomography is some sense: a radiocontrast agent is injected into the patient's blood,
which is then used to improve the visibility of the patient's internal structure in to the X-rays.
It is well known how the blood of healthy humans
percolates in the kidneys and if the agent is in the blood, it can be easily seen on tomography, how blood is percolating,
and are there any stones or tumors.}

\RU{Мы можем взять 32-битное число вроде \TT{0x0badf00d}, либо чью-то дату рождения вроде \TT{0x11101979} 
и записать это, занимающее 4 байта число, в какое-либо место файла используемого исследуемой нами программой.}
\EN{We can take a 32-bit number like \TT{0x0badf00d}, or someone's birth date like \TT{0x11101979}
and write this 4-byte number to some point in a file used by the program we investigate.}

\index{\GrepUsage}
\index{tracer}
\RU{Затем, при трассировки этой программы, в том числе, при помощи \tracer в режиме 
\IT{code coverage}, а затем при помощи
\IT{grep} или простого поиска по текстовому файлу с результатами трассировки, мы можем легко увидеть, в каких местах кода использовалось 
это значение, и как.}
\EN{Then, while tracing this program with \tracer in \IT{code coverage} mode, with the help of \IT{grep}
or just by searching in the text file (of tracing results), we can easily see where the value was used and how.}

\RU{Пример результата работы \tracer в режиме \IT{cc}, к которому легко применить утилиту \IT{grep}}\EN{Example 
of \IT{grepable} \tracer results in \IT{cc} mode}:

\begin{lstlisting}
0x150bf66 (_kziaia+0x14), e=       1 [MOV EBX, [EBP+8]] [EBP+8]=0xf59c934 
0x150bf69 (_kziaia+0x17), e=       1 [MOV EDX, [69AEB08h]] [69AEB08h]=0 
0x150bf6f (_kziaia+0x1d), e=       1 [FS: MOV EAX, [2Ch]] 
0x150bf75 (_kziaia+0x23), e=       1 [MOV ECX, [EAX+EDX*4]] [EAX+EDX*4]=0xf1ac360 
0x150bf78 (_kziaia+0x26), e=       1 [MOV [EBP-4], ECX] ECX=0xf1ac360 
\end{lstlisting}
% TODO: good example!
\RU{Это справедливо также и для сетевых пакетов.
Важно только, чтобы наш \IT{magic number} был как можно более уникален и не присутствовал в самом коде.}
\EN{This can be used for network packets as well.
It is important for the \IT{magic number} to be unique and not to be present in the program's code.}

\newcommand{\DOSBOXURL}{\href{http://go.yurichev.com/17222}{blog.yurichev.com}}

\index{DosBox}
\index{MS-DOS}
\RU{Помимо \tracer, такой эмулятор MS-DOS как DosBox, в режиме heavydebug, может писать в отчет информацию обо всех
состояниях регистра на каждом шаге исполнения программы\footnote{См. также мой пост в блоге об этой возможности в 
DosBox: \DOSBOXURL{}}, так что этот метод может пригодиться и для исследования программ под DOS.}\EN{Aside of 
the \tracer, DosBox (MS-DOS emulator) in heavydebug mode
is able to write information about all registers' states for each executed instruction of the program to a plain text file\footnote{See also my 
blog post about this DosBox feature: \DOSBOXURL{}}, so this technique may be useful for DOS programs as well.}



\section{\IFRU{Старые методы, тем не менее, интересные}{Old-school methods, nevertheless, interesting to know}}

\section{\RU{Сравнение ``снимков'' памяти}\EN{Memory ``snapshots'' comparing}}

\RU{Метод простого сравнения двух снимков памяти для поиска изменений часто применялся для взлома игр 
на 8-битных компьютерах и взлома файлов с записанными рекордными очками.}
\EN{The technique of straightforward two memory snapshots comparing in order to see changes, was often used to hack
8-bit computer games and hacking ``high score'' files.}

\RU{К примеру, если вы имеете загруженную игру на 8-битном компьютере (где самой памяти не очень много, но игра
занимает еще меньше), и вы знаете что сейчас у вас, условно, 100 пуль, вы можете сделать ``снимок'' всей
памяти и сохранить где-то. Затем просто стреляете куда угодно, у вас станет 99 пуль, сделать второй ``снимок'',
и затем сравнить эти два снимка: где-то наверняка должен быть байт, который в начале был 100, а затем стал 99.}
\EN{For example, if you got a loaded game on 8-bit computer (it is not much memory on these, but game is usually
consumes even less memory) and you know that you have now, let's say, 100 bullets, you can do a ``snapshot''
of all memory and back it up to some place. Then shoot somewhere, bullet count now 99, do second ``snapshot''
and then compare both: somewhere must be a byte which was 100 in the beginning and now it is 99.}
\RU{Если учесть, что игры на тех маломощных домашних компьютерах обычно были написаны на ассемблере и подобные
переменные там были глобальные, то можно с уверенностью сказать, какой адрес в памяти всегда отвечает за количество
пуль. Если поискать в дизассемблированном коде игры все обращения по этому адресу, несложно найти код,
отвечающий за уменьшение пуль и записать туда инструкцию \gls{NOP}
или несколько \gls{NOP}-в, так мы получим игру в которой у игрока всегда будет 100 пуль, например.}
\EN{Considering a fact these 8-bit games were often written in assembly language and such variables were global,
it can be said for sure, which address in memory holding bullets count. If to search all references to the
address in disassembled game code, it is not very hard to find a piece of code \glslink{decrement}{decrementing} bullets count,
write \gls{NOP} instruction there, or couple of \gls{NOP}-s, 
we'll have a game with e.g 100 bullets forever.}
\index{BASIC!POKE}
\RU{А так как игры на тех домашних 8-битных 
компьютерах всегда загружались по одним и тем же адресам, и версий одной игры редко когда было больше одной продолжительное время,
то геймеры-энтузиасты знали, по какому адресу (используя инструкцию языка BASIC \gls{POKE}) что записать после загрузки
игры, чтобы хакнуть её. Это привело к появлению списков ``читов'' состоящих из инструкций \gls{POKE}, публикуемых
в журналах посвященным 8-битным играм. См. также:}\EN{Games on these 8-bit computers was commonly loaded on the same
address, also, there were no much different versions of each game (commonly just one version was popular for a long span of time),
enthusiastic gamers knew, which byte must be written (using BASIC instruction \gls{POKE}) to which address in
order to hack it. This led to ``cheat'' lists containing of \gls{POKE} instructions published in magazines related to
8-bit games. See also:} \url{http://en.wikipedia.org/wiki/PEEK\_and\_POKE}.

\index{MS-DOS}
\RU{Точно так же легко модифицировать файлы с сохраненными рекордами (кто сколько очков набрал), впрочем, это может
сработать не только с 8-битными играми. Нужно заметить, какой у вас сейчас рекорд и где-то сохранить файл
с очками. Затем, когда очков станет другое количество, просто сравнить два файла, можно даже
DOS-утилитой FC\footnote{утилита MS-DOS для сравнения двух файлов побайтово} (файлы рекордов, часто, бинарные).}
\EN{Likewise, it is easy to modify ``high score'' files, this may work not only with 8-bit games. Let's notice 
your score count and back the file up somewhere. When ``high score'' count will be different, just compare two files,
it can be even done with DOS-utility FC\footnote{MS-DOS utility for binary files comparing} (``high score'' files
are often in binary form).}
\RU{Где-то будут отличаться несколько байт, и легко будет увидеть, какие именно отвечают за количество очков. 
Впрочем, разработчики игр осведомлены о таких хитростях и могут защититься от этого.}
\EN{There will be a point where couple of bytes will be different and it will be easy to see which ones are
holding score number.
However, game developers are aware of such tricks and may protect against it.}

\RU{В каком-то смысле похожий пример в этой книге здесь}
\EN{Somewhat similar example here in the book is}: \ref{Millenium_DOS_game}.

% TODO: пример с какой-то простой игрушкой?

\subsection{\RU{Реестр Windows}\EN{Windows registry}}

\RU{А еще можно вспомнить сравнение реестра Windows до инсталляции программы и после}
\EN{It is also possible to compare Windows registry before and after a program installation}.
\RU{Это также популярный метод поиска, какие элементы реестра программа будет использовать}
\EN{It is very popular method of finding, which registry elements a program will use}.
\EN{Probably, this is a reason why ``windows registy cleaner'' shareware is so popluar.}
\RU{Наверное это причина, почему так популярны shareware-программы для очистки реестра в Windows.}


\section{\IFRU{Форматы файлов}{File formats}}
\subsection{Win32 PE}

\acs{PE} \IFRU{это формат исполняемых файлов, принятый в Windows}{is a executable file format used in
Windows}.

\IFRU{Разница между .exe, .dll, и .sys в том, что у .exe и .sys обычно нет экспортов, только импорты}
{The difference between .exe, .dll and .sys is that .exe and .sys usually does not have exports, only imports}.

\index{OEP}
\IFRU{У \ac{DLL}, как и у всех PE-файлов, есть точка входа (\ac{OEP})
(там располагается ф-ция DllMain()), но обычно эта ф-ция ничего не делает.}
{A \ac{DLL}, just as any other PE-file, has entry point (\ac{OEP}) (the function DllMain() is located at it) 
but usually this function does nothing.}

.sys \IFRU{это обычно драйвера устройств}{is usually device driver}.

\IFRU{Для драйверов, Windows требует чтобы контрольная сумма в PE-файле была проставлена
и была верной}
{As of drivers, Windows require the checksum is present in PE-file and must be correct}
\footnote{\IFRU{Например}{For example}, Hiew(\ref{Hiew}) \IFRU{умеет её подсчитывать}{can calculate it}}.

\index{Windows!Windows Vista}
\IFRU{А начиная с}{Starting at} Windows Vista, 
\IFRU{PE-файлы-драйвера должны быть также подписаны при помощи электронной подписи, 
иначе они не будут загружаться.}
{driver PE-files must be also signed by digital signature. It will fail to load without signature.}

\index{MS-DOS}
\IFRU{В начале всякого PE-файла есть крохотная DOS-программа,
выводящая на консоль сообщение вроде}{Any PE-file begins with tiny DOS-program, printing a
message like} ``This program cannot be run in DOS mode.'' --- 
\IFRU{если запустить эту программу в DOS либо Windows 3.1, выведется это сообщение}
{if to run this program in DOS or Windows 3.1, this message will be printed}.

\subsubsection{\IFRU{Терминология}{Terminology}}

\begin{itemize}
\item
\IFRU{Модуль}{Module} --- \IFRU{это отдельный файл}{is a separate file}, .exe \OrENRU .dll.

\item
\IFRU{Процесс}{Process} --- \IFRU{это некая загруженная в память и работающая программа}{a program
loaded into memory and running}.
\IFRU{Как правило состоит из одного .exe-файла и массы .dll-файлов}{Commonly consisting of 
one .exe-file and bunch of .dll-files}.

\item
\IFRU{Память процесса}{Process memory} --- \IFRU{память с которой работает процесс}{the memory a process
works with}.
\IFRU{У каждого процесса --- своя}{Each process has its own}.
\IFRU{Там обычно имеются загруженные модули, память стека, кучи (heap), итд}{There can usually be 
loaded modules, memory of the stack, heap(s), etc}.

\item
\index{VA}
\ac{VA} --- \IFRU{это адрес, который будет использоваться в самой программе}{is address which will
be used in program}.

\item
\index{\IFRU{Базовый адрес}{Base address}}
\IFRU{Базовый адрес}{Base address} --- \IFRU{это адрес, по которому модуль будет загружен 
в пространство процесса}{is the address within a process memory at which a module will be loaded}.

\item
\index{RVA}
\ac{RVA} --- \IFRU{это}{is a} \ac{VA}-\IFRU{адрес минус базовый адрес}{address minus base address}.
\IFRU{Многие адреса в таблицах PE-файла используют именно}{Many addresses in PE-file tables
uses exactly} \ac{RVA}-\IFRU{адреса}{addresses}.

\item
\index{IAT}
\ac{IAT} --- \IFRU{самая главная таблица описывающая импорты}{most important table which describes
imports}\footnote{\cite{Pietrek1}}.
\end{itemize}

\subsubsection{\IFRU{Базовый адрес}{Base address}}

\IFRU{Дело в том что несколько авторов модулей могут готовить DLL-файлы для других, и нет возможности договориться о том, какие адреса и кому будут отведены.}
{The fact is that several module authors may prepare DLL-files for others and there is no possibility
to reach agreement, which addresses will be assigned to whose modules.}

\IFRU{Поэтому, если у двух необходимых для загрузки процесса DLL одинаковые базовые адреса,
одна из них будет загружена по этому базовому адресу, 
а вторая ~--- по другому свободному месту в памяти процесса, и все виртуальные адреса
во второй DLL будут скорректированы.}
{So that is why if two necessary for process loading DLLs has the same base addresses,
one of which will be loaded at this base address, and another ~--- at the other spare space in process memory,
and each virtual addresses in the second DLL will be corrected.} \\
\\
\IFRU{Очень часто линкер в}{Often, linker in} \ac{MSVC} \IFRU{генерирует .exe-файлы с базовым адресом}
{generates an .exe-files with the base address} \TT{0x400000},
\IFRU{и с секцией кода начинающейся с}{and with the code section started at} \TT{0x401000}.
\IFRU{Это значит что}{This mean} \ac{RVA} \IFRU{начала секции кода ~---}{of code section begin is} \TT{0x1000}.
\IFRU{А \ac{DLL} часто генерируются этим линкером с базовым адресом}
{DLLs are often generated by this linked with the base address} \TT{0x10000000}
\footnote{\IFRU{Это можно изменять опцией /BASE в линкере}{This can be changed by /BASE linker option}}.

\index{ASLR}
\IFRU{Помимо всего прочего, есть еще одна причина намеренно загружать модули по разным адресам, а точнее, 
по случайным}
{There is also another reason to load modules at different base addresses, rather at random ones}.

\IFRU{Это}{It is} \ac{ASLR}
\footnote{\IFRU{\url{https://ru.wikipedia.org/wiki/Address_Space_Layout_Randomization}}
{\url{https://en.wikipedia.org/wiki/Address_space_layout_randomization}}}.

\index{shellcode}
\IFRU{Дело в том, что некий шеллкод, пытающийся исполниться на зараженной системе, 
должен вызывать какие-то системные ф-ции}{The fact is that a shellcode trying to be executed on a compromised
system must call a system functions}.

\IFRU{И в старых}{In older} \ac{OS} (\IFRU{в линейке Windows NT: до}{in Windows NT line: before} Windows Vista),
\IFRU{системные}{system} DLL (\IFRU{такие как}{like} kernel32.dll, user32.dll) \IFRU{загружались все время
по одним и тем же адресам}{were always loaded at the known addresses}, 
\IFRU{а если еще и вспомнить, что версии этих DLL редко менялись}{and also if to recall
that its versions were rarely changed}, \IFRU{то адреса отдельных
ф-ций, можно сказать, фиксированы и шеллкод может вызывать их напрямую}{an addresses of functions were
fixed and shellcode can call it directly}.

\IFRU{Чтобы избежать этого, методика}{In order to avoid this,} \ac{ASLR}
\IFRU{загружает и вашу программу, и все модули ей необходимые, по случайным адресам, разным при каждом запуске}
{method loads your program and all modules it needs at random base addresses, each time different}.

\subsubsection{Subsystem}

\IFRU{Имеется также поле \IT{subsystem}, обычно это}{There is also \IT{subsystem} field, usually it is}
native (.sys-\IFRU{драйвер}{driver}), 
console (\IFRU{консольное приложение}{console application}) \OrENRU 
\ac{GUI} (\IFRU{не консольное}{non-console}).

\subsubsection{\IFRU{Версия ОС}{OS version}}

\IFRU{В PE-файле имеется минимальный номер версии Windows, необходимый для загрузки модуля.}
{A PE-file also has minimal Windows version needed in order to load it.}
\IFRU{Соответствие номеров версий в файле и кодовых наименований Windows, можно посмотреть}
{The table of version numbers stored in PE-file and corresponding Windows codenames is}
\href{https://en.wikipedia.org/wiki/Windows_NT#Releases}{\IFRU{здесь}{here}}.

\index{Windows!Windows NT4}
\index{Windows!Windows 2000}
\IFRU{Например}{For example}, \ac{MSVC} 2005 \IFRU{еще компилирует .exe-файлы запускающиеся на}{compiles
.exe-files running on} Windows NT4 (\IFRU{версия}{version} 4.00),
\IFRU{а вот}{but} \ac{MSVC} 2008 \IFRU{уже нет}{is not} 
(\IFRU{генерируемые файлы имеют версию}{files generated has version} 5.00, 
\IFRU{для запуска необходима как минимум Windows 2000}{at least Windows 2000 is needed to run them}).

\index{Windows!Windows XP}
\ac{MSVC} 2012 \IFRU{по умолчанию генерирует .exe-файлы версии}{by default generates .exe-files of version} 6.00, 
\IFRU{для запуска нужна как минимум Windows Vista}{targeting at least Windows Vista}, \IFRU{хотя}{however, by} 
\href{http://blogs.msdn.com/b/vcblog/archive/2012/10/08/10357555.aspx}{\IFRU{изменив настройки компиляции}
{by changing compiler's options}}, 
\IFRU{можно заставить генерировать и под Windows XP}{it is possible to force it to compile for Windows XP}.

\subsubsection{\IFRU{Секции}{Sections}}

\IFRU{Разделение на секции присутствует, по видимому, во всех форматах исполняемых файлов}{Division by sections,
as it seems, are present in all execuable file formats}.

\IFRU{Сделано это для того, чтобы отделить код от данных, а данные ~--- от константных данных.}
{It is done in order to separate code from data, and data ~--- from constant data.}

\begin{itemize}
\item
\IFRU{На секции кода будет стоять флаг}{There will be flag} 
\IT{IMAGE\_SCN\_CNT\_CODE} \OrENRU \IT{IMAGE\_SCN\_MEM\_EXECUTE}\IFRU{}{ on code section}
~--- \IFRU{это исполняемый код}{this is executable code}.

\item
\IFRU{На секции данных}{On data section} ~--- \IFRU{флаги }{}\IT{IMAGE\_SCN\_CNT\_INITIALIZED\_DATA}, 
\IT{IMAGE\_SCN\_MEM\_READ} \AndENRU \IT{IMAGE\_SCN\_MEM\_WRITE}\IFRU{}{ flags}.

\item
\IFRU{На пустой секции с неинициализированными данными}{On an empty section with uninitialized data} ~--- 
\IT{IMAGE\_SCN\_CNT\_UNINITIALIZED\_DATA}, \IT{IMAGE\_SCN\_MEM\_READ} \AndENRU \IT{IMAGE\_SCN\_MEM\_WRITE}.

\item
\IFRU{А на секции с константными данными, то есть, защищенными от записи}{On a constant data section, in other words,
protected from writing}, \IFRU{могут быть флаги}
{there are may be flags} \\
\IT{IMAGE\_SCN\_CNT\_INITIALIZED\_DATA} \AndENRU \IT{IMAGE\_SCN\_MEM\_READ} \IFRU{без}{without} \IT{IMAGE\_SCN\_MEM\_WRITE}. 
\IFRU{Если попытаться записать что-то в эту секцию, процесс упадет}{A process will crash if it would try to write to this
section}.
\end{itemize}

\IFRU{В PE-файле можно задавать название для секции, но это не важно}{Each section in PE-file may have a name, however,
it is not very important}.
\IFRU{Часто (но не всегда)}{Often (but not always)} \IFRU{секция кода называется}{code section have the name} \TT{.text}, 
\index{TLS}
\index{BSS}
\IFRU{секция данных}{data section} --- \TT{.data}, \IFRU{константных данных}{constant data section} --- \TT{.rdata} 
\IT{(readable data)}.
\IFRU{Еще популярные имена секций}{Other popular section names are}: 
\TT{.idata} (\IFRU{секция импортов}{imports section}),
\TT{.edata} (\IFRU{секция экспортов}{exports section}),
\TT{.reloc} (\IFRU{секция релоков}{relocs section}),
\TT{.bss} (\IFRU{неинициализированные данные}{uninitialized data (\ac{BSS})}),
\TT{.tls} ~--- thread local storage (\ac{TLS}),
\TT{.rsrc} (\IFRU{ресурсы}{resources}).

\IFRU{Запаковщики/зашифровщики PE-файлов часто затирают имена секций, или меняют на свои}
{PE-file packers/encryptors are often garble section names or replacing names to their own}.

\IFRU{В \ac{MSVC} можно объявлять данные в произвольно названной секции}
{\ac{MSVC} allows to declare data in arbitrarily named section}
\footnote{\url{http://msdn.microsoft.com/en-us/library/windows/desktop/cc307397.aspx}}.

\IFRU{Некоторые компиляторы и линкеры могут добавлять также секцию с отладочными символами 
и вообще отладочной информацией}
{Some compilers and linkers can add a section with debugging symbols and other debugging information}
(\IFRU{например}{for example}, MinGW).
\index{PDB}
\IFRU{Хотя это не так в современных версиях}{However it is not so in modern versions of} \ac{MSVC} 
(\IFRU{там принято отладочную информацию сохранять в отдельных PDB-файлах}
{a separate PDB-files are used there for this purpose}).

\subsubsection{\IFRU{Релоки}{Relocations (relocs)}}

\IFRU{Так же известны как FIXUP-ы}{\ac{AKA} FIXUP-s}.

\IFRU{Это также присутствует почти во всех форматах загружаемых и исполняемых файлов}
{This is also present in almost all executable file formats}
\footnote{\IFRU{Даже .exe-файлы в}{Even .exe-files in} MS-DOS}.

\IFRU{Как видно, модули могут загружаться по другим базовым адресам,
но как же тогда работать с глобальными переменными, например?}
{Obviously, modules are can be loaded on different base addresses, but how to work with global variables,
for example?}
\IFRU{Ведь нужно обращаться к ним по адресу}{They must be accessed by an address}.
\IFRU{Одно из решений это}{One solution is} \PICcode(\ref{sec:PIC}).
\IFRU{Но это далеко не всегда удобно}{But it is not always suitable}.

\IFRU{Поэтому имеется таблица релоков. 
Там просто перечислены адреса мест в модуле подлежащими коррекции при загрузке
по другому базовому адресу.}
{That is why relocations table is present.
The addresses of points needs to be corrected in case of loading on another base address 
are just enumerated in the table.}

\IFRU{Например, по}{For example, there is a global variable at the address}
\TT{0x410000} 
\IFRU{лежит некая глобальная переменная, и вот как обеспечивается её чтение}
{and this is how it is accessed}:

\begin{lstlisting}
A1 00 00 41 00         mov         eax,[000410000]
\end{lstlisting}

\IFRU{Базовый адрес модуля}{Base address of module is} \TT{0x400000},
\IFRU{а }{}\ac{RVA} \IFRU{глобальной переменной}{of global variable is} \TT{0x10000}.

\IFRU{Если загружать модуль по базовому адресу}{If the module is loading on the base address}
\TT{0x500000}, \IFRU{нужно чтобы адрес этой переменной в этой инструкции стал}{the factual address
of the global variable must be} \TT{0x510000}.

\index{x86!\Instructions!MOV}
\IFRU{Как видно, адрес переменной закодирован в самой инструкции}
{As we can see, address of variable is encoded in the instruction} \TT{MOV}, 
\IFRU{после байта}{after the byte} \TT{0xA1}.

\IFRU{Поэтому адрес четырех байт}{That is why address of 4 bytes}, \IFRU{после}{after} \TT{0xA1},
\IFRU{записыватся в таблицу релоков}{is written into relocs table}.

\IFRU{Если модуль загружается по другому базовому адресу},
\IFRU{загрузчик \ac{OS} обходит все адреса в таблице}{\ac{OS}-loader enumerates all addresses in table}, 
\IFRU{находит каждое 32-битное слово по этому адресу}
{finds each 32-bit word the address points on},
\IFRU{отнимает от него настоящий, оригинальный базовый адрес}{subtracts real, original base address of it}
(\IFRU{в итоге получается}{we getting} \ac{RVA}\IFRU{}{ here}),
\IFRU{и прибавляет к нему новый базовый адрес}{and adds new base address to it}.

\IFRU{А если модуль загружается по своему оригинальному базовому адресу, ничего не происходит}
{If module is loading on original base address, nothing happens}.

\IFRU{Так можно обходиться со всеми глобальными переменными}
{All global variables may be treated like that}.

\IFRU{Релоки могут быть разных типов}{Relocs may have various types}, 
\IFRU{однако в Windows для x86-процессоров, тип обычно}
{however, in Windows, for x86 processors, the type is usually} \\
\IT{IMAGE\_REL\_BASED\_HIGHLOW}.

\subsubsection{\IFRU{Экспорты и импорты}{Exports and imports}}

\IFRU{Как известно}{As all we know}, 
\IFRU{любая исполняемая программа должна как-то пользоваться сервисами \ac{OS} и прочими DLL-библиотеками}
{any executable program must use \ac{OS} services and other DLL-libraries somehow}.

\IFRU{Можно сказать, что нужно связывать функции из одного модуля (обычно DLL) и места их вызовов в 
другом модуле (.exe-файл или другая DLL)}
{It can be said, functions from one module (usually DLL) must be connected somehow to a points of their
calls in other module (.exe-file or another DLL)}.

\IFRU{Для этого, у каждой DLL есть ``экспорты'', это таблица ф-ций плюс их адреса в модуле}
{Each DLL has ``exports'' for this, this is table of functions plus its addresses in a module}.

\IFRU{А у .exe-файла, либо DLL, есть ``импорты'', (\ac{IAT}) это таблица ф-ций требующихся для исполнения 
включая список имен DLL-файлов}
{Each .exe-file or DLL has ``imports'', (\ac{IAT}) this is a table of functions it needs for execution including
list of DLL filenames}.

\IFRU{Загрузчик \ac{OS}, после загрузки основного .exe-файла, проходит по \ac{IAT}:
загружает дополнительные DLL-файлы, 
находит имена ф-ций среди экспортов в DLL и прописывает их адреса в \ac{IAT} в головном .exe-модуле}
{After loading main .exe-file, \ac{OS}-loader, processes \ac{IAT}: 
it loads additional DLL-files, finds function names
among DLL exports and writes their addresses down in an \ac{IAT} of main .exe-module}.

\index{Ordinal}
\IFRU{Как видно, во время загрузки, загрузчику нужно много сравнивать одни имена ф-ций с другими, а сравнение строк
это не очень быстрая процедура, так что,
имеется также поддержка ``ординалов'' или 
``hint''-ов, это когда в таблице импортов проставлены номера ф-ций вместо их имен}
{As we can notice, during loading, loader must compare a lot of function names, but strings comparison is not a very
fast procedure, so, there is a support of ``ordinals'' or ``hints'',
that is a function numbers stored in the table instead of their names}.

\IFRU{Так их быстрее находить в загружаемой DLL}
{That is how they can be located faster in loading DLL}.
\IFRU{В таблице экспортов ординалы присутствуют всегда}{Ordinals are always present in ``export'' table}.

\index{MFC}
\IFRU{К примеру}{For example}, \IFRU{программы использующие библиотеки}{program using} 
\ac{MFC}\IFRU{, обычно загружают mfc*.dll по ординалам}{ library usually loads mfc*.dll by ordinals},
\IFRU{и в таких программах, в \ac{IAT}, нет имен ф-ций \ac{MFC}}
{and in such programs there are no \ac{MFC} function names in \ac{IAT}}.

\IFRU{При загрузке такой программы в \IDA, она спросит у вас путь к файлу mfc*.dll,
чтобы установить имена ф-ций}{While loading such program in \IDA, it will asks for a path to mfs*.dll files,
in order to determine function names}.
\IFRU{Если в \IDA не указать путь к этой DLL, то вместо имен ф-ций будет что-то вроде}
{If not to tell \IDA path to this DLL, they will look like}
\IT{mfc80\_123}\IFRU{}{ instead of function names}.

\paragraph{\IFRU{Секция импортов}{Imports section}}

\IFRU{Под \ac{IAT} и всё что с ней связано иногда отводится отдельная секция (с названием вроде \TT{.idata}),
но это не обязательно}
{Often a separate section is allocated for \ac{IAT} table and everything related to it (with name like \TT{.idata}),
however, it is not a strict rule}.

\IFRU{Импорты это запутанная тема еще и из-за терминологической путанницы. Попробуем собрать всё в одно место.}
{Imports is also confusing subject because of terminological mess. Let's try to collect all information in one place.}

\begin{figure}[ht!]
\centering
\includegraphics[scale=0.66]{OS-specific/PE/unnamed0.png}
\caption{\IFRU{схема, объеденяющая все структуры в PE-файлы, связанные с импортами}
{The scheme, uniting all PE-file structures related to imports}}
\end{figure}

\IFRU{Самая главная структура, это массив}{Main structure is the array of} \IT{IMAGE\_IMPORT\_DESCRIPTOR}.
\IFRU{Каждый элемент на каждую импортируемую DLL}{Each element for each DLL being imported}.

\IFRU{У каждого элемента есть}{Each element holds} \ac{RVA}-\IFRU{адрес}{address} 
\IFRU{текстовой строки (имя DLL)}{of text string (DLL name)} (\IT{Name}).

\IT{OriginalFirstThink} \IFRU{это}{is a} \ac{RVA}-\IFRU{адрес}{address} \IFRU{массива}{of array of} 
\ac{RVA}-\IFRU{адресов}{addresses},
\IFRU{каждый из которых указывает на текстовую строку где записано имя ф-ции}
{each of which points to the text string with function name}. 
\IFRU{Каждую строку предваряет 16-битное число}{Each string is prefixed by 16-bit integer} (``hint'') ~--- 
\IFRU{``ординал'' ф-ции}{``ordinal'' of function}.

\IFRU{Если при загрузке удается найти ф-цию по ординалу, тогда сравнение текстовых строк не будет происходить.
Массив оканчивается нулем}{While loading, if it is possible to find function by ordinal,
then strings comparison will not occur. Array is terminated by zero}.
\IFRU{Есть также указатель с названием}
{There is also a pointer with the name} \IT{FirstThunk},
\IFRU{это просто}{it is just} \ac{RVA}-\IFRU{адрес}{address} 
\IFRU{места, где загрузчик будет проставлять адреса найденных ф-ций}{of the place where loader will
write addresses of functions resolved}.

\IFRU{Места где загрузчик проставляет адреса, \IDA именует их так}{The points where
loader writes addresses, \IDA marks like}: \IT{\_\_imp\_CreateFileA}, \IFRU{итд}{etc}.

\IFRU{Есть по крайней мере два способа использовать адреса проставленные загрузчиком}
{There are at least two ways to use addresses written by loader}.

\begin{itemize}
\index{x86!\Instructions!CALL}
\item
\IFRU{В коде будут просто инструкции вроде}{The code will have instructions like} 
\IT{call \_\_imp\_CreateFileA}, 
\IFRU{а так как, поле с адресом импортируемой ф-ции это как бы глобальная переменная}
{and since the field with the address of function imported is a global variable in some sense}, 
\IFRU{то в таблице релоков добавляется адрес (плюс 1 или 2) в инструкции \IT{call}}
{the address of \IT{call} instruction (plus 1 or 2) will be added to relocs table},
\IFRU{на случай если модуль будет загружен по другому базовому адресу}
{for the case if module will be loaded on different base address}.

\IFRU{Но как видно, это приводит к увеличению таблицы релоков}{But, obviously, this may enlarge
relocs table significantly}.
\IFRU{Ведь вызовов импортируемой ф-ции у вас в модуле может быть очень много}
{Because there are might be a lot of calls to imported functions in the module}.
\IFRU{К тому же, чем больше таблица релоков, тем дольше загрузка}
{Furthermore, large relocs table slowing down the process of module loading}.

\index{x86!\Instructions!JMP}
\index{thunk-\IFRU{функции}{functions}}
\item
\IFRU{На каждую импортируемую ф-цию выделяется только один переход на импортируемую ф-цию используя
инструкцию \JMP плюс релок на эту инструкцию}
{For each imported function, there is only one jump allocated, using \JMP instruction 
plus reloc to this instruction}.
\IFRU{Такие места-``переходники'' называются также ``thunk''-ами}{Such points are also called ``thunks''}.
\IFRU{А все вызовы импортируемой ф-ции это просто инструкция \CALL на соответствующий ``thunk''}
{All calls to the imported functions are just \CALL instructions to the corresponding ``thunk''}.
\IFRU{В данном случае, дополнительные релоки не нужны, потому что эти CALL-ы имеют относительный адрес,
и корректировать их не надо}{In this case, additional relocs are not necessary because these CALL-s
has relative addresses, they are not to be corrected}.
\end{itemize}

\IFRU{Оба этих два метода могут комбинироваться}{Both of these methods can be combined}.
\IFRU{Вероятно, линкер создает отдельный ``thunk'', если вызовов слишком много, но по умолчанию --- не создает}
{Apparently, linker creates individual ``thunk'' if there are too many calls to the functions,
but by default it is not to be created}. \\
\\
\IFRU{Кстати, массив адресов ф-ций, на который указывает FirstThunk,
не обязательно может быть в секции \ac{IAT}}{By the way, an array of function addresses to which FirstThunk is
pointing is not necessary to be located in \ac{IAT} section}.
\IFRU{К примеру, я написал утилиту}{For example, I once wrote the}
PE\_add\_import\footnote{\url{http://yurichev.com/PE_add_import.html}} 
\IFRU{для добавления импорта в уже существующий .exe-файл}{utility for adding import to an existing .exe-file}.
\IFRU{На месте ф-ции, вместо которой вы хотите подставить вызов в другую DLL,
моя утилита вписывает такой код}{At the place of the function you want to substitute by call to another DLL,
the following code my utility writes}:

\begin{lstlisting}
MOV EAX, [yourdll.dll!function]
JMP EAX
\end{lstlisting}

\IFRU{При этом, FirstThunk указывает прямо на первую инструкцию.
Иными словами, загрузчик, загружая yourdll.dll, 
прописывает адрес ф-ции \IT{function} прямо в коде.}
{FirstThunk points to the first instruction. In other words, while loading yourdll.dll,
loader writes address of the \IT{function} function right in the code.}

\IFRU{Надо также отметить что обычно секция кода защищена от записи}
{It also worth noting a
code section is usually write-protected}, \IFRU{так что, моя утилита
добавляет флаг}{so my utility adds} \IT{IMAGE\_SCN\_MEM\_WRITE} 
\IFRU{для секции кода. Иначе при загрузке такой программы, она упадет с ошибкой}
{flag for code section. Otherwise, the program will crash while loading with the error code}
5 (access denied). \\
\\
\IFRU{Может возникнуть вопрос: а что если я поставляю программу с набором DLL,
которые никогда не будут меняться, может как-то можно ускорить процесс загрузки?}
{One might ask: what if I supply a program with the DLL files set which are not supposed to change,
is it possible to speed up loading process?}

\IFRU{Да, можно прописать адреса импортируемых ф-ций в массивы FirstThunk зараннее}
{Yes, it is possible to write addresses of the functions to be imported into FirstThunk arrays in advance}.
\IFRU{Для этого в структуре}{The \IT{Timestamp} field is present in the}
\IT{IMAGE\_IMPORT\_DESCRIPTOR} \IFRU{имеется поле \IT{Timestamp}}{structure}.
\IFRU{И если там присутствует какое-то значение, то загрузчик сверяет это значение с датой-временем DLL-файла}
{If a value is present there, then loader compare this value with date-time of the DLL file}.
\IFRU{И если они равны, то загрузчик больше ничего не делает, и загрузка будет происходить быстрее}
{If the values are equal to each other, then the loader is not do anything, and loading process will be faster}.\IFRU{Это называется}{This is what called} ``old-style binding''
\footnote{\url{http://blogs.msdn.com/b/oldnewthing/archive/2010/03/18/9980802.aspx}.
\IFRU{Существует также}{There is also} ``new-style binding'',
\IFRU{про него напишу позже}{I will write about it in future}}.
\index{BIND.EXE}
\IFRU{В Windows SDK для этого имеется утилита BIND.EXE}
{There is the BIND.EXE utility in Windows SDK for this}.
\IFRU{Для ускорения загрузки вашей программы}{For speeding up of loading of your program}, 
Matt Pietrek \InENRU \cite{Pietrek1}, \IFRU{предлагает делать binding сразу после инсталляции
вашей программы на компьютере конечного пользователя}{offers to do binding shortly after your program
installation on the computer of the end user}. \\
\\
\IFRU{Запаковщики/зашифровщики PE-файлов могут также сжимать/шифровать \ac{IAT}}
{PE-files packers/encryptors may also comporess/encrypt \ac{IAT}}.
\IFRU{В этом случае, загрузчик Windows, конечно же, не загрузит все нужные DLL}
{In this case, Windows loader, of course, will not load all necessary DLLs}.
\index{Windows!LoadLibrary}
\index{Windows!GetProcAddress}
\IFRU{Поэтому распаковщик/расшифровщик делает это сам, при помощи вызовов}
{Therefore, packer/encryptor do this on its own, with the help of} 
\IT{LoadLibrary()} \AndENRU \IT{GetProcAddress()}\IFRU{}{ functions}. \\
\\
\IFRU{В стандартных DLL входящих в состав Windows, часто, \ac{IAT} находится в самом начале PE-файла}
{In the standard DLLs from Windows installation, often, \ac{IAT} is located right in the beginning of PE-file}.
\IFRU{Возможно это для оптимизации}{Supposedly, it is done for optimization}.
\IFRU{Ведь .exe-файл при загрузке не загружается в память весь 
(вспомните что инсталляторы огромного размера подозрительно быстро запускаются), он ``мапится'' (map), 
и подгружается в память частями по мере обращения к этой памяти.}
{While loading, .exe file is not loaded into memory as a whole (recall huge install programs which are
started suspiciously fast), it is ``mapped'', and loaded into memory by parts as they are accessed.}
\IFRU{И возможно в Microsoft решили что так будет быстрее}
{Probably, Microsoft developers decided it will be faster}.

\subsubsection{\IFRU{Ресурсы}{Resources}}

\label{PEresources}
\IFRU{Ресурсы в PE-файле это набор иконок, картинок, текстовых строк, описаний диалогов}
{Resources in a PE-file is just a set of icons, pictures, text strings, dialog descriptions}.
\IFRU{Возможно, их в свое время решили отделить от основного кода, чтобы все эти вещи были мультиязычными,
и было проще выбирать текст или картинку того языка, который установлен в \ac{OS}}
{Perhaps, they were separated from the main code, so all these things could be multilingual,
and it would be simpler to pick text or picture for the language that is currently set in \ac{OS}}. \\
\\
\IFRU{В качестве побочного эффекта, их легко редактировать и сохранять обратно в исполняемый файл,
даже не обладая специальными знаниями,
например, редактором}{As a side effect, they can be edited easily and saved back to the executable file,
even, if one does not have special knowledge, for example, using} ResHack\IFRU{}{ editor}(\ref{ResHack}).

\subsubsection{.NET}

\index{.NET}
\IFRU{Программы на .NET компилируются не в машинный код, а в свой собственный байткод}
{.NET programs are compiled not into machine code but into special bytecode}.
\index{OEP}
\IFRU{Собственно, в .exe-файлы байткод вместо обычного кода, однако, точка входа (\ac{OEP}) 
указывает на крохотный фрагмент x86-кода}{Strictly speaking, there is bytecode instead of usual x86-code
in the .exe-file, however, entry point (\ac{OEP}) is pointing to the tiny fragment of x86-code}:

\begin{lstlisting}
jmp         mscoree.dll!_CorExeMain
\end{lstlisting}

\IFRU{А в mscoree.dll и находится .NET-загрузчик, который уже сам будет работать с PE-файлом}
{.NET-loader is located in mscoree.dll, it will process the PE-file}.
\index{Windows!Windows XP}
\IFRU{Так было в \ac{OS} до Windows XP. Начиная с XP, загрузчик \ac{OS} уже сам определяет что это
.NET-файл и запускает его не исполняя этой инструкции \JMP}
{It was so in pre-Windows XP \ac{OS}. Starting from XP, \ac{OS}-loader able to detect the .NET-file
and run it without execution of that \JMP instruction}
\footnote{url{http://msdn.microsoft.com/en-us/library/xh0859k0(v=vs.110).aspx}}.

\index{TLS}
\subsubsection{TLS}

\IFRU{Эта секция содержит в себе инициализированные данные для}{This section holds initialized
data for} \ac{TLS}(\ref{TLS}) (\IFRU{если нужно}{if needed}).
\IFRU{При старте нового треда, его}{When new thread starting, its} 
\ac{TLS}-\IFRU{данные инициализируются данными из этой секции}
{data is initialized by the data from this section}. \\
\\
\index{C++!C++11}
\IFRU{В}{In the} C++11 \IFRU{ввели модификатор}{standard, a new} \IT{thread\_local} 
\IFRU{показывающий что каждый тред будет иметь свою версию этой переменной}
{modifier was added, showing that each thread will have its own version of the variable},
\IFRU{и её можно инициализировать, и она расположена в}{it can be initialized, and it is located in the} \ac{TLS}
\footnote{
\index{C11}
\IFRU{В C11 также есть поддержка тредов, хотя и опциональная}
{C11 also has thread support, optional though}}:

\begin{lstlisting}[caption=C++11]
#include <iostream>
#include <thread>

thread_local int tmp=3;

int main()
{
	std::cout << tmp << std::endl;
};
\end{lstlisting}
\footnote{\IFRU{Компилируется в}{Compiled in} MinGW GCC 4.8.1, \IFRU{но не в}{but not in} MSVC 2012}

\IFRU{В исполняемом файле значение}{In the resulting executable file, the} \IT{tmp} 
\IFRU{будет именно в}{variable will be stored in the} \ac{TLS}.
\\
\index{TLS!Callbacks}
\IFRU{Помимо всего прочего, спецификация PE-файла предусматривает инициализацию}
{Aside from that, PE-file specification also provides initialization of}
\ac{TLS}-\IFRU{секции, т.н.}{section, so called}, TLS callbacks.
\IFRU{Если они присутствуют, то они будут вызваны перед тем как передать управление на главную точку входа}
{If they are present, they will be called before control passing to the main entry point} (\ac{OEP}).
\IFRU{Это широко используется запаковщиками/защифровщиками PE-файлов}
{This is used widely in the PE-file packers/encryptors}.

\subsubsection{\IFRU{Инструменты}{Tools}}

\begin{itemize}
\item
\index{objdump}
\index{cygwin}
objdump (\IFRU{из}{from} cygwin) \IFRU{для вывода всех структур PE-файла}{for dumping all PE-file structures}.

\item
\index{Hiew}
Hiew(\ref{Hiew}) \IFRU{как редактор}{as editor}.

\item
pefile --- Python-\IFRU{библиотека для работы с PE-файлами}{library for PE-file processings}
\footnote{\url{https://code.google.com/p/pefile/}}.

\item
\label{ResHack}
ResHack \acs{AKA} Resource Hacker --- \IFRU{редактор ресурсов}{resources editor}
\footnote{\url{http://www.angusj.com/resourcehacker/}}.
\end{itemize}

\subsubsection{Further reading}

% FIXME: bibliography per chapter or section
\begin{itemize}
\item
Daniel Pistelli --- The .NET File Format \footnote{\url{http://www.codeproject.com/Articles/12585/The-NET-File-Format}}
\end{itemize}



\chapter{\IFRU{Задачи}{Tasks}}

\IFRU{Почти для всех задач, если не указано иное, два вопроса:}
{There are two questions almost for every task, if otherwise is not specified:}

1) \IFRU{Что делает эта функция? Ответ должен состоять из одной фразы.}
{What this function does? Answer in one-sentence form.}

2) \IFRU{Перепишите эту функцию на \CCpp}{Rewrite this function into \CCpp}.

\IFRU{Подсказки и ответы собраны в приложении к этой книге.}{Hints and solutions are in the appendix of
this book.}

\section{\IFRU{Легкий уровень}{Easy level}}

\subsection{\Task 1.1}

\IFRU{Это стандартная функция из библиотек Си. Исходник взят из OpenWatcom. Скомпилировано в MSVC 2010.}
{This is standard C library function. Source code taken from OpenWatcom. Compiled in MSVC 2010.}

\lstinputlisting{tasks/1_1_msvc.asm}

\IFRU{Это он же скомпилирован при помощи GCC 4.4.1 с опцией \Othree (максимальная оптимизация)}
{It is the same code compiled by GCC 4.4.1 with \Othree option (maximum optimization)}:

\lstinputlisting{tasks/1_1_gcc.asm}

Это он же скомпилирован для ARM при помощи Keil с опцией \Othree для режима ARM:

\lstinputlisting{tasks/1_1_ARM.s}

Это он же скомпилирован для ARM при помощи Keil с опцией \Othree для режима thumb:

\lstinputlisting{tasks/1_1_thumb.s}

\subsection{\Task 1.2}

\IFRU{Это также стандартная функция из библиотек Си. Исходник взят из OpenWatcom и немного переделан. 
Скомпилировано в MSVC 2010 с флагом (\Ox).}
{This is also standard C library function. Source code is taken from OpenWatcom and modified slightly.
Compiled in MSVC 2010 with \Ox optimization flag.}

\IFRU{Эта функция использует стандартные функции Си:}
{This function also use these standard C functions:} isspace() \AndENRU isdigit().

\lstinputlisting{tasks/1_2_msvc.asm}

\IFRU{То же скомпилировано в GCC 4.4.1. Задача немного усложняется тем, что GCC представил isspace() и isdigit() 
как inline-функции и вставил их тела прямо в код.}
{Same code compiled in GCC 4.4.1. This task is sligthly harder since GCC compiled isspace() and isdigit()
functions as inline-functions and inserted their bodies right into the code.}

\lstinputlisting{tasks/1_2_gcc.asm}

То же скомпилированное для ARM при помощи Keil с опцией \Othree для режима ARM:

\lstinputlisting{tasks/1_2_ARM.s}

То же скомпилированное для ARM при помощи Keil с опцией \Othree для режима thumb:

\lstinputlisting{tasks/1_2_thumb.s}

\subsection{\Task 1.3}

\IFRU{Это также стандартная функция из библиотек Си, а вернее, две функции, работающие в паре. 
Исходник взят из MSVC 2010 и немного переделан.}
{This is standard C function too, actually, two functions working in pair.
Source code taken from MSVC 2010 and modified sligthly.}

\IFRU{Суть переделки в том, что эта функция может корректно работать в мульти-тредовой среде, 
а я, для упрощения (или запутывания) убрал поддержку этого.}
{The matter of modification is that this function can work properly in multi-threaded environment,
and I removed its support for simplification (or for confusion).}

\IFRU{Скомпилировано в MSVC 2010 с флагом (\Ox)}{Compiled in MSVC 2010 with \Ox flag}.

\lstinputlisting{tasks/1_3_msvc.asm}

\IFRU{То же скомпилировано при помощи GCC 4.4.1}{Same code compiled in GCC 4.4.1}:

\lstinputlisting{tasks/1_3_gcc.asm}

То же скомпилированное для ARM при помощи Keil с опцией \Othree для режима ARM:

\lstinputlisting{tasks/1_3_ARM.s}

То же скомпилированное для ARM при помощи Keil с опцией \Othree для режима thumb:

\lstinputlisting{tasks/1_3_thumb.s}

\subsection{\Task 1.4}

\IFRU{Это стандартная функция из библиотек Си. Исходник взят из MSVC 2010. Скомпилировано в MSVC 2010 с флагом \Ox.}
{This is standard C library function. Source code taken from MSVC 2010. Compiled in MSVC 2010 with \Ox flag.}

\lstinputlisting{tasks/1_4_msvc.asm}

\IFRU{То же скомпилировано при помощи GCC 4.4.1}
{Same code compiled in GCC 4.4.1}:

\lstinputlisting{tasks/1_4_gcc.asm}

То же скомпилированное для ARM при помощи Keil с опцией \Othree для режима ARM:

\lstinputlisting{tasks/1_4_ARM.s}

То же скомпилированное для ARM при помощи Keil с опцией \Othree для режима thumb:

\lstinputlisting{tasks/1_4_thumb.s}

\subsection{\Task 1.5}

\IFRU{Задача, скорее, на эрудицию, нежели на чтение кода.}
{This task is rather on knowledge than on reading code.}

\IFRU{Функция взята из OpenWatcom. Скомпилировано в MSVC 2010 с флагом \Ox.}
{The function is taken from OpenWatcom. Compiled in MSVC 2010 with \Ox flag.}

\lstinputlisting{tasks/1_5_msvc.asm}

\subsection{\Task 1.6}

\IFRU{Скомпилировано в MSVC 2010 с ключом \Ox.}
{Compiled in MSVC 2010 with \Ox option.}

\lstinputlisting{tasks/1_6_msvc.asm}

То же скомпилированное для ARM при помощи Keil с опцией \Othree для режима ARM:

\lstinputlisting{tasks/1_6_ARM.s}

То же скомпилированное для ARM при помощи Keil с опцией \Othree для режима thumb:

\lstinputlisting{tasks/1_6_thumb.s}

\subsection{\Task 1.7}

\IFRU{Это взята функция из ядра Linux 2.6.}{This function is taken from Linux 2.6 kernel.}

\IFRU{Скомпилировано в MSVC 2010 с опцией \Ox:}{Compiled in MSVC 2010 with \Ox option:}

\lstinputlisting{tasks/1_7_msvc.asm}

То же скомпилированное для ARM при помощи Keil с опцией \Othree для режима ARM:

\lstinputlisting{tasks/1_7_ARM.s}

То же скомпилированное для ARM при помощи Keil с опцией \Othree для режима thumb:

\lstinputlisting{tasks/1_7_thumb.s}

\subsection{\Task 1.8}

\IFRU{Скомпилировано в MSVC 2010 с опцией \TT{/O1}\footnote{/O1: оптимизация по размеру кода}:}
{Compiled in MSVC 2010 with \TT{/O1} option\footnote{/O1: minimize space}:}

\lstinputlisting{tasks/1_8_msvc.asm}

То же скомпилированное для ARM при помощи Keil с опцией \Othree для режима ARM:

\lstinputlisting{tasks/1_8_ARM.s}

То же скомпилированное для ARM при помощи Keil с опцией \Othree для режима thumb:

\lstinputlisting{tasks/1_8_thumb.s}

\subsection{\Task 1.9}

\IFRU{Скомпилировано в MSVC 2010 с опцией \TT{/O1}:}
{Compiled in MSVC 2010 with \TT{/O1} option:}

\lstinputlisting{tasks/1_9_msvc.asm}

То же скомпилированное для ARM при помощи Keil с опцией \Othree для режима ARM:

\lstinputlisting{tasks/1_9_ARM.s}

То же скомпилированное для ARM при помощи Keil с опцией \Othree для режима thumb:

\lstinputlisting{tasks/1_9_thumb.s}

\subsection{\Task 1.10}

\IFRU{Если это скомпилировать и запустить, появится некоторое число. Откуда оно берется? 
Откуда оно берется если скомпилировать в MSVC с оптимизациями (\Ox)?}
{If to compile this piece of code and run, some number will be printed. Where it came from?
Where it came from if to compile it in MSVC with optimization (\Ox)?}

\begin{lstlisting}
#include <stdio.h>

int main()
{
	printf ("%d\n");

	return 0;
};
\end{lstlisting}

\section{\IFRU{Средний уровень}{Middle level}}

\subsection{\Task 2.1}

\IFRU{Довольно известный алгоритм, также включен в стандартную библиотеку Си. Исходник взят из glibc 2.11.1. 
Скомпилирован в GCC 4.4.1 с ключом \TT{-Os} (оптимизация по размеру кода). 
Листинг сделан дизассемблером IDA 4.9 из ELF-файла созданным GCC и линкером.}
{Well-known algorithm, also included in standard C library. Source code was taken from glibc 2.11.1.
Compiled in GCC 4.4.1 with \TT{-Os} option (code size optimization).
Listing was done by IDA 4.9 disassembler from ELF-file generated by GCC and linker.}

\IFRU{Для тех кто хочет использовать IDA в процессе изучения, вот здесь лежат .elf и .idb файлы, 
.idb можно открыть при помощи бесплатой IDA 4.9:}
{For those who wants use IDA while learning, here you may find .elf and .idb files,
.idb can be opened with freeware IDA 4.9:}

\url{http://yurichev.com/RE-tasks/middle/1/}

\lstinputlisting{tasks/2_1_gcc.asm}

\subsection{\Task 2.2}

Имеется небольшой исполняемый файл, внутри которого находится довольно известная криптосистема.
Попробуйте её идентифицировать.

\begin{itemize}
\item
\href{http://yurichev.com/RE-tasks/middle/2/unknown_cryptosystem.exe}{Windows x86}

\item
\href{http://yurichev.com/RE-tasks/middle/2/unknown_encryption_linux86.tar}{Linux x86}

\item
\href{http://yurichev.com/RE-tasks/middle/2/unknown_encryption_MacOSX.tar}{MacOSX (x64)}
\end{itemize}

\subsection{\Task 2.3}

Имеется небольшой исполняемый файл, некая утилита.
Она открывает другой файл, читает его, что-то вычисляет и показывает число с плавающей точкой.
Попробуйте разобраться, что она делает.

\begin{itemize}
\item
\href{http://yurichev.com/RE-tasks/middle/3/unknown_utility_2_3.exe}{Windows x86}

\item
\href{http://yurichev.com/RE-tasks/middle/3/unknown_utility_2_3_Linux86.tar}{Linux x86}

\item
\href{http://yurichev.com/RE-tasks/middle/3/unknown_utility_2_3_MacOSX.tar}{MacOSX (x64)}
\end{itemize}

\section{crackme / keygenme}

\IFRU{Несколько моих keygenme\footnote{программа имитирующая защиту вымышленной программы, 
для которой нужно сделать генератор ключей/лицензий.}:}
{Couple of my keygenmes\footnote{program which imitates fictional software protection, 
for which one needs to make a keys/licenses generator}:}

\url{http://crackmes.de/users/yonkie/}


\part{\RU{Инструменты}\EN{Tools}}

\chapter{\RU{Дизассемблер}\EN{Disassembler}}

\section{IDA}

\label{IDA}
\RU{Старая бесплатная версия доступна для скачивания}\EN{Older freeware version is available for downloading}
\footnote{\url{http://www.hex-rays.com/idapro/idadownfreeware.htm}}.

\ShortHotKeyCheatsheet: \ref{sec:IDA_cheatsheet}

\chapter{\RU{Отладчик}\EN{Debugger}}

\section{tracer}

\index{tracer}
\label{tracer}
\RU{Я использую}\EN{I use} \IT{tracer}\footnote{\url{http://yurichev.com/tracer-\LANG.html}}
\RU{вместо отладчика}\EN{instead of debugger}.

\RU{Со временем я отказался использовать отладчик, потому что все что мне нужно от него: это иногда подсмотреть 
какие-либо аргументы какой-либо функции во время исполнения или состояние регистров в определенном месте. 
Каждый раз загружать отладчик для этого это слишком, поэтому я написал очень простую утилиту \IT{tracer}. 
Она консольная, запускается из командной строки, позволяет перехватывать исполнение функций, 
ставить брякпойнты на произвольные места, смотреть состояние регистров, модифицировать их, и так далее.}
\EN{I stopped to use debugger eventually, since all I need from it is to spot a function's arguments while
execution, or registers' state at some point.
To load debugger each time is too much, so I wrote a small utility \IT{tracer}.
It has console-interface, working from command-line, enable us to intercept function execution,
set breakpoints at arbitrary places, spot registers' state, modify it, etc.}

\RU{Но для учебы, очень полезно трассировать код руками в отладчике, наблюдать как меняются значения регистров 
(например, как минимум классический SoftICE, OllyDbg, WinDbg подсвечивают измененные регистры), 
флагов, данные, менять их самому, смотреть реакцию, и т.д.}
\EN{However, as for learning purposes, it is highly advisable to trace code in debugger manually, watch how register's state
changing (e.g. classic SoftICE, OllyDbg, WinDbg highlighting changed registers), flags, data, change them
manually, watch reaction, etc.}

\section{\olly}
\index{\olly}

\RU{Очень популярный отладчик пользовательской среды win32}\EN{Very popular user-mode win32 debugger}:\\
\url{http://www.ollydbg.de/}.

\ShortHotKeyCheatsheet: \label{sec:Olly_cheatsheet}

\section{GDB}
\index{GDB}

\RU{Не очень популярный отладчик у реверсеров, тем не менее, крайне удобный}\EN{Not very popular
debugger among reverse engineers, but very comfortable nevertheless}.
\RU{Некоторые команды}\EN{Some commands}: \ref{sec:GDB_cheatsheet}.

\chapter{\RU{Трассировка системных вызовов}\EN{System calls tracing}}

\label{strace}
\index{strace}
\index{dtruss}
\subsection{strace / dtruss}

\index{syscalls}
\RU{Позволяет показать, какие системные вызовы (syscalls(\ref{syscalls})) прямо сейчас вызывает процесс}
\EN{Will show which system calls (syscalls(\ref{syscalls})) are called by process right now}.
\RU{Например}\EN{For example}:

\begin{lstlisting}
# strace df -h

...

access("/etc/ld.so.nohwcap", F_OK)      = -1 ENOENT (No such file or directory)
open("/lib/i386-linux-gnu/libc.so.6", O_RDONLY|O_CLOEXEC) = 3
read(3, "\177ELF\1\1\1\0\0\0\0\0\0\0\0\0\3\0\3\0\1\0\0\0\220\232\1\0004\0\0\0"..., 512) = 512
fstat64(3, {st_mode=S_IFREG|0755, st_size=1770984, ...}) = 0
mmap2(NULL, 1780508, PROT_READ|PROT_EXEC, MAP_PRIVATE|MAP_DENYWRITE, 3, 0) = 0xb75b3000
\end{lstlisting}

\index{\MacOSX}
\RU{В \MacOSX для этого же имеется dtruss}\EN{\MacOSX has dtruss for the same aim}.

\index{cygwin}
\RU{В Cygwin также есть strace, впрочем, если я верно понял, 
он показывает результаты только для .exe-файлов скомпилированных для среды самого cygwin}
\EN{The Cygwin also has strace, but if I understood correctly, it works only for .exe-files
compiled for cygwin environment itself}.

\chapter{\RU{Декомпиляторы}\EN{Decompilers}}

\RU{Пока существует только один, публично доступный, декомпилятор в Си высокого качества}
\EN{There are only one known, publically available, high-quality decompiler to C code}: Hex-Rays:\\
\url{https://www.hex-rays.com/products/decompiler/}

% TODO Java, .NET, VB, etc

\chapter{\RU{Прочие инструменты}\EN{Other tools}}

\begin{itemize}
\item
Microsoft Visual Studio Express\footnote{\url{http://www.microsoft.com/express/Downloads/}}:
\RU{Усеченная бесплатная версия Visual Studio, пригодная для простых экспериментов}
\EN{Stripped-down free Visual Studio version, convenient for simple experiments}.
Some useful options: \ref{sec:MSVC_options}.

\item
\label{Hiew}
Hiew\footnote{\url{http://www.hiew.ru/}} \RU{для мелкой модификации кода в исполняемых файлах}
\EN{for small modifications of code in binary files}.

\item
\index{binary grep}
binary grep: \RU{небольшая утилита для поиска констант (либо просто последовательности байт)
в большом  кол-ве файлов, включая неисполняемые: \BGREPURL.}
\EN{the small utility for constants searching (or just any byte sequence) in a big pile of files, 
including non-executable: \BGREPURL.}
\end{itemize}


\chapter{\IFRU{Что стоит почитать}{Books/blogs worth reading}}

\section{\IFRU{Книги}{Books}}

\subsection{Windows}

\cite{Russinovich}.

\subsection{\CCpp}

\begin{itemize}
\item
\IFRU{Стандарт языка Си++}{C++ language standard}: ISO/IEC 14882:2003\footnote{\url{http://www.iso.org/iso/catalogue_detail.htm?csnumber=38110}}
\end{itemize}

\subsection{x86 / x86-64}

\begin{itemize}
\item
\IFRU{Документация от Intel}{Intel manuals}: \url{http://www.intel.com/products/processor/manuals/}
\item
\IFRU{Документация от AMD}{AMD manuals}: \url{http://developer.amd.com/documentation/guides/Pages/default.aspx#manuals}
\end{itemize}

\subsection{ARM}

\IFRU{Документация от ARM}{ARM manuals}: \url{http://infocenter.arm.com/help/index.jsp?topic=/com.arm.doc.subset.architecture.reference/index.html}

\section{\IFRU{Блоги}{Blogs}}

\subsection{Windows}

\begin{itemize}
\item
\href{http://blogs.msdn.com/oldnewthing/}{Microsoft: Raymond Chen}
\item
\url{http://www.nynaeve.net/}
\end{itemize}


\part{\RU{Еще примеры}\EN{More examples}}

% chapters here
\chapter{\EN{Task manager practical joke}\RU{Шутка с task manager} (Windows Vista)}

\RU{У меня только 4 ядра в процессоре в компьютере, так что Task Manager в Windows показывает только 4
графика загрузки процессора.}
\EN{I have only 4 CPU cores on my computer, so the Windows Task Manager shows only 4 CPU load graphs.}

\RU{Посмотрим, сможем ли мы немного хакнуть Task Manager, чтобы он находил больше ядер в компьютере.}
\EN{Let's see if it's possible to hack Task Manager slightly so it would detect more CPU cores on a computer.}

\RU{В начале задумаемся, откуда Task Manager знает количество ядер?}
\EN{Let us first think, how Task Manager would know number of cores?}
\RU{В win32 имеется ф-ция \TT{GetSystemInfo()}, при помощи которой можно узнать.}
\EN{There are \TT{GetSystemInfo()} win32 function present in win32 userspace which can tell us this.}
\RU{Но она не импортируется в}\EN{But it's not imported in} \TT{taskmgr.exe}.
\RU{Есть еще одна в \gls{NTAPI}, \TT{NtQuerySystemInformation()}, которая используется в 
\TT{taskmgr.exe} в ряде мест.}
\EN{There are, however, another one in \gls{NTAPI}, \TT{NtQuerySystemInformation()}, 
which is used in \TT{taskmgr.exe} in several places.}
\RU{Чтобы узнать количество ядер, нужно вызвать эту ф-цию с константной \TT{SystemBasicInformation} в 
первом аргументе (а это ноль}
\EN{To get number of cores, one should call this function with \TT{SystemBasicInformation} constant 
in first argument (which is zero}
\footnote{MSDN: \url{http://msdn.microsoft.com/en-us/library/windows/desktop/ms724509(v=vs.85).aspx}}).

\RU{Второй аргумент должен указывать на буфер, который примет всю информацию.}
\EN{Second argument should point to the buffer, which will receive all the information.}

\RU{Так что нам нужно найти все вызовы ф-ции \TT{NtQuerySystemInformation(0, ?, ?, ?)}.}
\EN{So we need to find all calls to the \TT{NtQuerySystemInformation(0, ?, ?, ?)} function.}
\RU{Откроем}\EN{Let's open} \TT{taskmgr.exe} \InENRU IDA. 
\RU{Что всегда хорошо с исполняемыми файлами от Microsoft, это то что IDA может скачать соответствующуий 
\gls{PDB}-файл именно для этого файла и добавить все имена ф-ций.}
\EN{What is always good about Microsoft executables is that IDA can download corresponding \gls{PDB} 
file for exactly this executable and add all function names.}
\RU{Видимо, Task Manager написан на C++ и некоторые ф-ции и классы имеют говорящие за себя имена.}
\EN{It seems, Task Manager written in C++ and some of function names and classes are really 
speaking for themselves.}
\RU{Тут есть классы}\EN{There are classes} CAdapter, CNetPage, CPerfPage, CProcInfo, CProcPage, CSvcPage, 
CTaskPage, CUserPage.
\RU{Должно быть, каждый класс соответствует каждой вкладке в Task Manager.}
\EN{Apparently, each class corresponding each tab in Task Manager.}

\RU{Я прошелся по всем вызовам и добавил комментарий с числом, передающимся как первый аргумент.}
\EN{I visited each call and I add comment with a value which is passed as the first function argument.}
\RU{В некоторых местах я написал ``not zero'', потому что значение в тех местах однозначно не ноль, 
но что-то другое (больше об этом во второй части главы).}
\EN{There are ``not zero'' I wrote at some places, because, the value there was not clearly zero, 
but something really different (more about this in the second part of this chapter).}
\RU{А мы все-таки ищем ноль передаваемый как аргумент.}
\EN{And we are looking for zero passed as argument after all.}

\begin{figure}[H]
\centering
\includegraphics[scale=\FigScale]{examples/taskmgr/IDA_xrefs.png}
\caption{IDA: \RU{вызовы ф-ции}\EN{cross references to} NtQuerySystemInformation()}
\end{figure}

\RU{Да, имена действительно говорящие сами за себя.}
\EN{Yes, the names are really speaking for themselves.}

\RU{Когда я внимательно изучил каждое место, где вызывается \TT{NtQuerySystemInformation(0, ?, ?, ?)},
я быстро нашел то что нужно в ф-ции \TT{InitPerfInfo()}:}
\EN{When I closely investigating each place where \TT{NtQuerySystemInformation(0, ?, ?, ?)} is called,
I quickly found what I need in the \TT{InitPerfInfo()} function:}

\lstinputlisting[caption=taskmgr.exe (Windows Vista)]{examples/taskmgr/taskmgr.lst}

\TT{g\_cProcessors} \RU{это глобальная переменная и это имя присвоено IDA в соответствии с \gls{PDB}-файлом,
скачанным с сервера символов Microsoft}\EN{is a global variable, and this name was assigned by 
IDA according to \gls{PDB} loaded from the Microsoft symbol server}.

\RU{Байт берется из}\EN{The byte is taken from} \TT{var\_C20}. 
\RU{И}\EN{And} \TT{var\_C58} \RU{передается в}\EN{is passed to} \TT{NtQuerySystemInformation()} 
\RU{как указатель на принимающий буфер}\EN{as a pointer to the receiving buffer}.
\RU{Разница между}\EN{The difference between} 0xC20 \AndENRU 0xC58 \RU{это}\EN{is} 0x38 (56).
\RU{Посмотрим на формат структуры, который можно найти в MSDN:}
\EN{Let's take a look at returning structure format, which we can find in MSDN:}

\begin{lstlisting}
typedef struct _SYSTEM_BASIC_INFORMATION {
    BYTE Reserved1[24];
    PVOID Reserved2[4];
    CCHAR NumberOfProcessors;
} SYSTEM_BASIC_INFORMATION;
\end{lstlisting}

\RU{Это система x64, так что каждый PVOID занимает здесь 8 байт.}
\EN{This is x64 system, so each PVOID takes 8 byte here.}
\RU{Так что все \IT{reserved}-поля занимают $24+4*8=56$.}
\EN{So all \IT{reserved} fields in the structure takes $24+4*8=56$.}
\RU{О да, это значит что }\EN{Oh yes, this means, }\TT{var\_C20} \RU{в локальном стеке это именно поле}\EN{is the 
local stack is exactly} \TT{NumberOfProcessors} \RU{структуры}\EN{field of the} 
\TT{SYSTEM\_BASIC\_INFORMATION}\EN{ structure}.

\RU{Проверим, прав ли я}\EN{Let's check if I'm right}.
\RU{Скопируем}\EN{Copy} \TT{taskmgr.exe} \RU{из}\EN{from} \TT{C:\textbackslash{}Windows\textbackslash{}System32} 
\RU{в какую-нибудь другую папку}\EN{to some other folder} 
(\RU{чтобы}\EN{so the} \IT{Windows Resource Protection} \RU{не пыталась восстанавливать измененный}\EN{will not 
try to restore patched} \TT{taskmgr.exe}).

\RU{Откроем его в Hiew и найдем это место:}
\EN{Let's open it in Hiew and find the place:}

\begin{figure}[H]
\centering
\includegraphics[scale=\FigScale]{examples/taskmgr/hiew1.png}
\caption{Hiew: \RU{найдем это место}\EN{find the place to be patched}}
\end{figure}

\RU{Заменим инструкцию \TT{MOVZX} на нашу.}
\EN{Let's replace \TT{MOVZX} instruction by our.}
\RU{Сделаем вид что у нас 64 ядра процессора}\EN{Let's pretend we've got 64 CPU cores}.
\RU{Добавим дополнительную инструкцию \ac{NOP} (потому что наша инструкция короче чем та что там сейчас):}
\EN{Add one additional \ac{NOP} (because our instruction is shorter than original one):}

\begin{figure}[H]
\centering
\includegraphics[scale=\FigScale]{examples/taskmgr/hiew1.png}
\caption{Hiew: \RU{меняем инструкцию}\EN{patch it}}
\end{figure}

\RU{И это работает}\EN{And it works}!
\RU{Конечно же, данные в графиках неправильные}\EN{Of course, data in graphs is not correct}.
\RU{Иногда, Task Manager даже показывает общую загрузку CPU более 100\%.}
\EN{At times, Task Manager even shows overall CPU load more than 100\%.}

\begin{figure}[H]
\centering
\includegraphics[scale=\FigScale]{examples/taskmgr/taskmgr_64cpu_crop.png}
\caption{\RU{Обманутый}\EN{Fooled} Windows Task Manager}
\end{figure}

\RU{Я выбрал число 64, потому что Task Manager падает если установить б\'{о}льшее значение.}
\EN{I picked number of 64, because Task Manager is crashing if you try to set larger value.}
\RU{Должно быть, Task Manager в Windows Vista не тестировался на компьютерах с большим количеством ядер.}
\EN{Apparently, Task Manager in Windows Vista was not tested on computer with larger count of cores.}
\RU{И наверное там есть внутри какие-то статичные структуры данных, ограниченные до 64-х ядер.}
\EN{So there are probably some static data structures inside it limited to 64 cores.}

\section{\RU{Использование LEA для загрузки значений}\EN{Using LEA to load values}}

\RU{Иногда, \TT{LEA} используется в \TT{taskmgr.exe} вместо \TT{MOV} для установки первого аргумента 
\TT{NtQuerySystemInformation()}:}
\EN{Sometimes, \TT{LEA} is used in \TT{taskmgr.exe} instead of \TT{MOV} to set first argument of 
\TT{NtQuerySystemInformation()}:}

\lstinputlisting[caption=taskmgr.exe (Windows Vista)]{examples/taskmgr/taskmgr2.lst}

\RU{Честно говоря, я не знаю почему, но \ac{MSVC} часто так делает.}
\EN{I honestly, don't know why, but that is what \ac{MSVC} often does.}
\RU{Может быть, это какая-то оптимизация и \TT{LEA} работает быстрее или лучше чем загрузка значения 
используя \TT{MOV}?}
\EN{Maybe this some kind of optiization and \TT{LEA} works faster or better than load 
value using \TT{MOV}?}

\clearpage
\chapter{\RU{Шутка с игрой Color Lines}\EN{Color Lines game practical joke}}
\label{chap:color_lines}

\RU{Это очень популярная игра с большим количеством реализаций}\EN{This is a very popular game with several 
implementations in existence}.
\RU{Возьмем одну из них, с названием}\EN{We can take one of them, called} BallTriX, \RU{от}\EN{from} 1997, 
\RU{доступную бесплатно на}\EN{available freely at} 
\url{http://go.yurichev.com/17311}.
\RU{Вот как она выглядит}\EN{Here is how it looks}:

\begin{figure}[H]
\centering
\includegraphics[scale=\FigScale]{examples/lines/1.png}
\caption{\RU{Обычный вид игры}\EN{How this game looks usually}}
\label{fig:lines_1}
\end{figure}

\clearpage
\index{\CStandardLibrary!rand()}
\RU{Посмотрим, сможем ли мы найти генератор псевдослучайных чисел и и сделать с ним одну шутку.}
\EN{So let's see, is it be possible to find the random generator and do some trick with it.}
\IDA \RU{быстро распознает стандартную функцию}\EN{quickly recognize the standard} \TT{\_rand} \RU{в}\EN{function in} 
\TT{balltrix.exe} \RU{по адресу}\EN{at} \TT{0x00403DA0}.
\IDA \RU{также показывает, что она вызывается только из одного места}\EN{also shows that it is called 
only from one place}:

\lstinputlisting{examples/lines/random.lst}

\RU{Назовем её}\EN{We'll call it} \q{random}.
\RU{Пока не будем концентрироваться на самом коде функции}\EN{Let's not to dive into this function's code yet}.

\RU{Эта функция вызывается из трех мест}\EN{This function is referred from 3 places}.

\RU{Вот первые два}\EN{Here are the first two}:

\lstinputlisting{examples/lines/1.lst}

\EN{Here is the third one}\RU{Вот третье}:

\lstinputlisting{examples/lines/2.lst}

\RU{Так что у функции только один аргумент}\EN{So the function has only one argument}.
\RU{10 передается в первых двух случаях и 5 в третьем.}
\EN{10 is passed in first two cases and 5 in third.}
\RU{Мы также можем заметить, что размер доски 10*10 и здесь 5 возможных цветов}\EN{We can also notice 
that the board has a size of 10*10 and there are 5 possible colors}.
\RU{Это оно}\EN{This is it}!
\RU{Стандартная функция}\EN{The standard} \TT{rand()} \RU{возвращает число в пределах}\EN{function returns 
a number in the} \TT{0..0x7FFF} \RU{и это неудобно, так что многие программисты пишут свою функцию,
возвращающую случайное число в некоторых заданных пределах}\EN{range and this is often inconvenient,
so many programmers implement their own random functions which returns a random number in a specified range}.
\RU{В нашем случае, предел это}\EN{In our case, the range is} $0..n-1$ \AndENRU $n$ \RU{передается как
единственный аргумент в функцию}\EN{is passed as the sole argument of the function}.
\RU{Мы можем быстро проверить это в отладчике}\EN{We can quickly check this in any debugger}.

\RU{Сделаем так, чтобы третий вызов функции всегда возвращал ноль}\EN{So let's fix the third function call to always return zero}.
\RU{В начале заменим три инструкции}\EN{First, we will replace three instructions} (\TT{PUSH/CALL/ADD}) 
\RU{на}\EN{by} \ac{NOP}s.
\RU{Затем добавим инструкцию}\EN{Then we'll add} \INS{XOR EAX, EAX}\RU{, для очистки регистра \EAX}\EN{ instruction, 
to clear the \EAX register}.

\lstinputlisting{examples/lines/fixed.lst}

\RU{Что мы сделали, это заменили вызов функции}\EN{So what we did is we replaced a call to the} \TT{random()} 
\RU{на код, всегда возвращающий ноль}\EN{function by a code which always returns zero}.

\clearpage
\RU{Теперь запустим}\EN{Let's run it now}:

\begin{figure}[H]
\centering
\includegraphics[scale=\FigScale]{examples/lines/2.png}
\caption{\RU{Шутка сработала}\EN{Practical joke works}}
\end{figure}

\RU{О да, это работает}\EN{Oh yes, it works}\footnote{\RU{Автор этой книги однажды сделал это как 
шутку для его сотрудников, в надежде что они перестанут играть. 
Надежды не оправдались.}\EN{Author of this book once did this as a joke for his coworkers with 
the hope that they would stop playing. They didn't.}}.

\RU{Но почему аргументы функции}\EN{But why are the arguments to the} \TT{random()} \RU{это глобальные 
переменные}\EN{functions global variables}?
\RU{Это просто потому что в настройках игры можно изменять размер доски, так что эти параметры не 
фиксированы}\EN{That's just because it's possible to change the board size in the game's settings, 
so these values are not hardcoded}.
\EN{The }10 \AndENRU 5 \RU{это просто значения по умолчанию}\EN{values are just defaults}.

\chapter{\MinesweeperWinXPExampleChapterName}
\label{minesweeper_winxp}
\index{Windows!Windows XP}

\RU{Для тех, кто не очень хорошо играет в Сапёра (Minesweeper), можно попробовать найти все скрытые мины в отладчике.}%
\EN{For those who is not very good at playing Minesweeper, we could try to reveal the hidden mines in the debugger.}

\index{\CStandardLibrary!rand()}
\index{Windows!PDB}
\RU{Как мы знаем, Сапёр располагает мины случайным образом, так что там должен быть генератор случайных чисел
или вызов стандартной функции Си \TT{rand()}.}
\EN{As we know, Minesweeper places mines randomly, so there has to be some kind of random number generator or
a call to the standard \TT{rand()} C-function.}
\RU{Вот что хорошо в реверсинге продуктов от Microsoft, так это то что часто есть \gls{PDB}-файл со всеми
символами (имена функций, \etc{}.).}
\EN{What is really cool about reversing Microsoft products is that there are \gls{PDB} 
file with symbols (function names, \etc{}).}
\RU{Когда мы загружаем}\EN{When we load} \TT{winmine.exe} \RU{в}\EN{into} \IDA, \RU{она скачивает}\EN{it downloads the} 
\gls{PDB} \RU{файл именно для этого исполняемого файла и добавляет все имена}\EN{file exactly for this 
executable and shows all names}.

\RU{И вот оно, только один вызов}\EN{So here it is, the only call to} \TT{rand()} \RU{в этой 
функции}\EN{is this function}:

\begin{lstlisting}
.text:01003940 ; __stdcall Rnd(x)
.text:01003940 _Rnd@4          proc near               ; CODE XREF: StartGame()+53
.text:01003940                                         ; StartGame()+61
.text:01003940
.text:01003940 arg_0           = dword ptr  4
.text:01003940
.text:01003940                 call    ds:__imp__rand
.text:01003946                 cdq
.text:01003947                 idiv    [esp+arg_0]
.text:0100394B                 mov     eax, edx
.text:0100394D                 retn    4
.text:0100394D _Rnd@4          endp
\end{lstlisting}

\RU{Так её назвала \IDA и это было имя данное ей разработчиками Сапёра.}
\EN{\IDA named it so, and it was the name given to it by Minesweeper's developers.}

\RU{Функция очень простая}\EN{The function is very simple}:

\begin{lstlisting}
int Rnd(int limit)
{
    return rand() % limit;
};
\end{lstlisting}

\RU{(В \gls{PDB}-файле не было имени \q{limit}; это мы назвали этот аргумент так, вручную.)}
\EN{(There was no \q{limit} name in the \gls{PDB} file; we manually named this argument like this.)}

\RU{Так что она возвращает случайное число в пределах от нуля до заданного предела}\EN{So it returns 
a random value from 0 to a specified limit}.

\TT{Rnd()} \RU{вызывается только из одного места, это функция с названием}\EN{is called only from one place, 
a function called} \TT{StartGame()}, 
\RU{и как видно, это именно тот код, что расставляет мины}\EN{and as it seems, this is exactly 
the code which place the mines}:

\begin{lstlisting}
.text:010036C7                 push    _xBoxMac
.text:010036CD                 call    _Rnd@4          ; Rnd(x)
.text:010036D2                 push    _yBoxMac
.text:010036D8                 mov     esi, eax
.text:010036DA                 inc     esi
.text:010036DB                 call    _Rnd@4          ; Rnd(x)
.text:010036E0                 inc     eax
.text:010036E1                 mov     ecx, eax
.text:010036E3                 shl     ecx, 5          ; ECX=ECX*32
.text:010036E6                 test    _rgBlk[ecx+esi], 80h
.text:010036EE                 jnz     short loc_10036C7
.text:010036F0                 shl     eax, 5          ; EAX=EAX*32
.text:010036F3                 lea     eax, _rgBlk[eax+esi]
.text:010036FA                 or      byte ptr [eax], 80h
.text:010036FD                 dec     _cBombStart
.text:01003703                 jnz     short loc_10036C7
\end{lstlisting}

\RU{Сапёр позволяет задать размеры доски, так что X (xBoxMac) и Y (yBoxMac) это глобальные переменные.}
\EN{Minesweeper allows you to set the board size, so the X (xBoxMac) and Y (yBoxMac) of the board are global variables.}
\RU{Они передаются в}\EN{They are passed to} \TT{Rnd()} \RU{и генерируются случайные координаты}\EN{and random 
coordinates are generated}.
\RU{Мина устанавливается инструкцией}\EN{A mine is placed by the} \TT{OR} \RU{на}\EN{instruction at} \TT{0x010036FA}. 
\RU{И если она уже была установлена до этого}\EN{And if it was placed before} 
(\RU{это возможно, если пара функций}\EN{it's possible if the pair of} \TT{Rnd()} 
\RU{сгенерирует пару, которая уже была сгенерирована}\EN{generates a coordinates pair which was already 
was generated}), 
\RU{тогда}\EN{then} \TT{TEST} \AndENRU \TT{JNZ} \RU{на}\EN{at} \TT{0x010036E6} 
\RU{перейдет на повторную генерацию пары}\EN{jumps to the generation routine again}.

\TT{cBombStart} \RU{это глобальная переменная, содержащая количество мин. Так что это цикл.}
\EN{is the global variable containing total number of mines. So this is loop.}

\RU{Ширина двухмерного массива это 32 (мы можем это вывести, глядя на инструкцию \TT{SHL}, которая умножает
одну из координат на 32)}\EN{The width of the array is 32 
(we can conclude this by looking at the \TT{SHL} instruction, which multiplies one of the coordinates by 32)}.

\RU{Размер глобального массива}\EN{The size of the} \TT{rgBlk} 
\RU{можно легко узнать по разнице между меткой}\EN{global array can be easily determined by the difference 
between the} \TT{rgBlk} 
\RU{в сегменте данных и следующей известной меткой}\EN{label in the data segment and the next known one}. 
\RU{Это}\EN{It is} 0x360 (864):

\begin{lstlisting}
.data:01005340 _rgBlk          db 360h dup(?)          ; DATA XREF: MainWndProc(x,x,x,x)+574
.data:01005340                                         ; DisplayBlk(x,x)+23
.data:010056A0 _Preferences    dd ?                    ; DATA XREF: FixMenus()+2
...
\end{lstlisting}

$864/32=27$.

\RU{Так что размер массива}\EN{So the array size is} $27*32$?
\RU{Это близко к тому что мы знаем: если попытаемся установить размер доски в установках Сапёра на $100*100$, то он установит размер $24*30$}%
\EN{It is close to what we know: when we try to set board size to $100*100$ in Minesweeper settings, it fallbacks to a board of size $24*30$}.
\RU{Так что это максимальный размер доски здесь}\EN{So this is the maximal board size here}.
\RU{И размер массива фиксирован для доски любого размера}\EN{And the array has a fixed size for any board size}.

\RU{Посмотрим на всё это в}\EN{So let's see all this in} \olly.
\RU{Запустим Сапёр, присоединим (attach) \olly к нему и увидим содержимое памяти по адресу где массив \TT{rgBlk} (\TT{0x01005340})}%
\EN{We will ran Minesweeper, attaching \olly to it and now we can see the memory dump at the address of the \TT{rgBlk} array (\TT{0x01005340})}
\footnote{\RU{Все адреса здесь для Сапёра под}\EN{All addresses here are for Minesweeper for} Windows XP SP3 English. 
\RU{Они могут отличаться для других сервис-паков}\EN{They may differ for other service packs}.}.

\RU{Так что у нас выходит такой дамп памяти массива}\EN{So we got this memory dump of the array}:

\lstinputlisting{examples/minesweeper/1.lst}

\olly, \RU{как и любой другой шестнадцатеричный редактор, показывает 16 байт на строку}\EN{like any other 
hexadecimal editor, shows 16 bytes per line}.
\RU{Так что каждая 32-байтная строка массива занимает ровно 2 строки}\EN{So each 32-byte array row occupies
exactly 2 lines here}.

\RU{Это уровень для начинающих (доска 9*9)}\EN{This is beginner level (9*9 board)}.

\RU{Тут еще какая-то квадратная структура, заметная визуально (байты 0x10)}\EN{There is some square 
structure can be seen visually (0x10 bytes)}.

\RU{Нажмем \q{Run} \InENRU \olly чтобы разморозить процесс Сапёра, потом нажмем в случайное место окна Сапёра, попадаемся на мине, но теперь
видны все мины}%
\EN{We will click \q{Run} \InENRU \olly to unfreeze the Minesweeper process, then we'll clicked randomly at the Minesweeper window 
and trapped into mine, but now all mines are visible}:

\begin{figure}[H]
\centering
\includegraphics[scale=\FigScale]{examples/minesweeper/1.png}
\caption{\RU{Мины}\EN{Mines}}
\label{fig:minesweeper1}
\end{figure}

\RU{Сравнивая места с минами и дамп, мы можем обнаружить что 0x10 это граница, 0x0F\EMDASH{}пустой блок, 
0x8F\EMDASH{}мина.}
\EN{By comparing the mine places and the dump, we can conclude that 0x10 stands for border, 0x0F\EMDASH{}empty block, 0x8F---mine.}

\RU{Теперь добавим комментариев и также заключим все байты 0x8F в квадратные скобки:}%
\EN{Now we'll add comments and also enclose all 0x8F bytes into square brackets:}

\lstinputlisting{examples/minesweeper/2.lst}

\RU{Теперь уберем все байты связанные с границами (0x10) и всё что за ними:}%
\EN{Now we'll remove all \IT{border bytes} (0x10) and what's beyond those:}

\lstinputlisting{examples/minesweeper/3.lst}

\RU{Да, это всё мины, теперь это очень хорошо видно, в сравнении со скриншотом.}
\EN{Yes, these are mines, now it can be clearly seen and compared with the screenshot.}

\clearpage
\RU{Вот что интересно, это то что мы можем модифицировать массив прямо в \olly.}%
\EN{What is interesting is that we can modify the array right in \olly.}
\RU{Уберем все мины заменив все байты 0x8F на 0x0F, и вот что получится в Сапёре}%
\EN{We can remove all mines by changing all 0x8F bytes by 0x0F, and here is what we'll get in Minesweeper}:

\begin{figure}[H]
\centering
\includegraphics[scale=\FigScale]{examples/minesweeper/3.png}
\caption{\RU{Все мины убраны в отладчике}\EN{All mines are removed in debugger}}
\label{fig:minesweeper3}
\end{figure}

\RU{Также уберем их все и добавим их в первом ряду}\EN{We can also move all of them to the first line}: 

\begin{figure}[H]
\centering
\includegraphics[scale=\FigScale]{examples/minesweeper/2.png}
\caption{\RU{Мины, установленные в отладчике}\EN{Mines set in debugger}}
\label{fig:minesweeper2}
\end{figure}

\RU{Отладчик не очень удобен для подсматривания (а это была наша изначальная цель), так что напишем маленькую
утилиту для показа содержимого доски:}%
\EN{Well, the debugger is not very convenient for eavesdropping (which was our goal anyway), so we'll write a small utility
to dump the contents of the board:}

\lstinputlisting{examples/minesweeper/minesweeper_cheater.c}

\RU{Просто установите}\EN{Just set the} \ac{PID}
\footnote{PID \RU{можно увидеть в}\EN{it can be seen in} Task Manager 
(\RU{это можно включить в}\EN{enable it in} \q{View $\rightarrow$ Select Columns})} 
\RU{и адрес массива}\EN{and the address of the array} (\TT{0x01005340} \RU{для}\EN{for} Windows XP SP3 English) 
\RU{и она покажет его}\EN{and it will dump it}
\footnote{\RU{Скомпилированная версия здесь}\EN{The compiled executable is here}: 
\href{http://go.yurichev.com/17165}{beginners.re}}.

\RU{Она подключается к win32-процессу по \ac{PID}-у и просто читает из памяти процесса по этому адресу.}
\EN{It attaches itself to a win32 process by \ac{PID} and just reads process memory an the address.}

\section{\Exercises}

\begin{itemize}

\item \RU{Почему байты описывающие границы (0x10) присутствуют вообще?}
\EN{Why do the \IT{border bytes} (0x10) exist in the array?}
\RU{Зачем они нужны, если они вообще не видимы в интерфейсе Сапёра?}
\EN{What they are for if they are not visible in Minesweeper's interface?}
\RU{Как можно обойтись без них}\EN{How could it work without them}?

\item \RU{Как выясняется, здесь больше возможных значений (для открытых блоков, для тех на которых игрок установил
	флажок, \etc{}.).}
	\EN{As it turns out, there are more values possible (for open blocks, for flagged by user, \etc{}).}
\RU{Попробуйте найти значение каждого}\EN{Try to find the meaning of each one}.

\item \RU{Измените мою утилиту так, чтобы она в запущенном процессе Сапёра убирала все мины, 
или расставляла их в соответствии с каким-то заданным шаблоном.}
\EN{Modify my utility so it can remove all mines or set them in a fixed pattern that you want in the Minesweeper
process currently running.}

\item \RU{Измените мою утилиту так, чтобы она работала без задаваемого адреса массива и без \gls{PDB}-файла.}
\EN{Modify my utility so it can work without the array address specified and without a \gls{PDB} file.}
\RU{Да, вполне возможно автоматически найти информацию о доске в сегменте данных в запущенном процессе Сапёра.}
\EN{Yes, it's possible to find board information in the data segment of Minesweeper's running process automatically.}
%\RU{Подсказка}\EN{Hint}: \myref{minesweeper_winxp_hint}.
%\RU{Подсказка}\EN{Hint}: \RU{подумайте о байтах, описывающих границы (0x10).}
%\EN{think about  \IT{border bytes} (0x10).}

\end{itemize}

\chapter{\IFRU{Ручная декомпиляция + использование SMT-солвера Z3 для взлома любительской криптографии}
{Hand decompiling + using Z3 SMT solver for defeating amateur cryptography}}

\IFRU{Любительская криптография обычно (непреднамеренно) очень слабая и может быть легко сломана ---
для криптографов, конечно}{Amateur cryptography is usually (unintentionally) 
very weak and can be breaked easily---for cryptographers, of course}.

\IFRU{Но представим на время что мы не в числе этих профессионалов}
{But let's pretend we are not among these crypto-professionals}.

\IFRU{Я нашел эту необратимую хэш-ф-цию, которая конвертирует одно 64-битное значение в другое,
и нам нужно попытаться развернуть её работу назад}
{I once found this one-way hash function, converting 64-bit value to another one and we need to try
to reverse its flow back}.

\label{hash_func}
\begin{quotation}
\index{\IFRU{Хеш-функции}{Hash functions}}
\index{CRC32}
\IFRU{Но что такое хеш-функция}{But what is hash-function}?
\IFRU{Простейший пример это CRC32, алгоритм ``более мощный'' чем простая контрольная сумма,
для проверки целостности данных}
{Simplest example is CRC32, an algorithm providing ``stronger'' checksum for integrity checking purposes}.
\IFRU{Невозможно восстановить оригинальный текст из хеша, там просто меньше информации: ведь текст
может быть очень длинным, но результат CRC32 всегда ограничен 32 битами}
{it is impossible to restore original text from the hash value, it just has much less information:
there can be long text, but CRC32 result is always limited to 32 bits}.
\IFRU{Но CRC32 не надежна в криптографическом смысле: известны методы как изменить текст таким образом,
чтобы получить нужный результат}
{But CRC32 is not cryptographically secure: it is known how to alter a text in that way so the resulting
CRC32 hash value will be one we need}.
\IFRU{Криптографические хеш-функции защищены от этого}
{Cryptographical hash functions are protected from this}.
\index{MD5}
\index{SHA1}
\IFRU{Такие ф-ции как MD5, SHA1, и т.д, широко используются для хеширования паролей
для хранения их в базе}
{They are widely used to hash user passwords in order to store them in the database, 
like MD5, SHA1, etc}.
\IFRU{Действительно: БД форума в интернете может и не хранить пароли 
(иначе злоумышленник получивший доступ к БД сможет узнать все пароли), а только хеши}
{Indeed: an internet forum database may not contain user passwords (stolen database will compromise
all user's passwords) but only hashes (a cracker will not be able to reveal passwords)}.
\IFRU{К тому же, скрипту интернет-форума вовсе не обязательно знать ваш пароль, он только должен
cверить его хеш с тем что лежит в БД, и дать вам доступ если cверка проходит}
{Besides, an internet forum engine is not aware of your password, it should only check if its hash
is the same as in the database, then it will give you access in this case}.
\IFRU{Один из самых простых способов взлома это просто перебирать все пароли и ждать пока
результат будет такой же как тот что нам нужен}
{One of the simplest passwords cracking methods is just to brute-force all passwords in order to wait
when resulting value will be the same as we need}.
\IFRU{Другие методы намного сложнее}{Other methods are much more complex}. \\
\end{quotation}

\section{\IFRU{Ручная декомпиляция}{Hand decompiling}}

\IFRU{Вот листинг на ассемблере в}{Here its assembly language listing in} \IDA:

\lstinputlisting{examples/z3/algo_1.asm}

\index{Hex-Rays}
\IFRU{Если вы не имеете}{If} Hex-Rays\IFRU{, либо вы не доверяете его результатам, мы можем попробовать
переписать всё это на Си вручную}{ is not in list of our posessions, or we distrust to it, 
we may try to reverse this code manually}.
\IFRU{Один из методов, это представить регистры \ac{CPU} в виде локальных переменных Си и заменить каждую инструкцию
эквивалентным выражением, например}{One method is to represent \ac{CPU} registers as local C variables and 
replace each instruction by one-line equivalent expression, like}:

\lstinputlisting{examples/z3/algo_2.c}

\IFRU{Если быть очень аккуратным, этот код можно скомпилировать и он даже будет работать, 
точно так же как оригинальный}
{If to be careful enough, this code can be compiled and will even work in the same way as original one}.

\IFRU{Затем, будем переписывать его постепенно, не забывая об использовании регистров}
{Then, we will rewrite it gradually, keeping in mind all registers usage}.
\IFRU{Внимание и фокусирование здесь крайне важно --- любая самая мелкая опечатка может испортить всю работу}
{Attention and focusing is very important here---any tiny typo may ruin all your work}!

\IFRU{Первый шаг}{Here is a first step}:

\lstinputlisting{examples/z3/algo_3.c}

\IFRU{Следующий шаг}{Next step}:

\lstinputlisting{examples/z3/algo_4.c}

\IFRU{Мы находим деление через умножение}{We may spot division using multiplication} (\ref{sec:divisionbynine}).
\index{Wolfram Mathematica}
\IFRU{Действительно, найдем делитель в}{Indeed, let's calculate divider in} Wolfram Mathematica:

\begin{lstlisting}[caption=Wolfram Mathematica]
In[1]:=N[2^(64 + 5)/16^^8888888888888889]
Out[1]:=60.
\end{lstlisting}

\IFRU{Получаем}{We get this}:

\lstinputlisting{examples/z3/algo_5.c}

\IFRU{Еще один шаг}{Another step}:

\lstinputlisting{examples/z3/algo_6.c}

\IFRU{Простым сокращением, мы видим, что вычислялось вовсе не \glslink{quotient}{частное}, а остаток от деления}
{By simple reducing, we finally see that it's not \gls{quotient} calculated, but division remainder}:

\lstinputlisting{examples/z3/algo_7.c}

\IFRU{Заканчиваем на приятно отформатированном исходном коде}{We end up on something fancy formatted source-code}:

\lstinputlisting{examples/z3/algo_src.c}

\IFRU{Так как мы не криптоаналитики, мы не можем найти простой способ найти входное значение
для определенного выходного значения}{Since we are not cryptoanalysts we can't find an easy way to 
generate input value for some specific output value}.
\IFRU{Коэффициенты инструкций сдвигов выглядят очень пугающе --- это гарантия что ф-ция не биективная,
она имеет коллизии, или, говоря проще, возможны несколько значений на входе для одного на выходе}
{Rotate instruction coefficients are look frightening---it's a warranty that the function is not bijective,
it has collisions, or, speaking more simply, many inputs may be possible for one output}.

\IFRU{Брут-форс это тоже не решение, т.к., значения 64-битные, и это совершенно нереально}
{Brute-force is not solution because values are 64-bit ones, that's beyond reality}.

\section{\IFRU{Попробуем Z3 SMT-солвер}{Now let's use Z3 SMT solver}}
\index{Z3}

\IFRU{Но все же, без всяких специальных знаний из криптографии, мы можем попытаться взломать алгоритм при помощи
великолепного SMT-солвера от}{Still, without any special cryptographical knowledge, we may try to break this 
algorithm using excellent SMT solver from} Microsoft Research \IFRU{под названием}{named} 
Z3\footnote{\url{http://z3.codeplex.com/}}.
\IFRU{На самом деле, это автоматический доказыватель теорем, но мы будем использовать его как SMT-солвер}
{It is in fact theorem prover, but we will use it as SMT solver}.
\IFRU{Упрощенно говоря, мы можем думать о нем как о системе, способной решать очень большие системы уравнений}
{In terms of simplicity, we may think about it as a system capable of solving huge equation systems}.

\IFRU{Вот исходный код на Питоне}{Here is a Python source code}:

\lstinputlisting[numbers=left]{examples/z3/1.py}

\IFRU{Это будет наш первый солвер}{This will be our first solver}.

\IFRU{На строке 7 мы видим объявление переменных}{We see variable difinitions on line 7}.
\IFRU{Это просто 64-битные переменные}{These are just 64-bit variables}.
\TT{i1..i6} \IFRU{это промежуточные переменные, отражающие значения в регистрах между исполнениями инструкций}
{are intermediate variables, representing values in registers between instruction executions}.

\IFRU{Потом добавляем т.н. констрайнты, в строках}{Then we add so called constraints on lines} 10..15.
\IFRU{Самый последний констрайнт в строке 17 это наиболее важный: мы будем искать входное значение для
нашего алгоритма, при котором он выдаст на выходе}{The very last constraint at 17 is most important: 
we will try to find input value for which our algorithm will produce} $10816636949158156260$.

\IFRU{Собственно, SMT-солвер ищет (любые) значения, удовлетворяющие всем констрайнтам}
{Essentially, SMT-solver searches for (any) values that satisfy all constraints}.

RotateRight, RotateLeft, URem\EMDASH{}\IFRU{это ф-ции из Питоновского Z3 \ac{API} для описания выражений, 
они не связаны с ЯП Python}
{are functions from Z3 Python \ac{API}, they are not related to Python \ac{PL}}.

\IFRU{Запускаем}{Then we run it}:

\begin{lstlisting}
...>python.exe 1.py
sat
[i1 = 3959740824832824396,
 i3 = 8957124831728646493,
 i5 = 10816636949158156260,
 inp = 1364123924608584563,
 outp = 10816636949158156260,
 i4 = 14065440378185297801,
 i2 = 4954926323707358301]
 inp=0x12EE577B63E80B73
outp=0x961C69FF0AEFD7E4
\end{lstlisting}

``sat'' \IFRU{означает}{mean} ``satisfiable'', \IFRU{т.е., солвер нашел по крайней мере одно решение}
{i.e., solver was able to found at least one solution}.
\IFRU{Решение выведено внутри квадратных скобок}{The solution is printed inside square brackets}.
\IFRU{Две последние строки это пара входного/выходного значения в шестнадцатеричном виде}
{Two last lines are input/output pair in hexadecimal form}.
\IFRU{Да, действительно, если мы запустим нашу ф-цию с}{Yes, indeed, if we run our function with} 
\TT{0x12EE577B63E80B73} \IFRU{на входе, алгоритм выдаст искомое значение}
{on input, the algorithm will produce the value we were looking for}.

\IFRU{Но, как мы заметили раннее, ф-ция не биективная, так что тут могут быть и другие корректные входные значения}
{But, as we are noticed before, the function we work with is not bijective, so there are may be other correct
input values}.
\IFRU{Z3 SMT-солвер не выдает результаты больше одного, но мы можем хакнуть наш пример немного, 
добавив констрайнт в строке 19, означая, что мы ищем какие угодно другие результаты кроме этого}
{Z3 SMT solver is not capable of producing more than one result, but let's hack our example slightly, 
by adding line 19, meaining, look for any other results than this}:

\lstinputlisting[numbers=left]{examples/z3/2.py}

\IFRU{Действительно, получаем еще один верный результат}{Indeed, it found other correct result}:

\begin{lstlisting}
...>python.exe 2.py
sat
[i1 = 3959740824832824396,
 i3 = 8957124831728646493,
 i5 = 10816636949158156260,
 inp = 10587495961463360371,
 outp = 10816636949158156260,
 i4 = 14065440378185297801,
 i2 = 4954926323707358301]
 inp=0x92EE577B63E80B73
outp=0x961C69FF0AEFD7E4
\end{lstlisting}

\IFRU{Это можно автоматизировать}{This can be automated}.
\IFRU{Каждый найденный результат можно добавлять в качестве констрайнта и искать следующий}
{Each found result may be added as constraint and the next result will be searched for}.
\IFRU{Пример немного сложнее}{Here is slightly sophisticated example}:

\lstinputlisting[numbers=left]{examples/z3/3.py}

\IFRU{Получаем}{We got}:

\begin{lstlisting}
1364123924608584563
1234567890
9223372038089343698
4611686019661955794
13835058056516731602
3096040143925676201
12319412180780452009
7707726162353064105
16931098199207839913
1906652839273745429
11130024876128521237
15741710894555909141
6518338857701133333
5975809943035972467
15199181979890748275
10587495961463360371
results total= 16
\end{lstlisting}

\IFRU{Так что имеется 16 верных входных значений для}{So there are 16 correct input values are possible for} 
\TT{0x92EE577B63E80B73} \IFRU{на выходе}{as a result}.

\IFRU{Второй это}{The second is} $1234567890$\EMDASH{}\IFRU{действительно, я это и использовал изначально,
когда готовил этот пример}
{it is indeed a value I used originally while preparing this example}.

\IFRU{Попробуем изучить алгоритм немного больше}{Let's also try to research our algorithm more}.
\IFRU{В порыве садистских желаний, попробуем найти, есть ли здесь какая-нибудь возможная пара входов/выходов,
в которых младшие 32-битные части равны друг другу}
{By some sadistic purposes, let's find, are there any possible input/output pair in 
which lower 32-bit parts are equal to each other}?

\IFRU{Уберем констрайнт}{Let's remove} \IT{outp} \IFRU{и добавим другой, в строке 17}
{constraint and add another, at line 17}:

\lstinputlisting[numbers=left]{examples/z3/4.py}

\IFRU{И действительно}{It is indeed so}:

\begin{lstlisting}
sat
[i1 = 14869545517796235860,
 i3 = 8388171335828825253,
 i5 = 6918262285561543945,
 inp = 1370377541658871093,
 outp = 14543180351754208565,
 i4 = 10167065714588685486,
 i2 = 5541032613289652645]
 inp=0x13048F1D12C00535
outp=0xC9D3C17A12C00535
\end{lstlisting}

\IFRU{Можем упражняться в садизме и далее: пусть последние 16-бит всегда будут}
{Let's be more sadistic and add another constaint: last 16-bit should be} \TT{0x1234}:

\lstinputlisting[numbers=left]{examples/z3/5.py}

\IFRU{Это так же возможно}{Oh yes, this possible as well}:

\begin{lstlisting}
sat
[i1 = 2834222860503985872,
 i3 = 2294680776671411152,
 i5 = 17492621421353821227,
 inp = 461881484695179828,
 outp = 419247225543463476,
 i4 = 2294680776671411152,
 i2 = 2834222860503985872]
 inp=0x668EEC35F961234
outp=0x5D177215F961234
\end{lstlisting}

\IFRU{Z3 работает крайне быстро и это означает что алгоритм слаб, и вообще не относится к криптографическим 
(как и почти вся любительская криптография)}
{Z3 works very fast and it means that algorithm is weak, it is not cryptographical at all
(like the most of amateur cryptography)}.

\IFRU{Можно ли попытаться сделать что-то подобное с настоящими криптоалгоритмами этими методами}
{Will it be possible to tackle real cryptography by these methods}? 
\IFRU{Настоящие алгоритмы, такие как}{Real algorithms like} AES, RSA, \IFRU{итд, так же могут быть представлены
в виде огромных систем уравнений, но они большие настолько, что с ними нельзя работать на компьютерах,
ни сейчас, ни в обозримом будущем}{etc, can also be represented as huge system of equations, 
but these are that huge that are impossible to work with on computers, now or in near future}.
\IFRU{Разумеется, криптографы об этом всем прекрасно знают}{Of course, cryptographers are aware of this}.

\IFRU{Еще одна статья которую я написал о Z3:}{Another article I wrote about Z3 is} \cite{Rockey}.


\chapter{\RU{Донглы}\EN{Dongles}}
\label{dongles}

\RU{Иногда я делаю замену \glslink{dongle}{донглам} или ``эмуляторы донглов'' 
и здесь немного примеров, как это происходит}
\EN{Occasionally I do software copy-protection \gls{dongle} replacements, or ``dongle emulators'' and here
are couple examples of my work}
\footnote{\RU{Больше об этом читайте тут}\EN{Read more about it}: \url{http://yurichev.com/dongles.html}}.

\RU{Об одном неописанном здесь случае вы также можете прочитать здесь}
\EN{About one of not described cases you may also read here}: \cite{Rockey}.

\section{\RU{Пример}\EN{Example} \#1: MacOS Classic \AndENRU PowerPC}

\index{PowerPC}
\index{Mac OS Classic}
\RU{Вот пример программы для}\EN{Here is an example of a program for} MacOS Classic
\footnote{\RU{MacOS перед тем как перейти на UNIX}\EN{pre-UNIX MacOS}}, \RU{для}\EN{for} PowerPC.
\RU{Компания, разработавшая этот продукт, давно исчезла, так что (легальный) пользователь
боялся того что донгла может сломаться}\EN{The company who developed the software product
has disappeared a long time ago, so the (legal) customer was afraid of physical dongle damage}.

\RU{Если запустить программу без подключенной донглы, можно увидеть окно с надписью}
\EN{While running without a dongle connected, a message box with the text}
"Invalid Security Device"\EN{ appeared}.
\RU{Мне повезло потому что этот текст можно было легко найти внутри исполняемого файла}
\EN{Luckily, this text string could easily be found in the executable binary file}.

\RU{Представим, что мы не знакомы ни с Mac OS Classic, ни с PowerPC, но всё-таки попробуем}%
\EN{Let's pretend we are not very familiar both with Mac OS Classic and PowerPC, but will try anyway}.

\ac{IDA} \RU{открывает исполняемый файл легко, показывая его тип как}
\EN{opened the executable file smoothly, reported its type as} 
"PEF (Mac OS or Be OS executable)" (\RU{действительно, это стандартный тип файлов в Mac OS Classic}
\EN{indeed, it is a standard Mac OS Classic file format}).

\RU{В поисках текстовой строки с сообщение об ошибке, мы попадаем на этот фрагмент кода}%
\EN{By searching for the text string with the error message, we've got into this code fragment}:

\lstinputlisting{examples/dongles/1/1.lst}

\index{ARM}
\index{MIPS}
\RU{Да, это код PowerPC}\EN{Yes, this is PowerPC code}.
\RU{Это очень типичный процессор для \ac{RISC} 1990-х}
\EN{The CPU is a very typical 32-bit \ac{RISC} of 1990s era}.
\RU{Каждая инструкция занимает 4 байта (как и в MIPS и ARM) и их имена немного похожи на имена 
инструкций MIPS}
\EN{Each instruction occupies 4 bytes (just as in MIPS and ARM) and the names somewhat resemble
MIPS instruction names}.

\TT{check1()} \RU{это имя которое мы дадим этой функции немного позже}\EN{is a function name we'll give to it later}.
\TT{BL} \RU{это инструкция}\EN{is} \IT{Branch Link} 
\RU{т.е. предназначенная для вызова подпрограмм}\EN{instruction, e.g., intended for calling subroutines}.
\RU{Самое важное место\EMDASH{}это инструкция \ac{BNE}, срабатывающая, если проверка наличия донглы прошла
успешно, либо не срабатывающая в случае ошибки: и тогда адрес текстовой строки с сообщением об ошибке
будет загружен в регистр r3 для последующей передачи в функцию отображения диалогового окна}
\EN{The crucial point is the \ac{BNE} instruction which jumps if the dongle protection check passes 
or not if an error occurs: 
then the address of the text string gets loaded into the r3 register for the subsequent passing into a message box routine}.

\RU{Из}\EN{From the} \cite{PPCABI} \RU{мы узнаем, что регистр r3 используется для возврата
значений (и еще r4 если значение 64-битное)}%
\EN{we will found out that the r3 register is used for return values (and r4, in case of 64-bit values)}.

\index{x86!\Instructions!MOVZX}
\RU{Еще одна, пока что неизвестная инструкция}\EN{Another yet unknown instruction is} \TT{CLRLWI}. 
\RU{Из}\EN{From} \cite{PPC} \RU{мы узнаем, что эта инструкция одновременно и очищает и загружает}%
\EN{we'll learn that this instruction does both clearing and loading}. 
\RU{В нашем случае, она очищает 24 старших бита из значения в r3 и записывает всё это в r0, 
так что это аналог}\EN{In our case, it clears the 24 high bits from the value in r3
and puts them in r0, so it is analogical to} \MOVZX \InENRU x86 (\myref{movzx}),
\RU{но также устанавливает флаги, так что}\EN{but it also sets the flags, so} \ac{BNE} 
\RU{может проверить их потом}\EN{can check them afterwards}.

\RU{Посмотрим внутрь}\EN{Let's take a look into the} \TT{check1()}\EN{ function}:

\lstinputlisting{examples/dongles/1/check1.lst}

\RU{Как можно увидеть в \ac{IDA}, эта функция вызывается из многих мест в программе, но только значение
в регистре r3 проверяется сразу после каждого вызова}
\EN{As you can see in \ac{IDA}, that function is called from many places in the program, but only the r3 register's value
is checked after each call}.
\index{thunk-\RU{функции}\EN{functions}}
\RU{Всё что эта функция делает это только вызывает другую функцию, так что это}
\EN{All this function does is to call the other function, so it is a} \gls{thunk function}: 
\RU{здесь присутствует и пролог функции и эпилог, но регистр r3 не трогается, так что}
\EN{there are function prologue and epilogue, but the r3 register is not touched, so} \TT{checkl()} 
\RU{возвращает то, что возвращает}\EN{returns what} \TT{check2()}\EN{ returns}.

\ac{BLR} \RU{это похоже возврат из функции, но так как IDA делает всю разметку функций автоматически,
наверное, мы можем пока не интересоваться этим}\EN{looks like the return from the function, but since \ac{IDA} does the function layout, we probably do not need
to care about this}.
\RU{Так как это типичный \ac{RISC}, похоже, подпрограммы вызываются, используя}
\EN{Since it is a typical \ac{RISC}, it seems that subroutines are called using a} \gls{link register},
\RU{точно как в}\EN{just like in} ARM.

\EN{The}\RU{Функция} \TT{check2()} \RU{более сложная}\EN{function is more complex}:

\lstinputlisting{examples/dongles/1/check2.lst}

\index{USB}
\RU{Снова повезло: имена некоторых функций оставлены в исполняемом файле
(в символах в отладочной секции? Трудно сказать до тех пор, пока мы не знакомы с этим форматом файлов,
может быть это что-то вроде PE-экспортов (\myref{PE_exports_imports}))?
как например \TT{.RBEFINDNEXT()} and \TT{.RBEFINDFIRST()}.}
\EN{We are lucky again: some function names are left in the executable 
(debug symbols section? Hard to say while we are not very familiar with the file format, maybe it is
some kind of PE exports? (\myref{PE_exports_imports})),
like \TT{.RBEFINDNEXT()} and \TT{.RBEFINDFIRST()}.}
\RU{В итоге, эти функции вызывают другие функции с именами вроде}
\EN{Eventually these functions call other functions with names like} \TT{.GetNextDeviceViaUSB()}, 
\TT{.USBSendPKT()},
\RU{так что они явно работают с каким-то USB-устройством}\EN{so these are clearly dealing with an USB device}.

\RU{Тут даже есть функция с названием}\EN{There is even a function named} 
\TT{.GetNextEve3Device()}\EMDASH\RU{звучит знакомо, в 1990-х годах была донгла}\EN{sounds familiar, there was a} Sentinel Eve3 
\RU{для ADB-порта (присутствующих на Макинтошах)}\EN{dongle for ADB port (present on Macs) in 1990s}.

\RU{В начале посмотрим на то как устанавливается регистр r3 одновременно игнорируя всё остальное}
\EN{Let's first take a look on how the r3 register is set before return, while ignoring everything else}.
\RU{Мы знаем, что \q{хорошее} значение в r3 должно быть не нулевым, а нулевой r3 приведет
к выводу диалогового окна с сообщением об ошибке.}
\EN{We know that a \q{good} r3 value has to be non-zero, zero r3 leads the execution
flow to the message box with an error message.}

\RU{В функции имеются две инструкции}\EN{There are two} \TT{li \%r3, 1} 
\RU{и одна}\EN{instructions present in the function and one} \TT{li \%r3, 0} 
(\IT{Load Immediate}, \RU{т.е. загрузить значение в регистр}\EN{i.e., loading a value into a register}).
\RU{Самая первая инструкция находится на}\EN{The first instruction is at} 
\TT{0x001186B0}\EMDASH\RU{и честно говоря, трудно заранее понять, что это означает}%
\EN{and frankly speaking, it's hard to say what it means}.

\RU{А вот то что мы видим дальше понять проще}\EN{What we see next is, however, easier to understand}: 
\RU{вызывается }\TT{.RBEFINDFIRST()} \RU{и в случае ошибки, 0 будет записан в r3
и мы перейдем на \IT{exit}, а иначе будет вызвана функция \TT{check3()}\EMDASH{}если и она будет
выполнена с ошибкой, будет вызвана}\EN{is called:
if it fails, 0 is written into r3 and we jump to \IT{exit}, otherwise another
function is called (\TT{check3()})\EMDASH{}if it fails too, }
\TT{.RBEFINDNEXT()} \RU{вероятно, для поиска другого USB-устройства}
\EN{is called, probably in order to look for another USB device}.

N.B.: \TT{clrlwi. \%r0, \%r3, 16} \RU{это аналог того что мы уже видели, но она очищает 16 старших бит,
т.е.}\EN{it is analogical to what we already saw, but it clears
16 bits, i.e.}, \TT{.RBEFINDFIRST()} \RU{вероятно возвращает 16-битное значение}
\EN{probably returns a 16-bit value}.

\TT{B} (\RU{означает}\EN{stands for} \IT{branch}) \RU{\EMDASH{}безусловный переход}\EN{unconditional jump}.

\ac{BEQ} \RU{это обратная инструкция от}\EN{is the inverse instruction of} \ac{BNE}.

\RU{Посмотрим на}\EN{Let's see} \TT{check3()}:

\lstinputlisting{examples/dongles/1/check3.lst}

\RU{Здесь много вызовов}\EN{There are a lot of calls to} \TT{.RBEREAD()}. 
\RU{Эта функция вероятно читает какие-то значения из донглы, которые потом сравниваются здесь при помощи}
\EN{The function probably returns some values from the dongle,
so they are compared here with some hard-coded variables using} \TT{CMPLWI}.

\RU{Мы также видим в регистр r3 записывается перед каждым вызовом}
\EN{We also see that the r3 register is also filled before each call to} \TT{.RBEREAD()} 
\RU{одно из этих значений}\EN{with one of these values}: 0, 1, 8, 0xA, 0xB, 0xC, 0xD, 4, 5.
\RU{Вероятно адрес в памяти или что-то в этом роде}\EN{Probably a memory address or something like that}?

\RU{Да, действительно, если погуглить имена этих функций, можно легко найти документацию к}
\EN{Yes, indeed, by googling these function names it is easy to find the} Sentinel Eve3\EN{ dongle manual}!

\RU{Hаверное, уже и не нужно изучать остальные инструкции PowerPC: всё что делает эта функция это просто
вызывает}
\EN{Perhaps, we don't even need to learn any other PowerPC instructions: all this function does is just
call} \TT{.RBEREAD()}, \RU{сравнивает его результаты с константами и возвращает 1 если результат сравнения положительный или 0 в другом случае}\EN{compare its results with the constants and returns 1 if the comparisons
are fine or 0 otherwise}.

\RU{Всё ясно: \TT{check1()} должна всегда возвращать 1 или иное ненулевое значение}
\EN{OK, all we've got is that \TT{check1()} has always to return 1 or any other non-zero value}.
\RU{Но так как мы не очень уверены в своих знаниях инструкций PowerPC, будем осторожны и пропатчим переходы в \TT{check2} на адресах
\TT{0x001186FC} и \TT{0x00118718}.}
\EN{But since we are not very confident in our knowledge of PowerPC instructions, we are going to be careful: we will patch the jumps in \TT{check2()} at
\TT{0x001186FC} and \TT{0x00118718}.}

\RU{На}\EN{At} \TT{0x001186FC} \RU{мы записываем байты}\EN{we'll write bytes} 0x48 \AndENRU 0 
\RU{таким образом превращая инструкцию}\EN{thus converting the} \ac{BEQ} 
\RU{в инструкцию}\EN{instruction in an} 
\TT{B} (\RU{безусловный переход}\EN{unconditional jump}):
\RU{Мы можем заметить этот опкод прямо в коде даже без обращения к}%
\EN{We can spot its opcode in the code without even referring to} \cite{PPC}.

\RU{На}\EN{At} \TT{0x00118718} \RU{мы записываем байт}\EN{we'll write} 0x60 \AndENRU \RU{еще 3 нулевых байта,
таким образом превращая её в инструкцию}\EN{3 zero bytes, thus converting it to a}
\ac{NOP}\EN{ instruction}:
\RU{Этот опкод мы также можем подсмотреть прямо в коде}\EN{Its opcode we could spot in the code too}.

\RU{И всё заработало без подключенной донглы}\EN{And now it all works without a dongle connected}.

\RU{Резюмируя, такие простые модификации можно делать в \ac{IDA} даже с минимальными знаниями
ассемблера}\EN{In summary, such small modifications can be done with \ac{IDA} and minimal assembly language knowledge}.


\subsection{\IFRU{Пример}{Example} \#2: SCO OpenServer}

\index{SCO OpenServer}
\IFRU{Древняя программа для}{An ancient software for} SCO OpenServer \IFRU{от}{from} 1997 
\IFRU{разработанная давно исчезнувшей компанией}{developed
by a company disappeared long time ago}.

\IFRU{Специальный драйвер донглы инсталлируется в системе, он содержит такие текстовые строки}
{There is a special dongle driver to be installed in the system, containing text strings}:
``Copyright 1989, Rainbow Technologies, Inc., Irvine, CA''
\AndENRU
``Sentinel Integrated Driver Ver. 3.0 ''.

\IFRU{После инсталляции драйвера, в /dev появляются такие устройства}
{After driver installation in SCO OpenServer, these device files are appeared in /dev filesystem}:

\begin{lstlisting}
/dev/rbsl8
/dev/rbsl9
/dev/rbsl10
\end{lstlisting}

\IFRU{Без подключенной донглы, программа сообщает об ошибке, но сообщение об ошибке не удается
найти в исполняемых файлах}
{The program without dongle connected reports error, but the error string cannot be found in the executables}.

\index{COFF}
\IFRU{Еще раз спасибо \ac{IDA}, она легко загружает исполняемые файлы формата COFF использующиеся в}
{Thanks to \ac{IDA}, it does its job perfectly working out COFF executable used in} SCO OpenServer.

\IFRU{Я попробовал также поискать строку}{I've tried to find} ``rbsl'' 
\IFRU{, и действительно, её можно найти в таком фрагменте кода}
{and indeed, found it in this code fragment}:

\lstinputlisting{examples/dongles/2/1.lst}

\IFRU{Действительно, должна же как-то программа обмениваться информацией с драйвером}
{Yes, indeed, the program should comminicate with driver somehow and that is how it is}.

\index{thunk-\IFRU{функции}{functions}}
\IFRU{Единственное место где вызывается ф-ция}{The only place} \TT{SSQC()}
\IFRU{это}{function called is the} \gls{thunk function}:

\lstinputlisting{examples/dongles/2/2.lst}

\IFRU{А вот }{}SSQ() \IFRU{вызывается по крайней мерез из двух разных ф-ций}
{is called at least from 2 functions}.

\IFRU{Одна из них}{One of these is}:

\lstinputlisting{examples/dongles/2/check1.lst}

``\TT{3C}'' \AndENRU ``\TT{3E}'' \IFRU{~--- это звучит знакомо: когда-то была донгла}
{are sounds familiar: there was a} Sentinel Pro \IFRU{от Rainbow без памяти,
предоставляющая только одну секретную крипто-хеширующую ф-цию}{dongle by Rainbow with no memory,
providing only one crypto-hashing secret function}.

\begin{quotation}
\index{\IFRU{Хеш-функции}{Hash functions}}
\index{CRC32}
\IFRU{Но что такое хеш-функция}{But what is hash-function}?
\IFRU{Простейший пример это CRC32, алгоритм ``более мощный'' чем простая контрольная сумма,
для проверки целостности данных}
{Simplest example is CRC32, an algorithm providing ``stronger'' checksum for integrity checking purposes}.
\IFRU{Невозмжоно восстановить оригинальный текст из хеша, там просто меньше информации: ведь текст
может быть очень длинным, но результат CRC32 всегда ограничен 32 битами}
{It is not possible to restore original text from the hash value, it just has much less information:
there can be long text, but CRC32 result is always limited to 32 bits}.
\IFRU{Но CRC32 не надежна в криптографическом смысле: известны методы как изменить текст таким образом,
чтобы получить нужный результат}
{But CRC32 is not cryptographically secure: it is known how to alter a text in that way so the resulting
CRC32 hash value will be one we need}.
\IFRU{Криптографические хеш-функции защищены от этого}
{Cryptographical hash functions are protected from this}.
\index{MD5}
\index{SHA1}
\IFRU{Такие ф-ции как MD5, SHA1, итд, широко используются для хеширования паролей
для хранения их в базе}
{They are widely used to hash user passwords in order to store them in the database, 
like MD5, SHA1, etc}.
\IFRU{Действительно: БД форума в интернете может и не хранить пароли 
(иначе злоумышленник получивший доступ к БД сможет узнать все пароли), а только хеши}
{Indeed: an internet forum database may not contain user passwords (stolen database will compromise
all user's passwords) but only hashes (a cracker will not be able to reveal passwords)}.
\IFRU{К тому же, скрипту интернет-форума вовсе не обязательно знать ваш пароль, он только должен
cверить его хеш с тем что лежит в БД, и дать вам доступ если cверка проходит}
{Besides, an internet forum engine is not aware of your password, it should only check if its hash
is the same as in the database, then it will give you access in this case}.
\IFRU{Один из самых простых способов взлома это просто перебирать все пароли и ждать пока
результат будет такой же как тот что нам нужен}
{One of the simplest passwords cracking methods is just to brute-force all passwords in order to wait
when resulting value will be the same as we need}.
\IFRU{Другие методы намного сложнее}{Other methods are much more complex}. \\
\end{quotation}

\IFRU{Но вернемся к нашей программе}{But let's back to the program}.
\IFRU{Так что программа может только проверить подключена ли донгла или нет}
{So the program can only check the presence or absence dongle connected}.
\IFRU{Никакой больше информации в такую донглу без памяти записать нельзя}
{No other information can be written to such dongle with no memory}.
\IFRU{Двухсимвольные коды это команды}{Two-character codes are commands}
(\IFRU{можно увидеть как они обрабатывюатся в ф-ции}{we can see how commands are handled in} 
\TT{SSQC()}\IFRU{}{ function}) 
\IFRU{а все остальные строки хешируются внутри донглы превращаясь в 16-битное число}
{and all other strings are hashed inside the dongle transforming into 16-bit number}.
\IFRU{Алгоритм был секретный, так что нельзя было написать замену драйверу или сделать
электронную копию донглы идеально эмулирующую алгоритм}{The algorithm was secret,
so it was not possible to write driver replacement or to remake dongle hardware emulating it perfectly}.
\IFRU{С другой стороны, всегда можно было перехватить все обращения к ней и найти те константы, с которыми
сравнивается результат хеширования}
{However, it was always possible to intercept all accesses to it and to find what constants
the hash function results compared to}.
\IFRU{Но надо сказать, вполне возможно создать устойчивую защиту от копирования базирующуюся
на секретной хеш-функции: пусть она шифрует все файлы с которыми ваша программа работает}
{Needless to say it is possible to build a robust software copy protection scheme based on secret
cryptographical hash-function: let it to encrypt/decrypt data files your software working with}.

\IFRU{Но вернемся к нашему коду}{But let's back to the code}.

\IFRU{Коды}{Codes} 51/52/53 \IFRU{используются для выбора номера принтеровского LPT-порта}
{are used for LPT printer port selection}.
3x/4x \IFRU{используются для выбора}{is for} ``family'' \IFRU{так донглы Sentinel Pro
можно отличать друг от друга: ведь более одной донглы может быть подключено к LPT-порту}
{selection (that's how Sentinel Pro dongles are differentiated from each other: more than one
dongle can be connected to LPT port)}.

\IFRU{Единственная строка передающаяся в хеш-функцию это}
{The only non-2-character string passed to the hashing function is} "0123456789".
\IFRU{Затем результат сравнивается с несколькими правильными значениями}
{Then, the result is compared against the set of valid results}.
\IFRU{Если результат правилен}{If it is correct},
0xC \OrENRU 0xB \IFRU{будет записано в глобальную переменную}
{is to be written into global variable} \TT{ctl\_model}.

\IFRU{Еще одна строка для хеширования:}{Another text string to be passed is}
"PRESS ANY KEY TO CONTINUE: ", \IFRU{но результат не проверяется}{but the result is not checked}.
\IFRU{Не знаю зачем это, может быть по ошибке}{I don't know why, probably by mistake}.
(\IFRU{Это очень странное чувство: находить ошибки в столь древнем ПО}
{What a strange feeling: to reveal bugs in such ancient software}.)

\IFRU{Давайте посмотрим, где проверяется значение глобальной переменной}
{Let's see where the value from the global variable} \TT{ctl\_mode}\IFRU{}{ is used}.

\IFRU{Одно из таких мест}{One of such places is}:

\lstinputlisting{examples/dongles/2/4.lst}

\IFRU{Если оно 0, шифрованное сообщение об ошибке будет передано в ф-цию дешифрования и оно будет 
показано}{If it is 0, an encrypted error message is passed into decryption routine and printed}.

\index{x86!\Instructions!XOR}
\IFRU{Ф-ция дешифровки сообщений об ошибке похоже применяет простой \ac{XOR}}
{Error strings decryption routine is seems simple xoring}:

\lstinputlisting{examples/dongles/2/err_warn.lst}

% TODO: reverse the function with examples

\IFRU{Вот почему не получилось найти сообщение об ошибке в исполняемых файлах, потому что оно было
зашифровано, это очень популярная практика}
{That's why I was unable to find error messages in the executable files, because they are enrcypted,
this is popular practice}.

\IFRU{Еще один вызов хеширующей ф-ции передает строку}{Another call to \TT{SSQ()} hashing function passes}
``offln'' \IFRU{и сравнивает результат с константами}{string to it and comparing result with}
\TT{0xFE81} \AndENRU \TT{0x12A9}.
\IFRU{Если результат не сходится, происходит работа с какой-то ф-цией}
{If it not so, it deals with some} \TT{timer()} 
\IFRU{(может быть для ожидания плохо подключенной донглы и нового запроса?), затем дешифрует
еще одно сообщение об ошибке и выводит его}{function (maybe waiting for poorly
connected dongle to be reconnected and check again?) and then decrypt another error message to dump}.

\lstinputlisting{examples/dongles/2/check2.lst}

\IFRU{Заставить работать программу без донглы довольно просто: просто пропатчить все места после инструкций
\CMP где происходят соответствующие сравнения}{Dongle bypassing is pretty straightforward: just patch all jumps after \CMP the relevant instructions}.

\IFRU{Еще одна возможность это написать свой драйвер для SCO OpenServer}
{Another option is to write our own SCO OpenServer driver}.


\section{\RU{Пример}\EN{Example} \#3: MS-DOS}
\label{dongle_16bit_dos}

\index{MS-DOS}
\RU{Еще одна очень старая программа для}\EN{Another very old software for} MS-DOS \RU{от}\EN{from} 1995 
\RU{также разработанная давно исчезнувшей компанией}
\EN{also developed by a company that disappeared a long time ago}.

\index{Intel!8086}
\index{Intel!80286}
\RU{Во времена перед DOS-экстендерами, всё ПО для MS-DOS рассчитывалось на процессоры 8086 или 80286,
так что в своей массе весь код был 16-битным}
\EN{In the pre-DOS extenders era, all the software for MS-DOS mostly relied on 16-bit 8086 or 80286 CPUs,
so en masse the code was 16-bit}.
\RU{16-битный код в основном такой же, какой вы уже видели в этой книге, но все регистры 16-битные,
и доступно меньше инструкций}
\EN{The 16-bit code is mostly same as you already saw in this book, but all registers
are 16-bit and there are less instructions available}.

\label{IN_example}
\label{OUT_example}
\index{x86!\Instructions!IN}
\index{x86!\Instructions!OUT}
\RU{Среда MS-DOS не могла иметь никаких драйверов, и ПО работало с \q{голым} железом через порты,
так что здесь вы можете увидеть инструкции \TT{OUT}/\TT{IN}, 
которые в наше время присутствуют в основном только
в драйверах (в современных OS нельзя обращаться на прямую к портам из \gls{user mode})}
\EN{The MS-DOS environment has no system drivers, and any program can deal with the bare hardware via ports,
so here you can see the \TT{OUT}/\TT{IN} instructions, which are present in mostly in drivers in our times
(it is impossible to access ports directly in \gls{user mode} on all modern \ac{OS}es)}.

\RU{Учитывая это, ПО для MS-DOS должно работать с донглой обращаясь к принтерному LPT-порту
напрямую}
\EN{Given that, the MS-DOS program which works with a dongle has to access the LPT printer port directly}.
\RU{Так что мы можем просто поискать эти инструкции. И да, вот они}
\EN{So we can just search for such instructions. And yes, here they are}:

\lstinputlisting{examples/dongles/3/1.lst}

(\RU{Все имена меток в этом примере даны мною}\EN{All label names in this example were given by me}).

\RU{Функция }\TT{out\_port()} \RU{вызывается только из одной функции}
\EN{is referenced only in one function}:

\lstinputlisting{examples/dongles/3/2.lst}

\RU{Это также \q{хеширующая} донгла Sentinel Pro как и в предыдущем примере}
\EN{This is again a Sentinel Pro \q{hashing} dongle as in the previous example}.
\RU{Это заметно по тому что текстовые строки передаются и здесь, 16-битные значения также возвращаются и сравниваются с другими}%
\EN{It is noticeably because text strings are passed here, too, and 16 bit values are returned and compared with others}.

\RU{Так вот как происходит работа с Sentinel Pro через порты}
\EN{So that is how Sentinel Pro is accessed via ports}.
\RU{Адрес выходного порта обычно 0x378, т.е. принтерного порта, данные для него во времена
перед USB отправлялись прямо сюда}
\EN{The output port address is usually 0x378, i.e.,
the printer port, where the data to the old printers in pre-USB era was passed to}.
\RU{Порт однонаправленный, потому что когда его разрабатывали, никто не мог предположить,
что кому-то понадобится получать информацию из принтера}
\EN{The port is uni-directional, because when it was developed, no one imagined that someone
will need to transfer information from the printer}
\footnote{\RU{Если учитывать только Centronics и не учитывать последующий стандарт IEEE 1284\EMDASH{}
в нем из принтера можно получать информацию}
\EN{If we consider Centronics only. The following IEEE 1284 standard allows the transfer of information from
the printer}.}.
\RU{Единственный способ получить информацию из принтера это регистр статуса на порту 0x379,
он содержит такие биты как \q{paper out}, \q{ack}, \q{busy}\EMDASH{}так принтер может сигнализировать
о том, что он готов или нет, и о том, есть ли в нем бумага}
\EN{The only way to get information from the printer is a status register on port 0x379, which contains
such bits as \q{paper out}, \q{ack}, \q{busy}\EMDASH{}thus the printer may signal to the host computer
if it is ready or not and if paper is present in it}.
\RU{Так что донгла возвращает информацию через какой-то из этих бит, по одному биту на каждой
итерации}
\EN{So the dongle returns information from one of these bits, one bit at each iteration}.

\TT{\_in\_port\_2} \RU{содержит адрес статуса}\EN{contains the address of the status word} (0x379) \AndENRU 
\TT{\_in\_port\_1} \RU{содержит адрес управляющего регистра}\EN{contains the control register address} (0x37A).

\RU{Судя по всему, донгла возвращает информацию только через флаг \q{busy} на}
\EN{It seems that the dongle returns information via the \q{busy} flag at} \TT{seg030:00B9}: 
\RU{каждый бит записывается в регистре \TT{DI} позже возвращаемый в самом конце функции}
\EN{each bit is stored in the \TT{DI} register, which is returned at the end of the function}.

\RU{Что означают все эти отсылаемые в выходной порт байты}%
\EN{What do all these bytes sent to output port mean}?
\RU{Трудно сказать. Возможно, команды донглы.}\EN{Hard to say. Probably commands to the dongle.}
\RU{Но честно говоря, нам и не обязательно знать: нашу задачу можно легко решить и без этих знаний}
\EN{But generally speaking, it is not necessary to know: it is easy to solve our task without that knowledge}.

\RU{Вот функция проверки донглы}\EN{Here is the dongle checking routine}:

\lstinputlisting{examples/dongles/3/3.lst}

\RU{А так как эта функция может вызываться слишком часто, например, 
перед выполнением каждой важной возможности ПО,
а обращение к донгле вообще-то медленное (и из-за медленного принтерного порта, и из-за медленного
\ac{MCU} в донгле), так что они, наверное, добавили возможность пропускать проверку донглы слишком часто,
используя текущее время в функции \TT{biostime()}}
\EN{Since the routine can be called very frequently, e.g., before the execution of each important software feature, 
and accessing the dongle is generally slow (because of the slow printer port and also slow
\ac{MCU} in the dongle), they probably added a way to skip some dongle checks,
by checking the current time in the \TT{biostime()} function}.

\index{\CStandardLibrary!rand()}
\EN{The}\RU{Функция} \TT{get\_rand()} \RU{использует стандартную функцию Си}
\EN{function uses the standard C function}:

\lstinputlisting{examples/dongles/3/4.lst}

\RU{Так что текстовая строка выбирается случайно, отправляется в донглу и результат
хеширования сверяется с корректным значением}
\EN{So the text string is selected randomly, passed into the dongle, and then the result of the hashing 
is compared with the correct value}.

\RU{Текстовые строки, похоже, составлялись так же случайно, во время разработки ПО.}%
\EN{The text strings seem to be constructed randomly as well, during software development.}

\RU{И вот как вызывается главная процедура проверки донглы}
\EN{And this is how the main dongle checking function is called}:

\lstinputlisting{examples/dongles/3/5.lst}

\RU{Заставить работать программу без донглы очень просто: просто заставить функцию
\TT{check\_dongle()} возвращать всегда 0}
\EN{Bypassing the dongle is easy, just force the \TT{check\_dongle()} function to always return 0}.

\RU{Например, вставив такой код в самом её начале}\EN{For example, by inserting this code at its beginning}:

\begin{lstlisting}
mov ax,0
retf
\end{lstlisting}

\index{\CStandardLibrary!strcpy()}
\RU{Наблюдательный читатель может заметить, что функция Си \TT{strcpy()} имеет 2 аргумента, но здесь
мы видим, что передается 4}
\EN{The observant reader might recall that the \TT{strcpy()} C function usually requires two pointers in its arguments,
but we see that 4 values are passed}:

\begin{lstlisting}
seg033:088F 1E                          push    ds
seg033:0890 68 22 44                    push    offset aTrupcRequiresA ; "This Software Requires a Software Lock\n"
seg033:0893 1E                          push    ds
seg033:0894 68 60 E9                    push    offset byte_6C7E0 ; dest
seg033:0897 9A 79 65 00+                call    _strcpy
seg033:089C 83 C4 08                    add     sp, 8
\end{lstlisting}

\RU{Это связано с моделью памяти в MS-DOS. Об этом больше читайте здесь}
\EN{This is related to MS-DOS' memory model. You can read more about it here}: 
\myref{8086_memory_model}.

\RU{Так что, \TT{strcpy()}, как и любая другая функция принимающая указатель (-и) в аргументах,
работает с 16-битными парами}
\EN{So as you may see, \TT{strcpy()} and any other function that take pointer(s) in arguments
work with 16-bit pairs}.

\RU{Вернемся к нашему примеру}\EN{Let's get back to our example}.
\TT{DS} \RU{сейчас указывает на сегмент данных размещенный в исполняемом файле, там, где хранится текстовая
строка.}
\EN{is currently set to the data segment located in the executable,
that is where the text string is stored.}

\index{x86!\Instructions!LES}
\RU{В функции \TT{sent\_pro()} каждый байт строки загружается на \TT{seg030:00EF}: инструкция
\TT{LES} загружает из переданного аргумента пару ES:BX одновременно}
\EN{In the \TT{sent\_pro()} function, each byte of the string is loaded at \TT{seg030:00EF}: the \TT{LES} instruction
loads the ES:BX pair simultaneously from the passed argument}.
\RU{\MOV на \TT{seg030:00F5} загружает байт из памяти, на который указывает пара ES:BX.}
\EN{The \MOV at \TT{seg030:00F5} loads the byte from the memory at which the ES:BX pair points.}

% TODO rewrite
%\RU{На \TT{seg030:00F2} \glslink{increment}{инкрементируется} только вторая 16-битная пара адреса.}
%\EN{At \TT{seg030:00F2} only a second 16-bit part of address is \glslink{increment}{incremented}.}
%\RU{Это значит, что переданная в функцию строка не может находиться на границе двух сегментов.}
%\EN{This implies that the string passed to the function cannot be located on the boundary between two data segments.}



\chapter{\RU{\q{QR9}: Любительская криптосистема, вдохновленная кубиком Рубика}
\EN{\q{QR9}: Rubik's cube inspired amateur crypto-algorithm}}

\RU{Любительские криптосистемы иногда встречаются довольно странные.}
\EN{Sometimes amateur cryptosystems appear to be pretty bizarre.}

\RU{Однажды автора сих строк попросили разобраться с одним таким любительским криптоалгоритмом встроенным в 
утилиту для шифрования, исходный код которой был утерян\footnote{Он также получил разрешение от 
клиента на публикацию деталей алгоритма}.}
\EN{The author of this book was once asked to reverse engineer an amateur cryptoalgorithm of some data encryption utility, 
the source code for which was lost\footnote{He also got permission from the customer to publish the algorithm's details}.}

\RU{Вот листинг этой утилиты для шифрования, полученный при помощи \IDA}%
\EN{Here is the listing exported from \IDA for the original encryption utility}:

\lstinputlisting{examples/qr9/qr9_original.lst}

\RU{Все имена функций и меток даны мною в процессе анализа.}
\EN{All function and label names were given by me during the analysis.}

\RU{Начнем с самого верха. Вот функция, берущая на вход два имени файла и пароль.}
\EN{Let's start from the top. Here is a function that takes two file names and password.}

\begin{lstlisting}
.text:00541320 ; int __cdecl crypt_file(int Str, char *Filename, int password)
.text:00541320 crypt_file      proc near
.text:00541320
.text:00541320 Str             = dword ptr  4
.text:00541320 Filename        = dword ptr  8
.text:00541320 password        = dword ptr  0Ch
.text:00541320
\end{lstlisting}

\RU{Открыть файл и сообщить об ошибке в случае ошибки:}\EN{Open the file and report if an error occurs:}

\begin{lstlisting}
.text:00541320                 mov     eax, [esp+Str]
.text:00541324                 push    ebp
.text:00541325                 push    offset Mode     ; "rb"
.text:0054132A                 push    eax             ; Filename
.text:0054132B                 call    _fopen          ; open file
.text:00541330                 mov     ebp, eax
.text:00541332                 add     esp, 8
.text:00541335                 test    ebp, ebp
.text:00541337                 jnz     short loc_541348
.text:00541339                 push    offset Format   ; "Cannot open input file!\n"
.text:0054133E                 call    _printf
.text:00541343                 add     esp, 4
.text:00541346                 pop     ebp
.text:00541347                 retn
.text:00541348
.text:00541348 loc_541348:
\end{lstlisting}

\index{\CStandardLibrary!fseek()}
\index{\CStandardLibrary!ftell()}
\RU{Узнать размер файла используя}\EN{Get the file size via} \TT{fseek()}/\TT{ftell()}:

\lstinputlisting{examples/qr9/1.\LANG}

\RU{Этот фрагмент кода вычисляет длину файла, выровненную по 64-байтной границе.
Это потому что этот алгоритм шифрования работает только с блоками размерами 64 байта.
Работает очень просто: разделить длину файла на 64, забыть об остатке, прибавить 1,
умножить на 64.
Следующий код удаляет остаток от деления, как если бы это значение уже было разделено 
на 64 и добавляет 64. Это почти то же самое.}
\EN{This fragment of code calculates the file size aligned on a 64-byte boundary. 
This is because this cryptographic algorithm works with only 64-byte blocks. 
The operation is pretty straightforward: divide the file size by 64, forget about the remainder and add 1, 
then multiply by 64. 
The following code removes the remainder as if the value was already divided by 64 and adds 64. 
It is almost the same.}

\lstinputlisting{examples/qr9/2.\LANG}

\RU{Выделить буфер с выровненным размером:}\EN{Allocate buffer with aligned size:}

\begin{lstlisting}
.text:00541373                 push    esi             ; Size
.text:00541374                 call    _malloc
\end{lstlisting}

\index{\CStandardLibrary!calloc()}
\RU{Вызвать memset(), т.е. очистить выделенный буфер\footnote{malloc() + memset() можно было бы 
заменить на calloc()}.}\EN{Call memset(), e.g., clear the allocated buffer\footnote{malloc() + memset() could 
be replaced by calloc()}.}

\lstinputlisting{examples/qr9/3.\LANG}

\RU{Чтение файла используя стандартную функцию Си}\EN{Read file via the standard C function} \TT{fread()}.

\begin{lstlisting}
.text:00541392                 mov     eax, [esp+38h+Str]
.text:00541396                 push    eax             ; ElementSize
.text:00541397                 push    ebx             ; DstBuf
.text:00541398                 call    _fread          ; read file
.text:0054139D                 push    ebp             ; File
.text:0054139E                 call    _fclose
\end{lstlisting}

\RU{Вызов \TT{crypt()}. Эта функция берет на вход буфер, длину буфера (выровненную) и строку пароля.}
\EN{Call \TT{crypt()}. This function takes a buffer, buffer size (aligned) and a password string.}

\begin{lstlisting}
.text:005413A3                 mov     ecx, [esp+44h+password]
.text:005413A7                 push    ecx             ; password
.text:005413A8                 push    esi             ; aligned size
.text:005413A9                 push    ebx             ; buffer
.text:005413AA                 call    crypt           ; do crypt
\end{lstlisting}

\RU{Создать выходной файл. Кстати, разработчик забыл вставить проверку, создался ли файл успешно!
Результат открытия файла, впрочем, проверяется.}
\EN{Create the output file. By the way, the developer forgot to check if it is was created correctly! 
The file opening result is being checked, though.}

\begin{lstlisting}
.text:005413AF                 mov     edx, [esp+50h+Filename]
.text:005413B3                 add     esp, 40h
.text:005413B6                 push    offset aWb      ; "wb"
.text:005413BB                 push    edx             ; Filename
.text:005413BC                 call    _fopen
.text:005413C1                 mov     edi, eax
\end{lstlisting}

\RU{Теперь хэндл созданного файла в регистре \EDI. Записываем сигнатуру \q{QR9}.}
\EN{The newly created file handle is in the \EDI register now. Write signature \q{QR9}.}

\begin{lstlisting}
.text:005413C3                 push    edi             ; File
.text:005413C4                 push    1               ; Count
.text:005413C6                 push    3               ; Size
.text:005413C8                 push    offset aQr9     ; "QR9"
.text:005413CD                 call    _fwrite         ; write file signature
\end{lstlisting}

\RU{Записываем настоящую длину файла (не выровненную)}\EN{Write the actual file size (not aligned)}:

\begin{lstlisting}
.text:005413D2                 push    edi             ; File
.text:005413D3                 push    1               ; Count
.text:005413D5                 lea     eax, [esp+30h+Str]
.text:005413D9                 push    4               ; Size
.text:005413DB                 push    eax             ; Str
.text:005413DC                 call    _fwrite         ; write original file size
\end{lstlisting}

\RU{Записываем шифрованный буфер}\EN{Write the encrypted buffer}:

\begin{lstlisting}
.text:005413E1                 push    edi             ; File
.text:005413E2                 push    1               ; Count
.text:005413E4                 push    esi             ; Size
.text:005413E5                 push    ebx             ; Str
.text:005413E6                 call    _fwrite         ; write encrypted file
\end{lstlisting}

\RU{Закрыть файл и освободить выделенный буфер}\EN{Close the file and free the allocated buffer}:

\begin{lstlisting}
.text:005413EB                 push    edi             ; File
.text:005413EC                 call    _fclose
.text:005413F1                 push    ebx             ; Memory
.text:005413F2                 call    _free
.text:005413F7                 add     esp, 40h
.text:005413FA                 pop     edi
.text:005413FB                 pop     esi
.text:005413FC                 pop     ebx
.text:005413FD                 pop     ebp
.text:005413FE                 retn
.text:005413FE crypt_file      endp
\end{lstlisting}

\RU{Переписанный на Си код}\EN{Here is the reconstructed C code}:

\begin{lstlisting}
void crypt_file(char *fin, char* fout, char *pw)
{
	FILE *f;
	int flen, flen_aligned;
	BYTE *buf;

	f=fopen(fin, "rb");
	
	if (f==NULL)
	{
		printf ("Cannot open input file!\n");
		return;
	};

	fseek (f, 0, SEEK_END);
	flen=ftell (f);
	fseek (f, 0, SEEK_SET);

	flen_aligned=(flen&0xFFFFFFC0)+0x40;

	buf=(BYTE*)malloc (flen_aligned);
	memset (buf, 0, flen_aligned);

	fread (buf, flen, 1, f);

	fclose (f);

	crypt (buf, flen_aligned, pw);
	
	f=fopen(fout, "wb");

	fwrite ("QR9", 3, 1, f);
	fwrite (&flen, 4, 1, f);
	fwrite (buf, flen_aligned, 1, f);

	fclose (f);

	free (buf);
};
\end{lstlisting}

\RU{Процедура дешифрования почти такая же}\EN{The decryption procedure is almost the same}:

\begin{lstlisting}
.text:00541400 ; int __cdecl decrypt_file(char *Filename, int, void *Src)
.text:00541400 decrypt_file    proc near
.text:00541400
.text:00541400 Filename        = dword ptr  4
.text:00541400 arg_4           = dword ptr  8
.text:00541400 Src             = dword ptr  0Ch
.text:00541400
.text:00541400                 mov     eax, [esp+Filename]
.text:00541404                 push    ebx
.text:00541405                 push    ebp
.text:00541406                 push    esi
.text:00541407                 push    edi
.text:00541408                 push    offset aRb      ; "rb"
.text:0054140D                 push    eax             ; Filename
.text:0054140E                 call    _fopen
.text:00541413                 mov     esi, eax
.text:00541415                 add     esp, 8
.text:00541418                 test    esi, esi
.text:0054141A                 jnz     short loc_54142E
.text:0054141C                 push    offset aCannotOpenIn_0 ; "Cannot open input file!\n"
.text:00541421                 call    _printf
.text:00541426                 add     esp, 4
.text:00541429                 pop     edi
.text:0054142A                 pop     esi
.text:0054142B                 pop     ebp
.text:0054142C                 pop     ebx
.text:0054142D                 retn
.text:0054142E
.text:0054142E loc_54142E:
.text:0054142E                 push    2               ; Origin
.text:00541430                 push    0               ; Offset
.text:00541432                 push    esi             ; File
.text:00541433                 call    _fseek
.text:00541438                 push    esi             ; File
.text:00541439                 call    _ftell
.text:0054143E                 push    0               ; Origin
.text:00541440                 push    0               ; Offset
.text:00541442                 push    esi             ; File
.text:00541443                 mov     ebp, eax
.text:00541445                 call    _fseek
.text:0054144A                 push    ebp             ; Size
.text:0054144B                 call    _malloc
.text:00541450                 push    esi             ; File
.text:00541451                 mov     ebx, eax
.text:00541453                 push    1               ; Count
.text:00541455                 push    ebp             ; ElementSize
.text:00541456                 push    ebx             ; DstBuf
.text:00541457                 call    _fread
.text:0054145C                 push    esi             ; File
.text:0054145D                 call    _fclose
\end{lstlisting}

\RU{Проверяем сигнатуру (первые 3 байта)}\EN{Check signature (first 3 bytes)}:

\begin{lstlisting}
.text:00541462                 add     esp, 34h
.text:00541465                 mov     ecx, 3
.text:0054146A                 mov     edi, offset aQr9_0 ; "QR9"
.text:0054146F                 mov     esi, ebx
.text:00541471                 xor     edx, edx
.text:00541473                 repe cmpsb
.text:00541475                 jz      short loc_541489
\end{lstlisting}

\RU{Сообщить об ошибке если сигнатура отсутствует}\EN{Report an error if the signature is absent}:

\begin{lstlisting}
.text:00541477                 push    offset aFileIsNotCrypt ; "File is not encrypted!\n"
.text:0054147C                 call    _printf
.text:00541481                 add     esp, 4
.text:00541484                 pop     edi
.text:00541485                 pop     esi
.text:00541486                 pop     ebp
.text:00541487                 pop     ebx
.text:00541488                 retn
.text:00541489
.text:00541489 loc_541489:
\end{lstlisting}

\RU{Вызвать}\EN{Call} \TT{decrypt()}.

\begin{lstlisting}
.text:00541489                 mov     eax, [esp+10h+Src]
.text:0054148D                 mov     edi, [ebx+3]
.text:00541490                 add     ebp, 0FFFFFFF9h
.text:00541493                 lea     esi, [ebx+7]
.text:00541496                 push    eax             ; Src
.text:00541497                 push    ebp             ; int
.text:00541498                 push    esi             ; int
.text:00541499                 call    decrypt
.text:0054149E                 mov     ecx, [esp+1Ch+arg_4]
.text:005414A2                 push    offset aWb_0    ; "wb"
.text:005414A7                 push    ecx             ; Filename
.text:005414A8                 call    _fopen
.text:005414AD                 mov     ebp, eax
.text:005414AF                 push    ebp             ; File
.text:005414B0                 push    1               ; Count
.text:005414B2                 push    edi             ; Size
.text:005414B3                 push    esi             ; Str
.text:005414B4                 call    _fwrite
.text:005414B9                 push    ebp             ; File
.text:005414BA                 call    _fclose
.text:005414BF                 push    ebx             ; Memory
.text:005414C0                 call    _free
.text:005414C5                 add     esp, 2Ch
.text:005414C8                 pop     edi
.text:005414C9                 pop     esi
.text:005414CA                 pop     ebp
.text:005414CB                 pop     ebx
.text:005414CC                 retn
.text:005414CC decrypt_file    endp
\end{lstlisting}

\RU{Переписанный на Си код}\EN{Here is the reconstructed C code}:

\begin{lstlisting}
void decrypt_file(char *fin, char* fout, char *pw)
{
	FILE *f;
	int real_flen, flen;
	BYTE *buf;

	f=fopen(fin, "rb");
	
	if (f==NULL)
	{
		printf ("Cannot open input file!\n");
		return;
	};

	fseek (f, 0, SEEK_END);
	flen=ftell (f);
	fseek (f, 0, SEEK_SET);

	buf=(BYTE*)malloc (flen);

	fread (buf, flen, 1, f);

	fclose (f);

	if (memcmp (buf, "QR9", 3)!=0)
	{
		printf ("File is not encrypted!\n");
		return;
	};

	memcpy (&real_flen, buf+3, 4);

	decrypt (buf+(3+4), flen-(3+4), pw);
	
	f=fopen(fout, "wb");

	fwrite (buf+(3+4), real_flen, 1, f);

	fclose (f);

	free (buf);
};
\end{lstlisting}

\RU{OK, посмотрим глубже}\EN{OK, now let's go deeper}.

\RU{Функция}\EN{Function} \TT{crypt()}:

\begin{lstlisting}
.text:00541260 crypt           proc near
.text:00541260
.text:00541260 arg_0           = dword ptr  4
.text:00541260 arg_4           = dword ptr  8
.text:00541260 arg_8           = dword ptr  0Ch
.text:00541260
.text:00541260                 push    ebx
.text:00541261                 mov     ebx, [esp+4+arg_0]
.text:00541265                 push    ebp
.text:00541266                 push    esi
.text:00541267                 push    edi
.text:00541268                 xor     ebp, ebp
.text:0054126A
.text:0054126A loc_54126A:
\end{lstlisting}

\index{x86!\Instructions!MOVSD}
\RU{Этот фрагмент кода копирует часть входного буфера во внутренний буфер, который мы позже назовем \q{cube64}.}%
\EN{This fragment of code copies a part of the input buffer to an internal array we later name \q{cube64}.}
\RU{Длина в регистре \ECX. \TT{MOVSD} означает \IT{скопировать 32-битное слово}, так что, 16 32-битных слов
это как раз 64 байта.}\EN{The size is in the \ECX register. \TT{MOVSD} stands for \IT{move 32-bit dword}, so, 
16 32-bit dwords are exactly 64 bytes.}

\begin{lstlisting}
.text:0054126A                 mov     eax, [esp+10h+arg_8]
.text:0054126E                 mov     ecx, 10h
.text:00541273                 mov     esi, ebx   ; EBX is pointer within input buffer
.text:00541275                 mov     edi, offset cube64
.text:0054127A                 push    1
.text:0054127C                 push    eax
.text:0054127D                 rep movsd
\end{lstlisting}

\RU{Вызвать}\EN{Call} \TT{rotate\_all\_with\_password()}:

\begin{lstlisting}
.text:0054127F                 call    rotate_all_with_password
\end{lstlisting}

\RU{Скопировать зашифрованное содержимое из \q{cube64} назад в буфер}
\EN{Copy encrypted contents back from \q{cube64} to buffer}:

\begin{lstlisting}
.text:00541284                 mov     eax, [esp+18h+arg_4]
.text:00541288                 mov     edi, ebx
.text:0054128A                 add     ebp, 40h
.text:0054128D                 add     esp, 8
.text:00541290                 mov     ecx, 10h
.text:00541295                 mov     esi, offset cube64
.text:0054129A                 add     ebx, 40h  ; add 64 to input buffer pointer
.text:0054129D                 cmp     ebp, eax  ; EBP contain amount of encrypted data.
.text:0054129F                 rep movsd
\end{lstlisting}

\RU{Если \EBP не больше чем длина во входном аргументе, тогда переходим к следующему блоку.}%
\EN{If \EBP is not bigger that the size input argument, then continue to the next block.}

\begin{lstlisting}
.text:005412A1                 jl      short loc_54126A
.text:005412A3                 pop     edi
.text:005412A4                 pop     esi
.text:005412A5                 pop     ebp
.text:005412A6                 pop     ebx
.text:005412A7                 retn
.text:005412A7 crypt           endp
\end{lstlisting}

\RU{Реконструированная функция \TT{crypt()}}\EN{Reconstructed \TT{crypt()} function}:

\begin{lstlisting}
void crypt (BYTE *buf, int sz, char *pw)
{
	int i=0;
	
	do
	{
		memcpy (cube, buf+i, 8*8);
		rotate_all (pw, 1);
		memcpy (buf+i, cube, 8*8);
		i+=64;
	}
	while (i<sz);
};
\end{lstlisting}

\RU{OK, углубимся в функцию \TT{rotate\_all\_with\_password()}. Она берет на вход два аргумента: 
строку пароля и число.}\EN{OK, now let's go deeper in function \TT{rotate\_all\_with\_password()}. 
It takes two arguments: password string and a number.}
\RU{В функции \TT{crypt()}, число 1 используется и в \TT{decrypt()} (где \TT{rotate\_all\_with\_password()}
функция вызывается также), число 3.}
\EN{In \TT{crypt()}, the number 1 is used, and in the \TT{decrypt()} function (where \TT{rotate\_all\_with\_password()} function 
is called too), the number is 3.}

\begin{lstlisting}
.text:005411B0 rotate_all_with_password proc near
.text:005411B0
.text:005411B0 arg_0           = dword ptr  4
.text:005411B0 arg_4           = dword ptr  8
.text:005411B0
.text:005411B0                 mov     eax, [esp+arg_0]
.text:005411B4                 push    ebp
.text:005411B5                 mov     ebp, eax
\end{lstlisting}

\RU{Проверяем символы в пароле. Если это ноль, выходим:}\EN{Check the current character in the password. If it is zero, exit:}

\begin{lstlisting}
.text:005411B7                 cmp     byte ptr [eax], 0
.text:005411BA                 jz      exit
.text:005411C0                 push    ebx
.text:005411C1                 mov     ebx, [esp+8+arg_4]
.text:005411C5                 push    esi
.text:005411C6                 push    edi
.text:005411C7
.text:005411C7 loop_begin:
\end{lstlisting}

\index{\CStandardLibrary!tolower()}
\RU{Вызываем \TT{tolower()}, стандартную функцию Си.}\EN{Call \TT{tolower()}, a standard C function.}

\begin{lstlisting}
.text:005411C7                 movsx   eax, byte ptr [ebp+0]
.text:005411CB                 push    eax             ; C
.text:005411CC                 call    _tolower
.text:005411D1                 add     esp, 4
\end{lstlisting}

\RU{Хмм, если пароль содержит символ не из латинского алфавита, он пропускается!
Действительно, если мы запускаем утилиту для шифрования используя символы не латинского алфавита, 
похоже, они просто игнорируются.}
\EN{Hmm, if the password contains non-Latin character, it is skipped! 
Indeed, when we run the encryption utility and try non-Latin characters in the password, 
they seem to be ignored.}

\begin{lstlisting}
.text:005411D4                 cmp     al, 'a'
.text:005411D6                 jl      short next_character_in_password
.text:005411D8                 cmp     al, 'z'
.text:005411DA                 jg      short next_character_in_password
.text:005411DC                 movsx   ecx, al
\end{lstlisting}

\RU{Отнимем значение \q{a} (97) от символа.}\EN{Subtract the value of \q{a} (97) from the character.}

\begin{lstlisting}
.text:005411DF                 sub     ecx, 'a'  ; 97
\end{lstlisting}

\RU{После вычитания, тут будет 0 для \q{a}, 1 для \q{b}, и так далее. И 25 для \q{z}.}
\EN{After subtracting, we'll get 0 for \q{a} here, 1 for \q{b}, etc. And 25 for \q{z}.}

\begin{lstlisting}
.text:005411E2                 cmp     ecx, 24
.text:005411E5                 jle     short skip_subtracting
.text:005411E7                 sub     ecx, 24
\end{lstlisting}

\RU{Похоже, символы \q{y} и \q{z} также исключительные.
После этого фрагмента кода, \q{y} становится 0, а \q{z} ~--- 1.
Это значит, что 26 латинских букв становятся значениями в интервале 0..23, (всего 24).}
\EN{It seems, \q{y} and \q{z} are exceptional characters too. 
After that fragment of code, \q{y} becomes 0 and \q{z}~---1. 
This implies that the 26 Latin alphabet symbols become values in the range of 0..23, (24 in total).}

\begin{lstlisting}
.text:005411EA
.text:005411EA skip_subtracting:                       ; CODE XREF: rotate_all_with_password+35
\end{lstlisting}

\RU{Это, на самом деле, деление через умножение.
Читайте об этом больше в секции \q{\DivisionByNineSectionName}~(\myref{sec:divisionbynine}).}
\EN{This is actually division via multiplication. 
You can read more about it in the \q{\DivisionByNineSectionName} section~(\myref{sec:divisionbynine}).}

\RU{Это код, на самом деле, делит значение символа пароля на 3.}
\EN{The code actually divides the password character's value by 3.}
% TODO1: add Mathematica calculations
\begin{lstlisting}
.text:005411EA                 mov     eax, 55555556h
.text:005411EF                 imul    ecx
.text:005411F1                 mov     eax, edx
.text:005411F3                 shr     eax, 1Fh
.text:005411F6                 add     edx, eax
.text:005411F8                 mov     eax, ecx
.text:005411FA                 mov     esi, edx
.text:005411FC                 mov     ecx, 3
.text:00541201                 cdq
.text:00541202                 idiv    ecx
\end{lstlisting}

\RU{\EDX\EMDASH{}остаток от деления.}\EN{\EDX is the remainder of the division.}

\lstinputlisting{examples/qr9/4.\LANG}

\RU{Если остаток 2, вызываем \TT{rotate3()}. 
\EDX это второй аргумент функции \TT{rotate\_all\_with\_password()}. 
Как мы уже заметили, 1 это для шифрования, 3 для дешифрования.
Так что здесь цикл, функции rotate1/2/3 будут вызываться столько же раз, сколько значение переменной
в первом аргументе.}
\EN{If the remainder is 2, call \TT{rotate3()}. 
\EDI is the second argument of the \TT{rotate\_all\_with\_password()} function.
As we already noted, 1 is for the encryption operations and 3 is for the decryption. 
So, here is a loop. When encrypting, rotate1/2/3 are to be called the same number of times as 
given in the first argument.}

\begin{lstlisting}
.text:00541215 call_rotate3:
.text:00541215                 push    esi
.text:00541216                 call    rotate3
.text:0054121B                 add     esp, 4
.text:0054121E                 dec     edi
.text:0054121F                 jnz     short call_rotate3
.text:00541221                 jmp     short next_character_in_password
.text:00541223
.text:00541223 call_rotate2:
.text:00541223                 test    ebx, ebx
.text:00541225                 jle     short next_character_in_password
.text:00541227                 mov     edi, ebx
.text:00541229
.text:00541229 loc_541229:
.text:00541229                 push    esi
.text:0054122A                 call    rotate2
.text:0054122F                 add     esp, 4
.text:00541232                 dec     edi
.text:00541233                 jnz     short loc_541229
.text:00541235                 jmp     short next_character_in_password
.text:00541237
.text:00541237 call_rotate1:
.text:00541237                 test    ebx, ebx
.text:00541239                 jle     short next_character_in_password
.text:0054123B                 mov     edi, ebx
.text:0054123D
.text:0054123D loc_54123D:
.text:0054123D                 push    esi
.text:0054123E                 call    rotate1
.text:00541243                 add     esp, 4
.text:00541246                 dec     edi
.text:00541247                 jnz     short loc_54123D
.text:00541249
\end{lstlisting}

\RU{Достать следующий символ из строки пароля.}\EN{Fetch the next character from the password string.}

\begin{lstlisting}
.text:00541249 next_character_in_password:
.text:00541249                 mov     al, [ebp+1]
\end{lstlisting}

\RU{\glslink{increment}{Инкремент} указателя на символ в строке пароля:}\EN{\Gls{increment} the character pointer in the password string:}

\begin{lstlisting}
.text:0054124C                 inc     ebp
.text:0054124D                 test    al, al
.text:0054124F                 jnz     loop_begin
.text:00541255                 pop     edi
.text:00541256                 pop     esi
.text:00541257                 pop     ebx
.text:00541258
.text:00541258 exit:
.text:00541258                 pop     ebp
.text:00541259                 retn
.text:00541259 rotate_all_with_password endp
\end{lstlisting}

\RU{Реконструированный код на Си:}\EN{Here is the reconstructed C code:}

\begin{lstlisting}
void rotate_all (char *pwd, int v)
{
	char *p=pwd;

	while (*p)
	{
		char c=*p;
		int q;

		c=tolower (c);

		if (c>='a' && c<='z')
		{
			q=c-'a';
			if (q>24)
				q-=24;

			int quotient=q/3;
			int remainder=q % 3;

			switch (remainder)
			{
			case 0: for (int i=0; i<v; i++) rotate1 (quotient); break;
			case 1: for (int i=0; i<v; i++) rotate2 (quotient); break;
			case 2: for (int i=0; i<v; i++) rotate3 (quotient); break;
			};
		};

		p++;
	};
};
\end{lstlisting}

\RU{Углубимся еще дальше и исследуем функции rotate1/2/3.
Каждая функция вызывает еще две.
В итоге мы назовем их \TT{set\_bit()} и \TT{get\_bit()}.}%
\EN{Now let's go deeper and investigate the rotate1/2/3 functions. 
Each function calls another two functions. 
We eventually will name them \TT{set\_bit()} and \TT{get\_bit()}.}

\RU{Начнем с \TT{get\_bit()}:}\EN{Let's start with \TT{get\_bit()}:}

\begin{lstlisting}
.text:00541050 get_bit         proc near
.text:00541050
.text:00541050 arg_0           = dword ptr  4
.text:00541050 arg_4           = dword ptr  8
.text:00541050 arg_8           = byte ptr  0Ch
.text:00541050
.text:00541050                 mov     eax, [esp+arg_4]
.text:00541054                 mov     ecx, [esp+arg_0]
.text:00541058                 mov     al, cube64[eax+ecx*8]
.text:0054105F                 mov     cl, [esp+arg_8]
.text:00541063                 shr     al, cl
.text:00541065                 and     al, 1
.text:00541067                 retn
.text:00541067 get_bit         endp
\end{lstlisting}

\RU{\dots иными словами: подсчитать индекс в массиве cube64}\EN{\dots in other words: calculate an index in 
the cube64 array}: \IT{arg\_4 + arg\_0 * 8}.
\RU{Затем сдвинуть байт из массива вправо на количество бит заданных в arg\_8. 
Изолировать самый младший бит и вернуть его}\EN{Then shift a byte from the array by arg\_8 bits right. 
Isolate the lowest bit and return it.}

\RU{Посмотрим другую функцию}\EN{Let's see another function}, \TT{set\_bit()}:

\begin{lstlisting}
.text:00541000 set_bit         proc near
.text:00541000
.text:00541000 arg_0           = dword ptr  4
.text:00541000 arg_4           = dword ptr  8
.text:00541000 arg_8           = dword ptr  0Ch
.text:00541000 arg_C           = byte ptr  10h
.text:00541000
.text:00541000                 mov     al, [esp+arg_C]
.text:00541004                 mov     ecx, [esp+arg_8]
.text:00541008                 push    esi
.text:00541009                 mov     esi, [esp+4+arg_0]
.text:0054100D                 test    al, al
.text:0054100F                 mov     eax, [esp+4+arg_4]
.text:00541013                 mov     dl, 1
.text:00541015                 jz      short loc_54102B
\end{lstlisting}

\RU{\TT{DL} тут равно 1. Сдвигаем эту единицу на количество, указанное в arg\_8. Например, если в arg\_8 число 4,
тогда значение в \TT{DL} станет 0x10 или 1000b в двоичной системе счисления.}
\EN{The value in the \TT{DL} is 1 here. It gets shifted left by arg\_8.
For example, if arg\_8 is 4, the value in the \TT{DL} register is to be 
0x10 or 1000b in binary form.}

\begin{lstlisting}
.text:00541017                 shl     dl, cl
.text:00541019                 mov     cl, cube64[eax+esi*8]
\end{lstlisting}

\RU{Вытащить бит из массива и явно выставить его.}\EN{Get a bit from array and explicitly set it.} % TODO1: rewrite

\begin{lstlisting}
.text:00541020                 or      cl, dl
\end{lstlisting}

\RU{Сохранить его назад:}\EN{Store it back:} % TODO1: rewrite

\begin{lstlisting}
.text:00541022                 mov     cube64[eax+esi*8], cl
.text:00541029                 pop     esi
.text:0054102A                 retn
.text:0054102B
.text:0054102B loc_54102B:
.text:0054102B                 shl     dl, cl
\end{lstlisting}

\RU{Если arg\_C не ноль\dots}\EN{If arg\_C is not zero\dots}

\begin{lstlisting}
.text:0054102D                 mov     cl, cube64[eax+esi*8]
\end{lstlisting}

\index{x86!\Instructions!NOT}
\RU{\dots инвертировать DL. Например, если состояние DL после сдвига 0x10 или 1000b в двоичной системе,
здесь будет 0xEF после инструкции \NOT или 11101111b в двоичной системе.}
\EN{\dots invert DL. For example, if DL's state after the shift was 0x10 or 1000b in binary form, 
there is 0xEF to be after the \NOT instruction (or 11101111b in binary form).}

\begin{lstlisting}
.text:00541034                 not     dl
\end{lstlisting}

\RU{Эта инструкция сбрасывает бит, иными словами, она сохраняет все биты в \TT{CL} которые также
выставлены в \TT{DL} кроме тех в \TT{DL}, что были сброшены. Это значит, что если в \TT{DL}, например,
11101111b в двоичной системе, все биты будут сохранены кроме пятого (считая с младшего бита).}
\EN{This instruction clears the bit, in other words, it saves all bits in \TT{CL} which are also set in 
\TT{DL} except those in \TT{DL} which are cleared.
This implies that if \TT{DL} is 11101111b in binary form,
all bits are to be saved except the 5th (counting from lowest bit).}

\begin{lstlisting}
.text:00541036                 and     cl, dl
\end{lstlisting}

\RU{Сохранить его назад}\EN{Store it back:}

\begin{lstlisting}
.text:00541038                 mov     cube64[eax+esi*8], cl
.text:0054103F                 pop     esi
.text:00541040                 retn
.text:00541040 set_bit         endp
\end{lstlisting}

\RU{Это почти то же самое что и \TT{get\_bit()}, кроме того, что если arg\_C ноль, тогда функция сбрасывает
указанный бит в массиве, либо же, в противном случае, выставляет его в 1.}
\EN{It is almost the same as \TT{get\_bit()}, except, if arg\_C is zero, the function clears the specific bit in the array, 
or sets it otherwise.}

\RU{Мы также знаем что размер массива 64. Первые два аргумента и у \TT{set\_bit()} и у \TT{get\_bit()}
могут быть представлены как двумерные координаты. Таким образом, массив ~--- это матрица 8*8.}
\EN{We also know that the array's size is 64. The first two arguments both in the \TT{set\_bit()} and \TT{get\_bit()} functions
could be seen as 2D coordinates. Then the array is to be an 8*8 matrix.}

\RU{Представление на Си всего того, что мы уже знаем:}\EN{Here is a C representation of what we know up to now:}

\begin{lstlisting}
#define IS_SET(flag, bit)       ((flag) & (bit))
#define SET_BIT(var, bit)       ((var) |= (bit))
#define REMOVE_BIT(var, bit)    ((var) &= ~(bit))

static BYTE cube[8][8];

void set_bit (int x, int y, int shift, int bit)
{
	if (bit)
		SET_BIT (cube[x][y], 1<<shift);
	else
		REMOVE_BIT (cube[x][y], 1<<shift);
};

bool get_bit (int x, int y, int shift)
{
	if ((cube[x][y]>>shift)&1==1)
		return 1;
	return 0;
};
\end{lstlisting}

\RU{Теперь вернемся к функциям rotate1/2/3.}\EN{Now let's get back to the rotate1/2/3 functions.}

\begin{lstlisting}
.text:00541070 rotate1         proc near
.text:00541070
\end{lstlisting}

\RU{Выделение внутреннего массива размером 64 байта в локальном стеке:}
\EN{Internal array allocation in the local stack, with size of 64 bytes:}

\begin{lstlisting}
.text:00541070 internal_array_64= byte ptr -40h
.text:00541070 arg_0           = dword ptr  4
.text:00541070
.text:00541070                 sub     esp, 40h
.text:00541073                 push    ebx
.text:00541074                 push    ebp
.text:00541075                 mov     ebp, [esp+48h+arg_0]
.text:00541079                 push    esi
.text:0054107A                 push    edi
.text:0054107B                 xor     edi, edi        ; EDI is loop1 counter
\end{lstlisting}

\EBX \RU{указывает на внутренний массив}\EN{is a pointer to the internal array:}

\begin{lstlisting}
.text:0054107D                 lea     ebx, [esp+50h+internal_array_64]
.text:00541081
\end{lstlisting}

\RU{Здесь два вложенных цикла:}\EN{Here we have two nested loops:}

\lstinputlisting{examples/qr9/5.\LANG}

\RU{Мы видим, что оба счетчика циклов в интервале 0..7. 
Также, они используются как первый и второй аргумент \TT{get\_bit()}.
Третий аргумент \TT{get\_bit()} это единственный аргумент \TT{rotate1()}. 
То что возвращает \TT{get\_bit()} будет сохранено во внутреннем массиве.}
\EN{\dots we see that both loops' counters are in the range of 0..7. 
Also they are used as the first and second argument for the \TT{get\_bit()} function.
The third argument to \TT{get\_bit()} is the only argument of \TT{rotate1()}. 
The return value from \TT{get\_bit()} is placed in the internal array.}

\RU{Снова приготовить указатель на внутренний массив:}\EN{Prepare a pointer to the internal array again:}

\lstinputlisting{examples/qr9/6.\LANG}

\RU{\dots этот код помещает содержимое из внутреннего массива в глобальный массив cube используя функцию 
\TT{set\_bit()}, \IT{но}, в обратном порядке!
Теперь счетчик первого цикла в интервале 7 до 0, уменьшается на 1 на каждой итерации!}
\EN{\dots this code is placing the contents of the internal array to the cube global array via the \TT{set\_bit()} function, 
\IT{but} in a different order!
Now the counter of the first loop is in the range of 7 to 0, \glslink{decrement}{decrementing} at each iteration!}

\RU{Представление кода на Си выглядит так:}\EN{The C code representation looks like:}

\begin{lstlisting}
void rotate1 (int v)
{
	bool tmp[8][8]; // internal array
	int i, j;

	for (i=0; i<8; i++)
		for (j=0; j<8; j++)
			tmp[i][j]=get_bit (i, j, v);

	for (i=0; i<8; i++)
		for (j=0; j<8; j++)
			set_bit (j, 7-i, v, tmp[x][y]);
};
\end{lstlisting}

\RU{Не очень понятно, но если мы посмотрим в функцию \TT{rotate2()}:}
\EN{Not very understandable, but if we take a look at \TT{rotate2()} function:}

\lstinputlisting{examples/qr9/7.\LANG}

\RU{\IT{Почти} то же самое, за исключением иного порядка аргументов в \TT{get\_bit()} и \TT{set\_bit()}.
Перепишем это на Си-подобный код:}
\EN{It is \IT{almost} the same, except the order of the arguments of the \TT{get\_bit()} and \TT{set\_bit()} is different. 
Let's rewrite it in C-like code:}

\begin{lstlisting}
void rotate2 (int v)
{
	bool tmp[8][8]; // internal array
	int i, j;

	for (i=0; i<8; i++)
		for (j=0; j<8; j++)
			tmp[i][j]=get_bit (v, i, j);

	for (i=0; i<8; i++)
		for (j=0; j<8; j++)
			set_bit (v, j, 7-i, tmp[i][j]);
};
\end{lstlisting}

\RU{Перепишем так же функцию \TT{rotate3()}:}\EN{Let's also rewrite the \TT{rotate3()} function:}

\begin{lstlisting}
void rotate3 (int v)
{
	bool tmp[8][8];
	int i, j;

	for (i=0; i<8; i++)
		for (j=0; j<8; j++)
			tmp[i][j]=get_bit (i, v, j);

	for (i=0; i<8; i++)
		for (j=0; j<8; j++)
			set_bit (7-j, v, i, tmp[i][j]);
};
\end{lstlisting}

\RU{Теперь всё проще. Если мы представим cube64 как трехмерный куб 8*8*8, где каждый элемент это бит,
то \TT{get\_bit()} и \TT{set\_bit()} просто берут на вход координаты бита.}
\EN{Well, now things are simpler. If we consider cube64 as a 3D cube of size 8*8*8, where each element is a bit, 
\TT{get\_bit()} and \TT{set\_bit()} take just the coordinates of a bit as input.}

\RU{Функции rotate1/2/3 просто поворачивают все биты на определенной плоскости.
Три функции, каждая на каждую сторону куба и аргумент \TT{v} выставляет плоскость в интервале 0..7}
\EN{The rotate1/2/3 functions are in fact rotating all bits in a specific plane. 
These three functions are one for each cube side and the \TT{v} argument sets the plane in the range of 0..7.}


\RU{Может быть, автор алгоритма думал о кубике Рубика 8*8*8}
\EN{Maybe, the algorithm's author was thinking of a 8*8*8 Rubik's cube}
\footnote{\href{http://go.yurichev.com/17115}{wikipedia}}?!

\RU{Да, действительно.}\EN{Yes, indeed.}

\RU{Рассмотрим функцию \TT{decrypt()}, вот её переписанная версия:}%
\EN{Let's look closer into the \TT{decrypt()} function, here is its rewritten version:}

\begin{lstlisting}
void decrypt (BYTE *buf, int sz, char *pw)
{
	char *p=strdup (pw);
	strrev (p);
	int i=0;

	do
	{
		memcpy (cube, buf+i, 8*8);
		rotate_all (p, 3);
		memcpy (buf+i, cube, 8*8);
		i+=64;
	}
	while (i<sz);
	
	free (p);
};
\end{lstlisting}


\RU{Почти то же самое что и crypt(), \IT{но} строка пароля разворачивается стандартной функцией Си}
\EN{It is almost the same as for \TT{crypt()}, \IT{but} the password string is reversed by the}
strrev() \footnote{\href{http://go.yurichev.com/17249}{MSDN}}
\RU{и \TT{rotate\_all()} вызывается с аргументом 3.}
\EN{standard C function and \TT{rotate\_all()} is called with argument 3.} 

\RU{Это значит, что, в случае дешифровки, rotate1/2/3 будут вызываться трижды.}
\EN{This implies that in case of decryption, each corresponding rotate1/2/3 call is to be performed thrice.}

\RU{Это почти кубик Рубика!
Если вы хотите вернуть его состояние назад, делайте то же самое в обратном порядке и направлении!
Чтобы вернуть эффект от поворота плоскости по часовой стрелке, нужно повернуть её же против 
часовой стрелки, либо же трижды по часовой стрелке.}
\EN{This is almost as in Rubik'c cube! 
If you want to get back, do the same in reverse order and direction! 
If you need to undo the effect of rotating one place in clockwise direction, 
rotate it once in counter-clockwise direction, or thrice in clockwise direction.}

\RU{\TT{rotate1()}, вероятно, поворот \q{лицевой} плоскости. 
\TT{rotate2()}, вероятно, поворот \q{верхней} плоскости.
\TT{rotate3()}, вероятно, поворот \q{левой} плоскости.}
\EN{\TT{rotate1()} is apparently for rotating the \q{front} plane. 
\TT{rotate2()} is apparently for rotating the \q{top} plane. 
\TT{rotate3()} is apparently for rotating the \q{left} plane.}

\RU{Вернемся к ядру функции \TT{rotate\_all()}}\EN{Let's get back to the core of the \TT{rotate\_all()} function:}

\begin{lstlisting}
q=c-'a';
if (q>24)
	q-=24;

int quotient=q/3; // in range 0..7
int remainder=q % 3;

switch (remainder)
{
    case 0: for (int i=0; i<v; i++) rotate1 (quotient); break; // front
    case 1: for (int i=0; i<v; i++) rotate2 (quotient); break; // top
    case 2: for (int i=0; i<v; i++) rotate3 (quotient); break; // left
};
\end{lstlisting}

\RU{Так понять проще: каждый символ пароля определяет сторону (одну из трех) и плоскость (одну из восьми).
3*8 = 24, вот почему два последних символа латинского алфавита переопределяются так чтобы алфавит состоял
из 24-х элементов.}
\EN{Now it is much simpler to understand: each password character defines a side (one of three) and a plane (one of 8). 
3*8 = 24, that is why two the last two characters of the Latin alphabet are remapped to fit an alphabet of exactly 
24 elements.}

\RU{Алгоритм очевидно слаб: в случае коротких паролей, в бинарном редакторе файлов можно будет увидеть, 
что в зашифрованных файлах остались незашифрованные символы.}
\EN{The algorithm is clearly weak: in case of short passwords you can see
that in the encrypted file there are 
the original bytes of the original file in a binary file editor.}

\RU{Весь исходный код в реконструированном виде:}\EN{Here is the whole source code reconstructed:}

\lstinputlisting{examples/qr9/qr9.cpp}



\chapter{SAP}

\RU{\input{examples/SAP/sapgui/sapgui_RU}}
\RU{\input{examples/SAP/sapgui/sapgui_EN}}
\section{\RU{Функции проверки пароля в SAP 6.0}\EN{SAP 6.0 password checking functions}}

\index{SAP}
\RU{Когда автор этой книги в очередной раз вернулся к своему SAP 6.0 IDES заинсталлированному в виртуальной машине VMware, он обнаружил что забыл пароль, впрочем, затем он вспомнил его, но теперь получаем такую ошибку:}
\EN{One time when the author of this book have returned again to his SAP 6.0 IDES installed in a VMware box, he figured out that he forgot the password for the SAP* account, then he have remembered it, but then we got this error message} 
\IT{<<Password logon no longer possible - too many failed attempts>>}, 
\RU{потому что были потрачены все попытки на то, чтобы вспомнить его}
\EN{since he've made all these attempts in trying to recall it}.

\index{Windows!PDB}
\RU{Первая очень хорошая новость состоит в том, что с SAP поставляется полный \gls{PDB}-файл \IT{disp+work.pdb}, он содержит все: имена функций, структуры, типы, локальные переменные, имена аргументов, \etc{}. Какой щедрый подарок!}
\EN{The first extremely good news was that the full \IT{disp+work.pdb} \gls{PDB} file is supplied with SAP, and it contain almost everything: function names, structures, types, local variable and argument names, \etc{}. What a lavish gift!}

\RU{Существует утилита}\EN{There is} TYPEINFODUMP\footnote{\url{http://go.yurichev.com/17038}} \RU{для дампа содержимого \gls{PDB}-файлов во что-то более читаемое и grep-абельное}\EN{utility for converting \gls{PDB} files into something readable and grepable}.

\RU{Вот пример её работы: информация о функции + её аргументах + её локальных переменных:}\EN{Here is an example of a function information + its arguments + its local variables:}

\begin{lstlisting}
FUNCTION ThVmcSysEvent 
  Address:         10143190  Size:      675 bytes  Index:    60483  TypeIndex:    60484 
  Type: int NEAR_C ThVmcSysEvent (unsigned int, unsigned char, unsigned short*)
Flags: 0 
PARAMETER events 
  Address: Reg335+288  Size:        4 bytes  Index:    60488  TypeIndex:    60489 
  Type: unsigned int
Flags: d0 
PARAMETER opcode 
  Address: Reg335+296  Size:        1 bytes  Index:    60490  TypeIndex:    60491 
  Type: unsigned char
Flags: d0 
PARAMETER serverName 
  Address: Reg335+304  Size:        8 bytes  Index:    60492  TypeIndex:    60493 
  Type: unsigned short*
Flags: d0 
STATIC_LOCAL_VAR func 
  Address:         12274af0  Size:        8 bytes  Index:    60495  TypeIndex:    60496 
  Type: wchar_t*
Flags: 80 
LOCAL_VAR admhead 
  Address: Reg335+304  Size:        8 bytes  Index:    60498  TypeIndex:    60499 
  Type: unsigned char*
Flags: 90 
LOCAL_VAR record 
  Address: Reg335+64  Size:      204 bytes  Index:    60501  TypeIndex:    60502 
  Type: AD_RECORD
Flags: 90 
LOCAL_VAR adlen 
  Address: Reg335+296  Size:        4 bytes  Index:    60508  TypeIndex:    60509 
  Type: int
Flags: 90 
\end{lstlisting}

\RU{А вот пример дампа структуры:}\EN{And here is an example of some structure:}

\begin{lstlisting}
STRUCT DBSL_STMTID 
Size: 120  Variables: 4  Functions: 0  Base classes: 0
MEMBER moduletype 
  Type:  DBSL_MODULETYPE
  Offset:        0  Index:        3  TypeIndex:    38653
MEMBER module 
  Type:  wchar_t module[40]
  Offset:        4  Index:        3  TypeIndex:      831
MEMBER stmtnum 
  Type:  long
  Offset:       84  Index:        3  TypeIndex:      440
MEMBER timestamp 
  Type:  wchar_t timestamp[15]
  Offset:       88  Index:        3  TypeIndex:     6612
\end{lstlisting}

\RU{Вау!}\EN{Wow!}

\RU{Вторая хорошая новость: \IT{отладочные} вызовы, коих здесь очень много, очень полезны.}\EN{Another good news: \IT{debugging} calls (there are plenty of them) are very useful.} 

\RU{Здесь вы можете увидеть глобальную переменную \IT{ct\_level}}\EN{Here you can also notice the \IT{ct\_level} global variable}\footnote{\RU{Еще об уровне трассировки}\EN{More about trace level}: \url{http://go.yurichev.com/17039}}, \RU{отражающую уровень трассировки}\EN{that reflects the current trace level}.

\RU{В \IT{disp+work.exe} очень много таких отладочных вставок}\EN{There are a lot of debugging inserts in the \IT{disp+work.exe} file}:

\begin{lstlisting}
cmp     cs:ct_level, 1
jl      short loc_1400375DA
call    DpLock
lea     rcx, aDpxxtool4_c ; "dpxxtool4.c"
mov     edx, 4Eh        ; line
call    CTrcSaveLocation
mov     r8, cs:func_48
mov     rcx, cs:hdl     ; hdl
lea     rdx, aSDpreadmemvalu ; "%s: DpReadMemValue (%d)"
mov     r9d, ebx
call    DpTrcErr
call    DpUnlock
\end{lstlisting}

\RU{Если текущий уровень трассировки выше или равен заданному в этом коде порогу, 
отладочное сообщение будет записано в лог-файл вроде \IT{dev\_w0}, \IT{dev\_disp} 
и прочие файлы \IT{dev*}.}
\EN{If the current trace level is bigger or equal to threshold defined in the code here, 
a debugging message is to be written to the log files like \IT{dev\_w0}, \IT{dev\_disp}, 
and other \IT{dev*} files.}

\index{\GrepUsage}
\RU{Попробуем grep-ать файл недавно полученный при помощи утилиты TYPEINFODUMP:}
\EN{Let's try grepping in the file that we have got with the help of the TYPEINFODUMP utility:}

\begin{lstlisting}
cat "disp+work.pdb.d" | grep FUNCTION | grep -i password
\end{lstlisting}

\RU{Получаем:}\EN{We have got:}

\begin{lstlisting}
FUNCTION rcui::AgiPassword::DiagISelection 
FUNCTION ssf_password_encrypt 
FUNCTION ssf_password_decrypt 
FUNCTION password_logon_disabled 
FUNCTION dySignSkipUserPassword 
FUNCTION migrate_password_history 
FUNCTION password_is_initial 
FUNCTION rcui::AgiPassword::IsVisible 
FUNCTION password_distance_ok 
FUNCTION get_password_downwards_compatibility 
FUNCTION dySignUnSkipUserPassword 
FUNCTION rcui::AgiPassword::GetTypeName 
FUNCTION `rcui::AgiPassword::AgiPassword'::`1'::dtor$2 
FUNCTION `rcui::AgiPassword::AgiPassword'::`1'::dtor$0 
FUNCTION `rcui::AgiPassword::AgiPassword'::`1'::dtor$1 
FUNCTION usm_set_password 
FUNCTION rcui::AgiPassword::TraceTo 
FUNCTION days_since_last_password_change 
FUNCTION rsecgrp_generate_random_password 
FUNCTION rcui::AgiPassword::`scalar deleting destructor' 
FUNCTION password_attempt_limit_exceeded 
FUNCTION handle_incorrect_password 
FUNCTION `rcui::AgiPassword::`scalar deleting destructor''::`1'::dtor$1 
FUNCTION calculate_new_password_hash 
FUNCTION shift_password_to_history 
FUNCTION rcui::AgiPassword::GetType 
FUNCTION found_password_in_history 
FUNCTION `rcui::AgiPassword::`scalar deleting destructor''::`1'::dtor$0 
FUNCTION rcui::AgiObj::IsaPassword 
FUNCTION password_idle_check 
FUNCTION SlicHwPasswordForDay 
FUNCTION rcui::AgiPassword::IsaPassword 
FUNCTION rcui::AgiPassword::AgiPassword 
FUNCTION delete_user_password 
FUNCTION usm_set_user_password 
FUNCTION Password_API 
FUNCTION get_password_change_for_SSO 
FUNCTION password_in_USR40 
FUNCTION rsec_agrp_abap_generate_random_password 
\end{lstlisting}

\RU{Попробуем так же искать отладочные сообщения содержащие слова \IT{<<password>>} и \IT{<<locked>>}.}\EN{Let's also try to search for debug messages which contain the words \IT{<<password>>} and \IT{<<locked>>}.}
\RU{Одна из таких это строка}\EN{One of them is the string} \IT{<<user was locked by subsequently failed password logon attempts>>} \RU{на которую есть ссылка в}\EN{, referenced in} \\
\RU{функции}\EN{function} \IT{password\_attempt\_limit\_exceeded()}.

\RU{Другие строки, которые эта найденная функция может писать в лог-файл это}
\EN{Other strings that this function can write to a log file are}: 
\IT{<<password logon attempt will be rejected immediately (preventing dictionary attacks)>>}, \IT{<<failed-logon lock: expired (but not removed due to 'read-only' operation)>>}, \IT{<<failed-logon lock: expired => removed>>}.

\RU{Немного поэкспериментировав с этой функцией, мы быстро понимаем что проблема именно в ней}%
\EN{After playing for a little with this function, we noticed that the problem is exactly in it}.
\RU{Она вызывается из функции \IT{chckpass()}\EMDASH{}одна из функций проверяющих пароль}\EN{It is called from the \IT{chckpass()} function~---one of the password checking functions}.

\RU{В начале, давайте убедимся, что мы на верном пути}%
\EN{First, we would like to make sure that we are at the correct point}:

\RU{Запускаем}\EN{Run} \tracer:
\index{tracer}

\begin{lstlisting}
tracer64.exe -a:disp+work.exe bpf=disp+work.exe!chckpass,args:3,unicode
\end{lstlisting}

\begin{lstlisting}
PID=2236|TID=2248|(0) disp+work.exe!chckpass (0x202c770, L"Brewered1                               ", 0x41) (called from 0x1402f1060 (disp+work.exe!usrexist+0x3c0))
PID=2236|TID=2248|(0) disp+work.exe!chckpass -> 0x35
\end{lstlisting}

\RU{Функции вызываются так}\EN{The call path is}: \IT{syssigni()} -> \IT{DyISigni()} -> \IT{dychkusr()} -> \IT{usrexist()} -> \IT{chckpass()}.

\RU{Число}\EN{The number} 0x35 \RU{возвращается из}\EN{is an error returned in} \IT{chckpass()} \RU{в этом месте}\EN{at that point}:

\begin{lstlisting}
.text:00000001402ED567 loc_1402ED567:                          ; CODE XREF: chckpass+B4
.text:00000001402ED567                 mov     rcx, rbx        ; usr02
.text:00000001402ED56A                 call    password_idle_check
.text:00000001402ED56F                 cmp     eax, 33h
.text:00000001402ED572                 jz      loc_1402EDB4E
.text:00000001402ED578                 cmp     eax, 36h
.text:00000001402ED57B                 jz      loc_1402EDB3D
.text:00000001402ED581                 xor     edx, edx        ; usr02_readonly
.text:00000001402ED583                 mov     rcx, rbx        ; usr02
.text:00000001402ED586                 call    password_attempt_limit_exceeded
.text:00000001402ED58B                 test    al, al
.text:00000001402ED58D                 jz      short loc_1402ED5A0
.text:00000001402ED58F                 mov     eax, 35h
.text:00000001402ED594                 add     rsp, 60h
.text:00000001402ED598                 pop     r14
.text:00000001402ED59A                 pop     r12
.text:00000001402ED59C                 pop     rdi
.text:00000001402ED59D                 pop     rsi
.text:00000001402ED59E                 pop     rbx
.text:00000001402ED59F                 retn
\end{lstlisting}

\RU{Отлично, давайте проверим}\EN{Fine, let's check}:

\begin{lstlisting}
tracer64.exe -a:disp+work.exe bpf=disp+work.exe!password_attempt_limit_exceeded,args:4,unicode,rt:0
\end{lstlisting}

\begin{lstlisting}
PID=2744|TID=360|(0) disp+work.exe!password_attempt_limit_exceeded (0x202c770, 0, 0x257758, 0) (called from 0x1402ed58b (disp+work.exe!chckpass+0xeb))
PID=2744|TID=360|(0) disp+work.exe!password_attempt_limit_exceeded -> 1
PID=2744|TID=360|We modify return value (EAX/RAX) of this function to 0
PID=2744|TID=360|(0) disp+work.exe!password_attempt_limit_exceeded (0x202c770, 0, 0, 0) (called from 0x1402e9794 (disp+work.exe!chngpass+0xe4))
PID=2744|TID=360|(0) disp+work.exe!password_attempt_limit_exceeded -> 1
PID=2744|TID=360|We modify return value (EAX/RAX) of this function to 0
\end{lstlisting}

\RU{Великолепно! Теперь мы можем успешно залогиниться.}\EN{Excellent! We can successfully login now.}

\RU{Кстати, мы можем сделать вид что вообще забыли пароль, заставляя \IT{chckpass()} всегда возвращать ноль, и этого достаточно для отключения проверки пароля:}
\EN{By the way, we can pretend we forgot the password, fixing the \IT{chckpass()} function to return a value of 0 is enough to bypass the check:}

\begin{lstlisting}
tracer64.exe -a:disp+work.exe bpf=disp+work.exe!chckpass,args:3,unicode,rt:0
\end{lstlisting}

\begin{lstlisting}
PID=2744|TID=360|(0) disp+work.exe!chckpass (0x202c770, L"bogus                                   ", 0x41) (called from 0x1402f1060 (disp+work.exe!usrexist+0x3c0))
PID=2744|TID=360|(0) disp+work.exe!chckpass -> 0x35
PID=2744|TID=360|We modify return value (EAX/RAX) of this function to 0
\end{lstlisting}

\RU{Что еще можно сказать, бегло анализируя функцию \IT{password\_attempt\_limit\_exceeded()}, это то, что в начале
можно увидеть следующий вызов:}\EN{What also can be said while analyzing the 
\IT{password\_attempt\_limit\_exceeded()} 
function is that at the very beginning of it, this call can be seen:}

\begin{lstlisting}
lea     rcx, aLoginFailed_us ; "login/failed_user_auto_unlock"
call    sapgparam
test    rax, rax
jz      short loc_1402E19DE
movzx   eax, word ptr [rax]
cmp     ax, 'N'
jz      short loc_1402E19D4
cmp     ax, 'n'
jz      short loc_1402E19D4
cmp     ax, '0'
jnz     short loc_1402E19DE
\end{lstlisting}

\RU{Очевидно, функция \IT{sapgparam()} используется чтобы узнать значение какой-либо переменной конфигурации. Эта функция может вызываться из 1768 разных мест.}
\EN{Obviously, function \IT{sapgparam()} is used to query the value of some configuration parameter. This function can be called from 1768 different places.}
\RU{Вероятно, при помощи этой информации, мы можем легко находить те места кода, на которые влияют определенные переменные конфигурации.}\EN{It seems that with the help of this information, we can easily find the places in code, the control flow of which can be affected by specific configuration parameters.}

\RU{Замечательно! Имена функций очень понятны, куда понятнее чем в \oracle.}
\EN{It is really sweet. The function names are very clear, much clearer than in the \oracle.} 
\index{\Cpp}
\RU{По всей видимости, процесс \IT{disp+work} весь написан на \Cpp. Вероятно, он был переписан не так давно?}
\EN{It seems that the \IT{disp+work} process is written in \Cpp. It was apparently rewritten some time ago?}



\ifdefined\IncludeOracle
\chapter{\oracle}
\label{oracle}

% sections
\section{\IFRU{Таблица \TT{V\$VERSION} в \oracle}{\TT{V\$VERSION} table in the \oracle}}

\index{\oracle}
\index{Linux}
\index{Windows!ntoskrnl.exe}
\IFRU{\oracle 11.2 это очень большая программа, основной модуль \TT{oracle.exe} содержит около 124 тысячи функций.}{\oracle 11.2 is a huge program, main module \TT{oracle.exe} contain approx. 124,000 functions.} \IFRU{Для сравнения, ядро Windows 7 x64 (ntoskrnl.exe) ~--- около 11 тысяч функций, а ядро Linux 3.9.8 (с драйверами по умолчанию) ~--- 31 тысяч функций.}{For comparison, Windows 7 x86 kernel (ntoskrnl.exe)~---approx. 11,000 functions and Linux 3.9.8 kernel (with default drivers compiled)~---31,000 functions.}

\IFRU{Начнем с одного простого вопроса. Откуда \oracle берет информацию, когда мы в SQL*Plus пишем вот такой вот простой запрос:}{Let's start with an easy question. Where \oracle get all this information, when we execute such simple statement in SQL*Plus:}

\begin{lstlisting}
SQL> select * from V$VERSION;
\end{lstlisting}

\IFRU{И получаем}{And we've got}:

\begin{lstlisting}
BANNER
--------------------------------------------------------------------------------

Oracle Database 11g Enterprise Edition Release 11.2.0.1.0 - Production
PL/SQL Release 11.2.0.1.0 - Production
CORE    11.2.0.1.0      Production
TNS for 32-bit Windows: Version 11.2.0.1.0 - Production
NLSRTL Version 11.2.0.1.0 - Production
\end{lstlisting}

\IFRU{Начнем. Где в самом \oracle мы можем найти строку}{Let's start. Where in the \oracle we may find a string} \TT{V\$VERSION}?

\IFRU{Для win32-версии, эта строка имеется в файле \TT{oracle.exe}, это легко увидеть.}
{As of win32-version, \TT{oracle.exe} file contain the string,
which can be investigated easily.}
\IFRU{Но мы так же можем использовать объектные (.o) файлы от версии \oracle для Linux, потому что в них сохраняются имена функций и глобальных переменных, а в \TT{oracle.exe} для win32 этого нет.}{But we can also use object (.o) files from Linux version of \oracle since, unlike win32 version \TT{oracle.exe}, function names (and global variables as well) are preserved there.}

\IFRU{Итак, строка \TT{V\$VERSION} имеется в файле \TT{kqf.o}, в самой главной Oracle-библиотеке \TT{libserver11.a}.}{So, \TT{kqf.o} file contain \TT{V\$VERSION} string.
The object file is in the main Oracle-library \TT{libserver11.a}.}

\IFRU{Ссылка на эту текстовую строку имеется в таблице \TT{kqfviw}, размещенной в этом же файле \TT{kqf.o}}{A reference to this text string we may find in the \TT{kqfviw} table stored in the same file, \TT{kqf.o}}:

\begin{lstlisting}[caption=kqf.o]
.rodata:0800C4A0 kqfviw          dd 0Bh                  ; DATA XREF: kqfchk:loc_8003A6D
.rodata:0800C4A0                                         ; kqfgbn+34
.rodata:0800C4A4                 dd offset _2__STRING_10102_0 ; "GV$WAITSTAT"
.rodata:0800C4A8                 dd 4
.rodata:0800C4AC                 dd offset _2__STRING_10103_0 ; "NULL"
.rodata:0800C4B0                 dd 3
.rodata:0800C4B4                 dd 0
.rodata:0800C4B8                 dd 195h
.rodata:0800C4BC                 dd 4
.rodata:0800C4C0                 dd 0
.rodata:0800C4C4                 dd 0FFFFC1CBh
.rodata:0800C4C8                 dd 3
.rodata:0800C4CC                 dd 0
.rodata:0800C4D0                 dd 0Ah
.rodata:0800C4D4                 dd offset _2__STRING_10104_0 ; "V$WAITSTAT"
.rodata:0800C4D8                 dd 4
.rodata:0800C4DC                 dd offset _2__STRING_10103_0 ; "NULL"
.rodata:0800C4E0                 dd 3
.rodata:0800C4E4                 dd 0
.rodata:0800C4E8                 dd 4Eh
.rodata:0800C4EC                 dd 3
.rodata:0800C4F0                 dd 0
.rodata:0800C4F4                 dd 0FFFFC003h
.rodata:0800C4F8                 dd 4
.rodata:0800C4FC                 dd 0
.rodata:0800C500                 dd 5
.rodata:0800C504                 dd offset _2__STRING_10105_0 ; "GV$BH"
.rodata:0800C508                 dd 4
.rodata:0800C50C                 dd offset _2__STRING_10103_0 ; "NULL"
.rodata:0800C510                 dd 3
.rodata:0800C514                 dd 0
.rodata:0800C518                 dd 269h
.rodata:0800C51C                 dd 15h
.rodata:0800C520                 dd 0
.rodata:0800C524                 dd 0FFFFC1EDh
.rodata:0800C528                 dd 8
.rodata:0800C52C                 dd 0
.rodata:0800C530                 dd 4
.rodata:0800C534                 dd offset _2__STRING_10106_0 ; "V$BH"
.rodata:0800C538                 dd 4
.rodata:0800C53C                 dd offset _2__STRING_10103_0 ; "NULL"
.rodata:0800C540                 dd 3
.rodata:0800C544                 dd 0
.rodata:0800C548                 dd 0F5h
.rodata:0800C54C                 dd 14h
.rodata:0800C550                 dd 0
.rodata:0800C554                 dd 0FFFFC1EEh
.rodata:0800C558                 dd 5
.rodata:0800C55C                 dd 0
\end{lstlisting}

\IFRU{Кстати, нередко, при изучении внутренностей \oracle, появляется вопрос, почему имена функций и глобальных переменных такие странные}{By the way, often, while analysing \oracle internals, you may ask yourself, why functions and global variable names are so weird}. \IFRU{Вероятно, дело в том, что \oracle очень старый продукт сам по себе и писался на Си еще в 1980-х}
{Supposedly, since \oracle is very old product and was developed in C in 1980-s}. \IFRU{А в те времена стандарт Си гарантировал поддержку имен переменных длиной только до шести символов включительно}
{And that was a time when C standard guaranteed function names/variables support only up to 6 characters inclusive}: <<6 significant initial characters in an external identifier>>\footnote{\href{http://yurichev.com/ref/Draft\%20ANSI\%20C\%20Standard\%20(ANSI\%20X3J11-88-090)\%20(May\%2013,\%201988).txt}{Draft ANSI C Standard (ANSI X3J11/88-090) (May 13, 1988)}}

\IFRU{Вероятно, таблица \TT{kqfviw} содержащая в себе многие (а может даже и все) view с префиксом V\$, это служебные view (fixed views), присутствующие всегда.}{Probably, the table \TT{kqfviw} contain most (maybe even all) views prefixed with V\$, these are \IT{fixed views}, present all the time.}
\IFRU{Бегло оценив цикличность данных, мы легко видим, что в каждом элементе таблицы \TT{kqfviw} 12 полей 32-битных полей.}{Superficially, by noticing cyclic recurrence of data, we can easily see that each \TT{kqfviw} table element has 12 32-bit fields.}
\IFRU{В \IDA легко создать структуру из 12-и элементов и применить её ко всем элементам таблицы.}{It is very simple to create a 12-elements structure in \IDA and apply it to all table elements.}
\IFRU{Для версии \oracle 11.2, здесь 1023 элемента в таблице, то есть, здесь описываются 1023 всех возможных \IT{fixed view}.}{As of \oracle version 11.2, there are 1023 table elements, i.e., there are described 1023 of all possible \IT{fixed views}.}
\IFRU{Позже, мы еще вернемся к этому числу.}{We will return to this number later.}

\IFRU{Как видно, мы не очень много можем узнать чисел в этих полях. Самое первое число всегда равно длине строки-названия view (без терминирующего ноля).}
{As we can see, there is not much information in these numbers in fields. The very first number is always equals to name of view (without terminating zero.}
\IFRU{Это справедливо для всех элементов. Но эта информация не очень полезна.}{This is correct for each element. But this information is not very useful.}

\IFRU{Мы также знаем, что информацию обо всех fixed views можно получить из \IT{fixed view} под названием}
{We also know that information about all fixed views can be retrieved from \IT{fixed view} named} \TT{V\$FIXED\_VIEW\_DEFINITION}
\IFRU{(кстати, информация для этого view также берется из таблиц \TT{kqfviw} и \TT{kqfvip}).}{(by the way, the information for this view is also taken from \TT{kqfviw} and \TT{kqfvip} tables.)}
\IFRU{Кстати, там тоже 1023 элемента.}{By the way, there are 1023 elements too.}

\begin{lstlisting}
SQL> select * from V$FIXED_VIEW_DEFINITION where view_name='V$VERSION';

VIEW_NAME
------------------------------
VIEW_DEFINITION
--------------------------------------------------------------------------------

V$VERSION
select  BANNER from GV$VERSION where inst_id = USERENV('Instance')
\end{lstlisting}

\IFRU{Итак, \TT{V\$VERSION} это как бы \IT{thunk view} для другого, с названием \TT{GV\$VERSION}, который, в свою очередь:}
{So, \TT{V\$VERSION} is some kind of \IT{thunk view} for another view, named \TT{GV\$VERSION}, which is, in turn:}

\begin{lstlisting}
SQL> select * from V$FIXED_VIEW_DEFINITION where view_name='GV$VERSION';

VIEW_NAME
------------------------------
VIEW_DEFINITION
--------------------------------------------------------------------------------

GV$VERSION
select inst_id, banner from x$version
\end{lstlisting}

\IFRU{Таблицы с префиксом X\$ в \oracle ~--- это также служебные таблицы, они не документированы, не могут изменятся пользователем, и обновляются динамически.}{Tables prefixed as X\$ in the \oracle -- is service tables too, undocumented, cannot be changed by user and refreshed dynamically.}

\IFRU{Попробуем поискать текст}{Let's also try to search the text} \TT{select  BANNER from GV\$VERSION where inst\_id = USERENV('Instance')} \IFRU{в файле \TT{kqf.o} и находим ссылку на него в таблице \TT{kqfvip}}{in the \TT{kqf.o} file and we find it in the \TT{kqfvip} table}:

.\begin{lstlisting}[caption=kqf.o]
rodata:080185A0 kqfvip          dd offset _2__STRING_11126_0 ; DATA XREF: kqfgvcn+18
.rodata:080185A0                                         ; kqfgvt+F
.rodata:080185A0                                         ; "select inst_id,decode(indx,1,'data bloc"...
.rodata:080185A4                 dd offset kqfv459_c_0
.rodata:080185A8                 dd 0
.rodata:080185AC                 dd 0

...

.rodata:08019570                 dd offset _2__STRING_11378_0 ; "select  BANNER from GV$VERSION where in"...
.rodata:08019574                 dd offset kqfv133_c_0
.rodata:08019578                 dd 0
.rodata:0801957C                 dd 0
.rodata:08019580                 dd offset _2__STRING_11379_0 ; "select inst_id,decode(bitand(cfflg,1),0"...
.rodata:08019584                 dd offset kqfv403_c_0
.rodata:08019588                 dd 0
.rodata:0801958C                 dd 0
.rodata:08019590                 dd offset _2__STRING_11380_0 ; "select  STATUS , NAME, IS_RECOVERY_DEST"...
.rodata:08019594                 dd offset kqfv199_c_0
\end{lstlisting}

\IFRU{Таблица, по всей видимости, имеет 4 поля в каждом элементе. Кстати, здесь также 1023 элемента.}
{The table appear to have 4 fields in each element.
By the way, there are 1023 elements too.}
\IFRU{Второе поле указывает на другую таблицу, содержащую поля этого \IT{fixed view}.}
{The second field pointing to another table, containing table fields for this \IT{fixed view}.}
\IFRU{Для \TT{V\$VERSION}, эта таблица только из двух элементов, первый это $6$ и второй это строка 
\TT{BANNER} (число это длина строки) и далее \IT{терминирующий} элемент содержащий $0$ и \IT{нулевую} 
Си-строку:}{As of \TT{V\$VERSION}, this table contain only two elements, first is $6$ and second is 
\TT{BANNER} string (the number ($6$) is this string length) and after, \IT{terminating} element contain 
$0$ and \IT{null} C-string:}

\begin{lstlisting}[caption=kqf.o]
.rodata:080BBAC4 kqfv133_c_0     dd 6                    ; DATA XREF: .rodata:08019574
.rodata:080BBAC8                 dd offset _2__STRING_5017_0 ; "BANNER"
.rodata:080BBACC                 dd 0
.rodata:080BBAD0                 dd offset _2__STRING_0_0
\end{lstlisting}

\IFRU{Объединив данные из таблиц \TT{kqfviw} и \TT{kqfvip}, мы получим SQL-запросы, которые исполняются, когда пользователь хочет получить информацию из какого-либо \IT{fixed view}.}{By joining data from both \TT{kqfviw} and \TT{kqfvip} tables, we may get SQL-statements which are executed when user wants to query information from specific \IT{fixed view}.}

\IFRU{Я написал программу \oracletables, которая собирает всю эту информацию из объектных файлов от \oracle под Linux.}{So I wrote an \oracletables program, so to gather all this information from \oracle for Linux object files.}
\IFRU{Для \TT{V\$VERSION}, мы можем найти следующее:}{For \TT{V\$VERSION}, we may find this:}

\begin{lstlisting}[caption=\IFRU{Результат работы}{Result of} \OracleTablesName]
kqfviw_element.viewname: [V$VERSION] ?: 0x3 0x43 0x1 0xffffc085 0x4
kqfvip_element.statement: [select  BANNER from GV$VERSION where inst_id = USERENV('Instance')]
kqfvip_element.params:
[BANNER] 
\end{lstlisting}

\AndENRU:

\begin{lstlisting}[caption=\IFRU{Результат работы}{Result of} \OracleTablesName]
kqfviw_element.viewname: [GV$VERSION] ?: 0x3 0x26 0x2 0xffffc192 0x1
kqfvip_element.statement: [select inst_id, banner from x$version]
kqfvip_element.params:
[INST_ID] [BANNER] 
\end{lstlisting}

\IFRU{\IT{Fixed view} \TT{GV\$VERSION} отличается от \TT{V\$VERSION} тем, что содержит еще и поле отражающее идентификатор \IT{instance}.}
{\TT{GV\$VERSION} \IT{fixed view} is distinct from \TT{V\$VERSION} in only that way that it contains one more field with \IT{instance} identifier.}
\IFRU{Но так или иначе, мы теперь упираемся в таблицу \TT{X\$VERSION}. Как и прочие X\$-таблицы, она не документирована, однако, мы можем оттуда что-то прочитать}{Anyway, we stuck at the table \TT{X\$VERSION}. Just like any other X\$-tables, it is undocumented, however, we can query it}:

\begin{lstlisting}
SQL> select * from x$version;

ADDR           INDX    INST_ID
-------- ---------- ----------
BANNER
--------------------------------------------------------------------------------

0DBAF574          0          1
Oracle Database 11g Enterprise Edition Release 11.2.0.1.0 - Production

...
\end{lstlisting}

\IFRU{Эта таблица содержит дополнительные поля вроде \TT{ADDR} и \TT{INDX}.}{This table has additional fields like \TT{ADDR} and \TT{INDX}.}

\IFRU{Бегло листая содержимое файла \TT{kqf.o} в \IDA мы можем увидеть еще одну таблицу где есть ссылка на строку \TT{X\$VERSION}, это \TT{kqftab}:}{While scrolling \TT{kqf.o} in \IDA we may spot another table containing pointer to the \TT{X\$VERSION} string, this is \TT{kqftab}:}

\begin{lstlisting}[caption=kqf.o]
.rodata:0803CAC0                 dd 9                    ; element number 0x1f6
.rodata:0803CAC4                 dd offset _2__STRING_13113_0 ; "X$VERSION"
.rodata:0803CAC8                 dd 4
.rodata:0803CACC                 dd offset _2__STRING_13114_0 ; "kqvt"
.rodata:0803CAD0                 dd 4
.rodata:0803CAD4                 dd 4
.rodata:0803CAD8                 dd 0
.rodata:0803CADC                 dd 4
.rodata:0803CAE0                 dd 0Ch
.rodata:0803CAE4                 dd 0FFFFC075h
.rodata:0803CAE8                 dd 3
.rodata:0803CAEC                 dd 0
.rodata:0803CAF0                 dd 7
.rodata:0803CAF4                 dd offset _2__STRING_13115_0 ; "X$KQFSZ"
.rodata:0803CAF8                 dd 5
.rodata:0803CAFC                 dd offset _2__STRING_13116_0 ; "kqfsz"
.rodata:0803CB00                 dd 1
.rodata:0803CB04                 dd 38h
.rodata:0803CB08                 dd 0
.rodata:0803CB0C                 dd 7
.rodata:0803CB10                 dd 0
.rodata:0803CB14                 dd 0FFFFC09Dh
.rodata:0803CB18                 dd 2
.rodata:0803CB1C                 dd 0
\end{lstlisting}

\IFRU{Здесь очень много ссылок на названия X\$-таблиц, вероятно, на все те что имеются в \oracle этой версии.}
{There are a lot of references to X\$-table names, apparently, to all \oracle 11.2 X\$-tables.}
\IFRU{Но мы снова упираемся в то что не имеем достаточно информации.}{But again, we have not enough information.}
\IFRU{У меня нет никакой идеи, что означает строка \TT{kqvt}.}
{I have no idea, what \TT{kqvt} string means.} 
\IFRU{Вообще, префикс \TT{kq} может означать \IT{kernel} и \IT{query}.}
{\TT{kq} prefix may means \IT{kernel} and \IT{query}.} 
\IFRU{\TT{v}, может быть, \IT{version}, а \TT{t} ~--- \IT{type}?}
{\TT{v}, apparently, means \IT{version} and \TT{t}~---\IT{type}?} 
\IFRU{Я не знаю, честно говоря.}{Frankly speaking, I do not know.}

\IFRU{Таблицу с очень похожим названием мы можем найти в}{The table named similarly can be found in} \TT{kqf.o}:

\begin{lstlisting}[caption=kqf.o]
.rodata:0808C360 kqvt_c_0        kqftap_param <4, offset _2__STRING_19_0, 917h, 0, 0, 0, 4, 0, 0>
.rodata:0808C360                                         ; DATA XREF: .rodata:08042680
.rodata:0808C360                                         ; "ADDR"
.rodata:0808C384                 kqftap_param <4, offset _2__STRING_20_0, 0B02h, 0, 0, 0, 4, 0, 0> ; "INDX"
.rodata:0808C3A8                 kqftap_param <7, offset _2__STRING_21_0, 0B02h, 0, 0, 0, 4, 0, 0> ; "INST_ID"
.rodata:0808C3CC                 kqftap_param <6, offset _2__STRING_5017_0, 601h, 0, 0, 0, 50h, 0, 0> ; "BANNER"
.rodata:0808C3F0                 kqftap_param <0, offset _2__STRING_0_0, 0, 0, 0, 0, 0, 0, 0>
\end{lstlisting}

\IFRU{Она содержит информацию об именах полей в таблице \TT{X\$VERSION}.}{It contain information about all fields in the \TT{X\$VERSION} table.}
\IFRU{Единственная ссылка на эту таблицу имеется в таблице \TT{kqftap}:}{The only reference to this table present in the \TT{kqftap} table:}

\begin{lstlisting}[caption=kqf.o]
.rodata:08042680                 kqftap_element <0, offset kqvt_c_0, offset kqvrow, 0> ; element 0x1f6
\end{lstlisting}

\IFRU{Интересно что здесь этот элемент проходит также под номером \TT{0x1f6} (502-й), как и ссылка на строку 
\TT{X\$VERSION} в таблице \TT{kqftab}.}
{It is interesting that this element here is \TT{0x1f6th} (502nd), just as a pointer to the \TT{X\$VERSION} string in 
the \TT{kqftab} table.}
\IFRU{Вероятно, таблицы \TT{kqftap} и \TT{kqftab} дополняют друг друга, как и \TT{kqfvip} и \TT{kqfviw}.}
{Probably, \TT{kqftap} and \TT{kqftab} tables are complement each other, just like \TT{kqfvip} and \TT{kqfviw}.}
\IFRU{Мы также видим здесь ссылку на функцию с названием \TT{kqvrow()}. А вот это уже кое-что!}
{We also see a pointer to the \TT{kqvrow()} function. Finally, we got something useful!}

\IFRU{Я сделал так чтобы моя программа \oracletables могла дампить и эти таблицы. Для \TT{X\$VERSION} получается:}
{So I added these tables to my \oracletables utility too. For \TT{X\$VERSION} I've got:}

\begin{lstlisting}[caption=\IFRU{Результат работы}{Result of} \OracleTablesName]
kqftab_element.name: [X$VERSION] ?: [kqvt] 0x4 0x4 0x4 0xc 0xffffc075 0x3
kqftap_param.name=[ADDR] ?: 0x917 0x0 0x0 0x0 0x4 0x0 0x0
kqftap_param.name=[INDX] ?: 0xb02 0x0 0x0 0x0 0x4 0x0 0x0
kqftap_param.name=[INST_ID] ?: 0xb02 0x0 0x0 0x0 0x4 0x0 0x0
kqftap_param.name=[BANNER] ?: 0x601 0x0 0x0 0x0 0x50 0x0 0x0
kqftap_element.fn1=kqvrow
kqftap_element.fn2=NULL
\end{lstlisting}

\IFRU{При помощи \tracer, можно легко проверить, что эта ф-ция вызывается 6 раз кряду (из ф-ции \TT{qerfxFetch()}) при получении строк из \TT{X\$VERSION}.}{With the help of \tracer, it is easy to check that this function called 6 times in row (from the \TT{qerfxFetch()} function) while querying \TT{X\$VERSION} table.}

\IFRU{Запустим \tracer в режиме \TT{cc} (он добавит комментарий к каждой исполненной инструкции):}{Let's run \tracer in the \TT{cc} mode (it will comment each executed instruction):}

\begin{lstlisting}
tracer -a:oracle.exe bpf=oracle.exe!_kqvrow,trace:cc
\end{lstlisting}

\lstinputlisting{examples/oracle/VERSION_kqvrow.asm}

\IFRU{Так можно легко увидеть, что номер строки таблицы задается извне. Сама ф-ция возвращает строку, формируя её так:}
{Now it is easy to see that row number is passed from outside of function. The function returns the string constructing it as follows:}

\begin{center}
\begin{tabular}{ | l | l | }
\hline                        
\IFRU{Строка}{String} 1	& \IFRU{Использует глобальные переменные \TT{vsnstr}, \TT{vsnnum}, \TT{vsnban}. Вызывает \TT{sprintf()}.}{Using \TT{vsnstr}, \TT{vsnnum}, \TT{vsnban} global variables. Calling \TT{sprintf()}.} \\
\IFRU{Строка}{String} 2	& \IFRU{Вызывает}{Calling} \TT{kkxvsn()}. \\
\IFRU{Строка}{String} 3	& \IFRU{Вызывает}{Calling} \TT{lmxver()}. \\
\IFRU{Строка}{String} 4	& \IFRU{Вызывает}{Calling} \TT{npinli()}, \TT{nrtnsvrs()}. \\
\IFRU{Строка}{String} 5	& \IFRU{Вызывает}{Calling} \TT{lxvers()}. \\
\hline  
\end{tabular}
\end{center}

\IFRU{Так вызываются соответствующие ф-ции для определения номеров версий отдельных модулей.}
{That's how corresponding functions are called for determining each module's version.}


\input{examples/oracle/2_ksmlru.tex}
\section{\RU{Таблица \TT{V\$TIMER} в}\EN{\TT{V\$TIMER} table in} \oracle}
\index{\oracle}

\TT{V\$TIMER} \RU{это еще один служебный \IT{fixed view}, отражающий какое-то часто меняющееся значение:}
\EN{is another \IT{fixed view} that reflects a rapidly changing value:}

\begin{framed}
\begin{quotation}
V\$TIMER displays the elapsed time in hundredths of a second. Time is measured since the beginning of the epoch, 
which is operating system specific, and wraps around to 0 again whenever the value overflows four bytes 
(roughly 497 days).
\end{quotation}
\end{framed}(\RU{Из документации \oracle}\EN{From \oracle documentation}
\footnote{\url{http://go.yurichev.com/17088}})

\RU{Интересно что периоды разные в Oracle для Win32 и для Linux. Сможем ли мы найти функцию, отвечающую 
за генерирование этого значения?}
\EN{It is interesting that the periods are different for Oracle for win32 and for Linux. 
Will we be able to find the function that generates this value?}

\RU{Как видно, эта информация, в итоге, берется из системной таблицы \TT{X\$KSUTM}.}\EN{As we can see, 
this information is finally taken from the \TT{X\$KSUTM} table.}

\begin{lstlisting}
SQL> select * from V$FIXED_VIEW_DEFINITION where view_name='V$TIMER';

VIEW_NAME
------------------------------
VIEW_DEFINITION
--------------------------------------------------------------------------------

V$TIMER
select  HSECS from GV$TIMER where inst_id = USERENV('Instance')

SQL> select * from V$FIXED_VIEW_DEFINITION where view_name='GV$TIMER';

VIEW_NAME
------------------------------
VIEW_DEFINITION
--------------------------------------------------------------------------------

GV$TIMER
select inst_id,ksutmtim from x$ksutm
\end{lstlisting}

\RU{Здесь мы упираемся в небольшую проблему, в таблицах \TT{kqftab}/\TT{kqftap} нет указателей на функцию, 
которая бы генерировала значение}
\EN{Now we are stuck in a small problem, there are no references to value generating function(s) 
in the tables \TT{kqftab}/\TT{kqftap}}:

\begin{lstlisting}[caption=\RU{Результат работы}\EN{Result of} \OracleTablesName]
kqftab_element.name: [X$KSUTM] ?: [ksutm] 0x1 0x4 0x4 0x0 0xffffc09b 0x3
kqftap_param.name=[ADDR] ?: 0x10917 0x0 0x0 0x0 0x4 0x0 0x0
kqftap_param.name=[INDX] ?: 0x20b02 0x0 0x0 0x0 0x4 0x0 0x0
kqftap_param.name=[INST_ID] ?: 0xb02 0x0 0x0 0x0 0x4 0x0 0x0
kqftap_param.name=[KSUTMTIM] ?: 0x1302 0x0 0x0 0x0 0x4 0x0 0x1e
kqftap_element.fn1=NULL
kqftap_element.fn2=NULL
\end{lstlisting}

\RU{Попробуем в таком случае просто поискать строку \TT{KSUTMTIM}, и находим ссылку на нее в такой функции:}
\EN{When we try to find the string \TT{KSUTMTIM}, we see it in this function:}

\begin{lstlisting}
kqfd_DRN_ksutm_c proc near              ; DATA XREF: .rodata:0805B4E8

arg_0           = dword ptr  8
arg_8           = dword ptr  10h
arg_C           = dword ptr  14h

                push    ebp
                mov     ebp, esp
                push    [ebp+arg_C]
                push    offset ksugtm
                push    offset _2__STRING_1263_0 ; "KSUTMTIM"
                push    [ebp+arg_8]
                push    [ebp+arg_0]
                call    kqfd_cfui_drain
                add     esp, 14h
                mov     esp, ebp
                pop     ebp
                retn
kqfd_DRN_ksutm_c endp
\end{lstlisting}

\RU{Сама функция}\EN{The} \TT{kqfd\_DRN\_ksutm\_c()} \RU{упоминается в таблице}\EN{function is mentioned in the} 
\TT{kqfd\_tab\_registry\_0} \RU{вот так}\EN{table}:

\begin{lstlisting}
dd offset _2__STRING_62_0 ; "X$KSUTM"
dd offset kqfd_OPN_ksutm_c
dd offset kqfd_tabl_fetch
dd 0
dd 0
dd offset kqfd_DRN_ksutm_c
\end{lstlisting}

\RU{Упоминается также некая функция \TT{ksugtm()}}\EN{There is a function \TT{ksugtm()} referenced here}.
\RU{Посмотрим, что там (в Linux x86)}\EN{Let's see what's in it (Linux x86)}:

\begin{lstlisting}[caption=ksu.o]
ksugtm          proc near

var_1C          = byte ptr -1Ch
arg_4           = dword ptr  0Ch

                push    ebp
                mov     ebp, esp
                sub     esp, 1Ch
                lea     eax, [ebp+var_1C]
                push    eax
                call    slgcs
                pop     ecx
                mov     edx, [ebp+arg_4]
                mov     [edx], eax
                mov     eax, 4
                mov     esp, ebp
                pop     ebp
                retn
ksugtm          endp
\end{lstlisting}

\RU{В win32-версии тоже самое}\EN{The code in the win32 version is almost the same}.

\RU{Искомая ли эта функция? Попробуем узнать}\EN{Is this the function we are looking for? Let's see}:
\index{tracer}

\begin{lstlisting}
tracer -a:oracle.exe bpf=oracle.exe!_ksugtm,args:2,dump_args:0x4
\end{lstlisting}

\RU{Пробуем несколько раз}\EN{Let's try again}:

\begin{lstlisting}
SQL> select * from V$TIMER;

     HSECS
----------
  27294929

SQL> select * from V$TIMER;

     HSECS
----------
  27295006

SQL> select * from V$TIMER;

     HSECS
----------
  27295167
\end{lstlisting}

\begin{lstlisting}[caption=\RU{вывод \tracer}\EN{\tracer output}]
TID=2428|(0) oracle.exe!_ksugtm (0x0, 0xd76c5f0) (called from oracle.exe!__VInfreq__qerfxFetch+0xfad (0x56bb6d5))
Argument 2/2
0D76C5F0: 38 C9                                           "8.              "
TID=2428|(0) oracle.exe!_ksugtm () -> 0x4 (0x4)
Argument 2/2 difference
00000000: D1 7C A0 01                                     ".|..            "
TID=2428|(0) oracle.exe!_ksugtm (0x0, 0xd76c5f0) (called from oracle.exe!__VInfreq__qerfxFetch+0xfad (0x56bb6d5))
Argument 2/2
0D76C5F0: 38 C9                                           "8.              "
TID=2428|(0) oracle.exe!_ksugtm () -> 0x4 (0x4)
Argument 2/2 difference
00000000: 1E 7D A0 01                                     ".}..            "
TID=2428|(0) oracle.exe!_ksugtm (0x0, 0xd76c5f0) (called from oracle.exe!__VInfreq__qerfxFetch+0xfad (0x56bb6d5))
Argument 2/2
0D76C5F0: 38 C9                                           "8.              "
TID=2428|(0) oracle.exe!_ksugtm () -> 0x4 (0x4)
Argument 2/2 difference
00000000: BF 7D A0 01                                     ".}..            "
\end{lstlisting}

\RU{Действительно\EMDASH{}значение то, что мы видим в SQL*Plus, и оно возвращается через второй аргумент}
\EN{Indeed\EMDASH{}the value is the same we see in SQL*Plus and it is returned via the second argument}.

\RU{Посмотрим, что в функции}\EN{Let's see what is in} \TT{slgcs()} (Linux x86):

\begin{lstlisting}
slgcs           proc near

var_4           = dword ptr -4
arg_0           = dword ptr  8

                push    ebp
                mov     ebp, esp
                push    esi
                mov     [ebp+var_4], ebx
                mov     eax, [ebp+arg_0]
                call    $+5
                pop     ebx
                nop                     ; PIC mode
                mov     ebx, offset _GLOBAL_OFFSET_TABLE_
                mov     dword ptr [eax], 0
                call    sltrgatime64    ; PIC mode
                push    0
                push    0Ah
                push    edx
                push    eax
                call    __udivdi3       ; PIC mode
                mov     ebx, [ebp+var_4]
                add     esp, 10h
                mov     esp, ebp
                pop     ebp
                retn
slgcs           endp
\end{lstlisting}

(\RU{это просто вызов}\EN{it is just a call to} \TT{sltrgatime64()} \RU{и деление его результата на}
\EN{and division of its result by} 10~(\myref{sec:divisionbynine}))

\RU{И в win32-версии}\EN{And win32-version}:

\begin{lstlisting}
_slgcs          proc near               ; CODE XREF: _dbgefgHtElResetCount+15
                                        ; _dbgerRunActions+1528
                db      66h
                nop
                push    ebp
                mov     ebp, esp
                mov     eax, [ebp+8]
                mov     dword ptr [eax], 0
                call    ds:__imp__GetTickCount@0 ; GetTickCount()
                mov     edx, eax
                mov     eax, 0CCCCCCCDh
                mul     edx
                shr     edx, 3
                mov     eax, edx
                mov     esp, ebp
                pop     ebp
                retn
_slgcs          endp
\end{lstlisting}

\RU{Это просто результат}\EN{It is just the result of} \TT{GetTickCount()
\footnote{\href{http://go.yurichev.com/17248}{MSDN}}} 
\RU{поделенный на}\EN{divided by} 10~(\myref{sec:divisionbynine}).

\RU{Вуаля! Вот почему в win32-версии и версии Linux x86 разные результаты, потому что они получаются разными 
системными функциями \ac{OS}.}\EN{Voilà! That's why the win32 version and the Linux x86 version show different results, 
because they are generated by different \ac{OS} functions.}

\RU{\IT{Drain} по-английски \IT{дренаж, отток, водосток}. Таким образом, возможно имеется ввиду \IT{подключение} 
определенного столбца системной таблице к функции.}
\EN{\IT{Drain} apparently implies \IT{connecting} a specific table column to a specific function.}

\RU{Добавим поддержку таблицы}\EN{We will add support of the table} \TT{kqfd\_tab\_registry\_0} \RU{в}\EN{to} \oracletables, 
\RU{теперь мы можем видеть, при помощи каких функций, столбцы в системных таблицах \IT{подключаются} к значениям, 
например}\EN{now we can see how the table column's variables are \IT{connected} to a specific functions}:

\begin{lstlisting}
[X$KSUTM] [kqfd_OPN_ksutm_c] [kqfd_tabl_fetch] [NULL] [NULL] [kqfd_DRN_ksutm_c]
[X$KSUSGIF] [kqfd_OPN_ksusg_c] [kqfd_tabl_fetch] [NULL] [NULL] [kqfd_DRN_ksusg_c]
\end{lstlisting}

\IT{OPN}, \RU{возможно}\EN{apparently stands for}, \IT{open}, \RU{а}\EN{and} \IT{DRN}, \RU{вероятно, означает}
\EN{apparently, for} \IT{drain}.


\fi

\ifdefined\IncludeMSDOS
\chapter{\RU{Вручную написанный на ассемблере код}\EN{Handwritten assembly code}}

\section{\RU{Тестовый файл} EICAR\EN{ test file}}
\label{subsec:EICAR}

\index{MS-DOS}
\index{EICAR}
\RU{Этот .COM-файл предназначен для тестирования антивирусов, его можно запустить в MS-DOS
и он выведет такую строку}\EN{This .COM-file is intended for testing antivirus software, it is possible to run in
in MS-DOS and it prints this string}: \q{EICAR-STANDARD-ANTIVIRUS-TEST-FILE!}
\footnote{\href{\RU{http://go.yurichev.com/17005}\EN{http://go.yurichev.com/17006}}{wikipedia}}.
% FIXME1 \myref{} -> about .COM files

\RU{Он примечателен тем, что он полностью состоит только из печатных ASCII-символов, следовательно, его можно
набрать в любом текстовом редакторе}\EN{Its important property is that it's consists entirely of printable 
ASCII-symbols, which, in turn, makes it possible to create it in any text editor}:

\begin{lstlisting}
X5O!P%@AP[4\PZX54(P^)7CC)7}$EICAR-STANDARD-ANTIVIRUS-TEST-FILE!$H+H*
\end{lstlisting}

\RU{Попробуем его разобрать}\EN{Let's decompile it}:

\lstinputlisting{examples/handcoding/EICAR.lst.\LANG}

\RU{Добавим везде комментарии, показывающие состояние регистров и стека после каждой инструкции}%
\EN{We will add comments about the registers and stack after each instruction}.

\RU{Собственно, все эти инструкции нужны только для того чтобы исполнить следующий код}\EN{Essentially, all these
instructions are here only to execute this code}:

\begin{lstlisting}
B4 09     MOV AH, 9
BA 1C 01  MOV DX, 11Ch
CD 21     INT 21h
CD 20     INT 20h
\end{lstlisting}

\index{x86!\Instructions!INT}
\TT{INT 21h} \RU{с функцией 9 (переданной в \TT{AH}) просто выводит строку, адрес которой передан в}\EN{with 9th
function (passed in \TT{AH}) just prints a string, the address of which is passed in} \TT{DS:DX}.
\RU{Кстати, строка должна быть завершена символом '\$'}\EN{By the way, the string has to be terminated
with the '\$' sign}.
\RU{Вероятно, это наследие}\EN{Apparently, it's inherited from} \gls{CP/M} 
\RU{и эта функция в DOS осталась для совместимости}\EN{and this function was left in DOS for compatibility}.
\TT{INT 20h} \RU{возвращает управление в}\EN{exits to} DOS.

\RU{Но, как видно, далеко не все опкоды этих инструкций печатные}\EN{But as we can see, these instruction's
opcodes are not strictly printable}.
\RU{Так что основная часть EICAR-файла это}\EN{So the main part of EICAR file is}:

\begin{itemize}
\item \RU{подготовка нужных значений регистров (AH и DX)}\EN{preparing the register (AH and DX) values that we need};
\item \RU{подготовка в памяти опкодов для INT 21 и INT 20}\EN{preparing INT 21 and INT 20 opcodes in memory};
\item \RU{исполнение}\EN{executing} INT 21 \AndENRU INT 20.
\end{itemize}

\index{Shellcode}
\RU{Кстати, подобная техника широко используется для создания шеллкодов, 
где нужно создать x86-код, который будет нужно передать в виде текстовой строки}
\EN{By the way, this technique is widely used in shellcode construction, when one need to pass x86 code
in string form}.

\RU{Здесь также список всех x86-инструкций с печатаемыми опкодоами}\EN{Here is also a list of all 
x86 instructions which have printable opcodes}: \myref{printable_x86_opcodes}.
\fi

\mysection{\RU{Демо}\EN{Demos}}

\RU{Демо (или демомейкинг) были великолепным упражнением в математике, программировании компьютерной графики
и очень плотному программированию на ассемблере вручную}\EN{Demos (or demomaking) were an excellent 
exercise in mathematics, computer graphics programming and very tight x86 hand coding}.

% sections
\EN{\subsection{10 PRINT CHR\$(205.5+RND(1)); : GOTO 10}

All examples here are MS-DOS .COM files.
%FIXME1 -> about .COM files

\myindex{MS-DOS}
In [Nick Montfort et al, \IT{10 PRINT CHR\$(205.5+RND(1)); : GOTO 10}, (The MIT Press:2012)]
\footnote{\AlsoAvailableAs \url{http://go.yurichev.com/17286}}

we can read about one of the most simple possible random maze generators.

It just prints a slash or backslash characters randomly and endlessly, resulting in something like this:

\begin{figure}[H]
\centering
\includegraphics[width=0.6\textwidth]{examples/demos/10print/10print.png}
\end{figure}

There are a few known implementations for 16-bit x86.

\subsubsection{Trixter's 42 byte version}

\newcommand{\FNURLTRIXTER}{\footnote{\url{http://go.yurichev.com/17305}}}

The listing was taken from his website\FNURLTRIXTER, 
but the comments are mine.

\lstinputlisting[style=customasmx86]{examples/demos/10print/10print_42_EN.lst}

\myindex{Intel!8253}
The pseudo-random value here is in fact the time 
that has passed from the system's boot, taken from the 8253 time chip, the value increases by one 18.2 times per second.

By writing zero to port \TT{43h}, 
we send the command \q{select counter 0}, 
"counter latch", 
"binary counter" (not a \ac{BCD} value).

\myindex{x86!\Instructions!POPF}
The interrupts are enabled back with the \TT{POPF} instruction, which restores the \TT{IF} flag as well.

\myindex{x86!\Instructions!IN}
It is not possible to 
use the \TT{IN} instruction with registers other than \TT{AL}, 
hence the shuffling.

\subsubsection{
My attempt to reduce Trixter's version: 27 bytes}

We can say that since we use the timer not 
to get a precise time value, but a pseudo-random one, we do not need
to spend time (and code) to disable the interrupts.

Another thing we can say is that we need only one bit from the low 8-bit part, so let's read only it.

We can reduced the code slightly and we've got 27 bytes:

\lstinputlisting[style=customasmx86]{examples/demos/10print/10print_27_EN.lst}

\subsubsection{
Taking random memory garbage as a source of randomness}

Since it is MS-DOS, there is no memory protection at all, we can read from whatever address we want.
\myindex{x86!\Instructions!LODSB}
Even more than that: a simple \TT{LODSB} 
instruction reads a byte from the \TT{DS:SI} address, but it's not a problem
if the registers' values are not set up, let it read 1) random bytes; 2) from a random place in memory!

It is suggested in Trixter's webpage\FNURLTRIXTER 
to use \TT{LODSB} without any setup.

\myindex{x86!\Instructions!SCASB}
It is also suggested that the \TT{SCASB} 

instruction can be used instead, because it sets a flag according to the byte it reads.


Another idea to minimize the code is to use the \TT{INT 29h} DOS syscall, which just prints the character stored in the \TT{AL} register.

That is what Peter Ferrie and \HERMIT{} did (11 and 10 bytes)
\footnote{\url{http://go.yurichev.com/17087}}:

\lstinputlisting[caption=\HERMIT: 11 bytes,style=customasmx86]{examples/demos/10print/herm1t_11_EN.lst}

\myindex{x86!\Instructions!SCASB}
\TT{SCASB} also uses the value in the \TT{AL}
 register, it subtract a random memory byte's value from the \TT{5Ch} value in \TT{AL}.
\myindex{x86!\Instructions!JP}
\TT{JP} is a rare instruction, here it used for checking the parity flag (PF), 
which is generated by the formulae in the listing.
As a consequence, the output character 
is determined not by some bit in a random memory byte, but by a sum of bits, 
this (hopefully) makes the result more distributed.

\myindex{x86!\Instructions!SALC}
\myindex{x86!\Instructions!SETALC}
\myindex{NEC V20}
It is possible to 
make this even shorter by using the undocumented x86 instruction \TT{SALC} (\ac{AKA} \TT{SETALC}) (\q{Set AL CF}).
It was introduced in the NEC V20 \ac{CPU} and sets \TT{AL} to 
\TT{0xFF} if \TT{CF} is 1 or to 0 if otherwise.

\lstinputlisting[caption=Peter Ferrie: 10 bytes,style=customasmx86]{examples/demos/10print/ferrie_10_EN.lst}

So it is possible to get rid of conditional jumps at all.
The \ac{ASCII} code of backslash (\q{\textbackslash{}}) 
is \TT{0x5C} and \TT{0x2F} for slash (\q{/}).
So we have to convert one (pseudo-random) bit in the \TT{CF} flag to a value of \TT{0x5C} or \TT{0x2F}.

This is done easily: by \TT{AND}-ing all bits in \TT{AL} (where all 8 bits are set or cleared) with \TT{0x2D} we have just 0 or \TT{0x2D}.

By adding \TT{0x2F} to this value, we get \TT{0x5C} or \TT{0x2F}.

Then we just output it to the screen.

\subsubsection{\Conclusion{}}

\myindex{DOSBox}
It is also worth mentioning that the result may 
be different in DOSBox, \gls{Windows NT} and even MS-DOS, 

due to different
conditions: the timer chip can be emulated differently and the initial register contents may be different as well.
}
\RU{\subsection{10 PRINT CHR\$(205.5+RND(1)); : GOTO 10}

Все примеры здесь для .COM-файлов под MS-DOS.
%FIXME1 -> about .COM files

\myindex{MS-DOS}
В [Nick Montfort et al, \IT{10 PRINT CHR\$(205.5+RND(1)); : GOTO 10}, (The MIT Press:2012)]
\footnote{\AlsoAvailableAs \url{http://go.yurichev.com/17286}}
можно прочитать об одном из простейших генераторов случайных лабиринтов.
Он просто бесконечно и случайно печатает символ слэша или обратный слэша, выдавая в итоге что-то вроде:

\begin{figure}[H]
\centering
\includegraphics[width=0.6\textwidth]{examples/demos/10print/10print.png}
\end{figure}

Здесь несколько известных реализаций для 16-битного x86.

\subsubsection{Версия 42-х байт от Trixter}

\newcommand{\FNURLTRIXTER}{\footnote{\url{http://go.yurichev.com/17305}}}

Листинг взят с его сайта\FNURLTRIXTER, но комментарии --- автора.

\lstinputlisting[style=customasmx86]{examples/demos/10print/10print_42_RU.lst}

\myindex{Intel!8253}
Псевдослучайное число на самом деле это время, прошедшее со старта системы, получаемое из чипа таймера 8253, 
это значение
увеличивается на единицу 18.2 раза в секунду.

Записывая ноль в порт \TT{43h}, 
мы имеем ввиду что команда это \q{выбрать счетчик 0}, 
"counter latch", 
"двоичный счетчик" (а не значение \ac{BCD}).

\myindex{x86!\Instructions!POPF}
Прерывания снова разрешаются при помощи инструкции \TT{POPF}, которая
также возвращает флаг \TT{IF}.

\myindex{x86!\Instructions!IN}
Инструкцию \TT{IN} нельзя использовать с другими регистрами кроме \TT{AL}, поэтому здесь перетасовка.

\subsubsection{Моя попытка укоротить версию Trixter: 27 байт}

Мы можем сказать, что мы используем таймер не для того чтобы получить точное время, но псевдослучайное число,
так что мы можем не тратить время (и код) на запрещение прерываний.
Еще можно сказать, что так как мы берем бит из младшей 8-битной части, то мы можем считывать только её.

Немного укоротим код и выходит 27 байт:

\lstinputlisting[style=customasmx86]{examples/demos/10print/10print_27_RU.lst}

\subsubsection{Использование случайного мусора в памяти как источника случайных чисел}

Так как это MS-DOS, защиты памяти здесь нет вовсе, так что мы можем читать с какого
угодно адреса.
\myindex{x86!\Instructions!LODSB}
И даже более того: простая инструкция \TT{LODSB} 
будет читать байт по адресу \TT{DS:SI}, но это не проблема
если правильные значения не установлены в регистры, пусть она читает 1) случайные байты; 2) из случайного
места в памяти!

Так что на странице Trixter-а\FNURLTRIXTER 
можно найти предложение использовать \TT{LODSB} без всякой инициализации.

\myindex{x86!\Instructions!SCASB}
Есть также предложение использовать инструкцию \TT{SCASB} 
вместо, потому что она выставляет флаги в соответствии с прочитанным значением.

Еще одна идея насчет минимизации кода --- это использовать прерывание DOS
 \TT{INT 29h} которое просто печатает символ на экране
из регистра \TT{AL}.

Это то что сделали Peter Ferrie и \HERMIT{} (11 и 10 байт)
\footnote{\url{http://go.yurichev.com/17087}}:

\lstinputlisting[caption=\HERMIT: 11 байт,style=customasmx86]{examples/demos/10print/herm1t_11_RU.lst}

\myindex{x86!\Instructions!SCASB}
\TT{SCASB} также использует значение в регистре \TT{AL}, она вычитает значение
случайного байта в памяти из
 \TT{5Ch} в \TT{AL}.
\myindex{x86!\Instructions!JP}
\TT{JP} это редкая инструкция, здесь она используется для проверки флага четности (PF),
который вычисляется по формуле в листинге.
Как следствие, выводимый символ определяется не каким-то конкретным битом из случайного байта в памяти,
а суммой бит, и это (надеемся) сделает результат более распределенным.

\myindex{x86!\Instructions!SALC}
\myindex{x86!\Instructions!SETALC}
\myindex{NEC V20}
Можно сделать еще короче, если использовать недокументированную x86-инструкцию \TT{SALC} (\ac{AKA} \TT{SETALC}) (\q{Set AL CF}).
Она появилась в \ac{CPU} и выставляет \TT{AL} в 
\TT{0xFF} если \TT{CF} это 1 или 0 если наоборот.

\lstinputlisting[caption=Peter Ferrie: 10 байт,style=customasmx86]{examples/demos/10print/ferrie_10_RU.lst}

Так что можно избавиться и от условных переходов.
\ac{ASCII}-код обратного слэша (\q{\textbackslash{}}) 
это \TT{0x5C} и \TT{0x2F} для слэша (\q{/}).

Так что нам нужно конвертировать один (псевдослучайный) бит из флага \TT{CF} в значение \TT{0x5C} или \TT{0x2F}.

%
Это делается легко: применяя операцию \q{И} ко всем битам в \TT{AL} (где все 8 бит либо выставлены, либо сброшены) с \TT{0x2D} мы имеем просто 0 или \TT{0x2D}.

%
Прибавляя значение \TT{0x2F} к этому значению, мы получаем \TT{0x5C} или \TT{0x2F}.
И просто выводим это на экран.

\subsubsection{\Conclusion{}}

\myindex{DOSBox}
Также стоит отметить, что результат может быть разным в эмуляторе DOSBox, \gls{Windows NT} и даже MS-DOS, 
из-за разных условий:
чип таймера может эмулироваться по-разному, изначальные значения регистров также могут быть разными.
}
\EN{\clearpage
\subsection{Mandelbrot set}
\label{Mandelbrot_demo}

\epigraph{You know, if you magnify the coastline, it still looks like
a coastline, and a lot of other things have this property. Nature has
recursive algorithms that it uses to generate clouds and Swiss cheese
and things like that.}
{Donald Knuth, interview (1993)}

Mandelbrot set is a fractal, which exhibits self-similarity.

When you increase scale, you see that this characteristic pattern repeating infinitely.

Here is a demo\footnote{Download it \href{http://go.yurichev.com/17306} {here},} 
written by \q{Sir\_Lagsalot} in 2009, that draws 
the Mandelbrot set, which is just a x86 program with executable file size of only 64 bytes.
There are only 30 16-bit x86 instructions.

Here it is what it draws:

\begin{figure}[H]
\centering
\myincludegraphics{examples/demos/mandelbrot/1.png}
\end{figure}

Let's try to understand how it works.

\subsubsection{Theory}

\myparagraph{A word about complex numbers}

A complex number is a number that consists of two parts---real (Re) and imaginary (Im).


The complex plane is a two-dimensional plane where any complex number can be placed: the real part is one coordinate
and the imaginary part is the other.

Some basic rules we have to keep in mind:

\begin{itemize}
\item Addition: $(a+bi) + (c+di) = (a+c) + (b+d)i$

In other words:

$\operatorname{Re}(sum) = \operatorname{Re}(a) + \operatorname{Re}(b)$

$\operatorname{Im}(sum) = \operatorname{Im}(a) + \operatorname{Im}(b)$

\item Multiplication: $(a+bi) (c+di) = (ac-bd) + (bc+ad)i$

In other words:

$\operatorname{Re}(product) = \operatorname{Re}(a) \cdot \operatorname{Re}(c) - \operatorname{Re}(b) \cdot \operatorname{Re}(d)$

$\operatorname{Im}(product) = \operatorname{Im}(b) \cdot \operatorname{Im}(c) + \operatorname{Im}(a) \cdot \operatorname{Im}(d)$

\item Square: $(a+bi)^2 = (a+bi) (a+bi) = (a^2-b^2) + (2ab)i$

In other words:

$\operatorname{Re}(square) = \operatorname{Re}(a)^2-\operatorname{Im}(a)^2$

$\operatorname{Im}(square) = 2 \cdot \operatorname{Re}(a) \cdot \operatorname{Im}(a)$

\end{itemize}

\myparagraph{How to draw the Mandelbrot set}

The Mandelbrot set is a set of points for which the $z_{n+1} = {z_n}^2 + c$ recursive sequence
(where $z$ and $c$ are complex numbers and $c$ 
is the starting value)
does not approach infinity.\\
\\
In plain English language: 

\begin{itemize}
\item Enumerate all points on screen. 
\item Check if the specific point 
is in the Mandelbrot set.
\item Here is how to check it:

  \begin{itemize}
  \item Represent the point as a complex number.
  \item Calculate the square of it.
  \item Add the starting value of the point to it.
  \item Does it go off limits? If yes, break.
  \item Move the point to the 
new place at the coordinates we just calculated.
  \item Repeat all this for some reasonable 
number of iterations.
  \end{itemize}

\item The point is still in limits?
Then draw the point.

\item The point has eventually gone off limits?

  \begin{itemize}
    \item (For a black-white image) do not draw anything.
    \item 

(For a colored image) transform the number of iterations to some color. 
      So the color shows the speed with which point has gone off limits.
  \end{itemize}

\end{itemize}

%
Here is Pythonesque algorithm for both complex and integer number representations:

\lstinputlisting[caption=For complex numbers]{examples/demos/mandelbrot/algo_cplx_EN.lst}


The integer version is where the operations on complex numbers are replaced with integer operations according to the rules
which were explained above.

\lstinputlisting[caption=For integer numbers]{examples/demos/mandelbrot/algo_int_EN.lst}

Here is also a C\# source 
which is present in the Wikipedia article\footnote{\href{http://go.yurichev.com/17307}{wikipedia}}, but we'll modify it
so it will print the iteration numbers instead of some symbol
\footnote{Here is also the executable file: 
\href{http://go.yurichev.com/17163}{beginners.re}}:

\lstinputlisting[style=customc]{examples/demos/mandelbrot/dump_iterations.cs}

Here is the resulting file, 
which is too wide to be included here: \\
\href{http://go.yurichev.com/17164}{beginners.re}.

The maximal number of iterations is 40, so when you see 40 in this dump, it means that this point has been wandering
for 40 iterations but never got off limits. 

A number $n$ less than 40 means that point remained inside the bounds only for $n$ iterations, 
then it went outside them.

\clearpage
There is a cool demo available at 
\url{http://go.yurichev.com/17309}, which shows
visually how the point moves on the plane at each iteration for some specific point. 
Here are two screenshots.

%
First, we've clicked inside the yellow area and saw that the trajectory (green line)
eventually swirls at some point inside:

\begin{figure}[H]
\centering
\includegraphics[width=0.7\textwidth]{examples/demos/mandelbrot/demo1.png}
\caption{Click inside yellow area}
\end{figure}


This implies that the point we've clicked belongs to the Mandelbrot set.

\clearpage

Then we've clicked outside the yellow area and saw a much more chaotic point movement, 
which quickly went off bounds:

\begin{figure}[H]
\centering
\includegraphics[width=0.7\textwidth]{examples/demos/mandelbrot/demo2.png}
\caption{Click outside yellow area}
\end{figure}

This means the point doesn't belong to Mandelbrot set.

Another good demo is available here: 
\url{http://go.yurichev.com/17310}.

\clearpage
\subsubsection{Let's get back to the demo}


The demo, although very tiny (just 64 bytes or 30 instructions), implements the common algorithm 
described here, but using some coding tricks.

%
The source code is easily downloadable, so here is it, but let's also add comments:

\lstinputlisting[caption=Commented source code,numbers=left,style=customasmx86]{examples/demos/mandelbrot/Microbrot_commented_EN.asm}

Algorithm:

\begin{itemize}
\item Switch to 320*200 VGA video mode, 256 colors. 
$320*200=64000$ (0xFA00). 

Each pixel is encoded by one byte, so the buffer size is 0xFA00 bytes.
It is addressed using the ES:DI registers pair.

\myindex{x86!\Registers!ES}
ES must be 0xA000 here, because this is the segment address of 
the VGA video buffer, but storing 0xA000 to ES requires at least 4 bytes (\TT{PUSH 0A000h / POP ES}). 
You can read more about the 16-bit MS-DOS memory model here: 
\myref{8086_memory_model}.

\myindex{x86!\Instructions!LES}

Assuming that BX is zero here, and the Program Segment Prefix is at the zeroth
address, the 2-byte \TT{LES AX,[BX]} instruction stores 0x20CD to AX and 0x9FFF to ES.

So the program starts to draw 16 pixels (or bytes) before the actual video buffer.
But this is MS-DOS, 

there is no memory protection, so a write happens into the very end of conventional memory, and usually, there is nothing important.
That's why you see a red strip 16 
pixels wide at the right side.
The whole picture is shifted left by 16 pixels.
This is the price of saving 2 bytes.

\item An infinite loop processes each pixel.

Probably, the most common way to enumerate all pixels on the screen is with two loops: 
one for the X coordinate, another for the Y coordinate.
But then you'll need 
to multiply the coordinates to address a byte in the VGA video buffer.

The author of this demo decided to do it otherwise: enumerate all bytes in the video buffer by using one single loop instead 
of two, and get the coordinates of the current point using division.
The resulting coordinates are: X in the range of $-256..63$ and Y in the range of $-100..99$.
You can see on 
the screenshot that the picture is somewhat shifted to the right part of screen.

That's because the biggest heart-shaped black hole usually appears on coordinates 0,0 and these are shifted
here to right.
Could the author just 
subtract 160 from the value to get X in the range of $-160..159$? 
Yes, but the instruction \TT{SUB DX, 160} takes 4 bytes, 
while \TT{DEC DH}---2 bytes 
(which subtracts 0x100 (256) from DX). 
So the whole picture is shifted for the cost of 
another 2 bytes of saved space.

    \begin{itemize}
    \item Check, if the current 
point is inside the Mandelbrot set.
          The algorithm is the one that has been described here.
\myindex{x86!\Instructions!LOOP}
     \item The loop 
is organized using the \TT{LOOP} instruction, which uses the CX register as counter.

The author could set the number of iterations to some specific number, but he didn't: 320 is already present in CX 
(has been set at line 35), and this is good maximal iteration number anyway.
We save here some space 
by not the reloading CX register with another value.

\myindex{x86!\Instructions!SAR}
     \item 
\TT{IMUL} is used here instead of \TT{MUL}, because we work with signed values: 
keep in mind that the 0,0 coordinates has to be somewhere near the center of the screen.

It's the same with \TT{SAR} (arithmetic shift for signed values): it's used instead of \TT{SHR}.

     \item Another idea is to simplify the bounds check.
We must check a coordinate pair, i.e., two variables.
What the author does is to checks thrice for overflow: two squaring operations and one addition.

Indeed, we use 16-bit registers, which hold signed values in the range of -32768..32767, 
so if any of the coordinates is greater than 32767 during the signed multiplication, this point is definitely out 
of bounds: we jump to the \TT{MandelBreak} label.

     \item 
There is also a division by 64 (SAR instruction). 64 sets scale.

Try to increase the value and you can get a closer look, or to decrease if for a more distant look.

    \end{itemize}

\item We are at the \TT{MandelBreak} label, there are two ways of getting here: 
the loop ended with CX=0 (
the point is inside the Mandelbrot set); or because an overflow has happened (CX still holds some value).
Now we write the low 8-bit part of CX (CL) to the 
video buffer.

The default palette is rough, nevertheless, 0 is black: hence we see black holes in the places where the points are
in the Mandelbrot set.
The palette can be initialized at the program's start, but keep in mind, this is only a 64 bytes program!

\item The program runs in an infinite loop, 
because an additional check where to stop, or any user interface will result in additional instructions.

\end{itemize}

Some other optimization tricks:

\myindex{x86!\Instructions!CWD}
\begin{itemize}
\item The 1-byte CWD is used here 
for clearing DX instead of the 2-byte \TT{XOR DX, DX} or even the 3-byte \TT{MOV DX, 0}.

\item The 1-byte \TT{XCHG AX, CX} is used instead of the 2-byte 
\TT{MOV AX,CX}. 
The current value of AX is not needed here anyway.

\item DI (position in video buffer) is not initialized, and it is 0xFFFE at the start
\footnote{More information about initial register values: 
\url{http://go.yurichev.com/17004}}.

That's OK, because the program works for all DI in the range of 0..0xFFFF eternally, 
and the user can't notice
that it is started off the screen (the last pixel of a 320*200 video buffer is at address 0xF9FF).
So some work is actually done 
off the limits of the screen.

Otherwise, you'll need an additional instructions to set DI to 0 and check for the video buffer's end.

\end{itemize}

\newcommand{\MyFixedVersion}{My \q{fixed} version}
\subsubsection{\MyFixedVersion}

\lstinputlisting[caption=\MyFixedVersion,numbers=left,style=customasmx86]{examples/demos/mandelbrot/my_version_EN.asm}


The author of these lines made an attempt to fix all these oddities: now the palette is smooth grayscale, the video buffer is at the correct place 
(lines 19..20),
the picture is drawn on center of the screen (line 30), the program eventually ends and waits for the user's keypress 
(lines 58..68).

But now it's much bigger: 105 bytes (or 54 instructions)
\footnote{
You can experiment by yourself: get DosBox and NASM and compile it as: 
\TT{nasm fiole.asm -fbin -o file.com}}.

\begin{figure}[H]
\centering
\myincludegraphics{examples/demos/mandelbrot/fixed.png}
\caption{\MyFixedVersion}
\label{fig:mandelbrot_fixed}
\end{figure}
}
\RU{\clearpage
\subsection{Множество Мандельброта}
\label{Mandelbrot_demo}

\epigraph{You know, if you magnify the coastline, it still looks like
a coastline, and a lot of other things have this property. Nature has
recursive algorithms that it uses to generate clouds and Swiss cheese
and things like that.}
{Дональд Кнут, интервью (1993)}

Множество Мандельброта это фрактал, характерное свойство которого это самоподобие.

При увеличении картинки, вы видите, что этот характерный узор повторяется бесконечно.

Вот демо\footnote{Можно скачать \href{http://go.yurichev.com/17306} {здесь},} 
написанное автором по имени \q{Sir\_Lagsalot} в 2009, 
рисующее множество Мандельброта, и это программа для x86 с размером файла всего 64 байта.
Там только 30 16-битных x86-инструкций.

Вот что она рисует:

\begin{figure}[H]
\centering
\myincludegraphics{examples/demos/mandelbrot/1.png}
\end{figure}

Попробуем разобраться, как она работает.

\subsubsection{Теория}

\myparagraph{Немного о комплексных числах}

Комплексное число состоит из двух чисел (вещественная (Re) и мнимая (Im).

Комплексная плоскость --- это двухмерная плоскость, где любое комплексное число может быть расположено:
вещественная часть --- это одна координата и мнимая --- вторая.

Некоторые базовые правила, которые нам понадобятся:

\begin{itemize}
\item Сложение: $(a+bi) + (c+di) = (a+c) + (b+d)i$

Другими словами:

$\operatorname{Re}(sum) = \operatorname{Re}(a) + \operatorname{Re}(b)$

$\operatorname{Im}(sum) = \operatorname{Im}(a) + \operatorname{Im}(b)$

\item Умножение: $(a+bi) (c+di) = (ac-bd) + (bc+ad)i$

Другими словами:

$\operatorname{Re}(product) = \operatorname{Re}(a) \cdot \operatorname{Re}(c) - \operatorname{Re}(b) \cdot \operatorname{Re}(d)$

$\operatorname{Im}(product) = \operatorname{Im}(b) \cdot \operatorname{Im}(c) + \operatorname{Im}(a) \cdot \operatorname{Im}(d)$

\item Возведение в квадрат: $(a+bi)^2 = (a+bi) (a+bi) = (a^2-b^2) + (2ab)i$

Другими словами:

$\operatorname{Re}(square) = \operatorname{Re}(a)^2-\operatorname{Im}(a)^2$

$\operatorname{Im}(square) = 2 \cdot \operatorname{Re}(a) \cdot \operatorname{Im}(a)$

\end{itemize}

\myparagraph{Как нарисовать множество Мандельброта}

Множество Мандельброта --- это набор точек, для которых рекурсивное соотношение
 $z_{n+1} = {z_n}^2 + c$ 
(где $z$ и $c$ это комплексные числа и $c$ это начальное значение) не стремится к бесконечности.\\
\\
Простым русским языком: 

\begin{itemize}
\item Перечисляем все точки на экране. 
\item Проверяем, является ли эта точка в множестве Мандельброта.
\item Вот как проверить:

  \begin{itemize}
  \item Представим точку как комплексное число.
  \item Возведем в квадрат.
  \item Прибавим значение точки в самом начале.
  \item Вышло за пределы? Прерываемся, если да.
  \item Передвигаем точку в новое место, координаты которого только что вычислили.
  \item Повторять всё это некое разумное количество итераций.
  \end{itemize}

\item Двигающаяся точка в итоге не вышла за пределы?
Тогда рисуем точку.

\item Двигающаяся точка в итоге вышла за пределы?

  \begin{itemize}
    \item (Для черно-белого изображения) ничего не рисуем.
    \item 
(Для цветного изображения) преобразуем количество итераций в какой-нибудь цвет.
Так что цвет будет показывать, с какой скоростью точка вышла за пределы.

  \end{itemize}

\end{itemize}

Вот алгоритмы для комплексных и обычных целочисленных чисел (на языке, отдаленно напоминающем Python):%


\lstinputlisting[caption=Для комплексных чисел]{examples/demos/mandelbrot/algo_cplx_RU.lst}

Целочисленная версия, это версия где все операции над комплексными числами заменены на операции 
с целочисленными, в соответствии с изложенными ранее правилами.


\lstinputlisting[caption=Для целочисленных чисел]{examples/demos/mandelbrot/algo_int_RU.lst}

Вот также исходный текст на C\#, который есть в статье в Wikipedia\footnote{\href{http://go.yurichev.com/17307}{wikipedia}}, но мы немного изменим его,
чтобы он выдавал количество итераций, вместо некоторого символа
\footnote{Здесь также и исполняемый файл: 
\href{http://go.yurichev.com/17163}{beginners.re}}:

\lstinputlisting[style=customc]{examples/demos/mandelbrot/dump_iterations.cs}

Вот файл с результатом, который слишком широкий, чтобы привести его здесь: \\
\href{http://go.yurichev.com/17164}{beginners.re}.

Максимальное число итераций 40, так что если вы видите 40 в этом файле, это означает, что точка ходила
40 итераций, но так и не вышла за пределы.
 
Номер $n$ меньше 40 означает, что эта точка оставалась внутри пределов только $n$ итераций, и затем
вышла наружу.


\clearpage
Вот здесь есть неплохая демонстрация: 
\url{http://go.yurichev.com/17309}, она показывает визуально,
как определенная точка двигается по плоскости на каждой итерации. 
Вот два скриншота.

В начале кликаем внутри желтой области, и увидим траекторию (зеленые линии), которая в итоге
закручивается в какой-то точке внутри:%


\begin{figure}[H]
\centering
\includegraphics[width=0.7\textwidth]{examples/demos/mandelbrot/demo1.png}
\caption{Клик внутри желтой области}
\end{figure}

Это значит, что точка на которой кликнули, находится внутри множества Мандельброта.


\clearpage
Затем кликаем снаружи желтой области, и мы видим более хаотичные движения точки, которая быстро выходит
за пределы:


\begin{figure}[H]
\centering
\includegraphics[width=0.7\textwidth]{examples/demos/mandelbrot/demo2.png}
\caption{Клик снаружи желтой области}
\end{figure}

Это значит, что эта точка не принадлежит множеству Мандельброта.

Другая неплохая демонстрация там: 
\url{http://go.yurichev.com/17310}.

\clearpage
\subsubsection{Вернемся к демо}

Демо, хотя и крошечная (только 64 байта или 30 инструкций), реализует общий алгоритм, изложенный
здесь, но с некоторыми трюками.


Исходный код можно скачать, так что вот он, но также снабдим его своими комментариями:

\lstinputlisting[caption=Исходный код с комментариями,numbers=left,style=customasmx86]{examples/demos/mandelbrot/Microbrot_commented_RU.asm}

Алгоритм:

\begin{itemize}
\item Переключаемся в режим VGA 320*200 256 цветов. 
$320*200=64000$ (0xFA00). 
Каждый пиксель кодируется одним байтом, так что размер буфера 0xFA00 байт.

Он адресуется здесь при помощи пары регистров ES:DI.

\myindex{x86!\Registers!ES}
ES должен быть здесь 0xA000, потому что это сегментный адрес видеобуфера, но запись
числа 0xA000 в ES потребует по крайней мере 4 байта (\TT{PUSH 0A000h / POP ES}). 
О 16-битной модели памяти в MS-DOS, читайте больше тут: 
\myref{8086_memory_model}.

\myindex{x86!\Instructions!LES}
Учитывая, что BX здесь 0, и Program Segment Prefix находится по нулевому адресу, 2-байтная инструкция
\TT{LES AX,[BX]} запишет 0x20CD в AX и 0x9FFF в ES.

Так что программа начнет рисовать на 16 пикселей (или байт) перед видеобуфером.

Но это MS-DOS, 
здесь нет защиты памяти, так что запись происходит в самый конец обычной памяти, а там, как правило, ничего важного нет.

Вот почему вы видите красную полосу шириной 16 пикселей справа.
Вся картинка сдвинута налево на 16 пикселей.
Это цена экономии 2-х байт.

\item Вечный цикл, обрабатывающий каждый пиксель.
Наверное, самый общий метод обойти все точки на экране это два цикла:
один для X-координаты, второй для Y-координаты.

Но тогда вам придется перемножать координаты для поиска байта в видеобуфере VGA.
Автор этого демо решил сделать наоборот: перебирать все байты в видеобуфере при помощи одного цикла
вместо двух и затем получать координаты текущей точки при помощи деления.

В итоге координаты такие: X в пределах $-256..63$ и Y 
в пределах $-100..99$.
Вы можете увидеть на скриншоте что картинка как бы сдвинута в правую часть экрана.
Это потому что самая большая черная дыра в форме сердца обычно появляется на координатах 0,0 и они
здесь сдвинуты вправо.

Мог ли автор просто отнять 160 от X, чтобы получилось значение в пределах $-160..159$? 
Да, но инструкция \TT{SUB DX, 160} занимает 4 байта, 
тогда как \TT{DEC DH} --- 2 байта 
(которая отнимает 0x100 (256) от DX). 
Так что картинка сдвинута ценой экономии еще 2-х байт.

    \begin{itemize}
    \item Проверить, является ли текущая точка внутри множества Мандельброта.
          Алгоритм такой же, как и описанный здесь.
\myindex{x86!\Instructions!LOOP}
     \item Цикл организуется инструкцией \TT{LOOP}, которая использует регистр CX как счетчик.
Автор мог бы установить число итераций на какое-то число, но не сделал этого: потому что 320 уже
находится в CX (было установлено на строке 35), и это итак подходящее число как число максимальных
итераций.

Мы здесь экономим немного места, не загружая другое значение в регистр CX.

\myindex{x86!\Instructions!SAR}
     \item Здесь используется \TT{IMUL} вместо \TT{MUL}, потому что мы работаем с знаковыми значениями:
помните, что координаты 0,0 должны быть где-то рядом с центром экрана.

Тоже самое и с \TT{SAR} (арифметический сдвиг для знаковых значений): она используется вместо \TT{SHR}.


     \item Еще одна идея --- это упростить проверку пределов.
Нам бы пришлось проверять пару координат, т.е. две переменных.
Что делает автор это трижды проверяет на переполнение: две операции возведения в квадрат и одно 
прибавление.
Действительно, мы ведь используем 16-битные регистры, содержащие знаковые значения в пределах
 -32768..32767, так что
если любая из координат больше чем 32767 в процессе умножения, точка однозначно вышла за пределы,
и мы переходим на метку \TT{MandelBreak}.


     \item Здесь также имеется деление на 64 (при помощи инструкции SAR). 64 задает масштаб.

Попробуйте увеличить значение и вы получите более увеличенную картинку, или уменьшить для
меньшей.


    \end{itemize}

\item Мы находимся на метке \TT{MandelBreak}, есть только две возможности
попасть сюда: 
цикл закончился с CX=0 (точка внутри множества Мандельброта
); или потому что произошло переполнение (CX все еще содержит 
какое-то значение).
Записываем 8-битную часть CX (CL) в видеобуфер.
Палитра по умолчанию грубая, тем не менее, 0 это черный: поэтому видим черные дыры в местах где точки
внутри множества Мандельброта.

Палитру можно инициализировать в начале программы, но не забывайте, это всего лишь программа на 64 
байта!

\item Программа работает в вечном цикле, потому что дополнительная проверка, где остановится, 
или пользовательский интерфейс, это дополнительные инструкции.

\end{itemize}

Еще оптимизационные трюки:

\myindex{x86!\Instructions!CWD}
\begin{itemize}
\item 1-байтная CWD используется здесь для обнуления DX вместо двухбайтной \TT{XOR DX, DX} или даже трехбайтной \TT{MOV DX, 0}.

\item 1-байтная \TT{XCHG AX, CX} используется вместо двухбайтной 
\TT{MOV AX,CX}. 
Текущее значение в AX все равно уже не нужно.

\item DI (позиция в видеобуфере) не инициализирована, и будет 0xFFFE в
начале
\footnote{Больше о состояниях регистров на старте: 
\url{http://go.yurichev.com/17004}}.
Это нормально, потому что программа работает бесконечно для всех DI в пределах 0..0xFFFF, и пользователь
не может увидеть, что работала началась за экраном (последний пиксель видеобуфера 320*200 имеет адрес 0xF9FF).

Так что некоторая часть работы на самом деле происходит за экраном.
А иначе понадобятся дополнительные инструкции для установки DI в 0; добавить проверку на конец буфера.


\end{itemize}

\newcommand{\MyFixedVersion}{Моя \q{исправленная} версия}
\subsubsection{\MyFixedVersion}

\lstinputlisting[caption=\MyFixedVersion,numbers=left,style=customasmx86]{examples/demos/mandelbrot/my_version_RU.asm}

Автор сих строк попытался исправить все эти странности: теперь палитра плавная черно-белая, видеобуфер на правильном месте
(строки 19..20), картинка рисуется в центре экрана (строка 30), программа в итоге заканчивается и ждет,
пока пользователь нажмет какую-нибудь клавишу (строки 58..68).

Но теперь она намного больше: 105 байт (или 54 инструкции)

\footnote{Можете поэкспериментировать и сами: скачайте DosBox и NASM и компилируйте так:
 
\TT{nasm fiole.asm -fbin -o file.com}}.

\begin{figure}[H]
\centering
\myincludegraphics{examples/demos/mandelbrot/fixed.png}
\caption{\MyFixedVersion}
\label{fig:mandelbrot_fixed}
\end{figure}
}



\include{other}
\include{tasks_answers/tasks_answers}
\part*{\RU{Послесловие}\EN{Afterword}}
\addcontentsline{toc}{part}{\RU{Послесловие}\EN{Afterword}}

\chapter{\RU{Вопросы?}\EN{Questions?}}

\RU{Совершенно по любым вопросам, вы можете не раздумывая писать автору}
\EN{Do not hesitate to mail any questions to the author}: \TT{<\EMAIL>}

\EN{There is also a support forum, you can ask any questions there}
\RU{Есть также форум поддержки, вы можете задавать там абсолютно любые вопросы}:\\
\begin{center}
\url{http://go.yurichev.com/17010}
\end{center}
 
\RU{Пожалуйста, присылайте мне информацию о замеченных ошибках 
(включая грамматические), и т.д.}
\EN{Please, also do not hesitate to send me any corrections 
(including grammar ones (you see how horrible my English is?)), etc.}\\
\\
\RU{Я много работаю над книгой, поэтому номера страниц, листингов, и т.д., очень часто меняются.}
\EN{I'm working on the book a lot, so the page, listings numbers, etc, are changing very rapidly.}
\RU{Пожалуйста, в своих письмах мне не ссылайтесь на номера страниц и листингов.}
\EN{Please, do not refer to page/listing numbers in your emails to me.}
\RU{Есть метод проще: сделайте скриншот страницы, затем в графическом редакторе подчеркните место, где вы видите
ошибку, и отправьте мне. Так я исправлю её намного быстрее.}
\EN{There is much simpler method: just make a screenshot of the page, then underline the place where you see the error in a graphics editor,
and send  me it. I'll fix it much faster in this manner.}
\RU{Ну а если вы знакомы с git и \LaTeX\, вы можете исправить ошибку прямо в исходных текстах:}\EN{And if you familiar with git and \LaTeX\, you can fix the error right in the source code:}\\
\href{http://go.yurichev.com/17089}{GitHub}.

\part*{%
	\RU{Список принятых сокращений}%
	\EN{Acronyms used}%
	\NL{Gebruikte afkortingen}%
	\ES{Acr\'onimos utilizados}%
	\PTBRph{}%
	\DE{Verwendete Abkürzungen}%
	\PLph{}%
	\ITAph{}%
	\FRph{}
}
\addcontentsline{toc}{part}{%
	\RU{Список принятых сокращений}%
	\EN{Acronyms used}%
	\ES{Acr\'onimos utilizados}%
	\NL{Gebruikte afkortingen}%
	\PTBRph{}%
	\DE{Verwendete Abkürzungen}%
	\PLph{}%
	\ITAph{}%
	\FRph{Accronymes utilisés}
}
\begin{acronym}
\RU{
	\acro{OS}[ОС]{Операционная Система}
	\acro{FAQ}[ЧаВО]{Часто задаваемые вопросы}
	\acro{OOP}[ООП]{Объектно-Ориентированное Программирование}
	\acro{PL}[ЯП]{Язык Программирования}
	\acro{PRNG}[ГПСЧ]{Генератор псевдослучайных чисел}
	\acro{ROM}[ПЗУ]{Постоянное запоминающее устройство}
	\acro{ALU}[АЛУ]{Арифметико-логическое устройство}
	\acro{PID}{ID программы/процесса}
	\acro{LF}{Line feed (подача строки) (10 или '\textbackslash{}n' в \CCpp)}
	\acro{CR}{Carriage return (возврат каретки) (13 или '\textbackslash{}r' в \CCpp)}
	\acro{LIFO}{Last In First Out (последним вошел, первым вышел)}
}%
\EN{
	\acro{OS}{Operating System}
	\acro{FAQ}{Frequently Asked Questions}
	\acro{OOP}{Object-Oriented Programming}
	\acro{PL}{Programming language}
	\acro{PRNG}{Pseudorandom number generator}
	\acro{ROM}{Read-only memory}
	\acro{ALU}{Arithmetic logic unit}
	\acro{PID}{Program/process ID}
	\acro{LF}{Line feed (10 or '\textbackslash{}n' in \CCpp)}
	\acro{CR}{Carriage return (13 or '\textbackslash{}r' in \CCpp)}
	\acro{LIFO}{Last In First Out}
}%
\ES{
	\acro{OS}[SO]{\ES{Sistema Operativo}}
	\acro{FAQ}{\ES{Preguntas Frecuentes}}
	\acro{OOP}[POO]{\ES{Programaci\'on Orientada a Objetos}}
	\acro{PL}[LP]{\ES{Lenguaje de Programaci\'on}}
	\acro{PRNG}[GPAN]{\ES{Generador Pseudo-Aleatorio de N\'umeros}}
	\acro{ROM}{\ES{Memoria de Solo Lectura}}
	\acro{ALU}{\ES{Unidad Aritm\'etica L\'ogica}}
}%
\PTBR{
	\acro{OS}{\PTBRph{}}
	\acro{FAQ}{\PTBRph{}}
	\acro{OOP}{\PTBRph{}}
	\acro{PL}{\PTBRph{}}
	\acro{PRNG}{\PTBRph{}}
	\acro{ROM}{\PTBRph{}}
	\acro{ALU}{\PTBRph{}}
}%
\PL{
	\acro{OS}{\PLph{}}
	\acro{FAQ}{\PLph{}}
	\acro{OOP}{\PLph{}}
	\acro{PL}{\PLph{}}
	\acro{PRNG}{\PLph{}}
	\acro{ROM}{\PLph{}}
	\acro{ALU}{\PLph{}}
}%
\DE{
	\acro{OS}[BS]{Betriebssystem}
	\acro{FAQ}{\DEph{}}
	\acro{OOP}{\DEph{}}
	\acro{PL}{\DEph{}}
	\acro{PRNG}{\DEph{}}
	\acro{ROM}{\DEph{}}
	\acro{ALU}{\DEph{}}
}%
\ITA{
	\acro{OS}{\ITAph{}}
	\acro{FAQ}{\ITAph{}}
	\acro{OOP}{\ITAph{}}
	\acro{PL}{\ITAph{}}
	\acro{PRNG}{\ITAph{}}
	\acro{ROM}{\ITAph{}}
	\acro{ALU}{\ITAph{}}
}%
\THA{
	\acro{OS}{\THAph{}}
	\acro{FAQ}{\THAph{}}
	\acro{OOP}{\THAph{}}
	\acro{PL}{\THAph{}}
	\acro{PRNG}{\THAph{}}
	\acro{ROM}{\THAph{}}
	\acro{ALU}{\THAph{}}
}%
\NL{
	\acro{OS}{\NL{Operating System}}
	\acro{FAQ}{\NL{Veelvoorkomende vragen}}
	\acro{OOP}{\NL{Object-Oriented Programmeren}}
	\acro{PL}[PT]{\NL{Programmeertaal}}
	\acro{PRNG}{\NL{Pseudorandom number generator}}
	\acro{ROM}{\NL{Read-only memory}}
	\acro{ALU}{\NL{Arithmetic logic unit}}
}%
\FR{
	\acro{OS}[SE]{Système d'exploitation}
	\acro{FAQ}{Foire Aux Questions}
	\acro{OOP}[POO]{Programmation orientée objet}
	\acro{PL}[LP]{Language de programmation}
	\acro{PRNG}{Nombre généré pseudo-aléatoirement}
	\acro{ROM}{Mémoire morte}
	\acro{ALU}[UAL]{Unité arithmétique et logique}
}%
\acro{RA}{\ReturnAddress}
\acro{PE}{Portable Executable}
\acro{SP}{\gls{stack pointer}. SP/ESP/RSP \InENRU x86/x64. SP \InENRU ARM.}
\acro{DLL}{Dynamic-link library}
\acro{PC}{Program Counter. IP/EIP/RIP \InENRU x86/64. PC \InENRU ARM.}
\acro{LR}{Link Register}
\acro{IDA}{
	\RU{Интерактивный дизассемблер и отладчик, разработан \href{https://hex-rays.com/}{Hex-Rays}}%
	\EN{Interactive Disassembler and debugger developed by \href{https://hex-rays.com/}{Hex-Rays}}%
	\ES{Desensamblador Interactivo y depurador desarrollado por \href{https://hex-rays.com/}{Hex-Rays}}%
	\NL{Interactive Disassembler en debugger ontwikkeld door \href{https://hex-rays.com}{Hex-Rays}}
	\PTBRph{}%
	\PLph{}%
	\DEph{}%
	\ITAph{}%
	\THAph{}%
	\FRph{Désassembleur interactif et débuggueur développé par \href{https://hex-rays.com/}{Hex-Rays}}%
}
\acro{IAT}{Import Address Table}
\acro{INT}{Import Name Table}
\acro{RVA}{Relative Virtual Address}
\acro{VA}{Virtual Address}
\acro{OEP}{Original Entry Point}
\acro{MSVC}{Microsoft Visual C++}
\acro{MSVS}{Microsoft Visual Studio}
\acro{ASLR}{Address Space Layout Randomization}
\acro{MFC}{Microsoft Foundation Classes}
\acro{TLS}{Thread Local Storage}
\acro{AKA}{Also Known As%
	\RU{ - (Также известный как)}%
	\ES{ - (Tambi\'en Conocido Como)}%
	\NL{ - (Ook gekend als)}%
	\PTBRph{}%
	\PLph{}%
	\DEph{}%
	\ITAph{}%
	\THAph{}%
	\FRph{Aussi connu sous le nom}
}
\acro{CRT}{C runtime library}
\acro{CPU}{Central processing unit}
\acro{FPU}{Floating-point unit}
\acro{CISC}{Complex instruction set computing}
\acro{RISC}{Reduced instruction set computing}
\acro{GUI}{Graphical user interface}
\acro{RTTI}{Run-time type information}
\acro{BSS}{Block Started by Symbol}
\acro{SIMD}{Single instruction, multiple data}
\acro{BSOD}{Blue Screen of Death}
\acro{DBMS}{Database management systems}
\acro{ISA}{Instruction Set Architecture\RU{ (Архитектура набора команд)}}
\acro{CGI}{Common Gateway Interface}
\acro{HPC}{High-Performance Computing}
\acro{SOC}{System on Chip}
\acro{SEH}{Structured Exception Handling}
\acro{ELF}{\RU{Формат исполняемых файлов, использующийся в Linux и некоторых других *NIX}
\EN{Executable file format widely used in *NIX systems including Linux}\ESph{}\PTBRph{}\PLph{}\ITAph{}\DEph{}\NLph{}}
\acro{TIB}{Thread Information Block}
\acro{TEA}{Tiny Encryption Algorithm}
\acro{PIC}{Position Independent Code: \myref{sec:PIC}}
\acro{NAN}{Not a Number}
\acro{NOP}{No OPeration}
\acro{BEQ}{(PowerPC, ARM) Branch if Equal}
\acro{BNE}{(PowerPC, ARM) Branch if Not Equal}
\acro{BLR}{(PowerPC) Branch to Link Register}
\acro{XOR}{eXclusive OR\RU{ (исключающее \q{ИЛИ})}}
\acro{MCU}{Microcontroller unit}
\acro{RAM}{Random-access memory}
\acro{GCC}{GNU Compiler Collection}
\acro{EGA}{Enhanced Graphics Adapter}
\acro{VGA}{Video Graphics Array}
\acro{API}{Application programming interface}
\acro{ASCII}{American Standard Code for Information Interchange}
\acro{ASCIIZ}{ASCII Zero (\RU{ASCII-строка заканчивающаяся нулем}\EN{null-terminated ASCII string}\PTBRph{})}
\acro{IA64}{Intel Architecture 64 (Itanium): \myref{itanium}}
\acro{EPIC}{Explicitly parallel instruction computing}
\acro{OOE}{Out-of-order execution}
\acro{MSDN}{Microsoft Developer Network}
\acro{MSB}{Most significant bit/byte\RU{ (самый старший бит/байт)}}
\acro{LSB}{Least significant bit/byte\RU{ (самый младший бит/байт)}}
\acro{STL}{(\Cpp) Standard Template Library: \myref{sec:STL}}
\acro{PODT}{(\Cpp) Plain Old Data Type}
\acro{HDD}{Hard disk drive}
\acro{VM}{Virtual Memory\RU{ (виртуальная память)}}
\acro{WRK}{Windows Research Kernel}
\acro{GPR}{General Purpose Registers\RU{ (регистры общего пользования)}}
\acro{SSDT}{System Service Dispatch Table}
\acro{RE}{Reverse Engineering}
\acro{RAID}{Redundant Array of Independent Disks}
\acro{SSE}{Streaming SIMD Extensions}
\acro{BCD}{Binary-coded decimal}
\acro{BOM}{Byte order mark}
\acro{GDB}{GNU debugger}
\acro{FP}{Frame Pointer}
\acro{MBR}{Master Boot Record}
\acro{JPE}{Jump Parity Even (\RU{инструкция x86}\EN{x86 instruction})}
\acro{CIDR}{Classless Inter-Domain Routing}
\acro{STMFD}{Store Multiple Full Descending (\RU{инструкция ARM}\EN{ARM instruction})}
\acro{LDMFD}{Load Multiple Full Descending (\RU{инструкция ARM}\EN{ARM instruction})}
\acro{STMED}{Store Multiple Empty Descending (\RU{инструкция ARM}\EN{ARM instruction})}
\acro{LDMED}{Load Multiple Empty Descending (\RU{инструкция ARM}\EN{ARM instruction})}
\acro{STMFA}{Store Multiple Full Ascending (\RU{инструкция ARM}\EN{ARM instruction})}
\acro{LDMFA}{Load Multiple Full Ascending (\RU{инструкция ARM}\EN{ARM instruction})}
\acro{STMEA}{Store Multiple Empty Ascending (\RU{инструкция ARM}\EN{ARM instruction})}
\acro{LDMEA}{Load Multiple Empty Ascending (\RU{инструкция ARM}\EN{ARM instruction})}
\acro{APSR}{(ARM) Application Program Status Register}
\acro{FPSCR}{(ARM) Floating-Point Status and Control Register}
\acro{RFC}{Request for Comments}
\acro{TOS}{Top Of Stack\RU{ (вершина стека)}}
\acro{LVA}{(Java) Local Variable Array\RU{ (массив локальных переменных)}}
\acro{JVM}{Java virtual machine}
\acro{JIT}{Just-in-time compilation}
\acro{CDFS}{Compact Disc File System}
\acro{CD}{Compact Disc}
\acro{ADC}{Analog-to-digital converter}
\acro{EOF}{End of file\RU{ (конец файла)}}
\acro{TBT}{To be translated} % temporary...
\acro{DIY}{Do It Yourself}
\acro{MMU}{Memory management unit}
\acro{CPRNG}{Cryptographically secure Pseudorandom Number Generator}
\acro{DES}{Data Encryption Standard}
\acro{MIME}{Multipurpose Internet Mail Extensions}
\acro{XML}{Extensible Markup Language}
\acro{JSON}{JavaScript Object Notation}
\end{acronym}


\bibliographystyle{alpha}
\bibliography{books,articles,usenet,misc}

\clearpage
\printindex

\end{document}
