\vspace*{\fill}

\vspace*{\fill}

\normalsize \textbf{Объявление! Ищется программист на Си/Си++ с математическим уклоном}

\bigskip
\bigskip
\bigskip

Мои хорошие друзья в Киеве ищут программиста на Си/Си++ с уклоном в прикладную математику.
Если интересно, присылайте резюме: \TT{x@linuxenia.com}

\bigskip
\bigskip
\bigskip
% ---------------------------------------

\normalsize \textbf{Подписывайтесь на новости о моих новых статьях и постах в блоге:}

\bigskip
\bigskip
\bigskip

\begin{itemize}

\item \url{https://twitter.com/yurichev}

\item \url{https://www.facebook.com/dennis.yurichev.5}

\end{itemize}

\bigskip
\bigskip
\bigskip
% ---------------------------------------
\huge Пожалуйста, жертвуйте
\normalsize

\bigskip
\bigskip
\bigskip

\dots чтобы я мог продолжать работать над этой книгой и другими статьями: \\
\url{https://yurichev.com/donate.html}.

\bigskip
\bigskip
\bigskip

\huge Внимание: соц.опрос
\normalsize

\bigskip
\bigskip
\bigskip

У меня есть идея заменить все примеры на OllyDbg в этой книге на примеры с использованием другого отладчика.
Я не имею ничего против OllyDbg, но он GUI-шный и использует маленькие шрифты, и скриншоты не очень подходят для книги.
Может я бы использовал GDB, radare или WinDbg.
Или может быть, какой-нибудь другой консольный отладчик?

Что вы об этом думаете?
Должен ли я оставить примеры с OllyDbg, или примеры с radare будут ОК?

E-Mail: \GTT{\EMAIL}.

\vspace*{\fill}
\vfill
