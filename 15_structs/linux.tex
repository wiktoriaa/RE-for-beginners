\subsection{Linux}

\IFRU{В Линуксе, для примера, возьем структуру \TT{tm} из \TT{time.h}:}
{As of Linux, let's take \TT{tm} structure from \TT{time.h} for example:}

\lstinputlisting{15_structs/GCC_tm.c}

\IFRU{Компилируем при помощи}{Let's compile it in} GCC 4.4.1:

\IFRU{\lstinputlisting[caption=GCC 4.4.1]{15_structs/GCC_tm_ru.asm}}{\lstinputlisting{15_structs/GCC_tm_en.asm}}

\IFRU{К сожалению, по какой-то причине, \IDA не сформировала названия локальных переменных в стеке. 
Но так как мы уже опытные реверсеры :-) то можем обойтись и без этого в таком простом примере.}
{Somehow, \IDA didn't created local variables names in local stack.
But since we already experienced reverse engineers :-) we may do it without this information in 
this simple example.}

\IFRU{Обратите внимание на \TT{lea edx, [eax+76Ch]} ~--- эта инструкция прибавляет $0x76C$ к \EAX, 
но не модифицирует флаги. См. также соответствующий раздел об инструкции \LEA{}~\ref{sec:LEA}.}
{Please also pay attention to \TT{lea edx, [eax+76Ch]} ~--- this instruction just adding $0x76C$ to \EAX,
but not modify any flags. See also relevant section about \LEA{}~\ref{sec:LEA}.}

Чтобы проиллюстрировать то что структура это просто набор переменных лежащих в одном месте, переделаем немного
пример, заглянув предварительно в файл time.h:

\begin{lstlisting}[caption=time.h]
struct tm
{
  int	tm_sec;
  int	tm_min;
  int	tm_hour;
  int	tm_mday;
  int	tm_mon;
  int	tm_year;
  int	tm_wday;
  int	tm_yday;
  int	tm_isdst;
};
\end{lstlisting}

\lstinputlisting{15_structs/GCC_tm2.c}

Обратите внимание на то что в \TT{localtime\_r} передается указатель именно на \TT{tm\_sec}, 
т.е., на первый элемент ``структуры''.

В итоге:

\lstinputlisting[caption=GCC 4.7.3]{15_structs/GCC_tm2.asm}

Этот код почти идентичен уже рассмотренному, и нельзя сказать, была ли структура
в оригинальном исходном коде либо набор переменных.

И это работает. Однако, в реальности так лучше не делать. Обычно, компилятор располагает переменные в локальном
стеке в том же порядке, в котором они объявляются в функции. Тем не менее, никакой гарантии нет.

Я выбрал именно этот пример для иллюстрации, потому что члены структуры имеют тип \Tint, а члены структуры
\TT{SYSTEMTIME} ~--- 16-битные \TT{WORD}, и если их объявлять так же, то они будут выровнены по 32-битной границе 
и ничего не выйдет (потому что \TT{GetSystemTime()} заполнит их неверно). Читайте об этом в следующей секции
``\StructurePackingSectionName''.

Так что, структура это просто набор переменных лежащих в одном месте, рядом. Я мог бы сказать что структура
это такой синтаксический сахар, заставляющий компилятор удерживать их в одном месте. Впрочем, я не специалист
по языкам программирования, так что, скорее всего, ошибаюсь с этим термином.

