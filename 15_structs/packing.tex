\subsection{\StructurePackingSectionName}

\IFRU{Достаточно немаловажный момент, это упаковка полей в структурах\footnote{См.также: \URLWPDA}.}
{One important thing is fields packing in structures\footnote{See also: \URLWPDA}.}

\IFRU{Возьмем простой пример:}{Let's take a simple example:}

\lstinputlisting{15_structs/15_5.c}

\IFRU{Как видно, мы имеем два поля \Tchar (занимающий один байт) и еще два ~--- \Tint (по 4 байта).}
{As we see, we have two \Tchar fields (each is exactly one byte) and two more ~--- \Tint (each - 4 bytes).}

\IFRU{Компилируется это все в:}{That's all compiling into:}

\lstinputlisting{15_structs/15_5.asm}

\IFRU{Мы видим здесь что адрес каждого поля в структуре выравнивается по 4-байтной границе. 
Так что каждый \Tchar здесь занимает те же 4 байта что и \Tint. Зачем? 
Затем что процессору удобнее обращаться по таким адресам и кешировать данные из памяти.}
{As we can see, each field's address is aligned by 4-bytes border.
That's why each \Tchar using 4 bytes here, like \Tint. Why?
Thus it's easier for CPU to access memory at aligned addresses and to cache data from it.}

\IFRU{Но это не экономично по размеру данных.}{However, it's not very economical in size sense.}

\IFRU{Попробуем скомпилировать тот же исходник с опцией}{Let's try to compile it with option} (\TT{/Zp1}) 
(\IT{/Zp[n] pack structs on n-byte boundary}).

\lstinputlisting[caption=MSVC /Zp1]{15_structs/15_5_msvc_Zp1.asm}

\IFRU{Теперь структура занимает 10 байт и все \Tchar занимают по байту. Что это дает? 
Экономию места. Недостаток ~--- процессор будет обращаться к этим полям не так эффективно 
по скорости как мог бы.}
{Now the structure takes only 10 bytes and each \Tchar value takes 1 byte. What it give to us?
Size economy. And as drawback ~--- CPU will access these fields without maximal performance it can.}

\IFRU{Как нетрудно догадаться, если структура используется много в каких исходниках и объектных файлах, 
все они должны быть откомпилированы с одним и тем же соглашением об упаковке структур.}
{As it can be easily guessed, if the structure is used in many source and object files,
all these should be compiled with the same convention about structures packing.}

\newcommand{\FNURLMSDNZP}{\footnote{\href{http://msdn.microsoft.com/en-us/library/ms253935.aspx}
{MSDN: Working with Packing Structures}}}
\newcommand{\FNURLGCCPC}{\footnote{\href{http://gcc.gnu.org/onlinedocs/gcc/Structure_002dPacking-Pragmas.html}
{Structure-Packing Pragmas}}}

\IFRU{Помимо ключа MSVC \TT{/Zp}, указывающего, по какой границе упаковывать поля структур, есть также 
опция компилятора \TT{\#pragma pack}, её можно указывать прямо в исходнике. 
Это справедливо и для MSVC\FNURLMSDNZP и GCC\FNURLGCCPC{}.}
{Aside from MSVC \TT{/Zp} option which set how to align each structure field, here is also
\TT{\#pragma pack} compiler option, it can be defined right in source code.
It's available in both MSVC\FNURLMSDNZP and GCC\FNURLGCCPC{}.}

\IFRU{Давайте теперь вернемся к \TT{SYSTEMTIME}, которая состоит из 16-битных полей. 
Откуда наш компилятор знает что их надо паковать по однобайтной границе?}
{Let's back to \TT{SYSTEMTIME} structure consisting in 16-bit fields.
How our compiler know to pack them on 1-byte alignment method?}

\IFRU{В файле \TT{WinNT.h} попадается такое:}{\TT{WinNT.h} file has this:}

\begin{lstlisting}[caption=WinNT.h]
#include "pshpack1.h"
\end{lstlisting}

\IFRU{И такое:}{And this:}

\begin{lstlisting}[caption=WinNT.h]
#include "pshpack4.h"                   // 4 byte packing is the default
\end{lstlisting}

\IFRU{Сам файл PshPack1.h выглядит так:}{The file PshPack1.h looks like:}

\begin{lstlisting}[caption=PshPack1.h]
#if ! (defined(lint) || defined(RC_INVOKED))
#if ( _MSC_VER >= 800 && !defined(_M_I86)) || defined(_PUSHPOP_SUPPORTED)
#pragma warning(disable:4103)
#if !(defined( MIDL_PASS )) || defined( __midl )
#pragma pack(push,1)
#else
#pragma pack(1)
#endif
#else
#pragma pack(1)
#endif
#endif /* ! (defined(lint) || defined(RC_INVOKED)) */
\end{lstlisting}

\IFRU{Собственно, так и задается компилятору, как паковать объявленные после \TT{\#pragma pack} структуры.}
{That's how compiler will pack structures defined after \TT{\#pragma pack}.}

\subsection{\IFRU{Вложенные структуры}{Nested structures}}

\IFRU{Теперь, как насчет ситуаций, когда одна структура определяет внутри себя еще одну структуру?}
{Now what about situations when one structure define another structure inside?}

\lstinputlisting{15_structs/15_6.c}

\dots \IFRU{в этом случае, оба поля \TT{inner\_struct} просто будут располагаться между полями a,b и d,e в 
\TT{outer\_struct}.}
{in this case, both \TT{inner\_struct} fields will be placed between a,b and d,e fields of
\TT{outer\_struct}.}

\IFRU{Компилируем}{Let's compile} (MSVC 2010):

\lstinputlisting[caption=MSVC 2010]{15_structs/15_6_msvc.asm}

\IFRU{Очень любопытный момент в том, что глядя на этот код на ассемблере, мы даже не видим, 
что была использована какая-то еще другая структура внутри этой!
Так что, пожалуй, можно сказать, что все вложенные структуры в итоге разворачиваются в одну, \IT{линейную} 
или \IT{одномерную} структуру.}
{One curious point here is that by looking onto this assembly code, we do not even see that
another structure was used inside of it!
Thus, we would say, nested structures are finally unfolds into \IT{linear} or \IT{one-dimensional} structure.}

\IFRU{Конечно, если заменить объявление \TT{struct inner\_struct c;} на \TT{struct inner\_struct *c;} 
(объявляя таким образом указатель), ситауция будет совсем иная.}
{Of course, if to replace \TT{struct inner\_struct c;} declaration to \TT{struct inner\_struct *c;} 
(thus making a pointer here) situation will be significally different.}

